\section{Map pattern}

The map pattern is a widely used parallel programming technique where a function is applied independently to each element in a collection. 
The independence of each operation is key to the pattern's parallelizability, allowing for efficient computation, especially when combined with optimizations such as code fusion and cache management.

\paragraph*{Optimizations}
Applying multiple map operations in sequence can often be optimized by combining them into a single operation, reducing overhead and improving performance.
Code fusion combines multiple map operations into a single pass over the data, eliminating unnecessary intermediate steps and reducing memory traffic.
By optimizing memory access patterns, cache fusion reduces cache misses and ensures more efficient use of the processor cache, significantly improving performance.

\paragraph*{Independence}
The key advantage of the map pattern is its inherent independence. 
This makes it ideal for parallelization. 
In a map operation, there should be no shared state between iterations, meaning that each iteration operates only on its input and produces its output without interfering with other iterations.

This independence allows map operations to be executed in parallel, yielding significant speedups, often in the order of $\mathcal{O}(\log n)$, particularly for large datasets.

\paragraph*{Independence}
However, maintaining independence is critical. If shared state is modified within the map function, it can lead to issues such as:
\begin{itemize} 
    \item \textit{Non-determinism}: different executions may produce different results due to race conditions. 
    \item \textit{Data races}: multiple threads accessing and modifying shared data concurrently can lead to inconsistent results. 
    \item \textit{Undefined behavior}: incorrect program execution that might result in unexpected outputs or crashes. 
    \item \textit{Segfaults}: access violations or crashes due to improper memory handling. 
\end{itemize}
Thus, the map function should ideally be pure, meaning it should not modify any shared state, ensuring that each element can be processed independently.

\paragraph*{Multi-maps}
In many cases, the map pattern is extended to multiple collections, where a function is applied to elements across more than one collection simultaneously. 
This approach is useful when operations on multiple datasets need to be performed concurrently.
Chaining map operations is also common, where several map operations are performed in sequence.
A naïve implementation of these multi-map operations might store intermediate results in memory, leading to unnecessary memory bandwidth consumption and cache overhead. 
This can degrade performance, particularly on systems with limited memory bandwidth.

\paragraph*{Code fusion}
A better approach is to fuse map operations, performing them all in a single pass.
This reduces memory traffic and improves computational efficiency by eliminating redundant memory writes.
This optimization, known as fusion, increases arithmetic intensity and minimizes memory/cache usage, as intermediate results do not need to be written to memory.
In ideal situations, operations can be performed directly in registers, avoiding memory accesses altogether and further boosting performance.
However, there are cases where fusion is impractical, especially when operations cannot be combined into a single function or when it's difficult to maintain the independence of operations. 
In such cases, the workload can be partitioned into smaller blocks, with each CPU or thread handling one block at a time. 
This block-wise processing approach ensures that operations can still benefit from cache locality, improving performance even without full fusion.
By understanding and applying these optimization techniques, the map pattern can be effectively utilized to achieve high-performance parallel execution for a wide range of computational tasks.

\subsection{Patterns}
The main map patterns include the following:
\begin{itemize}
    \item \textbf{Stencil}: in this pattern, each instance of the map function accesses neighboring elements of its input, typically offset from its usual position. 
        This pattern is commonly used in applications like image processing and solving partial differential equations, where the value of an element depends on its surrounding neighbors.
    \item \textbf{Workpile}: the workpile pattern allows new work items to be dynamically added to the map during its execution. 
        As work is processed, it can generate additional tasks, which are then consumed by the map. 
        The pattern terminates when there are no more tasks to process. 
        This approach is useful in scenarios where work is generated dynamically during execution, such as in task scheduling or load balancing.
    \item \textbf{Divide-and-conquer}: this pattern applies when a problem can be recursively divided into smaller subproblems, with each subproblem being solved independently. 
        The subproblems continue to be divided until a base case is reached, which can be solved serially. 
        Afterward, the results from the subproblems are combined to form the solution to the original problem. 
        Divide-and-conquer is often used in sorting algorithms like quicksort and mergesort, as well as in matrix operations and tree-based computations.    
\end{itemize}

\subsection{Scaled vector addition}
Scaled vector addition is an operation where vector $x$ is scaled by a constant factor $a$, and the result is added element-wise to vector $y$. 
The operation modifies the input vector $y$, storing the final result in it.
The operation can be expressed as:
$y\leftarrow a\cdot x+y$
This operation is a key function in the Basic Linear Algebra Subprograms library, commonly used in scientific computing and machine learning applications.

The key feature of scaled vector addition is that each element of vectors $x$ and $y$ is independent of the others. 
As a result, the operation can be easily parallelized. 
When using more than 8 threads or processors, the performance benefits become more apparent, as the workload can be efficiently divided among multiple cores, improving execution speed.
The parallelization is particularly effective because there are no dependencies between the operations on individual elements of the vectors, allowing for high levels of concurrency.

This makes the scaled vector addition pattern well-suited for high-performance computing environments, where large vectors are processed in parallel to achieve faster computation.
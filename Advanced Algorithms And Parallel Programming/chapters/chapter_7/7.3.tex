\section{Map pattern}

The map pattern is a widely used parallel programming technique where a function is applied independently to each element in a collection. 
The independence of each operation is key to the pattern's parallelizability, allowing for efficient computation, especially when combined with optimizations such as code fusion and cache management.

\paragraph*{Independence}
The key advantage of the map pattern is its inherent independence. 
This makes it ideal for parallelization. 
In a map operation, there should be no shared state between iterations, meaning that each iteration operates only on its input and produces its output without interfering with other iterations.
This independence allows map operations to be executed in parallel, yielding significant speedups, often in the order of $\mathcal{O}(\log n)$, particularly for large datasets.

\paragraph*{Optimizations}
To optimize the mapping it is possible to use some techinques: 
\begin{itemize}
    \item \textit{Multi-maps}: the map pattern is extended to multiple collections, where a function is applied to elements across more than one collection simultaneously. 
    \item \textit{Code fusion}: fuse map operations, performing them all in a single pass.
\end{itemize}

\subsection{Implementation}
The main map patterns include the following:
\begin{itemize}
    \item \textit{Stencil}: each instance of the map function accesses neighboring elements of its input, typically offset from its usual position. 
    \item \textit{Workpile}: the workpile pattern allows new work items to be dynamically added to the map during its execution. 
    \item \textit{Divide-and-conquer}: this pattern applies when a problem can be recursively divided into smaller subproblems, with each subproblem being solved independently. 
\end{itemize}

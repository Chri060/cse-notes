\section{Random Access Machine}

\begin{definition}[\textit{Random Access Machine}]
    A Random Access Machine (RAM) is a theoretical computational model that features the following characteristics:
\end{definition}
\begin{itemize}
    \item \textit{Unbounded memory cells}: the machine has an unlimited number of local memory cells.
    \item \textit{Unbounded integer capacity}: each memory cell can store an integer of arbitrary size, without any constraints.
    \item \textit{Simple instruction set}: the instruction set includes basic operations such as arithmetic, data manipulation, comparisons, and conditional branching.
    \item \textit{Unit-time operations}: every operation is assumed to take a constant, unit time to complete.
\end{itemize}
The time complexity of a RAM is determined by the number of instructions executed during computation, while the space complexity is measured by the number of memory cells utilized.
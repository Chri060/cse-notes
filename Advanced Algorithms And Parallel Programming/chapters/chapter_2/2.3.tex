\section{Randomized algorithms}

Probabilistic analysis in algorithms assumes that the algorithm itself is deterministic; for a given fixed input, it will produce the same output and follow the same sequence of operations every time it runs. 
This analysis model considers a probability distribution over the possible inputs, evaluating the algorithm's performance based on this distribution.
While this can provide useful insights into average-case behavior, it has limitations.
or instance, certain specific inputs may lead to particularly poor performance, and if the assumed distribution does not accurately represent real-world inputs, the analysis may yield a misleading or overly optimistic view of the algorithm's expected efficiency.

In contrast, randomized algorithms incorporate randomness into their execution process, introducing variability in their behavior even when given a fixed input.
Due to this inherent randomness, a randomized algorithm may produce different results or follow different execution paths on the same input in separate runs.
Generally, randomized algorithms are designed to perform well with high probability across any input, though there remains a small probability of failure on any given run.

\subsection{Taxonomy}
\renewcommand*{\arraystretch}{2}
\begin{table}[H]
    \centering
    \begin{tabular}{l|l|l|}
    \cline{2-3}
                                              & \multicolumn{1}{c|}{\textbf{Las Vegas}}                        & \multicolumn{1}{c|}{\textbf{Monte Carlo}}                         \\ \hline
    \multicolumn{1}{|l|}{\textit{Randomness effect}} & Running time                                      & \makecell[l]{Running time \\ Solution correctness}               \\ \hline
    \multicolumn{1}{|l|}{\textit{Efficiency (polynomial bound)}} & Expected running time & Worst-case running time  \\ \hline
    \end{tabular}
\end{table}
\renewcommand*{\arraystretch}{1}
Monte Carlo algorithms are classified further based on their error probabilities. 
A Monte Carlo algorithm with two-sided error has a nonzero probability of error for both possible outputs. 
In contrast, a one-sided error algorithm guarantees correctness for at least one of the outputs, meaning it has zero error probability for that output.
\section{Primality problem}

The primality problem involves determining whether a given integer $n\geq 2$ is a prime number.
\begin{definition}[\textit{Prime number}]
    An integer $p\geq 2$ is called prime if and only if it has no positive divisors other than $1$ and itself. 
\end{definition}

\subsection{Naive algorithm}
The simplest way to test if a number $n$ is prime is to check if it has any divisors other than $1$ and itself.
Since any factor of $n$ greater than $\sqrt{n}$ would have a corresponding factor smaller than $\sqrt{n}$, we only need to check divisibility up to $\sqrt{n}$. 
\begin{algorithm}[H]
    \caption{Naive primality test}
    \begin{algorithmic}[1]
        \If{$n=2$} 
            \State \Return true 
        \EndIf 
        \If{$n\text{ is even}$}
            \State \Return false 
        \EndIf 
        \For{$i=1$ \textit{to} $\sqrt{\frac{n}{2}}$}
            \If{$2i+i$ divides $n$}
                \State \Return false 
            \EndIf 
        \EndFor 
        \State \Return true 
    \end{algorithmic}
\end{algorithm} 
The time complexity of this naive algorithm is $\mathcal{O}(\sqrt{n})$.

\subsection{Fermat primality test}
To improve efficiency, we can use a probabilistic primality test based on Fermat's Little Theorem, which states:
\begin{theorem}[Fermat]
    If $p$ is a prime number and $a$ and integer $a$ is an integer such that $0<a<p$, then $a^{p-1}\mod p= 1$. 
\end{theorem}
This theorem leads to a simple test for primality.
If $n$ is prime, $a^{n-1}\mod n= 1$ for some randomly chosen $a$. 
However, if $n$ is composite, it may still satisfy this condition for certain $a$, in which case it is called a pseudoprime to base $a$.
\begin{algorithm}[H]
    \caption{Fermat's primality test}
    \begin{algorithmic}[1]
        \If{$2^{n-1}\mod n=1$} 
            \State $n$ is possibly prime 
        \Else 
            \State $n$ is composite
        \EndIf
    \end{algorithmic}
\end{algorithm} 
The Fermat test runs in $\mathcal{O}(\log^2n)$ using modular exponentiation. 
However, it can mistakenly classify some composite numbers as prime (false positives).


\subsection{Carmichael primality test}
\begin{definition}[\textit{Carmichael number}]
    A composite number $n\geq 2$ is a Carmichael number if, for every integer $a$ coprime to $n$, it holds that $a^{n-1} \mod n = 1$
\end{definition}
\begin{algorithm}[H]
    \caption{Carmichael's primality test}
    \begin{algorithmic}[1]
        \State Randomly choose $a\in[2,n-1]$
        \If{$a^{n-1}\mod n=1$} 
            \State $n$ is possibly prime 
        \Else 
            \State $n$ is composite
        \EndIf
    \end{algorithmic}
\end{algorithm} 

\subsection{Miller-Rabin primality test}
The Miller-Rabin test improves on Carmichael's test by checking additional properties that only hold for prime numbers. 
Specifically, it looks for non-trivial square roots of $1\mod n$. 
\begin{definition}[\textit{Non-trivial square root}]
    An number $a$ is a non-trivial square root of $1 \mod n$ if: 
    \[a^2 \mod n = 1 \qquad a \neq 1 \qquad a \neq n - 1\]
\end{definition}
The Miller-Rabin test randomly selects bases and tests whether they exhibit properties consistent with a prime modulus.
\begin{algorithm}[H]
    \caption{Miller-Rabin primality test}
    \begin{algorithmic}[1]
        \Function{power}{$a,p,n$} 
            \If{$p=0$} \Comment compute $a^p \mod n$
                \State \Return 1
            \EndIf 
            \State $x=$ \Call{power}{$a, \frac{p}{2}, n$}
            \State $res= (x \cdot x)\:\%\:n$
            \If{$res=1$\textbf{ and }$x\neq 1$\textbf{ and }$x\neq n-1$} \Comment check $x^2 \mod n = 1$ and $x \neq 1, n-1$
                \State $isProbablyPrime=\text{false}$
            \EndIf
            \If{$p\:\%\:2=1$}
                \State $res=(a\cdot res)\:\%\:n$
            \EndIf
            \State \Return $res$
        \EndFunction
        \Statex 
        \Function{primalitytest}{$n$}
            \State $a = \Call{random}{2, n-1}$
            \State $isProbablyPrime = \text{true}$
            \State $result = \Call{power}{a, n-1, n}$
            \If{$res \neq 1$ \textbf{or} $!isProbablyPrime$}
                \State \Return false
            \Else
                \State \Return true
            \EndIf
        \EndFunction
    \end{algorithmic}
\end{algorithm}
Each iteration of the Miller-Rabin test has a low probability of incorrectly identifying a composite number as prime, and repeating the test $k$ times reduces this probability to $\left(\frac{1}{4}\right)^k$. 
\begin{theorem}
    If $p$ is prime and $0 < a < p$, the only solutions to $a^2 \mod p = 1$ are $a = 1$ and $a = p - 1$.
\end{theorem}
\begin{theorem}
    If $n$ is composite, the Miller-Rabin test incorrectly classifies $n$ as prime with probability at most is at most $\frac{1}{4}$.
\end{theorem}
The Miller-Rabin test runs in $\mathcal{O}(\log^2n)$ time, making it efficient and reliable for large numbers. 
It is commonly used in practice for cryptographic applications where probabilistic primality testing is acceptable.
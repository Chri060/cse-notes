\section{POSIX Threads}

Pthreads (POSIX Threads) provide a standardized API for thread management in multi-threaded programming. 
They offer mechanisms for thread creation, synchronization, and management, enabling concurrent execution of tasks.

Threads are peers, meaning that once they are created, there is no inherent hierarchy between them. 
Threads can also create additional threads, with no dependencies or constraints between them. 
The maximum number of threads depends on the system's implementation.

Pthreads offer several tools for synchronization and thread management, including mutexes for controlling access to shared resources and condition variables for inter-thread communication.

\subsection{Thread creation}
To create a new thread in Pthreads, the \texttt{pthread\_create} function is used:
\begin{lstlisting}[style=C]
int pthread_create(pthread_t * thread, const pthread_attr_t * attr, void * (* start_routine) (void *), void *arg)
\end{lstlisting}
Here's what each parameter means:
\begin{itemize}
    \item \texttt{thread}: a pointer to a \texttt{pthread\_t} variable, which will hold the identifier of the new thread.
    \item \texttt{attr}: a pointer to a \texttt{pthread\_attr\_t} structure that can be used to set thread attributes such as scheduling policies, stack size, and whether the thread is joinable or detached.
    \item \texttt{start\_routine}: the function that the new thread will execute once created.
    \item \texttt{arg}: a pointer to the argument passed to \texttt{start\_routine}. 
        It must be passed as a \texttt{void*}.
\end{itemize}

\subsection{Thread elimination}
A thread can be terminated in several ways:
\begin{itemize}
    \item By returning from its start routine.
    \item By calling \texttt{pthread\_exit}.
    \item If it is canceled by another thread using \texttt{pthread\_cancel}.
    \item When the entire process is terminated.
\end{itemize}
If the main thread calls \texttt{pthread\_exit}, any threads created by it will remain running until they complete or are canceled.

\subsection{Threads join}
To wait for a thread to finish execution, use the \texttt{pthread\_join} function:
\begin{lstlisting}[style=C]
int pthread_join(pthread_t thread, void ** retval)
\end{lstlisting}
Here's what each parameter means:
\begin{itemize}
    \item \texttt{pthread}: the thread whose execution is being waited on.
    \item \texttt{retval}: a pointer to store the return value of the terminated thread (if any). 
        This is the value passed to \texttt{pthread\_exit} by the thread.
\end{itemize}
When a thread is joined, the calling thread is blocked until the specified thread terminates.
Threads can either be created as joinable or detached. 
By default, threads are joinable. 
Once a joinable thread terminates, its resources remain allocated until the main thread calls \texttt{pthread\_join}.

\subsection{Barriers}
Barriers synchronize multiple threads at a specific point. 
All threads in a group wait for others to reach the barrier before continuing execution. 
This is useful when you want all threads to perform a specific task before moving to the next stage of execution. 
To use barriers:
\begin{lstlisting}[style=C]
int pthread_barrier_init(pthread_barrier_t * barrier, pthread_barrierattr_t * attr, unsigned int count)
int pthread_barrier_wait(pthread_barrier_t * barrier)
\end{lstlisting}
Here's what each parameter means:
\begin{itemize}
    \item \texttt{barrier}: the barrier object to synchronize threads.
    \item \texttt{count}: the number of threads that must reach the barrier before they can continue.
\end{itemize}
Not all implementations of Pthreads include barrier support.

\subsection{Thread mutex}
Mutexes (mutual exclusion) are used to protect shared data from concurrent modification. 
Only one thread can lock a mutex at a time, and other threads that attempt to lock the mutex will be blocked until it is unlocked. 
Here's an example of using a mutex:
\begin{lstlisting}[style=C]
pthread_mutex_lock(&my_lock);
/* critical section */
pthread_mutex_unlock(&my_lock);
\end{lstlisting}
If a thread fails to acquire a mutex, it will be blocked. 
Alternatively, \texttt{pthread\_mutex\_trylock} can be used to attempt locking the mutex without blocking, allowing the thread to continue if the mutex is not available.

\noindent There are three types of mutexes:
\begin{itemize}
    \item \textit{Normal}: basic mutex with no special properties.
    \item \textit{Recursive}: allows the same thread to lock the mutex multiple times without blocking.
    \item \textit{Error-check}: provides error checking to ensure proper mutex usage.
\end{itemize}

\subsection{Condition variables}
While mutexes serialize data access, condition variables allow threads to synchronize based on specific conditions. 
They are used to signal that a condition has been met, allowing waiting threads to proceed.

Without condition variables, a thread would have to continuously poll for a condition, wasting CPU resources.
With condition variables, threads can be made to wait until the condition is met and can be notified by another thread when that happens.

Condition variables are often used in conjunction with mutexes to implement producer-consumer scenarios, where threads must wait for a certain state before proceeding.
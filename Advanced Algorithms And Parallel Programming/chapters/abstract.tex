\begin{abstract}
    This course begins with an exploration of randomized algorithms, specifically Las Vegas and Monte Carlo algorithms, and the methods used to analyze them. 
    We will tackle the hiring problem and the generation of random permutations to build a strong foundation. 
    The course will then cover randomized quicksort, examining both worst-case and average-case analyses to provide a comprehensive understanding. 
    Karger-s Min-Cut Algorithm will be studied, along with its faster version developed by Karger and Stein.
    We will delve into randomized data structures, focusing on skip lists and treaps, to understand their construction and application. 
    Dynamic programming will be a key area, where we will learn about memoization and examine examples such as string matching and Binary Decision Diagrams (BDDs). 
    The course will also introduce amortized analysis, covering dynamic tables, the aggregate method, the accounting method, and the potential method to equip students with robust analytical tools. 
    Additionally, we will touch on approximate programming, providing an overview of this important concept. 
    Finally, the competitive analysis will be explored through self-organizing lists and the move-to-front heuristic.
    
    The second part of the course shifts to the design of parallel algorithms and parallel programming. 
    We will study various parallel patterns, including Map, Reduce, Scan, MapReduce, and Kernel Fusion, to understand their implementation and application. 
    Tools and languages essential for parallel programming, such as Posix Threads, OpenMP, and Message Passing Interface, will be covered, alongside a comparison of these parallel programming technologies.
    The course will also focus on optimizing and analyzing parallel performance, providing students with the skills needed to enhance and evaluate parallel computing systems.
    Practical examples of parallel algorithms will be reviewed to solidify understanding and demonstrate real-world applications.
\end{abstract}
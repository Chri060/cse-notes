\section{Potential method}

The potential method of amortized analysis conceptualizes the "bank account" as the potential energy of a dynamic sequence of operations.
The aim is to use a potential function to account for the work done by each operation and how it impacts the overall cost over time.

In this framework, we begin with an initial data structure, denoted as $D_0$. Each operation $i$ transitions the data structure from $D_{i-1}$ to $D_i$, incurring a cost $c_i$. 
To analyze the costs using the potential method, we define a potential function $\Phi$ that maps each data structure state $D_i$ to a real number ($\Phi: \{D_i\} \rightarrow \mathbb{R}$).
The potential function must satisfy the following properties:
\[\Phi(D_0) = 0 \qquad \Phi(D_i ) \geq 0\]
The amortized cost $\hat{c}_i$ for an operation $i$ is then defined as:
\[\hat{c}_i=c_i+\Phi(D_i)-\Phi(D_{i-1})=c_i+\Delta\Phi_i\]
Here, $\Delta\Phi_i$ is the potential difference between the states before and after the operation.

The potential difference $\Delta\Phi_i$ can be either positive or negative:
\begin{itemize}
    \item If $\Delta\Phi_i>0$, then $\hat{c}_i>c_i$, meaning the operation deposits work into the data structure to be used by future operations.
    \item If $\Delta\Phi_i<0$, then $\hat{c}_i<c_i$, meaning the data structure delivers stored work to help cover the current operation's cost.
\end{itemize}
The total amortized cost over $n$ operations is:
\[\sum_{i=1}^n\hat{c}_i\geq\sum_{i=1}^nc_i\]
This inequality ensures that the total amortized cost provides an upper bound on the true total cost of the operations.

\paragraph*{Hash table resizing}
To apply the potential method to the dynamic resizing of a hash table, we define the potential of the table after the $i$-th insertion as:
\[\Phi(D_i) =2i - 2^{\left\lceil \log i\right\rceil }\] 
Note that we assume $2^{\left\lceil \log 0\right\rceil }=0$, which accounts for the table's growth during resizing.

The amortized cost of the $i$-th insertion is:
\[\hat{c}_i=c_i+\Phi(D_i)-\Phi(D_{i-1})=c_i+(2i - 2^{\left\lceil \log i\right\rceil })-(2(i-1) - 2^{\left\lceil \log (i-1)\right\rceil })\]
The true cost $c_i$ of the $i$-th insertion is:
\[c_i=\begin{cases}
    i \qquad \text{if }i-1\text{ is an exact power of }2 \\
    1 \qquad \text{otherwise}
\end{cases}\]
When $i - 1$ is an exact power of 2 (i.e., the table needs to be resized), the amortized cost becomes:
\[\hat{c}_i=i + 2 - 2i + 2 + i - 1=3\]
In the case where $i - 1$ is not a power of 2, the amortized cost remains:
\[\hat{c}_i\approx 3\]
Thus, in both cases, the amortized cost per insertion is three. 
Consequently, after $n$ insertions, the total cost is $\Theta(n)$ in the worst case, ensuring that the dynamic hash table remains efficient even with frequent resizing operations.
\section{Introduction}

Amortized analysis is a technique used to assess the average cost per operation over a sequence of operations, ensuring that the overall performance remains efficient even if certain individual operations are more expensive.
Unlike probabilistic analysis, which relies on random events, amortized analysis provides a guaranteed bound on the average cost per operation, even in the worst-case scenario.

There are three primary methods of amortized analysis:
\begin{itemize} 
    \item \textit{Aggregate method}: this method calculates a simple average cost over all operations, providing an overall estimate, though it lacks the precision of more nuanced approaches. 
    \item \textit{Accounting method}: this approach employs a "banking" system, allocating an amortized cost to each operation to ensure that the total cost is distributed appropriately over a sequence of operations. 
    \item \textit{Potential method}: this method introduces a potential function to track the stored energy of a system, which helps to manage and distribute the amortized costs dynamically. 
\end{itemize}

\paragraph*{Hash table resizing}
A well-designed hash table strikes a balance between compactness and size to minimize overflow and maintain efficient access. 
However, determining an optimal size for the table upfront is often impractical, as it may not accurately reflect the growth of stored entries. 
To address this issue, dynamic resizing is employed. When the hash table reaches its capacity, a larger table is allocated, all existing entries are rehashed into the new table, and the memory from the old table is freed. 
This dynamic resizing mechanism ensures that the hash table adapts to changes in size without sacrificing efficiency, avoiding the need for a predefined size.
\section{Introduction}

Historically, both developers and users have approached problem-solving with a sequential mindset. 
This approach is reflected in the majority of existing algorithms, which are designed to execute one step at a time in a linear fashion. 
However, modern hardware architectures offer significant opportunities for parallelism, allowing for the simultaneous execution of multiple instructions or tasks.

\renewcommand*{\arraystretch}{2}
\begin{table}[H]
    \centering
    \begin{tabular}{|c|c|}
    \hline
    \textbf{Advantage} & \textbf{Description}  \\ \hline 
    Time efficiency & Parallel algorithms can complete tasks faster  \\ \hline  
    Cost efficiency & Parallel architectures use multiple inexpensive components  \\ \hline  
    Complex problems & Parallel algorithms solve some complex problems efficiently  \\   \hline  
    \end{tabular}
\end{table}
\renewcommand*{\arraystretch}{1}

\paragraph*{Moore's law}
According to Moore's Law, the number of transistors on a chip doubles approximately every 24 months, but single cores can no longer fully utilize the additional transistors. 
Moreover, continually increasing processor frequency has become impractical due to rising power consumption and heat dissipation concerns. 
Consequently, parallelism is increasingly essential to leverage these advances and continue improving computational performance.
\begin{abstract}
    The course begins with an introduction to decision-making problems and the primary steps involved in an operations research study, covering various types of optimization problems in mathematical programming. 
    This includes the formulation of optimization models, where decision variables, objective functions, constraints, and modeling techniques are discussed in depth.

    In the area of graph and network optimization, the course explores optimum spanning trees and shortest path problems, along with algorithms for these key variants, such as dynamic programming for acyclic graphs. 
    Project scheduling applications, including the critical path method and Gantt charts, are examined, followed by maximum flow problems using the Ford-Fulkerson algorithm and the maximum flow-minimum cut theorem. 
    Additional topics include minimum-cost flows, matching in bipartite graphs, and $\mathcal{NP}$-hard network optimization issues exemplified by the Traveling Salesman Problem.
    
    The section on Linear Programming addresses LP models with a focus on both geometric aspects and algebraic aspects.
    Students will study the fundamental properties of LP, the Simplex method, and duality theory, including weak and strong duality, complementary slackness, and their economic interpretations. 
    Sensitivity analysis and specific cases, such as assignment and transportation problems, are also included.
    
    In Integer Linear Programming, the course covers ILP models applied to transportation, routing, scheduling, and location problems. 
    Formulation techniques, LP relaxation, and exact solution methods are discussed, with emphasis on the Branch-and-Bound algorithm and the cutting plane method using fractional Gomory cuts.
\end{abstract}
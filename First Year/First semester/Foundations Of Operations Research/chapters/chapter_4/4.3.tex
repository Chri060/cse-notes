\section{Cutting plane method and Gomory fractional cuts}

Considering the ILP problem:
\begin{align*}
    \min                      \:&\: c^T x                       \\
    \text{such that }     &\: Ax \geq b                   \\
                                &\: x \in \mathbb{N}, x \geq 0
\end{align*}
with a feasible region:
\[ X = \left\{ x \in \mathbb{Z} \mid Ax \geq b,x \geq 0 \right\} \]
Assuming that $a_{ij} \in N$, $b_i \in N$, $c_i \in N$, the feasible region can be expressed by different sets of constraints, varying in strength. 
Although all infinite formulations are equivalent, the optimal solution of the linear relaxations of the ILP ($x^{\ast}_{LP}$) may exhibit significant variation.
\begin{definition}[\textit{Ideal formulation}]
    The ideal formulation describes the convex hull of $X$, denoted as $\text{conv}(X)$, representing the smallest convex subset containing $X$. 
\end{definition}
Given that all vertices have integer coordinates, for any $c \in \text{conv}(X)$, the following relation holds:
\[ z^{\ast}_{LP} = z^{\ast}_{ILP} \]
This implies that the optimum of the linear programming relaxation is also the optimum of the ILP.
\begin{theorem}[Ideal formulation]
    For any feasible region $X$ whether bounded or unbounded, there exists an ideal formulation—a description of $\text{conv}(X)$ involving a finite number of linear constraints.
\end{theorem}
However, the number of constraints can be substantial, growing exponentially with the size of the original formulation.

This theorem suggests that the solution of any ILP can be reduced to that of a single LP. 
Nevertheless, determining the ideal formulation is often challenging, as it tends to be either very extensive or difficult to ascertain.

\subsection{Cutting plane method}
A detailed characterization of $\text{conv}(X)$ isn't necessary; what's essential is a robust description of the vicinity surrounding the optimal solution.
\begin{definition}[\textit{Cutting plane}]
    A cutting plane is represented by the inequality $a^T x \leq b$.
\end{definition}
This inequality is not met by $x^{\ast}_{LP}$ but holds true for all feasible solutions within the ILP.

\subsection{Cutting plane methods and Gomory fractional cuts}
In the iterative process, cutting planes are successively added until the linear relaxation yields an optimal integer solution.
Consider the optimal solution $x^{\ast}_{LP}$ of the linear relaxation for the ILP problem formulated as:
\[ \min \left\{ c^T x \mid Ax = b, x \geq 0 \right\} \]
Let $x^{\ast}_{B{r}}$ denote the fractional basic variable. 
The corresponding row of the optimal tableau is expressed as:
\[ x_{B[r]} + \displaystyle \sum_{j \vert x_j \in \mathbb{N}} \overline{a}_{rj} x_j = \overline{b}_r \]
\begin{definition}[\textit{Gomory cut}]
    The Gomory cut, concerning the fractional basic variable $x_{B{r}}$, takes the form of the inequality:
    \[ \displaystyle \sum_{j \vert x_j \in \mathbb{N}} \left( \overline{a}_{rj} - \lfloor \overline{a}_{rj} \rfloor \right) x_j \geq \overline{b}_r - \lfloor \overline{b}_r \rfloor \]
\end{definition}
\begin{property}
    Both the integer form:
    \[ \displaystyle x_{B[r]} + \sum_{j \vert x_j \in \mathbb{N}} \lfloor\overline{a}_{rj}\rfloor x_j \leq \lfloor\overline{b}_r\rfloor \]
    and the fractional form:
    \[ \displaystyle x_{B[r]} + \sum_{j \vert x_j \in \mathbb{N}} \left( \overline{a}_{rj} - \lfloor \overline{a}_{rj} \rfloor \right) x_j \geq \overline{b}_r - \lfloor \overline{b}_r \rfloor \]
    are equivalent.
\end{property}

\begin{algorithm}[H]
    \caption{Cutting plane}
        \begin{algorithmic}[1]
            \State Solve the linear relaxation of the ILP problem $\min\left\{c^T x \vert Ax = b, x \geq 0 \right\}$
            \State Let $x^{\ast}_{LP}$ be the optimal solution of the linear relaxation
            \While {$x^{\ast}_{LP}$ is not integer}
                \State Select a basic variable $x_{B[r]}$ with fractional value
                \State Generate the corresponding Gomory cut
                \State Add constraint to the optimal tableau of the linear relaxation
                \State Perform one iteration of the dual simplex method
            \EndWhile
        \end{algorithmic}
\end{algorithm}
\begin{theorem}
    If the ILP has a finite number of optimal solutions, the Cutting plane method with Gomory cuts is guaranteed to find an optimal solution.
\end{theorem}
However, the algorithm's required number of iterations is unknown in advance and often proves to be very large.

\subsection{Alternative techniques}
\paragraph*{Other cutting planes}
Various generic cutting planes, tailored for specific problems, are available beyond Gomory cuts.
A comprehensive exploration of the combinatorial structure of different problems has resulted in:
\begin{itemize}
    \item The characterization of entire classes of facets.
    \item The development of efficient procedures for generating these facets.
\end{itemize}

\paragraph*{Branch-and-cut}
The approach of branch-and-cut integrates elements from both branch-and-bound and cutting plane methods, seeking to address the limitations of each. 
In the branch-and-cut method, for each subproblem, a set of cuts that are valid for that specific subproblem is generated and added to its formulation. 
When these cutting planes become less effective, the cut generation is halted, and a branching operation is executed. 
This process strengthens the formulation of various subproblems, interrupting long sequences of cuts without substantial improvements through branching operations.
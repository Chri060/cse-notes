\section{Mathematical programming problem}

\begin{table}[H]
    \centering
    \begin{tabular}{c|ccc|}
    \cline{2-4}
                                                               & \textbf{Decisions} & \textbf{Objective} & \textbf{Uncertainty} \\ \hline
    \multicolumn{1}{|c|}{\textit{Mathematical programming}}    & single                   & one                          & no                   \\
    \multicolumn{1}{|c|}{\textit{Multi-objective programming}} & single                   & multiple                     & no                   \\
    \multicolumn{1}{|c|}{\textit{Stochastic programming}}      & -                        & -                            & yes                  \\
    \multicolumn{1}{|c|}{\textit{Game theory}}                 & multiple                 & -                            & no                   \\ \hline
    \end{tabular}
\end{table}
We will delve into the realm of mathematical programming problems, where the primary aim typically revolves around minimizing or maximizing a specified function. 
Notably, it's worth mentioning that the maximization of a function $f(x)$ is essentially equivalent to the minimization of $-f(x)$. 
These problems are defined by the following characteristics:
\begin{itemize}
    \item \textit{Decision variables} $x \in \mathbb{R}^n$: these are numerical variables that serve as identifiers for potential solutions.
    \item \textit{Feasible region} $X \subseteq \mathbb{R}^n$: this is the set of admissible values for the decision variables.
    \item \textit{Objective function} $f:X \rightarrow\mathbb{R}$: this function quantitatively expresses the value of each feasible solution.
\end{itemize}
The core objective in solving a mathematical programming problem is to uncover a feasible solution that is globally optimal. 
In some cases, the problem may prove to be infeasible, unbounded, possess a unique optimal solution, or offer a multitude of optimal solutions. 
When dealing with particularly challenging problems, it may be necessary to settle for a feasible solution that represents a local optimum.

Mathematical programming can be categorized into the following classes:
\begin{enumerate}
    \item \textit{Linear Programming}.
    \item \textit{Integer Linear Programming}.
    \item \textit{Nonlinear Programming}. 
\end{enumerate}

\subsection{Multi-objective programming}
Multi-objective programming can be approached in various ways. 
Suppose our goal is to minimize $f_1(x)$ while simultaneously maximizing $f_2(x)$. 
In this context, we can:
\begin{enumerate}
    \item Combine the objectives into a single objective problem by representing both objectives in the same units. 
        This involves minimizing a weighted combination of the objectives, as follows:
        \[\min{\lambda_1f_1(x)-\lambda_2f_2(x)}\]
        with appropriate scalar values $\lambda_1$ and $\lambda_2$.
    \item Prioritize the primary objective and transform the other objective into a constraint. 
        In this approach, the focus is on optimizing the primary objective function while ensuring that the secondary objective satisfies a specified constraint. 
        This is achieved as follows:
        \[\max_{x \in X}f_2(x) \:\:\:\: f_1(x)\leq \varepsilon\]
        with an appropriate constant value $\varepsilon$. 
\end{enumerate}
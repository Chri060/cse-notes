\section{Dynamic programming}

Dynamic programming, introduced by Richard Bellman in 1953, is a versatile method employed to find optimal solutions consisting of a sequence of elementary decisions.
This is achieved by resolving a series of recursive equations.

Dynamic programming is well-suited for a wide range of sequential decision problems as long as they adhere to the optimality property.

In contemporary applications, dynamic programming finds utility across various domains, including optimal control, equipment maintenance and replacement, and the selection of inspection points along a production line.
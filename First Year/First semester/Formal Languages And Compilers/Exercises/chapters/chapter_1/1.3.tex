\section{Operations on languages}

Given two regular expressions: 
\[R_1=a((b|bb)a)^{+}\] 
\[R_2=(ab)^{*}ba\]
\begin{itemize}
    \item Define the quotient language $L=R_1-R_2$. 
    \item Write the three shortest strings of the language $L$.
    \item Write a regular expression that defines the language. 
\end{itemize}

\paragraph{Solution}
Initially, we enumerate all characters in the given regular expressions as follows:
\[R_1=a_1((b_2|b_3b_4)a_5)^{+}\]
\[R_2=(a_1b_2)^{*}b_3a_4\]
The $a_1$ is surely a prefix for every string in the language, and all the strings have $a_5$ as a suffix. 
The three shortest strings in this language are $aba$, $ababa$, and $abababa$. 

In the case of the second regular expression, every generated string starts with $ab$ and concludes with a single $ba$.  
The three shortest strings in this language are $ba$, $abba$ and $abababa$. 

It becomes apparent that all strings with the suffix $aba$ or containing at least two instances of $bb$ are certainly part of $L$. 
Consequently, the three shortest strings in $L$ are $aba$, $ababa$, and $abbaba$. 
The regular expression defining the language is:
\[L=\{(a(b|bb))^{*}aba\} \cup \{(a(b|bb))^{*}abba(a(b|bb))^{+}abba\}\]
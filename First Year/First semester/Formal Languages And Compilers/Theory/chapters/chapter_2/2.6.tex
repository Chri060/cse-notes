\section{Parenthesis languages}

Numerous constructed languages incorporate parenthesized or nested structures, which are created by matching pairs of opening and closing marks.
These parentheses may exhibit nesting, allowing for the inclusion of other parenthesized structures within a pair.
Additionally, nested structures can be arranged in sequences at the same level of nesting.
This abstraction, independent of specific representation and content, is referred to as Dyck language.
\begin{example}
    As an illustration, consider an alphabet for a Dyck language: $\Sigma=\{'(',')','[',']'\}$. 
    A valid sentence over this alphabet is: $()[[()[]]()]$.
\end{example}
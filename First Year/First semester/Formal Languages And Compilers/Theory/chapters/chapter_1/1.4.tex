\section{Regular expressions and languages}

The regular language family, the most fundamental among formal language families, can be defined in three distinct ways: algebraically, through generative grammars, and by employing recognizer automata.
\begin{definition}[\textit{Regular expression}]
    A regular expression is a string denoted as $r$, constructed over the alphabet $\Sigma=\{a_1,a_2,\dots,a_k\}$ and featuring metasymbols: union ($\cup$), concatenation ($\cdot$), star ($^{*}$), empty string ($\varepsilon$). 
\end{definition}
The metasymbols adhere to the following rules:
\begin{enumerate}
    \item \textit{Empty string}: $r=\varepsilon$.
    \item \textit{Unitary language}: $r=a$.
    \item \textit{Union of expressions}: $r=s \cup t$.
    \item \textit{Concatenation of expressions}: $r=(st)$.
    \item \textit{Iteration of an expression}: $r=s^{*}$. 
\end{enumerate}
In these rules, the symbols $s$ and $t$ represent regular expressions. 
The operator precedence is structured as follows: star has the highest precedence, followed by concatenation, and then union.
In addition to these operators, derived operators are frequently employed:
\begin{itemize}
    \item $\varepsilon$, defined as $\varepsilon=\varnothing^{*}$. 
    \item $e^{+}$, defined as $e \cdot e^{*}$. 
\end{itemize}
The interpretation of a regular expression $r$ corresponds to a language $L_r$ over the alphabet $\Sigma$, as outlined in the following table:
\begin{table}[H]
    \centering
    \begin{tabular}{cc}
    \hline
    \textbf{Expression $\boldsymbol{r}$} & \textbf{Language $\boldsymbol{L_r}$} \\ \hline
    $\varnothing$                        & $\varnothing$                        \\
    $\varepsilon$                        & $\{\varepsilon\}$                    \\
    $a \in \Sigma$                       & $\{a\}$                              \\
    $s \cup t \textnormal{ or } s|t$     & $L_s \cup L_t$                       \\
    $s \cdot t \textnormal{ or } st$     & $L_s \cdot L_t$                      \\
    $s^{*}$                              & $L_s^{*}$                            \\ \hline
    \end{tabular}
\end{table}
\begin{definition}[\textit{Regular language}]
    A regular language can be described by a regular expression.
\end{definition}

\begin{example}
    Consider the regular expression $e=(111)^{*}$, which denotes the language 
    \[L_e=\{\varepsilon,111,111111,\dots\}\]
    Similarly, the regular expression $e_1=11(1)^{*}$ represents the language 
    \[L_{e_1}=\{11,111,1111,11111,\dots\}\]
\end{example}
\begin{definition}[\textit{Family of regular languages}]
    The family of regular languages, denoted as REG, encompasses all languages that can be expressed by regular expressions.
\end{definition}
\begin{definition}[\textit{Family of finite languages}]
    The family of finite languages, denoted as FIN, comprises languages with finite cardinality.
\end{definition}
Every finite language is considered regular  since it can be articulated as the union of a finite set of strings, each formed by concatenating a finite number of alphabet symbols:
\[\left(x_1 \cup x_2 \cup \dots \cup x_k \right) = \left( a_{1_1}a_{1_2}\dots a_{1_n} \cup \dots \cup a_{k_1}a_{k_2}\dots a_{k_m} \right)\]
It is essential to recognize that the family of regular languages includes languages with infinite cardinality. 
Consequently, $\textnormal{FIN} \subset \textnormal{REG}$.

\paragraph*{Operations}
The union and repetition operators in regular expressions correspond to various choices, enabling the formulation of sub-expressions that identify specific sub-languages.
\begin{table}[H]
    \centering
    \begin{tabular}{cc}
    \hline
    \textbf{Expression $\boldsymbol{r}$}                                    & \textbf{Choice of $\boldsymbol{r}$}                       \\ \hline
    $e_1 \cup \dots \cup e_n \textnormal{ or } e_1 | \dots | e_n$           & $e_k$ for every $1 \leq k \leq n$                         \\
    $e^{*}$                                                                 & $\varepsilon$ or $e^n$ for every $n \geq 1$               \\
    $e^{+}$                                                                 & $e^n$ for every $n \geq 1$                                \\ \hline
    \end{tabular}
\end{table}
When manipulating a regular expression, it is possible to derive a new expression by substituting any outermost sub-expression with another that represents a choice of it.

\begin{definition}[\textit{Derivation}]
    We define a derivation relationship between two regular expressions, $e^{'}$ and $e^{''}$, denoted as $e^{'} \implies e^{''}$,  when these expressions can be decomposed as follows:
    \[e^{'}=\alpha \beta \gamma\]
    \[e^{''}=\alpha \delta \gamma\]
    Here, $\delta$ represents a choice that includes $\beta$.
\end{definition}
The derivation relation can be applied iteratively, leading to the following relations:
\begin{itemize}
    \item \textit{Power of $n$}: $\overset{n}{\implies}$ with $n \in \mathbb{N}$. 
    \item \textit{Transitive closure}: $\overset{*}{\implies}$ with $n \geq 0$. 
    \item \textit{Reflexive transitive closure}: $\overset{+}{\implies}$ with $n > 0$.
\end{itemize}
\begin{example}
    If $e_0 \overset{n}{\implies} e_n$, it implies that $e_n$ is derived from $e_0$ in $n$ steps.
    Similarly, $e_0 \overset{+}{\implies} e_n$ indicates that $e_n$ is derived from $e_0$ in $n \geq 1$ steps.
    Additionally, $e_0 \overset{*}{\implies} e_n$ suggests that $e_n$ is derived from $e_0$ in $n \geq 0$ steps.
\end{example}
Some derived regular expressions incorporate metasymbols, including operators and parentheses, while others consist solely of symbols from the alphabet $\Sigma$, also known as terminals, and the empty string $\varepsilon$. 
These expressions define the language specified by the regular expression.
It's crucial to observe that in derivations, operators must be selected from the outer to the inner layers. 
Premature choices could eliminate valid sentences from consideration.
\begin{definition}[\textit{Expression equivalence}]
    Two regular expressions are considered equivalent if they define the same language. 
\end{definition}

\subsection{Ambiguity}
A regular language may yield identical phrases through various equivalent derivations, introducing ambiguity.
The order of choices in these derivations can differ, leading to multiple valid outcomes.
To determine expressions with multiple derivations, it's essential to establish numbered subexpressions within a regular expression. 
Follow these steps:
\begin{itemize}
    \item Start with a regular expression, considering all possible parentheses.
    \item Derive a numbered version, represented as $e_N$, from the original expression, $e$.
    \item Identify all numbered subexpressions within the resulting expression.
\end{itemize}
\begin{example}
    For the regular expression $e=(a \cup(bb))^{*}(c^{+} \cup(a\cup(bb)))$, the corresponding numbered expression is:
    \[e_N=(a_1\cup(b_2b_3))^{*}(c_4^{+} \cup(a_5\cup(b_6b_7)))\]
    Sub-expressions can be derived by iteratively removing parentheses and union symbols.
\end{example}
\begin{definition}[\textit{Ambiguous regular expression}]
    A regular expression is termed ambiguous if its numbered version, denoted as $f'$, contains two distinct strings, $x$ and $y$, that become identical when the numbers are removed.
\end{definition}
\begin{example}
    Consider the regular expression $e=(aa|ba)^{*}a|b(aa|b)^{*}$; its numbered version is:
    \[e_N=(a_1a_2|b_3a_4)^{*}a_5|b_6(a_7a_8|b_9)^{*}\]
    Deriving $b_3a_4a_5$ and $b_6a_7a_8$ from this expression results in the same string, $baa$. 
    Therefore, the regular expression $e$ is deemed ambiguous.
\end{example}

\subsection{Extended regular expressions}
To formulate a regular expression, the introduction of the following operators does not alter its expressive power:
\begin{itemize}
    \item \textit{Power}: $a^h$ represents the repetition of $a$ for $h$ times, i.e., $aa\dots a$ repeated $h$ times.
    \item \textit{Repetition}: $[a]^n_k=a^k \cup a^{k+1} \cup \dots \cup a^n$.
    \item \textit{Optionality}: $(\varepsilon \cup a)$ or $[a]$ denotes the choice between an empty string and the character $a$.
    \item \textit{Ordered interval}: $(0\dots 9)(a \dots z)(A \dots Z)$ specifies an ordered range, combining numeric digits, lowercase, and uppercase letters.
    \item \textit{Intersection}: this operator proves useful in defining languages through the conjunction of conditions.
    \item \textit{Complement}: $\lnot L$ denotes the complement of language $L$.
\end{itemize}

\subsection{Closure properties of the REG family}
\begin{definition}[\textit{Closure}]
    Let $op$ denote a unary or binary operator.
    A family of languages is considered closed under $op$ if, and only if, applying the $op$ operator to languages within that family results in a language that remains within the same family.
\end{definition}
\begin{property}
    The REG family exhibits closure under concatenation, union, star, intersection, and complement operators.
\end{property}
This signifies that regular languages can be merged and manipulated using these operators without venturing outside the confines of the REG family.
\section{Multiple scanners}

Sometimes is useful to have more than one scanner together. 
To facilitate the support for multiple scanners, the following methods are employed:
\begin{itemize}
    \item Rules can be designated with the name of the associated scanner, known as the start condition.
    \item Special actions enable the transition between scanners.
\end{itemize}

A start condition, denoted by \texttt{S} is utilized to annotate rules as \texttt{<S>RULE} and activate rules when the scanner operates in the \texttt{S} start condition.
Start conditions can be:
\begin{itemize}
    \item \textit{Exclusive}, declared with \texttt{$\%$x S}; this disables unmarked rules when the scanner operates in the \texttt{S} start condition.
    \item \textit{Inclusive}, declared with \texttt{$\%$s S}; unmarked rules are active when the scanner operates in the \texttt{S} start condition.
\end{itemize}
The initial condition is inclusive by default.
Additionally:
\begin{itemize}
    \item The \texttt{*} start condition matches any start condition.
    \item The initial start condition is denoted as \texttt{INITIAL}.
    \item Start conditions are represented as integers.
    \item The current start condition is stored in the \texttt{YY$\_$START} variable.
\end{itemize}
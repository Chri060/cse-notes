\section{Regular expressions syntax}

The symbols employed in regular expressions include:
\begin{table}[H]
    \centering
    \begin{tabular}{c|l}
        \textbf{Syntax}                   & \textbf{Matches}                              \\ \hline
        \texttt{x}                        & The character \texttt{x}                      \\
        \texttt{.}                        & Any character except newline                  \\
        \texttt{[x,y,z]}                  & Any character in the set \texttt{x, y, z}     \\
        \texttt{[\textasciicircum x,y,z]} & Any character not in the set \texttt{x, y, z} \\
        \texttt{[a-z]}                    & Any character in the range \texttt{a-z}       \\
        \texttt{[\textasciicircum a-z]}   & Any character not in the range \texttt{a-z}   \\
    \end{tabular}
    \caption{Basic character sets}
\end{table}
\begin{table}[H]
    \centering
    \begin{tabular}{c|l}
        \textbf{Syntax}       & \textbf{Matches}                                                     \\ \hline
        \texttt{R}            & The regular expression \texttt{R}                                    \\
        \texttt{R S}          & The concatenation of \texttt{R} and \texttt{S}                       \\
        \texttt{R$|$S} & The alternation of \texttt{R} and \texttt{S}                         \\
        \texttt{R*}           & Zero or more occurrences of \texttt{R}                               \\
        \texttt{R+}           & One or more occurrences of \texttt{R}                                \\
        \texttt{R?}           & Zero or one occurrence of \texttt{R}                                 \\
        \texttt{R}\{\texttt{n}\}       & Exactly \texttt{n} occurrences of \texttt{R}                         \\
        \texttt{R}\{\texttt{n,}\}      & At least \texttt{n} occurrences of \texttt{R}                        \\
        \texttt{R}\{\texttt{n,m}\}     & At least \texttt{n} and at most \texttt{m} occurrences of \texttt{R} \\
    \end{tabular}
    \caption{Composition of regular Expressions}
\end{table}
\begin{table}[H]
    \centering
    \begin{tabular}{c|l}
        \textbf{Syntax}                 & \textbf{Matches}                                                                                         \\ \hline
            \texttt{(R)}                & Capture group or override precedence                                                                     \\
            \texttt{\textasciicircum R} & Match at the beginning of the line                                                                       \\
            \texttt{R\$}                & Match at the end of the line                                                                             \\
            \texttt{$\backslash$t}   & Tab character                                                                                            \\
            \texttt{$\backslash$n}   & Newline character                                                                                        \\
            \texttt{$\backslash$w}   & A word \textit{(same as \texttt{[a-zA-Z0-9\_]})}                                                         \\
            \texttt{$\backslash$d}   & A digit \textit{(same as \texttt{[0-9]})   }                                                             \\
            \texttt{$\backslash$s}   & A whitespace character \textit{(same as \texttt{[$\backslash$t$\backslash$s$\backslash$n]})} \\
            \texttt{$\backslash$W}   & A non-word character                                                                                     \\
            \texttt{$\backslash$D}   & A non-digit character                                                                                    \\
            \texttt{$\backslash$S}   & A non-whitespace character                                                                               \\
    \end{tabular}
    \caption{Regular expression utilities}
\end{table}
Regular expressions are ill-suited for input validation tasks. 
The complexity or impossibility of certain tasks becomes apparent when relying solely on regular expressions. 
In such cases, a comprehensive parser is essential.

\subsection{UNIX command line tools}
The following UNIX command line tools are particularly useful for regular expressions:
\begin{itemize}
    \item \texttt{grep -E ⟨regex⟩} \\
        Locates all lines in the input that match the specified regex.
    \item \texttt{find -E . -regex ⟨regex⟩} \\
        Discovers all files under the current directory whose names match the given regex.
    \item \texttt{sed -Ee s/ ⟨regex⟩ / ⟨replacement⟩ /g ⟨filename⟩} \\
        Reads \texttt{⟨filename⟩}, identifies all strings matching \texttt{⟨regex⟩}, and substitutes them with \texttt{⟨replacement⟩}. 
        Outputs the result to the standard output.
\end{itemize}
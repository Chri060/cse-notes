\section{Erroneous grammars}

\paragraph*{Clean grammar}
A grammar $G$ is deemed clean  if, and only if, for every nonterminal $A$:
\begin{itemize}
    \item $A$ is reachable from the axiom $S$, and thus contributes to the generation of the language. 
        In other words, there exists a derivation:
        \[S \overset{\ast}{\implies} \alpha A \beta\]
    \item  $A$ is defined, meaning it generates a non-empty language:
        \[L_A(G) \neq \varnothing\]
\end{itemize}
Note that the rule $L_A(G) \neq \varnothing$ also includes the case where no derivation from $A$ terminates with a terminal string $s$.

The process of cleaning a grammar involves a two-step algorithm:
\begin{enumerate}
    \item Establish the set UNDEF, comprising undefined nonterminals.
    \item Identify the set of unreachable nonterminals.
\end{enumerate}

\subsection{Phase one}
We define the set DEF as follows:
\[\textnormal{DEF}=\{A\mid( A \rightarrow u ) \in P,\textnormal{with } u \in \Sigma^{\ast}\}\]
We initiate the process by examining the terminal rules. 
Then, we apply the following update iteratively until a fixed point is reached:
\[\textnormal{DEF}=\textnormal{DEF} \cup \{B\mid( B \rightarrow D_1D_2\dots D_n)\in P \land \forall i(D_i \in\textnormal{DEF} \cup \Sigma)\}\]
During each iteration, two cases may occur:
\begin{enumerate}
    \item New nonterminals are discovered, and they all have their right-hand side symbols defined as nonterminals or terminals.
    \item No new nonterminals are found, and the algorithm terminates.
\end{enumerate}
At this stage, the nonterminals in UNDEF are removed.

\subsection{Phase two}
The produce relation, denoted as $A$ produce $B$, is valid only when there is a production rule $(A \rightarrow \alpha B \beta) \in P$, where $A \neq B$ and $\alpha,\beta$ can be any strings.
Subsequently, it can be asserted that a nonterminal $C$ is reachable from the initial symbol $S$ if and only if there exists a path in the produce relation graph from $S$ to $C$. 
Nonterminals that lack reachability from the initial symbol can be eliminated. 

\subsection{Additional requirement}
In addition to the aforementioned criteria for cleanliness, a third condition is frequently imposed:
\begin{enumerate}
    \item [3.] $G$ must avoid circular deviations, as they are non-essential and may introduce ambiguity.
\end{enumerate} 
Circular derivation occurs when, given $A \overset{+}{\implies} A$, the derivations $A \overset{+}{\implies} x$ and $A \overset{+}{\implies} A \overset{+}{\implies} x$ are possible.
It's important to note that even if a grammar is clean, it might still contain redundant rules that lead to ambiguity.
\begin{example}
    Instances of unclean grammars are illustrated below:
    \[\begin{cases}
        S \rightarrow aASb \\
        A \rightarrow b
    \end{cases} 
    \qquad
    \begin{cases}
        S \rightarrow a \\
        A \rightarrow b
    \end{cases} 
    \qquad
    \begin{cases}
        S \rightarrow aASb \\
        A \rightarrow S\mid b
    \end{cases} \]
    In the first case, the axiomatic rule fails to generate any phrase. 
    In the second case, $A$ is unreachable. 
    In the third case, the grammar exhibits circularity involving both $S$ and $A$.
\end{example}
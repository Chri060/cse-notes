\section{Operations on languages}

Typically, operations on a language are defined by applying string operations to each of its phrases.

\paragraph*{Reflection}
The reflection, denoted as $L^R$, of a language $L$ comprises a finite set of strings that are reversals of sentences present in $L$:
\[L^R = \{ x \mid \exists y \left( y \in L \land x=y^R \right)\}\]

\paragraph*{Prefix}  
The set of prefixes of a language $L$ is defined as:
\[\textnormal{Prefixes}(L)=\{y \mid y \neq \varepsilon \land \exists x \exists z \left( x \in L \land x=yz \land z \neq \varepsilon \right)\}\]
A language is deemed prefix-free if none of its sentences have proper prefixes within the language itself:
\[\textnormal{Prefixes}(L) \cap L = \varnothing\]
\begin{example}
    The language $L_1=\{x\mid x=a^nb^n \land n \geq 1\}$ is an example of a prefix-free language.
    On the contrary, the language $L_2=\{x\mid x=a^mb^n \land m > n \geq 1\}$ is not prefix-free. 
\end{example}

\paragraph*{Concatenation}  
When working with languages $L^{\prime}$ and $L^{\prime\prime}$, the concatenation operation is expressed as:
\[L^{\prime}L^{\prime\prime}=\{ xy \mid x \in L^{\prime} \land y \in L^{\prime\prime} \}\]

\paragraph*{Repetition}  
The repetition of languages is defined by:
\[ \begin{cases}
    L^m=L^{m-1}L \qquad \textnormal{for } m > 0 \\
    L^0=\{ \varepsilon \}
\end{cases}\]
The corresponding identities are:
\[\varnothing ^0 = \{ \varepsilon \} \]
\[L.\varnothing=\varnothing .L=\varnothing \]
\[L.\{\varepsilon\}=\{\varepsilon\} .L=L\]
The power operator offers a concise way to define the language of strings whose length does not exceed a specified integer $k$.
\begin{example}
    Consider the language $L=\{\varepsilon,a,b\}^k$ with $k=3$, which can be represented as:
    \[L=\{\varepsilon,a,b,aa,ab,ba,bb,aaa,\dots,bbb\}\] 
\end{example}

\paragraph*{Set operations}  
Given that a language is essentially a set, it naturally supports standard set operations, encompassing union ($\cup$), intersection ($\cap$), difference ($ \setminus $), inclusion ($ \subseteq $), strict inclusion ($ \subset $), and equality ($=$). 

\paragraph*{Universal language} 
The universal language is defined as the collection of all strings over an alphabet $\Sigma$, irrespective of length, including zero:
\[L_{universal}=\Sigma ^0 \cup \Sigma ^1 \cup \Sigma ^2 \cup \dots \]

\paragraph*{Complement} 
The complement of a language $L$ over an alphabet $\Sigma$, denoted as $\lnot L$, is determined by the set difference:
\[\lnot L=L_{universal}\setminus L\]
In simpler terms, it encompasses the strings over the alphabet $\Sigma$ that do not belong to the language $L$.
It's noteworthy that:
\[L_{universal} = \lnot \varnothing\]
While the complement of a finite language is inevitably infinite, the complement of an infinite language does not necessarily result in a finite set.

\paragraph*{Reflexive and transitive closures} 
Consider a set $A$ and a relation $R \subseteq A \times A$, where the pair $(a_1, a_2) \in R$ is often denoted as $a_1Ra_2$.
The relation $R^{\ast}$ is defined with the following properties:
\begin{itemize}
    \item Reflexive property:
        \[xR^{\ast}x \qquad \forall x \in A\]
    \item Transitive property: 
        \[x_1Rx_2 \land x_2Rx_3 \land \dots x_{n-1}Rx_n \implies x_1R^{\ast}x_n\]
\end{itemize}
\begin{example}
    For the given relation $R = \{(a, b), (b, c)\}$, its reflexive and transitive closure, denoted as $R^{\ast}$, will be:
    \[R^{\ast} = \{(a, a), (b, b), (c, c), (a, b), (b, c), (a, c) \}\]
\end{example}
The relation $R^{+}$  is defined based on the transitive property:
\[x_1Rx_2 \land x_2Rx_3 \land \dots x_{n-1}Rx_n \implies x_1R^{\ast}x_n\]
\begin{example}
    For the given relation $R = \{(a, b), (b, c)\}$, the transitive closure is: 
    \[R^{+} = \{ (a, b), (b, c), (a, c)\}\]
\end{example}

\paragraph*{Star operator} 
The star operator, also known as the Kleene star, represents the reflexive transitive closure concerning the concatenation operation.
It is defined as the union of all powers of the base language:
\[L^{\ast}=\bigcup_{h=0\dots\infty}L^h=L^0 \cup L^1 \cup L^2 \cup \dots = \varepsilon \cup L^1 \cup L^2 \cup \dots\]
\begin{example}
    Consider the language $L=\{ab,ba\}$. 
    Applying the star operation results in the following language:
    \[L^{\ast}=\{\varepsilon, ab, ba, abab, abba, baab, baba, \dots\}\]
    It's noteworthy that while $L$ is finite, $L^{\ast}$ is infinite, illustrating the generative power of the star operation.
\end{example}
Every string within the star language $L^{\ast}$ can be decomposed into substrings belonging to the base language $L$.
Consequently, the star language $L^{\ast}$ can be regarded as equivalent to the base language $L$.
If the alphabet $\Sigma$ is taken as the base language, then $\Sigma^{\ast}$ encompasses all possible strings constructed from that alphabet, making it the universal language of alphabet $\Sigma$.
It's common to express that a language $L$ is defined over the alphabet $\Sigma$ by indicating that $L$ is a subset of $\Sigma^{\ast}$, denoted as $L \subseteq \Sigma^{\ast}$. 
The properties of the star operator can be summarized as follows:
\begin{itemize}
    \item \textit{Monotonicity}: $L \subseteq L^{\ast}$. 
    \item \textit{Closure by concatenation}: if $x \in L^{\ast} \land y \in L^{\ast}$, then $xy \in L^{\ast}$. 
    \item \textit{Idempotence}: $(L^{\ast})^{\ast}=L^{\ast}$
    \item \textit{Commutativity of star and reflection}: $(L^{\ast})^R=(L^R)^{\ast}$
\end{itemize}
Additionally, if $L^{\ast}$ is finite, then it follows that $\varnothing^{\ast}=\{\varepsilon\}$ and $\{\varepsilon\}^{\ast}=\{\varepsilon\}$. 

\paragraph*{Cross operator} 
The cross operator, also referred to as the transitive closure under the concatenation operation, is defined as the union of all powers of the base language, excluding the first power $L^0$:
\[L^{+}=\bigcup_{h=1\dots\infty}L^h=L^1 \cup L^2 \cup \dots\]
\begin{example}
    Let's consider the language $L=\{ab,ba\}$. 
    The application of the cross operator yields the following language:
    \[L^{\ast}=\{ab, ba, abab, abba, baab, baba, \dots\}\]
\end{example}

\paragraph*{Quotient} 
The quotient operator operates on the languages $L_1$ and $L_2$ by eliminating suffixes from $L_1$ that belong to $L_2$. 
It is formally defined as follows:
\[L=L_1/L_2=\{y\mid\exists x \in L_1 \exists z \in L_2 (x=yz)\}\]
\begin{example}
    Let's consider the languages $L_1=\{a^{2n}b^{2n}\mid n>0\}$ and $L_2=\{b^{2n+1}\mid n \geq 0\}$. 
    The quotient language $L_1/L_2$ is:
    \[L_1/L_2=\{aab,aaaab,aaaabbb\}\]
    The quotient language $L_2/L_1$ is:
    \[L_2/L_1=\varnothing\]
    This is because no string in $L_2$ contains any string from $L_1$ as a suffix.
\end{example}
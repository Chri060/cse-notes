\begin{abstract}
    The lectures cover fundamental concepts in formal language theory, including the definition of language, language operations, regular expressions, and regular languages. 
    Topics such as deterministic and non-deterministic finite automata, BMC and Berry-Sethi algorithms, and properties of regular language families, including nested lists, are also explored. 
    Additionally, context-free grammars and languages are discussed, covering syntax trees, grammar ambiguity, context-free language properties, and the primary syntactic structures and limitations inherent to context-free languages.

    The analysis and recognition of phrases are studied, including parsing algorithms, automata, push-down automata, and deterministic languages. 
    Parsing methods such as bottom-up and recursive top-down syntactic analysis and the complexity of recognition are reviewed. 
    Translation topics encompass syntax-driven, direct, inverse, and syntactic translations, along with transducer automata, syntactic analysis, and translation. 
    Further, semantic properties and static flow analysis of programs are examined, covering syntax-driven semantic translation, semantic functions, attribute grammars, and one-pass and multi-pass attribute computation.
\end{abstract}
\section{First order predicate logic}

First-order logic extends propositional logic by introducing the capability to define predicates involving variables. 
It also introduces the existential ($\exists$) and universal ($\forall$) quantifiers.
In predicate logics, it is possible to deduce the truth value of a proposition through inferential mechanisms, such as Modus Ponens.
\begin{example}
    Given the sentences: "All man are mortal" and "Socrates is a man" we can infer that "Socrates is mortal".
\end{example}
Inference is employed to model a mental mechanism that allows us to store a reduced amount of information and establish a process for deriving additional information from existing information to deal with everyday situations.
\begin{definition}[\textit{Knowledge}]
    The combination of information and potential relationships constitutes what we refer to as knowledge.
\end{definition}
\section{Uncertainty modelling}

The choice of an uncertainty model is contingent upon the type of uncertainty, its origins, and the available information regarding uncertainty. 
It primarily pertains to the characterization and quantification of uncertainty. 
The conceivable models for representing uncertainty include statistical, logical, and cognitive models.

Artificial Intelligence and Machine Learning technologies are grounded in models that incorporate uncertainty models of various types. 
These models are crucial not only for constructing efficient models but also for defining learning models capable of addressing complex scenarios and evaluating the quality of learned or developed models.

\paragraph*{Uncertainty quantification}
Now, regarding uncertainty quantification, there are two principal categories of problems:
\begin{itemize}
    \item \textit{Forward propagation of uncertainty}: in this approach, various sources of uncertainty are propagated through the model to predict the overall uncertainty in the system's response. 
        This process entails:
        \begin{itemize}
            \item Evaluating low-order moments of the outputs (mean and variance).
            \item Assessing the reliability of the outputs.
            \item Determining the complete probability distribution of the outputs. 
        \end{itemize}
        This methodology is related to the principles applied in Bayesian networks and graphical models.
    \item \textit{Inverse assessment of model and parameter uncertainty}: in this scenario, model parameters are calibrated concurrently using test data. 
        Given experimental measurements of a system and results from its mathematical model, inverse uncertainty quantification estimates the discrepancy between the experiment and the mathematical model (bias correction). 
        It also determines the values of any unknown parameters in the model (parameter calibration).
\end{itemize}

\paragraph*{Models classification}
As for the classification of models used in Artificial Intelligence, they can be broadly categorized into three main types:
\begin{itemize}
    \item \textit{Symbolic models}: elements of these models are expressed as terms related to entities to be modeled. 
        The state of the world is represented by facts articulated in formal languages closely resembling natural languages.
    \item \textit{Sub-symbolic models}: in these models, elements are expressed through code.
    \item \textit{Black-box models}: these models are primarily seen as computational tools for mapping inputs to outputs, with their internal workings not explicitly considered.
\end{itemize}
In the context of symbolic models, a fact is true in a model if there is enough evidence to support it. 
The only facts considered truly accurate are those true by definition. 
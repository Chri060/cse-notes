\section{Inference rules}

\begin{definition}[\textit{Inference rule}]
    An inference rule is a model, essentially defining a mapping from input to output. 
    These rules are utilized to represent inferential relationships among various pieces of knowledge.
\end{definition}
We will primarily focus on forward chaining rules, which typically have the structure "IF $\left\langle \text{antecedent} \right\rangle$ THEN $\left\langle \text{consequent} \right\rangle$", where: 
\begin{itemize}
    \item $\left\langle \text{antecedent} \right\rangle$ is a set of clauses related by logical operators.
    \item $\left\langle \text{consequent} \right\rangle$ is a set of clauses related by logical operators.
\end{itemize}
In these inference rules, the clauses can be either propositions (sequences of symbols) or patterns (sequences of symbols and variables).

Inference rules play a crucial role in the implementation of Knowledge-Based Systems, with Expert Systems standing out as highly successful applications in the field of Artificial Intelligence. 
Expert Systems are carefully designed to replicate or enhance human expertise in problem-solving. 
The process of Knowledge Acquisition, which is inherently intricate, leads to the development of rule-based systems that are executed on computer platforms.

\paragraph*{Information generation}
A system can generate new information by following these steps:
\begin{enumerate}
    \item Pattern matching: identify the rules with antecedents that match the known facts stored in the fact base. 
        These rules can be considered for activation, provided the corresponding variables are assigned.
    \item Rule selection: among the rules identified through pattern matching (candidate rules), select the ones to be activated.
    \item Rule activation: assert the consequents of the selected rules in the fact base.
\end{enumerate}
\begin{example}
    Let's consider a rule base consisting of the following four rules:
    \begin{enumerate}
        \item IF X croaks AND X eats flies, THEN X is a frog.
        \item IF X chirps AND X sings, THEN X is a canary.
        \item IF X is a frog, THEN X is green.
        \item IF X is a canary, THEN X is yellow.
    \end{enumerate}
    Now, let's observe the following facts in the fact base:
    \begin{itemize}
        \item Fritz croaks.
        \item Fritz eats flies.
    \end{itemize}
    From rule 1 and the facts (a and b), we can add the following fact to the fact base:
    \[\textnormal{Fritz is a frog}\]
    With the updated fact base, we can use rule 3 to deduce the fact:
    \[\textnormal{Fritz is green}\]
\end{example}
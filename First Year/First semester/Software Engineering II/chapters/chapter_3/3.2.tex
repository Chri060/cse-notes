\section{Syntax}

Alloy presents bounded snapshots of the world that satisfy the given specification. 

It employs bounded exhaustive search to uncover counterexamples to asserted properties using SAT.        
\newpage
\begin{definition}[\textit{Atoms}]
    Atoms represent Alloy's fundamental entities, characterized by indivisibility, immutability, and a lack of interpretation.
\end{definition}
\begin{definition}[\textit{Relations}]
    Relations establish connections between atoms, forming sets of tuples, where tuples are sequences of atoms.
\end{definition}
In Alloy, relations are inherently typed, with their types determined by the declaration of the relation. 
The fundamental Alloy relation types include:
\begin{itemize}
    \item \texttt{none}: empty set. 
    \item \texttt{univ}: universal set.
    \item \texttt{iden}: identity relation.
\end{itemize}
The logical operators in Alloy encompass:
\begin{itemize}
    \item \textit{Union} ($\cup$).
    \item \textit{Intersection} ($\&$).
    \item \textit{Difference} ($-$).
    \item \textit{Subset} ($in$).
    \item \textit{Equality} ($=$).
    \item \textit{Cross product} ($\rightarrow$), analogous to a natural join.
    \item \textit{Dot join} ($.$), where the final element of the first relation joins with the corresponding first elements of the second relation, followed by the removal of the combined element from the relation.
\end{itemize}
Here are the various binary closures applicable to relations in Alloy:
\begin{itemize}
    \item \textit{Transpose} ($\sim$): inverts the order of the elements in the relation.
    \item \textit{Transitive} ($\land$): it signifies the transitive closure where $^{\land}r=r+r.r+r.r.r+\dots$. 
    \item \textit{Reflexive transitive} ($*$): it represents the reflexive transitive closure, where $^{*}r=\textnormal{iden}+^{\land}r$. 
\end{itemize}
The possible restrictions are: 
\begin{itemize}
    \item \textit{Domain restriction} ($<:$), that restricts the elements on the left side to the set on the right side.
    \item \textit{Range restriction} ($:>$), that is same as before, but the relations are inverted. 
    \item \textit{Override} ($++$), that removes the tuples on the left that are in the right relations and adds all the remaining relations of the right relation. 
\end{itemize}
Alloy also includes various Boolean operators:
\begin{itemize}
    \item \textit{Negation} ($!$ or \texttt{not}). 
    \item \textit{Conjunction} ($\&$ or \texttt{and}). 
    \item \textit{Disjunction} ($\mid \mid$ or \texttt{or}). 
    \item \textit{Implication} ($\implies$ or \texttt{implies}). 
    \item \textit{Alternative} ($,$ or \texttt{else}). 
    \item \textit{Coimplication} ($\iff$ or \texttt{iff}).
\end{itemize}
Alloy offers logic quantifiers:
\begin{itemize}
    \item \texttt{all}: holds for every element.
    \item \texttt{some}: holds for at least one element.
    \item \texttt{no}: holds for no elements.
    \item \texttt{lone}: holds for at most one element.
    \item \texttt{one}: holds for exactly one element.
\end{itemize}
For defining relations with singletons, you can use the following declaration \texttt{x: m e}, where \texttt{x} is the name of the relation, \texttt{m} is the multiplicity of the element and \texttt{e} is the name of the element within the relation. 
When the relation consists of pairs, the declaration appears as \texttt{r: Am }$\rightarrow$ \texttt{nB}, where \texttt{r} is the relation name, \texttt{A} and \texttt{B} denote the element names with multiplicities \texttt{m} and \texttt{n}, respectively.
    
Additionally, Alloy includes operators like $\#$ (counting the number of tuples in $r$), integers ($0,1,\dots$) for defining variable values, arithmetic operators ($+$ and $-$), and various comparison operators ($<$, $<=$, $=$, $=>$, $>$). 
There is also the \texttt{sum} operator, which adds all elements within a selected tuple.

Other useful keywords are: 
\begin{itemize}
    \item \texttt{let}: this keyword is utilized to establish local variables within a formula or expression. 
        These variables serve to store and reuse intermediate results, making it easier to simplify complex expressions.
    \item \texttt{enum}: this keyword is employed to define enumerations. 
        Enumerations enable the definition of a finite set of symbolic values, which can be utilized in models to represent various states, options, or categories.
    \item \texttt{var}: this keyword is used for declaring variables within an Alloy model or specification.
    \item \texttt{after}: this keyword is employed to specify the temporal ordering or sequence of events or states within a model. 
        It is commonly used in temporal logic to define the sequence of events that must occur before or after a specific state or action.
    \item \texttt{always}: this keyword is used to express a property that must remain true throughout a system's execution or under certain conditions.
        It is often utilized to specify invariant properties in Alloy models.
    \item \texttt{eventually}: this keyword is used to express temporal logic properties that indicate a particular condition or event will ultimately occur during the execution of a system. 
        It is commonly used to model and verify properties that describe what should happen at some point in the future.
    \item \texttt{historically}: keyword is used to express temporal logic properties that describe a condition or event that has remained true for a continuous duration of time leading up to the present or to a specified point in the model's execution.
        It is often employed to specify properties related to the historical behavior of a system.
    \item \texttt{before}. 
    \item \texttt{once}. 
\end{itemize}
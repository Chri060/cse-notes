\section{History}

In its early days, software was regarded as an art form. 
Computers primarily served the purpose of mathematical problem-solving, with the designers themselves acting as the end users. 
The initial programs were crafted using low-level languages and were subject to stringent resource constraints.

However, as the demand for customized software surged, this artistic endeavor transitioned into a more methodical craft. 
Developers began creating programs tailored for a broader audience, employing new high-level languages.
Towards the culmination of this era, a "software crisis" loomed, marked by the escalating complexity of software and a dearth of effective software development techniques.

With the intention of tackling this growing challenge, the term was coined during a NATO conference in 1968. 
The primary focus of this pivotal gathering encompassed the following key areas:
\begin{itemize}
    \item Development of software and the establishment of standards.
    \item Strategic planning and proficient management.
    \item Automation of software development processes.
    \item Modularization of software design.
    \item Rigorous quality assurance and verification.
\end{itemize}
\section{Requirements-level class diagrams}

The requirements-level class diagrams are conceptual models for the application domain. 
They may model objects that will not be represented in the software-to-be. 
Usually, they do not attach operations to objects: it's best to postpone this kind of decisions until software design. 
    
To find objects and classes we need to:
\begin{itemize}
    \item Analyze any description of the problem and application domain you may have.
    \item Analyze your scenarios and use cases descriptions.
\end{itemize}
Finding objects is the central piece in object modeling. 
A possible tool to use in the analysis is the Abbott Textual Analysis also called noun-verb analysis: nouns are good candidates for classes and verbs are good candidates for associations and operations. 
\begin{table}[H]
    \centering
    \begin{tabular}{ccc}
    \textbf{Example}                  & \textbf{Grammatical construct} & \textbf{UML component} \\ \hline
    "Monopoly"                        & concrete person, thing         & object                 \\
    "toy"                             & noun                           & class                  \\ \hline
    "3 years old"                     & adjective                      & attribute              \\ \hline
    "enters"                          & verb                           & operation              \\
    "depends on"                      & intransitive verb              & operation (event)      \\ \hline
    "is a", "either", "or", "kind of" & classifying verb               & inheritance            \\ \hline
    "has a", "consists of"            & possessive verb                & aggregation            \\ \hline
    "must be", "less than"            & modal verb                     & constraint             \\ \hline
    \end{tabular}
    \caption{Abbott textual analysis example}
\end{table}
\begin{abstract}
    The course centers on Transactional Systems, exploring their importance and relevance. 
    It introduces the concept of transactions and delves into the ACID properties that underpin them—atomicity, consistency, isolation, and durability.

    Key concepts in concurrency control theory are examined, including schedules, serializability, and various notions of equivalence and complexity in testing, along with view-serializability and conflict-serializability. 
    Advanced locking mechanisms such as two-phase locking, hierarchical locking, and deadlock analysis and resolution are discussed. 
    The course also covers timestamp-based concurrency control, multi-version concurrency control, and locking implementations in commercial systems.
    
    Database architecture topics include distributed databases, the concepts of fragmentation, allocation, and transparent access, query optimization, distributed transactions, and parallelism in database servers. 
    The differences between shared-memory and shared-nothing approaches, as well as scale-up, speed-up, and performance benchmarking, are covered. 
    Additionally, replicated databases are introduced, contrasting synchronous and asynchronous methods and symmetric versus primary-secondary replication strategies.
    
    An exploration of database server internals covers buffer and page management, data organization with sequential, direct, and indexed structures, and data access techniques such as B and B+ trees and hashing functions. 
    Access methods, including scans, ordering, and joins, are introduced, as well as fundamentals of query optimization, cost models, and execution plan selection using branch-and-bound methods.
    Topics in database administration and physical design, such as index and primary storage selection, are also addressed.
    
    Active databases are introduced with the Event-Condition-Action paradigm, execution methods, trigger languages, and rule analysis, covering formal properties of active rule sets, termination, confluence, and observable determinism.
    Active rule design is discussed for purposes such as integrity maintenance, automated data derivation, and enforcing business rules.
    
    Object-Relational Mapping concepts are presented, covering the mapping of entities, relationships, and hierarchies, the lifecycle of managed objects, and transactional support in business applications. 
    
    Finally, the course addresses ranking and its geometric interpretation in terms of data attributes, scoring functions, access methods, and algorithmic approaches to ranking. 
    It discusses concepts such as dominance, skylines, k-skybands, and optimality, emphasizing both theoretical and practical perspectives in data management and middleware scenarios.
\end{abstract}
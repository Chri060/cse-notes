\section{Schedule classification}

Determine the classification of the provided schedule with respect to the 2PL (Two-Phase Locking) and strict 2PL (Strict Two-Phase Locking) classes:
\[r_4(x) r_2(x) w_4(x) w_2(y) w_4(y) r_3(y) w_3(x) w_4(z) r_3(z) r_6(z) r_8(z) w_6(z) w_9(z) r_5(z) r_10(z)\]

\paragraph*{Solution}
In the context of strict 2PL, the assumption is made that all transactions commit and release all locks immediately after their final operation. 
The verification involves assessing whether these releases can be executed at the time of commit.
\begin{table}[H]
    \centering
    \begin{tabular}{c|ccccccccccccccc}
                & \textit{1} & \textit{2} & \textit{3} & \textit{4} & \textit{5} & \textit{6} & \textit{7} & \textit{8} & \textit{9} & \textit{10} & \textit{11} & \textit{12} & \textit{13} & \textit{14} & \textit{15} \\ \hline
    \textit{X} & $r_4$      & $r_2\searrow _2$          & $w_4$      &                              &            &            & $w_3$      &            &            &             &             &             &             &             &             \\
    \textit{Y} &            &                           &            & $w_2\downharpoonleft_2$      & $w_4$      & $r_3$      &            &            &            &             &             &             &             &             &             \\
    \textit{Z} &            &                           &            &                              &            &            &            & $w_4$      & $r_3$      & $r_6$       & $r_8$       & $w_6$       & $w_9$       & $r_5$       & $r_{10}$     
    \end{tabular}%
\end{table}
Hence, it is evident that the schedule cannot conform to strict 2PL. 
The conflict arises between transactions $T_2$ and $T_4$: while $T_2$ concludes after step 4, $T_4$ intends to write to $X$ at step 3.
As a result, $T_2$ would be compelled to release lock $X$ earlier, contradicting the requirement that it must maintain all locks until after step 4.

In the context of 2PL:
\begin{table}[H]
    \centering
    \resizebox{\columnwidth}{!}{%
    \begin{tabular}{c|ccccccccccccccc}
                & \textit{1} & \textit{2} & \textit{3} & \textit{4} & \textit{5} & \textit{6} & \textit{7} & \textit{8} & \textit{9} & \textit{10} & \textit{11} & \textit{12} & \textit{13} & \textit{14} & \textit{15} \\ \hline
    \textit{X} & $\nearrow_4r_4$        & $\nearrow_2r_2\searrow_2$         & $\nearrow_4w_4$       &                       & $\searrow_4$                          &                           & $\nearrow_3w_3$       &                       &                          & $\searrow_3$  &                       &             &             &             &             \\
    \textit{Y} &                        & $\nearrow_2$                      &                       & $w_2\searrow_2$       & $\nearrow_4w_4\searrow_4$             & $\nearrow_3r_3$           &                       &                       &                          & $\searrow_3$  &                       &             &             &             &             \\
    \textit{Z} &                        &                                   & $\nearrow_4$          &                       &                                       &                           &                       & $w_4\searrow_4$       & $\nearrow_3r_3$          & $r_6$         & $r_8\searrow_3$       & $w_6$       & $w_9$       & $r_5$       & $r_{10}$     
    \end{tabular}%
    }
\end{table}
We need to examine the acquisitions that must be anticipated and the releases that must be delayed to avoid violating the 2PL rules.
\begin{itemize}
    \item $T_4$ can only acquire the exclusive lock on $X$ after step 2 and on $Y$ after step 4. 
        It has to release $Y$ before step 6 and $X$ before step 7. 
        Consequently, the lock on $Z$ must be acquired before step 6.
    \item $T_2$ can acquire all locks at the beginning and release them immediately after each use.
    \item $T_3$ can acquire locks on $X$, $Y$, and $Z$ just before using them and release them all before step 12.
\end{itemize}
All other transactions ($T_6, T_9, T_5, T_{10}$) do not present any issues.
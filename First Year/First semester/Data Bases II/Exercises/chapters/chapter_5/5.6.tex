\section{Exercise 6}

Consider the following tables: 
\begin{itemize}
    \item USER(\underline{Email}, Password, LastName, FirstName): contains 128000 users and is stored on 25000 blocks, with a primary hash organization on the primary key.
    \item REVIEW(\underline{Email}, \underline{ISBN}, Date, Rating, ReviewText): has instead 4000000 tuples and is organized with a primary B+ tree with (email, ISBN) as a key; the average fan-out is equal to 35 and leaf nodes occupy 1000000 blocks.
\end{itemize}
Estimate the cost of joining the tables with the best technique.

\paragraph*{Solution}
The lookup cost for the USER table is one since it has a hash indexing. 
In the case of REVIEW we have that each user has an average of 32 reviews (4M/128K). 
Every user also have 9 nodes (1M leaf/128K tuples)
\begin{itemize}
    \item Indexed nested loop with USER as external:
        USER is external, so we have to check all its blocks (25000).
        REVIEW has a cost of 5 based on the levels, but since every user needs to be accessed 9 times (9 blocks associated to each user), we have that the cost for each user is given by the sum of $5-1$ levels and the number of leaf nodes. 
        The total final cost will be: 
        \[25000 + 128000 \cdot (4 + 9) = 1700000\]
    \item Indexed nested loop with REVIEW as external:
        REVIEW is external, so we have to check all its blocks (1000000) after scanning the 4 intermediate nodes, with a total of 1000004.
        In theory, we have to access 4 million tuples, but since the users are only 128000 we can do fewer accesses (all with unitary cost).
        The final cost will be: 
        \[4 + 1000000 + 128000 = 1130000\]
\end{itemize}
\section{Exercise 7}

Consider the following tables: 
CAR(\underline{PlateNo}, OwnerId, PurchaseDate) : 1000000 tuples primarily stored in a hash on PlateNo, with 20000 buckets, occupying 32K blocks overall (average cost per lookup: 1.6 I/O operations).
FINE(\underline{Id}, Date, PlateNo, MeasuredSpeed, PhotoRef) : 72000 fines given by a single speed camera in over 10 years, primarily stored sequentially ordered by Date on 4000 blocks.
Describe query plans and their costs for the query below in the following three scenarios: 
\begin{enumerate}
    \item There are no auxiliary structures. 
    \item Indexes: for FINE we have B+(Date), 3 levels, 2000 leaves and for CAR we have B+(PurchaseDate), 3 levels, 10000 leaves. 
    \item A hash index for Fine on PlateNo (same hash as Car, negligible overflow chains).
\end{enumerate}
Further data: val(PurchaseDate) = 5K, val(Date) = 3.6K, 75\% of the fines are for a speed over 70 km/h, 0.1\% of the fines are given to cars on their purchase date.
The query is: 
\begin{lstlisting}[style=SQL]
SELECT OwnerId 
FROM CAR C JOIN FINE F ON C.PlateNo = F.PlateNo
WHERE PurchaseDate = Date AND MeasuredSpeed > 70.0
\end{lstlisting}

\subsection*{Solution}
We have that the number of fines that are found by the query is equal to: 
\[72000 \cdot 0.75 \cdot 0.001=54\]
\begin{enumerate}
    \item In this case we scan the fines and lookup for the cars. 
        The cost required to perform this action is: 
        \[S_{fine}+L_{cars}=4000+\left(72000 \cdot 0.75\right) =90400\]
        We have considered only 75\% of the cars (and not only 54 fines) since we cannot filter on FINE. 
        The scan of CAR and FINE is not considered since the cost of scanning CAR is already greater than the cost of the cased analyzed. 
    \item The cost of a lookup on Car via the index is higher than via the hash, so a scan Fine and lookup Car option is dominated by the previous plan.
        A lookup on Fine retrieves the fines of a certain date (approx. 72K/val(Date) = 20 fines/day) and costs at least 3 I/O in the tree + 1 block with the 20 tuples (even more if the fines are split in consecutive blocks).
        The cost of a scan Car and lookup Fine is then at least:
        \[32000 + 1000000( 4 )> 4000000\]
    \item At join time there is no way to prune any plate/plate combination, so all fines must be retrieved (the index has no information on the date or the speed), and all 72000 blocks would be read one at a time. 
        Thus, cost of retrieving the fines is 72000:
        \[32000 + 20000 + 72000 = 124000\]
\end{enumerate}
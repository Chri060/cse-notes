\section{Exercise 2}

A table T(Name, Surname, City, Pathology, BloodType) has ten millions tuples and three multi-attribute indexes:
\begin{itemize}
    \item $IDX_1$(City, Name, Surname).
    \item $IDX_2$(Pathology, Name, BloodType).
    \item $IDX_3$(Pathology, Surname, City).
\end{itemize}
The access keys are made by the listed attributes from left to right. 
All attributes have a (reasonably) uniform and independent distribution, with:
\begin{itemize}
    \item $\text{val}(\text{Name}) = 1000$.
    \item $\text{val}(\text{Surname}) = 2000$.
    \item $\text{val}(\text{City}) = 1000$.
    \item $\text{val}(\text{Pathology}) = 100$.
    \item $\text{val}(\text{BloodType}) = 10$.
\end{itemize}
\begin{enumerate}
    \item Choose the most convenient index to evaluate the query:
        \begin{lstlisting}[style=SQL]
SELECT *
FROM T
WHERE Name = "N" AND Surname = "S" AND Pathology = "P" AND BloodType = "BT"
        \end{lstlisting}
    \item Choose the most convenient index to evaluate the same query, but with the following distribution: 
    \begin{itemize}
        \item $\text{val}(\text{Name}) = 1000$.
        \item $\text{val}(\text{Surname}) = 6000$.
        \item $\text{val}(\text{City}) = 1000$.
        \item $\text{val}(\text{Pathology}) = 100$.
        \item $\text{val}(\text{BloodType}) = 4$.
    \end{itemize}
\end{enumerate}

\paragraph*{Solution}
\begin{enumerate}
    \item The where clause is a conjunction of supported predicates.
        We estimate the selectivity allowed by each of the available indexes, only considering the predicates mentioned in the where clause:
        \begin{itemize}
            \item $IDX_1$: there is no condition on the name of the City, so in this case we will have to follow all the pointer starting with the cities name. 
                As a result it would force a full sequential scan of the table.
            \item $IDX_2$: the average number of candidate tuples is: 
                \[\dfrac{\left\lvert R \right\rvert }{\text{val}(\text{Pathology}) \cdot \text{val}(\text{Name}) \cdot \text{val}(\text{BloodType})}=\dfrac{10000000}{100 \cdot 1000 \cdot 10}=10\]
            \item $IDX_3$: the average number of candidate tuples is: 
                \[\dfrac{\left\lvert R \right\rvert }{\text{val}(\text{Pathology}) \cdot \text{val}(\text{Surname}) }=\dfrac{10000000}{2000 \cdot 100}=10\]
                We excluded the City attribute since there is no restriction on it. 
        \end{itemize}
        The best index is the more selective one, in this case we have that it is $IDX_2$. 
    \item We estimate the selectivity allowed by each of the available indexes, only considering the predicates mentioned in the where clause as before:
        \begin{itemize}
            \item $IDX_1$: useless for the same reasons as before. 
            \item $IDX_2$: the average number of candidate tuples is: 
                \[\dfrac{\left\lvert R \right\rvert }{\text{val}(\text{Pathology}) \cdot \text{val}(\text{Name}) \cdot \text{val}(\text{BloodType})}=\dfrac{10000000}{100 \cdot 1000 \cdot 4}=25\]
            \item $IDX_3$: the average number of candidate tuples is: 
                \[\dfrac{\left\lvert R \right\rvert }{\text{val}(\text{Pathology}) \cdot \text{val}(\text{Surname}) }=\dfrac{10000000}{100 \cdot 6000}=16.6\]
                We excluded the City attribute since there is no restriction on it. 
        \end{itemize}
        In this case, even if the third attribute of $IDX_3$ is of no use, its overall selectivity is higher, and therefore we choose $IDX_3$.
\end{enumerate}
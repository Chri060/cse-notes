\section{Exercise 19}

Explain why deadlocks can occur despite the presence of update locks. 

\subsection*{Solution}
While UL contribute to reducing the likelihood of deadlocks, they do not render deadlocks impossible.
Update locks specifically address one type of deadlock related to update patterns, such as when two transactions contend for the same resource ($r_1(x) r_2(x) w_1(x) w_2(x)$). 
However, in scenarios involving distinct resources, say $x$ and $y$, and transactions attempting to access them in the sequence $r_1(x) r_2(y) w_1(y) w_2(x)$, deadlocks may still occur.
This is particularly true in systems employing 2PL and when there is no update pattern involved.
In such cases, the effectiveness of update locks becomes irrelevant to preventing deadlock situations.
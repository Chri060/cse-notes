\section{Exercise 4}

Classify the given schedule with respect to CSR and VSR classes:
\[r_2(u) w_2(s) r_1(x) r_2(y) w_3(y) r_5(x) w_5(u) w_3(s) w_2(u) w_3(x) w_1(u) r_4(y) w_5(z) r_5(z)\]

\subsection*{Solution}
To analyze the schedule, we first divide it based on the resources:
\begin{itemize}
    \item $x: r_1 \: r_5 \: w_3$
    \item $y: r_2 \: w_3 \: r_4$
    \item $z: w_5 \: r_5$
    \item $s: w_2 \: w_3$
    \item $u: r_2 \: w_5 \: w_2 \: w_1$
\end{itemize}
The conflict graph is constructed based on write-write or write-read relations in the resource groups. 
The nodes are $\{1,2,3,4,5\}$, and the arcs are determined by the conflicts. 
As a result we have the following graph:
\begin{figure}[H]
    \centering
    \includegraphics[width=1.0\linewidth]{images/conflictgraph1.png}
\end{figure}
It becomes evident that a cycle exists between transactions two and five. 
According to the VSR definition, it is necessary to have the same reads-from relations and final writes.
To address this, we aim to identify a view-equivalent schedule that is also CSR. 
A viable solution involves a simple swap of the two writes on the resource $u$, effectively eliminating the cycle. 
Consequently, the modified schedule:
\[r_2(u) w_2(s) r_1(x) r_2(y) w_3(y) r_5(x) w_5(u) w_2(u) w_3(s) w_3(x) w_1(u) r_4(y) w_5(z) r_5(z)\]
is both CSR and VSR.
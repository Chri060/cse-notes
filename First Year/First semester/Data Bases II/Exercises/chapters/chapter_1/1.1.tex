\section{Exercise 1}

Check if the following schedules produce anomalies. 
$c_i$ and $a_i$ indicate the transactional decision commit and abort, respectively.
\begin{enumerate}
    \item $r_1(x) w_1(x) r_2(x) w_2(y)\:a_1\:c_2$
    \item $r_1(x) w_1(x) r_2(y) w_2(y)\:a_1\:c_2$
    \item $r_1(x) r_2(x) r_2(y) w_2(y) r_1(z)\:a_1\:c_2$
    \item $r_1(x) r_2(x) w_2(x) w_1(x)\:c_1\:c_2$
    \item $r_1(x) r_2(x) w_2(x) r_1(y)\:c_1\:c_2$
    \item $r_1(x) w_1(x) r_2(x) w_2(x)\:c_1\:c_2$
\end{enumerate}

\subsection*{Solution}
\begin{enumerate}
    \item This schedule exhibits a dirty read due to the abort of the first transaction, allowing the second transaction to read the modified value of $x$ before the abort.
    \item The schedule is free of anomalies as it represents a serial execution with transactions operating on different resources.
    \item No anomalies occur in this schedule since the last operation of the first transaction involves a different resource.
    \item Lost update anomaly is present as both transactions sequentially read and update the resource $x$ without considering the updated value by the other.
    \item Similar to case three, no anomalies arise because the last operation of the first transaction works on a different resource.
    \item This schedule is correct as it represents a serial execution without any anomalies.
\end{enumerate}
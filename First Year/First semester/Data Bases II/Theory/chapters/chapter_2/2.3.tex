\section{Concurrency theory}

\begin{definition}[\textit{Model}]
    A model is an abstraction of a system, object or process, which purposely disregards details to simplify the investigation of relevant properties. 
\end{definition}
Concurrency theory is founded on a model of transaction and concurrency control principles that aids in comprehending real systems.
Actual systems leverage implementation-level mechanisms to attain desirable properties postulated by the theory.
\begin{definition}[\textit{Operation}]
    An operation consist in a reading or in a writing of a specific datum by a specific transaction. 
\end{definition}
\begin{definition}[\textit{Schedule}]
    A schedule is a sequence of operations performed by concurrent transactions that respects the order of operations of each transaction. 
\end{definition}

\paragraph*{Schedule}
Transactions can be categorized as serial, interleaved, or nested. 
Let $N_S$ and $N_D$ be respectively the number of serial schedules and distinct schedules for $n$ transactions $\left\langle T_1,\dots,T_n\right\rangle $ each with $k_i$ operations. 
Then:
\[N_S=n!\]
\[N_D=\dfrac{\left( \sum_{i=1}^nk_i \right)!}{\prod_{i=1}^n \left( k_i! \right)}\]
\begin{example}
    For two transactions, $T_1$ and $T_2$, there are six possible schedules, with only two being serial:
    \begin{enumerate}
        \item $r_1(x) w_1(x) r_2(z) w_2(z) \rightarrow \textnormal{serial}$
        \item $r_2(z) w_2(z) r_1(x) w_1(x) \rightarrow \textnormal{serial}$
        \item $r_1(x) r_2(z) w_1(x) w_2(z) \rightarrow \textnormal{nested}$
        \item $r_2(z) r_1(x) w_2(z) w_1(x) \rightarrow \textnormal{nested}$
        \item $r_1(x) r_2(z) w_2(z) w_1(x) \rightarrow \textnormal{interleaved}$
        \item $r_2(z) r_1(x) w_1(x) w_2(z) \rightarrow \textnormal{interleaved}$
    \end{enumerate}
\end{example}
Concurrency control must reject all schedules that lead to anomalies.
\begin{definition}[\textit{Scheduler}]
    The scheduler is a component that accepts or rejects operations requested by the transactions. 
\end{definition}
\begin{definition}[\textit{Serial schedule}]
    The serial schedule is a schedule in which the actions of each transaction occur in a contiguous sequence.
\end{definition}
A serializable schedule leaves the database in the same state as some serial schedule of the same transactions, making it correct. 
To introduce other classes, two initial assumptions are made:
\begin{itemize}
    \item The transactions are observed a posteriori. 
    \item Commit-projection: consider only the committed transactions. 
\end{itemize}
\section{Skyline queries}

Skyline queries aim to identify superior objects across multiple perspectives, relying on the concept of dominance.
\begin{definition}[\textit{Domination}]
    A tuple $t$ dominates a tuple $s$ ($t \prec s$) if and only if $t$ in nowhere worse than $s$: 
    \[\forall i, 1 \leq i \leq m \rightarrow t[A_i] \leq s[A_i]\] 
    And is better at least once: 
    \[\exists j, 1 \leq j \leq m \land t[A_j] < s[A_j]\]
\end{definition}
\begin{definition}[\textit{Skyline}]    
    The skyline of a relation is the set of its non-dominated tuples.
\end{definition}

The convention on the skyline queries is that lower values are better than higher values.
The convention is that lower values are considered better. 
A tuple is part of the skyline if it is the top-1 result with respect to at least one monotone scoring function. 

\paragraph*{Query syntax}
The syntax for skyline queries can be expressed as follows:
\begin{lstlisting}[style=SQL]
SELECT <attributes>
FROM R1,R2,...,Rn
WHERE <conditions>
GROUP BY <conditions>
HAVING <conditions>
SKYLINE OF [DISTINCT] d1[MIN|MAX|DIFF], ..., dm[MIN|MAX|DIFF]
ORDER BY <conditions>
\end{lstlisting}
This query can be easily translated into a standard query, but the result is too slow and cannot be used in practice. 

\subsection{Block Nested Loop algorithm}
The Block Nested Loop algorithm takes a dataset $D$ of multidimensional points as input and outputs the skyline of $D$. 
Its complexity is $O(n^2)$, making it inefficient for large datasets.
\begin{algorithm}[H]
    \caption{Block nested loop algorithm}
        \begin{algorithmic}[1]
            \State $W \leftarrow \varnothing$
            \For {every point $p$ in $D$}
                \If {$p$ not dominated by any point in $W$}
                    \State remove from $W$ the points dominated by $p$
                    \State add $p$ to $W$
                \EndIf
            \EndFor
            \State \Return $W$
        \end{algorithmic}
\end{algorithm}

\subsection{Sort Filter Skyline algorithm}
The Sort Filter Skyline algorithm also takes a dataset $D$ of multidimensional points as input and outputs the skyline of $D$. 
It performs better than the Block Nested Loop algorithm, especially for large datasets, but still has a complexity of $O(n^2)$.
\begin{algorithm}[H]
    \caption{Sort filter skyline algorithm}
        \begin{algorithmic}[1]
            \State $S \leftarrow D$
            \State $W \leftarrow \varnothing$
            \For {every point $p$ in $S$}
                \If {$p$ not dominated by any point in $W$}
                    \State add $p$ to $W$
                \EndIf
            \EndFor
            \State \Return $W$
        \end{algorithmic}
\end{algorithm}
\begin{example}
    Consider the following unordered dataset: 
    \begin{table}[H]
        \centering
        \begin{tabular}{lcc}
        \textbf{Name}                 & \textbf{Cost} & \textbf{Complaints} \\ \hline
        \multicolumn{1}{l|}{Crillon}  & 0.25  & 0.1        \\
        \multicolumn{1}{l|}{Ibis}     & 0.08  & 0.3        \\
        \multicolumn{1}{l|}{Hilton}   & 0.175 & 0.3        \\
        \multicolumn{1}{l|}{Sheraton} & 0.2   & 0.2        \\
        \multicolumn{1}{l|}{Novotel}  & 0.15  & 0.1       
        \end{tabular}
    \end{table}
    The skyline, consisting of Novotel and Ibis, is identified by the algorithm.
\end{example}

\subsection{Summary}
Skyline queries are effective in identifying potentially interesting objects when user preferences are unknown. 
They are simple to use but return a large number of objects, and their computation is not highly efficient. 
Skyline queries are agnostic with respect to user preferences.

\paragraph*{K-skyband} 
To enhance skyline queries, the concept of $k$-skyband is introduced, where the result includes a set of tuples dominated by fewer than $k$ tuples.
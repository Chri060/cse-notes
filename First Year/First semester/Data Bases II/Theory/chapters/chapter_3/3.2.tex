\section{History}

In the 13th century, the problem of elections and polls was introduced. 
Borda proposed a solution involving assigning penalty points based on the position of candidates in a voter's ranking, aiming for the candidate with the lowest overall penalty to win. 
On the other hand, Condorcet suggested determining the winner as the candidate who defeats every other candidate in pairwise majority rule elections.
\begin{example}
    Given ten voters and three candidates with the following votes: 
    \begin{table}[H]
        \centering
        \begin{tabular}{|c|c|c|c|c|c|c|c|c|c|}
        \hline
        1 & 2 & 3 & 4 & 5 & 6 & 7 & 8 & 9 & 10 \\ \hline
        A & A & A & A & A & A & C & C & C & C  \\ 
        C & C & C & C & C & C & B & B & B & B  \\ 
        B & B & B & B & B & B & A & A & A & A  \\ \hline
        \end{tabular}
    \end{table}
    For Borda we have: 
    \begin{itemize}
        \item $A: 1 \cdot 6+3 \cdot 4 = 18$
        \item $B: 3 \cdot 6+2 \cdot 4 = 26$
        \item $C: 2 \cdot 6+1 \cdot 4 = 16$ (winner)
    \end{itemize}
    However, for Condorcet, A wins in pairwise majority. 
    The winner depends on the method used.
\end{example}
In 1950, Arrow proposed the axiomatic approach, defining aggregation as axioms.
He formulated Arrow's paradox, stating that no rank-order electoral system can satisfy the following fairness criteria simultaneously:
\begin{itemize}
    \item No dictatorship (nobody determines, alone, the group's preference). 
    \item If all prefer $X$ to $Y$, then the group prefers $X$ to $Y$. 
    \item If, for all voters, the preference between $X$ and $Y$ is unchanged, then the group preference between $X$ and $Y$ is unchanged. 
\end{itemize}
To address this paradox, researchers introduced the metric approach.
This involves finding a new ranking $R$ minimizing the total distance to the initial rankings $R_1,\dots,R_n$.
Distances between rankings can be measured using methods such as: 
\begin{itemize}
    \item Kendall tau distance $K(R_1, R_2)$, defined as the number of exchanges in a bubble sort to convert $R_1$ to $R_2$. 
    \item Spearman's foot-rule distance $F(R_1, R_2)$, which adds up the distance between the ranks of the same item in the two rankings. 
\end{itemize}
Finding an exact solution is computationally hard for Kendall tau ($\mathcal{NP}$-complete), but tractable for Spearman's foot-rule ($\mathcal{P}$ time). 
These distances are related as follows:
\[K(R_1, R_2) \leq F(R_1, R_2) \leq 2K(R_1, R_2)\]
Efficient approximations for $F(R_1, R_2)$ are also possible.
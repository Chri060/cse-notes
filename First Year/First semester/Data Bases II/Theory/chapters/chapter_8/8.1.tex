\section{Introduction}

The demand for multidimensional data emerged in the 1970s, primarily driven by the need to manage Geographic Information Systems (GIS) and Computer-Aided Design (CAD).
Two decades later, the inception of multimedia databases and the introduction of data mining techniques marked significant milestones.
These advancements paved the way for the evolution of spatial Database Management Systems (DBMS).
The application of this technique extends beyond spatial databases, finding utility in classical databases featuring multiple numerical attributes.

\paragraph*{Applications}
The applications of multidimensional data are diverse, encompassing:
\begin{itemize}
    \item \textit{Geographic Information Systems (GIS)}: in GIS applications, the data includes both points (such as places and cities) and objects with spatial extensions (e.g., regions, streets, and rivers).
    \item \textit{Multimedia databases}: multimedia content is represented by numerical characteristics known as features. 
        In this context, two pieces of content are considered equal if their respective features are similar.
    \item \textit{Relational databases}: this technique is employed to identify the optimal combination of multiple attributes based on user or system criteria.
\end{itemize}

\paragraph*{Query classification}
Multidimensional databases support various query types, including:
\begin{itemize}
    \item \textit{Point/lookup query}: examines the contents of a single point in space, useful for detecting duplicates within a specific location.
    \item \textit{Window query}: combines two different attributes to inspect all elements within a triangular region.
    \item \textit{Range query}: starting from a point and a radius, it searches for all elements within a circular area with the specified center and radius.
    \item \textit{K-nearest neighbor}: given a point in space, this query type identifies the best k nearest elements based on a defined criterion.
    \item \textit{Specific query types for GIS systems}: examples include queries related to streets crossing a railroad or buildings in proximity to a river.
\end{itemize}

\paragraph*{Using B+ trees}
Consider a window query involving two attributes, $A$ and $B$.
Assume that each query interval selects 10\%of the total number of points, resulting in an expectation of retrieving 1\% of the data when the attributes are independent.
When employing B+ trees, two potential solutions arise:
\begin{enumerate}
    \item Implement a single B+ tree for both attributes by sorting first on $A$ and then on $B$. 
        However, this approach leads to reading numerous leaves that do not contain any points in the result set.
    \item Use two separate B+ trees for each attribute. 
        In this scenario, 10\% of leaves are read on both indices, yet the intersection only contains 1\% of the data.
\end{enumerate}
In both cases, excessive effort is wasted due to the storage of spatially close points in distant leaves.
In the first case, this is attributed to the linearization of attributes, while in the second case, it is a consequence of disregarding other attributes.
As additional attributes are introduced, performance deterioration persists.

To address these challenges, multidimensional indices can be employed.
These specialized indices aim to maintain the spatial proximity of records (Local Order Preservation) by grouping objects in pages, ensuring that each page contains objects that are close in the $d$-dimensional space.
This task is non-trivial, given that a global order is not defined in $d$ dimensions.
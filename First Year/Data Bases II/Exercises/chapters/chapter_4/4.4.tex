\section{Ranking and skyline queries}

Consider the following lists reporting statistics about soccer players.
\begin{table}[H]
    \centering
    \begin{tabular}{cc|cc}
    \hline
    \textbf{Goals} & \textbf{Player} & \textbf{Assists} & \textbf{Player} \\ \hline
    25             & LEW             & 17               & DEB             \\
    21             & RON             & 15               & SAN             \\
    19             & MES             & 12               & MES             \\
    18             & MBA             & 6                & NEY             \\
    15             & ILI             & 5                & MBA             \\
    14             & SAN             & 5                & ILI             \\
    13             & NEY             & 3                & RON             \\
    8              & DEB             & 3                & LEW             \\ \hline
    \end{tabular}
\end{table}
\begin{enumerate}
    \item Apply the TA to determine the top-$2$ players according to the scoring function: 
        \[S(\textnormal{player})=\textnormal{player.Goals}+\textnormal{player.Assists}\]
    \item Reuse as much as possible the answer of the previous question to indicate the 
        reached depth and the number of sorted and random accesses to determine, with TA,
        the top-$1$ player according to the same scoring function as the previous point.
    \item Discuss whether any algorithm could attain a lower execution cost than the cost 
        incurred by TA to determine the top-$1$ player. 
    \item Apply now the TA to determine the top-$2$ players according to the scoring function: 
        \[S(\textnormal{player})=\left\lvert \textnormal{player.Goals}-\textnormal{player.Assists} \right\rvert\]
    \item Determine the skyline of the dataset. 
\end{enumerate}

\paragraph*{Solution}
\begin{enumerate}
    \item The execution of the TA algorithm needs four rounds to stop on the given dataset. At the fourth 
        round we have the following buffer. 
        \begin{table}[H]
            \centering
            \begin{tabular}{c|cc|c}
            \hline
            \textbf{Player} & \textbf{Goals} & \textbf{Assists} & \textbf{Score} \\ \hline
            MES             & 19             & 12               & 31             \\
            SAN             & 14             & 15               & 29             \\ \hline
            \end{tabular}
        \end{table}
        With a threshold value of $24$. We found that the best two players for the given scoring function
        are MES and SAN. 
    \item The procedure is the same as the previous point, but it stops at the third round since the threshold 
        condition is already verified at this step. The buffer is the following.
        \begin{table}[H]
            \centering
            \begin{tabular}{c|cc|c}
            \hline
            \textbf{Player} & \textbf{Goals} & \textbf{Assists} & \textbf{Score} \\ \hline
            MES             & 19             & 12               & 31             \\ \hline
            \end{tabular}
        \end{table}
        With a threshold of $31$. We have found that the best player is MES. 
    \item FA would stop at depth three as well, but only needs four random accesses to find the top player. 
    \item The given function is not monotone, so the TA can return wrong results. In this case the algorithm 
        will fail, returning DEB instead of RON. 
    \item After sorting the dataset we insert MES in the window. The only players that are not dominated by him
        are: SAN, LEW, and DEB. So, we have that the skyline is composed by these three players. 
\end{enumerate}
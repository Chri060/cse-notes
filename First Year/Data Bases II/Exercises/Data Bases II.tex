\documentclass[12pt, a4paper]{report}
\usepackage{graphicx, array, amsthm, amssymb, amsmath, algorithm, algpseudocode, float, xcolor, thmtools, thmbox}
\usepackage[english]{babel}

\makeatletter
\renewcommand\thmbox@headstyle[2]{\bfseries #1}
\makeatother
\newtheorem[style=M,bodystyle=\normalfont]{theorem}{Theorem}
\newtheorem[style=M,bodystyle=\normalfont]{corollary}{Corollary}
\newtheorem[style=M,bodystyle=\normalfont]{lemma}{Lemma}
\newtheorem[style=M,bodystyle=\normalfont]{definition}{Definition}


\title{Data Bases II \\ \textit{Exercises}}
\author{Christian Rossi}
\date{Academic Year 2023-2024}

\begin{document}

\maketitle

\newpage

\begin{abstract}
    The course aims to prepare software designers on the effective development of database applications. First, the course presents the fundamental features of current database 
    architectures, with a specific emphasis on the concept of transaction and its realization in centralized and distributed systems. Then, the course illustrates the main directions 
    in the evolution of database systems, presenting approaches that go beyond the relational model, like active databases, object systems and XML data management solutions.
\end{abstract}

\newpage

\tableofcontents

\newpage


    
\end{document}
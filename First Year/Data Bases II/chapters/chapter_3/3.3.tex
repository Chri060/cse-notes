\section{Opaque rankings}

The concept of opaque rankings focuses only on positions without considering associated scores.

\paragraph*{MedRank algorithm}
The MedRank algorithm, inspired by the median, offers an approximation of the foot-rule optimal aggregation.
The algorithm takes an integer $k$ and a ranked list $R_1,\dots,R_m$ of $N$ elements as inputs, producing the top $k$ elements based on the median ranking.
The algorithm proceeds as follows:
\begin{enumerate}
    \item Use sorted accesses in each list, extracting one element at a time until there are $k$ elements occurring in more than $m/2$ lists.
    \item Identify these as the top $k$ elements.
\end{enumerate}
\begin{definition}
    The maximum number of sorted accesses made on each list is called the \emph{depth reached} by the algorithm. 
\end{definition}
\begin{example}
    Consider sorting hotels based on three criteria (price, rating, and distance) using the MedRank algorithm. 
    With opaque ranks, the hotel ranks are as follows:
    \begin{table}[H]
        \centering
        \begin{tabular}{c|c|c}
        \textbf{Price} & \textbf{Rating} & \textbf{Distance} \\ \hline
        Ibis           & Crillon         & Le Roch           \\
        Etap           & Novotel         & Lodge In          \\
        Novotel        & Sheraton        & Ritz              \\
        Mercure        & Hilton          & Lutetia           \\
        Hilton         & Ibis            & Novotel           \\
        Sheraton       & Ritz            & Sheraton          \\
        Crillon        & Lutetia         & Mercure           \\
        $\dots$        & $\dots$         & $\dots$          
        \end{tabular}
    \end{table}
    Using MedRank with $k=3$, the resulting rank is:
    \begin{table}[H]
        \centering
        \begin{tabular}{cc}
        \hline
        \textbf{Top k hotels}       & \textbf{Median rank}          \\ \hline
        Novotel                     & median$\{2,3,5\}=3$           \\ 
        Hilton                      & median$\{4,5,?\}=5$           \\ 
        Ibis                        & median$\{1,5,?\}=5$           \\ \hline
        \end{tabular}
    \end{table}
    The depth in this case is equal to five.
\end{example}
\begin{definition}
    An algorithm is \emph{optimal} if its execution cost is never worse than any other algorithm on any input.

    An algorithm is \emph{instance-optimal} if is the best possible algorithm on every input of a specific instance. 
\end{definition}
\begin{property}
    MedRank is instace-optimal on ordered lists.
\end{property}
\begin{definition}
    Let $A$ be a family of algorithms, $I$ a set of problem instances. 
    Let cost be a cost metric applied to an algorithm-instance pair. 
    Algorithm $A^{*}$ is \emph{instance-optimal} with respect to $A$ and $I$ for the cost metric cost if there exist constants $k_1$ and $k_2$ such that, for all $A \in A$ and $I \in I$: 
    \[\textnormal{cost}(A^{*}, I) \leq k_1 \textnormal{cost}(A, I) + k_2\]
\end{definition}
If $A^{*}$ is instance-optimal, then any algorithm can improve with respect to $A^{*}$ by only a constant factor $r$, known as the optimality ratio of $A^{*}$.
Instance optimality is a much stronger notion than optimality in the average or worst case.
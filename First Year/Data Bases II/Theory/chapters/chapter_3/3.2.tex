\section{History}

In the $13^{\textnormal{th}}$ century this problem was introduced with the elections polls. Borda proposed to give a number of penalty points equivalent to the position 
of the electors in a voter's ranking, so the winner will be the candidate with the lowest overall penalty. Condorcet, instead, proposed to consider as winner the candidate who 
defeats every other candidate in pairwise majority rule election.
\begin{example}
    Given ten voters and three candidates with the following votes: 
    \begin{table}[H]
        \centering
        \begin{tabular}{|c|c|c|c|c|c|c|c|c|c|}
        \hline
        1 & 2 & 3 & 4 & 5 & 6 & 7 & 8 & 9 & 10 \\ \hline
        A & A & A & A & A & A & C & C & C & C  \\ 
        C & C & C & C & C & C & B & B & B & B  \\ 
        B & B & B & B & B & B & A & A & A & A  \\ \hline
        \end{tabular}
    \end{table}
    For Borda we have: 
    \begin{itemize}
        \item $A: 1 \cdot 6+3 \cdot 4 = 18$
        \item $B: 3 \cdot 6+2 \cdot 4 = 26$
        \item $C: 2 \cdot 6+1 \cdot 4 = 16$ (winner)
    \end{itemize}
    While for Condorcet we have that $A$ wins in pairwise majority. So, the winner depends on the method used. 
\end{example}

In 1950 Arrow proposed the axiomatic approach. He defined the aggregation as axioms and understood that a small set of natural requirements cannot be simultaneously achieved by 
any nontrivial aggregation function. So, he stated the Arrow's paradox: no rank-order electoral system can be designed that always satisfies these fairness criteria:
\begin{itemize}
    \item No dictatorship (nobody determines, alone, the group's preference). 
    \item If all prefer $X$ to $Y$, then the group prefers $X$ to $Y$. 
    \item If, for all voters, the preference between $X$ and $Y$ is unchanged, then the group preference between $X$ and $Y$ is unchanged. 
\end{itemize}

To solve this paradox, in the later years researchers tried to measure the values of all analyzed objects and created the metric approach. This consist in finding a new ranking 
$R$ whose total distance to the initial rankings $R_1,\dots,R_n$ is minimized. The distance between rankings can be found in several ways: 
\begin{itemize}
    \item Kendall tau distance $K(R_1, R_2)$, defined as the number of exchanges in a bubble sort to convert $R_1$ to $R_2$. 
    \item Spearman's foot-rule distance $F(R_1, R_2)$, which adds up the distance between the ranks of the same item in the two rankings. 
\end{itemize}
Finding an exact solution is computationally hard for Kendall tau (NP-complete), but tractable for Spearman's foot-rule (P time). These distances are related:
\[K(R_1, R_2) \leq F(R_1, R_2) \leq 2K(R_1, R_2)\]
And it is possible to find efficient approximation for $F(R_1, R_2)$. 
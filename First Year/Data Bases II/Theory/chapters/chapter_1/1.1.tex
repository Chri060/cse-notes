\section{Data Base Management System}

\paragraph*{DBMS}
A Data Base Management System is a software product capable of managing data collections that are: 
\begin{itemize}
    \item Large: much larger than the central memory available on the computers that run the software. 
    \item Persistent: with a lifetime which is independent of single executions of the programs that access them. 
    \item Shared: used by several applications at a time. 
    \item Reliable: ensuring tolerance to hardware and software failures. 
    \item Data ownership respectful: by disciplining and controlling accesses. 
\end{itemize}

\paragraph*{Evolution}
The key developments in DBMS innovation are highlighted in the chronological timeline below: 

\begin{chronology}[5]{1990}{2020}{0.9\textwidth}
    \event{1992}{SQL '92}
    \event{1999}{SQL '99}
    \event{2001}{Ranking in databases}
    \event{2003}{XML-related features}
    \event{2005}{NoSQL}
    \event{2006}{X-Query}
    \event{2009}{JPA final release}
    \event{2011}{Temporal databases}
    \event{2016}{JSON}
\end{chronology}

\paragraph*{Architecture} 
The architecture of a Database Management System (DBMS) is depicted in the figure below:
\begin{figure}[H]
    \centering
    \includegraphics[width=0.5\linewidth]{images/architecture.png}
    \caption{Architecture of a Data Base Management System}
\end{figure}
\section{Concurrency control in practice}

Checking conflict-serializability would be efficient if the conflict graph were known from the outset, but in practice, it is typically not.
Therefore, a scheduler must work online. 
It is impractical to maintain, update, and check the conflict graph at each operation request. 
Additionally, the assumption that concurrency control can rely solely on the commit-projection of the schedule is unrealistic because aborts do occur.
Thus, an online scheduler needs a simple decision criterion that avoids as many anomalies as possible with minimal overhead.

\paragraph*{Online concurrency control}
In the realm of online concurrency control, considering arrival sequences is crucial.
The concurrency control system translates an arrival sequence into an effective a posteriori schedule.
Two main families of techniques are commonly employed for online scheduling:
\begin{itemize}
    \item Pessimistic techniques (\textit{locks}): if a resource is taken, make the requester wait or pre-empt the holder.
    \item Optimistic techniques (\textit{timestamps}): serve as many requests as possible, possibly using out-of-date versions of the data. 
\end{itemize}
In practice, commercial systems often combine elements from both pessimistic and optimistic approaches to leverage the strengths of each.
\section{Multi-version timestamps}

The idea of multi-version is that writes generate new versions, and that reads access the right version. Writes generate new copies, each one with a new WTM, so each object $x$ 
always has $N \geq 1$ active versions. There is a unique global RTM($x$). Old versions are discarded when there are no transactions that need their values. 
In theory, it is possible to use the following rule: 
\begin{itemize}
    \item $r_{ts}(x)$ is always accepted. A copy $x_k$ is selected for reading such that:
        \begin{itemize}
            \item If $ts \geq \textnormal{WTM}_N(x)$, then $k=N$.
            \item Else take $k$ such that $\textnormal{WTM}_k(x) \leq ts < \textnormal{WTM}_{k+1}(x)$. 
        \end{itemize}
    \item $w_{ts}(x)$: 
        \begin{itemize}
            \item If $ts < \textnormal{RTM}(x)$ the request is rejected. 
            \item Else a new version is created for timestamp $ts$ ($N$ is incremented). 
        \end{itemize}
\end{itemize}
\begin{example}
    Let us assume $\textnormal{RTM}(x)=7$, $N=1$ and $\textnormal{WTM}_1(x)=4$ and the following schedule: 
    \[S=r_6(x) r_8(x) r_9(x) w_8(x) w_{11}(x) r_{10}(x) r_{12}(x) w_{14}(x) w_{13}(x)\]
    Using the multi-version we obtain: 
    \begin{table}[H]
        \centering
        \begin{tabular}{ccc}
        \textbf{Request} & \textbf{Response}         & \textbf{New value}  \\ \hline
        $r_6(x)$         & $\checkmark$              & -                   \\
        $r_8(x)$         & $\checkmark$              & $\textnormal{RTM}(x)=8$          \\
        $r_9(x)$         & $\checkmark$              & $\textnormal{RTM}(x)=9$          \\
        $w_8(x)$         & $\tikzxmark$              & $T_8$ killed        \\
        $w_{11}(x)$      & $\checkmark$              & $\textnormal{WTM}_2(x)=11,\:N=2$ \\
        $r_{10}(x)$      & $\checkmark$ on $x_{(1)}$ & $\textnormal{RTM}(x)=10$         \\
        $r_{12}(x)$      & $\checkmark$ on $x_{(2)}$ & $\textnormal{RTM}(x)=12$         \\
        $w_{14}(x)$      & $\checkmark$              & $\textnormal{WTM}_3(x)=14,\:N=3$ \\
        $w_{13}(x)$      & $\checkmark$              & $\textnormal{WTM}_4(x)=14,\:N=4$
        \end{tabular}
    \end{table}
\end{example}

In practice, it is possible to use the following rule: 
\begin{itemize}
    \item $r_{ts}(x)$ is always accepted. A copy $x_k$ is selected for reading such that:
        \begin{itemize}
            \item If $ts \geq \textnormal{WTM}_N(x)$, then $k=N$. 
            \item Else take $k$ such that $\textnormal{WTM}_k(x) \leq ts < \textnormal{WTM}_{k+1}(x)$. 
        \end{itemize}
    \item $w_{ts}(x)$:
        \begin{itemize}
            \item If $ts < \textnormal{RTM}(x)$ or $ts < \textnormal{WTM}_N(x)$ the request is rejected. 
            \item Else a new version is created for timestamp $ts$ ($N$ is incremented)
        \end{itemize}
\end{itemize}
\begin{example}
    Let us assume $\textnormal{RTM}(x)=7$, $N=1$ and $\textnormal{WTM}_1(x)=4$ and the following schedule: 
    \[S=r_6(x) r_8(x) r_9(x) w_8(x) w_{11}(x) r_{10}(x) r_{12}(x) w_{14}(x) w_{13}(x)\]
    Using the multi-version we obtain: 
    \begin{table}[H]
        \centering
        \begin{tabular}{ccc}
        \textbf{Request} & \textbf{Response}         & \textbf{New value}  \\ \hline
        $r_6(x)$         & $\checkmark$              & -                   \\
        $r_8(x)$         & $\checkmark$              & $\textnormal{RTM}(x)=8$          \\
        $r_9(x)$         & $\checkmark$              & $\textnormal{RTM}(x)=9$          \\
        $w_8(x)$         & $\tikzxmark$              & $T_8$ killed        \\
        $w_{11}(x)$      & $\checkmark$              & $\textnormal{WTM}_2(x)=11,\:N=2$ \\
        $r_{10}(x)$      & $\checkmark$ on $x_{(1)}$ & $\textnormal{RTM}(x)=10$         \\
        $r_{12}(x)$      & $\checkmark$ on $x_{(2)}$ & $\textnormal{RTM}(x)=12$         \\
        $w_{14}(x)$      & $\checkmark$              & $\textnormal{WTM}_3(x)=14,\:N=3$ \\
        $w_{13}(x)$      & $\tikzxmark$              & $T_{13}$ killed
        \end{tabular}
    \end{table}
\end{example}
\subsection*{Isolation level introduced with TS-multi}
The realization of TS-multi gives the opportunity to introduce into the DBMS another isolation level called snapshot isolation. In this level only WTM($x$) is used. The rule 
used in this level states that every transaction reads the version consistent with its timestamp, and defers writes to the end. If the scheduler detects that the writes of a
transaction conflict with writes of other concurrent transactions after the snapshot timestamp, it aborts. Snapshot isolation does not guarantee serializability: it is possible 
to have a new anomaly called write skew (non-determinism). 
\section{Multi-version timestamps}

The concept of multi-versioning involves generating new versions with each write operation, and reads access the relevant version.
Each write produces new copies, each with a new Write Timestamp (WTM($x$)), ensuring that each object $x$ always has $N \geq 1$ active versions.
A unique global Read Timestamp (RTM($x$)) is maintained, and old versions are discarded when there are no transactions requiring their values.
In theory, the following rules can be applied:
\begin{itemize}
    \item $r_{ts}(x)$ is always accepted. A copy $x_k$ is selected for reading, where:
        \begin{itemize}
            \item If $ts \geq \textnormal{WTM}_N(x)$, then $k=N$.
            \item Otherwise, $k$ is chosen such that $\textnormal{WTM}_k(x) \leq ts < \textnormal{WTM}_{k+1}(x)$. 
        \end{itemize}
    \item $w_{ts}(x)$: 
        \begin{itemize}
            \item If $ts < \textnormal{RTM}(x)$, the request is rejected. 
            \item  Otherwise, a new version is created for timestamp $ts$ (incrementing $N$).
        \end{itemize}
\end{itemize}
\begin{example}
    Assuming  $\textnormal{RTM}(x)=7$, $N=1$ and $\textnormal{WTM}_1(x)=4$, consider the schedule:
    \[S=r_6(x) r_8(x) r_9(x) w_8(x) w_{11}(x) r_{10}(x) r_{12}(x) w_{14}(x) w_{13}(x)\]
    Using multi-versioning, the results are:
    \begin{table}[H]
        \centering
        \begin{tabular}{ccc}
        \textbf{Request} & \textbf{Response}         & \textbf{New value}  \\ \hline
        $r_6(x)$         & $\checkmark$              & -                   \\
        $r_8(x)$         & $\checkmark$              & $\textnormal{RTM}(x)=8$          \\
        $r_9(x)$         & $\checkmark$              & $\textnormal{RTM}(x)=9$          \\
        $w_8(x)$         & $\tikzxmark$              & $T_8$ killed        \\
        $w_{11}(x)$      & $\checkmark$              & $\textnormal{WTM}_2(x)=11,\:N=2$ \\
        $r_{10}(x)$      & $\checkmark$ on $x_{(1)}$ & $\textnormal{RTM}(x)=10$         \\
        $r_{12}(x)$      & $\checkmark$ on $x_{(2)}$ & $\textnormal{RTM}(x)=12$         \\
        $w_{14}(x)$      & $\checkmark$              & $\textnormal{WTM}_3(x)=14,\:N=3$ \\
        $w_{13}(x)$      & $\checkmark$              & $\textnormal{WTM}_4(x)=14,\:N=4$
        \end{tabular}
    \end{table}
\end{example}

In practice, the rule set is modified slightly:
\begin{itemize}
    \item $r_{ts}(x)$ is always accepted. A copy $x_k$ is selected for reading, where:
        \begin{itemize}
            \item If $ts \geq \textnormal{WTM}_N(x)$, then $k=N$. 
            \item  Otherwise, $k$ is chosen such that $\textnormal{WTM}_k(x) \leq ts < \textnormal{WTM}_{k+1}(x)$. 
        \end{itemize}
    \item $w_{ts}(x)$:
        \begin{itemize}
            \item If $ts < \textnormal{RTM}(x)$ or $ts < \textnormal{WTM}_N(x)$, the request is rejected. 
            \item  Otherwise, a new version is created for timestamp $ts$ (incrementing $N$).
        \end{itemize}
\end{itemize}
\begin{example}
    Assuming $\textnormal{RTM}(x)=7$, $N=1$ and $\textnormal{WTM}_1(x)=4$, consider the schedule:
    \[S=r_6(x) r_8(x) r_9(x) w_8(x) w_{11}(x) r_{10}(x) r_{12}(x) w_{14}(x) w_{13}(x)\]
    Using multi-versioning, the results are:
    \begin{table}[H]
        \centering
        \begin{tabular}{ccc}
        \textbf{Request} & \textbf{Response}         & \textbf{New value}  \\ \hline
        $r_6(x)$         & $\checkmark$              & -                   \\
        $r_8(x)$         & $\checkmark$              & $\textnormal{RTM}(x)=8$          \\
        $r_9(x)$         & $\checkmark$              & $\textnormal{RTM}(x)=9$          \\
        $w_8(x)$         & $\tikzxmark$              & $T_8$ killed        \\
        $w_{11}(x)$      & $\checkmark$              & $\textnormal{WTM}_2(x)=11,\:N=2$ \\
        $r_{10}(x)$      & $\checkmark$ on $x_{(1)}$ & $\textnormal{RTM}(x)=10$         \\
        $r_{12}(x)$      & $\checkmark$ on $x_{(2)}$ & $\textnormal{RTM}(x)=12$         \\
        $w_{14}(x)$      & $\checkmark$              & $\textnormal{WTM}_3(x)=14,\:N=3$ \\
        $w_{13}(x)$      & $\tikzxmark$              & $T_{13}$ killed
        \end{tabular}
    \end{table}
\end{example}

\subsection*{Isolation level introduced with TS-multi}
The implementation of TS-multi opens the door to introducing another isolation level in the database management system (DBMS), known as snapshot isolation.
In this level, only Write Timestamp (WTM($x$)) is utilized.
The rule applied in snapshot isolation dictates that every transaction reads the version consistent with its timestamp and defers writes until the end. 
If the scheduler detects conflicts between the writes of a transaction and the writes of other concurrent transactions after the snapshot timestamp, it aborts.
It's essential to note that while snapshot isolation provides certain guarantees, it does not ensure serializability, and a new anomaly known as write skew (non-determinism) can occur.
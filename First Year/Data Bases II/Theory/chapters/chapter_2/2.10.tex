\section{Isolation levels in SQL '99}

SQL defines transaction isolation levels that specify the anomalies to be prevented when running at each level:

\begin{table}[H]
    \centering      
    \begin{tabular}{c|ccc|}
    \cline{2-4}
                                                    & \textbf{Dirty read} & \textbf{Non-repeatable read} & \textbf{Phantoms}    \\ \hline
    \multicolumn{1}{|c|}{\textbf{Read uncommitted}} & $\checkmark$        & $\checkmark$                 & $\checkmark$         \\
    \multicolumn{1}{|c|}{\textbf{Read committed}}   & $\tikzxmark$        & $\checkmark$                 & $\checkmark$         \\
    \multicolumn{1}{|c|}{\textbf{Repeatable reads}}  & $\tikzxmark$        & $\tikzxmark$                 & $\checkmark$(insert) \\
    \multicolumn{1}{|c|}{\textbf{Serializable}}     & $\tikzxmark$        & $\tikzxmark$                 & $\tikzxmark$         \\ \hline
    \end{tabular}
\end{table}
The four levels are implemented respectively with: no read locks, normal read locks, strict read locks, and strict locks with predicate locks. 
"Serializable" is not the default because its strictness can lead to the following problems:
\begin{itemize}
    \item Deadlock: two or more transactions are in endless mutual wait. 
    \item Starvation: a single transaction is in endless wait. 
\end{itemize}
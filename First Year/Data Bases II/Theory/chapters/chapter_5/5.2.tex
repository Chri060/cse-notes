\section{Introduction}

Triggers are built on the Event-Condition-Action paradigm, where an action A is automatically executed whenever a specific event E occurs, contingent upon the truth of a condition C.
To be more specific, the components of triggers can be described as follows:
\begin{itemize}
    \item \textit{Event}: typically, this pertains to a change in the database's status, encompassing operations like insertions, deletions, and updates.
    \item \textit{Condition}: this is a predicate that defines the specific circumstances in which the trigger's action must be executed.
    \item \textit{Action}: the action entails a general update statement or a stored procedure, typically involving database modifications such as insertions, deletions, and updates. 
        It may also encompass error notifications.
\end{itemize}

Triggers operate alongside integrity constraints and provide the capability to assess intricate conditions. 
These triggers are compiled and stored within the DBMS much like stored procedures. 
However, unlike stored procedures that are invoked by the client, triggers execute automatically in response to predefined events.
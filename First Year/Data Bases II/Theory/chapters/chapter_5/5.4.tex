\section{Triggers application}

Triggers introduce robust data management capabilities in a seamless and reusable fashion. 
This system empowers databases to incorporate business and management rules that would otherwise be scattered across various applications. 
Nevertheless, comprehending the interplay between triggers can be intricate, as certain database management system  providers leverage triggers to execute internal functions.

\paragraph*{View materialization}
A view is a virtual table defined through a query stored in the database catalog and subsequently utilized in queries as if it were a conventional table. 
When a view is referenced in a "select" query, the query processor revises the query by employing the view definition, ensuring that the executed query solely involves the base tables linked to the view.   
If the queries involving a view significantly outnumber the updates to the base tables that alter the view's contents, view materialization can be considered.  
Certain systems offer the "create materialized view" command, which enables the DBMS to automatically materialize the view. 
Alternatively, materialization can be implemented through the use of triggers.

\paragraph*{Design principles}
The design principles encompass the following guidelines:
\begin{enumerate}
    \item Employ triggers to ensure that specific operations trigger related actions.
    \item Avoid defining triggers that replicate functionality already inherent in the DBMS. 
    \item Keep trigger code concise. 
        If your trigger's logic extends beyond 60 lines of code, consider placing the majority of the code within a stored procedure and invoke the procedure from the trigger.
    \item Utilize triggers exclusively for centralized, global operations intended to be executed for the triggering statement, regardless of the issuing user or database application.
    \item Minimize the use of recursive triggers unless absolutely necessary, as triggers may inadvertently trigger one another until the DBMS exhausts its memory.
    \item Exercise caution when implementing triggers, as they are executed for every user each time the relevant trigger event occurs.
\end{enumerate}

\subsection{Summary}
All prominent relational DBMS vendors offer varying degrees of support for triggers. 
However, it's important to note that most products provide support for only a subset of the SQL-99 trigger standard, and they may not fully adhere to some of the more intricate aspects of the execution model. 
Additionally, certain trigger implementations rely on proprietary programming languages, making portability across different DBMS platforms a challenging endeavor.

It's crucial to emphasize that the central management of semantics within the database, under the control of the DBMS and not replicated across all applications, is imperative. 
This ensures the enforcement of data properties that cannot be explicitly specified through integrity constraints. 
As triggers often operate in the background, their behavior should always be well-documented, as it tends to be somewhat concealed from users and developers
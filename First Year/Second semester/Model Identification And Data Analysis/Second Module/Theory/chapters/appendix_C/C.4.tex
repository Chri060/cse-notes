\section{Sampling time choice}

The most critical decision in discretizing models lies in selecting the sampling time $\Delta T$. 
\begin{definition}[\textit{Nyquist frequency}]
    The Nyquist frequency is defined as:
    \[f_N=\dfrac{1}{2}f_S\]
\end{definition}
\begin{definition}[\textit{Nyquist angular speed}]
    The Nyquist angular speed is defined as:
    \[\omega_N=\dfrac{1}{2}\omega_S\]
\end{definition}
In general, we aim to choose the smallest possible sampling time. 
This is because a larger $\omega_N$ corresponds to a broader bandwidth over which the discretized model approximates the original one well.

\paragraph*{Problems}
There are several issues associated with selecting a very small sampling time:
\begin{itemize}
    \item Higher cost of sampling devices.
    \item Increased computational burden: updating algorithms at higher frequencies is more demanding.
    \item Memory cost: more memory is required for data logging or time buffering with smaller sampling times.
    \item Numerical precision cost: with small sampling times, poles are mapped very close to the edge of the unit circle in the z-plane, necessitating high numerical precision for discrete-time algorithm management.
\end{itemize}
For dynamical systems in control applications, a practical guideline for $f_S$ is approximately twenty times the control system's bandwidth:
\[f_S \approx 20 \times \text{bandwidth}_{\text{control system}}\]
In practice, given the desired design bandwidth of the closed-loop control system denoted as $\omega_C$, we can determine the Nyquist frequency and sampling frequency as follows:
\[\omega_N=10 \times \omega_C \qquad \omega_S=20 \times \omega_C\]
\section{Exercise one}

Consider a second-order system in state-space representation:
\[F=\begin{bmatrix} 0 & 2 \\ \frac{1}{2} & 3 \end{bmatrix}\quad G=\begin{bmatrix} \frac{1}{2} \\ \frac{1}{2} \end{bmatrix}\quad H=\begin{bmatrix} 1 & 0 \end{bmatrix}\quad D=\begin{bmatrix} 0 \end{bmatrix}\]
\begin{enumerate}
    \item Write the system of difference equations.
    \item Compute the system transfer function.
    \item Compute the poles.
    \item Compute five samples of the impulse response. 
    \item Check the observability and reachability of the system. 
    \item Compute the Hankel matrix. 
\end{enumerate}

\subsection*{Solution}
\begin{enumerate}
    \item The system of difference equations for the given matrices are:
        \[\begin{cases}
            x_1(t+1)=x_2(t)+\frac{1}{2}u(t) \\
            x_2(t+1)=\frac{1}{2}x_1+3x_2(t)+\frac{1}{2}u(t) \\
            y(t)=x_1(t)
        \end{cases}\]
    \item The transfer function can be derived using the formula:
        \[W(z)=H\left(zI-F\right)^{-1}G+D\]
        In this scenario, it simplifies to:
        \[W(z)=\begin{bmatrix} 1 & 0 \end{bmatrix}\begin{bmatrix} z & -2 \\ -\frac{1}{2} & z-3 \end{bmatrix}^{-1}\begin{bmatrix} \frac{1}{2} \\ \frac{1}{2} \end{bmatrix}=\dfrac{1}{2}\dfrac{z-1}{z^{2}-3z-1}\]

        Alternatively, we can compute the transfer function using the shift operators as follows:
        \[\begin{cases}
            zx_1(t)=x_2(t)+\frac{1}{2}u(t) \\
            zx_2(t)=\frac{1}{2}x_1+3x_2(t)+\frac{1}{2}u(t) \\
        \end{cases}\]
        Reformulating the system yields:
        \[\begin{cases}
            x_1(t)=\frac{x_2(t)}{x}+\frac{1}{2z}u(t) \\
            x_2(t)=\frac{\frac{1}{2}x_1+\frac{1}{2}u(t)}{z-3} \\
        \end{cases}\]
        Upon substitution, we obtain:
        \begin{align*}
            x_2(t)  &=\dfrac{\frac{1}{2}\left(\frac{x_2(t)}{x}+\frac{1}{2z}u(t)\right)+\frac{1}{2}u(t)}{z-3} \\
                    &=\dfrac{\frac{1}{4}+\frac{1}{2}z}{z^2-3z-1}u(t)
        \end{align*}
        Now, the expression for $x_1$ can be derived as:
        \[x_1(t)=\dfrac{\frac{1}{4}+\frac{1}{2}zu(t)}{z^2-3z-1}+\dfrac{1}{2z}u(t)=\dfrac{1}{2}\dfrac{z-1}{z^2-3z-1}u(t)\]
        Thus, the transfer function is given by:
        \[y(t)=\dfrac{1}{2}\dfrac{z-1}{z^2-3z-1}u(t)\]
    \item To find the poles of the function, we equate the denominator to zero:
        \[z^{2}-3z-1=0\rightarrow\lambda_{1,2}=\dfrac{3\pm\sqrt{13}}{2}\]
        As a result, one pole lies outside the unit circle, indicating that the system is unstable.
    \item We can approach solving this problem through two methods:
        \begin{itemize}
            \item Direct calculation of Impulse Response coefficients: compute the impulse response coefficients iteratively.
                \[\begin{cases}
                    \omega(0)=0 \\
                    \omega(1)=HG=\begin{bmatrix} 1 & 0 \end{bmatrix}\begin{bmatrix} \frac{1}{2} \\ \frac{1}{2} \end{bmatrix}=\dfrac{1}{2} \\
                    \omega(2)=HFG=\begin{bmatrix} 1 & 0 \end{bmatrix}\begin{bmatrix} 0 & 2 \\ \frac{1}{2} & 3 \end{bmatrix}\begin{bmatrix} \frac{1}{2} \\ \frac{1}{2} \end{bmatrix}=1 \\
                    \omega(3)=HF^2G=\begin{bmatrix} 1 & 0 \end{bmatrix}\begin{bmatrix} 0 & 2 \\ \frac{1}{2} & 3 \end{bmatrix}^2\begin{bmatrix} \frac{1}{2} \\ \frac{1}{2} \end{bmatrix}=\dfrac{7}{2} \\
                    \omega(4)=HF^3G=\begin{bmatrix} 1 & 0 \end{bmatrix}\begin{bmatrix} 0 & 2 \\ \frac{1}{2} & 3 \end{bmatrix}^3\begin{bmatrix} \frac{1}{2} \\ \frac{1}{2} \end{bmatrix}=\dfrac{23}{2}
                \end{cases}\]
                Note the increasing trend of the coefficients, indicating instability in the system.
            \item The alternative approach involves employing long division. 
                Beginning with the transfer function expressed in negative power notation as $\frac{\frac{1}{2}z^{-1}-\frac{1}{2}z^{-2}}{1-3z^{-1}-z^{-2}}$, we divide the numerator by the denominator through five steps to yield the quotient:
                \[E(z)=\dfrac{1}{2}z^{-1}+z^{-2}+\dfrac{7}{2}z^{-3}+\dfrac{23}{2}z^{-4}\]
                In this representation, the coefficients correspond to the impulse response of the system:
                \[\begin{cases}
                    \omega(0)=0 \\
                    \omega(1)=\frac{1}{2} \\
                    \omega(2)=1 \\
                    \omega(3)=\frac{7}{2} \\
                    \omega(4)=\frac{23}{2}
                \end{cases}\]
        \end{itemize}
    \item The observability matrix, denoted as $O$, is structured as:
        \[O=\begin{bmatrix} H \\ HF \end{bmatrix}=\begin{bmatrix} 1 & 0 \\ 0 & 2 \end{bmatrix}\]
        This matrix possesses full rank, implying that the system is entirely observable.
        
        On the other hand, the reachability matrix, denoted as $R$, takes the form:
        \[R=\begin{bmatrix} G \\ FG \end{bmatrix}=\begin{bmatrix} \frac{1}{2} & 1 \\ \frac{1}{2} & \frac{7}{4} \end{bmatrix}\]
        Similar to the observability matrix, this matrix also exhibits full rank, indicating that the system is fully reachable.
    \item The Hankel matrix, $H_n$, can be computed in two ways: 
        \begin{itemize}
            \item Utilizing the previously computed impulse response coefficients:
                \[H_n=\begin{bmatrix} \frac{1}{2} & 1 \\ 1 & \frac{7}{2} \end{bmatrix}\]
            \item Deriving it from the observability and reachability matrices:
                \[H_2=O_2R_2=\begin{bmatrix} 1 & 0 \\ 0 & 2 \end{bmatrix}\begin{bmatrix} \frac{1}{2} & 1 \\ \frac{1}{2} & \frac{7}{4} \end{bmatrix}=\begin{bmatrix} \frac{1}{2} & 1 \\ 1 & \frac{7}{2} \end{bmatrix}\]
        \end{itemize}
\end{enumerate}
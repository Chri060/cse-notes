\section{ White Noise}

\begin{definition}[\textit{ White Noise}]
    A stationary stochastic process is termed White Noise $WN$ with mean $\mu$ and variance $\lambda^2$ if it satisfies the following conditions:
\end{definition}
\begin{enumerate}
    \item The mean is constant: 
        \[\mathbb{E}\left[e(t)\right]=\mu \qquad \forall t\]
    \item The variance is equal to $\lambda^2$: 
        \[\text{Var}\left[e(t)\right]=\gamma_e(0)=\lambda^2\]
    \item The covariance is null: 
        \[\gamma_e(\tau)=\mathbb{E}\left[ \left(e(t)-\mu\right)\left(e(t-\tau)-\mu\right) \right]=0 \qquad \forall t,\tau \neq 0\] 
\end{enumerate}
Consequently, the realizations of $e(t)$ exhibit erratic and unpredictable behavior.

The probability distribution of each individual random variable $e(\bar{t},s)$ is not explicitly specified.
In the case of a Gaussian distribution, the  White Noise is denoted as $WGN$.

While a realization of  White Noise with a constant value over time is technically feasible, it is highly improbable.
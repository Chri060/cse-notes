\section{Frequency analysis}

The power spectral density (or spectrum) of a stationary stochastic process is defined as the Fourier transform of its correlation function:
\[\Gamma (\omega)=\mathcal{F} \left[ \tilde{\gamma}(\tau) \right]=\sum_{\tau = -\infty}^{\infty}\tilde{\gamma}(\tau)e^{-j\omega\tau}\]
Here, $\omega$ represents the angular frequency measured in $rad/s$. 
The Fourier series sum exists for stationary processes, specifically those with a correlation function $\tilde{\gamma}(\tau)$ that rapidly tends to 0 as $\tau$ approaches infinity.

A sufficient condition for the existence of the Fourier transform is the absolute convergence of $\tilde{\gamma}(\tau)$:
\[\sum_{\tau=-\infty}^{\infty}\left\lvert \tilde{\gamma}(\tau) \right\rvert < \infty\]
The anti-transformation formula establishes a one-to-one correspondence between $\tilde{\gamma}(\tau)$ and $\Gamma (\omega)$: 
\[\tilde{\gamma}(\tau)=\mathcal{F}^{-1}\left[ \Gamma (\omega) \right]=\dfrac{1}{2\pi}\int_{-\pi}^{\pi}\Gamma(\omega)e^{j\omega\tau}d\omega\]

From the definition, we derive that:
\begin{align*}
    \Gamma(\omega)  &=\dots+\tilde{\gamma}(-2)e^{j2\omega}+\tilde{\gamma}(-1)e^{j\omega}+\tilde{\gamma}(0)+\tilde{\gamma}(1)e^{-j\omega}+\tilde{\gamma}(2)e^{-j2\omega}+\dots \\
                    &=\tilde{\gamma}(0)+\tilde{\gamma}(1)(e^{j\omega}+e^{-j\omega})+\tilde{\gamma}(2)(e^{j2\omega}+e^{-j2\omega}) \\
                    &=\tilde{\gamma}(0)+2\tilde{\gamma}(1)\cos\omega+2\tilde{\gamma}(2)\cos2\omega 
\end{align*}
Therefore, the spectrum $\Gamma(\omega)$ is a non-negative, real, even, and periodic function with a period of $2\pi$. 

The maximum angular frequency for sinusoidal discrete-time signals is $\omega_{max} = \pi$.
This is due to the minimum period of a discrete signal, $T = 2 \left(f = \frac{1}{2}\right)$, corresponding to $\omega = \frac{2\pi}{2} = \pi = \omega_{max}$.

For processes with zero expected value, the variance of the process equals the area below the spectral density curve:
\[\gamma(0)=\tilde{\gamma}(0)=\dfrac{1}{2\pi}\int_{-\pi}^{\pi}\Gamma(\omega)d\omega\]
This can be directly derived from the anti-transformation formula with $\tau = 0$.

The area below the spectrum in a given angular frequency range represents the contribution to the overall variability of the process at those angular frequencies.
\begin{example}
    Consider a white noise with zero mean and variance equal to $\lambda^2$. 
    Both the covariance and correlation functions are zero everywhere except for $\tau = 0$, where $\gamma(0) = \lambda^2$. 
    Therefore, $\Gamma(\omega)=\lambda^2$ for all $\omega$: 
    \[\tilde{\gamma}(\tau)=\gamma(\tau)=\begin{cases}
        0 \qquad \tau \neq 0 \\
        \lambda^2 \qquad \tau = 0
    \end{cases}\]
    In the frequency domain, this translates to:
    \[\Gamma(\omega)=\tilde{\gamma}(0)e^{-j\omega 0}=\lambda^2\]
    This indicates that the power spectral density of a white noise is constant across all frequencies. 
    In other words, all frequencies contribute equally to the process variability, affirming the absolute unpredictability of a white noise.
\end{example}
The spectrum can be equivalently defined in two steps:
\begin{enumerate}
    \item Take the (bilateral) Z-transform of the correlation function:
        \[\Phi(z)=\sum_{\tau=-\infty}^{+\infty}\tilde{\gamma}(\tau)z^{-\tau}\]
    \item Evaluate $\Phi(z)$ at $z=e^{j\omega}$: 
        \[\Gamma(\omega)=\Phi(e^{j\omega})\]
\end{enumerate}

\subsection{Stable system spectrum}
For stationary processes resulting from filtering white noise through a stable dynamical system, the spectrum can be calculated using the transfer function.
Let's consider the process $y(\cdot)$ obtained by filtering an input process $u(\cdot)$ through an asymptotically stable dynamical system described by transfer function $W(z)$. 
The spectra of the two processes are related by the equation:
\[\Gamma_{yy}(\omega)=\left\lvert W(e^{j\omega})\right\rvert^2\Gamma_{uu}(\omega)\]

Given that the expected value of $u(\cdot)$ is zero, applying the convolution formula reveals that $y(\cdot)$ also has a null expectation.
To compute the input-output cross-covariance function, we start with the convolution formula:
\[y(t_2)=\sum_{i=0}^\infty w(i)u(t_2-i)\]
Multiplying both sides by $u(t_1)$ yields:
\[u(t_1)y(t_2)=u(t_1)\sum_{i=0}^{\infty}w(i)u(t_2-i)=\sum_{i=0}^\infty w(i)u(t_1)u(t_2-i)\]
Taking the expected value of both sides, we obtain:
\[\mathbb{E}\left[u(t_1)y(t_2)\right]=\mathbb{E}\left[u(t_1)\sum_{i=0}^{\infty}w(i)u(t_2-i)\right]=\sum_{i=0}^\infty w(i) \mathbb{E}\left[u(t_1)u(t_2-i)\right]\]
This leads to:
\[\gamma_{uy}(t_1,t_2)=\sum_{i=0}^\infty w(i)\gamma_{uu}(t_1,t_2-i)\]
Similarly, multiplying both sides of the convolution formula by $y(t_1)$ gives:
\[y(t_1)y(t_2)=y(t_1)\sum_{i=0}^{\infty}w(i)u(t_2-i)=\sum_{i=0}^\infty w(i)y(t_1)u(t_2-i)\]
Taking the expectation leads to:
\[\gamma_{yy}(t_1,t_2)=\sum_{i=0}^\infty w(i)\gamma_{yu}(t_1,t_2-i)\]
For stationary processes, these auto-covariance and cross-covariance functions depend only on the difference between the two temporal indices $\tau = t_2-t_1$:
\[\begin{cases}
    \gamma_{uy}(\tau)=\sum_{i=0}^\infty w(i)\gamma_{uu}(\tau-i) \\
    \gamma_{yy}(\tau)=\sum_{i=0}^\infty w(i)\gamma_{yu}(\tau-i)
\end{cases}\]
Let's introduce the (bilateral) Z-transforms of these covariance functions:
\[\Phi_{uu}(z)=\sum_{-\infty}^{+\infty}\gamma_{uu}(\tau)z^{-\tau} \qquad \Phi_{yy}(z)=\sum_{-\infty}^{+\infty}\gamma_{yy}(\tau)z^{-\tau} \qquad \Phi_{uy}(z)=\sum_{-\infty}^{+\infty}\gamma_{uy}(\tau)z^{-\tau}\]
These transforms are useful because they allow us to compute the power spectral densities of the input and output, as well as the cross-spectrum, by simply applying $z=e^{j\omega}$:
\[\Gamma_{uu}(\omega)=\Phi_{uu}(e^{j\omega}) \qquad \Gamma_{yy}(\omega)=\Phi_{yy}(e^{j\omega}) \qquad \Gamma_{uy}(\omega)=\Phi_{uy}(e^{j\omega})\]
It's worth noting that while $\Gamma_{uu}(\omega)=\Phi_{uu}(e^{j\omega})$ and $\Gamma_{yy}(\omega)=\Phi_{yy}(e^{j\omega})$ are real functions, the input-output cross spectrum can be complex. 
Additionally, since $\gamma_{uy}(\tau)=\gamma_{yu}(-\tau)$, we have:
\[\Phi_{yu}(z)=\Phi(z^{-1})\rightarrow\Gamma_{yu}(\omega)=\Gamma_{uy}(\omega)^\ast\]
Here, $\ast$ denotes the complex conjugate.
By performing Z-transforms on the expressions for $\gamma_{uy}(\cdot)$ and $\gamma_{yy}(\cdot)$, and recalling the properties of Z-transforms, specifically:
\begin{itemize}
    \item The Z-transform of the convolution product is equal to the product of the transforms of the two processes.
    \item The Z-transform of the impulse response $w(\cdot)$ is the transfer function of the system.
\end{itemize}
We arrive at:
\[\gamma_{uy}(\tau)=\sum_{i=0}^{\infty}w(i)\gamma_{uu}(\tau-i)\rightarrow\Phi_{uy}(z)=W(z)\Phi_{uu}(z)\]
\[\gamma_{yy}(\tau)=\sum_{i=0}^{\infty}w(i)\gamma_{yu}(\tau-i)\rightarrow\Phi_{yy}(z)=W(z)\Phi_{yu}(z)\]
Expanding these expressions yields:
\[\Phi_{yy}(z)=W(z)\Phi_{yu}(z)=W(z)\Phi_{uy}(z^{-1})=W(z)W(z^{-1})\Phi_{uu}(z^{-1})\]
Finally, since $\Phi_{uu}(z^{-1})=\Phi_{uu}(z)$, we also have:
\[\Phi_{yy}(z)=W(z)W(z^{-1})\Phi_{uu}(z)\]
To obtain the power spectral density, we simply substitute $z = e^{j\omega}$:
\[\Gamma_{yy}(\omega)=\left\lvert W(e^{j\omega})\right\rvert^2\lambda^2 \]
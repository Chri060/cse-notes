\section{Prediction problem}

To predict the value of $v(t)$ given a set of observations $\left\{v(1), v(2),\dots, v(t-1)\right\}$, we can devise a predictor using the formula:
\[\widehat{v}(t | t-1)=f\left(v(t-1), v(t-2),\dots, v(1)\right)\]
In this formulation, we impose certain constraints:
\begin{itemize}
    \item The function $f$ is linear. 
    \item Older data have diminishing importance compared to recent ones (finite memory predictor).
    \item The function remains invariant over time.
\end{itemize}
Under these simplifications, the predictor takes the form:
\[\widehat{v}(t | t-1)=a_1v(t-1)+a_2v(t-2)+\dots+a_nv(t-n)\]
Here, $v$ is represented as a vector:
\[v=\begin{bmatrix} a_1 & a_2 & \dots & a_n \end{bmatrix}^T\]
A reliable prediction is one that yields estimates closely aligned with the actual values.
The accuracy of these estimates hinges on the parameters $a_i$ in the $v$ vector. 
Determining these parameter values equates to identifying the model that best characterizes the data distribution. 
This task translates into an optimization problem.

\subsection{Model quality}
Uncertainty represents a critical aspect of noise in prediction problems, yet its precise magnitude cannot be predetermined.
The sole method to calculate an estimate of uncertainty is by comparing known values with those provided by the estimator at corresponding instants.
\begin{example}
    Consider the predictor given by:
    \[\widehat{v}(t | t-1)=a_1v(t-1)+a_2v(t-2)+a_3v(t-3)\]
    The values to be examined are:
    \begin{itemize}
        \item $\widehat{v}(4 | 3)=v(1)v(2)v(3)$: to be compared with $v(4)$.
        \item $\widehat{v}(5 | 4)=v(2)v(3)v(4)$: to be compared with $v(5)$.
    \end{itemize}
\end{example}

After comparing all possible sequences, we can generate a sequence of residuals using the formula:
\[\varepsilon(i) = v(i) - \widehat{v}(i | i - 1) \qquad i = n+1,\dots,N\]
From this sequence, we seek to find $v$  by minimizing the following function:
\[\mathcal{J}(v)=\sum_{n+1}^N\varepsilon(i)^2\]
It's worth noting that the error is squared to ensure it is always considered as positive.
A predictor is considered effective if the remaining error exhibits no discernible pattern, indicating that any remaining error is attributable solely to white noise.
\begin{definition}[\textit{White noise}]
    White noise refers to an error characterized by its lack of correlation between values at different points in time.
\end{definition}
Consequently, if the residual is white noise, it signifies that there's no meaningful information within it that can enhance predictions.

Finally, to derive the accurate value from an estimate, we must incorporate the residual into the estimate. 
Thus, the previous formula transforms to:
\[\widehat{v}(t)=a_1v(t-1)+a_2v(t-2)+\dots+a_nv(t-n)+\varepsilon(t)\]
This implies that addressing the prediction problem involves examining a stochastic system.

\subsection{Zeta transform}
The same system can be reformulated using the Z-transform defined as:
\[V(z)=\mathcal{Z} \left[v(t)\right]\]
When considering the same system at a time $t-1$ in the Z-transform formulation, we have:
\[\underbrace{z^{-1}}_{\text{unity delay operator}} \cdot V(z) =\left[v(t-1)\right]\]
By incorporating the system described above with the time-domain equation:
\[\widehat{v}(t)=a_1v(t-1)+a_2v(t-2)+\dots+a_nv(t-n)+\varepsilon(t)\]
It's feasible to rewrite the same model in the frequency domain with the Z-transform, resulting in:
\[V(z)=\mathcal{Z}\left[ a_1v(t-1)+a_2v(t-2)+\dots+a_nv(t-n)+\varepsilon(t) \right]\]
This simplifies to:
\[V(z)=a_1z^{-1}V(z)+a_2z^{-2}V(z)+\dots+a_nz^{-n}V(z)+\xi(z) \]
Note that this formula can also be expressed using operatorial notation:
\[V(z)=a_1z^{-1}v(t-1)+a_2z^{-2}v(t-2)+\dots+a_nz^{-n}v(t-n)+\xi(t) \]
Rearranging terms such that all elements multiplied by $V(z)$ are on the left side and the residual $\xi(z)$ is on the right side, we obtain:
\[V(z)\left(1 -a_1z^{-1}-a_2z^{-2}-\dots-a_nz^{-n}\right)=\xi(z) \]
Finally, from this expression, we can derive the transfer function as the ratio of $V(z)$ to the residual $\xi(z)$: 
\[\dfrac{V(z)}{\xi(z)}=\dfrac{1}{1 -a_1z^{-1}-a_2z^{-2}-\dots-a_nz^{-n}}\]

\subsection{Summary}
The components of an identification problem consist of:
\begin{itemize}
    \item A system $\mathcal{S}$ requiring modeling.
    \item A model $\mathcal{M}$ to be ascertained, describing the system.
    \item An identification algorithm $\mathcal{J}$ governing data processing.
    \item An identification experiment $\mathcal{E}$ providing the data.
\end{itemize}
From these elements, it's crucial to emphasize that the model cannot convey more information than what is inherent in the data.
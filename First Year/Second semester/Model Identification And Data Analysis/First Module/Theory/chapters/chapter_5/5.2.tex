\section{Spectral density estimator}

We aim to estimate the spectral density $\Gamma(\omega)$ of a stationary process based on a set of samples $\{v(1), v(2), ..., v(N)\}$. 
We assume that the process has a zero mean, implying $\tilde{\gamma}(\tau)=\gamma(\tau)$. 
By definition, the spectrum is expressed as an infinite sum:
\[\Gamma(\omega)=\sum_{\tau=-\infty}^{\infty}\gamma(\tau)e^{-j\omega\tau}\]
In practical applications, we approximate it using only a finite number of terms:
\[\widehat{\Gamma}_N(\omega)=\sum_{\tau=1-N}^{N-1}\widehat{\gamma}_N(\tau)e^{-j\omega\tau}\]
It's essential to note two sources of approximation in this definition:
\begin{itemize}
    \item The sample estimator $\widehat{\gamma}_N(\tau)$ is utilized instead of $\gamma(\tau)$. 
    \item The summation is restricted to $\pm(N-1)$ terms.
\end{itemize}
While this estimator is correct in the sense that it's derived from the available data, it's not consistent due to the finite number of terms used in the estimation process.
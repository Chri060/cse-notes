\section{Prediction of ARMAX processes}

Let's consider the generic ARMAX model expressed at time $t+k$: 
\[A(z)y(t+k)=B(z)u(t)+C(z)\xi(t+k)\]
The procedure for computing the predictor mirrors that of the ARMA process.
Initially, we conduct long division of $C(z)$ by $A(z)$ for $k$ steps, denoting the quotient as $E(z)$  and the remainder as $F_k(z)$: 
\[C(z) = A(z)E(z) + z^{-k} F_k(z)\]
Here, $E(z) = e_0 + e_1z^{-1} + \dots + e_{k-1}z^{-(k-1)}$, with $e_0 = 1$ due to the monic nature of both $A(z)$ and $C(z)$. 

Multiplying both sides of the ARMAX equation by $E(z)$, we obtain:
\begin{align*}
    &A(z)E(z)y(t+k) = B(z)E(z)u(t) + C(z)E(z)\xi(t+k) \implies \\
    &[C(z) - z^{-k} F_k(z)] y(t+k) = B(z)E(z)u(t) + C(z)E(z)\xi(t+k) \\
    &C(z)y(t+k) = z^{-k} F_k(z)y(t+k) + B(z)E(z)u(t) + C(z)E(z)\xi(t+k) \\
    &y(t+k)=\dfrac{F_k(z)}{C(z)}y(t) + \dfrac{B(z)E(z)}{C(z)}u(t) + E(z)\xi(t+k)
\end{align*}
Breaking down the structure of the equation:
\begin{itemize}
    \item The first term relies on $y(\cdot)$ up to time $t$. 
    \item The second term depends on $u(\cdot)$ up to time $t$.
    \item The third term, $e_0\xi(t+k) + e_1\xi(t+k-1) + \dots + e_{k-1}\xi(t+1)$, remains unpredictable from the past.
\end{itemize}
Hence, the optimal $k$-steps ahead predictor for $y(t+k)$ is:
\[\hat{y}(t+k|t)=\dfrac{F_k(z)}{C(z)}y(t) + \dfrac{B(z)E(z)}{C(z)}u(t)\]
Or equivalently: 
\[C(z)\hat{y}(t+k|t) = Fk(z)y(t) + B(z)E(z)u(t) \]
Specifically, for $k = 1$, $E(z) = e_0 = 1$ and $F_1(z) = z(C(z) - A(z))$, resulting in the predictor equation:
\[C(z)\hat{y}(t+1|t) = (C(z) - A(z))y(t+1) + B(z)u(t)\]
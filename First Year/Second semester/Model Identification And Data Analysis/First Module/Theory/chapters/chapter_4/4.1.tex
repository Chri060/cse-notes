\section{Power spectral density}

The power spectral density (or spectrum) of a stationary stochastic process is defined as the Fourier transform of its correlation function:
\[\Gamma (\omega)=\mathcal{F} \left[ \tilde{\gamma}(\tau) \right]=\sum_{\tau = -\infty}^{\infty}\tilde{\gamma}(\tau)e^{-j\omega\tau}\]
Here, $\omega$ represents the angular frequency measured in $rad/s$. 
The Fourier series sum exists for stationary processes, specifically those with a correlation function $\tilde{\gamma}(\tau)$ that rapidly tends to 0 as $\tau$ approaches infinity.

A sufficient condition for the existence of the Fourier transform is the absolute convergence of $\tilde{\gamma}(\tau)$:
\[\sum_{\tau=-\infty}^{\infty}\left\lvert \tilde{\gamma}(\tau) \right\rvert < \infty\]
The anti-transformation formula establishes a one-to-one correspondence between $\tilde{\gamma}(\tau)$ and $\Gamma (\omega)$: 
\[\tilde{\gamma}(\tau)=\mathcal{F}^{-1}\left[ \Gamma (\omega) \right]=\dfrac{1}{2\pi}\int_{-\pi}^{\pi}\Gamma(\omega)e^{j\omega\tau}d\omega\]

From the definition, we derive that:
\begin{align*}
    \Gamma(\omega)  &=\dots+\tilde{\gamma}(-2)e^{j2\omega}+\tilde{\gamma}(-1)e^{j\omega}+\tilde{\gamma}(0)+\tilde{\gamma}(1)e^{-j\omega}+\tilde{\gamma}(2)e^{-j2\omega}+\dots \\
                    &=\tilde{\gamma}(0)+\tilde{\gamma}(1)(e^{j\omega}+e^{-j\omega})+\tilde{\gamma}(2)(e^{j2\omega}+e^{-j2\omega}) \\
                    &=\tilde{\gamma}(0)+2\tilde{\gamma}(1)\cos\omega+2\tilde{\gamma}(2)\cos2\omega 
\end{align*}
Therefore, the spectrum $\Gamma(\omega)$ is a non-negative, real, even, and periodic function with a period of $2\pi$. 

The maximum angular frequency for sinusoidal discrete-time signals is $\omega_{max} = \pi$.
This is due to the minimum period of a discrete signal, $T = 2 \left(f = \frac{1}{2}\right)$, corresponding to $\omega = \frac{2\pi}{2} = \pi = \omega_{max}$.

For processes with zero expected value, the variance of the process equals the area below the spectral density curve:
\[\gamma(0)=\tilde{\gamma}(0)=\dfrac{1}{2\pi}\int_{-\pi}^{\pi}\Gamma(\omega)d\omega\]
This can be directly derived from the anti-transformation formula with $\tau = 0$.

The area below the spectrum in a given angular frequency range represents the contribution to the overall variability of the process at those angular frequencies.
\begin{example}
    Consider a white noise with zero mean and variance equal to $\lambda^2$. 
    Both the covariance and correlation functions are zero everywhere except for $\tau = 0$, where $\gamma(0) = \lambda^2$. 
    Therefore, $\Gamma(\omega)=\lambda^2$ for all $\omega$: 
    \[\tilde{\gamma}(\tau)=\gamma(\tau)=\begin{cases}
        0 \qquad \tau \neq 0 \\
        \lambda^2 \qquad \tau = 0
    \end{cases}\]
    In the frequency domain, this translates to:
    \[\Gamma(\omega)=\tilde{\gamma}(0)e^{-j\omega 0}=\lambda^2\]
    This indicates that the power spectral density of a white noise is constant across all frequencies. 
    In other words, all frequencies contribute equally to the process variability, affirming the absolute unpredictability of a white noise.
\end{example}
The spectrum can be equivalently defined in two steps:
\begin{enumerate}
    \item Take the (bilateral) Z-transform of the correlation function:
        \[\Phi(z)=\sum_{\tau=-\infty}^{+\infty}\tilde{\gamma}(\tau)z^{-\tau}\]
    \item Evaluate $\Phi(z)$ at $z=e^{j\omega}$: 
        \[\Gamma(\omega)=\Phi(e^{j\omega})\]
\end{enumerate}
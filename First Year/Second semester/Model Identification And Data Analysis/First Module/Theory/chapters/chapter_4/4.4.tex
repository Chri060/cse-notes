\section{Predictor implementation}

We have determined that the general form of the optimal predictor is:
\[\hat{y}(t+k|t)=\dfrac{F(z)}{C(z)}y(t)\]

Now, let's focus on a specific form of the optimal predictor:
\[\hat{y}(t+k|t)=\dfrac{1}{1+cz^{-1}}y(t)\]
For implementation in a program, it needs to be converted into a time-domain representation:
\[\hat{y}(t+k|t)=y(t)-c\hat{y}(t+k-1|t-1) \]
This form represents an AutoRegressive model.

However, there's a problem with the second term on the right side of the equation.
When trying to compute $\hat{y}(-1+k|-1)$ , we encounter an initialization problem because we only have samples for $t\geq 0$.

\paragraph*{Heuristic}
To solve this problem, a heuristic solution is employed. 
It sets $\hat{y}=m_y$ when data is not available. 
This effectively utilizes the trivial predictor.

This heuristic is acceptable due to the asymptotic stability of $\frac{F(z)}{C(z)}$, causing the effect of this initialization to rapidly diminish. 
Consequently, this approximation becomes negligible as $t$ grows sufficiently large.
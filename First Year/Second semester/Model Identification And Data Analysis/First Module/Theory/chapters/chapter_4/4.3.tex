\section{Purely deterministic processes}

Wold's decomposition conceptualizes a stochastic process as the sum of a purely nondeterministic component and a purely deterministic one:
\[v(t)=\tilde{v}(t)+\hat{v}(t)\]
he purely deterministic part of the process $v(t)$ can be represented as:
\[v(t) = \alpha_1v(t-1) + \alpha_2v(t-2) + \dots + \alpha_nv(t-n)\]

\subsection{Constant process}
Let's examine the constant process $v(t)=v(t-1)=v$ with a correlation function $\tilde{\gamma}(\tau)=v^2$. 
Consequently, the power spectral density is:
\[\Gamma(\omega)=v^2\delta(\omega)\]

\subsection{Alternated process}
Now, consider the alternating process $v(t)=-v(t-1)=v$ with a correlation function $\tilde{\gamma}(\tau)=(-1)^\tau v^2$. 
This leads to the power spectral density:
\[\Gamma(\omega)=v^2\delta(\omega-\pi)\]

\subsection{Sinusoidal process}
Lastly, let's consider the sinusoidal process $v(t)=A\cos(\omega_0 t)$ with a correlation function $\tilde{\gamma}(\tau)=\dfrac{A^2}{2}\mathbb{E}\left[\cos(\omega_0\tau)\right]$. 
This yields the power spectral density:
\[\Gamma(\omega)=\dfrac{A^2}{4}\left[\delta(\omega-\omega_0)+\delta(\omega+\omega_0)\right]\]
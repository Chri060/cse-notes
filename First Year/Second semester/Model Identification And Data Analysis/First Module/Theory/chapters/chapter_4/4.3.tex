\section{Spectral factorization}

Here are four distinct representations of a stochastic process resulting from filtering white noise through a stable digital filter:
\begin{itemize}
    \item Time-domain representation: 
        \[V(T)=a_1v(t-1)+\dots+a_{n_a}(t-n_a)+c_o\eta(t-1)+\dots+c_{n_c}\eta(t-n_c)\]
    \item Operatorial representation: 
        \[v(t)=\dfrac{C(z)}{A(z)}\eta(t)\]
    \item Probabilistic representation:
        \[\mu,\gamma(\tau)\]
    \item Frequency domain representation:
        \[\mu,\Gamma(\omega)\]
\end{itemize}
These representations are interchangeable, allowing for translation between them. 
While calculating the spectrum of a process from the transfer function has been addressed, in practice, it's often more useful to do the reverse. 
Given data with specific spectral characteristics, the challenge lies in describing the generating process, or finding the pair $(W(z), \eta(\cdot))$.

However, not every spectrum corresponds to a transfer function due to the constraints on spectrum shapes. 
Furthermore, even if a transfer function exists, it may not be unique. 
Multiple combinations of transfer functions and white noise can represent the same process.

\subsection{Representation choices}
Consider a generic process:
\[v(t)=W(z)\eta(t) \]
Here, $\eta(\cdot)\sim WN(o,\lambda^2)$
The complex spectrum is given by:
\[\Phi(z)=W(z)W(z^{-1})\lambda^2\]
The following are examples of equivalent representations for $v(t)=\tilde{W}(z),\tilde{\eta}(t)$ where $\tilde{\eta}\sim WN(0,\tilde{\lambda^2})$: 
\begin{enumerate}
    \item Multiplication by a constant $\alpha$:  
        \[\tilde{W}(z)=\dfrac{1}{\alpha}W(z) \qquad \tilde{\eta}\sim WN(0,\alpha^2\lambda^2)\]
        To demonstrate their equivalence, we show they have the same spectrum:
        \[\tilde{\Phi}(z)=\tilde{W}(z)\tilde{W}(z^{-1})\tilde{\lambda}^{2}=\dfrac{1}{\alpha}W(z)\cdot\dfrac{1}{\alpha}W(z^{-1})\cdot\alpha^2\lambda^2=W(z)W(z^{-1})\lambda^2=\Phi(z)\]
    \item Multiplication by $z^n$: 
        \[\tilde{W}(z)=z^nW(z) \qquad \tilde{\eta}=z^{-n}\eta(t)=\eta(t-n)\]
        To demonstrate their equivalence, we show they have the same spectrum:
        \[\tilde{\Phi}(z)=\tilde{W}(z)\tilde{W}(z^{-1})\lambda^{2}=z^nW(z)z^{-n}W(z^{-1})\lambda^2=W(z)W(z^{-1})\lambda^2=\Phi(z)\]
    \item Multiplication of both the numerator and the denominator by the same factor:
        \[\tilde{W}(z)=W(z)\dfrac{z-p}{z-p} \qquad \tilde{\eta}=\eta(t)\]
        To demonstrate their equivalence, we show they have the same spectrum, obtained after simplification:
        \[\tilde{\Phi}(z)=\dfrac{z-p}{z-p}\Phi(z)=\Phi(z)\]
    \item Multiplication by an all-pass filter:
        \[\tilde{W}(z)=W(z)T(z) \qquad \tilde{\eta}=\eta(t)\]
        Here, $T(z)=\dfrac{1}{q}\left(\dfrac{z-q}{z-\frac{1}{q}}\right)$
        To demonstrate their equivalence, we show they have the same spectrum: 
        \[\tilde{\Phi}(z)=\tilde{W}(z)\tilde{W}(z^{-1})\lambda^{2}=W(z)T(z)T(z^{-1})W(z^{-1})\lambda^2=W(z)W(z^{-1})\lambda^2=\Phi(z)\]
\end{enumerate}
We require a representation that is more suitable than others for solving the prediction problem. 
The following theorem facilitates the selection of the so-called canonical representation, which eliminates all potential sources of redundancy as outlined.
\begin{theorem}
    Let $v(t)$ be a stationary stochastic process with a rational spectrum.
    There exists a unique pair $\left\{ \hat{W}(z)=\dfrac{C(z)}{A(z)},\xi(t) \right\}$ such that:
    \begin{enumerate}
        \item $C(z)$ and $A(z)$ are monic (with the first coefficient being one).
        \item $C(z)$ and $A(z)$ have the same degree (relative degree equal to zero).
        \item $C(z)$ and $A(z)$ are co-prime (no common roots).
        \item $C(z)$ and $A(z)$ have all their roots in the closed and open unit circle, respectively ($\left\lvert z \right\rvert \leq 1, \forall z \text{ s.t. } C(z)=0$, $\left\lvert z \right\rvert < 1, \forall z \text{ s.t. } A(z)=0$).
    \end{enumerate}
\end{theorem}  
The function $\hat{W}(z)$ is termed the canonical spectral factor.
\begin{example}
    Let's examine the ARMA(1,1) process given by:
    \[v(t)=av(t-1)+\eta(t)+c\eta(t-1) \qquad \eta(\cdot)\sim WN(0,\lambda^2)\]
    For this process to be stationary, the condition $\left\lvert a \right\rvert < 1$ must be satisfied. 
    The spectrum is given by:
    \[\Gamma(\omega)=\dfrac{1+c^2+2c\cos(\omega)}{1+a^2-2a\cos(\omega)}\lambda^2\]
    If $\left\lvert c\right\rvert \leq 1$, the representation is canonical:
    \[\hat{W}(z)=W(z)=\frac{1+cz^{-1}}{1-az^{-1}} \qquad \xi(\cdot)=\eta(\cdot)\]
    On the other hand, if $\left\lvert c\right\rvert > 1$, the canonical representation changes to:
    \[\hat{W}(z)=W(z)=\frac{1+\frac{1}{c}z^{-1}}{1-az^{-1}} \qquad \xi(\cdot)\sim WN(0, c^2\lambda^2)\]
\end{example}
\begin{example}
    Let's examine the given process:
    \[v(t)=\eta_1(t-1)+\eta_2(t)-\eta_2(t-1)\]
    where $\eta_1\sim WN(0,lambda_1^2)$ and $\eta_2\sim WN(0,lambda_2^2)$ are independent white noise processes.
    This process is constructed as the sum of two independent and stationary processes, thus it is also stationary.
    The auto-covariance function is null for $\left\lvert \tau\right\rvert  > 1$. 
    The process can be reformulated as an MA(1):
    \[v(t)=\xi (t)+c\xi (t-1)\]
    where $\xi (\cdot)\sim WN(0,\lambda^2)$, provided it has the same spectral characteristics, or equivalently, the same auto-covariance function (same covariance implies same spectral representation).
    Now, considering the auto-covariance function:
    \[\gamma(0)=(1+c^2)\lambda^2=\lambda_1^2+2\lambda_2^2\]
    \[\gamma(1)=c\lambda^2=-\lambda_2^2\]
    Solving this system of equations yields only one solution with $\left\lvert c \right\rvert  < 1$ to this system of equations. 
    For instance, if $\lambda_1^2=\frac{1}{4}$ and $\lambda_2^2=\frac{1}{2}$, we have:
    \[\begin{cases}
        (1+c^2)\lambda^2=\frac{5}{4} \\
        c\lambda^2=-\frac{1}{2}
    \end{cases} \implies \begin{cases}
        c=-\frac{1}{2} \quad \lambda^2=1 \\
        c=-2 \quad \lambda^2=\frac{1}{4}
    \end{cases}\]
    Note that only the first solution is canonical.
\end{example}
\section{Mixed processes}

Let's now consider a mixed process represented as:
\[v(t)=\tilde{v}(t)+\hat{v}(t)\]
Here, $\tilde{v}(t)$ is a deterministic signal with spectrum $\gamma_{\tilde{v}}(\omega)$, while $\hat{v}(t)$ is a stationary stochastic process with zero expected value and spectrum $\gamma_{\hat{v}}(\omega)$.
As we've observed earlier, since $\tilde{v}(t)$ and $\hat{v}(t)$ are independent, it follows that:
\[\tilde{\gamma}_v(\tau)=\tilde{\gamma}_{\tilde{v}}(\tau)+\tilde{\gamma}_{\hat{v}}(\tau)\]
This relationship also implies:
\[\Gamma_v(\omega)=\Gamma_{\tilde{v}}(\omega)+\Gamma_{\hat{v}}(\omega)\]
\begin{example}
    Now, let's consider an example with a white noise process $\eta(\cdot)$ having a non-zero expected value $\bar{\eta}$ and variance $\lambda^2$. 
    Since the process $\eta(t)-\bar{\eta}$ is a standard white noise with zero expected value, its auto-covariance function is:
    \[\gamma(\tau)\begin{cases}
        0 \qquad \tau \neq 0 \\
        \lambda^2 \tau = 0
    \end{cases}\]
    From this, we can compute the auto-correlation function, recalling that:
    \[\tilde{\gamma}(\tau)=\gamma(\tau)+\bar{\eta}^2\]
    Hence, the power spectral density is:
    \[\Gamma(\omega)=\lambda^2+\bar{\eta}^2\delta(\omega)\]
\end{example}
This result aligns with the interpretation of $\eta(t)$ as a mixed process with:
\begin{itemize}
    \item A purely deterministic component equivalent to a constant process (which exhibits an impulsive spectrum).
    \item A purely nondeterministic component equivalent to a white noise with zero expected value (which has a constant spectrum).
\end{itemize}
Since these two components are independent, the spectrum of $\eta(t)$ is the sum of their spectra:
\[\Gamma_v(\omega)=\Gamma_{\tilde{v}}(\omega)+\Gamma_{\hat{v}}(\omega)\]
\section{Estimation problem}

An estimation problem involves quantifying an unknown parameter through estimation. 
This parameter, denoted as $\vartheta$, can be discrete or continuous, scalar or vectorial, and constant or time-variant.
We are provided with a set of observations, $d$, taken at various time points $t_1, t_2, \dots, t_N$, formally defined as: 
\[d=\left\{ d(t), t \in T\right\}\]
Our goal is to derive an estimator to obtain an estimate of the unknown variable $\vartheta$, expressed as:
\[\vartheta=f(d)\]
\begin{definition}[\textit{Estimate}]
    An estimate, denoted as $\hat{\vartheta}$, is a value produced by an estimator and depends on the input values of the estimator.
\end{definition}
For a parameter $\vartheta$ with a constant value, we seek the estimate $\hat{\vartheta}$. 
For a parameter $\vartheta(t)$ with a dynamic value, we aim to find the estimate $\hat{\vartheta}(t | t_N)$ where the value of $t_N$ is provided.
The choice of $t$ determines the nature of the estimation:
\begin{itemize}
    \item \textit{Prediction}: when $t > t_N$, indicating a time instant beyond $t_N$, we are forecasting a future event.
    \item \textit{Filtering}: when $t = t_N$, we are estimating the noise in the estimator. 
    \item \textit{Regularization} or \textit{interpolation} or \textit{smoothing}: when $t < t_N$, representing a time instant before $t_N$, we are estimating variables that are not directly accessible.
\end{itemize}
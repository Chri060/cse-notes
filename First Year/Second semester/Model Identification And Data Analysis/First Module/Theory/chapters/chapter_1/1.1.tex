\section{Model}

Definition of model 

Image 

A model can be built in different ways: 
\begin{enumerate}
    \item \textit{White-box} or \textit{direct} modeling: the model is created by means of physics laws. 
    \item \textit{Black-box} modeling: the model is created by estimating it with the information extracted from the data. 
\end{enumerate}


\begin{example}
    A mass moving with a spring like in the image 


    We can use direct modeling, obtaining the following equations: 
    \[mx^{''}(t)=f(t)-cx^{'}(t)-kx(t)\]
    Slide 7 and 8

    The same system can be modelled using black-box modeling, obtaining the following formula: 
    \[x(z)=\dfrac{b_0+b_1^{-1}+b_2z^{-2}}{1+a_1^{-1}+a_2z^{-2}}f(z)\]
    By interpreting the data we can obtain all the parameters in the previuous expression.
\end{example}
The first type of modeling is more efficient and simple to underestand because the formula that that describes the system is known. 
Tyhe main problems of this modeling are: 
\begin{itemize}
    \item We have to estimate the parameters in the equation. 
    \item The resulting equation could become non-linear. 
\end{itemize}

\subsection{Modeling error}

If the error has a pattern, it means that we have not extracted all information from the data. 
If the error has no patter, it is called white noise, and it cannot be extracted from the data. 
As a result a model is complete only if the error has a completely impredictable pattern. 

\subsection{System classification}
A system can be classified as: 
\begin{itemize}
    \item \textit{Static system}: a system for which the knowledge of the input variables is sufficient to determine the value of the output. 
    \item \textit{Dynamic system}: a system with memory, where the past behaviour of the output influences the value of the output itself. 
\end{itemize}

\begin{example}
    An example of static system is the Ohm's law that define the behavior of a resistor under a current $i(t)$. 

    The system shown in the image is controlled by the law: 
    \[i(t)=\dfrac{v(t)}{R}\]

    An example of dynamic system is 
\end{example}

A system can also be classified based on time description, that can be discrete or continuous. 
Natural and physical phenomena are intrinsically continuous and can be described mathematically via ordinary differential equations. 
The discrete systems can be described mathematically via difference equations. 





However, a computer can only handle a limited amount of data. 
Therefore, the signals must be sampled with a sampling time $T_s$, such as to store a finite amount of data at discrete times $t \cdot T_s$, where $t=1,\dots,N$: 
\[y(t)=y(t \cdot T_s)\]



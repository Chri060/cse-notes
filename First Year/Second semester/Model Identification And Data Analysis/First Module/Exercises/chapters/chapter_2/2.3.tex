\section{Exercise three}

Consider the MA ($2$) process generated by the expression:
\[y(t)=e(t)+0.5e(t-1)+0.5e(t-2) \qquad e(t) \sim WN(2,1)\]
\begin{enumerate}
    \item Determine the transfer function and verify if it is a stationary stochastic process.
    \item Calculate the expected value.
    \item Compute the covariance function.
\end{enumerate}

\subsection*{Solution}
\begin{enumerate}
    \item We express the formula in operatorial representation:
        \[y(t)=e(t)+0.5e(t)z^{-1}+0.5e(t)z^{-2}\rightarrow y(t)=\dfrac{z^2+0.5z+0.5}{z^2}e(t)\]
        The system has two zeros at $z_{1,2}=-\frac{1}{4}\pm \frac{\sqrt{7}}{4}i$  and a pole at $z=0$, indicating asymptotic stability.
        As the input, White Noise, is a stationary stochastic process, $y(t)$ is also a stationary stochastic process.
    \item The expected value is computed as:
        \[\mathbb{E}\left[ y(t) \right]=\mathbb{E}\left[ e(t)+0.5e(t-1)+0.5e(t-2) \right]=2+1+1=4\]

        Alternatively, it can be computed using the theorem: 
        \[\mathbb{E}\left[ y(t) \right]=W(1)\cdot\mathbb{E}\left[ e(t) \right]=2\cdot 2=4\]
    \item Define the unbiased process as: 
        \[\begin{cases}
            \tilde{y}(t)=y(t)-m_y \\
            \tilde{e}(t)=e(t)-m_e
        \end{cases}\]
        In this case, we have: 
        \[\tilde{y}(t)+m_y=\left(\tilde{e}(t)+m_e\right)+0.5\left(\tilde{e}(t-1)+m_e\right)+0.5\left(\tilde{e}(t-2)+m_e\right)\]
        Simplifying, we obtain: 
        \[\tilde{y}(t)=\tilde{e}(t)+0.5\tilde{e}(t-1)+0.5\tilde{e}(t-2)\]

        Since it is a Moving Average process, we can directly find the covariance as: 
        \[\begin{cases}
            \left(c_0^2+c_1^2+c_2^2\right)\lambda^2 \qquad \tau=0 \\
            \left(c_0c_1+c_1c_2\right)\lambda^2 \qquad\: \left\lvert \tau\right\rvert =1 \\
            \left(c_0c_2\right)\lambda^2 \qquad\qquad\:\:\:\:\: \left\lvert \tau\right\rvert =2 \\
            0 \qquad\qquad\qquad\qquad\: \left\lvert \tau\right\rvert \geq 3 \\
        \end{cases} \rightarrow \begin{cases}
            \frac{3}{2} \qquad \tau=0 \\
            \frac{3}{4} \qquad \left\lvert \tau\right\rvert =1 \\
            \frac{1}{2} \qquad \left\lvert \tau\right\rvert =2 \\
            0 \:\qquad \left\lvert \tau\right\rvert \geq 3 \\
        \end{cases}\]
\end{enumerate}
\section{Exercise two}

Consider the stochastic process defined by the following diagram:
\begin{figure}[H]
    \centering
    \includegraphics[width=0.5\linewidth]{images/block1.png}
\end{figure}
Here, $e(t) \sim WN(1,1)$ and $\xi(t) \sim WN(0,1)$ are uncorrelated.
\begin{enumerate}
    \item Find when the process is stationary. 
    \item Given $\gamma_y(0)=6$, $\gamma_y(1)=-2$, and $\gamma_y(\tau)=0$ for $\tau \geq 2$, we want to compute the values of $a$ and $c$. 
\end{enumerate}

\subsection{Solution}
\begin{enumerate}
    \item The process $y(t)$ is stationary when $\xi(t)$ and $y_1(t)$ are both stationary. 
        The process $\xi(t)$ is a White Noise, so it is stationary by definition. 
        The process $y_1(t)$ is stationary when $\left\lvert a \right\rvert<1$.
    \item Since we have that $\gamma_y(\tau)=0$ for $\tau \geq 2$, this is a Moving Average Process of order $1$. 
        Thus, $a=0$. 

        The process in the time domain is: 
        \[y(t)=-+e(t)+ce(t-1)+\xi(t)\]
        We can compute the covariance in $\tau=0$
        \[\gamma_y(0)=\mathbb{E}\left[ y(t)^2 \right]=0\]
        From which we obtain $c=-2$. 
\end{enumerate}

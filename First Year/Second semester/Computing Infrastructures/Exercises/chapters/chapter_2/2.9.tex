\section{Exercise nine}

In a RAID 6 system, eight $3\text{ TB}$ drives are used to store data along with the required parity bits. 
Compute how many identical drives would be required to construct a RAID 10 system with the same capacity.
For the RAID 1 configuration, consider a single replica of data blocks.

\subsection*{Solution}
The storage capacity for RAID 6 is:
\[S_C^{\text{RAID 6}}=\left(N-2\right)\cdot3\text{ TB}=6\cdot 3\text{ TB}=18\text{ TB}\]
The storage capacity for RAID 10 is:
\[S_C^{\text{RAID 10}}=\dfrac{N}{2}\cdot3\text{ TB}=18\text{ TB}\]
Hence, we need twelve drives. 
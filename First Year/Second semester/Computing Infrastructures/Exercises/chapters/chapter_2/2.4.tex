\section{Exercise four}

A system comprises four non-redundant components, with a 1-year reliability of 0.92.
A new version of the system incorporates a novel feature and utilizes six non-redundant components.
Compute the 1-year reliability of the new system, assuming all components have the same mean time to failure.

\subsection*{Solution}
The reliability of the initial system is:
\[R_{old}(1 \text{ year})=\left(e^{-\lambda}\right)^4=e^{-4\lambda}\]
Given this expression equals $0.92$, we can determine the value of the failure rate:
\[e^{-4\lambda}=0.92\rightarrow \lambda=\dfrac{\ln(0.92)}{-4}=0.02\]
From the failure rate, we can derive the mean time to failure:
\[\text{MTTF}=\dfrac{1}{\lambda}=\dfrac{1}{0.02}=47.97\text{ years}\]

In the new scenario, the reliability becomes:
\[R_{old}(1 \text{ year})=\left(e^{-\lambda}\right)^6=e^{-6\lambda}=0.88\]
\section{Exercise four}

A complex system has a failure rate of $\lambda = 0.25 \cdot 10^{-4}$ per hour and a mean time to repair of 72 hours in normal use.
\begin{enumerate}
    \item Compute the steady-state availability.
    \item If mean time to repair is increased to 120 hours, compute the failure rate that can be tolerated without decreasing the availability of the system.
\end{enumerate}

\subsection*{Solution}
\begin{enumerate}
    \item To calculate the steady-state availability, we first find the mean time to failure:
        \[\text{MTTF}=\dfrac{1}{\lambda}=\dfrac{1}{0.25 \cdot 10^{-4}}=40000 \text{ hours}\]
        Then, the availability is computed as:
        \[A=\dfrac{\text{MTTF}}{\text{MTTF}+\text{MTTR}}=\dfrac{40000}{40000+72}=0.9982\]
    \item If we increase the mean time to repair while maintaining the same availability of $0.9982$, we first find the new minimum mean time to failure:
        \[A=\dfrac{\text{MTTF}_{new}}{\text{MTTF}_{new}+\text{MTTR}}\rightarrow\text{MTTF}_{new}=-\dfrac{A\cdot\text{MTTR}}{A-1}\rightarrow\text{MTTF}_{new}=66666.66\text{ hours}\]
        From this, we derive the new failure rate:
        \[\lambda_{new}=\dfrac{1}{\text{MTTF}_{new}}=\dfrac{1}{66666.66}=1.5\cdot 10^{-5}\]
\end{enumerate}
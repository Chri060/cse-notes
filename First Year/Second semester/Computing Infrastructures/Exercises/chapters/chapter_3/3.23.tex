\section{Exercise twenty-three}

Let's consider an intranet accessible by a large number of users. 
A single request passes through an application server with a service time $S=300$ ms, then through a database server with a service time $S=250$ ms, and then back through the application server. 
Additionally, a request must pass through the system firewall before entering and exiting the intranet, with a firewall service time per visit of $S=10$ ms.
\begin{enumerate}
    \item Compute the maximum throughput of the system.
    \item It's possible for the response time to be less than 9 seconds. 
        Find the conditions to let this happen. 
\end{enumerate}

\subsection*{Solution}
\begin{enumerate}
    \item We compute the response time $D$ for each server as:
        \[\begin{cases}
            D_{AS}=S_{AS}\cdot V_{AS}= 300 \cdot 2 = 0.6\text{ s} \\
            D_{DS}=S_{DS}\cdot V_{DS}= 250 \cdot 1 = 0.25\text{ s} \\
            D_{FW}=S_{FW}\cdot V_{FW}= 10 \cdot 2 = 0.02\text{ s}
        \end{cases}\]
        We find the maximum response time among the three servers:
        \[D_{\max}=\max\{0.6,0.25,0.02\}=0.6\text{ s}\]
        The maximum throughput is the inverse of $D_{\max}$:
        \[X_{\max}=\dfrac{1}{D_{\max}}=\dfrac{1}{0.6}=1.6\:\dfrac{\text{job}}{\text{sec}}\]
    \item For a response time $R < 9$ s, we have:
        \[\max{D,N\cdot D_{\max}-Z} \leq R(N) \leq N \cdot D\]
        So, the maximum number of users $N$ is:
        \[N \cdot D \leq 9\rightarrow N \leq 10\]
        Thus, it's possible to have a response time less than 9 s if the number of users does not exceed 10.
\end{enumerate}
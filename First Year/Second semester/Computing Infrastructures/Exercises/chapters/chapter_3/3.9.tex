\section{Exercise nine}

Consider the following measurement data for an interactive system with a memory constraint:
\begin{itemize}
    \item $T$: 1 hour.
    \item $N$: 80.
    \item $R$: 1 second.
    \item $N$ in memory: 6.
    \item $C$: 36000.
    \item $U_{\text{CPU}}$: 75\%.
    \item $U_{D1}$: 50\%.
    \item $U_{D2}$: 50\%.
    \item $U_{D3}$: 25\%.
\end{itemize}
\begin{enumerate}
    \item Compute the throughput (in requests per second).
    \item Compute the average think time.
    \item Compute, on the average, how many users were attempting to obtain service. 
    \item Compute, on the average, how much time does a user spend waiting for memory. 
\end{enumerate}

\subsection*{Solution}
\begin{enumerate}
    \item The throughput of the system is:
        \[X=\dfrac{C}{T}=\dfrac{36000}{3600}=10\:\dfrac{\text{req}}{\text{sec}}\]
        Notice that:
        \[X^\prime=\dfrac{N}{R}=\dfrac{80}{1}=80\:\dfrac{\text{req}}{\text{sec}}\neq X\]
        That's because in an interactive system we have thinking users and thus we cannot use Little's law.
    \item The average think time is:
        \[Z=\dfrac{N}{X}-R=\dfrac{80}{10}-1=7\text{ s}\]
    \item The number of users not thinking can be computed using Little's law:
        \[N^{\prime\prime}=XR=10\cdot 1=10\:\text{req}\]
    \item The time spent waiting for memory is computed with Little's law:
        \[R_{\text{wait}}=\dfrac{N_{\text{wait}}}{X}=\dfrac{N^{\prime\prime}-N}{X}=\dfrac{10-6}{10}=0.4\text{ s}\]
\end{enumerate}
\section{Exercise seven}

Consider an interactive system that accommodates 100 users, each with a 15-second think time, and operates at a throughput of 5 interactions per second.
\begin{itemize}
    \item Compute the response time of the system.
    \item Now, suppose the service demands of the workload shift, leading to a 50\% reduction in system throughput, dropping to 2.5 interactions per second. 
        Assume that the user count remains unchanged at 100, with the same 15-second think times, compute their response time.
\end{itemize}

\subsection*{Solution}
Given:
\[N=100 \qquad Z=15 \qquad X=5\]
\begin{enumerate}
    \item The response time of the system can be determined using the response time law:
        \[R=\dfrac{N}{X}-Z=\dfrac{100}{5}-15=5\text{ s}\]
    \item If we adjust the throughput to $X^\prime=2.5$, the response time can be recalculated:
        \[R=\dfrac{N}{X}-Z=\dfrac{100}{2.5}-15=25\text{ s}\]
\end{enumerate}
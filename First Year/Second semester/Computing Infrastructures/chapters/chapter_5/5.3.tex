\section{Operational laws}

Operational laws provide a model for understanding the average behavior of various systems. 
These laws are highly general and impose minimal assumptions about the behavior of the random variables that characterize the system.
They are based on observable variables, which are values derived from observing a system over a finite period.
The underlying assumptions include:
\begin{enumerate}
    \item The system receives requests from its environment.
    \item Each request generates a job or customer within the system.
    \item Upon processing the job, the system responds to the environment by completing the corresponding request.
\end{enumerate}

\renewcommand*{\arraystretch}{2}
\begin{table}[H]
    \centering
    \begin{tabular}{|c|l|}
    \hline
    $T$ & Length of time we observe the system                 \\ \hline
    $A$ & Number of request arrivals we observe                \\ \hline
    $C$ & Number of request completions we observe             \\ \hline
    $B$ & Total amount of time during which the system is busy \\ \hline
    $N$ & Average number of jobs in the system                 \\ \hline
    \end{tabular}
\end{table}
From these observed values, we can derive four important quantities:
\begin{table}[H]
    \centering
    \begin{tabular}{|c|l|}
    \hline
    $\lambda=\frac{A}{T}$ & Arrival rate                        \\ \hline
    $X=\frac{C}{T}$       & Throughput                          \\ \hline
    $U=\frac{B}{T}$       & Utilization                         \\ \hline
    $S=\frac{B}{C}$       & Mean service time per completed job \\ \hline
    \end{tabular}
\end{table}
We will assume that the number of arrivals equals the number of completions during an observation period:
\[A=C\] 
If the system is job flow balanced, the arrival rate will be the same as the completion rate:
\[\lambda=X\]

A system can be viewed as comprising several devices or resources, each of which can be analyzed individually as a system under the framework of operational laws. 
An external request triggers a job within the system, which may then circulate among the resources until all necessary processing is completed. 
As it reaches each resource, it is treated as a request, generating a job internal to that resource.

\paragraph*{Resource-based variables}
We define the following variables for each resource $k$:
\begin{table}[H]
    \centering
    \begin{tabular}{|c|l|}
    \hline
    $A_k$ & Number of request arrivals we observe for resource $k$      \\ \hline
    $C_k$ & Number of request completions we observe at resource $k$    \\ \hline
    $B_k$ & Total amount of time during which resource $k$ is busy      \\ \hline
    $N_k$ & Average number of jobs in resource $k$                      \\ \hline
    \end{tabular}
\end{table}
From these observed values, we can derive four important quantities for resource $k$:
\begin{table}[H]
    \centering
    \begin{tabular}{|c|l|}
    \hline
    $\lambda=\frac{A_k}{T}$ & Arrival rate                        \\ \hline
    $X=\frac{C_k}{T}$       & Throughput                          \\ \hline
    $U=\frac{B_k}{T}$       & Utilization                         \\ \hline
    $S=\frac{B_k}{C_k}$     & Mean service time per completed job \\ \hline
    \end{tabular}
\end{table}
\renewcommand*{\arraystretch}{1}

\subsection{Utilization law}
Using the previously determined observed values, we find the utilization law:
\[U=XS\]
This can also be used in resource-based analysis:
\[U_k=X_k S_k\]

\subsection{Little's law}
We can derive Little's law as:
\[N = XR\]
Little's law can be applied to the entire system as well as to subsystems:
\[N_k= X_kR_k\]

\paragraph*{Derivation}
Let's denote $W$ as the accumulated time in the system (in jobs per second). 
Then, we can express:
\[N=\dfrac{W}{T}  \qquad  R=\dfrac{W}{C}\]
Thus, we can write:
\[N=\dfrac{W}{T}=\dfrac{C}{T}\dfrac{W}{C}=XR\]

\subsection{Interactive response time law}
In interactive systems, the think time $Z$ is the time where the system is waiting for processing. 
The think time represents the time between processing being completed and the job becoming available as a request again. 
The interactive response time law states:
\[R =\dfrac{N}{X}-Z\]

\subsection{Forced flow law}
During an observation interval, we can compute the number of completions at each resource within the system. 
We define the visit count, $V_k$, of the $k$-th resource to be the ratio of the number of completions at that resource to the number of system completions:
\[V_k=\dfrac{C_k}{C}\]
It's important to note that:
\begin{itemize}
    \item If $C_k>C$, resource $k$ is visited several times on average during each system-level request. 
        This occurs when there are loops in the model.
    \item If $C_k<C$, resource $k$ might not be visited during each system-level request.
    \item If $C_k=C$, resource $k$ is visited on average exactly once every request.
\end{itemize}
The forced flow law captures the relationship between the different components within a system. 
It states that the throughput or flows in all parts of a system must be proportional to one another:
\[X_k=V_kX\]

\subsection{Demand law}
If we know the processing required by each job at a resource, we can calculate the resource's utilization. 
The service time $S_k$ is not necessarily the same as the response time of the job at that resource. 
Generally, a job might have to wait for some time before processing begins. 
The total amount of service that a job in the system generates at the $k$-th resource is termed the service demand:
\[D_k=S_kV_k\]

\subsection{General response time law}
When considering nodes characterized by visits different from one, we can define two permanence times:
\begin{itemize}
    \item \textit{Response time $\tilde{R}_k$}: this accounts for the average time spent in station $k$ when the job enters the corresponding node.
    \item \textit{Residence time $R_k$}: this accounts for the average time spent by a job at station $k$ during its stay in the system.
\end{itemize}
The relation between residence time and response time is similar to that between demand and service time:
\[\begin{cases}
    D_k=V_kS_k \\
    R_k=V_k\tilde{R}_k
\end{cases}\]
For single queue open systems, $V_k=1$, implying that the average service time and service demand are equal, and response time and residence time are identical.

If the mean number of jobs in the system or the system-level throughput are not known, an alternative method can be used. 
Applying Little's Law to the $k$-th resource, we find that:
\[N_k=X_k\tilde{R}_k\]

From the forced flow law, we can deduce that:
\[\dfrac{N_k}{X}=V_k\tilde{R}_k=R_k\]
The total number of jobs in the system is clearly the sum of the number of jobs at each resource. 
The general response time law is:
\[R=\sum_kV_k\tilde{R}_k=\sum_kR_k\]
he average response time of a job in the system is the sum of the product of the average time for the individual access at each resource and the number of visits it makes to that resource.

\subsection{Summary}
Operational laws offer simple equations that serve as an abstract representation or model of the average behavior of nearly any system. 
These laws are highly general and make minimal assumptions about the behavior of the random variables characterizing the system. 
Their simplicity allows for quick and easy application, making them valuable tools in system analysis and modeling.
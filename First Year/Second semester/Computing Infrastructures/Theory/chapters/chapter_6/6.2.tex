\section{Cloud Computing}

Cloud Computing is a model designed to enable convenient, on-demand network access to a shared pool of configurable computing resources.
These resources encompass networks, servers, storage, applications, and services, which can be rapidly provisioned and released with minimal management effort or service provider interaction.

\subsection{Cloud computing services}
Cloud Computing offers a diverse array of services:
\begin{figure}[H]
    \centering
    \includegraphics[width=0.5\linewidth]{images/xaas.png}
    \caption{Cloud Computing services}
\end{figure}

The primary service layers include:
\begin{itemize}
    \item \textit{Cloud application layer}: this layer is primarily composed of Software-as-a-Service (SaaS). 
        Users access services through web portals, often paying for usage. 
        Cloud applications are either developed within cloud software environments or use cloud infrastructure components. 
        Examples include Gmail and Google Docs.
    \item \textit{Cloud software environment layer}: this layer is mainly represented by Platform-as-a-Service (PaaS). 
        It serves application developers by offering a programming environment with a well-defined API. 
        This environment facilitates application-platform interaction, accelerates deployment, and supports scalability. 
        Examples in Deep Learning include Microsoft Azure Machine Learning and Google TensorFlow.
    \item \textit{Cloud software infrastructure layer}: this layer includes three main types of services:
        \begin{itemize}
            \item \textit{Infrastructure-as-a-Service} (IaaS): offers computational resources, including VMs and dedicated hardware. 
                Benefits include flexibility (dynamic allocation and scaling of resources) and super-user access (fine-grained settings and customization). 
                Challenges include performance interference (due to shared physical resources) and difficulty in guaranteeing consistent performance levels.
                Examples include Amazon Web Services, and Microsoft Azure. 
            \item \textit{Data-as-a-Service} (DaaS): provides storage capabilities, enabling users to store data on remote disks accessible from anywhere. 
                Key requirements include high dependability, replication for reliability, and data consistency. 
                Examples include Dropbox, iCloud, and Google Drive.
            \item \textit{Communication-as-a-Service} (CaaS): manages communications, ensuring Quality of Service (QoS) in cloud environments. 
                Key aspects include service-oriented communication capabilities, network security, and dynamic provisioning. 
                Examples include VoIP and video conferencing services.
        \end{itemize}
\end{itemize}

\subsection{Clouds taxonomy}
\paragraph*{Public clouds}
Public clouds offer extensive infrastructure available for rental, emphasizing customer self-service. 
Service Level Agreements (SLAs) are prominently advertised, with resources accessed remotely via the Internet.
Accountability is managed through e-commerce mechanisms, including web-based transactions and customer service provisions.

\paragraph*{Private clouds}
Private clouds involve internally managed data centers where an organization sets up a virtualization environment on its own servers. 
Benefits include total control over the infrastructure and leveraging virtualization advantages.
Limitations include capital investment and less flexibility compared to public clouds.
Private clouds are suitable for organizations with significant IT investments prioritizing control and security.

\paragraph*{Community clouds}
Community clouds are managed by multiple federated organizations, combining resources for shared use. 
They resemble private clouds but require a more complex accounting system. 
Community clouds can be hosted locally or externally, typically using the infrastructure of participating organizations or a specific hosting entity.

\paragraph*{Hybrid clouds}
Hybrid clouds combine elements of public, private, and community clouds. 
They are used by companies with private cloud infrastructure that may need additional resources during demand spikes.
Common interfaces are crucial for simplifying deployment across hybrid environments, ensuring consistency in managing VMs, addresses, and storage.
The Amazon EC2 model is a leading example of such cloud infrastructure.

\subsection{Edge Computing}
While Cloud Computing has been the primary solution for large-scale data storage and processing, the rise of intelligent mobile devices and IoT technologies demands real-time responses, context-awareness, and mobility.
Due to WAN-induced delays and the need for location-agnostic resource provisioning, Edge Computing has emerged.
This approach brings computing and storage closer to data sources, enabling faster response times and improved support for these requirements.
\section{Cloud computing}

Cloud computing is a model designed to facilitate convenient and on-demand network access to a shared pool of configurable computing resources. 
These resources include networks, servers, storage, applications, and services. 
They can be rapidly provisioned and released with minimal management effort or interaction with service providers.

\subsection{X-as-a-Service}
Cloud computing offers a wide range of services described using the 'as-a-Service' (XaaS) terminology. 
\begin{figure}[H]
    \centering
    \includegraphics[width=0.5\linewidth]{images/xaas.png}
    \caption{Main services provided by the cloud}
\end{figure}

\paragraph*{Cloud application layer}
The Cloud Application Layer primarily consists of Software as a Service (SaaS). 
Users access services provided by this layer through web portals, often requiring payment for usage. 
Cloud applications can be developed within cloud software environments or infrastructure components.
Examples of SaaS applications include: GMail, and Google Docs.

\paragraph*{Cloud software environment layer}
The cloud software environment layer primarily consists of Platform as a Service (PaaS). 
In this layer, users are typically application developers. 
Providers offer developers a programming-language-level environment with a well-defined API. 
This environment facilitates interaction between applications and the platform, accelerates deployment, and supports scalability.
Examples of PaaS services in Deep Learning include: Microsoft Azure Machine Learning, and Google TensorFlow.

\paragraph*{Cloud Software Infrastructure Layer}
The Cloud Software Infrastructure Layer encompasses several key services:
\begin{itemize}
    \item Infrastructure as a Service (IaaS) for computational resources.
    \item Data as a Service (DaaS) for storage capabilities.
    \item Communication as a Service (CaaS) for managing communications.
\end{itemize}
These services provide resources to the higher-level layers, such as software and software environment. 
It's worth noting that while cloud applications and cloud software may bypass the cloud software infrastructure layer, doing so can reduce simplicity and increase development efforts.

\paragraph*{Infrastructure as a Service}
Infrastructure as a Service (IaaS) offers the choice between Virtual Machines and dedicated hardware for computing resources.
The benefits are: 
\begin{itemize}
    \item Flexibility: VMs allow for dynamic allocation and scaling of resources based on demand.
    \item Super-user access: users have fine granularity settings and customization options for installed software within VMs.
\end{itemize}
The main challenges are: 
\begin{itemize}
    \item Performance interference: VMs may experience performance degradation due to sharing physical resources with other VMs.
    \item Inability to provide strong guarantees about Service Level Agreements: VMs might struggle to ensure consistent performance levels due to resource sharing.
\end{itemize}

\paragraph*{Data as a Service}
Data as a Service (DaaS) enables users to store their data on remote disks and access it from anywhere at any time. 
This facilitates the scalability of cloud applications beyond the limitations of their local servers.
Key requirements for DaaS include:
\begin{itemize}
    \item High dependability: Ensuring availability, reliability, and performance scalability.
    \item Replication: Duplicating data across multiple locations to enhance reliability and availability.
    \item Data consistency: Maintaining coherence and accuracy of data across distributed systems.
\end{itemize}
Examples of DaaS include Dropbox, iCloud, and Google Drive. 

\paragraph*{Communications as a Service}
Communications as a Service (CaaS) plays a crucial role in ensuring Quality of Service (QoS) in cloud environments.
Key aspects of CaaS include:
\begin{itemize}
    \item \textit{Service-oriented communication capability}: configurable, schedulable, predictable, and reliable communication services.
    \item \textit{Network security}: ensuring secure communication channels through encryption and monitoring.
    \item \textit{Dynamic provisioning}: creating virtual overlays for traffic isolation, allocating dedicated bandwidth, and guaranteeing message delay.
\end{itemize}
Types of CaaS services include Voice over Internet Protocol (VoIP), and video conferencing services. 

\subsection{Clouds}
\paragraph*{Public clouds}
Public clouds offer extensive infrastructure available for rental purposes. 
They align with the conventional definition of cloud computing, emphasizing fully customer self-service. 
Service Level Agreements (SLAs) are prominently advertised, and requests are handled through web services, with resources granted accordingly. 
Customers access these resources remotely via the Internet. 
Accountability is primarily managed through e-commerce mechanisms, involving web-based transactions, pay-as-you-go and flat-rate subscription models, as well as customer service provisions including refunds when necessary.

\paragraph*{Private clouds}
Private clouds entail internally managed data centers where the organization establishes a virtualization environment on its own servers, either within its own data center or within the facilities of a managed service provider.
Key benefits of private clouds include:
\begin{itemize}
    \item Total control over every aspect of the infrastructure.
    \item Leveraging the advantages of virtualization.
\end{itemize}
However, private clouds face limitations such as:
\begin{itemize}
    \item Lack of freedom from capital investment.
    \item Limited flexibility compared to the almost infinite scalability of public cloud computing.
\end{itemize}
Private clouds are particularly useful for companies with significant existing IT investments who prioritize control and security over the benefits of public cloud solutions.

\paragraph*{Community Clouds}
Community clouds are single clouds managed by multiple federated organizations, leveraging economies of scale by combining resources. 
This setup allows sharing and utilization of resources among participating organizations, even when others are not actively using them.
Technically, community clouds resemble private clouds, sharing the same software and encountering similar challenges. 
However, a more complex accounting system is typically necessary.
Community clouds can be hosted either locally or externally. 
Typically, they utilize the infrastructure of participating organizations.
Alternatively, they may be hosted by a separate specific organization or only by a small subset of the partners.

\paragraph*{Hybrid clouds}
Hybrid clouds combine elements from any of the previous types of clouds. 
Typically, these are utilized by companies that maintain their private cloud infrastructure but may experience unpredictable spikes in demand. 
In such cases, the company can rent additional resources from other types of clouds to meet their needs.
To simplify the deployment process in hybrid environments, common interfaces are crucial. 
These interfaces should ensure consistency in starting and terminating virtual machines, assigning addresses, and accessing storage across different cloud environments. 
Although many standards are being developed in this direction, none have gained global acceptance yet.
Currently, the Amazon EC2 model serves as a leading example with the most compliant infrastructure in this regard.

\subsection{Summary}
Advantages of Cloud Computing:
\begin{itemize}
    \item Lower IT costs.
    \item Improved performance.
    \item Instant software updates.
    \item Unlimited storage capacity.
    \item Increased data reliability.
    \item Universal document access.
    \item Device independence.
\end{itemize}
Disadvantages of Cloud Computing:
\begin{itemize}
    \item Requires a constant Internet connection.
    \item Does not work well with low-speed connections.
    \item Features might be limited.
    \item Can be slow.
    \item Stored data might not be secure.
    \item Lock-in.
\end{itemize}

\paragraph*{Edge Computing}
Cloud Computing has traditionally been the go-to solution for storing and processing large volumes of data. 
However, with the proliferation of intelligent and mobile devices, along with technologies like Internet of Things (IoT), V2X Communications, and Augmented Reality (AR), there's a growing demand for: real-time responses, support for context-awareness, and mobility. 
Due to the inherent delays introduced by wide-area networks (WAN) and the requirement for location-agnostic provisioning of resources on the cloud, there's a necessity to bring cloud-like features closer to consumer devices.
This need has given rise to Edge Computing, where computing and storage capacity are made available at or near the data sources themselves, enabling faster response times and improved support for the aforementioned requirements.
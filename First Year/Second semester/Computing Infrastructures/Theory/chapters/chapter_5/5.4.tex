\section{Performance bound analysis}

Performance bounds provide valuable insights into the fundamental factors influencing the performance of a computer system.
They can be computed swiftly and effortlessly, making them an ideal initial modeling technique.
Additionally, performance bounds allow for the simultaneous treatment of multiple alternatives, offering a comprehensive understanding of system performance.

Bounding analysis focuses on single-class systems and aims to establish asymptotic bounds, both upper and lower, on performance indices such as throughput and response time.
By employing bounding analysis, critical influences of system bottlenecks can be highlighted and quantified, offering valuable insights into system performance.
\begin{definition}[\textit{Bottleneck}]
    A bottleneck refers to the resource within a system that experiences the highest service demand.
\end{definition}
This resource, also known as the bottleneck device, plays a crucial role in limiting the overall performance of the system. 
Typically, the bottleneck resource exhibits the highest utilization within the system, making it a key determinant of system performance.

Bounding analysis enables the evaluation of numerous candidate configurations, focusing on the dominant resource. 
By treating multiple configurations as a single alternative, bounding analysis streamlines decision-making based on preliminary estimates.

\paragraph*{Asymptotic bounds}
Asymptotic bounds are established by examining the extreme conditions of light and heavy loads, yielding both optimistic and pessimistic scenarios. 
These bounds provide insight into system performance under different operational conditions:
\begin{itemize}
    \item \textit{Optimistic bounds}: represent the upper limit for system throughput and the lower limit for system response time.
    \item \textit{Pessimistic bounds}: represent the lower limit for system throughput and the upper limit for system response time.
\end{itemize}
These bounds are determined under two extreme conditions: light load and heavy load. 
The analysis assumes that the service demand of a customer at a center remains consistent, irrespective of the number of other customers present in the system or their locations within the service centers.

\subsection{Open models asymptotic bounds}
\paragraph*{Throughput}
Throughput represents the maximum arrival rate that the system can effectively process. 
If the arrival rate exceeds this bound, the system becomes saturated, leading to indefinite wait times for new jobs.
The throughput bound is calculated as the reciprocal of the maximum service demand $D_{\max}$:
\[X\leq\dfrac{1}{D_{\max}}\]
\paragraph*{Response time}
Response time refers to the largest and smallest possible response times experienced at a given arrival rate. 
These bounds are explored only when the arrival rate is less than the saturation arrival rate, as the system becomes unstable otherwise. 
Two extreme situations are considered:
\begin{enumerate}
    \item When no customers interfere with each other, resulting in response time equal to the sum of all service demands.
    \item When there is no pessimistic bound on response times due to batch arrivals. 
        As the batch size increases, more customers wait increasingly longer times, leading to no pessimistic bound on response times regardless of how small the arrival rate might be.
\end{enumerate}
In the general case we have:
\[R\geq\sum_kD_k\]

\subsection{Closed models asymptotic bounds}
\paragraph*{Throughput}
With multiple customers in the system, the system throughput can be expressed as:
\[X=\dfrac{N}{N\sum_kD_k+Z}\]
The largest throughput is achieved when jobs consistently find the queue empty, and service begins immediately:
\[X=\dfrac{N}{\sum_kD_k+Z}\]
The system's utilization at each resource is constrained to be less than or equal to one. 
Since the bottleneck resource is the first to saturate, the throughput has this constraint:
\[X\leq\frac{1}{D_{\max}}\] 
Thus, the bounds for the throughput are:
\[\dfrac{N}{N\sum_kD_k+Z}\leq X \leq \min\left(\frac{1}{D_{\max}},\dfrac{N}{\sum_kD_k+Z}\right)\]
The parameter $N^\ast$ indicates the population size at which either the light or the heavy load optimistic bound is applied, determined by:
\[N^\ast=\dfrac{\sum_kD_k+Z}{D_{\max}}\]
\paragraph*{Response time}
From the throughput bounds, we obtain the following:
\[\max\left(\sum_kD_k, ND_{\max}-Z\right)\leq R \leq N\sum_kD_k\]

\subsection{What-if analysis}
What-if analysis is a decision-making technique that explores potential outcomes by varying one or more input variables within a model or system. 
This method allows users to evaluate the impact of different decisions or changes by simulating their effects on the system's results. 
By adjusting parameters and observing the corresponding changes, decision-makers can gain valuable insights into potential risks, opportunities, and optimal strategies.
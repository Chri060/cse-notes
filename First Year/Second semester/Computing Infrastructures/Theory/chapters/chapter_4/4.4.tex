\section{RAID four}

In RAID four, the $N$-th disk only stores parity information for the other $N-1$ disks. 
The parity bit is computed using the XOR operation.

\paragraph*{Writes}
There are two methods for updating parity when blocks are written:
\begin{itemize}
    \item Additive parity involves reading other blocks and then updating the parity block. 
        This method calculates the new parity value by XORing the old parity value with the new parity value.
    \item Subtractive parity involves updating the disk and then computing the new parity value. 
        This method calculates the new parity value by XORing the old parity value with the new parity value.
\end{itemize}

\paragraph*{Reads}
In RAID 4, reads are not a problem because the data is distributed evenly across all non-parity blocks in the stripe. 
This ensures that read operations can be performed efficiently without any significant performance reduction due to the parity disk.

RAID 4 has the same read and write performance, with parallelization across all non-parity blocks in the stripe. 
This means that all writes on the same stripe update the parity drive once.

\paragraph*{Random writes}
In RAID 4, random writes can cause a bottleneck due to the need to update the parity drive, which can cause serialization. 
The process of writing to a RAID 4 array involves reading the target block and the parity block, calculating the new parity block using subtraction, and then writing the target block and the new parity block. 
This process can cause performance issues, especially if the parity drive is slow or has limited bandwidth.

\subsection{Summary}
The following is a summary of the features of RAID 1:
\begin{itemize}
    \item The system has a capacity of $N-1$, meaning that the space on the parity drive is lost. 
    \item The system has a reliability of $1$, meaning that one drive can fail. 
        This can result in massive performance degradation during a partial outage.
    \item The system supports sequential read and write operations of $(N-1)\times S$, which means that it can perform parallelization across all non-parity blocks in the stripe.
    \item The system supports parallelization of read operations over all drives except the parity drive, with a maximum of $(N-1)\times R$ simultaneous reads.
    \item The system supports random write operations of $\frac{R}{2}$, which means that writes serialize due to the parity drive. 
        Each write requires one read and one write of the parity drive.
\end{itemize}
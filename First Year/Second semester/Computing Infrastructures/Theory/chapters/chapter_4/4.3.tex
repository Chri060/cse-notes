\section{Datacenters dependability}

In the context of data centers, downtime is a major concern. 
Aberdeen Research provides statistics on downtimes and incidents:
\begin{itemize}
    \item Average performing facilities experience downtime of 60 minutes with 2.3 incidents per year.
    \item Best-in-class organizations have downtime as low as 6 minutes with only 0.3 incidents per year.
\end{itemize}

Within a data center, the components requiring dependability include:
\begin{itemize}
    \item Computing systems, sensors, and actuators (referred to as nodes).
    \item Network infrastructure facilitating communication.
    \item Cloud services encompassing data storage and manipulation capabilities.
\end{itemize}
The smooth operation of all these elements is essential for the system's overall functionality.

\subsection{Dependability requirements}
To ensure dependability, a resilient computing system must adhere to the failure avoidance paradigm, which includes:
\begin{itemize}
    \item Employing a conservative design approach.
    \item Validating the design thoroughly.
    \item Conducting detailed testing of both hardware and software components.
    \item Implementing an infant mortality screen to identify and address early failures.
    \item Focusing on error avoidance strategies.
    \item Incorporating mechanisms for error detection and masking during system operation.
    \item Utilizing on-line monitoring tools.
    \item Implementing diagnostics for identifying issues.
    \item Enabling self-recovery and self-repair capabilities.
\end{itemize}
In practice, dependable systems can be obtained through:
\begin{itemize}
    \item Robust design, which focuses on error-free processes and design practices.
    \item Robust operation, achieved through fault-tolerant measures such as monitoring, detection, and mitigation.
\end{itemize}

\paragraph*{Safety-critical systems}
Safety-critical systems encompass all components collaborating to fulfill the safety-critical mission. 
These components may encompass input sensors, digital data devices, hardware, peripherals, drivers, actuators, controlling software, and other interfaces.
Their development necessitates thorough analysis, comprehensive design, and rigorous testing.

\subsection{Dependability in practice}
Dependability can be enhanced at various levels:
\begin{itemize}
    \item At the technological level: by incorporating reliable and robust components during design and manufacturing.
    \item At the architectural level: by integrating standard components with solutions designed to handle potential failures effectively.
    \item At the application level: by developing algorithms or operating systems that can mask and recover from failures.
\end{itemize}
However, it's important to note that regardless of the level, enhancing dependability typically entails increased costs and a potential reduction in performance.

The primary challenges of dependability include:
\begin{itemize}
    \item Designing robust systems using unreliable and cost-effective Commercial Off-The-Shelf (COTS) components and integrating them seamlessly.
    \item Addressing new challenges arising from technological advancements, such as process variations, stressed working conditions, and the emergence of failure mechanisms previously unnoticed at the system-level due to smaller geometries.
    \item Striking the optimal balance between dependability and costs, which is contingent upon factors like the application field, working scenario, employed technologies, algorithms, and applications.
\end{itemize}
In the application scenario achieving 100\% dependability may entail significant costs and overheads but is deemed justified.

Dependability becomes a trade-off with performance and power consumption, necessitating careful consideration of trade-offs.
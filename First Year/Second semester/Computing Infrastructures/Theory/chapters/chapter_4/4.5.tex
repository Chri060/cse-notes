\section{RAID five}

In RAID 5, the parity blocks are distributed evenly across all N disks. 
Unlike RAID 4, writes are spread evenly across all drives.
Random writes in RAID 5 involve the following steps:
\begin{enumerate}
    \item Read the target block and the parity block.
    \item Calculate the new parity block by subtracting the old parity block from the target block.
    \item Write the target block and the new parity block.
\end{enumerate}
This results in a total of 4 operations (2 reads and 2 writes) being distributed evenly across all drives.

\subsection{Summary}
The following is a summary of the features of RAID 1:
\begin{itemize}
    \item The system has a capacity of $N-1$, meaning that the space on the parity drive is lost. 
    \item The system has a reliability of $1$, meaning that one drive can fail. 
        This can result in massive performance degradation during a partial outage.
    \item The system supports sequential read and write operations of $(N-1)\times S$, which means that it can perform parallelization across all non-parity blocks in the stripe.
    \item The system supports parallelization of read operations over all drives, with a maximum of $N\times R$ simultaneous reads.
    \item The system supports random write operations of $\frac{N}{4}R$, which means that writes parallelize over all drives. 
        Each write requires two reads and two writes. 
\end{itemize}
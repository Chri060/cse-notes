\section{RAID six}

RAID 6 is a type of RAID level that provides more fault tolerance than RAID 5. 
It allows for two concurrent failures to be tolerated, which means that if two of the disks in the array fail, the data can still be recovered from the remaining disks. 
RAID 6 uses Solomon-Reeds codes with two redundancy schemes, which means that there are two different ways to encode the data on the disks. 
The (P+Q) distribution is distributed and independent, which means that the parity blocks are spread across all the disks in the array. 
This provides a higher level of fault tolerance than RAID 5. To implement RAID 6, a minimum of four data disks and two parity disks are required. 
Each write operation requires six disk accesses, as the data must be written to both the P and Q parity blocks, which can result in slow writes.
\section{Edge computing systems}

Edge computing is a distributed computing model in which data processing occurs as close as possible to where the data is generated, improving response times and saving on bandwidth. 
Processing data near the location where it is generated brings significant advantages in terms of processing latency, reduced data traffic, and increased resilience in case of data connection interruptions.
Edge computing systems can be categorized as follows:
\begin{itemize}
    \item \textit{Cloud}: providing virtualized computing, storage, and network resources with highly elastic capacity.
    \item \textit{Edge servers}: utilizing on-premises hardware resources for more computationally intensive data processing.
    \item \textit{IoT and AI-enabled edge sensors}: enabling data acquisition and partial processing at the edge of the network.
\end{itemize}
Edge computing offers several advantages, including high computational capacity, distributed computing capabilities, enhanced privacy and security, and reduced latency in decision-making. 
However, it comes with drawbacks such as the requirement for a power connection and dependence on a connection with the Cloud.

\subsection{Embedded PC}
An embedded system refers to a computer system that comprises a computer processor, computer memory, and input/output peripheral devices, all serving a specific function within a larger mechanical or electronic system.
Advantages of this approach include its pervasiveness in computing, high-performance units, availability of development boards, ease of programming similar to personal computers, and the support of a large community. 
On the other hand, it has disadvantages such as relatively high power consumption and the necessity for some hardware design work to be done.

\subsection{Internet of things}
The internet of things (IoT) encompasses devices equipped with sensors, processing capabilities, software, and other technologies. 
These devices are designed to connect and exchange data with other devices and systems over the internet or other communication networks.
Advantages of IoT devices include their high pervasiveness, wireless connectivity, battery-powered operation, low costs, and their ability to sense and actuate.
However, these devices also come with several disadvantages, such as their low computing ability, constraints on energy usage, limitations in memory (RAM/FLASH), and difficulties in programming them.
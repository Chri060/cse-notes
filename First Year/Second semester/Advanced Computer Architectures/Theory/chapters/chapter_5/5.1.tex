\section{Introduction}

\begin{definition}[\textit{Interrupt}]
    An interrupt refers to an external or internal event necessitating processing by another system program.
\end{definition}
Typically unexpected or infrequent from the program's perspective, interrupts can stem from various causes:
\begin{itemize}
    \item \textit{Asynchronous}: stemming from external events, such as input/output device service requests, timer expirations, power disruptions, or hardware failures.
    \item \textit{Synchronous}: arising from internal events (exceptions), including undefined opcodes, privileged instructions, arithmetic overflows, FPU exceptions, misaligned memory access, virtual memory exceptions (such as page faults, TLB misses, and protection violations), and traps (system calls).
\end{itemize}

\subsection{History}
The first system to incorporate exceptions was the Univac-I in 1951, where an arithmetic overflow would either trigger the execution of a two-instruction fix-up routine at address 0 or, optionally, cause the computer to halt.

The Univac 1103, modified in 1955, introduced external interrupts for gathering real-time wind tunnel data.

The DYSEAC in 1954 was the first system with I/O interrupts, featuring two program counters, and an I/O signal facilitated switching between them. 
Additionally, it was the first system to incorporate DMA (Direct Memory Access) by I/O devices.
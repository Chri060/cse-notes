\section{Taxonomy}

Exceptions can be categorized as follows:
\begin{itemize}
    \item \textit{Synchronous} or \textit{asynchronous}: asynchronous events stem from devices external to the CPU and memory, manageable after the current instruction completes (easier to handle).
    \item \textit{User requested} or \textit{coerced}: user requested exceptions are predictable and treated similarly to exceptions, utilizing the same mechanisms for saving and restoring state; they are handled after instruction completion. 
        Coerced exceptions arise from hardware events beyond the program's control.
    \item \textit{User maskable} or \textit{user nonmaskable}: the mask dictates whether the hardware responds to the exception.
    \item \textit{Within instructions} or \textit{between instructions}: exceptions occurring within instructions are typically synchronous as the instruction initiates the exception.
        The instruction must halt and restart. 
        Asynchronous exceptions between instructions arise from critical situations, leading to program termination.
    \item \textit{Resume} or \textit{terminate}: terminating events result in program execution always halting after the interrupt. 
        Resuming events allow program execution to continue after the interrupt.
\end{itemize}
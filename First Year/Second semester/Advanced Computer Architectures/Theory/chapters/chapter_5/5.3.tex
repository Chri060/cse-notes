\section{Interrupts}

\paragraph*{Asynchronous interrupt}
In case of asynchronous interrupt an I/O device signals the need for attention by activating a prioritized interrupt request line.
Upon the processor's decision to handle the interrupt:
\begin{enumerate}
    \item Execution halts at instruction $I_i$ of the current program, ensuring completion of all instructions up to $I_{i-1}$ (precise interrupt).
    \item The processor stores the program counter (PC) value of instruction $I_i$ in a dedicated register (EPC).
    \item Interrupts are disabled, and control shifts to a specified interrupt handler operating in kernel mode.
\end{enumerate}
The interrupt handler: 
\begin{itemize}
    \item Before enabling interrupts to accommodate nested interrupts:
        \begin{itemize}
            \item It saves the program counter (PC) to facilitate nested interrupts.
            \item Requires an instruction to transfer the PC into general-purpose registers (GPRs).
            \item Requires a mechanism to temporarily block further interrupts until the PC is saved.
        \end{itemize}
    \item It retrieves information about the interrupt cause from a designated status register.
    \item Utilizes a specialized indirect jump instruction called Return-From-Exception (RFE), which:
        \begin{itemize}
            \item Enables interrupts.
            \item Restores the processor to user mode.
            \item Reinstates hardware status and control state.
        \end{itemize}
\end{itemize}

\paragraph*{Synchronous interrupt}
A synchronous interrupt, also known as an exception, is triggered by a specific instruction.
Typically, the instruction cannot finish execution and must be restarted after handling the exception.
This necessitates undoing the impact of one or more partially executed instructions.
However, in the scenario of a system call trap, the instruction is deemed as fully executed.
This involves a special jump instruction that transitions to privileged kernel mode.

\subsection{Precise interrupt}
\begin{definition}[\textit{Precise interrupt}]
    An interrupt or exception is deemed precise when a singular instruction (or interrupt point) exists at which all preceding instructions have finalized their state, and no subsequent instructions, including the interrupting instruction, have altered any state.
\end{definition}
This implies that you can effectively resume execution from the interrupt point and obtain the correct outcome.

Precise interrupts are desirable for several reasons. 
They facilitate the restart of various interrupt and exception types. 
Additionally, they simplify the process of determining the exact cause of the interruption.

While restartability doesn't mandate preciseness, it significantly enhances the ease of restarting. 
Preciseness notably streamlines the task for the operating system: less state must be preserved when unloading processes, and restarts are quicker.
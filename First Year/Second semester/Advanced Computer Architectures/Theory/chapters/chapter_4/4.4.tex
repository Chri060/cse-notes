\section{Virtual memory}

Virtual memory is utilized to confine processes within their allocated memory space boundaries. 
This architecture serves multiple functions:
\begin{itemize}
    \item It facilitates user mode and supervisor mode.
    \item It safeguards specific CPU state components.
    \item It incorporates mechanisms for transitioning between user mode and supervisor mode.
    \item It provides tools for restricting memory accesses.
    \item It includes a Translation Lookaside Buffer (TLB) for address translation.
\end{itemize}

\subsection{Virtual machines}
Virtual memory's concept enables the creation of virtual machines.
These machines support isolation and security, allowing multiple unrelated users to share a computer. 
This capability is made feasible by the processors' raw speed, which mitigates the associated overhead.

Virtual machines enable the presentation of different Instruction Set Architectures (ISAs) and operating systems to user programs.

The software responsible for system virtual machines is called a hypervisor, with individual virtual machines operating under it referred to as guest virtual machines.
Each guest operating system maintains its set of page tables:
\begin{itemize}
    \item The hypervisor introduces a layer of memory between physical and virtual memory, termed real memory.
    \item The hypervisor maintains a shadow page table mapping guest virtual addresses to physical addresses. 
        This necessitates the hypervisor's ability to detect changes made by the guest to its page table, which naturally occurs if accessing the page table pointer is a privileged operation.
\end{itemize}
\section{Introduction}

Instruction-Level Parallelism (ILP) involves the simultaneous execution of independent instructions. 
Achieving ILP requires:
\begin{itemize}
    \item Absence of structural hazards.
    \item No stalls due to Read After Write (RAW), Write After Read (WAR), or Write After Write (WAW) dependencies.
    \item No stalls due to control dependencies.
\end{itemize}

The CPI for a pipeline is calculated using
\[\text{CPI}_{\text{pipeline}}=\text{CPI}_{\text{ideal pipeline}} + \text{stalls}_{\text{structural}} + \text{stalls}_{\text{data hazard}} + \text{stalls}_{\text{control}}\]
In this formula, $\text{CPI}_{\text{ideal pipeline}}$ denotes the optimal performance attainable by the pipeline.

Performance limitations due to hazards arise in several ways:
\begin{itemize}
    \item Structural hazards demand additional hardware resources.
    \item Data hazards require solutions such as forwarding and compiler scheduling.
    \item Control hazards can be addressed through early evaluation, program counter adjustments, delayed branching, and prediction techniques.
\end{itemize}
The influence of these hazards becomes more pronounced with longer pipelines.
Pipelining enhances instruction throughput rather than reducing execution latency.
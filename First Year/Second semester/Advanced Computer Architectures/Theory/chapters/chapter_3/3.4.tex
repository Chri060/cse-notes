\section{Benchmarks}

The conventional method for conducting performance tests on programs involves using benchmarks.
In this methodology, certain groups select programs available to the community to measure performance.
These programs are executed on machines, and their performance is reported, allowing for comparison with reports from other machines.

The most commonly used benchmarks include:
\begin{itemize}
    \item \textit{Real programs}: these are representative of real workloads and provide the most accurate way to characterize performance. 
        Occasionally, modified CPU-oriented benchmarks may eliminate I/O operations.
    \item \textit{Kernels} or microbenchmarks: these are representative program fragments that are useful for focusing on individual features.
    \item \textit{Synthetic benchmarks}: similar to kernels, these benchmarks attempt to match the average frequency of operations and operands from a large set of programs.
    \item \textit{Instruction mixes} for CPI. 
\end{itemize}

\subsection{System performance evaluation cooperative}
The System Performance Evaluation Cooperative was established in 1989 to address bench marketing issues. 
SPEC2000, for example, is based on 12 integer and 14 floating-point programs, and it is utilized to evaluate the CPU, memory architecture, and compilers' compute-intensive performance.

\subsection{Benchmark problems}
Benchmarks may not be representative if the workload is I/O bound, rendering certain benchmarks like SPECint ineffective.
Benchmarks also become obsolete over time, and aging benchmarks can be problematic as bench marketing pressure incentivizes vendors to optimize compiler/hardware/software to specific benchmarks. 
Therefore, benchmarks need to be periodically refreshed.

\subsection{Alternatives}
A straightforward method for comparing relative performance is to use the total execution time of the two programs. 
Another option is to calculate the arithmetic mean of the execution times, which is valid only if the programs run equally often. 
If the programs have different frequencies of execution, the weighted arithmetic mean is used:
\[\left\{ \sum_{i=1}^{n} \text{weight}(i) \cdot \text{time}(i) \right\} \div \dfrac{1}{n}\]
In general:
\begin{itemize}
    \item Use the arithmetic mean for times.
    \item Use the harmonic mean if rates must be used.
    \item Use the geometric mean if ratios must be used.
\end{itemize}
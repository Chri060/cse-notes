\section{Energy and power}

Energy and power consumption pose constraints for a variety of systems, including embedded systems, IoT devices, and mobile devices due to battery capacities, as well as for desktops, servers, and HPC clusters in terms of energy costs.
Consequently, it becomes necessary to establish an energy and power budget for these systems. 
Achieving optimization in this regard requires a thorough understanding of the underlying phenomenon.

\subsection{Energy consumption}
Thermal Design Power (TDP) serves as a metric to characterize sustained power consumption. 
It is used as a target for power supply and cooling systems and typically falls between peak power and average power consumption. 
The clock rate can be dynamically reduced to limit power consumption, and measuring energy per task often provides a more accurate assessment.

Various techniques are employed to reduce dynamic power consumption:
\begin{itemize}
    \item Optimizing existing processes.
    \item Implementing dynamic voltage-frequency scaling.
    \item Employing low-power states for DRAM and disks.
    \item Utilizing methods like overclocking and turning off cores.
\end{itemize}
Additionally, static power consumption, which scales with the number of transistors, needs to be considered. 
To mitigate static power, a technique called power gating is commonly employed.
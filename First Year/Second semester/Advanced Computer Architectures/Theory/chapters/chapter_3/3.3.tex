\section{Performance measures}

The system's performance can be described as follows:
\begin{itemize}
    \item \textit{Response time}: this encompasses the latency resulting from completing a task, which includes factors like disk accesses, I/O activity, and operating system overhead. 
        The elapsed time is calculated as the sum of CPU time and I/O wait time:
        \[\text{elapsed time}=\text{CPU time} + \text{I/O wait}\]
    \item \textit{CPU time}: this includes the time spent waiting for I/O operations and corresponds to the CPU's processing time. 
        It can be calculated as follows:
        \[\text{CPU time }(P) = \dfrac{\text{clock cycles needed to execute } P}{\text{clock frequency}}\] 
        \[\text{CPU time }(P) = \text{clock cycles needed to execute } P \times \text{clock cycles time}\] 
\end{itemize}

SLIDE 42






















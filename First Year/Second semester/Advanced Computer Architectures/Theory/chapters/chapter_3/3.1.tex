\section{Introduction}

In computer architecture, handling branches efficiently is crucial for optimizing processor performance. 
Branches introduce potential stalls in instruction execution as the correct path of execution is determined. 
The aim is to predict the outcome of branch instructions as early as possible to minimize these stalls and improve overall performance. 
The effectiveness of branch prediction techniques hinges on three primary factors:
\begin{enumerate}
    \item \textit{Accuracy}: this metric reflects the percentage of correct predictions made by the branch predictor. 
        Higher accuracy reduces unnecessary stalls caused by mispredictions.
    \item \textit{Cost of misprediction}: refers to the penalty incurred when a branch prediction turns out to be incorrect. 
        In deeply pipelined processors, mispredictions can lead to significant performance degradation due to the time lost in executing unnecessary instructions.
    \item \textit{Branch frequency}: the frequency at which branches occur within a program. 
        Programs with frequent branches benefit significantly from accurate prediction techniques to maintain performance.
\end{enumerate}
To mitigate the performance impact of branch hazards, several methods are employed:
\begin{itemize}
    \item \textit{Static branch prediction}: these techniques rely on predefined actions for branches that remain constant throughout program execution.
        Predictions are determined statically at compile time based on heuristics or program structure.
    \item \textit{Dynamic branch prediction }: unlike static techniques, dynamic approaches adjust branch predictions during program execution based on runtime behavior and historical data. 
        This adaptability improves prediction accuracy for varying branch conditions.
\end{itemize}
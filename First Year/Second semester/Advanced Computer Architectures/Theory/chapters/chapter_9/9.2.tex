\section{Parallel architectures}

Increasing the performance and clock frequency of a single core has become increasingly difficult due to several factors. 
Deep pipeline designs, which have been a common strategy, face significant issues:
\begin{itemize}
    \item \textit{Heat dissipation}: managing the heat generated by high-frequency operations is problematic.
    \item \textit{Speed of light limitations}: The physical limits of light speed affect transmission speed in wires.
    \item \textit{Design complexity}: designing and verifying deep pipelines is challenging and requires large design teams.
    \item \textit{Multithreaded applications}: many new applications are inherently multithreaded, demanding better support for parallel execution.
\end{itemize}

\paragraph*{Beyond Instruction-Level Parallelism}
ILP architectures, such as superscalar and VLIW, are designed to exploit fine-grained, instruction-level parallelism. 
However, they are not well-suited for supporting large-scale parallel systems. 
Multiple-issue CPUs, which attempt to issue multiple instructions per cycle, have become extremely complex, and the benefits of extracting additional parallelism are diminishing. 
Consequently, extracting parallelism at higher levels has become more attractive.

\subsection{Process and Thread-Level Parallel Architectures}
To achieve higher performance, the next step involves process- and thread-level parallel architectures. 
This approach focuses on connecting multiple microprocessors in a complex system to handle large-scale parallel tasks efficiently.
\begin{definition}[\textit{Parallel architecture}]
    A parallel computer is a collection of processing elements that cooperate and communicate to solve large problems quickly.
\end{definition}
The goal is to replicate processors to enhance performance rather than merely designing a faster single processor. 
Parallel architecture extends traditional computer architecture by incorporating a communication architecture that includes:
\begin{itemize}
    \item \textit{Abstractions}: defining the hardware/software interface.
    \item \textit{Efficient structures}: developing various structures to realize these abstractions efficiently.
\end{itemize}
By leveraging parallel architectures, systems can better support multithreaded applications and large-scale computational tasks, addressing the limitations of ILP and deep pipeline designs.
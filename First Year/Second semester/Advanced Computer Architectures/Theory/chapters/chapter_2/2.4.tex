\section{Virtual memory}

Virtual memory is a crucial architecture used to confine processes within their allocated memory space boundaries, providing several key functions:
\begin{itemize}
    \item \textit{User mode and supervisor mode}: facilitates distinct execution modes for user applications and privileged operating system tasks.
    \item \textit{CPU state protection}: safeguards critical components of the CPU state from unauthorized access.
    \item \textit{Mode transition mechanisms}: implements mechanisms for transitioning between user and supervisor modes securely.
    \item \textit{Memory access control}: provides tools to restrict memory accesses, ensuring security and isolation.
    \item \textit{Translation Lookaside Buffer} (TLB): accelerates virtual to physical address translation by caching recent mappings.
\end{itemize}

\subsection{Virtual machines}
Virtual memory forms the foundation for creating virtual machines (VMs), which enhance isolation and security in computing environments where multiple users or tasks share a single physical machine. 
The key functionalities of VMs include:
\begin{itemize}
    \item \textit{Isolation and security}: ensures that each VM is isolated from others, preventing interference and ensuring security for diverse workloads.
    \item \textit{Support for different architectures}: presents distinct ISAs and operating systems to user programs concurrently.
\end{itemize}

\paragraph*{Hypervisor and guest VM}
A hypervisor, or VM monitor (VMM), is software responsible for managing and running multiple guest VMs on a single physical machine. 
Key aspects of MV management include:
\begin{itemize}
    \item \textit{Guest page tables}: each guest operating system maintains its own set of page tables, which map virtual addresses used by applications running inside the guest to physical addresses in the real memory managed by the hypervisor.
    \item \textit{Real memory}: the hypervisor introduces a layer of abstraction between physical memory and virtual memory used by guest VMs.
    \item \textit{Shadow page tables}: to efficiently manage address translations, the hypervisor maintains shadow page tables, which mirror the guest's page tables but map directly to physical memory. 
        This allows the hypervisor to detect changes made by the guest to its page tables, ensuring consistency and security.
\end{itemize}
Virtual memory and VMs enable efficient resource utilization and flexibility in managing computing resources, making them essential components in modern computing environments.
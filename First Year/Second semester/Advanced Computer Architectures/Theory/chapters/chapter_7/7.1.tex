\section{Introduction}

Instruction-level parallelism refers to the potential overlap of execution among unrelated instructions.
Overlap is achievable when:
\begin{itemize}
    \item There are no structural hazards.
    \item There are no stalls due to Read-After-Write (RAW), Write-After-Read (WAR), or Write-After-Write (WAW) dependencies.
    \item There are no stalls due to control dependencies.
\end{itemize}

\subsection{Pipeline evaluation}
The CPI of a pipeline is computed as: 
\[\text{CPI}_{pipeline}=\text{CPI}_{ideal\:pipeline} + \text{Stalls}_{structural} + \text{Stalls}_{data\:hazard} + \text{Stalls}_{control}\]
Here: 
\begin{itemize}
    \item Ideal pipeline CPI represents the maximum performance achievable by the implementation.
    \item Structural hazards occur when the hardware cannot support a specific combination of instructions.
    \item Data hazards arise when an instruction depends on the result of a prior instruction still in the pipeline.
    \item Control hazards are caused by delays between the fetching of instructions and decisions regarding changes in control flow, such as branches, jumps, or exceptions.
\end{itemize}
Hazards constrain performance in several ways:
\begin{itemize}
    \item Structural hazards necessitate additional hardware resources.
    \item Data hazards require mechanisms like forwarding and compiler scheduling.
    \item Control hazards can be mitigated through early evaluation, program counter adjustments, delayed branch techniques, and predictors.
\end{itemize}
As the length of the pipeline increases, the impact of hazards becomes more significant.

Pipelining primarily enhances instruction bandwidth rather than reducing latency.

\subsection{Data hazard}
Data hazards can occur in the following scenarios:
\begin{itemize}
    \item \textit{Read-after-write} (RAW): when an instruction attempts to read data from a register or memory location that has been modified by a preceding instruction.
    \item \textit{Write-after-read} (WAR): when an instruction attempts to write data to a register or memory location before it has been read by a preceding instruction.
    \item \textit{Write-after-write} (WAW): when two instructions attempt to write to the same register or memory location in quick succession without an intervening read operation.
\end{itemize}
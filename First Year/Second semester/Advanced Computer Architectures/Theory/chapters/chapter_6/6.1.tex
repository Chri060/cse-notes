\section{Introduction}

Every branch incurs a single stall to fetch the correct instruction flow: either the (PC+4) or the branch target address. 
Our goal is to predict the outcome of a branch instruction as early as possible to enhance performance. 
The performance of a branch prediction technique depends on three key factors:
\begin{enumerate}
    \item \textit{Accuracy}: this is determined by the percentage of incorrect predictions made by the predictor.
    \item \textit{Cost of misprediction}: this refers to the time lost executing unnecessary instructions due to an incorrect prediction (misprediction penalty). 
        This cost increases notably in deeply pipelined processors.
    \item \textit{Branch frequency}: the frequency of branches within the application plays a significant role. 
        Accurate branch prediction is particularly crucial in programs with higher branch frequencies.
\end{enumerate}
Various methods address the performance degradation resulting from branch hazards:
\begin{itemize}
    \item \textit{Static branch prediction techniques}: predefined actions for each branch remain constant throughout program execution, determined at compile time.
    \item \textit{Dynamic branch prediction techniques}: decisions leading to branch prediction may alter during program execution.
\end{itemize}
In both approaches, caution is essential to avoid altering the processor state until the branch outcome is definitively determined.
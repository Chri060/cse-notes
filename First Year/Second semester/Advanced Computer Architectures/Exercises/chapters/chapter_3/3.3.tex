\section{Exercise 3}

Consider the following assembly program: 
\begin{verbnobox}[\verbarg]
LW 1, OFF(2)
ADDI 3, 1, 4
SUB 4, 1, 2
ADDI 2, 1, -8
SW 4, OFF(2)
\end{verbnobox}
No optimizations are applied in the MIPS pipeline. 
The processor operates with a clock cycle of $2$ ns. 
\begin{itemize}
    \item Draw the pipeline schema and highlight potential hazards.
    \item Illustrate the actual execution with stall cycles inserted.
    \item Calculate Instruction Count (IC), CPI, and MIPS.
\end{itemize}

\subsection*{Solution}
\begin{enumerate}
    \item The pipeline schema is:
        \begin{table}[H]
            \centering
            \begin{tabular}{l|ccccccccc}
            \textbf{Instruction}        & \textbf{1} & \textbf{2} & \textbf{3} & \textbf{4} & \textbf{5} & \textbf{6} & \textbf{7} & \textbf{8} & \textbf{9}  \\ \hline
            \textit{LW 1, OFF (2)}  & F          & D          & E          & M          & W          &            &            &            &            \\
            \textit{FADDI 3, 1, 4}   &            & F          & D          & E          & M          & W          &            &            &            \\
            \textit{SUB 4, 1, 3}  &            &            & F          & D          & E          & M          & W          &            &            \\
            \textit{ADDI 2, 1, -8}  &            &            &            & F          & D          & E          & M          & W          &            \\
            \textit{SW 5, OFF (2)}  &            &            &            &            & F          & D          & E          & M          & W         
            \end{tabular}
        \end{table}
        The potential hazards are:
        \begin{itemize}
            \item Instruction 1 writes to register $1$, and instructions 2, 3, and 4 read from it.
            \item Instruction 2 writes to register $3$, and instruction 3 reads from it.
            \item Instruction 4 writes to register $2$, and instruction 5 reads from it.
        \end{itemize}
    \item The real execution with stall cycles inserted is:
        \begin{table}[H]
            \centering
            \begin{tabular}{l|ccccccccccccccc}
            \textbf{Instruction}        & \textbf{1} & \textbf{2} & \textbf{3} & \textbf{4} & \textbf{5} & \textbf{6} & \textbf{7} & \textbf{8} & \textbf{9} & \textbf{10} & \textbf{11} & \textbf{12} & \textbf{13} & \textbf{14} & \textbf{15} \\ \hline
            \textit{LW 1, OFF (2)}  & F          & D          & E          & M          & W          &            &            &            &            &             &             &             &             &             &             \\
            \textit{ADDI 3, 1, 4}   &            & F          & \underline{S}    & \underline{S}     & D          & E          & M          & W          &            &             &             &             &             &             &             \\
            \textit{SUB 4, 1, 3}  &            &            &            &            & F          & \underline{S}     & \underline{S}     & D          & E          & M           & W           &             &             &             &             \\
            \textit{ADDI 2, 1, -8}  &            &            &            &            &            &            &            & F          & D          & E           & M           & W           &             &             &             \\
            \textit{SW 5, OFF (2)}  &            &            &            &            &            &            &            &            & F          & \underline{S}      & \underline{S}      & D           & E           & M           & W          
            \end{tabular}
        \end{table}
    \item The performance metrics are: 
        \[\text{IC}=5\]
        \[\text{CPI}=\dfrac{\text{CCs}}{\text{IC}}=\dfrac{15}{5}=3\]
        \[\text{MIPS}=\dfrac{\text{clock frequency}}{\text{CPI} \cdot 10^6}=\dfrac{0.5\cdot 10^9}{3 \cdot 10^6}=166\]
\end{enumerate}
\section{Exercise 5}

Consider the ARMA($1,1$) process described by the expression:
\[y(t)=\dfrac{1}{2}y(t-1)+e(t)-e(t-1) \qquad e(t)\sim WN(1,9)\]
\begin{enumerate}
    \item Determine the transfer function and verify if it is a stationary stochastic process.
    \item Calculate the expected value.
    \item Compute the covariance function.
\end{enumerate}

\subsection*{Solution}
\begin{enumerate}
    \item We express the formula in operatorial representation:
        \[y(t)=\dfrac{1}{2}y(t)z^{-1}+e(t)-e(t)z^{-1}\rightarrow y(t)=\dfrac{z-1}{z-\frac{1}{2}}e(t)\]
        The system exhibits a zero at $z=1$ and a pole in $z=\frac{1}{2}$, indicating asymptotic stability.
        As the input,  White Noise, is a stationary stochastic process, $y(t)$ is also a stationary stochastic process.
    \item The expected value is computed as:
        \[\mathbb{E}\left[ y(t) \right]=\mathbb{E}\left[ \dfrac{1}{2}y(t-1)+e(t)-e(t-1) \right]=\dfrac{1}{2}\mathbb{E}\left[ y(t-1)\right] + 1 - 1\]
        Since $y(t)$ is a stationary stochastic process, its mean is constant:
        \[m_y=\dfrac{1}{2}m_y \rightarrow m_y=0\]

        Alternatively, it can be computed using the theorem: 
        \[\mathbb{E}\left[ y(t) \right]=W(1)\cdot\mathbb{E}\left[ e(t) \right]=0\cdot 1=0\]
    \item Define the unbiased process as: 
        \[\begin{cases}
            \tilde{y}(t)=y(t)-m_y \\
            \tilde{e}(t)=e(t)-m_e
        \end{cases}\]
        In this case, we obtain: 
        \[\tilde{y}(t)+m_y=\dfrac{1}{2}\left(\tilde{y}(t-1)+m_y\right)+\tilde{e}(t)+m_e-\left(\tilde{e}(t-1)+m_e\right)\]
        Simplifying, we have: 
        \[\tilde{y}(t)=\dfrac{1}{2}\tilde{y}(t-1)+\tilde{e}(t)+1-\tilde{e}(t-1)-1\rightarrow\tilde{y}(t)=\dfrac{1}{2}\tilde{y}(t-1)+\tilde{e}(t)-\tilde{e}(t-1)\]

        Starting with the covariance at $\tau=0$: 
        \[\gamma_{\tilde{y}}(0)=\mathbb{E}\left[\tilde{y}{(t)}^2\right]=\mathbb{E}\left[ {\left(\dfrac{1}{2}\tilde{y}(t-1)+\tilde{e}(t)-\tilde{e}(t-1)\right)}^2\right]=\dfrac{1}{4}\gamma_{\tilde{y}}(0) +9-9 - 9\rightarrow \gamma_{\tilde{y}}(0)=12\]
        Next, compute the covariance at $\tau=1$: 
        \[\gamma_{\tilde{y}}(1)=\mathbb{E}\left[ \tilde{y}(t)\tilde{y}(t-1)\right]=\mathbb{E}\left[ \left(\dfrac{1}{2}\tilde{y}(t-1)+\tilde{e}(t)-\tilde{e}(t-1)\right)y(t-1)\right]\rightarrow \gamma_{\tilde{y}}(1)=-3\]
        Then, compute the covariance at $\tau=2$: 
        \[\gamma_{\tilde{y}}(2)=\mathbb{E}\left[ \tilde{y}(t)\tilde{y}(t-2)\right]=\mathbb{E}\left[ \left(\dfrac{1}{2}\tilde{y}(t-1)+\tilde{e}(t)-\tilde{e}(t-1)\right)y(t-2)\right]\rightarrow \gamma_{\tilde{y}}(1)=-\dfrac{3}{2}\]
        For a generic $\tau$: 
        \[\gamma_{\tilde{y}}(\tau)=\dfrac{1}{2}\gamma_{\tilde{y}}(\tau-1) \qquad \left\lvert \tau \right\rvert \geq 2\]
\end{enumerate}
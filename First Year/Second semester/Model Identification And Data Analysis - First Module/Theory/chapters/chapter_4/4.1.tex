\section{Possible representations}

An ARMA process can be expressed equivalently in various representations:
\begin{enumerate}
    \item Time domain representation: 
        \[y(t)=a_1y(t-1)+\cdots+a_m y(t-m)+c_0e(t)+\cdots+c_n e(t-n) \]
    \item Operatorial representation: 
        \[y(t)=\dfrac{C(z)}{A(z)}e(t)\]
    \item Probabilistic representation: 
        \[m_y,\gamma_y(\tau)\]
    \item Frequency domain representation: 
        \[m_y,\Gamma_y(\omega)\]
\end{enumerate}

\subsection{Canonical representation}
\paragraph*{Equivalent form one}
Let's introduce a parameter $\alpha \in \mathbb{R}$ and define a process as:
\[y(t)=W(z)\xi(t)\]
We can express it as:
\[y(t)=W(z)\xi(t)=W(z)\dfrac{\alpha}{\alpha}\xi(t)=\tilde{W}(z)\tilde{\xi}(t)\]
Here, $\tilde{W}(z)=\frac{1}{\alpha}W(z)$, and $\tilde{\xi}\sim WN(0,\alpha^2\lambda^2)$. 

\paragraph*{Equivalent form two}
Consider the process defined as:
\[y(t)=W(z)\xi(t)\]
We can represent it as:
\[y(t)=W(z)\xi(t)=W(z)\dfrac{z^n}{z^n}\xi(t)=\tilde{W}(z)\tilde{\xi}(t)\]
Here, $\tilde{W}(z)=z^n W(z)$, and $\tilde{\xi}=\xi(t-n)$. 

\paragraph*{Equivalent form three}
Let's introduce a complex number $p$ such that $\left\lvert p \right\rvert < 1$ and define a process as:
\[y(t)=W(z)\xi(t)\]
We can rewrite it as:
\[y(t)=W(z)\xi(t)=W(z)\dfrac{z-p}{z-p}\xi(t)=\tilde{W}(z)\xi(t)\]
Here, $\tilde{W}(z)=W(z)\frac{z-p}{z-p}$. 

\paragraph*{Equivalent form four}
Consider the process defined as:
\[y(t)=W(z)\xi(t)\]
Suppose $W(z)=W_1(z)(z-q)$, then:
\[y(t)=W(z)\xi(t)=W_1(z)(z-q)\dfrac{z-\frac{1}{q}}{z-q}\xi(t)=W_(z)\left(z-\dfrac{1}{q}\right)\xi(t)=\tilde{W}(z)\tilde{\xi}(t)\]
Here, $\tilde{\xi}(z)=\dfrac{z-q}{z-\frac{1}{q}}\xi(t)$. 
The fraction $\frac{z-q}{z-\frac{1}{q}}$ is called an all-pass filter.

\begin{theorem}[Spectral factorization]
    Let $y(t)$ be a stationary stochastic process with a rational spectral density $\Gamma_y(\omega)$, there exists a unique pair $\xi(t)$ and $W(z)$ such that $y(t)=W(z)\xi(t)$ if and only if:
\end{theorem}
\begin{enumerate}
    \item \textit{$C(z)$ and $A(z)$ are monic, i.e., $c_0=1$ and $a_0=1$}. 
    \item \textit{$C(z)$ and $A(z)$ have null relative degree (they have the same degree)}.
    \item \textit{$C(z)$ and $A(z)$ are co-prime (they do not have common factors)}.
    \item \textit{The absolute value of poles and zeros of $W(z)$ is less than one}.
\end{enumerate}
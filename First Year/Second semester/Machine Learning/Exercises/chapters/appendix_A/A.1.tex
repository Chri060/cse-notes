\section{Introduction}

Let's examine the Iris dataset, which comprises the following features for each sample:
\begin{enumerate}
    \item Sepal length.
    \item Sepal width.
    \item Petal length.
    \item Petal width.
    \item Species (including Iris setosa, Iris virginica, and Iris versicolor).
\end{enumerate}
This dataset consists of a total of $N = 150$ samples, with each species contributing $50$ samples.

Using linear regression, we can extract valuable insights from the data.
From the data, we can infer relationships between variables and make predictions based on them.
We can provide forecasts for various quantities using newly observed data.
Specifically, we can predict the petal width of a particular type of Iris setosa by leveraging the relationship with petal length through linear regression.
In this scenario, the target variable is continuous ($t_n \in \mathbb{R}$), indicating a regression problem.
\section{Exercise one}

Classify the following variable as quantitative or qualitative: 
\begin{enumerate}
    \item Height.
    \item Age.
    \item Speed.
    \item Color.
\end{enumerate}
Provide a technique for transforming qualitative data into quantitative format without imposing additional organization on the data.

\subsection*{Solution}
The classification is as follows:
\begin{enumerate}
    \item Quantitative variable: heights can be ordered and typically belong to a bounded continuous set.
    \item Quantitative variable: values are ordered natural numbers.
    \item Quantitative variable: takes real number values.
    \item Qualitative variable: since there's no inherent order among colors, one-hot encoding can be employed without imposing additional structure on the data.
\end{enumerate}
For a set of all possible colors, $\mathcal{C} = \{c_1, \dots , c_p\}$ one-hot encoding creates a binary variable  $z_i \in \{0, 1\}$ for each color $c_i$. 
This variable equals one when the color is $c_i$. 
Thus, color $c_i$ is represented by a binary vector $z_i = \begin{bmatrix} z_1 & \cdots & z_n \end{bmatrix}^T$, where $z_i = 1$ and all other $z_j = 0$ for $j \neq i$.
This method introduces $p$ new variables without further structuring the data.
Notably, any two vectors $zi \neq z_j$ are equally distant under reasonable metrics like Euclidean distance.

It's important to note that assigning a quantitative variable $i$ to each color would introduce additional structure to the data, which should generally be avoided.
While this approach requires only one variable instead of $p$, it imposes an ordering among the colors.
Moreover, it results in color $c_i$ being closer to color $c_i+1$ than to color $c_i+2$ in terms of Euclidean distance.
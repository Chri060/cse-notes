\section{Introduction}

Let's examine the Iris dataset, which comprises the following features for each sample:
\begin{enumerate}
    \item Sepal length.
    \item Sepal width.
    \item Petal length.
    \item Petal width.
    \item Species (including Iris setosa, Iris virginica, and Iris versicolor).
\end{enumerate}
This dataset consists of a total of $N = 150$ samples, with each species contributing $50$ samples.

Using petal and sepal length and width variables, it is possible to predict the species of Iris (target).
In this context, the target variables are discrete and non-metric, indicating a classification problem.
The targets are referred to as classes.

To tackle the problem of predicting the species of Iris based on petal and sepal measurements, several approaches can be considered:
\begin{itemize}
    \item \textit{Discriminant function approach}: in this method, the model functions as a mapping of inputs to classes, expressed as $f(x)=C_k\in\left\{C_1,\dots,C_K\right\}$. 
        The process involves fitting the model to the available data.
    \item \textit{Probabilistic discriminative approach}: here, the model represents a conditional probability distribution, $\text{P}(C_k|\textbf{x})\in[0,1]$. 
        The approach entails fitting the model to the data to establish probabilities.
    \item \textit{Probabilistic generative approach}: the model incorporates the likelihood $\text{P}(\textbf{x}|C_k)\in[0,1]$ and the prior  $\text{P}(C_k)\in[0,1]$
        The process involves fitting the model to the available data.
        Inference is made using the posterior probability with the Bayes rule.
        New samples can be generated from the joint distribution $\text{P}(C_k|\textbf{x})=\text{P}(\textbf{x}|C_k)\text{P}(C_k)$.
\end{itemize}
The classification problem can be addressed through various methods, including linear classification methods like the perceptron and logistic regression, alongside algorithms like naïve Bayes and $K$-nearest neighbor.

\subsection{Preliminary operations}
As usual before solving the problem we need to perform some preliminary operations:
\begin{itemize}
    \item Load the data.
    \item Consistency checks.
    \item Select and normalize the input.
    \item Shuffle the data (\texttt{shuffle ()}).
    \item Generate the output ($t_n \in \{0, 1\}$ or $t_n \in \{-1, 1\}$).
    \item Explore the selected data (\texttt{scatter}).
\end{itemize}
\section{Introduction}

To comprehensively analyze the variance and bias of a model, understanding the data generation process is essential. 
Consider the following data generation process:
\[t=f(x)+\varepsilon \qquad f(x)=1+\dfrac{1}{2}x+\dfrac{1}{10}x^2\]
Here are the specifics:
\begin{itemize}
    \item Inputs $x$ are uniformly distributed in the interval $[0, 5]$.
    \item The noise $\varepsilon$ follows a distribution $\text{P}(t|x)$ with $\mathbb{E}[\varepsilon|x] = 0$ and $\text{Var}[\varepsilon|x] = \sigma^2 = 0.7^2$
\end{itemize}

For the learning problem, assuming we don't know the true model, we consider two alternative models:
\begin{itemize}
    \item Linear model $\mathcal{H}_1:y(x)=a+bx$.
    \item Quadratic model $\mathcal{H}_2:y(x)=a+bx+cx^2$.
\end{itemize}
Both models can be interpreted as linear models: $y(x) = \textbf{w}^T\boldsymbol{\phi}(x)$ with the following feature mappings:
\begin{itemize}
    \item For $\mathcal{H}_1:\boldsymbol{\phi}(x)={\left(1,x\right)}^T$ and weights $\textbf{w}={(a,b)}^T$.
    \item For $\mathcal{H}_2:\boldsymbol{\phi}(x)={\left(1,x,x^2\right)}^T$ and weights $\textbf{w}={(a,b,c)}^T$.
\end{itemize}
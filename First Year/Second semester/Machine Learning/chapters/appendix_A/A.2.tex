\section{Matrices}

\paragraph*{Eigenvalues and eigenvectors}
For a square matrix $\mathbb{R}^{n \times n}$,  the corresponding eigenvector equations are given by:
\[A\textbf{v}_i=\lambda_i\textbf{v}_i\]
Here:
\begin{itemize}
    \item The eigenvectors $\textbf{v}_1,\dots,\textbf{v}_n$ represent directions unaffected by the transformation $A$. 
    \item The eigenvalues $\lambda_1,\dots,\lambda_n$ determine the scaling factor for the corresponding eigenvectors $\textbf{v}_i$. 
\end{itemize}
In matrix notation, we can express this relationship as:
\[(A-\lambda I_n)\textbf{v}=0\]
This equation has a non-trivial solution only if the rank of the matrix $A-\lambda I_n$ is full, or equivalently:
\[\left\lvert A-\lambda I_n \right\rvert=0\]
\begin{property}
    The rank of $A$ is equal to the number of non-zero eigenvalues. 
\end{property}   
\begin{property}
    The determinant of $A$ is equal to the product of its eigenvalues: 
    \[\left\lvert A \right\rvert=\prod_{i=1}^{n} \lambda_i\]
\end{property}   

\paragraph*{Trace}
The trace of $A$, denoted as $\text{Tr}(A)$, is equal to the sum of its eigenvalues: 
\[\text{Tr}(A)=\sum_{i=1}^{n}\lambda_i\]

\subsection{Properties}
\begin{definition}[\textit{Positive definite matrix}]
    A matrix $A$ is said to be positive definite if $\textbf{x}^T A \textbf{x}>0$ for all vectors $\textbf{x} \in \mathbb{R}^n\setminus\{0\}$. 
\end{definition}
A positive definite matrix has all positive eigenvalues, i.e., $\lambda_i>0$ for all $i$. 
\begin{definition}[\textit{Semi-positive definite matrix}]
    A matrix $A$  is said to be semi-positive definite if $\textbf{x}^T A \textbf{x}\geq 0$ for all vectors $\textbf{x} \in \mathbb{R}^n\setminus\{0\}$. 
\end{definition}
A semi-positive definite matrix has all non-negative eigenvalues, i.e., $\lambda_i \geq 0$ for all $i$. 
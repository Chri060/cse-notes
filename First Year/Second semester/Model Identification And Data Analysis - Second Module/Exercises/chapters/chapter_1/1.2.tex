\section{Exercise two}

Given the impulse response:
\[\omega(t)=\begin{cases}
    0 \qquad\qquad\qquad \text{if }t\leq 1 \\
    (-2)^{2-t} \:\qquad\quad \text{if }t>1
\end{cases}\]
\begin{enumerate}
    \item Compute the transfer function.
    \item Write a state space representation of the system.
    \item Apply the change of coordinates $\tilde{x}=Tx$, where $T=\begin{bmatrix} 1 & 1 \\ 1 & -1 \end{bmatrix}$.
    \item Compute the transfer function in the new coordinates.
    \item Compute the Hankel matrix $H_3$ and verify that $H_3=O_3R_3$.
    \item Verify that $F=O_3(1:2,:)$ and $F=R_3(2:3,:)$.
\end{enumerate}

\subsection*{Solution}
\begin{enumerate}
    \item In general, the output of a system can be represented in the Z-domain as:
        \[y(t)=\left[\sum_{k=0}^{+\infty}\omega(k)z^{-k}\right]u(t)\]
        In our specific case, we start with the impulse response $\omega(t)$ and compute the transfer function $W(z)$ as follows:
        \begin{align*}
            W(z)    &=\sum_{k=0}^{+\infty}\omega(k)z^{-k} \\
                    &=\sum_{k=0}^{1}\underbrace{\omega(k)z^{-k}}_{0}  + \sum_{k=2}^{1}\underbrace{\omega(k)z^{-k}}_{(-2)^{2-k}} \\
                    &=\sum_{k=2}^{1}(-2)^{2-k}z^{-k} \\
                    &=z^{-2}\sum_{k=2}^{1}(-2)^{-(k-2)}z^{-(k-2)}
        \end{align*}
        We then redefine the index $t=k-2$,  leading to:
        \[W(z)=z^{-2}\sum_{t=0}^{1}(-2)^{-t}z^{-t}=z^{-2}\sum_{t=0}^{1}\left(-\dfrac{1}{2}z^{-1}\right)^{t}\]
        By recalling the geometric series formula (applicable for $\left\lvert a \right\rvert<1$): 
        \[\sum_{k=0}^{+\infty}s^{k}=\dfrac{1}{1-a}\]
        We rewrite our expression as:
        \begin{align*}
            W(z)    &=z^{-2}\sum_{t=0}^{1}\left(-\dfrac{1}{2}z^{-1}\right)^{t} \\
                    &=z^{-2}\dfrac{1}{1+\frac{1}{2}z^{-1}}
        \end{align*}
    \item To obtain the transfer function $W(z)$ in canonical form, we rewrite it as:
        \[W(z)=\dfrac{0z+1}{z^{2}+\frac{1}{2}z+0}\]
        The matrix $F$, which represents the coefficients of the denominator polynomial, is extracted from the lowest terms coefficient of the numerator and denominator:
        \[F=\begin{bmatrix} 0 & 1 \\ 0 & -\frac{1}{2} \end{bmatrix}\]
        The matrix $G$, related to the input polynomial, depends on the highest order monomials of the numerator and denominator:
        \[G=\begin{bmatrix} 0 \\ 1 \end{bmatrix}\]
        The $H$ contains the numerator coefficients read backwards:
        \[H=\begin{bmatrix} 1 & 0 \end{bmatrix}\]
        Finally, since there's no direct feedthrough term, $D$ is equal to $0$. 
    \item The original system undergoes a change of coordinates according to the following equations:
        \[\begin{cases}
            \tilde{x}(t+1)=TFT^{-1} \tilde{x}(t)+TG\tilde{u}(t) \\
            y(t)=HT^{-1}\tilde{x}(t)+Du(t)
        \end{cases}\]
        Substituting the matrices into the above equations, we obtain the transformed system as follows:
        \begin{itemize}
            \item $\tilde{F}=TFT^{-1}=\begin{bmatrix} 1 & 1 \\ 1 & -1 \end{bmatrix}\begin{bmatrix} 0 & 1 \\ 0 & -\frac{1}{2} \end{bmatrix} \begin{bmatrix} 1 & 1 \\ 1 & -1 \end{bmatrix}^{-1}=\dfrac{1}{4}\begin{bmatrix} 1 & -1 \\ 3 & -3 \end{bmatrix}$
            \item $\tilde{G}=TG=\begin{bmatrix} 1 & 1 \\ 1 & -1 \end{bmatrix}\begin{bmatrix} 0 \\ 1 \end{bmatrix}=\begin{bmatrix} 1 \\ -1 \end{bmatrix}$
            \item $\tilde{H}=HT^{-1}=\begin{bmatrix} 1 & 0 \end{bmatrix}\begin{bmatrix} 1 & 1 \\ 1 & -1 \end{bmatrix}^{-1}=\dfrac{1}{2}\begin{bmatrix} 1 & 1 \end{bmatrix}$
            \item $\tilde{D}=0$
        \end{itemize}
    \item The transfer function now takes the form:
        \[\tilde{W}=\tilde{H}(zI-\tilde{F})^{-1}\tilde{G}+\tilde{D}=W(z)\]
        This remains the same because the input-output relation remains unchanged despite the change in coordinates.
    \item The third-order Hankel matrix, $H_3$, is given by:
    
        \[H_3=\begin{bmatrix} \omega(1) & \omega(2) & \omega(3) \\ \omega(2) & \omega(3) & \omega(4) \\ \omega(3) & \omega(4) & \omega(5) \\ \end{bmatrix}\]
        Given the values of $\omega(t)$ from the provided formula:
        \[\begin{cases}
            \omega(0)=0 \\
            \omega(1)=0 \\
            \omega(2)=1 \\
            \omega(3)=-\frac{1}{2} \\
            \omega(4)=\frac{1}{4} \\
            \omega(5)=-\frac{1}{8}
        \end{cases}\]
        The Hankel matrix becomes:
        \[H_3=\begin{bmatrix} 0 & 1 & -\frac{1}{2} \\ 1 & -\frac{1}{2} & \frac{1}{4} \\ -\frac{1}{2} & \frac{1}{4} & -\frac{1}{8} \end{bmatrix}\]
        The extended reachability matrix, $R_3$, and the extended observability matrix, $O_3$, are computed as follows:
        \[R_3=\begin{bmatrix} G & FG & F^2G \end{bmatrix}=\begin{bmatrix} 0 & 1 & -\frac{1}{2} \\ 1 & -\frac{1}{2} & \frac{1}{4} \\ \end{bmatrix}\]
        \[O_3=\begin{bmatrix} H \\ HF \\ HF^2 \end{bmatrix}=\begin{bmatrix} 1 & 0 \\ 0 & 1 \\ 0 & -\frac{1}{2} \end{bmatrix}\]
        To demonstrate their equality, we can multiply these matrices.
    \item Upon visual inspection, we can confirm that the equality holds true between the extended reachability matrix $R_3$ and the extended observability matrix $O_3$, as they both match the form of the Hankel matrix $H_3$.
\end{enumerate}
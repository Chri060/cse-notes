\section{Parametric 4SID algorithm}

The parametric 4SID method comprises the following steps:
\begin{enumerate}
    \item \textit{Data collection}: gather a dataset consisting of input-output pairs:
        \[\left\{ u(1),u(2),\dots,u(N)\right\}\quad\left\{ y(1),y(2),\dots,y(N)\right\}\]
    \item \textit{Model selection}: choose a parametric model, denoted as $\mathcal{M}$, which relates the output $y(t)$ to the input $u(t)$:
        \[\mathcal{M}:y(t)=W(z,\vartheta)\cdot u(t)\]
    \item \textit{Performance index definition}: define a performance index to evaluate the goodness-of-fit of the chosen model. 
        For instance, utilize the sample variance of the output error generated by the model:
        \[J(\vartheta)=\dfrac{1}{N}\sum_{t=1}^N\left(y(t)-W(z,\vartheta)u(t)\right)^2\]
        This index serves to rank the effectiveness of different models. 
        Specifically, if $J(\vartheta_1)<J(\vartheta_2)$, then $\mathcal{M}(\vartheta_1)$ is deemed superior to $\mathcal{M}(\vartheta_2)$. 
    \item \textit{Optimization}: minimize the performance index with respect to the model parameter $\vartheta$:
        \[\hat{\vartheta}_N=\argmin_{\vartheta} J(\vartheta)\]
        Consequently, $\mathcal{M}(\hat{\vartheta}_N)$ represents the optimally estimated model.
\end{enumerate}
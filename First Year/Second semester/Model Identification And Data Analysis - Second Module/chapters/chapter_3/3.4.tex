\section{Asymptotic solution of the Kalman Filter}

In this section, we delve into the asymptotic or steady-state solution of the Kalman Filter.

It's important to note that although the system $\mathcal{S}$ may be linear time-invariant, the Kalman Filter itself is a linear time-variant system due to the presence of the gain $K(t)$. 
This introduces two significant challenges:
\begin{enumerate}
    \item Ensuring and verifying the asymptotic stability of the Kalman Filter becomes notably intricate. 
        Even when all eigenvalues of the matrix $F(t)$ remain strictly inside the unit circle at any given time $t$, stability isn't guaranteed. 
        Consequently, assessing stability, especially for higher-order systems, can be exceptionally complex.
    \item There's a computational challenge. 
        With each sampling instance, we need to update the computations for $K(t)$ and $P(t)$. 
        Notably, solving the Differential Riccati Equation necessitates inverting a $p \times p$ matrix.
\end{enumerate}
These inherent difficulties often lead to the adoption and implementation of the asymptotic version of the Kalman Filter in practical applications.

\subsection{Time invariant Kalman Filter}
In this simplified version, the fundamental concept is that if: $P(t)\rightarrow \bar{P}$, then: 
\[K(t)\rightarrow \bar{K}\]
Thus, we can utilize this constant correction gain $\bar{K}$ to convert the Kalman Filter into a linear time-invariant system.

Before delving into the existence of $\bar{K}$, let's examine the asymptotic stability of the Kalman Filter, assuming the existence of $\bar{K}$.

Consider the core state equation of the Kalman Filter:
\begin{align*}
    \hat{x}(t+1|t)  &=F\hat{x}(t|t-1)+\bar{K}e(t) \\
                    &=F\hat{x}(t|t-1)+\bar{K}\left(y(t)-\hat{y}(t|t-1)\right) \\
                    &=F\hat{x}(t|t-1)+\bar{K}\left(y(t)-H\hat{x}(t|t-1)\right) \\
                    &=\left(F-\bar{K}H\right)\hat{x}(t|t-1)+\bar{K}y(t)
\end{align*}
Here, the state matrix of the Kalman Filter is $F-\bar{K}H$. 
Hence, the Kalman Filter is asymptotically stable if and only if all eigenvalues of $F-\bar{K}H$ lie strictly within the unit circle. 
Particularly:
\begin{itemize}
    \item The stability of the system $\mathcal{S}$ is contingent upon the matrix $F$.
    \item The stability of the Kalman Filter hinges on $F-\bar{K}H$.
\end{itemize}
Thus, the Kalman Filter can achieve asymptotic stability even if the system $\mathcal{S}$ itself is unstable.
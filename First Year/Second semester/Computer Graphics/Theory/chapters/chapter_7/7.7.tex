\section{Diffuse reflection models}

This section will introduce the following diffuse reflection models:
\begin{itemize}
    \item Lambert.
    \item Oren-Nayar.
    \item Toon.
\end{itemize}

\subsection{Lambert reflection}
The simplest form of BRDF consists solely of the diffuse part, which entails a constant term. This BRDF, known as Lambert reflection, is foundational in radiosity techniques.

According to Lambert's law of reflection, each point on an object hit by a ray of light reflects it with a uniform probability distribution in all directions above the surface. 
This reflection is independent of the viewing angle and corresponds to a constant BRDF:
\[f_r(x,\omega_i,\omega_r)=\rho_x\]

However, the amount of light received by an object varies with the angle between the light ray and the reflecting surface (the geometric term $G(x,y)$  of the rendering equation).

Let $n_x$ be the unit normal vector to the surface, $\mathbf{d}=\overrightarrow{lx}$ be the direction of the light ray, and $\alpha$ be the angle between the two vectors.
Lambert demonstrated that the reflected light is proportional to $\cos\alpha$: 
\[R_l=S_l\cos\alpha\]
Here, $S_l$ is the sent light, $R_l$ is the received light. 
Utilizing the geometric properties of the dot product, $\cos\alpha$ can be calculated as the dot product between the unit vectors corresponding to the normal vector $n_x$ and the light direction $\mathbf{d}$.

Let $m_D$ be a vector expressing the material's capability to perform Lambert reflection for each RGB color frequency (i.e., $\rho_x=m_D$): 
\[m_D=(m_R,m_G,m_B)\]
Remarkably, when an object with a reflection factor $m_D$ is illuminated by a perfectly white source $\left(l = (1, 1, 1)\right)$, it will display colors with RGB components equal to $m_D$.
Hence, $m_D$ is commonly treated as a color parameter.

The BRDF of Lambert reflection for scan-line rendering can be expressed as:
\[f_r(x,\overrightarrow{lx},\omega_r)=f_{diffuse}(x,\overrightarrow{lx})=\mathbf{m}_D\max(\overrightarrow{lx}\mathbf{n}_x,0)\]
When a face is opposite to a light source, it cannot be illuminated, as the cosine becomes negative. 
Clamping the value at zero ensures that the faces are not illuminated. 
Since the Lambert reflection model solely comprises the diffuse component of lighting, the vector $m_D$ is often referred to as the diffuse color of the object, representing its primary color.

It's worth noting that the pixel color remains independent of $\omega_r$; when Lambert diffuse reflection is applied, the viewing angle has no effect, and the final image depends solely on the objects' positions and the directions of the lights.

\subsection{Toon shading}
Toon shading simplifies the color output range by using discrete values based on predefined thresholds, resulting in a cartoon-like rendering style. 
This technique can be applied to both the diffuse and specular components of the BRDF.

To start, a standard Lambert BRDF is used for the diffuse component, and either a Phong or Blinn BRDF with $\gamma=1$ is employed for the specular component.
Two colors $(\textbf{m}_{D1} , \textbf{m}_{D2})$ or $(\textbf{m}_{S1} , \textbf{m}_{S2})$ and a threshold ($t_D$ or $t_S$) are utilized to determine which color to choose.
For the diffuse component:
\[f_{diffuse}(x,\overrightarrow{lx})=\begin{cases}
    \mathbf{m}_{D1}\qquad\overrightarrow{lx}\mathbf{n}_x\geq t_d \\
    \mathbf{m}_{D2}\qquad\overrightarrow{lx}\mathbf{n}_x< t_d
\end{cases}\]
For the specular component:
\[\mathbf{r}_{l,x}=2\mathbf{n}_x\left(\overrightarrow{lx}\mathbf{n}_x\right)-\overrightarrow{lx}\]
\[f_{specular}(x,\overrightarrow{lx},\omega_r)=\begin{cases}
    \mathbf{m}_{S1}\qquad\omega_r\mathbf{r}_{l,x}\geq t_s \\
    \mathbf{m}_{S2}\qquad\omega_r\mathbf{r}_{l,x}< t_s
\end{cases}\]
To achieve better visual results, more than two colors are often used for both the specular and diffuse parts. 
Additionally, small gradients are added to smooth transitions between different colors. 
This smoothing is usually implemented using a color that is a function of the cosine of the angles between the considered rays. 
These functions are implemented as 1D textures, which offers performance benefits in modern GPU hardware, as texture look-up is generally more efficient than program branching.

\subsection{Oren-Nayar}
Certain real-world materials like clay, dirt, and certain types of cloth exhibit a unique optical phenomenon known as retro-reflection. 
These materials tend to reflect light back in the direction of the light source due to their rough surfaces, which cannot be accurately simulated using the Lambert diffuse reflection model. 
To address this, the Oren-Nayar diffuse reflection model has been developed to accurately represent such materials.

In most cases, materials exhibiting retro-reflection do not display specular reflections and are characterized solely by a diffuse component. 
The model requires three vectors: the direction of the light $\mathbf{d}$, the surface normal vector $\mathbf{n}$, and the viewing direction $\omega_r$.

These vectors define three angles:
\begin{itemize}
    \item $\theta_i$ between $\mathbf{d}$ and $\mathbf{n}$.
    \item $\theta_r$ between $\omega_r$ and $\mathbf{n}$.
    \item $\gamma$ between the projections of $\omega_r$ and $\mathbf{d}$ on the plane perpendicular to $\mathbf{n}$, denoted as $v_i$ and $v_r$, respectively.
\end{itemize}
The model is characterized by two parameters:
\begin{itemize}
    \item $m_D = (m_R, m_G, m_B)$, representing the main color of the material. 
    \item $\sigma\in\left[0,\frac{\pi}{2}\right]$, which denotes the roughness of the material. 
        Higher values of $\sigma$ correspond to rougher surfaces.
\end{itemize}
The model converges to the Lambert diffusion model when $\sigma = 0$
\begin{align*}
    \theta_i &= \cos^{-1}(d \cdot n_x) \\
    \theta_r &= \cos^{-1}(\omega_r \cdot n_x) \\
    \alpha &= \max(\theta_i, \theta_r) \\
    \beta &= \min(\theta_i, \theta_r) \\
    A &= 1 - 0.5 \cdot \frac{\sigma^2}{\sigma^2 + 0.33} \\
    B &= 0.45 \cdot \frac{\sigma^2}{\sigma^2 + 0.09} \\
    v_i &= \text{normalize}(d - (d \cdot n_x)n_x) \\
    v_r &= \text{normalize}(\omega_r - (\omega_r \cdot n_x)n_x) \\
    G &= \max(0, v_i \cdot v_r) \\
    L &= m_d \cdot \text{clamp}(d \cdot n_x) \\
    \text{diffuse}(x, L_x, \omega_r) &= L \cdot (A + B \cdot G \sin\alpha \tan\beta)
\end{align*}
Parameter $A$ controls how closely the surface adheres to the Lambert principle, while parameter $B$ controls the extent of the retro-reflection effect.
In practice, materials often directly specify parameters $A$ instead of $\sigma$, and $B\sin \alpha\tan \beta$ is pre-computed and interpolated from a texture function $f(\cdot)$, dependent on $d\mathbf{n}_x$ and $\omega_r\mathbf{n}_x$.
\section{Introduction}

As previously described, both in scan-line rendering and ray-casting, the scene is constructed using a finite set of light sources. 
The contributions of all these light sources, denoted as $l$, are combined to determine the final color of each pixel.

Initially, we'll overlook the scenario where objects emit small amounts of light, which allows us to simplify the equation as follows:
\[L(x,\omega_r)=\underbrace{L_e(x,\omega_r)}_{\text{ignored}} +\sum_{l}L(l,\overrightarrow{lx})f_{r^\prime}(x,\overrightarrow{lx},\omega_r) \]
Here, each term in the summation represents the product of two components: the light model, responsible for determining the quantity and direction of the relevant light source, and the Bidirectional Reflectance Distribution Function (BRDF), which governs how the surface reflects incoming light.

The light model defines how light is emitted in various directions within space. 
Given the position of a point $x$ on an object as input, it provides two outputs: a vector representing the direction of the light, and a color value representing the intensity of light received by point $x$ across different wavelengths.

The light direction is typically denoted by a vector $\overrightarrow{lx}$, with a convention that the sign of this vector points towards the light source.
Additionally, it's important to note that the direction of the light vector is normalized, ensuring it remains a unit vector.

\subsection{Light color}
A vector $L(l,\overrightarrow{lx})$ with RGB components specifies the intensity of light for each wavelength, thereby defining its color.
These components are not restricted to the 0 to 1 range; larger values can represent stronger light sources. 
However, it's imperative that the components remain non-negative.

Given that the light color $L(l,\overrightarrow{lx})$ is encoded in a vector, the Bidirectional Reflectance Distribution Function $f_{r}(x,\overrightarrow{lx},\omega_r)$ also returns a color vector.

We will explore three fundamental direct light models commonly used in real-time graphics: direct point, and spot lights. 
\section{Direct light models}

Directional lights are commonly employed to simulate distant sources such as sunlight. 
These sources are positioned far away from objects, thereby uniformly influencing the entire scene.

Because of their distant location, rays from directional lights are parallel across all positions in space, maintaining a constant color and intensity.
The direction of such lights is represented by a constant vector $\textbf{d}=(d_{x},d_{y},d_{z})$ , independent of the position $x$ on the object.
Similarly, the light color is specified by another constant vector $\textbf{l}=(l_{R},l_{G},l_{B})$. 

For each point on an object, the direction of the light and its color are expressed using these constant values $\mathbb{l}$ and $\mathbb{d}$: 
\begin{itemize}
    \item Light intensity: $L(l,\overrightarrow{lx})=\textbf{l}_l$. 
    \item Light direction: $\overrightarrow{lx}=\textbf{d}_l$.
\end{itemize}
In the case of a single directional light, the rendering equation simplifies to:
\[L(x,\omega_r)=\textbf{l}\times f_r(x,\textbf{d},\omega_r)\]

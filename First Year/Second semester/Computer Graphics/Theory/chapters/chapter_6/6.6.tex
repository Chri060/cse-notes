\section{Montecarlo techniques}

Photorealistic results in rendering can only be achieved by approximating the solution of the complete rendering equation. 
Due to its complexity, Monte Carlo techniques are commonly employed: the integral is computed by averaging several random points and directions chosen from the equations. 
Many alternative approaches are possible, and each advanced rendering engine typically exploits one of them.

Numerous techniques extend ray tracing: instead of sending a single ray in the mirror direction, a sampling of the most probable output directions is considered (importance sampling). 
A ray is then traced for each selected direction, and the radiance is computed using the Bidirectional Reflectance Distribution Function (BRDF) of the considered material.

Photon mapping, on the other hand, emulates the movements of photons in the scene, taking into account bounces, focalizations, and other advanced phenomena such as caustics.

Due to the inherent randomness in these techniques, Monte Carlo-based rendering algorithms tend to produce noisy images. 
This noise can only be reduced by increasing the number of rays (and consequently, the rendering time).
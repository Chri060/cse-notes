\section{Radiosity}

Radiosity offers a distinct simplification of the rendering equation, focusing on materials with constant Bidirectional Reflectance Distribution Function (BRDF). 
This simplification reduces the unknowns in the rendering equation to one variable per point on the objects, as reflection is independent of the direction from which it is observed. 
This variable is termed the radiosity of the object, which corresponds to the output counterparts of irradiance:
\[L(x)=L_e(x)+\rho_x\int L_(y)G(x,y)V(x,y)\,dy\]
In this equation, the surfaces of the objects are divided into patches, each with its own unknown radiosity value.
Light sources are implemented as patches that emit radiosity ($L_e(\cdot)$ term). 
The rendering equation transforms into a system of linear equations, which can be solved iteratively:
\[L(x_i)=L_e(x_i)+\rho_{x_i}\sum_{y_j}L(y_j)G(x_i,y_j)V(x_i,y_j)=L_e+\sum_{y_j}L(y_j)R(x_i,y_j)\]
In matrix notation, the vector $L$ contains one element (per color frequency) per patch, representing its radiosity.
Matrix $R$ includes visibility, constant BRDF, and geometric relations between any two patches in the scene. 
The solutions of these equations are then interpolated to generate a view of the scene.

\subsection{Algorithm}
The pseudo-code of a radiosity rendering algorithm is the following:
\begin{algorithm}[H]
    \caption{Radiosity rendering algorithm}
        \begin{algorithmic}[1]
            \State{Discretize the scene, and compute matrix $R$}
            \State{Compute the solution of equation $L = L_e + R \cdot L$}
            \State{Render the scene using either scan-line or ray tracing}
            \State{Interpolate $L$ to obtain the color for each pixel}
        \end{algorithmic}
\end{algorithm}
Indeed, the most time-consuming steps are the first two.
However, once they have been computed, the scene can be rendered very quickly from any point of view. 
In most cases, the solution of the equation can be computed with a fixed-point iteration, starting from $L = 0$, and refining its value at every iteration.
\begin{algorithm}[H]
    \caption{Rendering equation algorithm}
        \begin{algorithmic}[1]
            \State{$L_{old} = L_{new} = 0$}
            \Repeat{}
                \State{$L_{old} = L_{new}$}
                \State{$L_{new} = L_e + R \cdot L_{old}$}
            \Until{$\left\lvert L_{new} - L_{old}\right\rvert  > \text{threshold}$}
        \end{algorithmic}
\end{algorithm}
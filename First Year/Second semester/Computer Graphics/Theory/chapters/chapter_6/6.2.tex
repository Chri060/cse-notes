\section{Rendering equation}

Light emitted by sources is characterized by its radiance. 
The material comprising an object's surface determines the intensity of light reflected in a particular direction.

In practice, the surface's microscopic structure determines the direction of light bounces, which can vary significantly between materials.
The surface properties of an object can be described by a function known as the bidirectional reflectance distribution function (BRDF).

\subsection{Bidirectional reflectance distribution function}
The BRDF function takes as inputs the incoming ($\omega_i$) and outgoing ($\omega_r$) light directions, both represented as unit vectors.
It quantifies the amount of irradiance from the incoming angle that is reflected towards the outgoing angle and is measured in steradians per unit solid angle ($sr^{-1}$):
\[f_r(\theta_i,\phi_i,\theta_r,\phi_r)=f_r(\omega_i,\omega_r)\]
This allows for the consideration of various reflection angles for incoming radiance, depending on the surface material.

A good BRDF should satisfy two main properties: reciprocity and energy conservation. 

\paragraph*{Reciprocity}
Reciprocity implies that swapping the incoming and outgoing directions doesn't change the function's value (hence the term bidirectional):
\[f_r(\omega_i,\omega_r)=f_r(\omega_r,\omega_i)\]

\paragraph*{Energy conservation}
Energy conservation dictates that the BRDF cannot increase the total irradiance leaving a point on a surface:
\[\int f_r(\omega_i,\omega_r)\cos\theta_r\, d\omega_r \leq 1\]
The integral may be less than one if the point absorbs energy.

\paragraph*{BRDF functions}
Various approximations to common BRDF functions have been proposed in the literature.
Some of these approximations provide satisfactory results during rendering, even if they don't fully adhere to the aforementioned properties.

\subsection{The equation}
The BRDF facilitates the connection between the irradiance in all directions for every point on the objects within a scene. 
This connection is formalized through the rendering equation:
\[L(x,\omega_r)=L_e(x,\omega_r)+\int L(y,\overrightarrow{yx}) f_r(x,\overrightarrow{yx},\omega_r)G(x,y)V(x,y)\, dy\]
In this equation, the following components are present:
\begin{itemize}
    \item $L(x,\omega_r)$ determines the radiance of each point $x$ of any object in the 3D world, for any output direction $\omega$.
    \item $L_e(x,\omega_r)$ accounts for the light that the object at $x$ emits in direction $\omega$. 
        This parameter mainly characterizes light sources (e.g., bulbs or neon) and serves as the initial injection of photons into the scene.
    \item The integral accounts for the light that hits the considered point $x$ from all the points $y$ of the surfaces (including the light sources) of all the objects and lights in the scene. 
        It also includes other points of the same object to allow for the computation of effects such as self-shadowing or self-reflection. 
        It is composed of the following terms:
        \begin{itemize}
            \item $L(y,\overrightarrow{yx})$: for each object, the equation considers the radiance emitted toward point $x$. 
                Here $\overrightarrow{yx}$ represents the direction of the line that connects point $y$ to point $x$.
            \item $f_r(x,\overrightarrow{yx},\omega_r)$ is the BRDF of the material of the object at point $x$. 
                Since the materials of objects might change over the surface, the BRDF depends also on the position of the considered point $x$. 
                The input angle corresponds to the direction of the incoming light, oriented along the segment that connects $y$ to $x$. 
                The output angle corresponds to the direction from which the output radiance is being computed, represented by $\omega_r$ on the left-hand side of the equation.
            \item $G(x,y)$ encodes the geometric relation between points $x$ and $y$. 
                This is necessary because the angle between the surfaces has an impact on both the light emitted and reflected. 
                It considers both the relative orientation and the distance of the two points and is defined as follows:
                \[G(x,y)=\dfrac{\cos\theta_x\cos\theta_y}{r_{xy}^2}\]
                The two cosine terms account for the angle relative to the respective normal vectors, and $r_{xy}^2$ represents the squared distance of the two points.
            \item $V(x,y)$: finally, term $V(x,y)$ considers the visibility between two points $x$ and $y$: $V(x,y) = 1$ if the two points can see each other, and $V(x,y) = 0$ if point $y$ is hidden by some other objects in between. 
                Term $V(x,y)$ allows for the computation of shadows and ensures that in each input direction, at most a single object is considered.
        \end{itemize}
\end{itemize}
The term $L(x, \omega)$ represents the unknown quantity in the equation. 
As it appears on both sides of the expression, the rendering equation is mathematically classified as an integral equation of the second kind. 
This classification is denoted using mathematical notation as follows:
\[\varphi(x)=f(x)+\lambda\int_a^b K(x,t)\varphi(t)\, dt\]

\paragraph*{Colors}
The rendering equation is applied for every wavelength $\lambda$ of light, typically involving repetition for the three different RGB channels. 
Geometric and visibility terms remain constant across wavelengths and are unique for each pair of points.

Light sources are associated with RGB values representing the photons emitted for each of the three main frequencies. 
Objects are characterized by a BRDF with distinct parameters for each of the three primary colors. 
Three images are generated independently, each considering either the red, green, or blue components individually. 
Subsequently, the three colors are combined to produce the final output image.

Due to the separation of color components, the interaction between light and material colors results in outcomes that are not immediately intuitive.

\paragraph*{Extension}
The proposed rendering equation accurately computes various effects such as reflections, shadows, matte, and glossy objects. 
However, it falls short in simulating other effects like participating media (e.g., rendering of gases and fumes) or transparent objects like glass or water. 
To account for these materials, extensions to the BRDF and rendering equations have been made.

One extension involves considering transmitted light, i.e., transparency, which is achieved by defining the bidirectional transmittance distribution function (BTDF). 
The BTDF has a similar definition to the BRDF but is used in the opposite direction:
\[f_t(\theta_i,\phi_i,\theta_r,\phi_r)=f_t(\omega_i,\omega_r)\]
Since the angles for the BRDF and BTDF typically do not overlap, they are often combined into a single function called the bidirectional scattering distribution function (BSDF). 
The rendering equations need to be updated to incorporate both the BRDF and the BTDF.\@ 
Here, $x^\prime$ denotes the point on the other side of the surface (assuming it is unique) from which light is directed toward point $x$.
\begin{multline*}
    L(x,\omega_r)=L_e(x,\omega_r)+\int L(y,\overrightarrow{yx}) f_r(x,\overrightarrow{yx},\omega_r)G(x,y)V(x,y) + \\ + L(y,\overrightarrow{yx}) f_t(x^\prime,\overrightarrow{yx},\omega_r)G(x^\prime,y)V(x^\prime,y) \, dy
\end{multline*}
In certain materials, light can bounce inside the object and exit from another point. 
This phenomenon is known as subsurface scattering, observed in materials like human skin and marble.
For such cases, a more complex function called the bidirectional surface reflectance distribution function (BSSRDF) is utilized. 
The BSSRDF incorporates an extra parameter $x^\prime$ to account for the point on the surface from which light enters at angle $\omega_i$. 
The rendering equation now integrates over all points of an object to consider the quantity of light that can enter from a given position and direction and exit from a specific position and direction:
\[L(x,\omega_r)=L_e(x,\omega_r)+\iint L(y,\overrightarrow{yx}) f_{ss}(x,\overrightarrow{yx},x^\prime,\omega_r)G(x^\prime,y)V(x^\prime,y) \, dx^\prime dy\]

\paragraph*{Solution}
Solving the rendering equations is inherently challenging and often necessitates complex discretization techniques. 
In the subsequent sections, we will explore straightforward approximations to the rendering equation that yield satisfactory results with manageable complexity. 
Some of these techniques are supported in Vulkan through its specific features.

\subsection{Lights basics}
The simplest approximations to the rendering equation categorize light sources into direct and indirect sources.

Direct sources emanate from specific positions and directions, such as lamps, studio spotlights, or the sun in outdoor scenes.

Indirect sources encompass all other forms of illumination, primarily caused by light bounces and reflections among surfaces. 
Photographers often utilize these sources to introduce soft lighting in their images.

Images illuminated solely by direct sources may appear overly dark and lack realism.
Points not struck by any light source would appear pitch black. 
While this setup is unrealistic, it is often the simplest assumption made to simplify real-time rendering equation solutions in computer graphics.

Projected shadows are formed by the obstruction of direct light sources and can be approximated using specific techniques.

Indirect lighting enhances realism by accounting for light that bounces off other surfaces. 
However, achieving accurate computations for indirect lighting demands considerable effort and is challenging to implement in real-time scenarios.
Nonetheless, various approximations exist to recreate indirect lighting effects.

In many cases, the light contribution for a single point and direction is computed offline and stored in an image-based structure. 
This precomputed data is later used to perform approximations for indirect lighting during real-time rendering.
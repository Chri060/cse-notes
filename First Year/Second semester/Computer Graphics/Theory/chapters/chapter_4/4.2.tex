\section{Mesh encoding}

Mesh encoding relies on sets of vertices, which serve as the building blocks for defining the geometry. 
The rendering engine utilizes these vertices to establish the endpoints of the triangles composing the mesh. 
Typically, vertex coordinates are stored as Cartesian coordinates to optimize memory usage. 
Before applying transformations using the world view projection matrix, the rendering engine adds a fourth component set to one to ensure homogeneous coordinates.

It's evident that many triangles within a mesh share a significant number of vertices. 
Leveraging this characteristic is crucial for minimizing the memory footprint required to encode an object.

Various types of mesh encoding have been proposed in the literature, but only two are standard in Vulkan: 
\begin{itemize}
    \item \textit{Triangle lists}: each triangle is encoded as a trio of distinct coordinates, without reusing any vertices. 
        They are suitable for encoding disconnected triangles. 
        Encoding $N$ triangles necessitates $3N$ vertices.
    \item \textit{Triangle strips}: A series of contiguous triangles forming a band-like surface. 
        The encoding starts with the first two vertices, with each subsequent vertex connected to the preceding two. 
        To encode $N$ triangles, $N+2$ vertices are required.
\end{itemize}

\paragraph*{Properties}
In some cases, triangle strips may not be applicable even when the topology appears suitable for this type of encoding. 
This limitation arises because a vertex is typically defined by more parameters than just its local coordinates. 
For triangle strip encoding to be viable, shared vertices must be identical across all parameters.
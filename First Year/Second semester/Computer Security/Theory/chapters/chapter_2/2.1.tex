\section{Introduction}

\begin{definition}[\textit{Cryptography}]
    Cryptography is the study of techniques to allow secure communication and data storage in presence of attackers.     
\end{definition}
The features provided by cryptography are: 
\begin{itemize}
    \item \textit{Confidentiality}: data can be accessed only by chosen entities. 
    \item \textit{Integrity/freshness}: detect or prevent tampering or replays. 
    \item \textit{Authenticity}: data and their origin are certified. 
    \item \textit{Non-repudiation}: data creator cannot repudiate created data. 
    \item \textit{Advanced features}: proofs of knowledge or computation. 
\end{itemize}

\subsection{History}
Cryptography is as old as written communication. 
It was born for commercial or military uses. 
The first cryptographic algorithms were computed by hand with pen and paper. 

The original approach consisted in a battle of wits between cryptographers (ideate a method to obfuscate a test) and cryptanalyst (break the cipher). 

A turning point came when Bellaso (1553) was the first to have the idea to separate encryption method from the key. 

In 1883 Kerchoff found six principles to obtain a good cipher: 
\begin{enumerate}
    \item It must be practically, if not mathematically, unbreakable. 
    \item It should be possible to make it public, even to the enemy. 
    \item The key must be communicable without written notes and changeable whenever the correspondents want. 
    \item It must be applicable to telegraphic communication. 
    \item It must be portable, and should be operable by a single person. 
    \item Finally, given the operating environment, it should be easy to use, it shouldn't impose excessive mental load, nor require a large set of rules to be known. 
\end{enumerate}

In 1917 the first mechanical computation (given by the rotor machine by Hebern) changed the cryptography. 
This rotor machine were commercialized to people at the beginning of 1920. 
At the start of World War II the Germans upgraded this rotor machine with a new version called Enigma. 
Enigma's workflow where then decrypted by cryptanalyst at Bletchey park, that led Eisenhower to the win of the war. 

After World War II, in 1949 Shannon proved that a mathematically secure ciphers exists. 

Later, in 1955, Nash argued that computationally secure ciphers are also ok. 
Consider a cipher with a finite, $\lambda$ bit long, key. 
The conjecture is that if parts of the key interact complexly in the determination of their effects on the ciphertext, the attacker effort to break the cipher would be $\mathcal{O}(2^{\lambda})$. 
This means that the owner of the keys takes $\mathcal{O}(\lambda^2)$ to compute the cipher. 
This means that the computational gap is unsurmountable for large $\lambda$. 


\begin{chronology}[25]{1875}{1975}{0.9\textwidth}
    \event{1883}{Kerchoff principles}
    \event{1917}{First rotor machine}
    \event{1949}{Unbreakable ciphers proof}
    \event{1955}{Computational security proof}
\end{chronology}









\section{Introduction}

\begin{definition}[\textit{Cryptography}]
    Cryptography refers to the field of study concerned with developing techniques that enable secure communication and data storage in the presence of potential adversaries.
\end{definition}
Cryptography offers several essential features, including:
\begin{itemize}
    \item \textit{Confidentiality}: ensures that data can only be accessed by authorized entities.
    \item \textit{Integrity/freshness}: detects or prevents tampering or unauthorized replays of data.
    \item \textit{Authenticity}: certifies the origin of data and verifies its authenticity.
    \item \textit{Non-repudiation}: ensures that the creator of data cannot deny their responsibility for creating it.
    \item \textit{Advanced features}: includes capabilities such as proofs of knowledge or computation.
\end{itemize}

\subsection{History}
Cryptography has a history as ancient as written communication itself, originating primarily for commercial and military purposes. 
Initially, cryptographic algorithms were devised and executed manually, using pen and paper.

The early approach to cryptography involved a contest of intellect between cryptographers, who devised methods to obscure messages, and cryptanalysts, who sought to break these ciphers.

A significant development occurred in 1553 when Bellaso pioneered the idea of separating the encryption method from the key.

In 1883, Kerchoff formulated six principles for designing robust ciphers:
\begin{enumerate}
    \item The cipher should be practically, if not mathematically, unbreakable.
    \item It should be possible to disclose the cipher to the public, including enemies.
    \item The key must be communicable without written notes and changeable at the discretion of correspondents.
    \item It should be suitable for telegraphic communication.
    \item The cipher should be portable and operable by a single person.
    \item Considering the operational context, it should be user-friendly, imposing minimal mental burden and requiring a limited set of rules.
\end{enumerate}

The landscape of cryptography underwent a significant transformation in 1917 with the introduction of mechanical computation, exemplified by Hebern's rotor machine, which became commercially available in the 1920s.
This technology evolved into the German Enigma machine during World War II, whose encryption methods were eventually deciphered by cryptanalyst at Bletchley Park, contributing significantly to the Allied victory.

After World War II, in 1949 Shannon proved that a mathematically secure ciphers exists. 

Following World War II, in 1949, Shannon demonstrated the existence of mathematically secure ciphers. 
Subsequently, in 1955, Nash proposed the concept of computationally secure ciphers, suggesting that if the interaction of key components in a cipher's determination of ciphertext is sufficiently complex, the effort required for an attacker to break the cipher would grow exponentially with the length of the key ($\mathcal{O}(2^{\lambda})$), surpassing the computational capabilities of the key owner ($\mathcal{O}(\lambda^2)$) for sufficiently large key lengths ($\lambda$).

\begin{chronology}[25]{1875}{1975}{0.9\textwidth}
    \event{1883}{Kerchoff principles}
    \event{1917}{First rotor machine}
    \event{1949}{Unbreakable ciphers proof}
    \event{1955}{Computational security proof}
\end{chronology}

\subsection{Definitions}
\begin{definition}[\textit{Plaintext space}]
    A plaintext space $P$ is the set of possible messages $ptx \in P$. 
\end{definition}
\begin{definition}[\textit{Ciphertext space}]
    A ciphertext space $C$ is the set of possible ciphertext $ctx \in C$. 
\end{definition}
It's worth noting that the ciphertext space $C$ may have a larger cardinality than the plaintext space $P$.
\begin{definition}[\textit{Key space}]
    A key space $K$ is the set of possible keys. 
\end{definition}
The length of the key often correlates with the desired level of security.
\begin{definition}[\textit{Encryption function}]
    An encryption function $\mathbb{E}$ is a mapping that takes an element from the plaintext space $P$ and a key from the key space $K$, and produces an element from the ciphertext space $C$:
    \[\mathbb{E}:P \times K \rightarrow C\]
\end{definition}
\begin{definition}[\textit{Decryption function}]
    A decryption function $\mathbb{D}$ is a mapping that takes an element from the ciphertext space $C$ and a key from the key space $K$, and yields an element from the plaintext space $P$:
    \[\mathbb{D}:C \times K \rightarrow P\]
\end{definition}
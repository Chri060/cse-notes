\section{Computational security}

The objective of ensuring confidentiality is to prevent unauthorized individuals from comprehending the data. 
Various methods can compromise confidentiality:
\begin{itemize}
    \item Passive interception by an attacker.
    \item Knowledge of a set of potential plaintexts by the attacker.
    \item Data manipulation by the attacker to observe the reactions of an entity capable of decryption.
\end{itemize}

\begin{definition}[\textit{Perfect cipher}]
    In a perfect cipher, for any plaintext $ptx$ in the plaintext space $P$ and any corresponding ciphertext $ctx$ in the ciphertext space $C$, the probability of the plaintext being sent is equal to the conditional probability of that plaintext given the observed ciphertext:
    \[\text{P}(ptx\:sent=ptx)=\text{P}(ptx\:sent=ptx|ctx\:sent=ctx)\]
\end{definition}
In other words, observing a ciphertext $c \in C$ provides no information about the corresponding plaintext it represents.
\begin{theorem}[Shannon 1949]
    Any symmetric cipher $\left\langle P,K,C,\mathbb{E},\mathbb{D} \right\rangle $ with $\left\lvert P \right\rvert=\left\lvert K \right\rvert=\left\lvert C \right\rvert$, achieves perfect security if and only if every key is utilized with equal probability $\dfrac{1}{\left\lvert K\right\rvert }$, and each plaintext is uniquely mapped to a ciphertext by a unique key:
    \[\forall \left(ptx,ctx\right) \in P \times C, \exists !k \in K \text{ such that } \mathbb{E}\left(ptx,k\right)=ctx\]
\end{theorem}
\begin{example}
    Let's consider $P$, $K$, and $C$ as sets of binary strings. 
    The encryption function selects a uniformly random, fresh key $k$ from $K$ each time it's invoked and computes the ciphertext as $ctx=ptx \oplus k$.

    Gilbert Vernam patented a telegraphic machine in 1919 that implemented $ctx=ptx \oplus k$ using the Baudot code. 
    Joseph Mauborgne proposed utilizing a random tape containing the key $k$.

    Combining Vernam's encryption machine with Mauborgne's approach results in a perfect cipher implementation.
\end{example}

It's crucial to understand that while a cipher may achieve perfect security, this doesn't necessarily mean it's practical or user-friendly.
Managing key material and regularly changing keys can be exceptionally challenging.

In practice, perfect ciphers often face vulnerabilities due to issues such as key theft or reuse. 
Additionally, the generation of truly random keys has historically been problematic, leading to potential vulnerabilities and breaches.
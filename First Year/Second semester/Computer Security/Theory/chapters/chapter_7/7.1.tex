\section{Introduction}

Malware, a portmanteau of malicious software, refers to code that is intentionally designed to violate security policies. 
Malware can be categorized into several types, each with distinct characteristics:
\begin{itemize}
    \item \textit{Viruses}: these are pieces of code that self-propagate by infecting other files, typically executables, but also documents with macros and boot loaders. 
        They are not standalone programs and require a host file to spread.
    \item \textit{Worms}: these are standalone programs that self-propagate, often remotely. 
        They spread by exploiting vulnerabilities in hosts or through social engineering tactics, such as email worms.
    \item \textit{Trojan horses}: these are programs that appear benign but conceal malicious functionality. 
        They often enable remote control by an attacker, allowing unauthorized access to the infected system.
\end{itemize}

\subsection{History}
In detail, the main steps are: 
\begin{itemize}
    \item 1971: Creeper is the first self-replicating program on PDP-10.
    \item 1981: First outbreak of Elk Cloner on Apple II floppy disks.
    \item 1983: The first documented experimental virus, as part of Fred Cohen's pioneering work. The term "virus" was coined by Len Adleman.
    \item 1987: Christmas worm (mass mailer) hits IBM Mainframes, causing 500,000 replications per hour and paralyzing many networks.
    \item 1988: Internet worm (November 2, 1988) created by Robert Morris Jr., leading to the birth of CERT.
    \item 1995: Concept virus, the first macro virus, appears.
    \item 1998: Back Orifice trojan, demonstrating the lack of security in Microsoft systems, is released for the IRC masses.
    \item 1999: Melissa virus, a large-scale email macro-virus, spreads widely.
    \item 1999: First DDoS attacks via trojaned machines (zombies) occur.
    \item 1999: Kernel Rootkits become public with tools like Knark, which modify the system call table.
    \item 2000: ILOVEYOU worm spreads widely through email, employing social engineering techniques.
    \item 2001: Code Red worm, a large-scale exploit-based worm, emerges.
    \item 2003: SQL Slammer worm propagates extremely quickly through UDP.
    \item 2004: Malware creating botnet infrastructures begins to appear, with examples like Storm Worm, Torpig, Koobface, Conficker, and Stuxnet.
    \item 2010: Scareware, ransomware, and state-sponsored malware become more prevalent.
\end{itemize}
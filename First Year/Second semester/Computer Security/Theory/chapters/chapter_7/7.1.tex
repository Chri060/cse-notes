\section{Introduction}

Malware, a portmanteau of malicious software, refers to code that is intentionally designed to violate security policies. 
Malware can be categorized into several types, each with distinct characteristics:
\begin{itemize}
    \item \textit{Viruses}: these are pieces of code that self-propagate by infecting other files, typically executables, but also documents with macros and boot loaders. 
        They are not standalone programs and require a host file to spread.
    \item \textit{Worms}: these are standalone programs that self-propagate, often remotely. 
        They spread by exploiting vulnerabilities in hosts or through social engineering tactics, such as email worms.
    \item \textit{Trojan horses}: these are programs that appear benign but conceal malicious functionality. 
        They often enable remote control by an attacker, allowing unauthorized access to the infected system.
\end{itemize}

\subsection{History}
In detail, the main steps are: 
\begin{itemize}
    \item 1971: Creeper is the first self-replicating program on PDP-10.
    \item 1981: First outbreak of Elk Cloner on Apple II floppy disks.
    \item 1983: The first documented experimental virus, as part of Fred Cohen's pioneering work. The term "virus" was coined by Len Adleman.
    \item 1987: Christmas worm (mass mailer) hits IBM Mainframes, causing 500,000 replications per hour and paralyzing many networks.
    \item 1988: Internet worm (November 2, 1988) created by Robert Morris Jr., leading to the birth of CERT.
    \item 1995: Concept virus, the first macro virus, appears.
    \item 1998: Back Orifice trojan, demonstrating the lack of security in Microsoft systems, is released for the IRC masses.
    \item 1999: Melissa virus, a large-scale email macro-virus, spreads widely.
    \item 1999: First DDoS attacks via trojaned machines (zombies) occur.
    \item 1999: Kernel Rootkits become public with tools like Knark, which modify the system call table.
    \item 2000: ILOVEYOU worm spreads widely through email, employing social engineering techniques.
    \item 2001: Code Red worm, a large-scale exploit-based worm, emerges.
    \item 2003: SQL Slammer worm propagates extremely quickly through UDP.
    \item 2004: Malware creating botnet infrastructures begins to appear, with examples like Storm Worm, Torpig, Koobface, Conficker, and Stuxnet.
    \item 2010: Scareware, ransomware, and state-sponsored malware become more prevalent.
\end{itemize}

\subsection{Theory of computer viruses}
In 1983, Fred Cohen theorized the existence of computer viruses and produced the first examples. 
From a theoretical computer science perspective, these viruses represent an intriguing concept of self-modifying and self-propagating code. 
However, the security challenges posed by viruses quickly became apparent.
One significant challenge is the impossibility of creating a perfect virus detector.

Let $P$ be a perfect detection program.
We can construct a virus $V$ that includes $P$ as a subroutine:
\begin{itemize}
    \item If $P(V)=\text{True}$, $V$ halts and does not spread, thus $V$ is not a virus.
    \item If $P(V)=\text{False}$, $V$ spreads, thus $V$ is a virus.
\end{itemize}

\subsection{Infection techniques}
Boot viruses target the Master Boot Record (MBR) of the hard disk, which is the first sector on the disk, or the boot sector of partitions. 
Early examples include the Brain virus, with more recent instances like Mebroot/Torpig.
Although considered somewhat outdated, interest in boot viruses is resurging, particularly with the advent of diskless workstations and virtual machines, such as SubVirt.

File infectors, on the other hand, come in several forms. 
Simple overwrite viruses damage the original program, while parasitic viruses append code and modify the program's entry point. 
Multi-cavity viruses inject code into unused regions of the program code.

\subsection{Attackers' motivations}
Modern attackers are primarily focused on monetizing their malware.
Direct monetization methods include the abuse of credit cards and connecting to premium numbers. 
Indirect monetization involves information gathering, abuse of computing resources, and renting or selling botnet infrastructures. 
This has led to the development of a growing underground (black) economy. 
Within this cybercrime ecosystem, organized groups engage in various activities such as exploit development and procurement, site infection, victim monitoring, and selling exploit kits. 
They also provide support to their clients.

\subsection{Antivirus and anti-malware}
The basic strategy for antivirus and anti-malware protection is signature-based detection. 
This method utilizes a database of byte-level or instruction-level signatures that match known malware, often employing wildcards and regular expressions. 
Heuristics are used to check for signs of infection, such as code execution starting in the last section, incorrect header size in the PE header, suspicious code section names, and patched import address tables. 
Behavioral detection aims to identify the signs or behaviors of known malware and detect common behaviors associated with malware.
\section{Worms}

In November 1988, a program written by Robert Morris Jr., a Ph.D. student at Cornell, brought down the ARPANET.
This program, later known as the Morris Worm, was capable of connecting to other computers and exploiting several vulnerabilities, such as buffer overflow in the service and password cracking, to replicate itself onto a second computer. 
Once the copy was established, it would begin running and repeat the process, causing an infinite loop of self-replication across the network. 
This unexpected behavior resulted in widespread disruption.

\subsection{Mass mailers}
The introduction of email software that allowed attached files, including executables, led to the emergence of mass mailer worms. 
These worms could spread by emailing themselves to others, often by exploiting the address book to appear more trustworthy. 
Modern variations of mass mailers use social networks to spread, such as suspicious-looking messages on Twitter or Facebook from friends.

\subsection{Mass scanners}
Modern worms often use mass scanning techniques to spread rapidly. 
The basic pattern involves infecting a computer and seeking out new targets, with the potential to spread within minutes and infect hundreds of thousands of hosts. 
Scanning methods include selecting random addresses, favoring local networks, permutation scanning (dividing up the IP address space), hit list scanning, and combining techniques, as seen with the Warhol worm.

\subsection{Worm activity and the internet}
Despite initial fears that the Internet would be plagued by increasingly sophisticated worms, major worm outbreaks have been rare since 2004. 
Although vulnerabilities existed and there were times when the community braced for significant impacts, no worm has specifically targeted the Internet infrastructure. 
However, attackers have typically avoided targeting the infrastructure directly, likely due to their need for the infrastructure to remain operational for their own purposes.
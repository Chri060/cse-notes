\section{Stealth techniques}

\subsection{Virus}
Virus scanners typically detect viruses by searching around the entry point of programs. 
To evade detection, viruses employ several stealth techniques. 
One such technique is Entry Point Obfuscation, where viruses, such as multi-cavity viruses, hijack control later in the execution process after the program is launched. 
This can involve overwriting import table addresses (like libraries) or function call instructions.

Polymorphism is another technique where a virus changes its layout with each infection. 
The same payload is encrypted using a different key for each infection, making signature analysis practically impossible, although antivirus software might still detect the encryption routine. 
Metamorphism goes further by creating different versions of the code that appear different but perform the same functions, complicating detection even more.

\subsection{Malware}
Malware often includes a dormant period during which it exhibits no malicious behavior. 
Its payload can be event-triggered, frequently through a command and control (C\&C) channel. 
Anti-virtualization techniques are used to evade analysis, as malware often detects if it is running in a virtual environment, which suggests it might be under scrutiny by security labs, antivirus sandboxes, or security specialists.

Modern malware detects execution environments to complicate analysis. 
It can identify virtual machines, hardware-supported virtual machines, and emulators through timing attacks and environment detection. 
Encryption and packing techniques are also employed, where malicious content is encrypted, and a small encryption/decryption routine changes the key with each execution. 
Typical functions of these routines include compressing/decompressing, encrypting/decrypting, incorporating metamorphic components, and implementing anti-debugging and anti-VM techniques.

These advanced techniques make it difficult for antivirus software to detect and analyze complex malware, which often utilizes rootkit techniques to further obscure its presence.

\paragraph*{Analysis}
The process of analyzing suspicious executables typically follows a workflow: first, a suspicious executable is reported, then it is automatically analyzed, followed by manual analysis, and finally, an antivirus signature is developed.

Dynamic analysis involves observing the runtime behavior of the executable. 
It has the advantage of dealing with obfuscation techniques like metamorphism, encryption, and packing, but it might not cover all the code, particularly dormant code.

Static analysis involves parsing the executable code, which provides comprehensive code coverage and detects dormant code, but it can be hindered by obfuscation techniques such as metamorphism, encryption, and packing.
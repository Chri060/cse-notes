\section{Rootkit}

Historically, rootkits emerged as a way for attackers to maintain root access on a compromised machine. 
These tools allow attackers to make files, processes, users, and directories disappear, effectively rendering themselves invisible.
Rootkits can operate in either userland or kernel-space.

In userland, rootkits can backdoor login mechanisms and password files. 
They often trojanize utilities to hide their presence. 

\paragraph*{Rootkit types}
Userland rootkits are easier to build but are often incomplete and more easily detected through cross-layer examination and the use of non-trojaned tools. 
Kernel space rootkits, on the other hand, are more challenging to construct but can completely hide artifacts. 
These rootkits can only be detected via postmortem analysis. 

\paragraph*{Syscall hijacking}
Syscall hijacking involves manipulating the syscall table, the Interrupt Descriptor Table (IDT), or the Global Descriptor Table (GDT). 

\paragraph*{Advanced rootkits}
Advanced rootkits extend beyond software to embed themselves in hardware components and firmware. 
Brossard introduced a rootkit that operates independently of the BIOS. 
Rootkits can also target the firmware of Network Interface Cards (NICs) or video cards. 
In virtualization systems, rootkits can act as hypervisors, making detection extremely difficult.
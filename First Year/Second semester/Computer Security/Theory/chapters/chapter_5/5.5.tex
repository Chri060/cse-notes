\section{Other issues}

\subsection{Information leakage}
While detailed error messages enhance Human-Computer Interaction (HCI), they can inadvertently lead to security vulnerabilities. 
Information leakage can occur through various avenues, including active debug traces in production environments. 
Additionally, risks stem from the insertion of user-supplied data in errors, which may expose vulnerabilities like Reflected Cross-Site Scripting. 
Side channels are another potential source of data leaks.

\subsection{Password security}
The essentials of password security stay the same: passwords should never be stored in plain text in web applications to reduce the risk of exposure in case of a breach. 
Employing techniques like salting and hashing is crucial to prevent attacks like rainbow tables. 
It's crucial to handle password reset processes carefully: Typically, resetting a password involves providing an alternative login method. 
Common methods include sending a reset link to a registered email. 
Less secure practices include sending temporary passwords or relying solely on security questions for verification.

\paragraph*{Brute-forcing protection}
Protecting against brute-force attacks requires thoughtful measures:
\begin{itemize}
    \item A naive solution involves locking an account after a certain number of failed login attempts. 
        However, this can lead to reverse brute-forcing, where attackers focus on other accounts.
    \item Making accounts non-enumerable can prevent attackers from cycling through usernames systematically.
    \item  Blocking IP addresses may seem intuitive, but it's not foolproof due to proxies and NATs. 
        Moreover, this approach can inadvertently lead to Denial of Service (DoS) attacks.
\end{itemize}
\section{Web Application security}

Web Applications have become the primary method for delivering software across various environments, including corporate intranets, SaaS platforms, and Cloud services.
These applications are often designed to be publicly accessible, similar to public web services, and operate on the stateless HTTP protocol.
This protocol's stateless nature requires additional mechanisms to simulate state management for maintaining user sessions and data between requests. 
Moreover, HTTP's built-in authentication methods are relatively weak, necessitating the implementation of more robust authentication strategies to enhance security.

The key principle in web application security is that the client should never be trusted implicitly. 
It is crucial to rigorously filter and validate all data received from the client to mitigate risks associated with potentially malicious inputs or attacks.

\paragraph*{Data filtering}
Filtering data is a complex task, but employing a variety of validation techniques can significantly improve security:
\begin{itemize}
    \item \textit{Allow listing}: only permit data that meets predefined criteria, effectively restricting input to expected and safe values.
    \item \textit{Black listing}: exclude known malicious content in addition to using allow listing. This method is supplementary and helps catch specific threats.
    \item \textit{Escaping}: convert special characters into safer forms to prevent them from being interpreted as executable code or commands.
\end{itemize}
A fundamental principle to follow is that allow listing generally provides stronger security compared to black listing, as it proactively controls what data is permitted rather than reacting to known threats.
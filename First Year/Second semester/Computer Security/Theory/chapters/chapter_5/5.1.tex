\section{Web application security}

Web applications have become the predominant method for delivering software in various contexts. 
They are commonly utilized within corporate intranets and are prevalent in SAAS and cloud environments. 
These applications are often designed to be accessible to the public, resembling public web services in their functionality.
They are typically built on top of the stateless HTTP protocol, which necessitates the emulation of state management to maintain user sessions and data between requests.
Additionally, HTTP's built-in authentication mechanisms are relatively weak, prompting the implementation of additional authentication measures to bolster security.

\paragraph*{Client}
The golden rule in web application security asserts that the client should never be considered trustworthy. 
Therefore, it is imperative to thoroughly filter and scrutinize any data received from the client. 
This approach helps mitigate the risks associated with potential malicious inputs or attacks originating from users interacting with the application.

\paragraph*{Data filtering}
Filtering data presents a formidable challenge, but implementing a series of validation techniques can bolster security:
\begin{itemize}
    \item \textit{Allowlisting}: Authorizing only anticipated data to traverse.
    \item \textit{Blocklisting}: Additional filtration to exclude recognized malicious content in addition to allowlisting.
    \item \textit{Escaping}: Transforming special characters into less harmful counterparts.
\end{itemize}
A foundational principle: allowlisting typically offers greater security compared to blocklisting.
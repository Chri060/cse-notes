\section{SQL injection}

SQL injection is a vulnerability that allows attackers to manipulate SQL queries by injecting malicious input.
This often involves terminating a query prematurely, commenting out parts of the query, or adding additional SQL commands.

To terminate an SQL statement, a semicolon followed by a double dash (--) is used to comment out the remainder of the line. 
This technique can be exploited to alter the behavior of a query. For example:
\begin{verbatim}
' OR '1'='1';--
\end{verbatim}
This injection alters the query to always return true, potentially bypassing authentication or other security checks.

\paragraph*{Unions}
Union-based SQL injection allows attackers to retrieve data from different tables within a single query. 
For example:
\begin{verbatim}
SELECT name, phone, address FROM Users
WHERE Id='' UNION ALL SELECT name,creditCardNumber,CCV2 
FROM CreditCardTable;--';
\end{verbatim}
This query combines results from two different tables. 
It works only if the number and types of columns in the original and injected queries match.

\paragraph*{Insertions}
Insert injection involves modifying an insert statement to insert arbitrary data or execute sub-queries. For example:
\begin{verbatim}
INSERT INTO results VALUES (NULL, 's.zanero',
    (SELECT password from USERS where user='admin')
    )--', '18')
\end{verbatim}
Here, the injected sub-query retrieves the password of an admin user, which is then inserted into the results table.

\paragraph*{Blind injections}
Blind SQL injection occurs when the SQL queries do not return data directly but still allow an attacker to infer information based on the application's behavior. 
For example, you may infer whether a condition is true or false based on the response or timing of the application.

\subsection{Countermeasures}
To protect against SQL injection attacks, consider the following measures:
\begin{itemize}
    \item \textit{Input sanitization}: validate and filter user inputs to ensure they do not contain harmful SQL syntax.
    \item \textit{Prepared statements}: use parameterized queries instead of directly concatenating user inputs into SQL statements. 
        Prepared statements use placeholders for variables, which prevents injection.
    \item \textit{Avoid dynamic queries}: minimize the use of dynamic SQL and avoid using table names or other sensitive identifiers as field names to prevent information leakage.
    \item \textit{Restrict database privileges}: implement least privilege principles by limiting the types of queries that different users can execute on the database. 
        Ensure that users have only the permissions necessary for their roles.
\end{itemize}
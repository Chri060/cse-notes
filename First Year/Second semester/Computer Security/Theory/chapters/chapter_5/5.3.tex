\section{SQL injection}

To end a statement in SQL we simply put a semicolon followed by a double dash line that it is used to comment a part of the line. 
In that way we can modify the query as we desire. 

Another way to modify a query is to use the following line where a parameter is requested: 
\begin{verbatim}
' OR '1'='1';--
\end{verbatim}
In that way the query will be modified resulting in always true. 

\paragraph*{Unions}
By using unions we can retrieve data from different tables: 
\begin{verbatim}
SELECT name, phone, address FROM Users
WHERE Id='' UNION ALL SELECT name,creditCardNumber,CCV2 
FROM CreditCardTable;--';
\end{verbatim}
Will show contents of a different table.
Will work only if the number and the data types of the columns are the same.

\subsection{Insert injection}
On insert we can use the double dash line to comment a section of the statement and change the inserted values. 
We can also modify the insertion query by adding a sub-query that retrieve some wanted data like passwords and then retrieve the value of the data with another query: 
\begin{verbatim}
INSERT INTO results VALUES (NULL, 's.zanero',
    (SELECT password from USERS where user='admin')
    )--', '18')
\end{verbatim}

\subsection{Blind injections}
Some SQL queries do not display returned values. 
Rather, they do, or do not do, stuff based on the return value. 
As a result, we cannot use them to directly display data, but we can play with their behavior to infer data. 

\subsection{Solutions}
We can avoid SQL injections in the following ways: 
\begin{itemize}
    \item Input sanitization (validation and filtering). 
    \item Using prepared statements (parametric query) instead of building query strings (if languages allows). 
        In this case we have variable placeholders, that is not a string concatenation. 
        For instance, in PHP we have: 
        \begin{verbatim}
            $stmt = $db->prepare(SELECT * 
                                 FROM users
                                 WHERE username = ? AND password = ?” )
            $stmt -> execute(array($username,$psw));
        \end{verbatim}
    \item Not using table names as field names (to avoid information leakage). 
    \item Limitations on query privileges. 
        Different users can execute different types of queries on different tables/DBs, so separate privileges from DB admin point of view. 
\end{itemize}
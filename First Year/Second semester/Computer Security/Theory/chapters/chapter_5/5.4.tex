\section{Cross-Site Request Forgery}

Cross-Site Request Forgery (CSRF) is an attack that tricks a user into performing unwanted actions on a Web Application where they are currently authenticated. 
This is achieved by leveraging the user's existing credentials, such as cookies, which are automatically included with each request made from their browser. 
Consequently, attackers can craft malicious requests that are processed by the vulnerable Web Application using the victim's credentials. 
Since the Web Application cannot differentiate between legitimate user actions and those initiated by an attacker, it is at risk of executing unauthorized commands.

\subsection{Countermeasures}
To protect against CSRF attacks, several effective countermeasures can be implemented:
\begin{itemize}
    \item \textit{Anti-CSRF tokens}: use a unique, random token associated with each user session. 
        This token should be included in any form or request that performs sensitive operations. 
        The server generates and stores this token, and it must match the token submitted with the request for the operation to proceed. 
        Importantly, these tokens should not be stored in cookies to avoid exposure to cross-site requests.
    \item \textit{SameSite cookies}: utilize the SameSite attribute for cookies to restrict how cookies are sent with cross-site requests. 
        This attribute helps prevent session cookies from being included in requests originating from other sites:
        \begin{itemize}
            \item SameSite=Strict: cookies are sent only in requests originating from the same site, providing strong protection against CSRF.
            \item SameSite=Lax: cookies are sent with top-level navigation requests but not with cross-site post requests, images, or iframes, offering a balance between usability and security.
        \end{itemize}
\end{itemize}

\section{Firewall}

\begin{definition}[\textit{Firewall}]
    A firewall is a network access control system that inspects and verifies all packets flowing through it.
\end{definition}
The main functions of a firewall typically include:
\begin{itemize}
    \item IP packet filtering.
    \item Network Address Translation (NAT).
\end{itemize}
A firewall must serve as the single enforcement point between a screened network and outside networks.

\paragraph*{Insider attacks}
Firewalls inspect all traffic passing through them but are powerless against insider attacks unless the network is partitioned.
They are generally ineffective against unchecked paths within the network.

\paragraph*{Firewall rules}
Firewalls are essentially computers and can have vulnerabilities. 
Most firewalls are single-purpose machines, often embedded appliances with minimal firmware that offer few or no additional services, resulting in a smaller attack surface. 
Firewalls act as bouncers at the network boundary, enforcing security rules. 
Ineffective rules result in inadequate protection. 
Firewall rules implement higher-level security policies, which should be based on a default-deny approach.

\paragraph*{Taxonomy}
Firewalls are categorized based on their packet inspection capabilities, from lower to higher layers:
\begin{itemize}
    \item \textit{Network layer firewalls}: packet filters and stateful packet filters. 
    \item \textit{Application layer firewalls}: circuit-level gateways and application proxies. 
\end{itemize}

\subsection{Packet filters}
Packet filters process packets individually, decoding the IP and part of the TCP headers. 
They are stateless and cannot track TCP connections or fully inspect payloads. 
Often found on routers as Access Control Lists (ACLs), packet filters predicate actions (block, allow, log, etc.) based on packet conditions.

\subsection{Stateful packet filters}
Dynamic packet filters enhance standard packet filters by tracking the state of TCP connections, ensuring the correct sequence of packets (e.g., SYN followed by SYN-ACK). 
This tracking improves the safety of deny rules but can impact performance as it is connection-based rather than packet-based. 
Dynamic filters can also perform deeper content inspection, reconstruct application-layer protocols, and offer features like NAT, packet defragmenting, and reassembly to handle fragmented packets.

\paragraph*{Session handling}
A session is an atomic, transport-layer exchange of application data between two hosts. Main transport protocols include:
\begin{itemize}
    \item TCP (Transmission Control Protocol): the session is the TCP connection itself.
    \item UDP (User Datagram Protocol): connectionless, but sessions can be inferred for NAT and control purposes.
        UDP is widely used (e.g., DNS, VoIP, video streaming) despite its lack of connection orientation, making it difficult to secure.
\end{itemize}

\paragraph*{Application layer}
Application-layer inspection is crucial for protocols that transmit network information. 
Stateful firewalls must manage these protocols to ensure security.

\subsection{Circuit firewalls}
Circuit firewalls relay TCP connections, acting as a TCP-level proxy. 
The client connects to a specific port on the firewall, which then connects to the desired server.

\subsection{Application proxies}
Application proxies operate at the application layer, inspecting, validating, and manipulating protocol data (e.g., rewriting HTTP frames). 
They can authenticate users, apply specific filtering policies, perform advanced logging, and offer content filtering or scanning (e.g., anti-virus, spam filtering). 
Proxies may require client modifications and specific protocol servers.
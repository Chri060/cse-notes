\section{Network attacks}

In typical network operations, a Network Interface Card (NIC) is configured to intercept and forward only the packets addressed specifically to its host IP address. 
However, when a NIC is set to promiscuous mode, it captures all packets it reads from the network, regardless of their destination address.
This mode can be used for network diagnostics, but it also exposes all traffic to potential snooping.

\paragraph*{Historical context}
Originally, Ethernet networks utilized a shared medium with BNC cables, meaning that all traffic was broadcast to every device on the network.
Even though RJ-45 cables replaced BNC, the use of hubs maintained this broadcast behavior. 
Hubs transmit all incoming traffic to every device within the broadcast domain, making all network traffic visible to all connected hosts.

Modern networks often use switches instead of hubs. 
Switches enhance performance by forwarding packets only to the specific port associated with the destination MAC address, rather than broadcasting to all ports. 
This selective forwarding improves efficiency but is not designed for security purposes.

\paragraph*{Spoofing}
UDP (User Datagram Protocol) does not authenticate the source IP address, making it vulnerable to spoofing.
Attackers can easily alter the source address in UDP or ICMP packets, leading to various malicious outcomes.
If an attacker is on a different network than the target, they won't receive responses to their spoofed packets, as these responses are sent to the forged source address. 
This is known as blind spoofing.
Conversely, if the attacker is on the same network as the target, they can capture responses or use techniques like ARP spoofing to exploit further network vulnerabilities.

TCP (Transmission Control Protocol) differs from UDP by using sequence numbers for packet ordering and acknowledgment. 
During connection establishment, TCP uses a semi-random Initial Sequence Number (ISN).
An attacker who can accurately predict the ISN might complete the TCP three-way handshake without receiving responses. 
However, this requires that the attacker's spoofed source address does not receive any responses. 
If response packets are received, the target may send a Reset (RST) packet, which can disrupt the connection and potentially alert the target to the attack.

\subsection{Killer packets attacks}
We may use killer packets to perform a DoS on a user. 
We may use: 
\begin{itemize}
    \item \textit{Smurf attack}: this attack involves sending ICMP echo requests (pings) with a spoofed source address (the victim) to a network's broadcast address. 
        This results in a flood of responses from all devices on the network, overwhelming the victim and potentially causing system crashes or erratic behavior.
    \item \textit{Teardrop attack}: this exploit takes advantage of vulnerabilities in the TCP reassembly process.
        By sending fragmented packets with overlapping offsets, the attacker can cause the target system's kernel to hang or crash during packet reassembly.
    \item \textit{Land attack}: this attack targets older systems, such as Windows 95. 
        It involves sending a packet where the source IP address is the same as the destination IP address, with the SYN flag set. 
        This can cause the TCP/IP stack to enter a loop, leading to a system lock-up.
\end{itemize}

\subsection{Syn flood attacks}
SYN Flood attacks exploit the TCP three-way handshake process. 
An attacker sends a high volume of SYN requests with spoofed source addresses, filling the server's queue with half-open connections. 
This congestion prevents legitimate SYN requests from being processed, effectively denying service to legitimate users.

\paragraph*{Countermeasures}
To mitigate SYN Flood attacks, SYN cookies can be employed. 
This technique involves sending a SYN+ACK response while discarding the half-open connection. 
The server waits for a subsequent ACK from the client to complete the connection establishment, reducing the impact of the attack.

\subsection{Botnets}
A botnet is a network of compromised computers, known as bots, that are controlled remotely by an attacker through a command-and-control infrastructure. 
Botnets are used for various malicious activities, including spamming, phishing, information theft, and conducting large-scale DDoS attacks.

\subsection{ARP spoofing}
The Address Resolution Protocol (ARP) is used to map 32-bit IPv4 addresses to 48-bit MAC addresses. 
ARP operates through a simple request-reply mechanism: a device sends an ARP request to find the MAC address associated with a given IP address, and the device holding that IP address responds with an ARP reply containing its MAC address.

However, ARP lacks authentication, making it vulnerable to spoofing attacks. In ARP spoofing, an attacker sends forged ARP replies to a network, associating their own MAC address with the IP address of another device, such as the gateway. 
This can lead to a range of issues including MITM attacks, traffic interception, and DoS.

\paragraph*{Countermeasures}
To counter ARP spoofing, it's essential to verify responses before trusting them, especially if they conflict with existing address mappings. 
Validate ARP replies before accepting them, particularly if they conflict with known or expected address mappings. 
This helps ensure that ARP responses are legitimate.

\subsection{MAC flooding}
Content Addressable Memory (CAM) tables are essential components of network switches, used to map MAC addresses to specific ports. 
This mapping allows switches to efficiently direct traffic only to the appropriate port based on the destination MAC address.
However, CAM tables are vulnerable to attacks such as ARP spoofing, which can compromise their integrity.

One common attack on CAM tables is MAC flooding, where attackers use tools to overwhelm the CAM table with a large volume of spoofed MAC addresses. 
When the CAM table becomes saturated, the switch loses its ability to store MAC address-to-port mappings effectively. 
As a result, the switch defaults to broadcasting all incoming traffic to every port, similar to the behavior of a hub, leading to reduced network performance and potential security risks.

\paragraph*{Countermeasures}
Implementing port security, a concept often associated with Cisco, can help mitigate MAC flooding attacks. 
Port security features allow administrators to define and limit the number of MAC addresses that can be learned on a switch port, thus preventing CAM table saturation.

\subsection{TCP hijacking}
TCP session hijacking involves taking over an active TCP connection between two parties. 
Here's a typical process for executing this attack:
\begin{enumerate}
    \item \textit{Intercept and monitor}: the attacker (C) intercepts and observes packets exchanged between two parties (A and B), capturing critical information such as sequence numbers.
    \item \textit{Disrupt the connection}: the attacker disrupts A's connection, potentially through methods like a SYN Flood attack. 
        This causes A to experience a random or seemingly arbitrary service interruption.
    \item \textit{Impersonate and engage}: the attacker, then, impersonates A by spoofing A's IP address and using a correct Initial Sequence Number (ISN) to initiate communication with B. 
        B is unaware that they are now interacting with C instead of A.
\end{enumerate}
Various tools and scripts can automate this attack, making it more efficient. 
When an attacker is positioned as a MITM, they can inject content into the communication flow without disrupting B's session, allowing them to control or resynchronize all traffic passing through.
\begin{definition}[\textit{Man In The Middle}]
    A Man In The Middle (MITM) attack encompasses various techniques where an attacker impersonates either the server or the client to intercept or manipulate communication between them.
\end{definition}

\subsection{DNS poisoning}
The Domain Name System (DNS) is essential for converting domain names into IP addresses. 
Instead of using a single comprehensive file or hash for mappings, DNS relies on a distributed database system:
\begin{itemize}
    \item \textit{Hierarchical servers}: the DNS system is structured with a hierarchy of servers, each maintaining a cache of domain-to-IP mappings.
    \item \textit{UDP communication}: DNS queries and responses use UDP on Port 53 and do not include authentication.
    \item \textit{Query process}: when a domain name is requested and not found in the local cache, the system queries a series of DNS servers.
\end{itemize}
Each DNS server in this hierarchy holds resource records that link domain names with IP addresses.

\paragraph*{Cache poisoning attack}
A cache poisoning attack manipulates DNS responses to redirect users to malicious sites. 
Here's how it typically unfolds:
\begin{itemize}
    \item \textit{Initiate query}: the attacker sends a recursive query to a target DNS server.
    \item \textit{Contact authoritative server}: the target DNS server queries the authoritative DNS server for the domain.
    \item \textit{Spoofed response}: the attacker intercepts or guesses the DNS query ID and sends a forged response, pretending to be the authoritative server.
    \item \textit{Cache malicious record}: the target DNS server accepts and caches the fraudulent record, believing it to be legitimate.
\end{itemize}
As a result, any client querying the poisoned DNS server will be redirected to the attacker's malicious website.

\subsection{DHCP poisoning}
The Dynamic Host Configuration Protocol (DHCP) is used to automatically assign IP addresses and network configuration parameters to devices on a network. 
Here's how DHCP operates:
\begin{itemize}
    \item \textit{Automatic IP assignment}: when a device connects to the network, DHCP assigns it a new IP address automatically.
    \item \textit{Centralized management}: network administrators can centrally manage and distribute configuration settings such as IP addresses, router information, and subnet masks.
\end{itemize}
Despite its convenience, DHCP has some limitations:
\begin{itemize}
    \item \textit{Lack of authentication}: DHCP does not authenticate requests or responses, making it vulnerable to attacks.
    \item \textit{UDP dependency}: DHCP operates over UDP, which does not guarantee delivery or order of messages.
    \item \textit{Server dependency}: DHCP servers must be continuously operational.
        If a DHCP server becomes unavailable, devices may lose access to network resources.
\end{itemize}

\paragraph*{DHCP poisoning}
In a DHCP poisoning attack, the lack of authentication is exploited to manipulate network configurations. 
Here's how the attack typically works:
\begin{enumerate}
    \item \textit{Intercept requests}: the attacker intercepts DHCP requests from clients on the network.
    \item \textit{Spoof responses}: the attacker sends spoofed DHCP responses before the legitimate server can reply.
    \item \textit{Manipulate configurations}: through these forged responses, the attacker can assign malicious IP addresses, DNS servers, and default gateways to the victim devices.
\end{enumerate}
As a result, victim clients may receive incorrect network settings, potentially leading to network breaches, traffic interception, or DoS.

\subsection{ICMP redirect}
The Internet Control Message Protocol (ICMP) is used for sending error messages and diagnostic information between network devices, such as hosts and routers. 
ICMP messages are categorized into three main types: requests, responses, and error messages.

Routing information is primarily managed and updated by routers, not hosts.
Routers determine and maintain the best paths for reaching different destinations. 
Initially, hosts may have limited routing knowledge and rely on routers to learn and update routes.

\paragraph*{ICMP redirect}
An ICMP Redirect message notifies a host that a more efficient route to a specific destination is available and provides the address of the better gateway. 
The typical process is as follows:
\begin{itemize}
    \item \textit{Detection}: a router detects that a host is using a suboptimal route for a particular destination.
    \item \textit{Notification}: the router sends an ICMP Redirect message to the host and forwards the original packet.
    \item \textit{Update}: the host updates its routing table to use the suggested gateway.
\end{itemize}
However, attackers can exploit ICMP Redirect messages by sending spoofed redirects. 
This can hijack traffic and facilitate DoS attacks. 
\section{Denial Of Service}

Denial of Service (DoS) attacks aim to disrupt the availability of services. 
Examples include Killer Packets, SYN floods, Smurf attacks, amplification attacks, and Distributed DoS (DDoS).

\subsection{Killer packets}
\paragraph*{Ping of Death}
A pathological ICMP echo request that exploits memory errors in the protocol implementation, causing the target system to crash or behave unpredictably.

\paragraph*{Teardrop}
This attack exploits vulnerabilities in the TCP reassembly process.
It sends fragmented packets with overlapping offsets, causing the target system's kernel to hang or crash during reassembly.

\paragraph*{Land attack}
\paragraph*{Land Attack}
In older systems, such as Windows 95, sending a packet with the source IP equal to destination IP, and the SYN flag set could cause the TCP/IP stack to loop and lock up.

\subsection{Syn flood attacks}
These attacks exploit the TCP three-way handshake. 
The attacker sends a large volume of SYN requests with spoofed source addresses, filling up the queue with half-open connections. 
As a result, SYN requests from legitimate clients are dropped.
\paragraph*{Mitigation}
SYN cookies mitigate this attack by replying with a SYN+ACK while discarding the half-open connection, waiting for a subsequent ACK to establish a connection.

\paragraph*{Botnet case}
A botnet is a network of compromised computers, known as bots, controlled by an attacker through a command-and-control (C&C) infrastructure. 
Botnets are used for various malicious activities, including spamming, phishing, information theft, and DDoS attacks.

\subsection{Smurf}
The attacker sends ICMP packets with a spoofed sender address (victim) to a network's broadcast address, causing a flood of responses to overwhelm the victim.
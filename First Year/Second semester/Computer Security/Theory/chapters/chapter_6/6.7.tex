\section{Multi-zone architecture}

Perimeter defense typically assumes internal network safety while keeping external threats out. 
However, scenarios like remote access to resources (e.g., web servers, FTP, mail transfer) and remote user access to corporate networks challenge this assumption. 
Mixing externally accessible servers with internal clients reduces internal network security.

A practical solution is a multi-zone architecture, splitting the network by privilege levels and using firewalls to regulate access. 
This involves creating a DMZ (demilitarized zone) for public servers (web, FTP, DNS, SMTP) while keeping critical data away from the DMZ, which is considered as risky as the internet.

\subsection{Virtual Private Network}
VPNs enable remote employees to work as if they were in the office and connect remote sites without dedicated lines, ensuring confidentiality, integrity, and availability (CIA) of data over public networks. 
VPNs create encrypted overlay connections over public networks, supporting full and split tunneling:
\begin{itemize}
    \item \textit{Full tunneling}: all traffic goes through the VPN, allowing complete control and application of security policies.
    \item \textit{Split tunneling}: only traffic to the corporate network uses the VPN; other traffic goes directly to the ISP, improving efficiency but reducing control.
\end{itemize}

\paragraph*{VPN technologies}
The possible technologies for a VPN are: 
\begin{itemize}
    \item PPTP (Point-to-Point Tunneling Protocol): a proprietary Microsoft protocol, an extension of PPP with added authentication and confidentiality.
    \item TLS-Based VPNs: using pure TLS, SSH tunnels, or OpenVPN (TLS-based).
    \item IPsec: provides security extensions for IPv6, backported to IPv4, offering authentication and confidentiality at the IP layer.
\end{itemize}
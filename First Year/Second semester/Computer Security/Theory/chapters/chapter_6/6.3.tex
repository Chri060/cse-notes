\section{Sniffing}

Normally, a network interface card (NIC) intercepts and forwards to the operating system only the packets addressed to the host's IP.

In promiscuous mode, the NIC passes any packet it reads off the wire to the operating system, regardless of the packet's destination address.

Originally, Ethernet networks operated on a shared medium, using BNC cables. 
Although RJ-45 cables changed the physical connector, the medium remained shared due to the use of hubs, which broadcast traffic to every host within the broadcast domain.

In contrast, switches selectively relay traffic only to the specific wire corresponding to the correct NIC, based on the Ethernet address. 
This selective relaying is primarily for performance enhancement, not for security reasons.

\subsection{ARP spoofing}
Address Resolution Protocol (ARP) is responsible for mapping 32-bit IPv4 addresses to 48-bit hardware, or MAC, addresses.
RP works with a simple request-reply mechanism with a request and a reply. 
However, ARP lacks authentication, making it susceptible to spoofing. 
An attacker can easily forge ARP replies, leading to a lack of trust in the received information. 
Each host caches these replies, which can be viewed using the command \texttt{arp -a}.
\paragraph*{Mitigation}
To counter ARP spoofing, it's essential to verify responses before trusting them, especially if they conflict with existing address mappings. 
One approach is to add a sequence or ID number in the request to help validate the responses.

\subsection{CAM tables}
CAM tables are integral to switches as they cache the mapping of MAC addresses to specific ports.
Switches are susceptible to ARP spoofing attacks, compromising the integrity of CAM tables.
A CAM table can be saturated in the following way: 
\begin{itemize}
    \item Switches rely on CAM tables to efficiently direct traffic based on MAC addresses.
    \item Tools like Dsniff (macof) can inundate switches with a massive volume of spoofed packets, saturating the CAM table within seconds (known as MAC flooding).
    \item When the CAM table reaches full capacity, the switch can no longer effectively cache ARP replies. 
        Consequently, it resorts to broadcasting all incoming traffic to every port, resembling the behavior of a hub.
\end{itemize}
\paragraph*{Mitigation}
Implementing PORT Security, a term coined by CISCO, helps mitigate these vulnerabilities.

\subsection{Spanning Tree Protocol}
The Spanning Tree Protocol (STP) 802.1d is crucial for preventing loops in switched networks by constructing a spanning tree (ST). 
Switches determine the structure of the ST through the exchange of Bridge Protocol Data Unit (BPDU) packets to elect the root node.

However, since BPDU packets are not authenticated, attackers can manipulate the topology of the spanning tree for malicious purposes such as sniffing or ARP spoofing.
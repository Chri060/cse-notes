\section{Firewall}

\begin{definition}[\textit{Firewall}]
    A firewall is a network security system designed to monitor and control incoming and outgoing network traffic based on predetermined security rules.
\end{definition}
The primary functions of a firewall include:
\begin{itemize}
    \item \textit{IP packet filtering}: examining and controlling packet flow based on IP address and port number.
    \item \textit{Network Address Translation} (NAT): modifying IP address information in packet headers to improve network security and efficiency.
\end{itemize}
A firewall acts as a gatekeeper, enforcing security policies between a protected internal network and external networks.

While firewalls are effective at managing traffic between different networks, they are generally ineffective against insider threats unless the network is properly segmented.
They cannot prevent unauthorized activities originating from within the network itself.

Firewalls are essentially specialized computers and may have vulnerabilities of their own. 
Most firewalls are purpose-built appliances with minimal firmware and limited additional services, reducing their attack surface. 
They enforce security policies by applying rules that should follow a default-deny approach, where all traffic is denied unless explicitly allowed.

Firewalls can be categorized based on their packet inspection capabilities:
\begin{itemize}
    \item \textit{Network layer firewalls}: include packet filters and stateful packet filters that operate at the network layer.
    \item \textit{Application layer firewalls}: include circuit-level gateways and application proxies that operate at the application layer.
\end{itemize}

\subsection{Packet filters}
Packet filters process individual packets by examining IP and part of the TCP headers. 
They are stateless, meaning they do not track the state of TCP connections or fully inspect packet payloads. 
Packet filters, often implemented as Access Control Lists (ACLs) on routers, base their decisions on packet-specific conditions such as IP addresses and port numbers.

\subsection{Stateful packet filters}
Stateful packet filters build upon basic packet filtering by tracking the state of active TCP connections. 
They ensure packets follow the correct sequence and provide more robust security.
While they offer deeper inspection and additional features such as NAT, packet defragmentation, and reassembly, they can also impact performance due to their connection-oriented nature.

\subsection{Circuit firewalls}
Circuit firewalls function as TCP-level proxies. 
They relay TCP connections by allowing clients to connect to a specific port on the firewall, which then establishes a connection to the desired server on behalf of the client.

\subsection{Application proxies}
pplication proxies operate at the application layer, inspecting, validating, and modifying protocol data. 
They provide additional features such as user authentication, specific filtering policies, advanced logging, and content filtering. 
Application proxies may require modifications to clients and servers to function properly.

\subsection{Multi-zone architecture}
Traditional perimeter defense strategies often assume that once external threats are blocked, the internal network remains secure. 
However, this assumption is challenged by scenarios such as remote access to resources and remote user access to corporate networks. 
These scenarios can compromise internal network security by mixing externally accessible servers with internal clients.

A robust approach to addressing these security concerns is the implementation of a multi-zone architecture. 
This approach divides the network into distinct zones based on privilege levels and uses firewalls to control access between them. 
A key component of this architecture is the creation of a Demilitarized Zone (DMZ) where public-facing servers are placed. 
The DMZ is considered a high-risk area similar to the internet, and critical data and systems are kept separate from it.

\paragraph*{Virtual Private Network}
Virtual Private Networks (VPN) allow remote employees to securely access corporate resources and connect remote sites without requiring dedicated leased lines. 
They ensure the confidentiality, integrity, and availability (CIA) of data over public networks by creating encrypted connections. 
VPNs support two main types of tunneling:
\begin{itemize}
    \item \textit{Full tunneling}: all traffic is routed through the VPN, which allows comprehensive control over security policies and traffic monitoring.
    \item \textit{Split tunneling}: only traffic intended for the corporate network is routed through the VPN, while other traffic goes directly to the internet. 
        This method can improve efficiency but reduces control over non-corporate traffic.
\end{itemize}
\section{ICMP redirect attack}

ICMP (Internet Control Message Protocol) is utilized for sending debugging information and error reports between hosts, routers, and other network devices at the IP level. 
ICMP messages fall into three categories: requests, responses, and error messages.
Key ICMP functions include:
\begin{itemize}
    \item \textit{Echo request/reply}: used to test connectivity (ping).
    \item \textit{Redirect}: used to inform hosts about more efficient routes (gateways).
\end{itemize}

\paragraph*{Route change requests}
Routers, not hosts, are responsible for maintaining and updating routing information. 
They are expected to discover and manage the best routes for every destination. 
Initially, hosts start with minimal routing information and gradually learn new routes from routers.
A host may boot up knowing the address of only one router, which might not always be the most optimal route.

\subsection{ICMP redirect}
An ICMP Redirect message informs a host that a better route exists for a given destination and provides the gateway for that route.
When a router detects a host using a suboptimal route, it:
\begin{itemize}
    \item Sends an ICMP Redirect message to the host and forwards the original packet.
    \item The host is expected to update its routing table accordingly.
\end{itemize}
An attacker can exploit this by forging a spoofed ICMP Redirect packet to re-route traffic to a specific route or a particular host, which may not be a legitimate router. This attack can:
\begin{itemize}
    \item Hijack traffic, directing it through the attacker's computer.
    \item Facilitate a denial-of-service (DoS) attack.
\end{itemize}

\paragraph*{Weak authentication}
An ICMP message includes the IP header and a portion of the payload (usually the first 8 bytes) of the original IP datagram. 
To forge a convincing ICMP Redirect message, an attacker must intercept a packet in the original connection, necessitating their presence on the same network. 
This creates a half-duplex man-in-the-middle (MITM) situation.
Handling of ICMP Redirects varies by operating system.

\paragraph*{Route mangling}
If an attacker can announce routes to a router, they can manipulate routing in various ways. 
Common routing protocols with no or weak authentication include:
\begin{itemize}
    \item IGRP (Interior Gateway Routing Protocol).
    \item RIP (Routing Information Protocol).
    \item OSPF (Open Shortest Path First).
\end{itemize}
More secure protocols like EIGRP (Enhanced Interior Gateway Routing Protocol) and BGP (Border Gateway Protocol) offer authentication options, but these are often underutilized.
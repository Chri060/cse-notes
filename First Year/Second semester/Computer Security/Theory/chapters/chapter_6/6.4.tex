\subsection{Spoofing}

\paragraph*{UDP spoofing}
In UDP (User Datagram Protocol), the IP source address lacks authentication, making it susceptible to modification in UDP or ICMP packets.
This process is relatively simple for attackers. 
However, if the attacker is on a different network, they won't receive responses to their spoofed packets, as these responses will be directed to the spoofed host (known as blind spoofing).
However, if the attacker is on the same network, they can sniff the remaining traffic or employ ARP spoofing to further exploit vulnerabilities.
\paragraph*{TCP spoofing}
Unlike UDP, TCP (Transmission Control Protocol) employs sequence numbers for packet reordering and acknowledgment. 
It utilizes a semi-random Initial Sequence Number (ISN) during the connection setup.
If a blind spoofer can predict the ISN, they can complete the TCP three-way handshake blindly without receiving responses. 
However, it's crucial that the spoofed source does not receive response packets. 
Otherwise, it might respond with a Reset (RST) packet, potentially disrupting the connection.

\subsection{TCP session hijacking}
TCP session hijacking involves seizing control of an active TCP session. 
Here's how it typically occurs:
\begin{enumerate}
    \item The attacker (C) intercepts and monitors packets between two parties (A and B), recording sequence numbers.
    \item C disrupts A's connection, such as by launching a SYN Flood attack, causing A to experience a seemingly random disruption in service.
    \item C assumes A's identity by spoofing A's address and using a correct Initial Sequence Number (ISN) to engage in dialogue with B. B remains unaware of the deception.
\end{enumerate}
Numerous tools, like hunt and dsniff, automate this attack process.
If the attacker is a man in the middle, they can inject content into the flow without disrupting B's session, thereby controlling or resynchronizing all traffic passing through.
\begin{definition}[\textit{Man in the Middle}]
    A man in the middle attack encompasses various techniques where an attacker impersonates either the server or the client to intercept or manipulate communication between them.
\end{definition}

\subsection{Domain Names System}
The Domain Name System (DNS) is responsible for translating domain names into numerical IP addresses. 
Rather than employing a hash or a file of all mappings, DNS operates through a distributed database system:
\begin{itemize}
    \item A hierarchy of servers maintains mappings, with each server retaining a small cache of these mappings.
    \item DNS operates over UDP on Port 53, with messages lacking authentication.
    \item When a domain name is requested and not found in the local cache, the system queries a DNS server.
\end{itemize}
This hierarchy of DNS servers contains resource records to match domain names with IP addresses.

\paragraph*{Domain name resolution}
Consider the example of typing "sports.polimi.com":
\begin{itemize}
    \item he system checks /etc/hosts and the DNS cache.
    \item If not found locally, it traverses the hierarchy, querying DNS servers from the root to the authoritative server of "polimi.com".
    \item Finally, it sends an HTTP request to the obtained IP address.
\end{itemize}
When a non-authoritative DNS server receives a resolution request, it may:
\begin{itemize}
    \item Respond from cache if available.
    \item Recursively resolve the name on behalf of the client.
    \item Iteratively provide the authoritative DNS address.
\end{itemize}

\paragraph*{Cache poisoning attack}
Here's how a cache poisoning attack unfolds:
\begin{enumerate}
    \item The attacker initiates a recursive query to the victim DNS server.
    \item The victim DNS server contacts the authoritative server.
    \item Impersonating the authoritative DNS server, the attacker sniffs or guesses the DNS query ID and spoofs the answer.
    \item The victim DNS server, trusting the response, caches the malicious record. 
\end{enumerate}
Subsequently, all clients redirected to the poisoned DNS server are led to the malicious website.

\subsection{Dynamic Host Configuration Protocol}
The Dynamic Host Configuration Protocol (DHCP) is a protocol designed to dynamically allocate IP addresses and network parameters to devices within a network. 
Here's how it functions:
\begin{itemize}
    \item DHCP automatically assigns a new IP address to a computer when it connects to the network.
    \item It enables network administrators to centrally manage and distribute configuration parameters to network hosts, including IP addresses, router information, and subnet masks.
\end{itemize}

\paragraph*{Limitations}
However, DHCP lacks authentication, potentially impacting performance. 
It operates over UDP, and a DHCP server must run continuously. 
If the DHCP server becomes unavailable, clients may lose access to the enterprise network.

\paragraph*{DHCP Operations}
DHCP involves several key operations:
\begin{enumerate}
    \item DHCP DISCOVER: Client requests an IP address.
    \item  DHCP OFFER: DHCP server offers an IP address to the client.
    \item Upon receiving DHCP ACK from the server, the client can begin using the assigned IP address.
    \item DHCP RELEASE: Client voluntarily relinquishes the IP address.
\end{enumerate}

\paragraph*{DHCP poisoning attack}
Since DHCP is unauthenticated, attackers can intercept DHCP requests, respond first, and manipulate client configurations. 
Through spoofed DHCP responses, attackers can set IP addresses, DNS addresses, and default gateways for victim clients, leading to potential network breaches.
\section{Inherent factor}

The most commonly utilized inherent factors encompass Biometric authentication. 

\paragraph*{Advantages}
Inherent factors offer several advantages, including a high level of security and the absence of extra hardware requirements.

\paragraph*{Disadvantages}
However, inherent factors come with notable drawbacks, such as being challenging to deploy, probabilistic matching, invasive measurement techniques, susceptibility to cloning, changes in biometric characteristics over time, privacy concerns, and issues for users with disabilities.

\paragraph*{Countermeasures}
To mitigate these vulnerabilities, potential countermeasures may include regularly re-measuring biometric data, securing the biometric authentication process, and providing alternative authentication methods for users who may face difficulties with biometric authentication.

\subsection{Technologies}
Examples of biometric technology include:
\begin{itemize}
    \item Fingerprint recognition.
    \item Facial geometry analysis.
    \item Hand geometry (palm print) recognition.
    \item Retina scanning.
    \item Iris scanning.
    \item Voice analysis.
    \item DNA analysis.
    \item Typing dynamics analysis.
\end{itemize}

\paragraph*{Fingerprint}
Fingerprint authentication involves several steps. First, during enrollment, a reference sample of the user's fingerprint is acquired by a fingerprint reader. 
From this sample, features are derived, focusing on minutiae such as end points of ridges, bifurcation points, core, delta, loops, and whorls. 
To enhance accuracy, features from multiple fingers and different positions may be recorded.

These extracted feature vectors are then securely stored in a database.

During authentication, when the user attempts to log in, a new reading of the fingerprint is taken. 
The features of this newly captured fingerprint are compared with the reference features stored in the database. 
Access is granted if the similarity between the captured and reference features exceeds a predefined threshold.

However, a main challenge in fingerprint authentication is the occurrence of false positives and false negatives, where the system either incorrectly matches the user's fingerprint or fails to recognize it, respectively.
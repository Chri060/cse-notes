\section{Single sign on}

Single Sign-On (SSO) addresses the complexity of managing and remembering multiple passwords, a common challenge for users. 
The issue often leads to password reuse across different sites, which can compromise security. 
Additionally, replicating password policies across various platforms can be costly and inefficient.

The solution offered by SSO is to establish one identity, typically supported by one or two authentication factors, and designate one trusted host. 
Users authenticate or sign on to this trusted host. 
Then, other hosts can verify a user's authentication status by querying the trusted host. 
This streamlined approach simplifies the authentication process for users while maintaining security standards across multiple platforms.

\paragraph*{Disadvantages}
The drawbacks of Single Sign-On (SSO) include having a single point of trust: the trusted server. 
If this server is compromised, all affiliated sites are compromised as well. 
Additionally, the password reset scheme needs to be flawlessly secure, with email serving as the trusted component.

Developers often find it challenging to implement SSO correctly. 
The process involves a complex flow, and while libraries are available to assist, they may contain bugs that introduce vulnerabilities.

\subsection{Password managers}
Dealing with and recalling numerous passwords poses a complex challenge. 
Users often resort to using the same passwords across various sites, leading to the duplication of password policies.

A potential solution involves adopting a single identity approach, incorporating one or two authentication factors, and relying on a trusted host. 
Once a trusted host is selected, users authenticate themselves on it using a master password.
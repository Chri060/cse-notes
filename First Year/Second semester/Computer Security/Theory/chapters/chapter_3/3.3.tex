\section{Possession-based factor}

The most commonly utilized possession-based factors encompass tokens, smart cards, and smartphones.

\paragraph*{Advantages}
Possession-based offer several advantages, including the human factor (making it less likely to hand out a key), relatively low cost, and a good level of security.

\paragraph*{Disadvantages}
However, passwords come with notable drawbacks, such as being hard to deploy and susceptible to being lost or stolen.

\paragraph*{Countermeasures}
To address these vulnerabilities, potential countermeasures may involve implementing a second factor alongside passwords or exploring alternative authentication methods.

\subsection{Technologies}
The possible solution to implement this solution are: 
\begin{itemize}
    \item One-time password generators operate on the principle of a secret key synchronized with a counter on the host system. 
        The process involves the client computing a Message Authentication Code (MAC) using the counter and key, which is then verified by the host system. 
        The system ensures that the counter matches the expected value, with the counter typically changing every 30 to 60 seconds.
        Examples of applications utilizing one-time password generators include online banking platforms and administrative consoles like Amazon AWS.
    \item Smart cards, including those with embedded readers in USB keys, consist of a CPU and non-volatile RAM housing a private key. 
        In the authentication process, the smart card verifies its identity to the host system through a challenge-response protocol. 
        It utilizes the private key to sign the challenge, ensuring the private key remains secure within the device and does not leave it. 
        Smart cards are designed to be tamper-resistant to some degree.
    \item Static OTP lists consist of sequences known to both the client and the host. 
        The host selects challenges, typically random numbers or specific criteria. 
        The client responds to these challenges, ideally transmitting the response over an encrypted channel for added security. 
        To safeguard the list, the host employs techniques like hashing to avoid storing it in plain text.
    \item Time-based one-time password software replicates the functionality of password generators. 
        However, a key distinction lies in their implementation:
        \begin{itemize}
            \item Password generators are typically closed, embedded systems.
            \item Password-generation apps operate on general-purpose software and hardware platforms.
        \end{itemize}
\end{itemize}
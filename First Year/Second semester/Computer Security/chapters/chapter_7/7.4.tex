\section{Bots}

The rise of bots began with the abuse of IRC bots, notably during the IRCwars, which included one of the first documented DDoS attacks. 
In 1999, the trinoo DDoS attack tool emerged, initially running on Solaris and later ported to Windows.
Setting up botnets with trinoo was mostly a manual process. 
In August 1999, a DDoS attack using at least 227 bots targeted a server at the University of Minnesota. 
The 2000s saw high-profile DDoS attacks against major websites like Amazon, CNN, and eBay, drawing significant media attention.

\begin{definition}[\textit{Botnets}]
    A botnet is a network that consists of several malicious bots controlled by a commander, commonly known as a bot-master or bot-herder.
\end{definition}

Botnets pose various potential threats. 
For the infected host, they can harvest identity, financial, and private data, including email address books and any other type of data present on the victim's machine. 
For the rest of the Internet, botnets can be used for spamming, DDoS attacks, propagation via network or email worms, and supporting illegal internet activities such as hosting phishing sites and drive-by-download sites.

To defend against malware, several mitigation strategies are employed. 
Patching is crucial, as most worms exploit known vulnerabilities, though patches are ineffective against zero-day worms. 
Signature-based detection must be developed automatically because worms spread too quickly for human response.
Intrusion or anomaly detection systems can notice fast-spreading and suspicious activity, potentially driving the automated generation of signatures.
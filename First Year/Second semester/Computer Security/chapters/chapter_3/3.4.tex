\section{Inherent factor}

Inherent factors, most commonly utilized in biometric authentication, offer several advantages, including a high level of security and the absence of extra hardware requirements. 
However, they also have notable drawbacks, such as deployment challenges, probabilistic matching, invasive measurement techniques, susceptibility to cloning, changes in biometric characteristics over time, privacy concerns, and issues for users with disabilities.

To address these vulnerabilities, potential countermeasures may include:
\begin{itemize}
    \item Regularly re-measuring biometric data.
    \item Securing the biometric authentication process.
    \item Providing alternative authentication methods for users who may face difficulties with biometric authentication.
\end{itemize}

\paragraph*{Fingerprint authentication}
Fingerprint authentication involves several steps:
\begin{enumerate}
    \item \textit{Enrollment}: a reference sample of the user's fingerprint is acquired by a fingerprint reader. 
        From this sample, features are derived. 
        These extracted feature vectors are then securely stored in a database.
    \item \textit{Authentication}: a new fingerprint reading is taken. 
        The features of this newly captured fingerprint are compared with the reference features stored in the database. 
        Access is granted if the similarity between the captured and reference features exceeds a predefined threshold.
\end{enumerate}
However, a main challenge in fingerprint authentication is the occurrence of false positives and false negatives.
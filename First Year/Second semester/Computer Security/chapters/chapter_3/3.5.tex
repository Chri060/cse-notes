\section{Single Sign-On}

Single Sign-On (SSO) addresses the complexity of managing and remembering multiple passwords. 
This issue often leads to password reuse across different sites, which can compromise security. 
Additionally, replicating password policies across various platforms can be costly and inefficient.

SSO offers a solution by establishing a single identity, typically supported by one or two authentication factors, and designating one trusted host. 
Users authenticate or sign on to this trusted host, and other hosts can verify a user's authentication status by querying the trusted host. 

The main drawbacks of SSO include having a single point of trust: the trusted server. 
If this server is compromised, all affiliated sites are compromised as well. 
Implementing SSO correctly is often challenging for developers. 
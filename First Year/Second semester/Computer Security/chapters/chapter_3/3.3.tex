\section{Possession-based factor}

Possession-based factors commonly include tokens, smart cards, and smartphones. 
These factors offer several advantages, such as reducing the likelihood of users handing out keys, being relatively low-cost, and providing a good level of security. 
However, they also have drawbacks, including deployment challenges and the risk of being lost or stolen.

To mitigate these vulnerabilities, implementing a second factor alongside passwords or exploring alternative authentication methods can be effective.
Possible solutions include:
\begin{itemize}
    \item \textit{One-Time Password generators} (OTP): these operate on the principle of a secret key synchronized with a counter on the host system. 
        The client computes a MAC using the counter and key, which is then verified by the host system. 
    \item \textit{Smart cards}: these devices contain a CPU and non-volatile RAM housing a private key. 
        During authentication, the smart card verifies its identity to the host system through a challenge-response protocol. 
    \item \textit{Static OTP}: these consist of sequences known to both the client and the host. 
        The host selects challenges, typically random numbers or specific criteria, and the client responds, ideally transmitting the response over an encrypted channel. 
    \item \textit{Time-based OTP}: these replicate the functionality of password generators but differ in implementation. 
        Password generators are typically embedded systems that operate on general-purpose software and hardware platforms.
\end{itemize}
\section{Exercise 5}

Tomb-Bino is a novel malware family that spreads through email attachments, turning infected computers into peers of a command and control infrastructure. 
An attacker can then command an infected computer remotely, for instance, to perform DDoS, steal data, etc.
Its body comprises three functions, which are loaded into memory when the victim clicks on the email attachment: spread, mutate, and sit\_and\_wait.
spread performs the following operations (in order):
\begin{enumerate}
    \item Steals the list of the user's contacts.
    \item Calls mutate.
    \item Creates a new file, new\_attachment.jpeg.exe, whose instructions are dumped from the process memory.
    \item Sends new\_attachment.jpeg.exe to every user in the contact list as an email attachment.
\end{enumerate}
mutate takes as input (start\_address, number\_of\_bytes) and substitutes each instruction found in memory at [start\_address, (start\_address + number\_of\_bytes)] with a semantically equivalent one. 
For instance, instruction sub eax, 0x1 is substituted with add eax, -0x1. 
Substitutions are performed directly to the memory.
Lastly, sit\_and\_wait simply waits for commands from the server and executes them.
The following is a simplified view of the process memory when a Tomb-Bino instance is executed:
\begin{verbatim}
0x00100000 <spread>
(1000 bytes long)
0x001003e8 <sit_and_wait>
(2000 bytes long)
0x00100bb8 <mutate>
(4000 bytes long)
\end{verbatim}
\begin{enumerate}
    \item What type of class does this malware most likely belong to (encrypted, polymorphic, metamorphic, evasive)?
        Motivate your answer. 
    \item Suppose that the headers (i.e., library imports, file details, etc) of each “new\_attachment.jpeg.exe” cannot be used to generate signatures. 
        Moreover, suppose that spread invokes mutate with parameters (0x00100000, 3000).
        Could this family of malware be detected through signatures? 
        If yes, explain why. 
        If not, propose a different detection strategy. 
    \item Given the nature of Tomb-Bino (i.e., a command and control trojan that spreads via email), how would you employ dynamic analysis to detect it? 
        In particular, what kind of resources or operations would you monitor? 
        (Note: a description of command and control malware is provided at the beginning of this section)
\end{enumerate}

\subsection*{Solution}
\begin{enumerate}
    \item Metamorphic: mutate produces semantically equivalent but different code. 
    \item Yes, it can be detected through signatures.
        When the malware mutates (i.e., when the function mutate is invoked), it only changes the first two functions, leaving the mutation engine (i.e., mutate) unchanged. 
        Therefore, an antivirus could use mutate as a target for its signatures.
    \item Email access (retrieve contacts, send emails) and general networking operations (listen on a port, sending/receiving data to specific IPs).
\end{enumerate}
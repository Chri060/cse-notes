\section{Exercise 3}

Our systems have been compromised by a very powerful malware that encrypted all the files in the /home directories of our systems.
Luckily, we succeeded in collecting two samples. 
Luckily, we succeeded in collecting two versions of the same malware. 
The code of the two malware samples is reported below.
\begin{verbatim}
.text
0x08048047<evil>:
0x08048047: push ebp
0x08048049: mov ebp,esp
0x08048050: mov eax,0x8090000
0x08048055: mov ebx,0x8048086
0x0804805a: mov cl,BYTE PTR [eax]
0x0804806c: mov dl,BYTE PTR [ebx]
0x0804806e: cmp cl,0x0
0x08048071: je 0x08048086
0x08048077: xor dl,cl
0x08048079: mov BYTE PTR [ebx],dl
0x0804807b: add eax,0x1
0x0804807e: add ebx,0x1
0x08048081: jmp 0x0804805a
0x08048086: sub DWORD PTR [ebx+0x2e],eax
0x08048089: push DWORD PTR [edi]
0x0804808b: sub ebx,DWORD PTR [eax-0x3f66bf99]
0x08048091: retf
0x08048092: mov bh,0xdf
0x08048094: sub ebp,DWORD PTR [ebx+0x3ef5045b]
0x0804809a: imul ecx,DWORD PTR [eax+0x40],0x31
0x0804809e: and al,0x39
0x080480a0: pop eax
0x080480a1: mov edi,0x5d97e942
0x080480a6: loope 0xffffffb2
0x080480a8: cli
0x080480a9: (bad)
0x080480aa: (bad)
0x080480ab: movs DWORD PTR es:[edi],DWORD PTR ds:[esi]
0x080480ac: shl BYTE PTR [ecx],1
0x080480ae: leave
0x080480af: ret
.rodata
0x08090000: <Data with length 40, null terminated>
\end{verbatim}

\begin{verbatim}
.text
0x08048046<evil>:
0x08048046: push ebp
0x08048048: mov ebp,esp
0x0804804a: mov ebx,0x0804806b
0x0804804f: xor eax,eax
0x08048051: cmp eax,0x28
0x08048054: jge 0x0804806b
0x0804805a: mov cl,BYTE PTR [ebx + eax]
0x0804805d: sub cl,0x3
0x08048060: mov BYTE PTR [ebx + eax],cl
0x08048063: add eax,1
0x08048066: jmp 0x08048051
0x0804806b: fstp QWORD PTR [eax+eiz*2+0x22e70b6d]
0x08048072: jg 0xfffffffa
0x08048074: push ebp
0x08048075: lea ecx,ds:0xf130af40
0x0804807b: fisub WORD PTR [edx-0x703c814d]
0x08048081: mov eax,edi
0x08048083: or eax,DWORD PTR ds:0xd19fe443
0x08048089: bswap esp
0x0804808b: stos DWORD PTR es:[edi],eax
0x0804808c: sbb eax,DWORD PTR [eax+0x50]
0x0804808f: fimul WORD PTR [esp+edi*4]
0x08048092: (bad)
0x08048093: leave
0x08048094: ret 
\end{verbatim}
\begin{enumerate}
    \item It is clear that the malware is showing evasive behavior.
        What technique is implemented? Explain how it is implemented in this specific case. 
        If this malware is metamorphic, explain which instructions implement the obfuscation functionality.
        If this malware is evasive, explain how it is evading the environment. 
        If this malware is polymorphic, describe how the decryption routine is implemented. 
    \item We executed the two samples in a Virtual Environment. 
        Tracing the system calls of two different executions with strace yields the following outputs (trimmed for readability):
        \begin{verbatim}
...
open(“/home/law/.bashrc”, O_RDONLY) = 3
fstat(3, {st_mode=S_IFREG|0644, st_size=3979, ...}) = 0
read(3, 0x7f508000, 3979) = 3979
close(3) = 0
open(“/home/law/.bashrc”, O_WRONLY|O_TRUNC) = 3
write(3, 0x7f508000, 3984) = 3984
close(3) = 0
...
(similar lines as above)
...
exit(0)
        \end{verbatim}
        \begin{verbatim}
...
sleep(0x1000)
exit(0)
        \end{verbatim}
        What kind of technique does the malware employ? Is such a technique adopted by both samples? How can you spot it in both cases?
    \item Is it possible to analyze this malware through static analysis? 
        Why? 
        Which instructions would you consider for fingerprint based detection? 
        Do you expect this would be effective?
 \end{enumerate}
\subsection*{Solution}
\begin{enumerate}
    \item Polymorphism, the two malware samples employ an OTP (1st sample) and a shift-cipher (2nd sample) with $K = 3$ (Caesar's Cipher) to decrypt/unpack the malicious code.
        Can't say if the malware is evasive or not since some part of the malware is packed.
    \item The malware employs an evasive technique called dormant code.
        Since it detects that it is running in a VM, it doesn't show its malicious behavior. 
        Such a technique is performed with a sleep of 0x1000 seconds.
        Only the second sample employs dormant code by performing a sleep syscall, clearly visibile in the strace of its execution.
    \item No, because we only see the decryption routines.
        No relevant instructions can be considered for a fingerprint based detection since the ones of the decryption routines are too generic and would lead to many false positives.
\end{enumerate}
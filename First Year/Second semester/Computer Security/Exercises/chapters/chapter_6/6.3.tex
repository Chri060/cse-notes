\section{Exercise 3}

A web application contains three pages to handle login, post comments, and read comments, all served over a secure HTTPS connection.
Here you can find code snippet of these pages:
Show comments
\begin{verbnobox}[\verbarg]
var id = request.get['id'];
var prep_query = prepared_statement("SELECT username FROM users WHERE id=? LIMIT 1");
var username = query(prep_query, id);
var prep_query = prepared_statement("SELECT * FROM comments WHERE username=?");
var comments = query(prep_query, username);
for comment in comments {
    echo htmlentities(comment);
}   
\end{verbnobox}
Login 
\begin{verbnobox}[\verbarg]
var password = md5(request.post['password']);
var username = request.post['username'];
var prep_query = prepared_statement("SELECT username FROM users WHERE username=? AND password=? LIMIT 1");
var result = query(prep_query, username, password);
if (result) {
    session.set('username', username);
    echo "Logged in.";
} else {
    echo "User" + username + "does not exists!";
}
\end{verbnobox}
Write comment
\begin{verbnobox}[\verbarg]
var username = session.get['username']; // You need to be logged in
var comment = request.get['comment'];
var res = query("INSERT INTO comments (username, comment, timestamp) VALUES ( “ + username + ” , “+ comment + ” , NOW())");
    echo "Comment saved.";
\end{verbnobox}
Assume the following:
\begin{itemize}
    \item The framework used to develop the web application securely and transparently manages the users' sessions through the object session.
    \item The dictionaries request.get and request.post store the content of the parameters passed through a GET or POST request respectively. 
    \item The function htmlentities() converts special characters such as <, >, ", and ' to their equivalent HTML entities (i.e., \&lt;, \&gt;, \&quot; and \&apos; respectively).
\end{itemize}
As it is clear from the code, this application uses a database to store data.
These are tables of the database:
\begin{itemize}
    \item Users: Id, Name, Password. 
    \item Comments: Id, User, Comment, Timestamp.
\end{itemize}
\begin{enumerate}
    \item Only considering the code above, identify which of the following classes of web application vulnerabilities are present. 
    \item Exploiting one of the vulnerability detected before, write down an exploit to get the hash of the password of admin. 
        You must also specify all the steps and assumptions.
    \item You are the database administrator and have no way to modify the above code. 
        How would you mitigate the damage that an attacker can do?
\end{enumerate}

\subsection*{Solution}
\begin{enumerate}
    \item The vulnerabilities in the given code are: 
        \begin{itemize}
            \item Reflected XSS on line ten of the second code. 
                An adversary can set up a form (hidden form) that submits a request with an username containing a malicious script e.g.,
\begin{verbatim}
<script>alert(document.cookie)</script>
\end{verbatim}
                and the web server would print that script tag to the browser, and the browser will run the code.
                The simplest procedure to prevent this vulnerability is to apply escaping/filtering to the “username” variable. 
            \item CSRF: on lines zero and four of the third code. 
                An adversary can set up a form that submits a request to send a message, as this request will be honored by the server. 
                To solve this problem, include a CSRF token with every legitimate request, and check that cookie['csrftoken']==param['csrftoken']. 
            \item SQL injection: on lines three and four of the third code. 
                The simplest procedure to prevent this vulnerability is to apply escaping/filtering to the “comment/username” variable.
        \end{itemize}
    \item The query is: 
\begin{verbatim}
... comment = '( SELECT password from users where name =“admin”)
\end{verbatim}
    \item As this page/application needs only to read data from the users table, we could restrict, at the database level, the privileges of the user of this application to only perform SELECTs involving the user table (and no operation involving the account\_balance table).
\end{enumerate}
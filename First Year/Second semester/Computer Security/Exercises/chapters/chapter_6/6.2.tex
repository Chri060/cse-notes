\section{Exercise 2}

A new malware just broke out, causing a world-wide infection and a huge amount of damages. 
Unfortunately, all the anti-malware systems are not able to detect this malware. 
You were able to retrieve a couple of samples.
Consider the code snippets reported below, extracted from the two malware samples you retrieved:
\begin{verbatim}
0000000000000675 <decrypt>:
[...]
    6a3: 83 f1 42 xor ecx,0x42
[...]
00000000000007b0 <payload>:
    7b0: 28 00 sub BYTE PTR
[rax],al
    7b2: 1a bc 86 0a db 10 0a sbb bh,BYTE PTR
[rsi+rax*4+0xa10db0a]
    7b9: fd std
    7ba: 6d ins DWORD PTR
es:[rdi],dx
    7bb: 20 2b and BYTE PTR
[rbx],ch
    7bd: 2c 6d sub al,0x6d
    7bf: 6d ins DWORD PTR
es:[rdi],dx
    7c0: 31 2a xor DWORD PTR
[rdx],ebp
    7c2: 15 16 1c 0b cb adc eax,0xcb0b1c16
    7c7: 92 xchg edx,eax
    7c8: 0b cb or ecx,ebx
    7ca: 90 nop
    7cb: 4d rex.WRB
    7cc: 47 rex.RXB
\end{verbatim}

\begin{verbatim}
0000000000000675 <decrypt>:
[...]
    6a3: 83 f1 42 xor ecx,0x12
[...]
00000000000007b0 <payload>:
    7b0: 78 50 js 802
<__GNU_EH_FRAME_HDR+0x32>
    7b2: 4a ec rex.WX in al,dx
    7b4: d6 (bad)
    7b5: 5a pop rdx
    7b6: 8b 40 5a mov eax,DWORD PTR
[rax+0x5a]
    7b9: ad lods eax,DWORD PTR
ds:[rsi]
    7ba: 3d 70 7b 7c 3d cmp eax,0x3d7c7b70
    7bf: 3d 61 7a 45 46 cmp eax,0x46457a61
    7c4: 4c 5b rex.WR pop rbx
    7c6: 9b fwait
    7c7: c2 5b 9b ret 0x9b5b
    7ca: c0 .byte 0xc0
    7cb: 1d .byte 0x1d
    7cc: 17 (bad) 
\end{verbatim}
\begin{enumerate}
    \item It is clear that the malware is showing evasive behavior. 
        What technique is implemented? How this technique works?
\end{enumerate}
In order to avoid signature detection, a malware sample saves his own assembly code in text format on the victim machine, and then uses a standard assembler to generate and execute the real malicious object code on the machine. 
\begin{enumerate}
    \item [2. ] How can a signature-based detection method (e.g., antivirus) detect this kind of malware ? 
    \item [3. ] You suspect that your machine have been compromised with a kernel rootkit. 
        You tried to use network traffic tools from your machine but you do not see any malicious traffic. 
        Can you conclude that your machine is safe? If is not there are other way to prove you have been compromised?
    \item [4. ] A colleague suggests to replace the hard drive of a machine to be sure to get rid of a very sophisticated rootkit.
        However, after reinstalling the operating system, it seems like that the machine is infected by the same rootkit. 
        Provide an explanation of what happened. 
        Whatever your answer is, explain why.
\end{enumerate}

\subsection*{Solution}
\begin{enumerate}
    \item Polymorphism. 
    \item Have a signature to detect the malware in its textual assembly format. 
        Note that having a signature to detect the assembler is a wrong solution, as it leads to lots of false positives (the system assembler it's a legitimate program, after all).
    \item No you cannot conclude that the machine have not been compromised. 
        Because the malware can hide its own traffic from tools running on the compromised machine. 
        You could inspect network traffic using an external machine as a MitM between your machine and the router.
    \item If it is a BIOS rootkit then no. 
        If it is a kernel rootkit it is ok to just replace the HD or even just reinstall the OS.
\end{enumerate}

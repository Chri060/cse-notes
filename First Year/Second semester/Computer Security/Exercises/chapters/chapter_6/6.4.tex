\section{Exercise 4}

Last week, we succeeded in blocking a new malware able to cause a world-wide infection and a huge amount of damage.
Consider the code snippets reported below, extracted from the malware sample you retrieved:
\begin{verbatim}
08048454 <main>:
    8048454: 8d 4c 24 04 lea ecx,[esp+0x4]
    8048458: 83 e4 f0 and esp,0xfffffff0
    804845b: ff 71 fc push DWORD PTR [ecx-0x4]
    804845e: 55 push ebp
    804845f: 89 e5 mov ebp,esp
    8048461: 51 push ecx
    8048462: 83 ec 04 sub esp,0x4
    8048465: 83 ec 0c sub esp,0xc
    8048468: 68 00 01 00 00 push 0x1000
    804846d: e8 8e fe ff ff call 8048300 <sleep@plt>
    8048472: 83 c4 10 add esp,0x10
    8048475: e8 c1 ff ff ff call 804843b <evil>
    804847a: b8 00 00 00 00 mov eax,0x0
    804847f: 8b 4d fc mov ecx,DWORD PTR [ebp-0x4]
    8048482: c9 leave
    8048483: 8d 61 fc lea esp,[ecx-0x4]
    8048486: c3 ret 
\end{verbatim}
\begin{enumerate}
    \item It is clear that the malware is showing evasive behavior.
        What technique is implemented? 
        Explain how it works in this specific case. 
\end{enumerate}
Today, a new version of the same malware has been released.
Unfortunately, all the anti-malware systems are not able to detect it.
You were able to retrieve a couple of samples. 
Consider the code snippets reported below, extracted from the two malware samples you retrieved:
\begin{verbatim}
08048340 <main>:
    8048340: 8d 4c 24 04 lea ecx,[esp+0x4]
    8048344: 83 e4 f0 and esp,0xfffffff0
    8048347: ff 71 fc push DWORD PTR [ecx-0x4]
    804834a: 55 push ebp
    804834b: 89 e5 mov ebp,esp
    804834d: 51 push ecx
    804834e: 83 ec 10 sub esp,0x10
    8048351: 21 c0 and eax,eax
    8048353: 90 nop
    8048354: 40 inc eax
    8048355: 09 c0 or eax,eax
    8048357: 48 dec eax
    8048358: 6a 01 push 0x1000
    804835a: e8 a1 ff ff ff call 8048300 <sleep@plt>
    804835f: 21 c0 and eax,eax
    8048361: 90 nop
    8048362: 40 inc eax
    8048363: 09 c0 or eax,eax
    8048365: 48 dec eax
    8048366: e8 15 01 00 00 call 8048480 <evil>
    804836b: 21 c0 and eax,eax
    804836d: 90 nop
    804836e: 40 inc eax
    804836f: 09 c0 or eax,eax
    8048371: 48 dec eax
    8048372: 8b 4d fc mov ecx,DWORD PTR [ebp-0x4]
    8048375: 83 c4 10 add esp,0x10
    8048378: 31 c0 xor eax,eax
    804837a: c9 leave
    804837b: 8d 61 fc lea esp,[ecx-0x4]
    804837e: c3 ret
\end{verbatim}

\begin{verbatim}
08048443 <main>:
    8048443: 8d 4c 24 04 lea ecx,[esp+0x4]
    8048447: 83 e4 f0 and esp,0xfffffff0
    804844a: ff 71 fc push DWORD PTR [ecx-0x4]
    804844d: 55 push ebp
    804844e: 89 e5 mov ebp,esp
    8048450: 51 push ecx
    8048451: 83 ec 04 sub esp,0x4
    8048454: 68 00 00 01 00 push 0x10000
    8048459: e8 c6 ff ff ff call 8048424 <what_am_i_doing>
    804845e: 83 c4 04 add esp,0x4
    8048461: e8 a5 ff ff ff call 804840b <evil>
    8048466: b8 00 00 00 00 mov eax,0x0
    804846b: 8b 4d fc mov ecx,DWORD PTR [ebp-0x4]
    804846e: c9 leave
    804846f: 8d 61 fc lea esp,[ecx-0x4]
    8048472: c3 ret
08048424 <what_am_i_doing>:
    8048424: 55 push ebp
    8048425: 89 e5 mov ebp,esp
    8048427: 83 ec 10 sub esp,0x10
    804842a: c7 45 fc 00 00 00 00 mov DWORD PTR [ebp-0x4],0x0
    8048431: eb 05 jmp 8048438 <what_am_i_doing+0x14>
    8048433: 90 nop
    8048434: 83 45 fc 01 add DWORD PTR [ebp-0x4],0x1
    8048438: 8b 45 fc mov eax,DWORD PTR [ebp-0x4]
    804843b: 3b 45 08 cmp eax,DWORD PTR [ebp+0x8]
    804843e: 7c f3 jl 8048433 <what_am_i_doing+0xf>
    8048440: 90 nop
    8048441: c9 leave
    8048442: c3 ret
\end{verbatim}
\begin{enumerate}
    \item [2. ] It is clear that the malware is showing evasive behavior.
        What technique/s is/are implemented?
        Explain how it works in these specific cases. 
    \item [3. ] Consider now malware analysis techniques. 
        Briefly discuss which is the best technique to detect the malware behavior implemented in the previous samples. 
        Motivate your answer by also providing the information you are able to extract in the case of this specific malware.
\end{enumerate}

\subsection*{Solution}
\begin{enumerate}
    \item Dormant code and then a call to the evil function is performed. 
    \item Metamorphism to evade static analysis.
        1st snippet: useless instructions added (nop; and eax, eax;).
        2nd snippet: sleep substituted by a cycle that does not do anything (nop).
    \item Static analysis for the dormant code; the function sleep@plt easy to spot through static analysis.
        It is harder to spot loops that act as sleeps. 
        No signature can be extracted. 
        Perhaps, detection of loops that iterates many times.
        Dynamic analysis can be tried. 
        For instance, waiting for malware behavior over a fixed period of time.
\end{enumerate}
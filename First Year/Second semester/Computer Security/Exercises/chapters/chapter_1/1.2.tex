\section{Exercise two}

Consider a scenario involving a self-driving, internet-connected vehicle operating within a taxi service context:
\begin{enumerate}
    \item Identify the three most valuable assets at risk in this scenario.
    \item Suggest at least two potential attack surfaces on the vehicles.
    \item Provide, in rough order of prevalence, the two most likely potential digital attacks against such vehicles and their operating companies.
\end{enumerate}

\subsection{Solution}
\begin{enumerate}
    \item The valuable assets at risk are: passengers inside the car, pedestrians and other individuals outside the car, and the vehicle itself.
    \item Potential attack surfaces include:
        \begin{itemize}
            \item The Controller Area Network (CAN) bus via the diagnostic port, which may be vulnerable to manipulation.
            \item The remote interface to the car, which could be exploited if not properly secured.
        \end{itemize}
    \item Likely potential digital attacks include:
    \begin{itemize}
        \item Local attack: an attacker inside the car may manipulate the packets transmitted on the CAN bus via the diagnostic port to gain control of the vehicle.
        \item Remote attack: an attacker could manipulate communication between the car and the backend systems, potentially diverting the car to a different location or disrupting its normal operation.
        \end{itemize}
\end{enumerate}
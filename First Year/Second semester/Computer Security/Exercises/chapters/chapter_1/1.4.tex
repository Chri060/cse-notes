\section{Exercise four}

Consider the SmartCar device, a new plug-in device designed to monitor driving habits, patterns, and the location of a car via a smartphone application.
All modern cars are equipped with an internal wired network that connects together all the electronic control units. 
This network is used to exchange commands and data, including safety-related ones. 
This network is based on the standard known as CAN (Controller Area Network): all messages are broadcast to all control units connected to the network, are not encrypted, and their sender is not authenticated. 
In order to gather information about how the vehicle is driven, SmartCar must be physically connected to the car's internal CAN network, where it actively exchanges messages with the car's control units in order to gather the required data.
Furthermore, to display real-time data, SmartCar is connected via Bluetooth to the vehicle owner's smartphone, and sends information about the vehicle's location to a remote server over a cellular network (3G or 4G), so that the vehicle's owner can constantly track its movements—for instance to remotely locate the vehicle in case of theft.
Consider the following scenario: a vehicle owner installs SmartCar in their car.
\begin{enumerate}
    \item Identify the most valuable assets at risk in this scenario.
    \item Suggest at least two potential attack surfaces of the SmartCar device.
    \item Suggest potential digital attacks in this scenario.
\end{enumerate}

\subsection{Solution}
\begin{enumerate}
    \item The most valuable assets at risk in this scenario include:
        \begin{itemize}
            \item Life/Health of individuals: safety of people inside and around the car is paramount.
            \item Owner's private driving data: confidential driving habits and patterns.
            \item Device vendor/car manufacturer reputation: reputation of the device vendor and car manufacturer.
            \item Vehicle: physical integrity and functionality of the vehicle.
            \item Smartphone: security and privacy of the owner's smartphone.
        \end{itemize}
    \item  Potential attack surfaces of this smart speaker:
        \begin{itemize}
            \item Smartphone application: vulnerabilities in the application used to interact with the device.
            \item Company's backend: weaknesses in the backend infrastructure and services.
            \item Physical access to the vehicle: unauthorized access to the vehicle's physical components.
            \item Bluetooth/cellular network: vulnerabilities in the communication channels used by the device.
        \end{itemize}
    \item The most likely potential digital attacks in this scenario are:
        \begin{itemize}
            \item Compromise of company's backend: attackers may breach the company's backend to access user data and compromise safety by re-flashing the device or sending unauthorized data within the network.
            \item Physical compromise of device: attackers could physically compromise the device to send remote commands to the vehicle.
            \item Compromise of application: attackers may target the application to access user data or gather real-time data on specific users.
        \end{itemize}
\end{enumerate}
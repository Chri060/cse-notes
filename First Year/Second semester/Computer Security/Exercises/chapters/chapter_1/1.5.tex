\section{Exercise five}

Consider object tracking devices, such as those developed by Apple or Tile, designed to assist in locating personal items like keys, bags, and electronic devices. 
These devices utilize a smartphone app and a crowdsourced network of devices emitting Bluetooth Low Energy 4.0 signals for location tracking. 
If reported as lost and detected by nearby smartphones running the tracking app, the device's location is anonymously updated for the owner. 
The devices also include features such as a built-in speaker for close-range sound alerts and a "Find My Phone" function to locate paired smartphones. 
They typically have a battery life of about one year, with easily replaceable batteries.
\begin{enumerate}
    \item Identify the main assets at risk in this scenario. Suggest at least two assets.
    \item Provide, in rough order of prevalence, the most likely potential security threats against such infrastructure and their operating companies.
    \item Suggest, in rough order of prevalence, the most likely potential security threat agents against such infrastructure and their operating companies.
    \item Recommend, in rough order of prevalence, potential security solutions to counter the identified threats and threat agents.
\end{enumerate}

\subsection*{Solution}
\begin{enumerate}
    \item The primary assets at risk in this scenario include:
        \begin{itemize}
            \item Personal information and location data: users rely on tracking devices to locate lost items, potentially sharing personal information and location data with the device's infrastructure and operating companies, as well as other users in the crowdsourced network. 
                This data could be vulnerable to security breaches or mishandling by the company.
            \item Physical assets: the tangible items being tracked, such as keys, bags, apparel, small electronic devices, and vehicles, are also at risk if lost or stolen, despite the assistance of tracking devices.
            \item Network and infrastructure: the tracking devices depend on a network of smartphones and a centralized infrastructure for locating lost items, which could be compromised by cyberattacks or other security breaches.
            \item Business reputation: failure to protect users' personal information and location data, or performance issues with the tracking devices, could lead to negative publicity and damage the company's reputation.
        \end{itemize}
    \item The most likely potential security threats against such infrastructure and their operating companies include:
        \begin{itemize}
            \item Privacy concerns: users may have concerns about privacy as their location data is shared anonymously with other users via the crowdsourced network.
            \item Security breaches: the infrastructure and operating companies could be vulnerable to security breaches, exposing personal information and location data of users.
            \item Physical tampering and theft: tracking devices themselves could be tampered with or stolen, compromising personal information and location data.
            \item Malware and cyberattacks: infrastructure and operating companies may face attacks that compromise personal information and location data, disrupting services.
            \item Denial of service: infrastructure and operating companies may be targeted with denial-of-service attacks, disrupting the service and preventing users from locating lost items.
        \end{itemize}
    \item The most likely potential security threat agents against such infrastructure and their operating companies include:
        \begin{itemize}
            \item Hackers and cybercriminals: individuals or groups may attempt to gain unauthorized access to the network and infrastructure to steal or misuse personal information and location data.
            \item Insider threats: current or former employees with access to sensitive information may misuse it for personal gain or to disrupt the service.
            \item State-sponsored actors: nation-states or agents may target the companies for political or strategic reasons.
            \item Competitors: other companies in the same industry may seek to gain an advantage by stealing proprietary technology or information.
        \end{itemize}
    \item The most likely potential security solutions to counter these threats and threat agents include:
        \begin{itemize}
            \item Encryption: encrypting personal information and location data in transit and at rest can protect it from unauthorized access.
            \item Multifactor authentication: implementing multifactor authentication, such as using passwords and biometric factors, can ensure only authorized users access personal information and location data.
        \end{itemize}
\end{enumerate}
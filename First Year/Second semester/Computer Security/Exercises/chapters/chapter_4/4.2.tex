\section{Exercise 2}

“SHIPSTUFF” is a new online service that allow registered users to send stuff to other registered users by filling a form. 
The form must contain the product\_id of the product to send and the receiver\_id of the receiver. 
After clicking on the submit button, the web browser will make the following GET request to the web server:
\begin{verbatim}
https://shipstuff.org/ship?product_id=<product_id>&receiver_id=<receiver_id>
\end{verbatim}
The Python-like pseudocode that will handle the shipment is the following:
\begin{verbnobox}[\verbarg]
def ship_stuff(request):
    # code to send HTTP header (not relevant) 
    user = check_cookie(request.cookie)
    if user is None:
        print "Please log in first"
        return

    product_id = request.params['product_id'] # GET parameter
    receiver_id = request.params['receiver_id'] # GET parameter

    query1 = 'SELECT p_id, product_name FROM warehouse, ownership WHERE p_id = ' + product_id + 'AND user.id = ownership.u_id ' + 'AND ownership.p_id = warehouse.p_id;'
    db.execute(query)
    row = db.fetchone()
    if row is None:
        print "Product", product_id, "is not existent"
        return

    query2 = 'SELECT u_id, username FROM accounts WHERE u_id = ' + receiver_id + ';'
    db.execute(query)
    row = db.fetchone()
    if row is None:
        print "User", receiver_id, "is not existent"
        return
    # code to actually send the product and print the product name 
\end{verbnobox}
The above code checks if the user is logged in using the function check\_cookie, which returns the username of the authenticated user checking the session cookie. 
Then, the code attempts to retrieve the product\_id or the receiver\_id from the database and, if they cannot be located the page will contain an error message.
Assume that request.params['product\_id'] and request.params[receiver\_id'] are controllable by the user, and that all the functionalities concerning the user authentication (i.e., check\_cookie) are securely implemented and do not contain vulnerabilities.

Now assume that SHIPSTUFF executes all the database queries against the following tables:
\begin{itemize}
    \item Accounts: UserID, Username, Password. 
    \item Ownership: UserID, ProductID. 
    \item Warehouse: ProductID, ProductName.
\end{itemize}
\begin{enumerate}
    \item Only considering the code above, identify which of the following classes of web application vulnerabilities are present. 
    \item Write an exploit for one of the vulnerability/ies just identified to get the username and the password of the only user that owns the product excalibur.
    By assuming that products are unique, state all the necessary steps and conditions for the exploit to take place.
\end{enumerate}

\subsection*{Solution}
\begin{enumerate}
    \item We have the following vulnerabilities: 
        \begin{itemize}
            \item Reflected XSS: an attacker can supply a product\_id or the receiver\_id containing e.g.,
\begin{verbatim}
<script>alert(document.cookie)</script>
\end{verbatim}
                and the web server would print that script tag to the browser, and the browser will run the code from the URL.
                The simplest procedure to prevent this vulnerability is to apply escaping/filtering to the vulnerable variable. For example: \texttt{product\_id}. 
            \item CSFR: an attacker can send a link to a victim and let the victim ship a product to him by just visiting the link. 
                The simplest procedure to prevent this vulnerability is to apply CSRF token.
            \item SQL injection: user-controlled data is concatenated a query, allowing an attacker to modify such query.
                The simplest procedure to prevent this vulnerability is to apply prepared statement
        \end{itemize}
    \item We need to modify the queries as follows: 
\begin{verbatim}
0 AND 0=1 UNION SELECT a.u_id, a.password FROM accounts AS a,
ownership AS o, products AS p WHERE o.u_id = a.u_id AND o.p_id 
= p.p_id AND p.product_name = 'excalibur';--

0 AND 0=1 UNION SELECT a.u_id, a.username FROM accounts AS a,
ownership AS o, products AS p WHERE o.u_id = a.u_id AND o.p_id 
= p.p_id AND p.product_name = 'excalibur';--
\end{verbatim}
\end{enumerate}
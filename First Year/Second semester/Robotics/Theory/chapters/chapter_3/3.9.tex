\section{Inverse kinematics}

When faced with a desired position or velocity, several strategies can be employed to achieve it.
While it's relatively straightforward to find some solution, determining the best solution can pose significant challenges. 
This best solution could be defined by various criteria such as:
\begin{itemize}
    \item Shortest time to reach the goal.
    \item Most energy-efficient path or trajectory.
    \item Smoothest velocity profiles for comfortable operation.
\end{itemize}
Furthermore, if we encounter non-holonomic constraints and are limited to just two control variables, it becomes impossible to directly reach any of the three degrees of freedom final positions.

\subsection{Differential drive}
To tackle the problem effectively, we can decompose it and focus on controlling only a few degrees of freedom at a time:
\begin{enumerate}
    \item Begin by turning the robot so that the wheels align parallel to the line between its original and final positions:
    \[-v_L(t)=v_R(t)=v_{max}\]
    \item Proceed to drive straight until the robot's origin coincides with the destination:
    \[v_L(t)=v_R(t)=v_{max}\]
    \item Finally, rotate again to achieve the desired final orientation:
    \[-v_L(t)=v_R(t)=v_{max}\]
\end{enumerate}

\subsection{Synchro drive}
To address the challenge systematically, we can break it down and manage only a select few degrees of freedom at each stage:
\begin{enumerate}
    \item Initiate a turn to align the wheels parallel to the line connecting the robot's original and final positions:
    \[\omega(t)=\omega_{max}\]
    \item Proceed to drive straight until the robot's origin reaches the destination:
    \[v(t)=v_{max}\]
    \item Rotate once more to achieve the desired final orientation:
    \[\omega(t)=\omega_{max}\]
\end{enumerate}
\section{Landmark model}

Landmark sensors provide information on distance, bearing, or both. 
These measurements can be obtained through active beacons such as radio or GPS, or passive methods like visual or retro-reflective techniques.
The standard approach for utilizing this data is triangulation.

Explicitly incorporating uncertainty into sensing processes is crucial for ensuring robustness:
\begin{enumerate}
    \item Establish a parametric model for noise-free measurements.
    \item Analyze sources of noise, such as distance and angle.
    \item Introduce appropriate noise to parameters, possibly incorporating densities for noise distribution.
    \item Learn and verify parameters by fitting the model to empirical data.
\end{enumerate}
The likelihood of a measurement is determined by probabilistically comparing actual measurements with expected ones.

\subsection{Landmark detection}
The measurement $z = (i, d, \alpha)$ for a robot positioned at $(x, y, \theta)$, relative to landmark $i$ on map $m$ (denoted as $m_i$), is expressed as follows:
\[\hat{d}=\sqrt{{\left(m_x(i)-x\right)}^2+{\left(m_y(i)-y\right)}^2}\]
\[\hat{\alpha}=\arctan\left[2(m_y(i)-y,m_x(i)-x)-\theta\right]\]
The detection probability of a particular landmark may rely on either the distance or the bearing:
\[\text{P}_{\text{det}}=\text{P}(\hat{d}-d,\varepsilon_d)\text{P}(\hat{\alpha}-\alpha,\varepsilon_\alpha)\]
Additionally, consideration for false positives is necessary:
\[z_{\text{det}}\text{P}_{\text{det}}+z_{\text{fp}}\text{P}_{\text{uniform}}(z|x,m)\]
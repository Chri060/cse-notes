\section{Scan sensor model}

The beam sensor model, while assuming independence between beams and the physical causes of measurements, exhibits several issues:
\begin{itemize}
    \item It tends to be overconfident due to its independence assumptions.
    \item Parameters need to be learned from data, adding complexity.
    \item A distinct model is required for different angles relative to obstacles, leading to increased complexity.
    \item It's inefficient as it relies on ray tracing for calculations.
\end{itemize}
To address these challenges, the scan sensor model simplifies the beam sensor model by:
\begin{itemize}
    \item Utilizing a Gaussian distribution with the mean set at the distance to the closest obstacle.
    \item Employing a uniform distribution for random measurements.
    \item Incorporating a small uniform distribution for maximum range measurements.
\end{itemize}
In this model, we calculate the likelihood of encountering an obstacle along the trajectory of the ray.

From the occupancy grid map, that is the real map with the real obstacles, we can compute the likelihood field.

The likelihood field enables matching of various scans (except for sonars). 
It operates highly efficiently, relying solely on 2D tables. 
It maintains smoothness concerning minor shifts in robot position, facilitating gradient descent pose optimization.
However, it disregards the physical attributes of beams.

\paragraph*{Summary}
In highly dynamic environments, opting for the beam sensor model is crucial as it accounts for temporary obstacles, unlike the scan model. 
Conversely, in static environments, this approach proves faster due to precomputed maps.
\section{Middlewares}

Several middleware have been developed in recent years:
\begin{itemize}
    \item \textit{OROCOS}: originating in December 2000 as an initiative of the EURON mailing list, OROCOS evolved into a European project with three partners: K.U. Leuven (Belgium), LAAS Toulouse (France), and KTH Stockholm (Sweden).
        The requirements for OROCOS encompass open source licensing, modularity, flexibility, independence from specific robot industries, compatibility with various devices, software components covering kinematics, dynamics, planning, sensors, and controllers, and a lack of dependency on a singular programming language.
        The structure includes:
        \begin{itemize}
            \item Real-Time Toolkit (RTT): offering infrastructure and functionalities tailored for real-time robot systems and component-based applications.
            \item Component Library (OCL): supplying ready-to-use components such as drivers, debugging tools, path planners, and task planners.
            \item Bayesian Filtering Library (BFL): featuring an application-independent framework encompassing (Extended) Kalman Filter and Particle Filter functionalities.
            \item Kinematics and Dynamics Library (KDL): facilitating real-time computations for kinematics and dynamics.
        \end{itemize}
    \item \textit{ORCA}: the project's objective is to emphasize software reuse in both scientific and industrial applications. 
        Its key properties include enabling software reuse through the definition of commonly-used interfaces, simplifying software reuse via high-level libraries, and encouraging software reuse through regularly updated software repositories. 
        ORCA defines itself as an "unconstrained component-based system."
        
        The primary distinction between OROCOS and ORCA lies in their communication toolkits. 
        OROCOS utilizes CORBA, whereas ORCA employs ICE. 
        ICE, a contemporary framework created by ZeroC, functions as an open-source commercial communication system. 
        ICE offers two fundamental services: the IceGrid registry (Naming service), which facilitates logical mapping between various components, and the IceStorm service (event service), which forms the foundation for publisher-subscriber architecture.
    \item \textit{OpenRTM}: RT-Middleware (RTM) serves as a widely adopted platform standard for assembling robot systems by integrating software modules known as robot functional elements (RTCs). 
        These modules include components such as camera, stereo vision, face recognition, microphone, speech recognition, conversational, head and arm, and speech synthesis. 
        OpenRTM-AIST (Advanced Industrial Science and Technology) is built upon CORBA technology to realize the extended specifications of RTC implementation.
    \item \textit{BRICS}: the objective is to uncover the most effective strategies for developing robotic systems by examining several critical areas:
        Initially, by thoroughly examining the weaknesses found in current robotic projects.    
        Subsequently, by delving into the integration between hardware and software components within these systems.    
        Thirdly, by advocating for the incorporation of model-driven engineering principles in the development process.    
        Moreover, by creating a tailored Integrated Development Environment (IDE) specifically for robotic projects, known as BRIDE.    
        Finally, by defining benchmarks aimed at evaluating the strength and effectiveness of projects based on BRICS principles.
    \item \textit{ROS}: introduced by Willow Garage in 2009, the Robot Operating System (ROS) serves as a meta-operating system tailored for robotics, boasting a diverse ecosystem replete with tools and programs.
        ROS has expanded to encompass a vast global community of users. 
        The developer community stands out as one of ROS's most significant features.
\end{itemize}  
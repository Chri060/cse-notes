\section{Mapping}

Mapping in ROS is accomplished using \texttt{gmapping}, a ROS wrapper for the OpenSLAM GMapping algorithm. 
This Simultaneous Localization and Mapping algorithm facilitates real-time map creation and localization using laser scans and odometry data.

\paragraph*{Requirements}
To utilize \texttt{gmapping}, the following are required:
\begin{itemize}
    \item Odometry data. 
    \item A horizontally-mounted, fixed laser range-finder
    \item A complete tf tree that includes base to laser transformation, and base to odometry transformation.
\end{itemize}

\paragraph*{Usage}
To generate a map using \texttt{gmapping}, follow these steps:
\begin{enumerate}
    \item Drive your robot around the environment.
        \begin{itemize}
            \item Explore the entire area you want to map.
            \item Collect as much data as possible.
            \item Create loops to provide the algorithm with reference points.
        \end{itemize}
    \item Save the collected data in a ROS bag file.
    \item Play back the bag file.
    \item Start \texttt{gmapping} to process the data.
    \item Save the generated map.
\end{enumerate}

\paragraph*{Bag and real time}
Using bag files:
\begin{itemize}
    \item Processing is faster since you can use pre-collected data.
    \item Allows for multiple trials and parameter tuning.
    \item Useful for experimenting with different settings without needing to rerun the physical robot.
\end{itemize}
Processing in real-time:
\begin{itemize}
    \item Immediate feedback allows for early stopping if issues arise.
    \item  Ability to restart quickly in case of problems.
    \item Direct visualization of results ensures complete coverage of the mapped area.
\end{itemize}
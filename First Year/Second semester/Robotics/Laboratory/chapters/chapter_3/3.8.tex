\section{Services}

To create a new service in ROS, you typically follow these steps:
\begin{enumerate}
    \item Create a folder named \texttt{service} within your ROS package.
    \item Inside the \texttt{service} folder, create a folder named \texttt{srv}.
    \item Within the \texttt{srv} folder, create a file with a \texttt{.srv} extension. 
        This file will define your service request and response format, with inputs and outputs separated by a line with three dashes (\texttt{---}).
    \item In the \texttt{src} folder of your package, define a \texttt{.cpp} file that implements the service.
    \item In your \texttt{.cpp} file, initialize the node and create a function that executes the service. This function should return \texttt{true} to send the response.
    \item Optionally, define a client that calls the service with the proper values.
\end{enumerate}

To call the service, use the command:
\begin{verbatim}
rosrun service client_name parameters_list
\end{verbatim}
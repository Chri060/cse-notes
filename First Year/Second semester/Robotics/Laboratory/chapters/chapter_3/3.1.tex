\section{Basic commands}

To initiate ROScore within a tmux terminal, execute the following command: 
\begin{verbatim}
roscore
\end{verbatim}

\subsection{Topics}
To view active topics, use the command:
\begin{verbatim}
rostopic list
\end{verbatim}
To publish a new permanent topic, utilize:
\begin{verbatim}
rostopic pub /topic_name message_type message_name data
\end{verbatim}
For publishing a new temporary topic, use:
\begin{verbatim}
rostopic pub --once /topic_name message_type message_name data
\end{verbatim}
To publish a new topic at a specific frequency, employ:
\begin{verbatim}
rostopic pub -r frequency /topic_name message_type message_name data
\end{verbatim}
To check the running frequency of a particular topic, execute:
\begin{verbatim}
rostopic hz /topic_name
\end{verbatim}
To display the content of a specific topic, input:
\begin{verbatim}
rostopic echo /topic_name
\end{verbatim}
To view information about a topic with a graphical user interface (GUI), employ:
\begin{verbatim}
rqt_topic
\end{verbatim}

\subsection{Nodes}
To view active nodes, utilize the command:
\begin{verbatim}
rosnode list
\end{verbatim}
To access information about a specific node, use:
\begin{verbatim}
rosnode info /name
\end{verbatim}
For viewing information about a specific node with a graphical user interface (GUI), use:
\begin{verbatim}
rqt_graph
\end{verbatim}
To initiate a node, the command is:
\begin{verbatim}
rosrun package_name node_name
\end{verbatim}
For redirecting the topic associated with a particular node, utilize:
\begin{verbatim}
rosrun package_name node_name /old_topic:=/new_topic
\end{verbatim}
To generate a node resembling an existing one but with an altered name, execute:
\begin{verbatim}
rosrun \package_name \node_name --name:=\new_node_name
\end{verbatim}

\subsection{Services}
To invoke a service, employ:
\begin{verbatim}
rosservice call /service_name parameters
\end{verbatim}
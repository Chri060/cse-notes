\section{Publisher}

The structure of a publisher is as follows: 
\begin{verbatim}
#include "ros/ros.h"
#include "std_msgs/String.h"

int main(int argc, char **argv) {

// initialize the node with the name "talker"
// ros::init_options::AnonymousName initialize the node without a name 
ros::init(argc, argv, "talker"); 

// handle used to call all the ros releated functions
ros::NodeHandle n;

// create the publisher called "chatter_pub" with the type of message
// the message is a simple "String" and it is publish on the topic "chatter"
ros::Publisher chatter_pub = n.advertise<std_msgs::String>("chatter", 1);

// we want a loop with a running frequency of ten
ros::Rate loop_rate(10);

int count = 0;

// automatic check on the condition
while (ros::ok()) {

        // message "msg" creation
        std_msgs::String msg;
            
        // setting the field in the message, "String" has only this field
        msg.data = "hello world!";

        // print on the log console of the message
        ROS_INFO("%s", msg.data.c_str());

        // message publishing
        chatter_pub.publish(msg);

        // used for the loop to let the system know that one loop is done
        ros::spinOnce();

        // sleep for the requested time
        loop_rate.sleep();
        ++count;
}
return 0;
}
\end{verbatim}
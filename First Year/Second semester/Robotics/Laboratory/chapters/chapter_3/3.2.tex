\section{Bags and saves}

To record a series of commands given to a specific topic we write: 
\begin{verbatim}
rosbag record /topic_name
\end{verbatim}

To check the data inside a bag we write 
\begin{verbatim}
rosbag info bag_name.bag
\end{verbatim}

To play the data inside a bag we write 
\begin{verbatim}
rosbag play bag_name.bag
\end{verbatim}

To play in loop the data inside a bag we write 
\begin{verbatim}
rosbag play -l bag_name.bag
\end{verbatim}

To visualize data from a playing a bag we write: 
\begin{verbatim}
rviz
\end{verbatim}
Alternatively, to visualize simple numeric data we write: 
\begin{verbatim}
rosrun plotjuggler plotjuggler
\end{verbatim}
And select ROS topic subscriber as the source streaming. 
From the time series list we can finally select the specific topic. 

\subsection{Tf}
Tf files are included into a bag file. 
They can be also static, if the element does not change during time, and it is published only at the beginning of the execution. 
\section{Introduction}

The key features of ROS:
\begin{enumerate}
    \item \textit{Decentralized framework}: ROS operates on a decentralized architecture, facilitating robust communication and coordination among various components.
    \item \textit{Code reusability}: ROS promotes the reuse of code, allowing developers to efficiently leverage existing modules and algorithms for faster development.
    \item \textit{Language neutrality}: ROS supports multiple programming languages, enabling developers to work in their preferred language while seamlessly integrating with the ROS ecosystem.
    \item \textit{Seamless real robot and simulation testing}: ROS provides an environment for easy testing on both real robots and simulations, ensuring smooth transition from development to deployment.
    \item \textit{Scalability}: ROS is designed to scale efficiently, accommodating projects of various sizes and complexities without compromising performance or functionality.
\end{enumerate}
ROS is composed by the following elements: 
\begin{itemize}
    \item \textit{File system utilities}: ROS provides tools for managing files and directories, simplifying data organization and access within the ROS environment.
    \item \textit{Construction tools}: ROS offers construction tools to streamline the development process, facilitating the creation and configuration of robotic systems and components.
    \item \textit{Package management}: ROS includes robust package management capabilities, allowing users to easily install, update, and manage software packages and dependencies.
    \item \textit{Monitoring and graphical user interfaces} (GUIs): ROS features monitoring tools and graphical user interfaces to visualize and analyze system behavior, aiding in debugging, optimization, and user interaction.
    \item \textit{Data logging}: ROS supports data logging mechanisms for recording and analyzing sensor data, system state, and other relevant information, facilitating research, analysis, and debugging tasks.
\end{itemize}
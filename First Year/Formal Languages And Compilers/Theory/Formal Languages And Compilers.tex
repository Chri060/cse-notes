\documentclass[12pt, a4paper]{report}
\usepackage{graphicx, array, amsthm, amssymb, amsmath, algorithm, algpseudocode, float, xcolor, thmtools, thmbox}
\usepackage[english]{babel}

\makeatletter
\renewcommand\thmbox@headstyle[2]{\bfseries #1}
\makeatother
\newtheorem[style=M,bodystyle=\normalfont]{theorem}{Theorem}
\newtheorem[style=M,bodystyle=\normalfont]{corollary}{Corollary}
\newtheorem[style=M,bodystyle=\normalfont]{lemma}{Lemma}
\newtheorem[style=M,bodystyle=\normalfont]{definition}{Definition}


\title{Formal Languages And Compilers \\ \textit{Theory}}
\author{Christian Rossi}
\date{Academic Year 2023-2024}

\begin{document}

\maketitle

\newpage

\begin{abstract}
    The lectures are about those topics: 
    \begin{itemize}
        \item Definition of language, theory of formal languages, language operations, regular expressions, regular languages, finite deterministic and non-deterministic automata, 
            BMC and Berry-Sethi algorithms, properties of the families of regular languages, nested lists and regular languages.
        \item Context-free grammars, context-free languages, syntax trees, grammar ambiguity, grammars of regular languages, properties of the families of context-free languages, 
            main syntactic structures and limitations of the context-free languages.
        \item Analysis and recognition (parsing) of phrases, parsing algorithms and automata, push down automata, deterministic languages, bottom-up and recursive top-down syntactic 
            analysis, complexity of recognition.
        \item Syntax-driven translation, direct and inverse translation, syntactic translation schemata, transducer automata, and syntactic analysis and translation. Definition of 
            semantics and semantic properties. Static flow analysis of programs. Semantic translation driven by syntax, semantic functions and attribute grammars, one-pass and 
            multiple-pass computation of the attributes.
    \end{itemize}
    The laboratory sessions are about those topics: 
    \begin{itemize}
        \item Modellization of the lexicon and the syntax of a simple programming language (C-like).
        \item Design of a compiler for translation into an intermediate executable machine language (for a register-based processor).
        \item Use of the automated programming tools Flex and Bison for the construction of syntax-driven lexical and syntactic analyzers and translators.
    \end{itemize}
\end{abstract}

\newpage

\tableofcontents

\newpage

\chapter{Syntax}
    \section{Formal language theory}
    A \emph{formal language} consists of words whose letters are taken from an alphabet and are well-formed according to a specific set of rules.
    \begin{definition}
        An \emph{alphabet} is a finite set of elements called terminal symbols or \emph{characters}. 
        The \emph{cardinality} of an alphabet \[\Sigma =\{ a_1,a_2,\dots, a_k \}\] is the number of characters that it contains: $\left\lvert \Sigma \right\rvert = k$. 
        A \emph{string} or word is a sequence of characters. 
    \end{definition}
    \begin{example}
        The alphabet $\Sigma =\{ a,b \}$ has a cardinality of two. Some possible languages derived from this alphabet can be:
        \begin{itemize}
            \item $L_1=\{aa,aaa\}$
            \item $L_2=\{aba,aab\}$
            \item $L_3=\{ab,ba,aabb,abab,\dots,aaabbb,\dots\}$
        \end{itemize}
    \end{example}
    \begin{definition}
        Given a language, a string belonging to it is called a \emph{sentence} or \emph{phrase}. The \emph{cardinality} or size of a language is the number of sentence it contains.
        If the cardinality is finite, the language is called \emph{vocabulary}. 
    \end{definition}
    \begin{example}
        Given the language (that is a vocabulary) $L_2=\{ bc,bbc \}$ we have that its cardinality is equal to two. 
    \end{example}
    \begin{definition}
        The number of repetitions of a certain letter in a word is called \emph{number of occurrences}. The \emph{length} of a string is the number of its elements. 
        Two strings are \emph{equal} if and only if: 
        \begin{itemize}
            \item They have the same length.
            \item Their elements, from left to right, coincide. 
        \end{itemize}
    \end{definition}
    \begin{example}
        The number of occurrences of $a$ and $c$ in $aab$ is indicated with:
        \[{\left\lvert aab \right\rvert}_a = 2\]
        \[{\left\lvert aab \right\rvert}_c = 0\]
        The length of the string $aab$ is equal to: 
        \[\left\lvert aab \right\rvert = 3\]
    \end{example}
    In order to manipulate strings, it is convenient to introduce several operations:
    \begin{itemize}
        \item Concatenation: given two strings $x=a_1a_2\dots a_h$ and $y=b_1b_2\dots b_k$ this operation is defined as:
            \[x \cdot y = a_1a_2\dots a_h b_1b_2\dots b_k\]
            Concatenation is non-commutative and associative ($x(yz)=(xy)z$). The length of the result is the sum of the length of the concatenated strings
             ($\left\lvert xy \right\rvert = \left\lvert x \right\rvert + \left\lvert y \right\rvert$). 
        \item Empty string: $\varepsilon$ is the neutral element for concatenation that satisfies the identity:
            \[x\varepsilon=\varepsilon x=x\]
            It is important to note that $\left\lvert \varepsilon \right\rvert = 0$ and that the set that contains this operator is not the empty set. 
        \item Substring: let string $x=xyv$ be written as the concatenation of three, possibly empty, strings $x,y$ and $v$. Then, strings $x,y$ and $v$ are substrings of $x$. 
            Moreover, string $u$ is a prefix of $x$ and $v$ is a suffix of $x$. A non-empty substring is called proper if it does not coincide with string $x$. 
        \item Reflection: the reflection of a string $x=a_1a_2\dots a_h$ is:
            \[x^R=a_ha_{h-1}\dots a_1\]
            The following identities are immediate: 
            \[(x^R)^R=x \:\:\:\:\:\: (xy)^R=y^Rx^R \:\:\:\:\:\: \varepsilon^R=\varepsilon\]
        \item Repetition: the $m$-th power $x^m$ of a string $x$ is the concatenation of $x$ with himself $m-1$ times. The formal definition is the following: 
            \[x^m=x^{m-1}x \:\: for m \geq 1 \:\:\:\:\:\: x^0=\varepsilon\]
    \end{itemize}
    Repetition and reflection take precedence over concatenation. 

    Operations are typically defined on a language by extending the string operation to all its phrases: 
    \begin{itemize}
        \item Reflection: the reflection $L^R$ of a language $L$ is the finite set of strings that are the reflection of a sentence of $L$: 
            \[L^R = \{ x | \exists y \left( y \in L \land x=y^R \right)\}\]
        \item Prefixes: the set of prefixes of a language $L$ is defined as: 
            \[Prefixes(L)=\{y | y \neq \varepsilon \land \exists x \exists z \left( x \in L \land x=yx \land z \neq \varepsilon \right)\}\]
            A language is prefix-free if none of the proper prefixes of its sentences is in the language. 
        \item Concatenation: given languages $L^{'}$ and $L^{''}$ we have that concatenation is defined as: 
            \[L^{'}L^{''}=\{ xy | x \in L^{'} \land y \in L^{''} \}\]
        \item Repetition is redefined as: 
            \[L^m=L^{m-1}L \:for \: m \geq 1 \:\:\:\:\:\: L^0=\{ \varepsilon \}\]
            The identity now became: 
            \[\varnothing ^0 = \{ \varepsilon \} \:\:\:\:\:\: L.\varnothing=\varnothing .L=\varnothing \:\:\:\:\:\: L.\{\varepsilon\}=\{\varepsilon\} .L=L\]
            The power operator allows one to define concisely the language of strings whose length is not greater than a given integer $K$. 
        \item Since a language is a set, the classical set operation of union ($\cup$), intersection ($\cap$), difference ($ \setminus $), inclusion ($ \subseteq $), strict 
            inclusion ($ \subset $), and equality ($=$). 
        \item The universal language is defined as the set of all the strings, over an alphabet $\Sigma$, of any length including zero: 
            \[L_{universal}=\Sigma ^0 \cup \Sigma ^1 \cup \Sigma ^2 \cup \dots \]
        \item The complement of a language $L$ over an alphabet $\Sigma$, denoted by $\lnot L$, is the set difference: 
            \[ \lnot L = L_{universal} - L\]
            that is, the set of the strings over the alphabet $\Sigma$ that are not in $L$. Note that: 
            \[ L_{universal} = \lnot \varnothing \]
            The complement of a finite language is always infinite. The complement of an infinite one is not necessarily finite. 
    \end{itemize}








    \section{Regular expressions and languages}


\end{document}
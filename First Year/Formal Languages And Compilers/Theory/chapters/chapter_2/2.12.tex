\section{Comparison of regular and context-free languages}

Regular languages are a special case of free languages that are generated with strong constraints on the form of rules. 
Due to these constraints the sentences of regular languages present inevitable repetitions. 
The rules used to transform a regular expression into a grammar that generates the same regular language are the following: 
\begin{table}[H]
    \centering
    \begin{tabular}{cc}
    \hline
    \textbf{Regular expression}         & \textbf{Corresponding grammar}                \\ \hline
    $r=r_1r_2\dots r_k$                 & $E=E_1E_2 \dots E_k$                          \\ 
    $E=r_1\cup r_2\cup \dots\cup r_k$   & $E=E_1 \cup E_2 \cup \dots \cup E_k$          \\
    $r=(r_1)^{*}$                       & $E=EE_1|\varepsilon$ or $E=E_1E|\varepsilon$  \\
    $r=(r_1)^{+}$                       & $E=EE_1|E_1$ or $E=E_1E|E_1$                  \\
    $r = b \in \Sigma$                  & $E=b$                                         \\
    $r=\varepsilon$                     & $E=\varepsilon$                               \\ \hline
    \end{tabular}
\end{table}

In general, we have that the regular expressions are a subset of the context-free language: 
\[\textnormal{REG} \subset \textnormal{CF}\] 

\begin{definition}
    A grammar is \emph{unilinear} if and only if its rules are either all right-linear or all left-linear. 
\end{definition}
We can require that a unilinear grammar follows these constraints: 
\begin{itemize}
    \item Strictly unilinear rules: with at most one terminal $A \rightarrow aB$ with $A \in (\Sigma \cup \varepsilon)$ and $B \in (V \cup \varepsilon)$. 
    \item All terminal rules are empty. 
\end{itemize}
Therefore, we can assume only rules $A \rightarrow aB|\varepsilon$ for the right case and $A \rightarrow Ba|\varepsilon$ for the left case. 

It is possible to demonstrate that the regular expressions can be translated into strictly unilinear grammars. 
Therefore, the regular language set is a subset of unilinear grammars: $\textnormal{REG} \subseteq \textnormal{UNILIN}$. 
We can also show that from any unilinear grammar one can obtain an equivalent regular expression: $\textnormal{UNILIN} \subseteq \textnormal{REG}$.
As a result we have that: 
\[\textnormal{UNILIN}=\textnormal{REG}\]

Due to this property we can see the rules of the unilinear right grammar as equations, where the unknowns are the languages generated by every nonterminal. 
Let $G$ be a strictly unilinear right grammar with all terminal rules empty. 
A string $x \in \Sigma^{*}$ is in $L_A$ in the following cases: 
\begin{enumerate}
    \item $x$ is the empty string: we have a rule $P: A \rightarrow \varepsilon$. 
    \item $x=ay$: we have a rule $P: A \rightarrow aB$ and $y \in L_B$. 
\end{enumerate}
For every nonterminal $A_0$ defined by $A_0 \rightarrow a_1A_1|a_2A_2|\dots|a_kA_k|\varepsilon$ we have $L_A=a_1L_{a_1} \cup a_2L_{a_2} \cup \dots \cup a_kL_{a_k} \cup \varepsilon$. 
Therefore, we obtain a system of $n=\left\lvert V \right\rvert$ equations in $n$ unknowns to be solved with the method with substitution and by applying the Arden identity. 
\begin{definition}[Arden identity]
    Equation $KX \cup L$, with $K$ nonempty language and $L$ any language, has exactly one solution, which is. 
    \[X=K^{*}L=KK^{*}L \cup L\]
\end{definition}
\begin{example}
    Consider the grammar: 
    \[
    \begin{cases}
        S \rightarrow sS | eA \\
        A \rightarrow sS | \varepsilon
    \end{cases}
    \]
    This grammar can be transformed into a system of equation as follows: 
    \[
    \begin{cases}
        L_S \rightarrow sL_S \cup eL_A \\
        L_A \rightarrow sL_S \cup \varepsilon
    \end{cases}
    \]
    By substituting the second equation into the first one, and the applying the concatenation operation of the union operator we obtain: 
    \[
    \begin{cases}
        L_S \rightarrow (s \cup es)L_S \cup e \\
        A \rightarrow sL_S \cup \varepsilon
    \end{cases}
    \]
    We can now apply the Arden identity, obtaining: 
    \[
    \begin{cases}
        L_S \rightarrow (s \cup es)^{*}e \\
        A \rightarrow s(s \cup es)^{*}e \cup \varepsilon
    \end{cases}
    \]
\end{example}

We can note that regular languages exhibits inevitable repetitions. 
\begin{property}
    Let $G$ be a unilinear grammar. 
    Every sufficiently long sentence $x$ (i.e. longer than a grammar-dependent constant $k$) can be factorized as $x=tuv$ (with $u$ non-empty) so that, for all $i \geq 1$, the string $tu^nv \in L(G)$. 
\end{property}
In other words, the sentence can be pumped by injecting string $u$ an arbitrary number of times. 
\begin{proof}
    Consider a strictly right-linear grammar $G$ with $k$ nonterminal symbols. 
    In the derivation of a sentence $x$ whose length is $k$ or more, there is necessarily a nonterminal $A$ that appears at least two times. 
    Then, it is also possible to derive $tv$, $tuv$, $tuuv$, etc.
\end{proof}
This property is useful to demonstrate whether a grammar generates a regular language or not. 

A grammar generates a regular language only if it has no self-nested derivations. 
Note that the inverse is not necessarily true: a regular language may be generated by a grammar with self-nested derivations. 
The lack of self-nested derivations allows solving language of equations of unilinear grammars. 

In the context-free languages all sufficiently long sentences necessarily contain two substring that can be repeated arbitrarily many times, thus originating self-nested structures. 
This hinders the derivation of string with three or more parts that are repeated the same number of times (e.g., $a^nb^nc^n$). 
As a result, the language of three or more power is not context-free
Therefore, the language of copies is also not context-free. 

\subsection*{Closure properties}
The regular language is closed under reverse, star, complement, union, and intersection operators. 
On the other hand, the context-free language is closed under reverse, star, and union operators. 

We can also prove that the intersection between a context-free language and a regular language is still part of the context-free language. 

To make a grammar more selective one can filter it through a regular language. 
The result of this filtering is always context-free.
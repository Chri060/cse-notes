\section{Syntax trees and canonical derivations}

\begin{definition}
    A \emph{tree} is an oriented and ordered graph not containing a circuit, such that every pair of nodes is connected by exactly one oriented path.
    
    An \emph{arc} $\langle N_1,N_2 \rangle$ define the $\langle \textnormal{father,son} \rangle$ relation, customarily visualized from top to bottom as in genealogical 
    trees. The sides of a node are ordered from left to right. 

    The \emph{degree} of a node is the number of its siblings. 
    
    A \emph{tree} contains one node without father, termed root.

    Consider an internal node $N$: the subtree with root $N$ is the tree having $N$ as root and containing all descendants of $N$. Nodes without sibling are termed leaves or \emph{terminal nodes}. 

    The sequence of all leaves, read from left to right, is the \emph{frontier} of the tree.

    A \emph{syntax tree} has as root the axiom and as frontier a sentence.
\end{definition}
A syntax tree of a sentence $x$ can also be encoded in a text, by enclosing each subtree between brackets. Brackets are subscribed with the non-terminal symbol. The representation can be simplified 
by dropping the non-terminal labels, thus obtaining a skeleton tree. A further simplification of the skeleton tree consists in shortening non bifurcating paths, resulting in the condensed skeleton tree. 
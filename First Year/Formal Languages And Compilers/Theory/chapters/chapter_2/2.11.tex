\section{Free grammars extended with regular expressions}

The class of EBNF is useful to construct grammars that are more readable thanks to star, cross and union operators. 

These grammars also allow for the definition of syntax diagrams which can be viewed as a blueprint of the syntax analyzer flowchart. 

Note that since the context-free family is closed under all regular operations, therefore the generative power of EBNF is the same as that of BNF. 

\begin{definition}
    An EBNF grammar is defined as a four-tuple $\{V, \Sigma, P, S\}$, where we have exactly $\left\lvert V \right\rvert$ rules in the form $A \rightarrow \eta$ with $\eta$ being a regular expression over $\Sigma \cup V$.
\end{definition}
The BNF grammar is longer and less readable than an EBNF. 
Furthermore, the choice of nonterminal symbols names can be arbitrary. 

The derivation relation in EBNF is defined by considering an equivalent BNF with infinite rules. 
\begin{definition}
    Given string $\eta_1$ and $\eta_2$ within $(\Sigma \cup V)^{*}$. 
    The string $\eta_2$ is said to be \emph{derived} immediately in $G$ from $\eta_1$, denoted as $\eta_1 \implies \eta_2$ if the two strings can be factorized as: 
    \[\eta_1=\alpha A \gamma\]
    \[\eta_2=\alpha \vartheta \gamma\]
    and there exists a rule: 
    \[A \rightarrow e\]
    Such that the regular expression $e$ admits the derivation $e \overset{*}{\implies} \vartheta$. 
\end{definition}
Note that $\eta_1$ and $\eta_2$ does not contain regular expressions' operators nor parenthesis. 
Only string $e$ is a regular expression, but it does not appear in the derivation if it is not terminal.

With EBNF we have unbounded node degree. As a result, the tree is in general wider and reduced in depth. 
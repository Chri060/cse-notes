\section{Derivation and Language Generation}

We reconsider and formalize the notion of string derivation. Let $\beta=\delta A \eta$ be a string containing a non-terminal, where $\delta$ and $\eta$ are any, 
possibly empty strings. Let $A \rightarrow \alpha$ be a rule of $G$ and let $\gamma=\delta\alpha\eta$ be the string obtained replacing in $\beta$ non-terminal $A$with 
the right part $\alpha$. The relation between such two strings is called derivation. We say that $\beta$ derives $\gamma$ for grammar $G$, written:
\[\beta \implies \gamma\]
$A\rightarrow \alpha$ is applied in such derivation and string $\alpha$ reduced to non-terminal $A$. The possible closures are: power ($\implies^n$), 
reflexive ($\implies^{*}$), and transitive ($\implies^{+}$). 
\begin{definition}
    If $A \implies^{*} \alpha$ we have that $\alpha \in (V \cup \Sigma)$ is called \emph{string form} generated by $G$. 

    If $S \implies^{*} \alpha$ we have that $\alpha$ is called \emph{sentential} or phrase form.

    If $A \implies^{*} s$ we have that $s \in \Sigma^{*}$ is called \emph{phrase} or sentence. 

    Language is \emph{context-free} if a context-free grammar exists that generates it. 
    
    Two grammars $G$ and $G^{'}$ are \emph{equivalent} if they generate the same language. 
\end{definition}
\section{Derivation and language generation}

\begin{definition}
    Given $\beta, \gamma \in (V \cup \Sigma)^{*}$, we state that $\beta$ \emph{derives} $\gamma$ within a grammar $G$, denoted as $\beta\underset{G}{\implies}\gamma$ or $\beta\implies\gamma$, if and only if we have the following conditions:
    \begin{itemize}
        \item $\beta=\delta A \eta$. 
        \item There exists a rule $A\rightarrow a$ in the grammar $G$. 
        \item $\gamma=\delta\alpha\eta$
    \end{itemize}
\end{definition}
We can establish the following closure properties:
\begin{itemize}
    \item Power: $\beta_0 \overset{n}{\implies} \beta_n$. 
    \item Reflexive: $\beta_0 \overset{*}{\implies} \beta_n$. 
    \item Transitive: $\beta_0 \overset{+}{\implies} \beta_n$. 
\end{itemize}
\begin{definition}
    If $A \overset{*}{\implies} \alpha$, then $\alpha \in (V \cup \Sigma)$ is called \emph{string form generated by $G$}. 

    If $S \overset{*}{\implies} \alpha$, then $\alpha$ is called \emph{sentential} or phrase form.

    If $A \overset{*}{\implies} s$, then $s \in \Sigma^{*}$ is called \emph{phrase} or sentence. 
\end{definition}
\begin{example}
    Let's consider the grammar $G_l$ responsible for generating the structure of a book. 
    This grammar consists of a front page $f$ and a series $A$ of one or more chapters. 
    Each chapter starts with a title $t$ and contains a sequence $B$ of one or more lines $l$. 
    The corresponding grammar rules are as follows:
    \[
    \begin{cases}
        S \rightarrow fA \\
        A \rightarrow AtB | tB \\
        B \rightarrow lB | l
    \end{cases}
    \]
    In this context:
    \begin{itemize}
        \item From $A$, one can generate the string form $tBtB$ and the phrase $tlltl \in L_A(G_l)$. 
        \item From $S$, one can generate the phrase forms $fAtlB$ and $ftBtB$. 
        \item The language generated from $B$ is $L_B(G_l)=l^{+}$.
        \item The language $L(G_l)$ is generated by the context-free grammar $G_l$, making it a context-free language.
    \end{itemize}
\end{example}
\begin{definition}
    A language is considered \emph{context-free} if there exists a context-free grammar that generates it.
    
    Two grammars, denoted as $G$ and $G^{'}$ are \emph{equivalent} if they both generate the same language.
\end{definition}
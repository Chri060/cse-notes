\section{Context-free generative grammars}

Regular expressions are very practical for describing lists but fall short of the capacity needed to define other frequently occurring constructs. 
For defining other useful languages, regular or not, we move to the formal model of generative grammars.  A generative grammar or syntax is a set of multiple 
rules that can be repeatedly applied in order to generate all and only the valid strings. 
\begin{definition}
    A \emph{context-free grammar} $G$ is defined by four entities: 
    \begin{enumerate}
        \item $V$ non-terminal alphabet, is the set of non-terminal symbols.
        \item $\Sigma$ terminal alphabet, is the set of the symbols of which phrases or sentences are made.
        \item $P$ is the set of rules or productions.
        \item $S \in V$ is the specific non-terminal, called the axiom ($S$), from which derivations start. 
    \end{enumerate}
\end{definition}
A rule of set $P$ is an order pair $X \rightarrow \alpha$, with $X \in V$ and $\alpha \in (V \cup \Sigma)^{*}$. Two or more rules: 
\[X \rightarrow \alpha_1 \:\:\:\: X \rightarrow \alpha_2 \:\:\:\: \dots \:\:\:\: X \rightarrow \alpha_n\]
with the same left part $X$ can be concisely groped in:
\[X \rightarrow \alpha_1 | \alpha_2 | \dots | \alpha_n\]
We say that the strings $\alpha_1,\alpha_2,\dots,\alpha_n$ are the alternative of $X$. 

In professional practice, different styles are used to represent terminals and non-terminals. We usually adopt these conventions: 
\begin{itemize}
    \item Lowercase Latin letters $\{a,b,\dots\}$ for terminal characters. 
    \item Uppercase Latin letters $\{A,B,\dots\}$ for non-terminal symbols. 
    \item Lowercase Latin letters $\{r,s,\dots,z\}$ for strings over the alphabet $\Sigma$. 
    \item Lowercase Greek letters $\{r,s,\dots,z\}$ for both terminals and non. 
    \item $\sigma$ only for non-terminals. 
\end{itemize}
The classification of grammar rule forms is the following. 
\begin{figure}[H]
    \centering
    \includegraphics[width=1\linewidth]{images/grammars.png}
\end{figure}
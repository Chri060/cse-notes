\section{Regular composition of context-free languages}

Context-free languages exhibit closure properties under union, concatenation, and star operations.
Consider two grammars, $G_1=(\Sigma_1,V_1,P_1,S_1)$ and $G_2=(\Sigma_2,V_2,P_2,S_2)$ defining languages $L_1$ and $L_2$, respectively.
Let's assume that the sets of nonterminal symbols, $V_{N_1}$ and $V_{N_2}$ are disjoint and that $S \notin (V_{N_1} \cup V_{N_2})$. 

\paragraph*{Union}
The union of languages $L_1 \cup L_2$ is defined by a grammar that incorporates the rules of both individual grammars, along with the additional initial rule $S\rightarrow S_1|S_2$. 
In formulaic terms, the grammar is expressed as:
\[G=\left(\Sigma_1 \cup \Sigma_2,\{S\} \cup V_{N_1} \cup V_{N_2},\{S\rightarrow S_1|S_2\} \cup P_1 \cup P_2,S\right)\]
\begin{example}
    The language $L=\{a^ib^jc^k|i=j \lor j=k\}$ can be defined as the union of two languages:
    \[L=\{a^ib^ic^{*}|i \geq 0\} \cup \{a^{*}b^ic^i|i \geq 0\}=L_1 \cup L_2\]
    The grammars for these two languages, $G_1$ and $G_2$, are defined as follows:
    \[G_1=\begin{cases}
        S_1 \rightarrow XC \\
        X \rightarrow aXb|\varepsilon \\
        C \rightarrow cC|\varepsilon
    \end{cases}\qquad G_2=
    \begin{cases}
        S_2 \rightarrow AY \\
        Y \rightarrow bYc|\varepsilon \\
        A \rightarrow aA|\varepsilon
    \end{cases}\]
    The union language is defined with the rule:
    \[S \rightarrow S_1|S_2\]
    It's important to note that the nonterminal sets of grammars $G_1$ and $G_2$ are distinct.
\end{example}
If the nonterminals in the grammars are not disjoint, meaning they share some common nonterminals, the resulting grammar generates a superset of the union language.
This may lead to the generation of spurious additional sentences.

\paragraph*{Concatenation}
The concatenation $L_1L_2$ is defined by the grammar a grammar that incorporates the rules of both individual grammars, along with the additional initial rule $S\rightarrow S_1S_2$.
The grammar is formulated as follows:
\[G=\left(\Sigma_1 \cup \Sigma_2,\{S\} \cup V_{N_1} \cup V_{N_2},\{S\rightarrow S_1S_2\} \cup P_1 \cup P_2,S\right)\] 

\paragraph*{Star}
For the star language $(L_1)^{*}$, the grammar $G$ is obtained by augmenting $G_1$ with the rules $S \rightarrow SS_1|\varepsilon$.

\paragraph*{Cross}
The language $(L_1)^{+}$ is generated by the grammar $G$, which is derived by adding to $G_1$ the rules $S \rightarrow SS_1|S1$.

\paragraph*{Mirror language}
The mirror language of $L(G)$, denoted as $(L(G))^R$, is generated by a mirror grammar. This grammar is obtained by reversing the right-hand side of the rules.
\section{Parenthesis languages}

Many  artificial  languages  include  parenthesized  or  nested  structures,  made  by matching pairs of opening/closing marks. Any such occurrence may contain other matching pairs.
The marks are abstract elements that have different concrete representations indistinct settings.
\begin{definition}
    When a marked construct may contain another construct of the same kind, it is called \emph{self-nested}.
\end{definition}
Self-nesting is potentially unbounded in artificial languages, whereas in natural languages its use is moderate, because it causes difficulty of comprehension by breaking the flow of discourse. 
Abstracting from concrete representation and content, this paradigm is known as a Dyck language. The terminal alphabet contains one or more pairs of opening/closing marks. 
Dyck sentences are characterized by the following cancelation rule that checks parentheses are well nested: given a string, repeatedly substitute the empty string for a pair of adjacent matching parentheses:
\[[\:]\implies\varepsilon \:\:\:\:\:\: (\:)\implies\varepsilon\]
Thus obtaining another string. Repeat until the transformation no longer applies; the original string is correct if, and only if, the last string is empty.
\begin{definition}
    Let $G=(V,\Sigma,P,S)$ be a grammar with an alphabet $\Sigma$ not containing parentheses. The \emph{parenthesized grammar} $G_p$ has alphabet $\Sigma \cup \{'(',')'\}$ and rules:
    \[A \rightarrow (\alpha) \textnormal{ where } A \rightarrow (\alpha) \textnormal{ is a rule of } G\]
    The grammar is distinctly parenthesized if every rule has form:
    \[A \rightarrow (_A \alpha)_A \:\:\:\:\:\: B \rightarrow (_B \alpha)_B\]
    where $(_A$ and $)_A$ are parentheses subscripted with the non-terminal name.
\end{definition}
Clearly each sentence produced by such grammars exhibits parenthesized structure. A notable effect of the presence of parentheses is to allow a simpler checking of string correctness. 
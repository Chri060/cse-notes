\section{Regular composition of context-free languages}

The context-free languages are closed under union, concatenation, and star. 

Let $G_1=(\Sigma_1,V_1,P_1,S_1)$ and $G_2=(\Sigma_2,V_2,P_2,S_2)$ be the grammars defining languages $L_1$ and $L_2$. 
Le's also suppose that $V_{N_1} \cap V_{N_2}=\varnothing$ and $S \notin (V_{N_1} \cup V_{N_2})$. 

\subsection*{Union}
The union $L_1 \cup L_2$ is defined by the grammar containing the rules of both grammars, plus the initial rules $S\rightarrow S_1|S_2$. In formulas, the grammar is: 
\[G=\left(\Sigma_1 \cup \Sigma_2,\{S\} \cup V_{N_1} \cup V_{N_2},\{S\rightarrow S_1|S_2\} \cup P_1 \cup P_2,S\right)\]

\begin{example}
    The language $L=\{a^ib^jc^k|i=j \lor j=k\}$ can be defined as the union of two languages:
    \[L=\{a^ib^ic^{*}|i \geq 0\} \cup \{a^{*}b^ic^i|i \geq 0\}=L_1 \cup L_2\]
    Those two languages are defined by the following two grammars: 
    \[
    G_1=
    \begin{cases}
        S_1 \rightarrow XC \\
        X \rightarrow aXb|\varepsilon \\
        C \rightarrow cC|\varepsilon
    \end{cases}    
    G_2=
    \begin{cases}
        S_2 \rightarrow AY \\
        Y \rightarrow bYc|\varepsilon \\
        A \rightarrow aA|\varepsilon
    \end{cases}  
    \]
    The union language is defined with the rule: 
    \[S \rightarrow S_1|S_2\]
    It's worth noting that the nonterminal sets of grammars $G_1$ and $G_2$ are distinct.
\end{example}
If the nonterminals in the grammars are not disjoint, it means that they have some common nonterminals. 
In this case, the grammar generates a superset of the union language, which results in spurious additional sentences being generated.

\subsection*{Concatenation}
The concatenation $L_1L_2$ is defined by the grammar containing the rules of both grammars, plus the initial rule $S\rightarrow S_1S_2$. The grammar is: 
\[G=\left(\Sigma_1 \cup \Sigma_2,\{S\} \cup V_{N_1} \cup V_{N_2},\{S\rightarrow S_1S_2\} \cup P_1 \cup P_2,S\right)\] 

\subsection*{Star}
The grammar $G$ of the star language $(L_1)^{*}$ is obtained by adding to $G_1$ and rules $S \rightarrow SS_1|\varepsilon$.

\subsection*{Cross}
The grammar $G$ of language $(L1)^{+}$ is obtained by adding to $G_1$ and rules $S \rightarrow SS_1|S1$. 

\subsection*{Mirror language}
The mirror language of $L(G)$, denoted as $(L(G))^R$, is generated by a mirror grammar, which is obtained by reversing the right-hand side of the rules.
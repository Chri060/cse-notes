\section{Ambiguity}

The common linguistic phenomenon of ambiguity in natural language shows up when a sentence has two or more meanings. Ambiguity is of two kinds, semantic or syntactic.
\begin{definition}
    A sentence $x$ defined by grammar $G$ is \emph{syntactically ambiguous}, if it is generated with two different syntax trees. Then the grammar too is called ambiguous.

    The \emph{degree of ambiguity} of a sentence $x$ of  language $L(G)$ is the number of distinct syntax trees deriving the sentence. For a grammar the degree of ambiguity is the maximum degree of 
    any ambiguous sentence.
\end{definition}
The ambiguity can be: 
\begin{itemize}
    \item From bilateral recursion
\end{itemize}


PAG 47 A 79
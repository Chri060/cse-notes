\section{Parenthesis languages}

Many artificial languages include parenthesized or nested structures, formed by matching pairs of opening and closing marks. 
These parentheses can be nested, meaning that inside a pair, there can be other parenthesized structures (recursion). 
Nested structures can also be placed in sequences at the same level of nesting. 
This paradigm, abstracted from concrete representation and content, is known as a Dyck language. 
\begin{example}
    For example, an alphabet of a Dyck language could be $\Sigma=\{'(',')','[',']'\}$, and a valid sentence over this alphabet is $()[[()[]]()]$.
\end{example}
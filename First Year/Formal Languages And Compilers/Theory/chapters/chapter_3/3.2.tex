\section{Introduction to finite automata}

Conforming to the general scheme, a finite automaton comprises: the input tape with the source string $x \in \Sigma^{*}$, the control unit, and the reading head scanning 
the string until its end, unless an error occurs before. Upon reading a character, the automaton updates the state  of the  control unit and advances the reading head. 
Upon reading the last character, the automaton accepts $x$ if and only if the state is an accepting one.

A well-known representation of an automaton is by a state-transition diagram or graph. This is a directed graph whose nodes are the states of the control unit. Each arc
is labeled with a terminal and represents the change of state or transition caused by reading the terminal.

An automaton may have several final states, but only one initial state. 
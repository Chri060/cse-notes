\section{Elimination of non-determinism}

Every non-deterministic finite automaton can always be transformed into an equivalent deterministic one. Consequently, every right linear grammar always admits an 
equivalent non-ambiguous right linear one. Thus,  every ambiguous regular expression can always be transformed into a non-ambiguous one. The algorithm to transform a 
non-deterministic automaton into a deterministic one is structured in two phases: 
\begin{enumerate}
    \item Elimination of the spontaneous moves. As such moves correspond to copy rules, it suffices to apply the algorithm for removing the copy rules. 
    \item Replacement of the non-deterministic multiple transitions by changing the automaton state set. This is the well known subset construction. 
\end{enumerate}
\begin{example}
    Given the following automaton: 
    \begin{figure}[H]
        \centering
        \includegraphics[width=0.5\linewidth]{images/oaut.png}
    \end{figure}
    After applying the algorithm we have: 
    \begin{figure}[H]
        \centering
        \includegraphics[width=0.5\linewidth]{images/faut.png}
    \end{figure}
\end{example}
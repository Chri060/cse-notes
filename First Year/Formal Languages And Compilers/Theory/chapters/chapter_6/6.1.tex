\section{Introduction}

In a general sense, a translation refers to a function (or mapping) from a source language to a target language. 
Typically, two approaches are employed: through coupled grammars or using a transducer (an automaton similar to an acceptor with the ability to produce output).
These methods fall under the category of purely syntactic translations, expanding and completing the language definition and parsing methods studied so far. 
However, they do not inherently provide a semantic interpretation (meaning) of the input string. 
In contrast, an alternative method known as the attribute grammar model capitalizes on the syntactic modularity of grammar rules.
The distinction between syntax and semantics lies in the former dealing with the form of a sentence, while the latter is concerned with its meaning. 
In computer science, this difference is reflected in the domain of entities and operations permitted by each:
\begin{itemize}
  \item Syntax employs concepts and operations from formal language theory, representing algorithms as automata. 
    Entities include alphabets, strings, and syntax trees, with operations such as concatenation, union, intersection, and complementation.
  \item Semantics utilizes concepts and operations from logic, representing algorithms as programs. 
    Entities in semantics are not as limited as in syntax, encompassing numbers, strings, and any data structures. The complexity level is higher, with operations extending beyond formal language theory.
\end{itemize}
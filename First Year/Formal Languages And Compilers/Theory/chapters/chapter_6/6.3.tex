\section{Semantic Translation}

The previously detailed syntactic translation methods prove insufficient for handling slightly more intricate translations, such as converting a number from binary to decimal. 
These methods rely on tools that are too basic to accomplish such tasks. To address this limitation, a more robust approach, known as syntax-directed translation, becomes necessary. 
The term "directed" emphasizes the departure from the purely syntactic methods discussed in preceding sections.

Syntax-directed translation involves semantic techniques, including tree-walking procedures that traverse the syntax tree and compute variables known as semantic attributes. 
These attributes encapsulate the meaning or semantics of a given source text. 
It's crucial to note that a syntax-directed method is not a formal model because the procedures for computing attributes lack formalization.

A syntax-directed compiler carries out two consecutive phases:
\begin{enumerate}
    \item Parsing or syntax: this phase computes a syntax tree, typically condensed into an abstract syntax tree, containing essential information for the subsequent phase.
    \item Semantic evaluation or semantic: the semantic phase involves applying a set of semantic functions to each node of the tree until all attributes are evaluated. 
        The set of evaluated attribute values represents the meaning or translation, and this information is found at the root of the tree.
\end{enumerate}
This two-phase approach is termed two-pass compilation and is the most common and straightforward method of compilation. 
The separation of parsing and semantic evaluation provides compiler designers with greater flexibility in creating these phases.
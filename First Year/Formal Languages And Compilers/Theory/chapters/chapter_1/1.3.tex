\section{Operations on languages}

Operations on a language are usually defined by applying the string operations to all of its phrases.

\subsection*{Reflection}
The reflection $L^R$ of a language $L$ consists of a finite set of strings that are reversals of sentences in $L$: 
\[L^R = \{ x | \exists y \left( y \in L \land x=y^R \right)\}\]

\subsection*{Prefix}  
The set of prefixes of a language $L$ is defined as follows:
\[\textnormal{Prefixes}(L)=\{y | y \neq \varepsilon \land \exists x \exists z \left( x \in L \land x=yz \land z \neq \varepsilon \right)\}\]
A language is considered prefix-free if it contains none of the proper prefixes of its sentences:
\[\textnormal{Prefixes}(L) \cap L = \varnothing\]
\begin{example}
    The language $L_1=\{x|x=a^nb^n \land n \geq 1\}$ is prefix-free. 
    
    The language $L_2=\{x|x=a^mb^n \land m > n \geq 1\}$ is not prefix-free. 
\end{example}

\subsection*{Concatenation}  
When dealing with languages $L^{'}$ and $L^{''}$, the concatenation operation is defined as:
\[L^{'}L^{''}=\{ xy | x \in L^{'} \land y \in L^{''} \}\]

\subsection*{Repetition}  
The definition of repetition for languages is as follows:
\[  
\begin{cases}
    L^m=L^{m-1}L \:\:\:\:\:\: \textnormal{for } m > 0 \\
    L^0=\{ \varepsilon \}
\end{cases} 
\]
The corresponding identities are:
\[\varnothing ^0 = \{ \varepsilon \} \]
\[L.\varnothing=\varnothing .L=\varnothing \]
\[L.\{\varepsilon\}=\{\varepsilon\} .L=L\]
Utilizing the power operator provides a concise way to define the language of strings whose length does not exceed a specified integer $k$. 
\begin{example}
    The language $L=\{\varepsilon,a,b\}^k$ with $k=3$ can be represented as follows: 
    \[L=\{\varepsilon,a,b,aa,ab,ba,bb,aaa,\dots,bbb\}\] 
\end{example}

\subsection*{Set operations}  
As a language is a set, it supports the standard set operations, including union ($\cup$), intersection ($\cap$), difference ($ \setminus $), inclusion ($ \subseteq $), strict inclusion ($ \subset $), and equality ($=$). 

\subsection*{Universal language} 
The universal language is defined as the collection of all the strings, over an alphabet $\Sigma$, of any length including zero: 
\[L_{universal}=\Sigma ^0 \cup \Sigma ^1 \cup \Sigma ^2 \cup \dots \]

\subsection*{Complement} 
The complement of a language $L$ over an alphabet $\Sigma$, indicated by $\lnot L$, is defined as the set difference:
\[\lnot L=L_{universal}\backslash L\]
In other words, it comprises the strings over the alphabet $\Sigma$ that do not belong to the language $L$.
It's important to note that:
\[L_{universal} = \lnot \varnothing\]
The complement of a finite language is always infinite.
However, the complement of an infinite language is not necessarily finite.

\subsection*{Reflexive and transitive closures} 
Given a set $A$ and a relation $R \subseteq A \times A$, the pair $(a_1, a_2) \in R$ is often represented as $a_1Ra_2$. 
The relation $R^{*}$ is a relation defined by the following properties:
\begin{itemize}
    \item Reflexive property:
        \[xR^{*}x \:\:\:\:\:\: \forall x \in A\]
    \item Transitive property: 
        \[x_1Rx_2 \land x_2Rx_3 \land \dots x_{n-1}Rx_n \implies x_1R^{*}x_n\]
\end{itemize}
\begin{example}
    For the given relation $R = \{(a, b), (b, c)\}$, its reflexive and transitive closure, denoted as $R^{*}$, will be:
    \[R^{*} = \{(a, a), (b, b), (c, c), (a, b), (b, c), (a, c) \}\]
\end{example}
The relation $R^{+}$ is a relation defined by the following property: 
\begin{itemize}
    \item Transitive property: 
        \[x_1Rx_2 \land x_2Rx_3 \land \dots x_{n-1}Rx_n \implies x_1R^{*}x_n\]
\end{itemize}
\begin{example}
    For the given relation $R = \{(a, b), (b, c)\}$, the transitive closure will be: 
    \[R^{+} = \{ (a, b), (b, c), (a, c)\}\]
\end{example}

\subsection*{Star operator} 
The star operator, also known as the Kleene star, is the reflexive transitive closure with respect to the concatenation operation.
It is defined as the union of all the powers of the base language:
\[L^{*}=\bigcup_{h=0\dots\infty}L^h=L^0 \cup L^1 \cup L^2 \cup \dots = \varepsilon \cup L^1 \cup L^2 \cup \dots\]
\begin{example}
    Consider the language $L=\{ab,ba\}$. 
    Applying the star operation results in the following language:
    \[L^{*}=\{\varepsilon, ab, ba, abab, abba, baab, baba, \dots\}\]
    It's noticeable that $L$ is finite, while $L^{*}$ is infinite, demonstrating the generative power of the star operation.
\end{example}
Every string within the star language $L^{*}$ can be divided into substrings belonging to the base language $L$. 
Consequently, the star language $L^{*}$ can be equivalent to the base language $L$. 
If we take the alphabet $\Sigma$ as the base language, then $\Sigma^{*}$ contains all possible strings constructed from that alphabet, making it the universal language of alphabet $\Sigma$.
It's common to express that a language $L$ is defined over the alphabet $\Sigma$ by indicating that $L$ is a subset of $\Sigma^{*}$, denoted as $L \subseteq \Sigma^{*}$. 
The properties of the star operator can be summarized as follows:
\begin{itemize}
    \item Monotonicity: $L \subseteq L^{*}$. 
    \item Closure by concatenation: if $x \in L^{*} \land y \in L^{*}$ then $xy \in L^{*}$. 
    \item Idempotence: $(L^{*})^{*}=L^{*}$
    \item Commutativity of star and reflection: $(L^{*})^R=(L^R)^{*}$
\end{itemize}
Additionally, if $L^{*}$ is finite, then we observe that $\varnothing^{*}=\{\varepsilon\}$ and $\{\varepsilon\}^{*}=\{\varepsilon\}$. 

\subsection*{Cross operator} 
The cross operator, also known as the transitive closure under the concatenation operation, is defined as the union of all the powers of the base language, excluding the first power $L^0$:
\[L^{+}=\bigcup_{h=1\dots\infty}L^h=L^1 \cup L^2 \cup \dots\]
\begin{example}
    Consider the language $L=\{ab,ba\}$. 
    Applying the cross operator results in the following language:
    \[L^{*}=\{ab, ba, abab, abba, baab, baba, \dots\}\]
\end{example}

\subsection*{Quotient} 
The quotient operator reduces the phrases in $L_1$ by removing a suffix that belongs to $L_2$ and is defined as follows:
\[L=L_1/L_2=\{y|\exists x \in L_1 \exists z \in L_2 (x=yz)\}\]
\begin{example}
    Consider the languages $L_1=\{a^{2n}b^{2n}|n>0\}$ and $L_2=\{b^{2n+1}|n \geq 0\}$. 
    The quotient language $L_1/L_2$ is:
    \[L_1/L_2=\{aab,aaaab,aaaabbb\}\]
    The quotient language $L_2/L_1$ is:
    \[L_2/L_1=\varnothing\]
    This is because no string in $L_2$ contains any string from $L_1$ as a suffix.
\end{example}
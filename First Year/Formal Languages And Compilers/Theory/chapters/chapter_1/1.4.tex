\section{Regular expressions and languages}

The family of regular languages is our simplest formal language family. It can be defined in three ways: algebraically, by means of generative grammars, and by means of 
recognizer automata. 
\begin{definition}
    A \emph{regular expression} is a string $r$ containing the terminal characters of the alphabet $\Sigma$ and the following meta-symbols: union ($\cup$), concatenation ($.$), 
    star ($^{*}$), empty string ($\varepsilon$), and parenthesis in accordance with the following rules:
    \begin{table}[H]
        \centering
        \begin{tabular}{|cc|}
        \hline
        $r=\varepsilon$ & Empty string                 \\
        $r=a$           & Unitary language             \\
        $r=s \cup t$    & Union of expressions         \\
        $r=(st)$        & Concatenation of expressions \\
        $r=s^{*}$       & Iteration of an expression   \\ \hline
        \end{tabular}
    \end{table}
    where the symbols $s$ and $t$ are regular sub-expression. 
\end{definition}
For expressivity, the metasymbol cross is allowed. The operators precedence is: star, concatenation, and union. 
\begin{definition}
    A \emph{regular language} is a language denoted by a regular expression. 

    The \emph{family of regular languages} (REG) is the collection of all regular languages. 

    The \emph{family of finite languages} (FIN) is the collection of all languages having a finite cardinality
\end{definition}
We have that every finite language is regular because it is the union of a finite number of strings each one being the concatenation of a finite 
number of alphabet symbols. The family of regular languages also includes languages having infinite cardinality (hence $FIN \subset REG$)

The union and repetition operators correspond to possible choices. One obtains a sub-expression by making a choice that identifies a sub-language. Given a regular expression 
one can derive another one by replacing any outermost sub-expression with another that is a choice of it. 
\begin{definition}
    We say that a regular expression $e^{'}$ \emph{derives} a regular expression $e^{''}$, written $e^{'} \implies e^{''}$, if the two regular expressions can be factorized as 
    \[e^{'}=\alpha \beta \gamma \:\:\:\:\:\: e^{''}=\alpha \delta \gamma\]
    where $\delta$ is a choice of $\beta$.
\end{definition}
The derivation relation can be applied repeatedly, yielding relation $\implies^{n}$ ($n$ steps), $\implies^{*}$ ($n \geq 0$ steps), and $\implies^{+}$ ($n > 0$ steps). 
\begin{definition}
    Two regular expressions are \emph{equivalent} if they define the same language. 

    A regular expression is \emph{ambiguous} if the language of the numbered version $f^{'}$ includes two distinct strings $x$ and $y$ that coincide when numbers are erased. 
\end{definition}
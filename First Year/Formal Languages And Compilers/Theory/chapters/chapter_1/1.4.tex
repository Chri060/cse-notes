\section{Regular expressions and languages}

The family of regular languages is the most basic among formal language families and can be defined in three different ways: algebraically, through generative grammars, and by using recognizer automata.
\begin{definition}
    A \emph{regular expression} is a string denoted as $r$, constructed over the alphabet $\Sigma=\{a_1,a_2,\dots,a_k\}$ and featuring metasymbols: union ($\cup$), concatenation ($\cdot$), star ($^{*}$), empty string ($\varepsilon$), subject to the following rules:
    \begin{enumerate}
        \item Empty string: $r=\varepsilon$.
        \item Unitary language: $r=a$.
        \item Union of expressions: $r=s \cup t$.
        \item Concatenation of expressions: $r=(st)$.
        \item Iteration of an expression: $r=s^{*}$. 
    \end{enumerate}
    Here, the symbols $s$ and $t$ represent regular expressions.
\end{definition}
The operator precedence is as follows: star has the highest precedence, followed by concatenation, and then union.
\newpage
In addition to these operators, we often make use of derived operators:
\begin{itemize}
    \item $\varepsilon$, defined as $\varepsilon=\varnothing^{*}$. 
    \item $e^{+}$, defined as $e \cdot e^{*}$. 
\end{itemize}
The interpretation of a regular expression $r$ corresponds to a language $L_r$ over the alphabet $\Sigma$, as outlined in the following table:
\begin{table}[H]
    \centering
    \begin{tabular}{cc}
    \hline
    \textbf{Expression $\boldsymbol{r}$} & \textbf{Language $\boldsymbol{L_r}$} \\ \hline
    $\varnothing$                        & $\varnothing$                        \\
    $\varepsilon$                        & $\{\varepsilon\}$                    \\
    $a \in \Sigma$                       & $\{a\}$                              \\
    $s \cup t \textnormal{ or } s|t$     & $L_s \cup L_t$                       \\
    $s \cdot t \textnormal{ or } st$     & $L_s \cdot L_t$                      \\
    $s^{*}$                              & $L_s^{*}$                            \\ \hline
    \end{tabular}
\end{table}
\begin{definition}
    A \emph{regular language} is a language that is represented by a regular expression.
\end{definition}
\begin{example}
    The regular expression $e=(111)^{*}$ represents the language $L_e=\{\varepsilon,111,111111,\dots\}$. 

    The regular expression $e_1=11(1)^{*}$ represents the language $L_e=\{11,111,1111,11111,\dots\}$. 
\end{example}
\begin{definition}
    The \emph{family of regular languages}, denoted as REG, is the collection of all regular languages. 

    The \emph{family of finite languages}, denoted as FIN, is the collection of all languages with finite cardinality.
\end{definition}
Every finite language is considered regular because it can be expressed as the union of a finite number of strings, each of which is formed by concatenating a finite number of alphabet symbols: 
\[\left(x_1 \cup x_2 \cup \dots \cup x_k \right) = \left( a_{1_1}a_{1_2}\dots a_{1_n} \cup \dots \cup a_{k_1}a_{k_2}\dots a_{k_m} \right)\]
It's important to note that the family of regular languages includes languages with infinite cardinality as well.
Therefore, we can conclude that $\textnormal{FIN} \subset \textnormal{REG}$.

The union and repetition operators in regular expressions correspond to possible choices, allowing for the creation of sub-expressions that identify specific sub-languages.
\begin{table}[H]
    \centering
    \begin{tabular}{cc}
    \hline
    \textbf{Expression $\boldsymbol{r}$}                                    & \textbf{Choice of $\boldsymbol{r}$}                       \\ \hline
    $e_1 \cup \dots \cup e_n \textnormal{ or } e_1 | \dots | e_n$           & $e_k$ for every $1 \leq k \leq n$                         \\
    $e^{*}$                                                                 & $\varepsilon$ or $e^n$ for every $n \geq 1$               \\
    $e^{+}$                                                                 & $e^n$ for every $n \geq 1$                                \\ \hline
    \end{tabular}
\end{table}
When working with a regular expression, it's possible to derive a new one by replacing any outermost sub-expression with another that represents a choice of it.
\newpage
\begin{definition}
    We state that a regular expression $e^{'}$ \emph{derives} a regular expression $e^{''}$, denoted as $e^{'} \implies e^{''}$, when the two regular expressions can be factorized as:
    \[e^{'}=\alpha \beta \gamma\]
    \[e^{''}=\alpha \delta \gamma\]
    Here, $\delta$ represents a choice involving $\beta$.
\end{definition}
The derivation relation can be applied iteratively, resulting in the following relations:
\begin{itemize}
    \item Power of $n$: $\overset{n}{\implies}$ with $n \in \mathbb{N}$. 
    \item Transitive closure: $\overset{*}{\implies}$ with $n \geq 0$. 
    \item Reflexive transitive closure: $\overset{+}{\implies}$ with $n > 0$.
\end{itemize}
\begin{example}
    The expression $e_0 \overset{n}{\implies} e_n$ implies that $e_n$ is derived from $e_0$ in $n$ steps. 

    The expression $e_0 \overset{+}{\implies} e_n$ implies that $e_n$ is derived from $e_0$ in $n \geq 1$ steps. 

    The expression $e_0 \overset{*}{\implies} e_n$ implies that $e_n$ is derived from $e_0$ in $n \geq 0$ steps. 
\end{example}
Some derived regular expressions incorporate metasymbols, including operators and parentheses, while others consist solely of symbols from the alphabet $\Sigma$, also known as terminals, and the empty string $\varepsilon$. 
These latter define the language specified by the regular expression.

It's essential to note that in derivations, operators must be selected from the external to the internal layers.
Making a premature choice could eliminate valid sentences from consideration.
\begin{definition}
    Two regular expressions are considered \emph{equivalent} if they define the same language. 
\end{definition}

\subsection*{Ambiguity}
A phrase from a regular language can be derived through different equivalent derivations. 
These derivations may vary in the order of the choices made during the derivation process.
To determine the expression that can be derived in multiple ways, we need to establish the numbered subexpressions of a regular expression. 
To achieve this, follow these steps:
\begin{itemize}
    \item Begin with a regular expression and consider all possible parentheses.
    \item Derive a numbered version, denoted as $e_N$, of the original regular expression, $e$.
    \item Identify all the numbered subexpressions within the expression.
\end{itemize}
\begin{example}
    Taking the regular expression $e=(a \cup(bb))^{*}(c^{+} \cup(a\cup(bb)))$, the corresponding numbered regular expression is:
    \[e_N=(a_1\cup(b_2b_3))^{*}(c_4^{+} \cup(a_5\cup(b_6b_7)))\]
    From this expression, we can derive its subexpressions by iteratively removing the parentheses and union symbols.
\end{example}
\newpage
\begin{definition}
    A regular expression is considered \emph{ambiguous} of its numbered version, denoted as $f'$, contains two distinct strings, $x$ and $y$, that become identical when the numbers are removed.
\end{definition}
\begin{example}
    Taking the regular expression $e=(aa|ba)^{*}a|b(aa|b)^{*}$, its corresponding numbered version is $e_N=(a_1a_2|b_3a_4)^{*}a_5|b_6(a_7a_8|b_9)^{*}$.
    
    From this expression, we can derive $b_3a_4a_5$ and $b_6a_7a_8$, both of which map to the string $baa$. 
    Consequently, it can be concluded that the regular expression $e$ is ambiguous.
\end{example}
Ambiguity is often a source of problems. 

\subsection*{Extended regular expressions}
To define a regular expression, we can introduce the following operators without altering its expressive power: 
\begin{itemize}
    \item Power: $a^h=aa\dots a$ for $h$ times. 
    \item Repetition: $[a]^n_k=a^k \cup a^{k+1} \cup \dots \cup a^n$.
    \item Optionality: $(\varepsilon \cup a)$ or $[a]$.
    \item Ordered interval: $(0\dots 9)(a \dots z)(A \dots Z)$.
    \item Intersection: useful to define languages through conjunction of conditions. 
    \item Complement: $\lnot L$.
\end{itemize}

\subsection*{Closure properties of the REG family}
\begin{definition}
    Suppose $op$ represents a unary or binary operator.
    A family of languages is said to be closed under $op$ if and only if every language obtained by applying the $op$ operator to languages within that family remains within the same family.
\end{definition}
\begin{property}
    The REG family is closed under concatenation, union, star, intersection, and complement operators. 
\end{property}
This implies that regular languages can be combined using these operators without going beyond the boundaries of the REG family.
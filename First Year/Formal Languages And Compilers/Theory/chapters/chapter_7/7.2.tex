\section{Lexical analysis}

The purpose of the lexical analysis is:
\begin{enumerate}
    \item To recognize the tokens of the language.
    \item To (possibly) decorate the tokes with additional information.
\end{enumerate}
Such analysis is performed through a scanner, which basically is just a big FSA.
Since coding a scanner is a hard task, scanner generators based on regular expressions such as \texttt{flex} are used.
In a compiler, the scanner prepares the input for the parser:
\begin{enumerate}
    \item It \textit{detects} the tokens of the language.
    \item It \textit{cleans} the input.
    \item It \textit{adds} information to the tokens.
\end{enumerate}

\paragraph*{Words}
Words cannot be enumerated in artificial languages, as there are too many of them (despite being bounded).
However, technical words are simpler than natural words:
\begin{itemize}
    \item Their structure is simple.
    \item They follow specific rules.
    \item They are (normally) a regular language.
\end{itemize}
\begin{property}[\texttt{C} identifiers]
    The first character must be a letter or an underscore. 
    The following characters must be letters, digits or underscores. 
\end{property}
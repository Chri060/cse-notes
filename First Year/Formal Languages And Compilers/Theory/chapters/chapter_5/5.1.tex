\section{Introduction}

To systematically construct a bottom-up syntax analyzer we have: 
\begin{enumerate}
    \item Construction of the pilot graph: the pilot drives the PDA. In each macro-state the pilot incorporates all the information about any possible phrase form that
        reaches the $m$-state (with lookahead). 
    \item The $m$ states are used to build a few analysis threads in the stack, which correspond to possible derivations: computations of the machine network, or paths 
        with $\varepsilon$-arcs at each machine change, labeled with the scanned string. 
    \item Verification of determinism conditions on the pilot graph: shift-reduce conflicts, reduce-reduce conflicts, and convergence conflicts. 
    \item If the determinism test is passed, the PDA can analyze the string deterministically.
    \item The PDA uses the information stored in the pilot graph and in the slack. 
\end{enumerate}
\begin{definition}
    The \emph{set of initials} is the set of chars found starting from state $q_A$ of machine $M_A$ of the net $M$.
    
    An \emph{item} is: 
    \[\left\langle q_B,a\right\rangle \textnormal{ in } Q \times (\Sigma \cup \{\dashv\})\]

    The function \emph{closure} computes a kind of closure of a set $C$ of items with look-ahead. 

    The \emph{shift operation} is defined as: 
    \[
    \begin{cases}
        \theta(\left\langle p_A,\rho\right\rangle,X)=\left\langle q_A,\rho\right\rangle \textnormal{ if the arc } p_a \rightarrow^X q_a \textnormal{ exists} \\
        \textnormal{the empty set otherwise}
    \end{cases}    
    \]
\end{definition}

The pilot is a DFA, named $\mathcal{P}$, defined by the following entities:
\begin{itemize}
    \item The set $R$ of $m$-states. 
    \item The pilot alphabet is the union $\Sigma\cup V$ of the terminal and non-terminal alphabets, to be also named the grammar symbols.
    \item The initial $m$-state, $I_0$, is the set $I_0=closure(\left\langle 0_S,\dashv \right\rangle )$. 
    \item The $m$-state set $R={I_0,I_1,\dots}$ and the state-transition function $\theta:R \times (\Sigma \cup V) \rightarrow R$ are computed starting from $I_0$. 
\end{itemize}
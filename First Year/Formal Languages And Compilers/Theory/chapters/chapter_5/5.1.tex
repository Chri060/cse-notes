\section{Top-down and bottom-up constructions}

Consider a grammar $G$. If a source string is in the language $L(G)$, a syntax analyzer or parser scans the string and computes a derivation or syntax tree; 
otherwise it stops and prints the configuration where the error was detected (diagnosis); afterwards it may resume parsing and skip the substrings contaminated 
by the error (error recovering), in order to offer as much diagnostic help as possible with a single scan of the source string. If the source string is ambiguous, 
the result of the analysis is a set of trees, also called tree forest. 

We know the same syntax tree corresponds to many derivations. Depending on the derivation being leftmost or rightmost and on the order it is constructed, we obtain 
two important parser classes: 
\begin{itemize}
    \item Top-down analysis: constructs the leftmost derivation by starting from the axiom.
    \item Bottom-up analysis constructs the rightmost derivation but in the reversed order.
\end{itemize}
\section{Grammar as network of finite automata}

Let $\Sigma$ and $V=\{S,A,B,\dots\}$ be, respectively, the terminal alphabet and non-terminal alphabet, and $S$ be the axiom of an extended context-free grammar $G$.

For each non-terminal $A$ there is exactly one (extended) grammar rule $A\rightarrow\alpha$ and the right part $\alpha$ of the rule is a regular expression over the 
alphabet $\Sigma \cup V$.

Let the grammar rules be denoted by $S\rightarrow\sigma,A\rightarrow\alpha,B\rightarrow\beta,\dots$. The symbols $R_S,R_A,R_B,\dots$ denote the regular languages 
over the alphabet $\Sigma \cup V$, definedby the regular expression $\sigma,\alpha,\beta,\dots$, respectively. 

The symbols $M_S,M_A,M_B,\dots$ are the names of the (finite deterministic) machines accepting the corresponding regular languages $R_S,R_A,\dots$ The set of all 
such machines is denoted by symbol $\mathcal{M}$.

To prevent confusion, the names of the states of any two machines are made disjoint, say, by appending the machine name as a subscript. The state set of a machine
$M_A$ is denoted $Q_A={0_A,\dots,q_A,\dots}$, its only initial state is $0_A$ and its set of final states is $F_A\subseteq Q_A$. The state set $Q$ of a net $\mathcal{M}$
is the union of all states:
\[Q=\bigcup_{M_a \in \mathcal{M}}{Q_A}\]
The transition function of all machines will be denoted by the same name $\delta$ as for the individual machines, at no risk of confusion as the machine state sets 
are all disjoint.

For a state $q_A$, the symbol $R(M_A,q_A)$ or for brevity $R(q_A)$, denotes the regular language over the alphabet $\Sigma \cup V$, accepted by the machine $M_A$ 
starting from state $q_A$. For the initial state, we have $R(0_A)\equiv R_A$.

It is convenient to stipulate that for every machine $M_A$, there is no arc as with $c \in \Sigma \cup V$, which enters the initial state $0_A$. Such a normalization 
ensures that the initial state is not visited twice within a computation that does not leave machine $M_A$.

We need to consider also the terminal language defined by a generic machine $M_A$, when starting from a state possibly other than the initial one. For any state
$q_A$,not necessarily initial, we write as: 
\[L(M_A,q_A)=L(q_A)=y \in \Sigma^{*}|\eta\in R(q_A) \land \eta^{*} \implies y\]
The formula above contains a string $\eta$ over terminals and non-terminals, accepted by machine $M_A$ when starting in the state $q_A$. The derivations 
originating from $\eta$ produce all the terminal strings of language $L(q_A)$. In particular, from previous stipulations it follows that: 
\[L(M_A,0_A)=L(0_A)\equiv L_A(G)\]
and for the axiom it is:
\[L(M_S,0_S)=L(0_S)=L(M)\equiv L(G)\]
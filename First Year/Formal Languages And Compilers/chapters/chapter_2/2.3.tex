\section{Erroneous grammars}

\begin{definition}
    A grammar $G$ is called \emph{clean} (or reduced) if and only if for every nonterminal $A$: 
    \begin{itemize}
        \item $A$ is reachable from the axiom $S$, and hence contribute to the generation of the language.
            That is, there exists a derivation: 
            \[S \overset{*}{\implies} \alpha A \beta\]
        \item $A$ is defined, that is, it generates a non-empty language: 
            \[L_A(G) \neq \varnothing\]
    \end{itemize}
\end{definition}
Note that the rule $L_A(G) \neq \varnothing$ includes also the case when no derivation from $A$ terminates with a terminal string $s$.

The process of grammar cleaning involves a two-step algorithm:
\begin{enumerate}
    \item Establish the set UNDEF, which comprises undefined nonterminals.
    \item Identify the set of unreachable nonterminals.
\end{enumerate}

\subsection*{Phase one}
We define the set DEF as follows:
\[\textnormal{DEF}:=\{A|( A \rightarrow u ) \in P,\textnormal{with } u \in \Sigma^{*}\}\]
We initiate the process by examining the terminal rules. 
Then, we apply the following update iteratively until a fixed point is reached:
\[\textnormal{DEF}:=\textnormal{DEF} \cup \{B|( B \rightarrow D_1D_2\dots D_n)\in P \land \forall i(D_i \in\textnormal{DEF} \cup \Sigma)\}\]
During each iteration, two cases may occur:
\begin{enumerate}
    \item New nonterminals are discovered, and they have all their right-hand side symbols defined as nonterminals or terminals.
    \item No new nonterminals are found, and the algorithm terminates.
\end{enumerate}
At this stage, the nonterminals in UNDEF are removed.

\subsection*{Phase two}
The produce relation, denoted as $A$ produce $B$, holds if and only if there exists a production rule $(A \rightarrow \alpha B \beta) \in P$, where $A \neq B$ and $\alpha,\beta$ can be any strings.

We can now state that a nonterminal $C$ is reachable from the start symbol $S$ if and only if there exists a path in the graph of the produce relation from $S$ to $C$. 
Nonterminals that are not reachable from the start symbol can be eliminated.

\subsection*{Additional requirement}
In addition to the above cleanliness conditions, a third requirement is often added:
\begin{enumerate}
    \item [3.] $G$ must not allow for circular deviations because they are non-essential and may introduce ambiguity.
\end{enumerate} 
A circular derivation occurs when given $A \overset{+}{\implies} A$, the derivation $A \overset{+}{\implies} x$ is possible, and also $A \overset{+}{\implies} A \overset{+}{\implies} x$ (and many others) are possible.

It's important to note that even if a grammar is clean, it can have redundant rules that lead to ambiguity.
\begin{example}
    Examples of unclean grammars are as follows:
    \[
    \begin{cases}
        S \rightarrow aASb \\
        A \rightarrow b
    \end{cases} 
    \begin{cases}
        S \rightarrow a \\
        A \rightarrow b
    \end{cases} 
    \begin{cases}
        S \rightarrow aASb \\
        A \rightarrow S|b
    \end{cases} 
    \]
    In the first case, the axiomatic rule does not produce any phrase. 
    In the second case, $A$ is not reachable. 
    In the third case, the grammar is circular on $S$ and $A$.
\end{example}
\newpage
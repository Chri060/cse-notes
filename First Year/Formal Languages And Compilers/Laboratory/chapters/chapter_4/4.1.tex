\section{Introduction}

The primary objective of a compiler is to translate a program written in a source language $L_0$ into a semantically equivalent program expressed in a target language $L_1$.

A compiler is structured as a pipeline, where each stage applies a specific transformation to the input program, resulting in an output program. 
Each stage serves a distinct purpose:
\begin{itemize}
    \item \textit{Front-end}: This stage converts the source program into an intermediate form.
    \item \textit{Middle-end}: here, transformations and optimizations are applied to the intermediate form.
    \item \textit{Back-end}: this stage converts the optimized intermediate form into the target machine language.
\end{itemize}
\section{File format}

A FLEX file is divided into three sections delineated by $\%\%$:
\begin{enumerate}
    \item \textit{Definitions}: where helpful regular expressions are declared.
    \item \textit{Rules}: which associate regular expression combinations with actions.
    \item \textit{User code}: typically containing C code, including helper functions.
\end{enumerate}

\subsection{Definitions}
In lexical analysis, a definition links a name to a group of characters. 
Here are some key points about definitions:
\begin{itemize}
    \item Regular expressions are utilized to define character sets.
    \item Literal strings are represented within quotes, and they are counted as a single symbol for precedence purposes.
    \item Definitions are commonly employed to define straightforward concepts, such as digits.
    \item They function similarly to C's preprocessor macros.
    \item To invoke a definition, its name is enclosed within curly braces.
\end{itemize}
\begin{example}
    Here we have some examples of definitions: \\
    \begin{lstlisting}[style=C]
LETTER              [a-zA-Z$\_$] 
DIGIT               [0-9]
HELLOWORLD          "*hello world*"
LETTERDIGIT         {LETTER}{DIGIT}
    \end{lstlisting}
\end{example}

\subsection{Rules}
A rule defines a complete token to be identified. Here are the key aspects of rules:
\begin{itemize}
    \item The token is characterized by a regular expression.
    \item Rules utilize definitions to create composite concepts, like numbers or identifiers.
    \item Each match triggers a semantic action specified within the rule.
\end{itemize}
\begin{example}
    Considering the definitions given in the previous example we can create for instance the following definitions:  \\
    \begin{lstlisting}[style=C]
{LETTER}({LETTER}|{DIGIT})*         {return 1;}
{DIGIT}+                            {return 2;}
[ t]+                               /* do nothing */
"if"                                {return 3;}
    \end{lstlisting}
\end{example}
Semantic actions, executed each time a rule is matched and have access to the matched textual data.
The global variables defined for semantic actions are: 
\begin{itemize}
    \item \texttt{yytext} of type \texttt{char*} that contains the matched text. 
    \item \texttt{yyleng} of type \texttt{int} that contains the matched text. 
\end{itemize}

In straightforward applications, the business logic is often embedded directly within semantic actions.
However, in more intricate applications like compilers that employ a separate parser, the approach involves:
\begin{enumerate}
    \item Assigning a value to the recognized token, known as the lexical value.
    \item Returning the token type.
\end{enumerate}

\subsection{User code}
The user-provided C code is directly replicated into the generated scanner without alteration. 
It's advantageous to include the following components:
\begin{itemize}
    \item The main function.
    \item Any additional routines called by semantic actions.
    \item Scanner-wrapping routines.
\end{itemize}

Arbitrary code can be inserted within the definitions and rules sections by escaping from FLEX using $\%\{$, $\%\}$ braces. 
This code is directly copied into the generated scanner without modification.
It's commonly utilized for tasks such as header inclusions, defining global variables, and declaring functions.
\section{Generated scanner}

The scanner generated by FLEX is a C file called \texttt{lex.yy.c}. 
It exports: \\ 
\texttt{FILE *yyin = stdin;} \\
\texttt{int yylex(void);} \\
The \texttt{yylex} function parses the file \texttt{yyin} until a semantic action returns or the file ends (return value zero). 

\paragraph*{File ending}
FLEX necessitates the implementation of a single function: \texttt{int yywrap(void)}. 
This function is invoked when the file concludes, offering the chance to open another file and resume scanning from that point onward: 
\begin{itemize}
    \item Return 0 to indicate that parsing should continue.
    \item Return 1 to signify that parsing should cease.
\end{itemize}
If this behavior is undesired, the following line must be included in the scanner source: 
\texttt{\%option noyywrap}.

\paragraph*{Scanner behaviour}

The behavior of the scanner is governed by the following rules:
\begin{itemize}
    \item \textit{Longest matching rule}: if multiple matching strings are found, the rule that generates the longest match is selected.
    \item \textit{First rule}: in case of multiple strings with the same length being matched, the rule listed first will be triggered.
    \item \textit{Default action}: if no rules are found, the next character in the input is implicitly considered matched and printed to the output stream as is.
\end{itemize}

\paragraph*{Scanner workflow}
The generated parser operates as a non-deterministic finite state automaton:
\begin{itemize}
    \item The automaton endeavors to match all potential tokens simultaneously.
    \item Upon recognizing one:
        \begin{enumerate}
            \item The associated semantic action is executed.
            \item The stream advances beyond the end of the token.
            \item The automaton resets.
        \end{enumerate}
\end{itemize}
In practice, the NFA is converted into a deterministic automaton using a modified version of the Berry-Sethi algorithm.
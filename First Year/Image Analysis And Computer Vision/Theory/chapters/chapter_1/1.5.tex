\section{Pin-hole camera}

To achieve a sharply focused image, it is essential that each light ray converges precisely onto a single pixel on the camera's focal plane. 
To ensure this, we must satisfy the following conditions:
\begin{itemize}
    \item The distance between the lens and the source of the ray, denoted as $Z(P)$, should be significantly greater than the lens aperture $a$, preferably at least a factor of 1000 times.
    \item By positioning the screen at a distance $Z$ from the lens, we enable all the rays to maintain parallel trajectories as they pass through the lens, resulting in a well-focused image.
\end{itemize}
The camera described so far is commonly known as a pin-hole camera, and it requires the following characteristics:
\begin{enumerate}
    \item A thin spherical lens.
    \item Utilization of small angles.
    \item Ensuring that $Z(P) \gg a$.
    \item Maintaining $Z=f$, where $f$ represents the focal length.
\end{enumerate}
\section{Introduction}

When we aim to recover a model of an unknown planar scene based on an image of the scene, which is a projective transformation denoted as $x_i^{'}=Hx_i$, we face a significant challenge.
The key issue is that we know the values of $x_i$ (the scene points), but we don't have direct knowledge of the transformation matrix $H$, which complicates a straightforward inversion of the mapping.

The general problem is inherently unsolvable due to the excessive number of unknown variables. 
To tackle this challenge, we can adopt two primary strategies:
\begin{enumerate}
    \item Reduce unknowns: in many cases, it is unnecessary to precisely recover the original scene configuration. 
        Instead, the objective is to retrieve the overall shape of the scene, known as shape reconstruction. 
        By doing so, we can reduce the number of unknowns from eight to four, and the matrix $H$ takes on the following form:
        \[H=    
        \begin{bmatrix}
            s\cos \vartheta & -s\sin \vartheta & t_x \\
            s\sin \vartheta & s\cos \vartheta & t_y \\
            0 & 0 & 1
        \end{bmatrix}\]
        Additionally, it's possible to perform similarity reconstruction, which reduces the unknowns by two, or affine reconstruction, which reduces the unknowns by six.
    \item Add constraints: this strategy involves utilizing extra information to recover a model of the scene. 
        The valuable information typically pertains to parameters that remain invariant under the desired class of mappings but are not invariant under more general classes.
\end{enumerate}

The reconstruction can fall into one of two categories:
\begin{itemize}
    \item Affine reconstruction: in this scenario, the reconstructed scene is an affine mapping of the original scene.
    \item Shape reconstruction: in this case, the reconstructed scene follows a similarity mapping of the original scene, aiming to capture the overall shape while simplifying the problem.
\end{itemize}
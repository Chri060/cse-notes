\section{Planes}

In the homogeneous coordinates, planes are defined by using the matrix: 
\[\pi=\begin{bmatrix} a \\ b \\c \\d \end{bmatrix}\]
Here, the direction normal to the plane is given by $(a, b, c)$, and the distance from the origin to the plane is calculated as:
\[\textnormal{distance}=-\dfrac{d}{\sqrt{a^2+b^2+c^2}}\]
Similar to homogeneous point coordinates, this representation of planes also exhibits the homogeneity property.
Any vector $\pi$ is equivalent to all its non-zero multiples $\lambda \pi$, where $\lambda \neq 0$, as they all represent the same plane.
The parameters $a, b, c, d$ are referred to as the homogeneous parameters of the plane.
As with points, there are an infinite number of equivalent representations for a single plane, which includes all non-zero multiples of the unit normal vector.
The null vector does not represent any plane. 
If $d=0$, it signifies that the plane $\pi$ passes through the origin of space.

To determine whether a point lies on a plane or if a plane passes through a point, you can solve the following system of equations:
\[
\begin{cases}
    ax+by+cz+dw=0 \\
    \pi^TX=X^T\pi=0
\end{cases}
\]
\begin{definition}
    The plane 
    \[\begin{bmatrix} 0 & 0 & 0 & 1 \end{bmatrix} \begin{bmatrix} x \\ y \\ z \\ w \end{bmatrix}=w=0\] 
    is called the \emph{plane at the infinity} $\pi_{\infty}={\begin{bmatrix} 0 & 0 & 0 & 1 \end{bmatrix}}^T$. 
\end{definition}
It's important to note that this plane has an undefined normal direction.
\section{Points}

To define points in Cartesian coordinates, a Euclidean plane is established, with a designated origin. 
Each point is uniquely represented by a pair of Cartesian coordinates, $(x, y)$.

For the analysis of images, it is advantageous to use homogeneous coordinates.
To define this coordinate system, we construct a 3D space with axes labeled $x$, $y$, and $w$. 
Now, to represent a point, we assign three values. 
This implies that every point can have an infinite number of representations by altering the value of $w$.

To analyze images it is better to use homogeneous coordinates. To define this type of coordinates we need to construct a 3D space with $x,y,w$ axis. So, now to define a point
we have to assign three values. This means that every number can be represented in infinite ways by changing the value of $w$. 
The relationship between Cartesian and homogeneous coordinates can be expressed as follows:
\[
x=
\begin{bmatrix}
    x \\
    y \\
    w 
\end{bmatrix}
=w
\begin{bmatrix}
    X \\
    Y \\
    1 
\end{bmatrix}
\]

Consequently, a vector $x = {\begin{bmatrix} x & y & w \end{bmatrix}}^T$ and all its nonzero multiples, including ${\begin{bmatrix} \frac{x}{w} & \frac{y}{w} & 1 \end{bmatrix}}^T$, represent the same point in Cartesian coordinates ${\begin{bmatrix} X & Y \end{bmatrix}}^T={\begin{bmatrix}  \frac{x}{w} &  \frac{y}{w} \end{bmatrix}}^T$ on the Euclidean plane. 
This representation adheres to the homogeneity property: any vector $x$ is equivalent to all its nonzero multiples $\lambda x$, where $\lambda \neq 0$, since they denote the same point.
The null vector does not represent any point.
\newpage
\begin{definition}
    Let's define the \emph{projective plane} as:
    \[\mathbb{P}^2=\{{\begin{bmatrix} x & y & w \end{bmatrix}}^T \in \mathbb{R}^3\}-\{{\begin{bmatrix} 0 & 0 & 0 \end{bmatrix}}^T\}\]
\end{definition}
\begin{example}
    The origin of the plane is defined as:
    \[{\begin{bmatrix} 0 & 0 & 1 \end{bmatrix}}^T\]
    A generic point in homogeneous coordinates can be easily transformed into a pair of Cartesian coordinates by a straightforward division. 
    For example, the point:
    \[{\begin{bmatrix} 0 & 8 & 4 \end{bmatrix}}^T\]
    in Cartesian coordinates is equal to:
    \[{\begin{bmatrix} \frac{x}{w} & \frac{y}{w} \end{bmatrix}}^T=\begin{bmatrix} \frac{0}{4} & \frac{8}{4} \end{bmatrix}=\begin{bmatrix} 0 & 4 \end{bmatrix}\]
\end{example}
Consider a point $x={\begin{bmatrix} x & y & w \end{bmatrix}}^T$, and let $w$ slowly decrease from $w=1$. 
As $w$ decreases, moves along a constant direction $\begin{bmatrix} x & y \end{bmatrix}$, while distancing itself from the origin. 
As $w$ approaches $0$, tends towards infinity along the direction $\begin{bmatrix} x & y \end{bmatrix}$. 
\begin{definition}
    We define the \emph{point at the infinity along the direction} $\begin{bmatrix} x & y \end{bmatrix}$ as: 
    \[x=\begin{bmatrix} x \\ y \\ w \end{bmatrix}\]
\end{definition}
Points at the infinity, representing directions, exist outside the Euclidean plane, and they are additional points well-defined within the projective plane. 
Therefore, the projective plane encompasses not only the Euclidean plane but also these points at infinity.
\documentclass[12pt, a4paper]{report}
\usepackage{graphicx, array, amsthm, amssymb, amsmath, algorithm, algpseudocode, float, xcolor, thmtools, thmbox, matlab-prettifier, exercise}
\usepackage[english]{babel}

\makeatletter
\renewcommand\thmbox@headstyle[2]{\bfseries #1}
\makeatother
\newtheorem[style=M,bodystyle=\normalfont]{theorem}{Theorem}
\newtheorem[style=M,bodystyle=\normalfont]{corollary}{Corollary}
\newtheorem[style=M,bodystyle=\normalfont]{lemma}{Lemma}
\newtheorem[style=M,bodystyle=\normalfont]{definition}{Definition}


\title{Numerical Analysis\\ \textit{Exercises}}
\author{Christian Rossi}
\date{Academic Year 2023-2024}

\begin{document}

\maketitle

\newpage

\begin{abstract}
    The topics of the course are:
    \begin{itemize}
        \item Floating-point arithmetic: different sources of the computational error; absolute vs relative errors; the floating point representation 
            of real numbers; the round-off unit; the machine epsilon; floating-point operations; over- and under-flow; numerical cancellation.
        \item Numerical approximation of nonlinear equations: the bisection and the Newton methods; the fixed-point iteration; convergence analysis 
            (global and local results); order of convergence; stopping criteria and corresponding reliability; generalization to the system of 
            nonlinear equations (hints).
        \item Numerical approximation of systems of linear equations: direct methods (Gaussian elimination method; LU and Cholesky factorizations; 
            pivoting; sparse systems: Thomas algorithm for tridiagonal systems); iterative methods (the stationary and the dynamic Richardson scheme; 
            Jacobi, Gauss-Seidel, gradient, conjugate gradient methods (hints); choice of the preconditioner; stopping criteria and corresponding 
            reliability); accuracy and stability of the approximation; the condition number of a matrix; over- and under-determined systems: the 
            singular value decomposition (hints).
        \item Numerical approximation of functions and data: Polynomial interpolation (Lagrange form); piecewise interpolation; cubic interpolating 
            splines; least-squares approximation of clouds of data.
        \item Numerical approximation of derivatives: finite difference schemes of the first and second order; the undetermined coefficient method.
        \item Numerical approximation of definite integrals: simple and composite formulas; midpoint, trapezoidal, Cavalieri-Simpson quadrature rules; 
            Gaussian formulas; degree of exactness and order of accuracy of a quadrature rule. 
        \item Numerical approximation of ODEs: the Cauchy problem; one-step methods (forward and backward Euler and Crank-Nicolson schemes); 
            consistency, stability, and convergence (hints).
    \end{itemize}
\end{abstract}

\newpage

\tableofcontents

\newpage

\chapter{Introduction to MATLAB}
    \section{Main MATLAB operators}
    Assignment operator: 
    \begin{lstlisting}[frame=single, numbers=none, style=Matlab-bw]
% Print output
a = 1 
% Does not print output
b = 2;
    \end{lstlisting}
    The active variables can be found in the workspace and the value can be checked on the command window with: 
    \begin{lstlisting}[frame=single, numbers=none, style=Matlab-bw]
% Value of all variables
whos
% Value of a
whos a
    \end{lstlisting}
    If you want to save the file: 
    \begin{lstlisting}[frame=single, numbers=none, style=Matlab-bw]
% Save the command history
diary file_name.txt 
% Save the whole workspace
save file_name 
% Save only the variable a
save file_name_only_a a 
% Load only the variable a
load file_name_only_a 
% Load the whole workspace
load file_name 
    \end{lstlisting}
    It is possible to clear variables with the following commands: 
    \begin{lstlisting}[frame=single, numbers=none, style=Matlab-bw]
% Clear only the variable a
clear a 
% Clear the whole workspace
clear all 
    \end{lstlisting}

    \section{Vector and matrices}
    Most of the entities in MATLAB are matrices, even real numbers. The matrices can be defined in the following ways: 
    \begin{lstlisting}[frame=single, numbers=none, style=Matlab-bw]
% Row vector definition
c = [1 2 3]
% Column vector definition
c = [1; 2; 3]
% Vectorn transposition
c = [1 2 3]'
% 2D matrix definition
D = [ 1 2 3; 
        4 5 6; 
        7 8 9 ]
    \end{lstlisting}
    It is also possible to define various types of matrices: 
    \begin{lstlisting}[frame=single, numbers=none, style=Matlab-bw]
% Zeros vector/matrix
A = zeros(row_length,column_length)	
% Ones vector/matrix
A = ones(row_length,column_length)	
% Identity matrix
A = eye(row_length,column_length)   
% Diagonal matrix
d = [1:4]
D = diag(d)
% Set a not principal diagonal 
D = diag(d, diagonal_index)
% Select only upper o lower trinagular
Ml = tril(M)
Mu = triu(M)
% Access an element in vector
C(1)
% Access an element in matrix
C([2,3]);
% Access a part of the matrix
Q(rows,columns)    
% Access the element in position (n,m)
Q(end, end)   
% Dimension of a matrix
length(a);
numel(b);
size(a);
    \end{lstlisting}  
    The operations on vectors are done in the following way: 
    \begin{lstlisting}[frame=single, numbers=none, style=Matlab-bw]
% Given two row vectors a and b
% Vector sum
a + b    
% Vector difference
a - b   
% Scalar product
a * b'      
dot(a,b)   
% Tensor product
a' * b      
% Elementwise product
a .* b  
% Elementwise division    
a ./ b  
% Elementwise exponentiation    
a .^ 2      
    \end{lstlisting}  
    The operations on matrices are done in the following way: 
    \begin{lstlisting}[frame=single, numbers=none, style=Matlab-bw]
% Givcen two matrices A and B (both 3x2)
% Matrix sum
A + B
% Matrix difference
A - B
% Matrix product
K * L'
% Elementwise product
A .* B
% Elementwise division
A ./ B      
% Elementwise exponentiation
A .^ 2      
% Power matrix (useful only square)
A ^ 2         
% Other useful values of the matrices
% Determinant
det(A)
% Trace
trace(A)  
% Inverse of small matrix      
inv(A)          
% Given a column vector b ths olutio of Ax=b
A \ b       
    \end{lstlisting}  
    The function used to plot a graph are the following: 
    \begin{lstlisting}[frame=single, numbers=none, style=Matlab-bw]
% To plot y=f(x) in [a,b]
x = a:step_length:b;
y = f(x);   
figure         
plot(x,y,color)
% To add y2=f2(x) in [c,d]
hold on 
x2 = c:step_length:d;
y2 = f2(x);
plot(x2,y2,color)     
% Show graph's grid
grid on 
% Set the axis limit
axis([xmin xmax ymin ymax]) 
% Set the same scaling for both axis
axis equal 
    \end{lstlisting} 
    To handle functions the commands are: 
    \begin{lstlisting}[frame=single, numbers=none, style=Matlab-bw]
% Define a function handle to g(x)
f = @g(x);
% Evaluation of f in a
f(a) 
% Define an anonymous function
% It is useful to modify other functions
f = @(argument-list) expression
    \end{lstlisting} 
    The operators that u logical values are:  
    \begin{lstlisting}[frame=single, numbers=none, style=Matlab-bw]
% Smaller than
a < b     
% Greater than
a > b
% Smaller or equal than
a <= b   
% Equal to
a == b    
% Different from
a ~= b  
% And
(a < b) & (b > c)  
% Or   
(a < b) | (b > c)     
    \end{lstlisting} 
    The control-flow statement are: 
    \begin{lstlisting}[frame=single, numbers=none, style=Matlab-bw]
% if-then-else statements
if (condition1)
    block1
elseif (condition2)
    block2
else
    block3
end
% for loops
for (index=start:step:end)
    instruction block
end
% while loops
while (condition)
    instruction block
end
    \end{lstlisting}    
    There are two categories of m-files: 
    \begin{itemize}
        \item Scripts: these files contain instructions that are executed in sequence in the command line if the script file is called. 
            The variables are saved in the current workspace.
        \item Functions: they take some input arguments and return some outputs after a series of instructions are performed. 
            The variables defined in the function are local to the scope of the function itself.
    \end{itemize}

\newpage

\chapter{Laboratory I}
    \begin{Exercise}[label=1]
        Define the row vector: 
        \[ \bar{v_k} = [1,9,25,\dots,\left( 2k+1 \right)^2] \in \mathbb{R} \]
        with $k=8$ using the following strategies:
        \begin{enumerate}
            \item A for loop to define one by one each element of the vector.
            \item The vector syntax to build it in just one shot.
        \end{enumerate}
    \end{Exercise}
    \begin{Answer}[ref=1]
        \begin{lstlisting}[frame=single, numbers=left, style=Matlab-bw]
% Point one
k = 8;
A = 0:k;
for i=1:1:(k + 1)
    A(i) = (2 * (i - 1) + 1).^2;
end

% Point two
k = 8;
A = 0:k;
B = 2 * A + 1;
C = B.^2;
        \end{lstlisting}  
    \end{Answer}

    \newpage

    \begin{Exercise}[label=2]
        Define a function which, for an input value $k$, returns the corresponding vector $v_k$ as defined in the previous exercise.
    \end{Exercise}
    \begin{Answer}[ref=2]
        \begin{lstlisting}[frame=single, numbers=left, style=Matlab-bw]
function A = vector(k)
    A = 0:k;
    B = 2 * A + 1;
    A = B.^2;
end
        \end{lstlisting}
    \end{Answer}

    \newpage

    \begin{Exercise}[label=3]
        Using the function of the previous exercise write another function that returns, for a generic value $k$, the $2(k + 1) \times 2(k + 1)$ matrix.
        \[
            m_k=
            \begin{bmatrix}
                2 & 0 & 0 & 0 & 0 & 0 & \cdots & 0 & 0 \\
                0 & \sqrt[2]{2} & 0 & 0 & 0 & 0 & \cdots & 0 & 1 \\
                0 & 0 & \sqrt[3]{2} & 0 & 0 & 0 & \cdots & 0 & 0 \\
                0 & 0 & 0 & \sqrt[4]{2} & 0 & 0 & \cdots & 0 & 9 \\
                0 & 0 & 0 & 0 & \sqrt[5]{2} & 0 & \cdots & 0 & 0 \\
                0 & 0 & 0 & 0 & 0 & \sqrt[6]{2} & \cdots & 0 & 25 \\
                \vdots  & \vdots  & \vdots  & \vdots  & \vdots  & \vdots  & \ddots & \vdots  & \vdots  \\
                0 & 0 & 0 & 0 & 0 & 0 & \cdots & \sqrt[(2k+1)]{2} & 0 \\
                1 & 1 & 9 & 9 & 25 & 25 & \cdots & (2k+1)^2 & (2k+1)^2 
            \end{bmatrix} 
        \]
    \end{Exercise}
    \begin{Answer}[ref=3]

    \end{Answer}

    \newpage

    \begin{Exercise}[label=4]
        Compare the results of the following code segments:
        \begin{lstlisting}[frame=single, numbers=left, style=Matlab-bw]
% Code A
x = 0;
while (x ~= 1)
    x = x + 1/16
end

% Code B
x = 0;
while (x ~= 1)
    x = x + 0.1
end
        \end{lstlisting}
    \end{Exercise}
    \begin{Answer}[ref=4]
        Code A work as expected: the while loop is repeated $16$ times, and the final value of $x$ is $1$. Instead, code B does not work as expected, and results in an infinite loop.
    \end{Answer}
    
    \newpage

    \begin{Exercise}[label=5]
        Find the machine epsilon by implementing an ad hoc procedure. Comment and justify the obtained results.
    \end{Exercise}
    \begin{Answer}[ref=5]
        b
    \end{Answer}
    
    \newpage

    \begin{Exercise}[label=6]
        \begin{lstlisting}[frame=single, numbers=left, style=Matlab-bw]
realmax
a = 1.0e+308;
b = -a;
c = 1.1e+308;
(a + b) + c
(a + c) + b  
        \end{lstlisting}
    \end{Exercise}
    \begin{Answer}[ref=6]
        b
    \end{Answer}

    \newpage

    \begin{Exercise}[label=7]
        Consider the following function:
        \[f(x)=\dfrac{e^x-1}{x}\]
        \begin{enumerate}
            \item Evaluate $f(x)$ for values of $x$ around zero (try with $x_k = 10^{-k}$, $k \in [1, 20]$). What do you obtain? Explain the results.
            \item Propose an approach for fixing the problem. (Hint: Use Taylor expansions to get an approximation of $f(x)$ around $x = 0$). 
            \item How many terms in the Taylor expansion are needed to get double precision accuracy (16 decimal digits) $\forall x \in \left[0, \dfrac{1}{2}\right]$?
        \end{enumerate}
    \end{Exercise}
    \begin{Answer}[ref=7]
        b
    \end{Answer}

    \newpage

    \begin{Exercise}[label=8]
        The sequence: 
        \[1, \dfrac{1}{3}, \dfrac{1}{9}, \dots, \dfrac{1}{3^n},\dots\]
        can be generated with the following recursive relations: 
        \[
        \begin{cases}
            p_n=\dfrac{10}{3}p_{n-1}=p_{n-2}  \\
            p_1 = \dfrac{1}{3}, \: p_0=1
        \end{cases}
        \]
        \[
        \begin{cases}
            q_n=\dfrac{1}{3}q_{n-1}  \\
            q_0=1
        \end{cases}
        \]
        \begin{enumerate}
            \item Implement the two relations in order to generate the first $100$ terms of the sequence.
            \item Study the stability of the two algorithms and justify the obtained results. 
        \end{enumerate}
    \end{Exercise}
    \begin{Answer}[ref=8]
        
    \end{Answer}

    \newpage

    \begin{Exercise}[label=9]
        Find the minimum positive number representable in MATLAB/Octave by implementing an ad hoc procedure. Compare with $realmin$.
    \end{Exercise}
    \begin{Answer}[ref=9]
        
    \end{Answer}

    \newpage

    \begin{Exercise}[label=10]
        \begin{enumerate}
            \item Use Taylor polynomial approximation to avoid the loss of significance errors in the following function when $x$ approaches $0$:
                \[f(x)=\dfrac{1-\cos(x)}{x^2}\]
            \item Reformulate the following function $g(x)$ to avoid the loss of significance error in its evaluation for increasing values of $x$ towards $+ \infty$:
                \[g(x)=x \left( \sqrt{x+1} - \sqrt{x} \right)\]
        \end{enumerate}
    \end{Exercise}
    \begin{Answer}[ref=10]
        
    \end{Answer}

    \newpage

    \begin{Exercise}[label=11]
        We can compute $e^{-x}$ around $x=0$ using Taylor polynomials in two ways, either using:
        \[e^{-x} \approx 1-x-\dfrac{1}{2}x^2+\dfrac{1}{6}x^3+\dots\]
        or using 
        \[e^{-x}=\dfrac{1}{e^x} \approx \dfrac{1}{1-x-\dfrac{1}{2}x^2+\dfrac{1}{6}x^3+\dots}\]
        Which approach is the most accurate?
    \end{Exercise}
    \begin{Answer}[ref=11]
        
    \end{Answer}

    \newpage

    \begin{Exercise}[label=12]
        Consider the following integral: 
        \[I_n(\alpha)=\int_{0}^{1} \dfrac{x^n}{x+\alpha} \,dx \:\:\:\:\:\: \forall n \in \mathbb{N},\alpha>0\]
        \begin{enumerate}
            \item Give an upper bound for $I_n(\alpha)$, $\forall n \in \mathbb{N}, \alpha > 0$.
            \item Prove the following recursive relation between $In(\alpha)$ and $I_{n-1}(\alpha)$:
            \[
                \begin{cases}
                    I_n(\alpha)=-\alpha I_{n-1}(\alpha)+\dfrac{1}{n} \\
                    I_o(\alpha)=\ln{\left( \dfrac{\alpha + 1}{\alpha} \right)}
                \end{cases}
            \]
            \item Employing the previous relation, compute $I_40(\alpha = 8)$ and comment the obtained results. 
            \item Write a numerically stable recursive relation for $I_40(\alpha = 8)$.
        \end{enumerate}
    \end{Exercise}
    \begin{Answer}[ref=12]
        
    \end{Answer}

    \newpage

    \begin{Exercise}[label=13]
        Given the following sequence:
        \[
          \begin{cases}
                x_{n+1}=2^{n+1}\left[ \sqrt{1+\dfrac{x_n}{2^n}}-1 \right] \\
                x_0>-1
          \end{cases}  
        \]
        for which $\lim_{n \rightarrow + \infty}x_n=\ln\left({1+x_0}\right)$
        \begin{enumerate}
            \item Set $x_0 = 1$, compute $x_1, x_2, \dots , x_{71}$ and explain the obtained results.
            \item Transform the sequence in an equivalent one that converges to the theoretical limit.
        \end{enumerate}


    \end{Exercise}
    \begin{Answer}[ref=13]
        
    \end{Answer}
\end{document}
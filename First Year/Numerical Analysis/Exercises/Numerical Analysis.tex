\documentclass[12pt, a4paper]{report}
\usepackage{graphicx, array, amsthm, amssymb, amsmath, algorithm, algpseudocode, float, xcolor, thmtools, thmbox, matlab-prettifier, exercise}
\usepackage[english]{babel}

\makeatletter
\renewcommand\thmbox@headstyle[2]{\bfseries #1}
\makeatother
\newtheorem[style=M,bodystyle=\normalfont]{theorem}{Theorem}
\newtheorem[style=M,bodystyle=\normalfont]{corollary}{Corollary}
\newtheorem[style=M,bodystyle=\normalfont]{lemma}{Lemma}
\newtheorem[style=M,bodystyle=\normalfont]{definition}{Definition}

\definecolor{dkgreen}{rgb}{0,0.6,0}
\definecolor{gray}{rgb}{0.5,0.5,0.5}
\definecolor{mauve}{rgb}{0.58,0,0.82}
\lstset{frame=tb,
  language=Java,
  aboveskip=3mm,
  belowskip=3mm,
  showstringspaces=false,
  columns=flexible,
  basicstyle={\small\ttfamily},
  numbers=none,
  numberstyle=\tiny\color{gray},
  keywordstyle=\color{blue},
  commentstyle=\color{dkgreen},
  stringstyle=\color{mauve},
  breaklines=true,
  breakatwhitespace=true,
  tabsize=3
}

\title{Numerical Analysis\\ \textit{Exercises}}
\author{Christian Rossi}
\date{Academic Year 2023-2024}

\begin{document}

\maketitle

\newpage

\begin{abstract}
    The topics of the course are:
    \begin{itemize}
        \item Floating-point arithmetic: different sources of the computational error; absolute vs relative errors; the floating point representation 
            of real numbers; the round-off unit; the machine epsilon; floating-point operations; over- and under-flow; numerical cancellation.
        \item Numerical approximation of nonlinear equations: the bisection and the Newton methods; the fixed-point iteration; convergence analysis 
            (global and local results); order of convergence; stopping criteria and corresponding reliability; generalization to the system of 
            nonlinear equations (hints).
        \item Numerical approximation of systems of linear equations: direct methods (Gaussian elimination method; LU and Cholesky factorizations; 
            pivoting; sparse systems: Thomas algorithm for tridiagonal systems); iterative methods (the stationary and the dynamic Richardson scheme; 
            Jacobi, Gauss-Seidel, gradient, conjugate gradient methods (hints); choice of the preconditioner; stopping criteria and corresponding 
            reliability); accuracy and stability of the approximation; the condition number of a matrix; over- and under-determined systems: the 
            singular value decomposition (hints).
        \item Numerical approximation of functions and data: Polynomial interpolation (Lagrange form); piecewise interpolation; cubic interpolating 
            splines; least-squares approximation of clouds of data.
        \item Numerical approximation of derivatives: finite difference schemes of the first and second order; the undetermined coefficient method.
        \item Numerical approximation of definite integrals: simple and composite formulas; midpoint, trapezoidal, Cavalieri-Simpson quadrature rules; 
            Gaussian formulas; degree of exactness and order of accuracy of a quadrature rule. 
        \item Numerical approximation of ODEs: the Cauchy problem; one-step methods (forward and backward Euler and Crank-Nicolson schemes); 
            consistency, stability, and convergence (hints).
    \end{itemize}
\end{abstract}

\newpage

\tableofcontents

\newpage

\chapter{Introduction to MATLAB}
    \section{Main MATLAB operators}
    Assignment operator: 
    \begin{lstlisting}[language=Matlab]
% Print output
a = 1 
% Does not print output
b = 2;
    \end{lstlisting}
    The active variables can be found in the workspace and the value can be checked on the command window with: 
    \begin{lstlisting}[language=Matlab]
% Value of all variables
whos
% Value of a
whos a
    \end{lstlisting}
    If you want to save the file: 
    \begin{lstlisting}[language=Matlab]
% Save the command history
diary file_name.txt 
% Save the whole workspace
save file_name 
% Save only the variable a
save file_name_only_a a 
% Load only the variable a
load file_name_only_a 
% Load the whole workspace
load file_name 
    \end{lstlisting}
    It is possible to clear variables with the following commands: 
    \begin{lstlisting}[language=Matlab]
% Clear only the variable a
clear a 
% Clear the whole workspace
clear all 
    \end{lstlisting}

    \section{Vector and matrices}
    Most of the entities in MATLAB are matrices, even real numbers. The matrices can be defined in the following ways: 
    \begin{lstlisting}[language=Matlab]
% Row vector definition
c = [1 2 3]
% Column vector definition
c = [1; 2; 3]
% Vectorn transposition
c = [1 2 3]'
% 2D matrix definition
D = [ 1 2 3; 
        4 5 6; 
        7 8 9 ]
    \end{lstlisting}
    It is also possible to define various types of matrices: 
    \begin{lstlisting}[language=Matlab]
% Zeros vector/matrix
A = zeros(row_length,column_length)	
% Ones vector/matrix
A = ones(row_length,column_length)	
% Identity matrix
A = eye(row_length,column_length)   
% Diagonal matrix
d = [1:4]
D = diag(d)
% Set a not principal diagonal 
D = diag(d, diagonal_index)
% Select only upper o lower trinagular
Ml = tril(M)
Mu = triu(M)
% Access an element in vector
C(1)
% Access an element in matrix
C([2,3]);
% Access a part of the matrix
Q(rows,columns)    
% Access the element in position (n,m)
Q(end, end)   
% Dimension of a matrix
length(a);
numel(b);
size(a);
    \end{lstlisting}  
    The operations on vectors are done in the following way: 
    \begin{lstlisting}[language=Matlab]
% Given two row vectors a and b
% Vector sum
a + b    
% Vector difference
a - b   
% Scalar product
a * b'      
dot(a,b)   
% Tensor product
a' * b      
% Elementwise product
a .* b  
% Elementwise division    
a ./ b  
% Elementwise exponentiation    
a .^ 2      
    \end{lstlisting}  
    The operations on matrices are done in the following way: 
    \begin{lstlisting}[language=Matlab]
% Givcen two matrices A and B (both 3x2)
% Matrix sum
A + B
% Matrix difference
A - B
% Matrix product
K * L'
% Elementwise product
A .* B
% Elementwise division
A ./ B      
% Elementwise exponentiation
A .^ 2      
% Power matrix (useful only square)
A ^ 2         
% Other useful values of the matrices
% Determinant
det(A)
% Trace
trace(A)  
% Inverse of small matrix      
inv(A)          
% Given a column vector b ths olutio of Ax=b
A \ b       
    \end{lstlisting}  
    The function used to plot a graph are the following: 
    \begin{lstlisting}[language=Matlab]
% To plot y=f(x) in [a,b]
x = a:step_length:b;
y = f(x);   
figure         
plot(x,y,color)
% To add y2=f2(x) in [c,d]
hold on 
x2 = c:step_length:d;
y2 = f2(x);
plot(x2,y2,color)     
% Show graph's grid
grid on 
% Set the axis limit
axis([xmin xmax ymin ymax]) 
% Set the same scaling for both axis
axis equal 
    \end{lstlisting} 
    To handle functions the commands are: 
    \begin{lstlisting}[language=Matlab]
% Define a function handle to g(x)
f = @g(x);
% Evaluation of f in a
f(a) 
% Define an anonymous function
% It is useful to modify other functions
f = @(argument-list) expression
    \end{lstlisting} 
    The operators that u logical values are:  
    \begin{lstlisting}[language=Matlab]
% Smaller than
a < b     
% Greater than
a > b
% Smaller or equal than
a <= b   
% Equal to
a == b    
% Different from
a ~= b  
% And
(a < b) & (b > c)  
% Or   
(a < b) | (b > c)     
    \end{lstlisting} 
    The control-flow statement are: 
    \begin{lstlisting}[language=Matlab]
% if-then-else statements
if (condition1)
    block1
elseif (condition2)
    block2
else
    block3
end
% for loops
for (index=start:step:end)
    instruction block
end
% while loops
while (condition)
    instruction block
end
    \end{lstlisting}    
    There are two categories of m-files: 
    \begin{itemize}
        \item Scripts: these files contain instructions that are executed in sequence in the command line if the script file is called. 
            The variables are saved in the current workspace.
        \item Functions: they take some input arguments and return some outputs after a series of instructions are performed. 
            The variables defined in the function are local to the scope of the function itself.
    \end{itemize}

\newpage

\chapter{Laboratory I}
    \begin{Exercise}[label=1]
        Define the row vector: 
        \[ \bar{v_k} = [1,9,25,\dots,\left( 2k+1 \right)^2] \in \mathbb{R} \]
        with $k=8$ using the following strategies:
        \begin{enumerate}
            \item A for loop to define one by one each element of the vector.
            \item The vector syntax to build it in just one shot.
        \end{enumerate}
    \end{Exercise}
    \begin{Answer}[ref=1]
        \begin{lstlisting}[language=Matlab]
k = 8;
% For loop strategy
vk = zeros(1, k+1);
for (ii = 0:k)
    vk(ii+1) = (2*ii + 1)^2;
end
vk
% Vector syntax
vk = [1:2:2*k+1].^2;            
        \end{lstlisting}  
    \end{Answer}

    \newpage

    \begin{Exercise}[label=2]
        Define a function which, for an input value $k$, returns the corresponding vector $v_k$ as defined in the previous exercise.
    \end{Exercise}
    \begin{Answer}[ref=2]
        \begin{lstlisting}[language=Matlab]
function vk = ex_1_2(k)
vk = [1:2:2*k+1].^2;
end
        \end{lstlisting}
    \end{Answer}

    \newpage

    \begin{Exercise}[label=3]
        Using the function of the previous exercise write another function that returns, for a generic value $k$, the $2(k + 1) \times 2(k + 1)$ matrix.
        \[
            m_k=
            \begin{bmatrix}
                2 & 0 & 0 & 0 & 0 & 0 & \cdots & 0 & 0 \\
                0 & \sqrt[2]{2} & 0 & 0 & 0 & 0 & \cdots & 0 & 1 \\
                0 & 0 & \sqrt[3]{2} & 0 & 0 & 0 & \cdots & 0 & 0 \\
                0 & 0 & 0 & \sqrt[4]{2} & 0 & 0 & \cdots & 0 & 9 \\
                0 & 0 & 0 & 0 & \sqrt[5]{2} & 0 & \cdots & 0 & 0 \\
                0 & 0 & 0 & 0 & 0 & \sqrt[6]{2} & \cdots & 0 & 25 \\
                \vdots  & \vdots  & \vdots  & \vdots  & \vdots  & \vdots  & \ddots & \vdots  & \vdots  \\
                0 & 0 & 0 & 0 & 0 & 0 & \cdots & \sqrt[(2k+1)]{2} & 0 \\
                1 & 1 & 9 & 9 & 25 & 25 & \cdots & (2k+1)^2 & (2k+1)^2 
            \end{bmatrix} 
        \]
    \end{Exercise}
    \begin{Answer}[ref=3]
        \begin{lstlisting}[language=Matlab]
function Mk = ex_1_3(k)
% Create a diagonal matrix with the right elements and dimension
Mk = diag(2.^(1./[1:2*(k+1)]))
% The bottom right element will be overwritten
% Last column
Mk(2:2:end, end) = ex_1_2(k)
% Last line
Mk(end, 1:2:end) = ex_1_2(k)
Mk(end, 2:2:end) = ex_1_2(k)
end
        \end{lstlisting}
    \end{Answer}

    \newpage

    \begin{Exercise}[label=4]
        Find the machine epsilon by implementing an ad hoc procedure. Comment and justify the obtained results.
    \end{Exercise}
    \begin{Answer}[ref=4]
        \begin{lstlisting}[language=Matlab]
k = 0;
EPS = 1/2;
while (1 + EPS) > 1
    EPS_old = EPS;    % keep track of the value
    EPS = EPS / 2;
    k = k + 1;
end
format long
EPS_old		
(1 + EPS_old) > 1
EPS
(1 + EPS) > 1
k			% Number of iterations, which is also the numer of digits in the mantissa, according to the standard
eps			 
        \end{lstlisting}
    \end{Answer}

    \newpage

    \begin{Exercise}[label=5]
        Consider the following function:
        \[f(x)=\dfrac{e^x-1}{x}\]
        \begin{enumerate}
            \item Evaluate $f(x)$ for values of $x$ around zero (try with $x_k = 10^{-k}$, $k \in [1, 20]$). What do you obtain? Explain the results.
            \item Propose an approach for fixing the problem. (Hint: Use Taylor expansions to get an approximation of $f(x)$ around $x = 0$). 
            \item How many terms in the Taylor expansion are needed to get double precision accuracy (16 decimal digits) $\forall x \in \left[0, \dfrac{1}{2}\right]$?
        \end{enumerate}
    \end{Exercise}
    \begin{Answer}[ref=5]
        \begin{lstlisting}[language=Matlab]
%   Evaluate f(x) for values of x around zero (try with xk = 10^{-k}, k in [1,20]). What do you obtain? Explain the results.
k = [1:20]';
x = 10.^(-k);
f = @(x) (exp(x) - 1) ./ x;
format long
[k x f(x)]
figure;
plot(x, f(x), '*')
f_taylor_5 = @(x) 1 + 1/2*x + 1/6*x.^2 + 1/24*x.^3 + 1/120*x.^4;
[k x f(x) f_taylor_5(x)]
format short e
n = [1:20]';
err = 1./factorial(n+2) .* (0.5).^(n+1).*exp(0.5);
[n err]
%   Therefore n* = 13$.            
        \end{lstlisting}
    \end{Answer}

    \newpage

    \begin{Exercise}[label=6]
        The sequence: 
        \[1, \dfrac{1}{3}, \dfrac{1}{9}, \dots, \dfrac{1}{3^n},\dots\]
        can be generated with the following recursive relations: 
        \[
        \begin{cases}
            p_n=\dfrac{10}{3}p_{n-1}=p_{n-2}  \\
            p_1 = \dfrac{1}{3}, \: p_0=1
        \end{cases}
        \]
        \[
        \begin{cases}
            q_n=\dfrac{1}{3}q_{n-1}  \\
            q_0=1
        \end{cases}
        \]
        \begin{enumerate}
            \item Implement the two relations in order to generate the first $100$ terms of the sequence.
            \item Study the stability of the two algorithms and justify the obtained results. 
        \end{enumerate}
    \end{Exercise}
    \begin{Answer}[ref=6]
        \begin{lstlisting}[language=Matlab]
p(1) = 1;
p(2) = 1/3;
for i = 2:100
    p(i+1) = 10/3*p(i) - p(i-1);
end
figure
subplot(2,1,1)
plot(0:100, p, 'LineWidth',3)
% gca return the current axes 
% setting the fontsize on axes
set(gca,'FontSize',16)
xlabel('n','FontSize',16)
ylabel('p_n','FontSize',16)
% The sequence explodes!

q(1) = 1;
for i=1:100
    q(i+1) = 1/3*q(i);
end
subplot(2,1,2)
plot(0:100, q, 'LineWidth',3)
set(gca,'FontSize',16)
xlabel('n','FontSize',16)
ylabel('q_n','FontSize',16)
% The sequence is ok.
        \end{lstlisting}
    \end{Answer}

\newpage

\chapter{Laboratory session II}
    \begin{Exercise}[label=7]
        Consider the following function
        \[f(x) = x^3-(2+e)x^2+(2e+1)x+(1-e)-\cosh(x-1) \:\: x \in [0.5, 5.5]\]
        \begin{itemize}
            \item Plot the function f and determine two intervals that contain its roots.
            \item Implement the bisection method:
                \begin{lstlisting}[language=Matlab]
function [x,x_iter]=bisection(f,a,b,tol)
                \end{lstlisting}
                where $x$ is the solution, $x\_iter$ is the vector of the approximations at each iteration, $f$ is the function, defined as handle function, $a,b$ are the end points 
                of the interval, $tol$ is the required tolerance.
            \item For which roots the bisection method can be used? Compute the number of needed iterations for the bisection method to converge with a tolerance of $10^{-3}$, 
                when the interval $[3, 5]$ is chosen as starting interval.
        \end{itemize}
    \end{Exercise}
    \begin{Answer}[ref=7]
        \begin{lstlisting}[language=Matlab]
1.  f= @(x) x.^3-(2 + exp(1))*x.^2 + (2*exp(1) + 1)*x + (1 - exp(1)) - cosh(x - 1);
    a=0.5;
    b=5.5;
    x_plot=linspace(a,b,1000);
    plot(x_plot,f(x_plot));
    grid on

2.  function [x,x_iter]=bisection(f,a,b,tol)
        Nmax = ceil(log((b-a)/tol)/log(2));
        for i=1:Nmax
            x_iter(i)=(b+a)/2;
            if f(x_iter(i))*f(a)<0
                b=x_iter(i);
            else
                a=x_iter(i);
            end
        end
        x=x_iter(end);
    end
3.  a=3;
    b=5;
    tol=1.e-3;
    [x,x_iter]=bisection(f,a,b,tol);
    x
    x_plot=linspace(a,b,1000);
    figure
    plot(x_plot,f(x_plot));
    title("Iterations required to reach the tollerance:", length(x_iter))
    grid on
    hold on
        \end{lstlisting}  
    \end{Answer}

    \newpage

    \begin{Exercise}[label=8]
        Consider the following function in the interval $[-0.5, 1.5]$
        \[f (x) = sin(x)(1 - x)^2\]
        \begin{enumerate}
            \item Plot $f$ in order to find some intervals containing the roots. 
            \item Implement the Newton method by using a stopping criterion based on the error estimator $\left\lvert x^k - x^{k-1}\right\rvert$. The signature of the function is:
                \begin{lstlisting}[language=Matlab]
function [x,x_iter]=newton(f,df,x0,tol,Nmax)
                \end{lstlisting}  
                where $x$ is the approximate, $x\_iter$ is the vector of the approximations at each iteration, $f$, $df$ are the function and its first derivative, defined as
                handle functions, $x0$ is the initial guess, $tol$ is the tolerance demanded by user and $Nmax$ is the maximum number of allowed iterations.
            \item Use Newton method to find the roots with a tolerance equal to $10^{-6}$, by considering as initial guess $x0 = 0.3$ and $x0 = 0.5$.
            \item Compute an estimate of the convergence rate. 
        \end{enumerate}
    \end{Exercise}
    \begin{Answer}[ref=8]
        \begin{lstlisting}[language=Matlab]
1.  f=@(x) sin(x).*(1-x).^2;
    df=@(x) cos(x).*(1-x).^2-2*sin(x).*(1-x);
    x_plot=linspace(-0.5,1.5,1000);
    plot(x_plot,f(x_plot));
    grid on
2.  function [x,x_iter]=newton(f,df,x0,tol,Nmax)
        i=1;
        err=1+tol;
        x_iter(i)=x0;
        while i<=Nmax && err>tol
            if(abs(df(x_iter(i))) < 1e-8) 
                break;
            end
            x_iter(i+1)=x_iter(i)-f(x_iter(i))/df(x_iter(i));
            err=abs(x_iter(i+1)-x_iter(i));
            i=i+1;
        end
        x=x_iter(end);
    end
3.  [x1,x1_iter]=newton(f,df,0.3,1.e-6,100);
    err1=abs(x1_iter-0)
4.  function p = conv_order(e) % e is the error
        p = log(e(3:end)./e(2:end-1))./log(e(2:end-1)./e(1:end-2));
    end
    p = conv_order(err1)
    figure;
    plot(p);
    title('convergence order, root x1=0');
    [x2,x2_iter]=newton(f,df,0.5,1.e-6,100);
    err2=abs(x2_iter-1);
    p = conv_order(err2)
    figure;
    plot(p);
    title('convergence order, root x2=1');
        \end{lstlisting}  
    \end{Answer}

    \newpage

    \begin{Exercise}[label=9]
        \begin{enumerate}
            \item Implement the Newton method for systems. The signature of the function is:
                \begin{lstlisting}[language=Matlab]
function [x,res,niter,difv,x_vect]=newtonsys(Ffun,Jfun,x0,tol, kmax,normtype,varargin)
                \end{lstlisting}  
                where $Ffun$ is the function handle to vector function f(x), $Jfun$ is Jacobian handle matrix, $x0$ is the initial value for iterative process, $nmax$ is the 
                maximum number of allowed iterations, $tol$ is the absolute error tolerance, $normtype$ is the type of norm used in the error estimation, and $varargin$ is an input 
                variable in a function definition statement that enables the function to accept any number of input arguments. 
            \item Use Newton method for the given system and the given parameters. 
        \end{enumerate}
        
    \end{Exercise}
    \begin{Answer}[ref=9]
        \begin{lstlisting}[language=Matlab]
1.  function [x,res,niter,difv,x_vect] = newtonsys(Ffun,Jfun,x0,tol, kmax,normtype,varargin)
    k = 0;
    x_vect = x0;
    err = tol + 1; difv=[ ];
    x = x0;
    if normtype == 2
        nor = 2;
    else
        nor = inf;
    end
    while err >= tol && k < kmax
        J = Jfun(x,varargin{:});
        F = Ffun(x,varargin{:});
        delta = - J\F;
        x = x + delta;
        err = norm(delta,nor); 
        difv=[difv; err];
        x_vect = [x_vect x];
        k = k + 1;
    end
    res = norm(Ffun(x,varargin{:}));
    if (k==kmax && err> tol)
        fprintf(['The method does not converge'],F);
    end
    niter=k;          
2.  F = @(x) [x(1).^2+x(2).^2-1;  sin(pi*x(1)/2)+x(2).^3];
    J = @(x) [2*x(1), 2*x(2);     cos(pi*x(1)/2)*pi/2, 3*x(2).^2];
    nmax = 200;
    tol = 1e-10;
    x0 = [-1;-1];
    [x,res,niter,difv,x_vect] = newtonsys(F,J,x0,tol,nmax,2);
    format long                       
    x
    niter
        \end{lstlisting}  
    \end{Answer}

\newpage

\chapter{Laboratory session III}
    \begin{Exercise}[label=10]
        Given $\xi=\sqrt{5}$, using the fixed point iteration method: 
        \begin{enumerate}
            \item Check if $\phi(x)=5+x-x^2$ converges in $\xi$.
            \item Check if $\phi(x)=5/x$ converges in $\xi$.
            \item Check if $\phi(x)=1+x-(1/5)*x^2$ converges in $\xi$.
            \item Check if $\phi(x)=(1/2)*(x + 5/x)$ converges in $\xi$ and the order of convergence. 
            \item Apply the fixed point iteration method in all previous cases. 
        \end{enumerate}
    \end{Exercise}
    \begin{Answer}[ref=10]
        \begin{lstlisting}[language=Matlab]
1.  xi = sqrt(5);
    phi1  = @(x) 5 + x - x.^2;
    dphi1 = @(x) 1 - 2*x;
    abs(dphi1(xi))
    % The absolute value of the derivative of phi at xi
    % is greater than 1: the method will not converge.
2.  xi = sqrt(5);
    phi2  = @(x) 5./x;
    dphi2 = @(x) -5./x.^2;
    abs(dphi2(xi))
    % The absolute value of the derivative of phi at xi
    % is equal to 1: no theoretical conclusion can be stated in this case.
3.  xi = sqrt(5);
    phi3  = @(x) 1 + x - 1/5*x.^2;
    dphi3 = @(x) 1 - 2/5*x;
    abs(dphi3(xi))
    % The absolute value of the derivative of phi at xi
    % is less than 1: the method will converge provided that the initial guess x{(0)} is close enough to xi (local convergence).
4.  xi = sqrt(5);
    phi4  = @(x) 1/2*(x + 5./x);
    dphi4 = @(x) 1/2 - 5./(2*x.^2);
    abs(dphi4(xi))
    % The absolute value of the derivative of phi at xi
    % is zero (i.e. less than 1): the method will converge provided that the initial guess x{(0)} is close enough to xi (local convergence).
    d2phi4 = @(x) 5./x.^3;
    abs(d2phi4(xi))
    % The second derivative is different from zero,
    % so method 4 is expected to be of second order.
5.  function [xi, x_iter] = fixed_point(phi, x0, tol, maxit)
        x_iter(1) = x0;
        for (iter = 1:maxit)
        x_iter(iter+1) = phi(x_iter(iter));
        if (abs (x_iter(iter+1) - x_iter(iter)) < tol)
            break;
        end
        end
        xi = x_iter(end);
    end

    tol = 1e-6;
    maxit = 1000;
    x0 = xi + 0.001;
    [xi1, x1] = fixed_point(phi1, x0, tol, maxit);
    xi1
    iter1 = numel(x1) - 1
    [xi1, x1] = fixed_point_FV(phi1, x0, tol, maxit);
    xi1
    iter1 = numel(x1) - 1
    % The approximation xi is incorrect and the number of performed iterations is the maximum: as expected, the method did not converge.

    x0 = 3;
    [xi2, x2] = fixed_point(phi2, x0, tol, maxit);
    xi2
    % number of iterations of method 2
    iter2 = numel(x2) - 1
    % different implementation of fixed point method
    [xi2, x2] = fixed_point_FV(phi2, x0, tol, maxit);
    xi2
    iter2 = numel(x2) - 1
    % The approximation xi is incorrect and the number of performed iterations is the maximum: the method did not converge.

    x0 = 4;
    [xi3, x3] = fixed_point(phi3, x0, tol, maxit);
    xi3
    iter3 = numel(x3) - 1
    [xi3, x3] = fixed_point_FV(phi3, x0, tol, maxit);
    xi3
    iter3 = numel(x3) - 1
    % The method converged locally to xi.

    x0 = 10;
    [xi3, x3] = fixed_point(phi3, x0, tol, maxit);
    xi3
    iter3 = numel(x3) - 1
    [xi3, x3] = fixed_point_FV(phi3, x0, tol, maxit);
    xi3
    iter3 = numel(x3) - 1
    % With a different initial guess, the method may not converge to xi.
    
    x0 = 4;
    [xi4, x4] = fixed_point(phi4, x0, tol, maxit);
    xi4
    iter4 = numel(x4) - 1
    [xi4, x4] = fixed_point_FV(phi4, x0, tol, maxit);
    xi4
    iter4 = numel(x4) - 1
    % The method converged to xi.
        \end{lstlisting}  
    \end{Answer}

    \newpage 

    \begin{Exercise}[label=11]
        Consider the following function in the interval $[-1, 6]$
        \[f (x) = \arctan\left[7\left(x-\dfrac{\pi}{2}\right)\right]+\sin\left[\left(x-\dfrac{\pi}{2}\right)^{3}\right]\]
        \begin{enumerate}
            \item Plot $f$ in order to find an interval containing a root. What is the multiplicity of the root?
            \item Use the Newton method to find the root with a tolerance of $10^{-10}$ and initial guess $x^{(0)}=1.5$. Compute the error.
            \item Use the Newton method to find the root with a tolerance of $10^{-10}$ and initial guess $x^{(0)}=4$. Compute the error.
            \item If possible, apply the bisection method on the interval $[a, b] = [-1, 6]$ and tolerance $\dfrac{b-a}{20^{30}}$. Compute the error.
            \item Write a function $bisection\_newton.m$ to find $\xi$ using the Newton method starting from an initial guess obtained after few iterations of a bisection method. 
                Test with $[a, b] = [-1, 6]$, $5$ iterations of the bisection method and tolerance $10^{-10}$ for the Newton method.
        \end{enumerate}
    \end{Exercise}
    \begin{Answer}[ref=11]
        \begin{lstlisting}[language=Matlab]
1.  function rootfinding_function_plot(f, a, b, new_figure)
        if ((nargin < 4) || new_figure)
        figure
        end
        hold on, box on
        x_plot = linspace(a, b, 1000);
        plot(x_plot, f(x_plot), 'LineWidth', 2)
        plot(x_plot, 0*x_plot, 'k-', 'LineWidth', 1)
        xlabel('x','FontSize', 16)
        ylabel('f(x)','FontSize', 16)
        set(gca,'FontSize', 16)
        set(gca,'LineWidth', 1.5)
    end

    a = -1;
    b = 6;
    f = @(x) atan(7*( x - pi/2)) + sin((x-pi/2).^3);
    rootfinding_function_plot(f, a, b, true);
    xi_ex = pi/2;
    df = @(x) 7 ./ ( 1 + 49 * ( x-pi/2 ).^2 ) + 3 * (x-pi/2).^2 .* cos( (x-pi/2).^3 );
    df(xi_ex); 
2.  x0 = 1.5;
    tol = 1e-10;
    maxit = 1000;
    [xi1, x_iter1] = newton(f, df, x0, tol, maxit);
    xi1
    iter1 = numel(x_iter1) - 1
    err1 = abs( xi1 - xi_ex)
    % Newton method converges to xi
3.  x0 = 4;
    tol = 1e-10;
    maxit = 1000;
    [xi2, x_iter2] = newton(f, df, x0, tol, maxit);
    xi2
    iter2 = numel(x_iter2) - 1
    err2 = abs( xi2 - xi_ex)
    % Newton method does not converge to xi
4.  tol = (b-a)/(2^30);
    [xi3, x_iter3] = bisection(f, a, b, tol);
    xi3
    iter3=numel(x_iter3)  
    tol_bisection = (b-a)/(2^5);
    tol_newton = 1e-10;
    maxit_newton = 1000;
    [xi4, x_iter4_bisection, x_iter4_newton] = bisection_newton(f, df, a, b, tol_bisection, tol_newton, maxit_newton);
    xi4
    iter4_bisection=numel(x_iter4_bisection)
    iter4_newton=numel(x_iter4_newton)
    err4 = abs( xi4 - xi_ex)
5.  function [xi, x_iter_bisection, x_iter_newton] = bisection_newton(f, df, a, b, tol_bisection, tol_newton, maxit_newton, multiplicity)
        if (nargin < 8)
        multiplicity = 1;
        end
        [xi_bisection, x_iter_bisection] = bisection(f, a, b, tol_bisection);
        [xi_newton, x_iter_newton] = newton(f, df, xi_bisection, tol_newton, maxit_newton, multiplicity);
        xi = xi_newton;
    end

    function [xi, x_iter] = bisection(f, a, b, tol)
        max_iterb = ceil( log((b - a)/tol)/log(2) );

        for (iter = 1:max_iterb)
        x_iter(iter) = a + (b-a)/2;
        f_iter(iter) = f(x_iter(iter));

        if ( f(b)*f_iter(iter) < 0 )
            a = x_iter(iter);
        elseif ( f(a)*f_iter(iter) < 0 )
            b = x_iter(iter);
        else % f(x) = 0
            break;
        end
        end

        xi = x_iter(end);
    end

    function [xi, x_iter] = newton(f, df, x0, tol, maxit, multiplicity)
        if (nargin < 6)
        multiplicity = 1;
        end
        x_iter(1) = x0;
        for (iter = 1:maxit)
        newton_method = @(x) x - multiplicity*f(x)/df(x);
        x_iter(iter+1) = newton_method(x_iter(iter));

        if (abs (x_iter(iter+1) - x_iter(iter)) < tol)
            break;
        end
        end
        xi = x_iter(end);
    end
        \end{lstlisting}  
    \end{Answer}

\newpage

\chapter{Laboratory session IV}



\end{document}
\documentclass[12pt, a4paper]{report}
\usepackage{graphicx, array, amsthm, amssymb, amsmath, algorithm, algpseudocode, float, xcolor, thmtools, thmbox}
\usepackage[english]{babel}

\makeatletter
\renewcommand\thmbox@headstyle[2]{\bfseries #1}
\makeatother
\newtheorem[style=M,bodystyle=\normalfont]{theorem}{Theorem}
\newtheorem[style=M,bodystyle=\normalfont]{corollary}{Corollary}
\newtheorem[style=M,bodystyle=\normalfont]{lemma}{Lemma}
\newtheorem[style=M,bodystyle=\normalfont]{definition}{Definition}


\title{Numerical Analysis \\ \textit{Theory}}
\author{Christian Rossi}
\date{Academic Year 2023-2024}

\begin{document}

\maketitle

\newpage

\begin{abstract}
The topics of the course are:
\begin{itemize}
    \item Floating-point arithmetic: different sources of the computational error; absolute vs relative errors; the floating point representation 
        of real numbers; the round-off unit; the machine epsilon; floating-point operations; over- and under-flow; numerical cancellation.
    \item Numerical approximation of nonlinear equations: the bisection and the Newton methods; the fixed-point iteration; convergence analysis 
        (global and local results); order of convergence; stopping criteria and corresponding reliability; generalization to the system of 
        nonlinear equations (hints).
    \item Numerical approximation of systems of linear equations: direct methods (Gaussian elimination method; LU and Cholesky factorizations; 
        pivoting; sparse systems: Thomas algorithm for tridiagonal systems); iterative methods (the stationary and the dynamic Richardson scheme; 
        Jacobi, Gauss-Seidel, gradient, conjugate gradient methods (hints); choice of the preconditioner; stopping criteria and corresponding 
        reliability); accuracy and stability of the approximation; the condition number of a matrix; over- and under-determined systems: the 
        singular value decomposition (hints).
    \item Numerical approximation of functions and data: Polynomial interpolation (Lagrange form); piecewise interpolation; cubic interpolating 
        splines; least-squares approximation of clouds of data.
    \item Numerical approximation of derivatives: finite difference schemes of the first and second order; the undetermined coefficient method.
    \item Numerical approximation of definite integrals: simple and composite formulas; midpoint, trapezoidal, Cavalieri-Simpson quadrature rules; 
        Gaussian formulas; degree of exactness and order of accuracy of a quadrature rule. 
    \item Numerical approximation of ODEs: the Cauchy problem; one-step methods (forward and backward Euler and Crank-Nicolson schemes); 
        consistency, stability, and convergence (hints).
\end{itemize}
\end{abstract}

\newpage

\tableofcontents

\newpage

\chapter{Introduction}




\end{document}
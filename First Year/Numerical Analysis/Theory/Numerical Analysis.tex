\documentclass[12pt, a4paper]{report}
\usepackage{graphicx}
\usepackage{array}
\usepackage[english]{babel}
\usepackage{amsthm}
\usepackage{amssymb}
\usepackage{amsmath}
\usepackage{algorithm}
\usepackage{algpseudocode}
\usepackage{float}
\usepackage{xcolor}
\usepackage{thmtools}

\newtheorem{theorem}{Theorem}[section]
\newtheorem{corollary}{Corollary}[theorem]
\newtheorem{lemma}[theorem]{Lemma}
\theoremstyle{remark}
\newtheorem*{remark}{Definition}
\declaretheoremstyle[
spaceabove=\topsep, spacebelow=\topsep,
headfont=\bfseries\color{black},
notefont=\mdseries\color{black}, notebraces={(}{)},
bodyfont=\normalfont,
postheadspace=\newline,
headpunct=,
numbered=no,
postheadhook=\leavevmode%
  \interlinepenalty 10000%
  \vskip-1.3\baselineskip%
  \noindent{\color{black}\rule{\textwidth}{1pt}}%
  \interlinepenalty 10000%
  \vskip0.3\baselineskip\noindent,
qed=\textcolor{gray}{$\blacksquare$}
]{mystyle}
\declaretheorem[style=mystyle]{example}
\renewcommand\theexample{\arabic{example}}

\title{Numerical Analysis \\ \textit{Theory}}
\author{Christian Rossi}
\date{Academic Year 2023-2024}

\begin{document}

\maketitle

\newpage

\begin{abstract}



\end{abstract}

\newpage

\tableofcontents

\newpage

\chapter{Introduction}
\section{Numerical analysis and errors}
Numerical analysis is the field of mathematics dealing with methods to find the solutions of certain mathematical problems with an electronic 
calculator. It is the intersection between math and computer science. 

On the other hand, scientific computing also deals with the model formalization and so it needs also engineering knowledge. 
\begin{figure}[H]
    \centering
    \includegraphics[width=0.75\linewidth]{images/difference.png}
    \caption{Difference between numerical analysis and scientific computing}
\end{figure}
As it is possible to see from the diagram every step of the computation have to deal with errors. The possible types of errors are: 
\begin{itemize}
    \item Absolute: $\left\lvert x - \tilde{x} \right\rvert$
    \item Relative: $\dfrac{\left\lvert x - \tilde{x} \right\rvert}{\left\lvert x \right\rvert}$, where $x \neq 0$
\end{itemize}
The relative error is more precise because it compares the error with the measure quantity. 
\begin{example}
    Let us consider $x=100$ and $\tilde{x}=100.1$. The errors in this case are: 
    \[e_{abs}=\left\lvert x - \tilde{x} \right\rvert=\left\lvert 100 - 100.1 \right\rvert=0.1\]
    \[e_{rel}=\dfrac{\left\lvert x - \tilde{x} \right\rvert}{\left\lvert x \right\rvert}=\dfrac{\left\lvert 100 - 100.1 \right\rvert}{\left\lvert 100 \right\rvert}=0.001\]
    Let us consider $x=0.2$ and $\tilde{x}=0.1$. The errors in this case are: 
    \[e_{abs}=\left\lvert x - \tilde{x} \right\rvert=\left\lvert 0.2 - 0.1 \right\rvert=0.1\]
    \[e_{rel}=\dfrac{\left\lvert x - \tilde{x} \right\rvert}{\left\lvert x \right\rvert}=\dfrac{\left\lvert 0.2 - 0.1 \right\rvert}{\left\lvert 0.2 \right\rvert}=0.5\]
    The result are that the measures have the same absolute error ($10\%$), but the relative error is much grater in the second example ($50\%$ vs $0.1\%$).
    This result proves that the relative error is the most precise.
\end{example}

\section{Floating point}
A calculator can only handle a finite quantity of numbers and compute a finite number of operations. For those reason the set of the real numbers 
$\mathbb{R}$ is indeed represented by a finite set of machine numbers $\mathbb{F}=\{-\tilde{a}_{min}, \dots , \tilde{a}_{max} \}$ called
floating points numbers. The function used to find the corresponding value in $\mathbb{F}$ to a number in $\mathbb{R}$ is $fl(x)$ that does an 
operation called truncation and rounding.

The set $\mathbb{F}=\mathbb{F}(\beta,t,L,U)$ is characterized by four parameters $\beta,t,L$ and $U$ such that every real number $fl(x) \in \mathbb{F}$ 
can be written as:
\[fl(x)=(-1)^sm\beta^{e-t}=(-1)^s(a_1a_2\dots a_t)_{\beta}\beta^{e-t}\]
where:
\begin{itemize}
    \item $\beta \geq 2$ is the basics, an integer that determines the numeric system. 
    \item $m=(a_1a_2\dots a_t)_{\beta}$ is the mantissa, $(0<m<\beta^t-1)$ where $t$ is the number of digits such that $0<a_1 \leq \beta - 1$
        and $0 \leq a_i \leq \beta - 1$ for $i=2, \dots, t$. 
    \item $e=\mathbb{Z}$ is the exponent such that $L<e<U$, with $L<0$ and $U>0$. 
    \item $s=\{0,1\}$ is the sign. 
\end{itemize}
In the definition of the numbers in the mantissa set we have to set the constraint $a_1 \neq 0$ to ensure the uniqueness of the representation. 

The set of floating points has the following characteristic values:
\begin{itemize}
    \item Machine epsilon, that is the distance between one and the smallest floating point number greater than one, and it is equal to: 
        \[\epsilon_M=\beta^{1-t}\]
    \item Round-off error, that is the relative error that is committed when substituting $x \in \mathbb{R}-\{0\}$ with his corresponding 
        $fl(x) \in \mathbb{F}$. It is limited by: 
        \[\dfrac{\left\lvert x-fl(x) \right\rvert}{\left\lvert x \right\rvert }\leq \dfrac{1}{2}\epsilon_M\]
        where $x \neq 0$.
    \item Cardinality of the floating point set:
        \[\#\mathbb{F}=2 \beta^{t-1}(\beta -1)(U-L+1)+1\]
        where: 
        \begin{itemize}
            \item $2$ is needed to consider both positive and negative numbers. 
            \item $\beta^{t-1}$ is the cardinality of values that can be taken by all digits.
            \item $(\beta -1)$ is the cardinality of values that can be taken by $a_1$.
            \item $(U-L+1)$ considers all the possible variations for the exponent.
            \item $1$ is needed to consider also the zero. 
        \end{itemize}
    \item The biggest and the smallest numbers in the set are found with the formula:
        \[x_{min}=\beta^{L-1}\]
        \[x_{max}=\beta^U(1-\beta^{-t})\]
\end{itemize}
\begin{example}
    In MATHLAB the floating point set is defined with the following variables:
    \[(\beta=2,t=53,L=-1021,U=1024)\] 
    With the command $eps$ we can find the machine epsilon, that in MATLAB case is:
    \[\epsilon_M=2.22 \cdot 10^{-16}\]
    With the command $realmin$ and $realmax$ we can find the smallest and the largest numbers representable that are equal to:
    \[x_{min}=2.225073858507201 \cdot 10^{-308}\]
    \[x_{max}=1.797693134862316 \cdot 10^{308}\]
\end{example}

Since not all the numbers in the $\mathbb{R}$ set are also in the $\mathbb{F}$ set, in the second one there is no continuity. It is possible to 
demonstrate that while we are increasing the values of the numbers we are also increasing the distance between two consecutive numbers in $\mathbb{F}$.
\begin{example}
    Let us consider the floating number set $\mathbb{F}(2,2,-1,2)$. The characteristic values of this set are: 
    \begin{itemize}
        \item $\epsilon_M=\beta^{1-t}=0.5$.
        \item $x_{min}=\beta^{L-1}=0.25$.
        \item $x_{max}=\beta^U(1-\beta^t)=3$.
        \item $\#\mathbb{F}=2 \beta^{t-1}(\beta -1)(U-L+1)+1=16$. 
    \end{itemize}
    The exponent can have the values $-1,0,1$ and $2$. The mantissa will be like $(a_1a_2)_{\beta}$ because $t=2$. The possible positive values are
    reported in the figure. 
    \begin{figure}[H]
        \centering
        \includegraphics[width=0.9\linewidth]{images/numbers.png}
    \end{figure}

\end{example}

The other important aspect is that the passage between the two sets causes the loss of two important properties such as associativity end the 
neutral number for the sum. 

\newpage

\chapter{Nonlinear equations}









\end{document}
\section{Introduction}

Nonlinear equations systems can be represented as:
\[Ax=B\]
Here, $A$ is a non-singular square matrix of dimension $n \times n$, with elements $a_{ij}$ that can be real or complex, while $x$ and $B$ are column vectors of size $n$.
The vector $x$ represents the unknown solution, while $B$ is a predefined vector. 
In component-wise form, it can be expressed as:
\[
\begin{cases}
    a_{11}x_1+a_{12}x_2+\dots+a_{1n}x_n=b_1 \\
    a_{21}x_1+a_{22}x_2+\dots+a_{2n}x_n=b_2 \\
    \vdots                                  \\
    a_{n1}x_1+a_{n2}x_2+\dots+a_{nn}x_n=b_n
\end{cases}
\]
\begin{definition}
    The matrix $A$ is \emph{non-singular} if and only if: 
    \[Av=0\leftrightarrow v=0\]
\end{definition}
In other words, the kernel of $A$ only contains the vector $0$ or equivalently: 
\[\dim{\left(\ker{\left(A\right)}\right)}=0\]
Additionally, it is essential that $\textnormal{rank}(A)=n$ or $\det(A)\neq 0$. 

The solution to this system can be determined using Cramer's rule:
\[x_i=\dfrac{\det{A_i}}{\det{A}} \:\:\:\:\:\: i=1,\dots,n\]
Here, $A_i$ is the matrix derived from $A$ by replacing the $i$-th column with $B$. 
However, the time complexity of Cramer's rule operations is on the order of $3(n+1)!$, making it impractical for most cases.
There are two alternative approaches to tackle this: 
\begin{itemize}
    \item Direct methods, which provide the system's solution in a finite number of steps
    \item Iterative methods, which, in principle, demand an infinite number of steps.
\end{itemize}
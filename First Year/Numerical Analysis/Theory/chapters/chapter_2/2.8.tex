\section{Aitken method}

We have the following relation:
\[x^{(k+1)}-\alpha=\phi(x^{(k+1)})-\phi(\alpha)=\lambda_k\left( x^{(k+1)}-\alpha \right)\]
We know that $\lambda_k$ is the first derivative of $\phi$ at a certain point $\xi$. 
If we know $\lambda_k$, we can express $\alpha$ as: 
\[\alpha=\dfrac{\phi(x^{(k)})-\lambda_k x^{k}}{1-\lambda_k}\]
Additionally, the quantity:
\[A_k=\dfrac{\phi(\phi(x^{(k)}))-\phi(x^{(k)})}{\phi(x^{(k)})-x^{(k)}}\]
provides a good approximation of $\lambda_k$, so we can replace $\lambda_k$ with $A_k$. 
By substituting, we obtain the formula used in the Aitken method.
\begin{theorem}
    Let $\phi(x)=x-f(x)$ and $f(\alpha)$ be a simple zero. 
    Let $\phi^{(k)}$ converge to the first order to $\alpha$, then $\phi_A$ converge to the second order. 
    If $\phi(x)$ converges with order $p$ and $f(x)$ is still a simple zero, then $\phi_A$ converges with order $2p-1$. 
\end{theorem}
Occasionally, $\phi_A$ converges even when $\phi$ does not.

\subsection*{Algorithm}
\begin{algorithm}[H]
    \caption{Algorithm for the Aitken method}
        \begin{algorithmic}[1]
            \For {$k=0,1,\dots,n$}
                \State $x^{(k+1)}=x^{(k)}-\dfrac{\left[ \phi(x^{(k)})-x^{(k)} \right]^2}{\phi(\phi(x^{(k)}))-2\phi(x^{(k)})+x^{(k)}}$
                \If {stopping criterion is satisfied}
                    \State \Return $x^{(k+1)}$
                \EndIf
            \EndFor
        \end{algorithmic}
\end{algorithm}
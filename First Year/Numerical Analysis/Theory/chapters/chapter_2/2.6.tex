\section{Secant method}

In situations where the derivative of a function $f$ is not readily available in analytical form, the Newton method cannot be employed for finding its zeros. 
However, we still have the capability to compute the function $f$ at arbitrary points, and in such cases, we can replace the exact value of $f^{'}(x^{(k)})$ with an incremental ratio based on previously computed values of $f$. 
The secant method capitalizes on this strategy, and it converges super-linearly with a rate of $q=1.6$. 

\subsection*{Algorithm}
The algorithm takes two initial guesses, $x^{(0)} \in \mathbb{R}$ and $x^{(1)} \in \mathbb{R}$, as inputs. 
The output of the algorithm is an approximate value for the zero of the function.
\begin{algorithm}[H]
    \caption{Algorithm for the secant method}
        \begin{algorithmic}[1]
            \For {$k=0,1,\dots,n$}
                \State $x^{(k+1)}=x^{(k)}-f(x^{(k)})\dfrac{x^{(k)}-x^{(k-1)}}{f(x^{(k)})-f(x^{(k-1)})}$
                \If {$k>k_{max} \lor \left\lvert x^{(k)}-x^{(k-1)} \right\rvert \leq \epsilon_s \lor \left\lvert f\left(x^{(k+1)}\right) \right\rvert \leq \epsilon_r$}
                    \State \Return $x^{(k+1)}$
                \EndIf
            \EndFor
        \end{algorithmic}
\end{algorithm}
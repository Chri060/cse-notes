\section{Bisection method}

\newpage
\begin{theorem}
    Let $f(x)$ be a continuous function on the interval $I=(a,b)$, that is $f \in C^0([a,b])$. 
    If $f(a)f(b)<0$, then there exists at least one zero $\alpha \in I$ of $f(x)$. 
\end{theorem}
Suppose there exists a unique zero, denoted as $\alpha$. 
The bisection method employs a strategy involving the given interval and selects sub-intervals within which the function $f$ exhibits a change in sign.
By following this procedure, it is assured that each interval chosen in this manner will encompass $\alpha$. 

The sequence $\{x^{(k)}\}$ consisting of the midpoints of these sub-intervals will inevitably converge to $\alpha$. 
This convergence occurs because the lengths of these sub-intervals decrease to zero as $k$ approaches infinity.
We can establish the following relationship:
\[\left\lvert x^{(k)} - \alpha \right\rvert \leq \dfrac{1}{2} \left\lvert b^{(k)}-a^{(k)} \right\rvert \]
Given the similarity between the left-hand side of this equation and the condition for the error, we can determine the stopping criterion as follows:
\[\left\lvert b^{(k)}-a^{(k)} \right\rvert \leq 2 \epsilon_e\]
Now, with the tolerance $\epsilon$ at our disposal, we can calculate the minimum number of iterations required:
\[k_{min}= \left\lceil {\log_2{\left( \dfrac{\left\lvert b-a \right\rvert}{\epsilon} \right)}-1}\right\rceil\]

\subsection*{Algorithm}
The algorithm takes as input a continuous function $f$ belonging to $C(\mathbb{R})$ and an interval $[a,b]$ with the property that $f(a)f(b) \leq 0$. 
The output of the algorithm is an approximate value for the zero of the function.
\begin{algorithm}[H]
    \caption{Algorithm for the bisection method}
        \begin{algorithmic}[1]
            \For {$k=0,1,\dots,n$}
                \State $x^{(k)}=\dfrac{a+b}{2}$
                \If {$\left\lvert b^{(k)}-a^{(k)} \right\rvert \leq 2 \epsilon_e$}
                    \State \Return $x^{(k)}$
                \ElsIf {$f(x^{(k)})f(a) < 0$}
                    \State $b \leftarrow x^{(k)}$
                \Else 
                    \State $a \leftarrow x^{(k)}$
                \EndIf
            \EndFor
        \end{algorithmic}
\end{algorithm}

\subsection*{Summary}
The advantages of this method are as follows:
\begin{itemize}
    \item I can control the maximum allowable error.
    \item Convergence is assured.
    \item It relies solely on the evaluation of $f$
\end{itemize}
However, there are some drawbacks:
\begin{itemize}
    \item It is effective only when $f$ changes sign within the interval.
    \item Convergence tends to be slow.
\end{itemize}
\section{Stopping conditions}

Iterative methods necessitate a stopping criterion to determine when to halt the iteration process. 
This criterion can be based on one of four possible conditions:
\begin{itemize}
    \item Error criterion: terminate if the absolute error satisfies $\left\lvert x^{(k)}-\alpha \right\rvert \leq \epsilon_e$.
    \item Residual criterion: Halt the process if the absolute value of the function's residual meets the condition $\left\lvert f\left(x^{(k)}\right) \right\rvert \leq \epsilon_r$. 
    \item Step length criterion: stop the iteration when the absolute difference between consecutive steps adheres to $\left\lvert x^{(k)}-x^{(k-1)} \right\rvert \leq \epsilon_s$. 
    \item Maximum iterations criterion: terminate the iterations if the number of iterations reaches or exceeds a specified maximum value $k_{max}$. 
\end{itemize}
If none of the first three stopping criteria are satisfied, it indicates a lack of convergence in the iterative process.
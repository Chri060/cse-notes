\section{Iterative methods}

The set comprising all polynomials of degree $n$ is represented by the symbol $\mathbb{P}_n$ encompassing all polynomials with degrees less than or equal to $n$.
\begin{theorem}
    There is no solution in radicals to general polynomial equations of degree five or higher with arbitrary coefficients. 
\end{theorem}
Therefore, for polynomials with degrees exceeding four, iterative methods are required for solutions. 
The fundamental concept of these methods can be outlined as follows:
\begin{enumerate}
    \item Begin with an initial guess, denoted as $x^{(0)}$, which serves as a speculative value for $\alpha$.
    \item Utilize this selected value as an input for a black-box function.
    \item Take the output of the black-box function as the new $x^{(0)}$ and return to step one. 
\end{enumerate}
After several iterations, a sequence of values $\{ x^{(n)} \}$ converges such that:
\[ \lim_{n \rightarrow \infty} = \alpha\]
Additionally, the error associated with the approximated value for $\alpha$ approaches zero:
\[ \lim_{n \rightarrow \infty}e^n = 0\]
This implies that the error can also be expressed as: 
\[e^n=\alpha-x^{(n)}\]
All the methods we are going to discuss generate a sequence $x^{(1)},x^{(2)},\dots,x^{(n)}$ of numbers that ideally converges to $\alpha$:
\[ \lim_{k \rightarrow + \infty} \left\lvert x^{(k)}-\alpha \right\rvert =0\]
\begin{definition}[order of convergence]
    An iterative method for approximating the zero $\alpha$ of the function $f(x)$ is considered to have a convergence order $q$ if and only if for $k > k_0$:
    \[\left\lvert x^{(k)} - \alpha \right\rvert \leq c {\left\lvert x^{(k+1)} - \alpha \right\rvert}^q  \]
    There are two possible cases:
    \begin{itemize}
        \item If $q=1$, it is termed linear convergence, with the constraint $0<c<1$.
        \item If $q>1$, $c$ can be any positive number greater than zero.
    \end{itemize}
\end{definition}
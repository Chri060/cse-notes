\section{Testing}

Consider the function \texttt{foo}, written in a C-like language:
\begin{lstlisting}[style=C]
0: void foo(int a, int b) {
1:      for (int i=a; i<b; i++) {
2:          if (i % 5 == 0) {
3:              print(i)
4:          }
5:      }
6: }
\end{lstlisting}
\begin{enumerate}
    \item Run a concolic execution starting from the following input $\{a = 1, b = 3\}$. 
    \item Run a concolic execution with concrete values that allow no execution of the loop. 
    \item Run a concolic execution with concrete values that allow the execution of the loop two times. 
        Line three must be executed only in the second iteration. 
\end{enumerate}

\paragraph*{Solution}
\begin{enumerate}
    \item The path executed with the given input is the following: 
        \begin{enumerate}
            \item [0: ] $a=A(1), b=B(3)$
            \item [1: ] $i=A(1), A<B$
            \item [2: ] $A\%5 \neq 0$
            \item [1: ] $i = A+1(2), A+1<B$
            \item [2: ] $A+1\%5 \neq 0$
            \item [1: ] $i=A+2(3), A+2=B$
        \end{enumerate}
        In this case we have that the test executes the loop 2 times without executing the location 3 as requested. 
        The condition for this path are: 
        \[\begin{cases}
            A\%5 \neq 0 \\ A+1\%5 \neq 0 \\ A+2=B
        \end{cases}\]
    \item To avoid the loop we have to negate the condition to enter in it, that is: 
        \[\neg( A < B ) \Rightarrow A \geq B\]
        One test case that satisfies this condition is: 
        \[\{a=3,b=1\}\]
        In this case the final path will be $\left\langle 0,1 \right\rangle$. 
    \item Starting from the execution of the first test case we have only to negate the condition: 
        \[A+1\%5 \neq 0\]
        This results in the following conditions: 
        \[\begin{cases}
            A\%5 \neq 0 \\ A+1\%5 = 0 \\ A+2=B
        \end{cases}\]  
        One possible concrete set is $\{a = 4, b = 6\}$. 
        After running the concrete execution we have that the request is satisfied. 
\end{enumerate}
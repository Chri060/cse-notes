\section{General quality of documentation}

A well-crafted documentation should possess the following qualities:
\begin{itemize}
    \item Completeness:
        \begin{itemize}
            \item Regarding goals: all requirements must satisfy the goals within specified domain assumptions.
            \item Regarding inputs: software behavior should be specified for all possible inputs.
            \item Structural completeness. 
        \end{itemize}
    \item Pertinence: 
        \begin{itemize}
            \item Each requirement or domain assumption should be necessary for achieving a goal.
            \item Each goal should be genuinely needed by the stakeholders.
            \item The documentation should not contain items unrelated to requirement definitions.
        \end{itemize}
    \item Consistency: there should be no contradictions in the formulation of goals, requirements, and assumptions.
    \item Unambiguity: 
        \begin{itemize}
            \item Clear and well-defined vocabulary.
            \item Unambiguous assertions. 
            \item Verifiability of requirements.
            \item Clear delineation of responsibilities between the software and its environment.
        \end{itemize}
    \item Feasibility: the goals and requirements must be achievable within the allocated budget and schedules.
    \item Comprehensibility: the documentation should be easily understandable by the target audience.
    \item Good structuring: every item must be defined before it is used.
    \item Modifiability: the document should be adaptable, and the impact of modifications should be assessable.
    \item Traceability: 
        \begin{itemize}
            \item Indication of the sources of goals, requirements, and assumptions.
            \item Linking requirements and assumptions to underlying goals.
            \item Facilitating referencing of requirements in future documentation.
        \end{itemize}
\end{itemize}   
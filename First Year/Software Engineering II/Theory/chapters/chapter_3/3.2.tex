\section{Syntax}

Alloy presents bounded snapshots of the world that satisfy the given specification. 

It employs bounded exhaustive search to uncover counterexamples to asserted properties using SAT.        
\newpage
\begin{definition}
        \emph{Atoms} represent Alloy's fundamental entities, characterized by indivisibility, immutability, and a lack of interpretation.

        \emph{Relations} establish connections between atoms, forming sets of tuples, where tuples are sequences of atoms.
    \end{definition}
    In Alloy, relations are inherently typed, with their types determined by the declaration of the relation. 
    The fundamental Alloy relation types include:
    \begin{itemize}
        \item "none" (signifying an empty set). 
        \item "univ" (representing the universal set).
        \item "iden" (denoting the identity relation).
    \end{itemize}
    The logical operators in Alloy encompass:
    \begin{itemize}
        \item Union "$\cup$".
        \item Intersection "$\&$".
        \item Difference "$-$".
        \item Subset "$in$".
        \item Equality "$=$".
        \item Cross product "$\rightarrow$", analogous to a natural join.
        \item Dot join "$.$" or "$[\:]$", where the final element of the first relation joins with the corresponding first elements of the second relation, followed by the removal of the combined element from the relation.
    \end{itemize}
    Here are the various binary closures applicable to relations in Alloy:
    \begin{itemize}
        \item Transpose "$\sim$": inverts the order of the elements in the relation.
        \item Transitive "$\land$": it signifies the transitive closure where $^{\land}r=r+r.r+r.r.r+\dots$. 
        \item Reflexive transitive "$*$": it represents the reflexive transitive closure, where $^{*}r=\textnormal{iden}+^{\land}r$. 
    \end{itemize}
    The possible restrictions are: 
    \begin{itemize}
        \item Domain restriction "$<:$", that restricts the elements on the left side to the set on the right side.
        \item Range restriction "$:>$", that is same as before, but the relations are inverted. 
        \item Override "$++$", that removes the tuples on the left that are in the right relations and adds all the remaining relations of the right relation. 
    \end{itemize}
    Alloy also includes various Boolean operators:
    \begin{itemize}
        \item Negation ("$!$" or "not"). 
        \item Conjunction ("$\&$" or "and"). 
        \item Disjunction ("$\mid \mid$" or "or"). 
        \item Implication ("$\implies$" or "implies"). 
        \item Alternative ("$,$" or "else"). 
        \item Bi-Implication ("$\iff$" or "iff").
    \end{itemize}
    Alloy offers logic quantifiers:
    \begin{itemize}
        \item "all": holds for every element.
        \item "some": holds for at least one element.
        \item "no": holds for no elements.
        \item "lone": holds for at most one element.
        \item "one": holds for exactly one element.
    \end{itemize}
    For defining relations with singletons, you can use the following declaration "x: m e", where "x" is the name of the relation, "m" is the multiplicity of the element (e.g., "set", "one", "lone", or "some") and "e" is the name of the element within the relation. 
    When the relation consists of pairs, the declaration appears as "r: Am $\rightarrow$ nB", where "r" is the relation name, "A" and "B" denote the element names with multiplicities "m" and "n", respectively.
        
    Additionally, Alloy includes operators like "$\#$" (counting the number of tuples in $r$), integers ($0,1,\dots$) for defining variable values, arithmetic operators ("$+$" and "$-$"), and various comparison operators ("$<$", "$<=$", "$=$", "$=>$", "$>$"). 
    There is also the "sum" operator, which adds all elements within a selected tuple.
    
    Other useful keywords are: 
    \begin{itemize}
        \item "let:" this keyword is utilized to establish local variables within a formula or expression. 
            These variables serve to store and reuse intermediate results, making it easier to simplify complex expressions.
        \item "enum": this keyword is employed to define enumerations. 
            Enumerations enable the definition of a finite set of symbolic values, which can be utilized in models to represent various states, options, or categories.
        \item "var": this keyword is used for declaring variables within an Alloy model or specification.
        \item "after": this keyword is employed to specify the temporal ordering or sequence of events or states within a model. 
            It is commonly used in temporal logic to define the sequence of events that must occur before or after a specific state or action.
        \item "always": this keyword is used to express a property that must remain true throughout a system's execution or under certain conditions.
            It is often utilized to specify invariant properties in Alloy models.
        \item "eventually": this keyword is used to express temporal logic properties that indicate a particular condition or event will ultimately occur during the execution of a system. 
            It is commonly used to model and verify properties that describe what should happen at some point in the future.
        \item "historically": keyword is used to express temporal logic properties that describe a condition or event that has remained true for a continuous duration of time leading up to the present or to a specified point in the model's execution.
            It is often employed to specify properties related to the historical behavior of a system.
        \item "before". 
        \item "once".
    \end{itemize}
\section{Introduction}

Alloy is a formal notation designed for specifying models of both systems and software. 
It exhibits characteristics akin to a declarative object-oriented language, underpinned by a robust mathematical foundation.   
To aid in the practical application of Alloy, a supporting tool is available, facilitating the simulation of specifications and enabling property verification.

Alloy has been crafted with the objective of combining the expressive power found in the Z language with the rigorous automated analysis capabilities offered by the SMW model checker.
In the realm of requirement engineering, Alloy finds utility in formally describing the domain, its inherent properties, or the operations that a machine is required to provide.
In software design, Alloy serves as a valuable tool for the formal modeling of components and their interdependencies.    

Alloy represents a fusion of first-order logic and relational calculus. 
Key features of this formal language include:
\begin{itemize}
    \item A thoughtfully curated subset of relational algebra, offering a uniform model for individuals, sets, and relations, with an absence of high-order relations.
    \item Minimal reliance on arithmetic operations.
    \item Support for modules and hierarchies to enhance organization and scalability.
    \item Designed for succinct, illustrative specifications.
    \item Employs a potent and efficient analysis tool.
\end{itemize}
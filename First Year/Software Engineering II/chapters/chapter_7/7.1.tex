\section{General quality of documentation}

A well-crafted documentation should possess the following qualities:
\begin{itemize}
    \item \textit{Completeness}: for the goals we need that all requirements must satisfy the goals within specified domain assumptions.
        For the input we want that the software behavior should be specified for all possible inputs.
        We also need to check for structural completenss. 
    \item \textit{Pertinence}: each requirement or domain assumption should be necessary for achieving a goal.
        Each goal should be genuinely needed by the stakeholders.
        The documentation should not contain items unrelated to requirement definitions.
    \item \textit{Consistency}: there should be no contradictions in the formulation of goals, requirements, and assumptions.
    \item \textit{Unambiguity}: clear and well-defined vocabular, unambiguous assertions, and verifiability of requirements.
        It must also define a clear delineation of responsibilities between the software and its environment.
    \item \textit{Feasibility}: the goals and requirements must be achievable within the allocated budget and schedules.
    \item \textit{Comprehensibility}: the documentation should be easily understandable by the target audience.
    \item \textit{Good structuring}: every item must be defined before it is used.
    \item \textit{Modifiability}: the document should be adaptable, and the impact of modifications should be assessable.
    \item \textit{Traceability}: indication of the sources of goals, requirements, and assumptions, and the link between requirements and assumptions to underlying goals.
        This property facilitate referencing of requirements in future documentation.
\end{itemize}   
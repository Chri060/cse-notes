\section{Risk management plan}

The primary goal of the risk management plan is to establish a comprehensive framework for the ongoing identification, analysis, treatment, and continuous monitoring of risks.
\begin{definition}[\textit{Risk}]
    Risk is characterized as a potential problem or threat.
\end{definition}
Various techniques are employed to identify risks, including:
\begin{itemize}
    \item Drawing from prior experiences.
    \item Engaging in brainstorming sessions.
    \item Extracting lessons learned from analogous projects.
    \item Utilizing checklists for well-known risks.
\end{itemize}
Risks are formulated according to a rule of thumb, requiring each risk to be articulated in the following structure:
\[\textnormal{if }\left\langle \textnormal{situation} \right\rangle\textnormal{ then }\left\langle \textnormal{consequence} \right\rangle\textnormal{, for }\left\langle \textnormal{stakeholder} \right\rangle \]

\paragraph*{Measurement}
Risk assessment is based on evaluating the likelihood (probability) and impact (dependent on the event's nature). 
To diminish the probability and impact of a specific event, a mitigation strategy can be introduced, aiming to lower the levels of these two components (or at least one of them).
\documentclass[12pt, a4paper]{report}
\usepackage{graphicx, array, amsthm, amssymb, amsmath, algorithm, algpseudocode, float, xcolor, thmtools, thmbox}
\usepackage[english]{babel}

\makeatletter
\renewcommand\thmbox@headstyle[2]{\bfseries #1}
\makeatother
\newtheorem[style=M,bodystyle=\normalfont]{theorem}{Theorem}
\newtheorem[style=M,bodystyle=\normalfont]{corollary}{Corollary}
\newtheorem[style=M,bodystyle=\normalfont]{lemma}{Lemma}
\newtheorem[style=M,bodystyle=\normalfont]{definition}{Definition}


\title{Foundation Of Operations Research \\ \textit{Theory}}
\author{Christian Rossi}
\date{Academic Year 2023-2024}

\begin{document}

\maketitle

\newpage

\begin{abstract}
    Operations Research is the branch of applied mathematics dealing with quantitative methods to analyze and solve
    complex real-world decision-making problems. 
    
    The course covers some of the fundamental concepts and methods of Operations Research pertaining to graph optimization, 
    linear programming and integer linear programming. 
    
    The emphasis is on optimization models and efficient algorithms with a wide range of important applications in 
    engineering and management.  
\end{abstract}

\newpage

\tableofcontents

\newpage
 
\chapter{Introduction}
    \section{Definition}
    \begin{definition}
        \emph{Operations Research} is the branch of mathematics in which mathematical models and quantitative methods are used to analyze complex decision-making problems and find
        near-optimal solutions.
    \end{definition}
    It is an interdisciplinary field at the interface of applied mathematics, computer science, economics and industrial engineering. 

    \section{Decision-making problems}
    \begin{definition}
        The \emph{decision-making problems} are problems in which we must choose a feasible solution among many alternatives based on one or several criteria. 
    \end{definition}
    The more complex decision-making problems are tackled via a mathematical modelling approach (mathematical models, algorithms and computer implementations). 
    Those problems can be classified in the following categories: 
    \begin{enumerate}
        \item Assignment problem: given $m$ jobs and $m$ machines, suppose that each job can be executed by any machine and that $t_{ij}$ is the execution time of job $J_i$ one
            machine $M_j$. We want to decide which job assign to each machine to minimize the total execution time. Each job must be assigned to exactly one machine, and each 
            machine to exactly one job. The number of feasible solution is equal to $m!$. 
        \item Network design: we want to decide how to connect $n$ cities via a collection of possible links to minimize the total link cost. 
            Given a graph $G=(N,E)$ with a node $i \in N$ for each city and an edge $\{i,j\} \in E$ of cost $c_ij$, select a subset of edges of minimum total cost, guaranteeing that 
            all pairs of nodes are connected. The number of feasible solution is equal to $2^{\left\lvert E \right\rvert}$. 
        \item Shortest path: given a direct graph that represents a road network with distances (traveling times) for each arc, determine the shortest path between two points (nodes).
        \item Personnel scheduling: determine the week schedule for the hospital personnel, to minimize the number of people involved while meeting the daily requirements.
        \item Service management: determine how many desks to open at a given time of the day so that the average customer waiting time does not exceed a certain value. 
        \item Multi-criteria problem: decide which laptop to buy considering the price, the weight and the performance. 
        \item Maximum clique (community detection in social networks): determine the complete sub-graph of a graph, with the maximum number of vertices.
    \end{enumerate}

    \section{History}
    In the World War II, teams of scientists were asked to do research on the most efficient way to conduct the operations.
    In the decades after the war, the techniques became public and began to be applied more widely to problems in business, industry and society.
    During the industrial boom, the substantial increase in the size of the companies and organizations gave rise to more complex decision-making problems. 
    The favorable circumstances that permitted this were: 
    \begin{itemize}
        \item Fast progress in Operations Research and in numerical analysis methodologies. 
        \item Advent and diffusion of computers (more computing power and widespread software).
    \end{itemize}

    \section{Operations Research workflow}
    \begin{figure}[H]
        \centering
        \includegraphics[width=1\linewidth]{images/study.png}
    \end{figure}
    The main steps in studying an Operations Research problem are: 
    \begin{enumerate}
        \item Define the problem.
        \item Build the model.
        \item Select or develop an appropriate algorithm. 
        \item Implement it or use an existing program. 
    \end{enumerate}
    After all this process we need to analyze the results with feedbacks (and eventually modify some previous step). 

    The model obtained with this process is a simplified representation of a real-world problem. To define it we must identify the fundamental elements of the problem and the main
    relationships among them. 
    \begin{example}
        A company produces three types of electronic devices: $D_1,D_2,D_3$, going through three main phases of the production process: assembly, refinement and quality control.
        The time required for each phase and product is: 
        \begin{table}[H]
            \centering
            \begin{tabular}{c|ccc|}
            \cline{2-4}
            \textbf{}                             & \textbf{$D_1$} & \textbf{$D_2$} & $D_3$ \\ \hline
            \multicolumn{1}{|c|}{Assembly}        & 80             & 70             & 120   \\
            \multicolumn{1}{|c|}{Refinement}      & 70             & 90             & 20    \\
            \multicolumn{1}{|c|}{Quality control} & 40             & 30             & 20    \\ \hline
            \end{tabular}
        \end{table}
        The available resources within the planning horizon in minutes are: 
        \begin{table}[H]
            \centering
            \begin{tabular}{|c|c|c|}
            \hline
            \textbf{Assembly} & \textbf{Refinement} & \textbf{Quality control} \\ \hline
            30 000            & 25 000              & 18 000                   \\ \hline
            \end{tabular}
        \end{table}
        The unary product for each product in: 
        \begin{table}[H]
            \centering
            \begin{tabular}{|c|c|c|}
            \hline
            $D_1$          & $D_2$          & $D_3$ \\ \hline
            1600           & 1000           & 2000  \\ \hline
            \end{tabular}
        \end{table}
        The main assumption is that the company can sell whatever it produces. 

        The mathematical model that describes the problem given before is the following: 
        \begin{itemize}
            \item Decision variables: $x_j$ is the number of devices $D_j$ produced for $j=1,2,3$.
            \item Objective function: we need to maximize the earning, so we have: 
                \[\max{z}=1.6x_1+1x_2+2x_3\]
            \item Constraints: they are on the production limit of each phase, that are: 
                \[80x_1+70x_2+120x_3 \leq 30000\]
                \[70x_1+90x_2+20x_3 \leq 25000\]
                \[40x_1+30x_2+20x_3 \leq 18000\]
            \item Variable type: the variables must be non-negative values, so we have $x_1,x_2,x_3 \geq 0$.
        \end{itemize}
    \end{example}
    \begin{example}
        An insurance company must decide which investments to select out of a given set of possible assets.
        \begin{table}[H]
            \centering
            \begin{tabular}{|c|ccc|}
            \hline
            \textbf{Investments} & \textbf{Area} & \textbf{Capital ($c_j$)} & \textbf{Return ($r_j$)} \\ \hline
            A (automotive)       & Germany       & $150 000$                & $11\%$                  \\
            B (automotive)       & Italy         & $150 000$                & $9\%$                   \\
            C (ICT)              & USA           & $60 000$                 & $13\%$                  \\
            D (ICT)              & Italy         & $100 000$                & $10\%$                  \\
            E (real estate)      & Italy         & $125 000$                & $8\%$                   \\
            F (real estate)      & France        & $100 000$                & $7\%$                   \\
            G (treasury bonds)   & Italy         & $50 000$                 & $3\%$                   \\
            H (treasury bonds)   & UK            & $80 000$                 & $5\%$                   \\ \hline
            \end{tabular}
        \end{table}
        The available capital is $600\:000$ euro. It is required to take at most five different investments. It is also required to take at maximum three investments in Italy and 
        maximum three abroad. 

        The mathematical model that describes the problem given before is the following:
        \begin{itemize}
            \item Decision variables: boolean value to communicate if the investment is selected or not: $x_j=1$ if the $j$-th investment is selected and $x_j=0$ otherwise, for
                $j=0,\dots, 8$.
            \item Objective function: we need to maximize the expected return, so we have: 
                \[\max{z}=\sum_{j=1}^8{c_jr_jx_j}\]
            \item Constraints: there is a constraint on the capital that insurance
                \[\sum_{j=1}^8{c_jx_j} \leq 800\]
                There is a constraint also on the max number of general investment and on the region they are coming from formalized asked
                \[\sum_{j=1}^8{x_j} \leq 5\]
                \[x_2+x_4+x_5+x_7 \leq 3\]
                \[x_1+x_3+x_6+x_8 \leq 3\]
            \item Variable type: the variables are binary integer defined as $x_j \in \{0,1\} \:\: 1 \leq j \leq 8$. 
        \end{itemize}
        The variant requires that if any of the ICT investment is selected, then at least one of the treasury bond must be select. This requires one new constraint that is: 
        \[\dfrac{x_3+x_4}{2} \leq x_7+x_8\]
        It is divided by two because if both ICT are selected at least one treasury bound must be selected and not two. 
    \end{example}

\end{document}
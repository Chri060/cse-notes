\documentclass[12pt, a4paper]{report}
\usepackage{graphicx}
\usepackage[english]{babel}
\usepackage{amsthm}
\usepackage{amssymb}
\usepackage{amsmath}
\usepackage{algorithm}
\usepackage{algpseudocode}

\newtheorem{theorem}{Theorem}[section]
\newtheorem{corollary}{Corollary}[theorem]
\newtheorem{lemma}[theorem]{Lemma}
\theoremstyle{remark}
\newtheorem*{remark}{Definition}

\title{Foundation Of Operations Research \\ \textit{Theory}}
\author{Christian Rossi}
\date{Academic Year 2023-2024}

\begin{document}

\maketitle

\newpage

\begin{abstract}
    Operations Research is the branch of applied mathematics dealing with quantitative methods to analyze and solve
    complex real-world decision-making problems. 
    
    The course covers some of the fundamental concepts and methods of Operations Research pertaining to graph optimization, 
    linear programming and integer linear programming. 
    
    The emphasis is on optimization models and efficient algorithms with a wide range of important applications in 
    engineering and management.  
\end{abstract}

\newpage

\tableofcontents

\newpage

\chapter{Introduction}
\section{Definition}
The beginning of the activity called Operations Research has generally been attributed to the military services early in 
World War II. As its name implies, it involves "research on operations". Thus, Operations Research is applied to problems that
concern how to conduct and coordinate the operations within an organization. 

Operations Research is used as a support in decision making in any complex systems using: mathematical models, quantitative methods
and efficient algorithms. The decisions involving this field are at variuos level: strategic (design of a railroad),
tactical (quantity of trains to use) and operational (schedule the train routes).

The Operations Research expert needs the following abilities: 
\begin{itemize}
    \item Analyze the problem.
    \item Knowledge of the Operations Research methods.
    \item Ability in building mathematical models. 
    \item Interpret the results.
\end{itemize}

To define an Operations Research multiperiod problem we need the following elements:
\begin{enumerate}
    \item Sets (possible value for the time period).
    \item Parameters (all the known parameters in the problem).
    \item Variables (values that can vary in the variuos time periods).
    \item Constraints (constraints on variables possible values).
    \item Objective function (functions that indicates the reuqest with a maximum or minimum formula).
\end{enumerate}
\section{Variables}
The variables can be of various types: 
\begin{itemize}
    \item Non-negative absolute ($0,1,2,\dots$).
    \item Unrestricted absolute ($\dots,-1,0,1,\dots$): those values can be obtaining by subtracting two non-negative numbers.
    \item Relative ($\%$).
    \item Logical ($1$ (true) or $0$ (false)).
    \item Discrete (finite set of values).
\end{itemize}
\section{Constraints}
The constraints (that are the rules that the problem has to follow) can be of various types: 
\begin{itemize}
    \item Availability: the number of the used elements cannot exceed the available quantity.
    \item Requirement: requires at least a fixed number of some elements.
    \item Blending: requires a certain percentage presence of an element.
    \item Logical: used for logical operators in propositional logic. It's solved as a SAT problem.
    \item Flow: set a constraints on a graph.
\end{itemize}
\section{Objective functions}
A problem can have from zero to multiple objective functions. They are used to maximize or minimize the overall cost of the problem.
If the problem has multiple objective functions is called multicriteria analysis. 

\newpage

\chapter{Graph and network optimization}
\section{The terminology of networks}
A network consist of a set of points and a set of lines connecting certain pairs of the points.
\begin{remark}
    The points are called \emph{nodes}. 
\end{remark}
\begin{remark}
    The lines are called \emph{arcs}. If flow through an arc is allowed in only in one direction the arc is said to be a 
    \emph{directed arc}. The direction of a directed arc is indicated by adding an arrowhead at the end of the line representing the arc. 
    If flow through an arc is allowed in either directions the arc is said to be an \emph{undirected arc}.
    A network that has only directed arcs is called a \emph{directed networks}. Similarly, if all its arcs are undirected,
    the network is said to be an \emph{undirected network}. 
\end{remark}
\begin{remark}
    A \emph{path} between tho nodes is a sequence of distinct arcs connecting these nodes. A \emph{directed path} from node $i$ to
    node $j$ is a sequence of connecting arcs whose direction is toward node $j$ along this path, so that flow from node $i$ to node
    $j$ along this path is feasible. An \emph{undirected path} from node $i$ to node $j$ is a sequence of connecting arcs whose 
    direction can be either toward or away from node $j$ along this path. A \emph{simple path} is a path with no repeated arcs. 
    An \emph{elementary path} is a path without repeated nodes. A \emph{hemiltonian cycle} is a cycle where all the nodes of the graph 
    are visited.
\end{remark}
\begin{remark}
    A path that begins and ends at the same node is called a \emph{cycle}. In a directed network, a cycle is either a directed or an
    undirected cycle, depending on wheather the path involved is a directed or an undirected path. A \emph{simple cycle} is a cycle with no repeated arcs.
    An \emph{elementary cycle} is a cycle with no repeated nodes.
\end{remark}
\begin{remark}
    Two nodes are said to be \emph{connected} if the network contains at least one undirected path between them. A \emph{connected network} 
    is a network where every pair of nodes is connected.
\end{remark}
\begin{remark}
    A \emph{spanning tree} is a connected network that contains no undirected cycles for every nodes.
\end{remark}
Other useful definitions are: 
\begin{itemize}
    \item Eulerian property: if the number of nodes in a graph is odd is impossible to return to a given node without using one of the 
    arcs at least two times. 
    \item Eulerian tour: is a path where all the arcs are visited once.
\end{itemize}
\section{The minimum spanning tree problem}
The minimum spanning tree problem requires that the chosen links among all the possible ones must provide a path between each pair of nodes.
This can be summarized as: 
\begin{enumerate}
    \item You are given the nodes of a network but not the links. Instead, you are given the potential links and the positive length
    for each if it is inserted into the network. 
    \item You wish to design the network by inserting enough links to satisfy the requirement that there be a path between every pair of nodes.
    \item The objective is to satisfy this requirement ina a way that minimizes the total length of the links inserted into the network. 
\end{enumerate}
A network with $n$ nodes requires only $(n-1)$ links to provide a path between each pair of nodes. These links needs to be chosen in such 
a way that the resulting network forms a spanning tree. Therefore, the problem is to find the spanning tree with a minimum total length of
the links. This problem can be solved in a greedy way and the result is overall optimal. The algorithm that solves this problem is based on
the following steps: 
\begin{enumerate}
    \item Select any node arbitrarily, and connect it to the nearest distinct node. 
    \item Identify the unconnected node that is closest to a connected node, and then connect these two nodes. Repeat this step until all
    nodes have been connected.
    \item Tie breaking: ties for the nearest distinct node or the closest unconnected node may be broken arbitrarily, and the algorithm
    must still yield an optimal solution. However, such ties are a signal that there may be multiple optimal solutions.
\end{enumerate}
It is important to note that the choice of the initial node will not affect the resulting final solution. 
Another possible approach is given by the Kruskal algorithm. The problem can be formulated matematically in the following way: given an 
undirected graph $G=(N,A)$ with weights on the arcs $w_{ij} \geq 0 \: \forall (i,j) \in A$ find a connected subgraph 
$G^{'}=(N,A^{'}),A^{'} \subseteq A$ minimizing $\sum_{(i,j) \in A^{'}}w_{ij}$.  
\begin{algorithm}
    \caption{Kruskal algorithm for minimum spanning tree problem}
        \begin{algorithmic}[1]
            \State sort arcs in increasing order of $w_{ij}$
            \State $A^{'}=\varnothing$
            \While {$\left\lvert A^{'} \right\rvert != (n-1)$}
                \State consider the next arc $(1,j)$ in the order
                \If {$A^{'} \cup (i,j)$ does not induce a cycle}
                \State $A^{'} \leftarrow A^{'} \cup (i,j)$
                \EndIf
            \EndWhile
            \State \Return
        \end{algorithmic}
\end{algorithm}





\newpage

\chapter{Linear programming}



\newpage

\chapter{Integer linear programming}










    

    
\end{document}
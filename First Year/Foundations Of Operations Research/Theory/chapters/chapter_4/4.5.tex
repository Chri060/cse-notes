\section{Linear programming duality}

We can associate a closely related maximization (minimization) linear program with any minimization (maximization) linear program, based on the same parameters.
In both cases, we encounter different spaces and objective functions, but, in general, the optimal objective function values coincide.
This duality is valuable because finding the best lower bound (i.e., the maximum) is challenging, while finding the best upper bound is relatively simpler.

The general strategy involves combining the constraints linearly with non-negative multiplicative factors.
\begin{example}
    For instance, consider the following original problem:
    \begin{align*}
        \max                      \:&\: 4x_1+x_2+5x_3+3x_4          \\
        \textnormal{such that }     &\: x_1-x_2-x_3+3x_4 \leq 1     \\
                                    &\: 5x_1+x_2+3x_3+8x_4 \leq 55  \\
                                    &\: -x_1+2x_2+3x_3-5x_4 \leq 3  \\
                                    &\: x_1,x_2,x_3,x_4 \geq 0      
    \end{align*}
    We can construct the dual problem to solve it more easily:
    \begin{align*}
        \min                      \:&\: y_1+55y_2+3y_3              \\
        \textnormal{such that }     &\: y_1+5y_2-y_3 \geq 4         \\
                                    &\: -y_1+y_2+2y_3 \geq 1        \\
                                    &\: -y_1+3y_2+3y_3 \geq 5       \\
                                    &\: 3y_1+8y_2-5y_3 \geq 3       \\   
                                    &\: y_1,y_2,y_3 \geq 3          \\
    \end{align*}
\end{example}

\begin{property}
    The dual of the dual problem coincides with the primal problem. 
\end{property}

\begin{table}[H]
    \centering
    \begin{tabular}{cc}
    \hline
    \textbf{Primal (minimization)} & \textbf{Dual (maximization)}   \\ \hline
    $m$ constraints                & $m$ variables                  \\
    $n$ variables                  & $n$ constraints                \\
    right-hand side                & coefficient objective function \\
    coefficient objective function & right-hand side                \\
    $A$                            & $A^T$                          \\
    equality constraints           & unrestricted variables         \\
    unrestricted variables         & equality constraints           \\
    inequality constraints         & non-negative variables         \\
    non-negative variables         & inequality constraints         \\ \hline
    \end{tabular}
\end{table}

For the weak duality we have the following theorem. 
\begin{theorem}
    Given the primal problem: 
    \begin{align*}
        \min                      \:&\: z=c^T            \\
        \textnormal{such that }     &\: Ax\geq b         \\
                                    &\: x \geq 0
    \end{align*}
    And its dual: 
    \begin{align*}
        \max                      \:&\: w=b^Ty              \\
        \textnormal{such that }     &\: A^ty\leq c          \\
                                    &\: y \geq 0
    \end{align*}
    With $X=\{x|Ax \geq b, x \geq 0\} \neq \varnothing$ and $Y=\{y|A^Ty \leq c, y \geq 0\} \neq \varnothing$.
    For every feasible solution $x \in X$ of the primal problem and every feasible solution $y \in Y$ of the dual problem, we have: 
    \[b^Ty \leq c^Tx\]
\end{theorem}




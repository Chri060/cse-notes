\section{Linear programming duality}

We can associate a closely related maximization (minimization) linear program with any minimization (maximization) linear program, based on the same parameters.
In both cases, we encounter different spaces and objective functions, but, in general, the optimal objective function values coincide.
This duality is valuable because finding the best lower bound (i.e., the maximum) is challenging, while finding the best upper bound is relatively simpler.

The general strategy involves combining the constraints linearly with non-negative multiplicative factors.
\begin{example}
    For instance, consider the following original problem:
    \begin{align*}
        \max                      \:&\: 4x_1+x_2+5x_3+3x_4          \\
        \textnormal{such that }     &\: x_1-x_2-x_3+3x_4 \leq 1     \\
                                    &\: 5x_1+x_2+3x_3+8x_4 \leq 55  \\
                                    &\: -x_1+2x_2+3x_3-5x_4 \leq 3  \\
                                    &\: x_1,x_2,x_3,x_4 \geq 0      
    \end{align*}
    We can construct the dual problem to solve it more easily:
    \begin{align*}
        \min                      \:&\: y_1+55y_2+3y_3              \\
        \textnormal{such that }     &\: y_1+5y_2-y_3 \geq 4         \\
                                    &\: -y_1+y_2+2y_3 \geq 1        \\
                                    &\: -y_1+3y_2+3y_3 \geq 5       \\
                                    &\: 3y_1+8y_2-5y_3 \geq 3       \\   
                                    &\: y_1,y_2,y_3 \geq 3          \\
    \end{align*}
\end{example}

\begin{property}
    The dual of the dual problem coincides with the primal problem. 
\end{property}

\begin{table}[H]
    \centering
    \begin{tabular}{cc}
    \hline
    \textbf{Primal (minimization)} & \textbf{Dual (maximization)}   \\ \hline
    $m$ constraints                & $m$ variables                  \\
    $n$ variables                  & $n$ constraints                \\
    right-hand side                & coefficient objective function \\
    coefficient objective function & right-hand side                \\
    $A$                            & $A^T$                          \\
    equality constraints           & unrestricted variables         \\
    unrestricted variables         & equality constraints           \\
    inequality constraints         & non-negative variables         \\
    non-negative variables         & inequality constraints         \\ \hline
    \end{tabular}
\end{table}

\begin{theorem}[\textit{Weak duality theorem}]
    Given the primal problem: 
    \begin{align*}
        \min                      \:&\: z=c^T            \\
        \textnormal{such that }     &\: Ax\geq b         \\
                                    &\: x \geq 0
    \end{align*}
    And its dual: 
    \begin{align*}
        \max                      \:&\: w=b^Ty              \\
        \textnormal{such that }     &\: A^ty\leq c          \\
                                    &\: y \geq 0
    \end{align*}
    With $X=\{x|Ax \geq b, x \geq 0\} \neq \varnothing$ and $Y=\{y|A^Ty \leq c, y \geq 0\} \neq \varnothing$.
    For every feasible solution $x \in X$ of the primal problem and every feasible solution $y \in Y$ of the dual problem, we have: 
    \[b^Ty \leq c^Tx\]
\end{theorem}
\begin{proof}
    For every pair $x \in X$ and $y \in Y$, we have $Ax \geq b$, $x \geq O$ and $A^Ty \leq c$, $y \geq 0$ which imply that: 
    \[b^Ty \leq x^TA^Ty \leq x^Tc=c^Tx\]
\end{proof}
As a result we have that if $x$ is a feasible solution of $(P)$ $(x \in X)$, $y$ is a feasible solution of $(D)$ $(y \in Y)$, and the values of the respective objective functions coincide, $c^Tx=b^Ty$, then $x$ is optimal for $(P)$ and y is optimal for $(D)$. 
The optimal solutions are denoted by $x^{*}$ and $y^{*}$. 

\subsection{Strong duality theorem}
\begin{theorem}[\textit{Strong duality theorem}]
    If $X=\{x|Ax \geq b, b,x \geq 0\}$ and $\min\{c^Tx|x \in X\}$ is finite, there exist $x^{*} \in X$ and $y^{*} \in Y$ such that $c^Tx^{*}=b^Ty^{*}$. 
\end{theorem}
This is equivalent to say: 
\[\min\{c^Tx|x \in X\}=\max\{b^Ty|y \in Y\}\]
\begin{proof}
    Derive an optimal solution of $(D)$ from one of $(P)$. Given: 
    Given the primal problem: 
    \begin{align*}
        \min                      \:&\: z=c^T            \\
        \textnormal{such that }     &\: Ax = b         \\
                                    &\: x \geq 0
    \end{align*}
    And its dual: 
    \begin{align*}
        \max                      \:&\: w=b^Ty              \\
        \textnormal{such that }     &\: y^tA \leq c^T       \\
                                    &\: y \in \mathbb{R}^m
    \end{align*}
    And: 
    \[x^{*}=\begin{bmatrix}
        x^{*}_B \\ 
        x^{*}_N
    \end{bmatrix}\]
    Here, $x^{*}_B=B^{-1}b$ and $x^{*}_N=0$ an optimal feasible solution of $(P)$, provided (after a finite number of iterations) by the Simplex algorithm with Bland's rule. 
    Consider $\overline{y}^T=c_B^TB^{-1}$.
    
    Verify that $\overline{y}$ is a feasible solution of $(D)$. 
    For the non-basic variables: 
    \[\overline{c}_N^T=c_N^T-(c_B^TB^{-1})N=c_N^T-\overline{y}^TN \geq 0^T \implies \overline{y}^TN \geq c_N^T\]
    For the basic variables 
    \[\overline{c}_B^T=c_B^T-(c_B^TB^{-1})B=c_B^T-\overline{y}^TB \geq 0^T \implies \overline{y}^TB \geq c_B^T\]
    
    According to weak duality, $\overline{y}$ is an optimal solution of $(D)$: 
    \[\overline{y}^Tb?(c_B^TB^{-1})b=c_B^T(B^{-1}b)=c_B^Tx_B^{*}=c^Tx^{*}\]

    Hence, $\overline{y}=y^{*}$
\end{proof}

\paragraph*{Consequences}
Since in a linear programming problem, only one of the three following cases can occur: 
\begin{enumerate}
    \item There exists an optimal solution. 
    \item The problem is unbounded: the optimal cost is $+\infty$ for maximization problems and $-\infty$ for minimization problems. 
    \item The problem is infeasible. 
\end{enumerate}
\begin{table}[H]
    \centering
    \begin{tabular}{ccccc}
                                                     &                         & \multicolumn{3}{c}{\textbf{Dual}}                                  \\
                                                     &                         & \textit{Finite optimum} & \textit{Unbounded} & \textit{Infeasible} \\
    \multirow{3}{*}{\rotatebox{90}{\textbf{Primal}}} & \textit{Finite optimum} & \checkmark              & \tikzxmark         & \tikzxmark          \\
                                                     & \textit{Unbounded}      & \tikzxmark              & \tikzxmark         & \checkmark          \\
                                                     & \textit{Infeasible}     & \tikzxmark              & \checkmark         & \checkmark                   
    \end{tabular}
\end{table}

\subsection{Complementary slackness}
\begin{theorem}[Complementary slackness]
    Let $x \in X^\ast$ and $y \in Y^\ast$ be feasible solutions to the primal and dual problem, respectively.
    Then $x$ and $p$ are optimal for their respective problems if and only if:
    \[y_i^\ast \left( a^T_i x - b_i \right) = 0 \quad  \forall, i \in \left\{ i = 1, \ldots, m \right\}\]
    \[\left( c_j - y^T A_j \right) x_j^\ast = 0 \quad  \forall, j \in \left\{ j = 1, \ldots, n \right\}\]
    Where $a_i$ denotes the $i$-th row of $A$ and $A_j$ the $j$-th column of $A$.
\end{theorem}

At optimality, the product of each variable with the corresponding slack variable of the constraint of the relative dual problem must be zero. 
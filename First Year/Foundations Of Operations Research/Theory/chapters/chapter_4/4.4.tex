\section{Simplex method}

The idea of the simplex method is that, given a linear program in standard form: 
\begin{align*}
    \min                      \:&\: z=c^Tx              \\
    \textnormal{such that }     &\: Ax=b                \\
                                &\: x \geq 0
\end{align*}
we will examine a sequence of basic feasible solutions with non-increasing objective function values until an optimal solution is reached, or the  linear program is 
found to be unbounded. 

\subsection{Optimality test}
Because a basic feasible solution is characterized by $x_B=B^{-1}b,x_N=0$, we can express the objective function in terms of only the non-basic variables as follows:
\[c^Tx=c_B^TB^{-1}b+\left(c_N^T-c_B^TB^{-1}N\right)x_N\]
\begin{definition}[\textit{Vector of reduced cost}]
    The vector: 
    \[\overline{c}^T=c^T-C_B^TB^{-1}A=\left[c^T-C_B^TB^{-1}B,c^T-C_B^TB^{-1}N\right]\]
    is the vector of reduced costs with respect to the basis $B$. 
\end{definition}
$\overline{c}_j$ represents the change in the objective function value if non-basic $x_j$ is increased from 0 to 1 while keeping all other non-basic variables at 0. 
The solution value changes by:
\[\Delta z=\theta \cdot \overline{c}_j\]

Consider a linear program $\min\{c^Tx|Ax=b,x\geq 0\}$ and a feasible basis $B$. 
\begin{proposition}
    If $\overline{c}_N \geq 0$, then the basic feasible solution $(x_B^T,x_N^T)$, where $x_B=B^{-1}b \geq 0$ and $x_N = 0$, with a cost of  $c_B^TB^{-1}b$ is a global optimum. 
\end{proposition}
\begin{proof}
    If $\overline{c}_N \geq 0$, it implies that: 
    \[c^Tx=c_B^TB^{-1}b+\overline{c}_N^Tx_N \geq c_B^TB^{-1}b \:\: \forall x \geq 0, Ax=b\]
\end{proof}
This optimality condition is sufficient, but in general, it's not necessary.
\begin{example}
    Consider the following linear program: 
    \begin{align*}
        \min                      \:&\: -x_1-x_2          \\
        \textnormal{such that }     &\: x_1-x_2+s_1=1  \\
                                    &\: x_1+x_2+s_2=3  \\
                                    &\: x_1,x_2,s_1,s_2 \geq 0
    \end{align*}
    The matrices associated with the given constraints are:
    \[
    A=
    \begin{bmatrix}
        1 & -1 & 1 & 0  \\
        1 & 1 & 0 & 1 
    \end{bmatrix}
    \:\:\:\:\:\:
    b=
    \begin{bmatrix}
        -1 \\
        -1 \\
        0  \\
        0
    \end{bmatrix}
    \]
    Consider the solution with $x_B=(x_1,s_2)=(1,2)$, $x_n=(x_2,s_1)$, and $z=-1$. 
    We have:
    \[
    B=
    \begin{bmatrix}
        1 & 0  \\
        1 & 1  
    \end{bmatrix}
    \:\:\:\:\:\:
    N=
    \begin{bmatrix}
        -1 & 1  \\
        1  & 0  
    \end{bmatrix}
    \]
    Therefore, we have:
    \[c_B^T=c_N^T=\begin{bmatrix} -1 & 0 \end{bmatrix} \:\:\:\:\:\: B^{-1}=
    \begin{bmatrix}
        1  & 0  \\
        -1 & 1  
    \end{bmatrix}
    \]
    The reduced costs for $x_2,s_1$ are: 
    \[\overline{c}_N^T=c_N^T-c_B^TB^{-1}N=\begin{bmatrix} -2 & 0 \end{bmatrix}\]
    Since $\overline{c}_2=-2<0$, increasing $x_2$ to 1 will improve the solution by $-2$. 
\end{example}

\newpage
\subsection{Vertex selection}
\begin{example}
    Consider the following linear problem:
    \begin{align*}
        \min                      \:&\: -x_1-3x_2          \\
        \textnormal{such that }     &\: x_1+x_2+s_1 = 6  \\
                                    &\: 2x_1+x_2+s_2 = 8  \\
                                    &\: x_1,x_2,s_1,s_2 \geq 0
    \end{align*}
    Moving from vertex one to vertex five occurs as follows:
    \begin{itemize}
        \item In vertex one, $x_1 = 0, X_2=0 \implies s_1 = 6, s_2 = 8$ with $x_B = (s_1, s_2), x_N = (x_1, x_2)$.
        \item In vertex five, $x_2=0,s_2=0 \implies x_1=4,s_1=2$ with $x_B = (x_1, s_1), x_N = (x_2, s_2)$.
    \end{itemize}
    As a result, $x_1$ enters the basis $B$ while $s_2$ exits the basis $B$.

    By expressing the basic variables in terms of the non-basic variables, we obtain:
    \[s_1=6-x_1-x_2\]
    \[s_2=8-2x_1-x_2\]
    Now, we increase $x_1$ while keeping $x_2 = 0$.
    Since $s_1=6-x_1 \geq 0$ implies $x_1 \leq 6$ and $s_2=8-2x_1 \geq 0$ implies $x_1 \leq 4$, the upper limit on the increase of $x_1$ is: 
    \[x_1 \leq \min\{6,4\}=4\]
    We move from vertex one to vertex five by letting $x_1$ enter the basis and $s_2$ exit the basis ($s_1=2$ and $s_2=0$). 
\end{example}
Note that when transitioning from the current vertex to an adjacent one, we substitute one column of the basis matrix $B$ with one column of the non-basis matrix $N$.

Given a basis $B$, the system:
\[Ax=b \Leftrightarrow \sum_{j=1}^{m}a_{ij}x_i=b_j \:\:\:\:\:\: \textnormal{for }i=1,\dots,m\]
Can be expressed in canonical form: 
\[x_B+B^{-1}N_{x_N}=B^{-1}b \Leftrightarrow x_B+\overline{N}_{x_N}=\overline{b}\]
This form emphasizes the basic feasible solution as $(x_B,x_N)=(B^{-1}b,0)$. 
This involves pre-multiplying the system by $B^{-1}$: 
\[B^{-1}Bx_B+B^{-1}Nx_N=B^{-1}b \implies Ix_B+\overline{N}x_N=\overline{b}\]
In the canonical form: 
\[x_{B_i}+\sum_{j=1}^{n-m}\overline{a}_{ij}x_{N_j}=\overline{b}_i \:\:\:\:\:\: \textnormal{for }i=1,\dots,m\]
The basic variables are expressed in terms of non-basic variables: $x_B=\overline{b}-\overline{N}_{x_N}$. 
If we increase the value of a non-basic variable $x_s$ (starting from 0) while keeping all other non-basic variables at 0, the system becomes: 
\[x_{B_i}+\overline{a}_{is}x_s=\overline{b}_i \Leftrightarrow x_{B_i}=\overline{b}_i-\overline{a}_{is}x_s \:\:\:\:\:\: \textnormal{for }i=1,\dots,m\]
To ensure that $x_{B_i} \geq 0$ for each $i$, we need to satisfy: 
\[\overline{b}_i-\overline{a}_{is}x_s \geq 0 \implies x_s \geq \dfrac{\overline{b}_i}{\overline{a}_{is}} \:\:\:\:\:\: \textnormal{for }\overline{a}_{is} \geq 0\]
The value of $x_s$ can be increased up to: 
\[\theta^{*}=\min_{i=1,\dots,m}{\left[\dfrac{\overline{b}_i}{\overline{a}_{is}} \textnormal{ such that }\overline{a}_{is} \geq 0\right]}\]
The value of $x_r$ where $r=\textnormal{argmin}_{i=1,\dots,m}{\left[\frac{\overline{b}_i}{\overline{a}_{is}} \textnormal{ such that }\overline{a}_{is} \geq 0\right]}$ decreases to 0 and exits the basis. 

\subsection{Change of basis}
Consider a feasible basis $B$ and a non-basic variable $x_s$ (part of $x_N$) with a reduced cost $\overline{c}_ < 0$.
To improve the objective function value, we aim to increase $x_s$ as much as possible while keeping all other non-basic variables at zero.
Among the basic variables $x_r$ (in $x_B$) where $x_r>0$, one imposes the strictest upper limit, denoted as $\theta^{*}$ on the increase of $x_s$.
If $\theta^{*} > 0$, the resulting basic feasible solution yields a superior objective function value.
This new basis differs from the previous one by just a single column, representing adjacent vertices in the solution space.
Transitioning from the canonical form of the current basic feasible solution:
$B^{-1}Bx_B+B^{-1}Nx_N=B^{-1}b$
to that of an adjacent basic feasible solution doesn't require computing $B^{-1}$ from scratch.
The inverse matrix $B^{-1}$ of the new basis $B$ can be obtained by applying a unique pivoting operation to the inverse of the previous basis, which differs by just one column.

The pivoting operation aligns with the Gaussian elimination method used to solve systems of linear equations. 
For a given system $Ax=b$, the steps are as follows:
\begin{enumerate}
    \item Select a non-zero coefficient $\overline{a}_{rs}$ as the pivot.
    \item Divide the $r$-th row by $\overline{a}_{rs}$. 
    \item For each row $i$ where $i \neq r$ and $\overline{a}_{rs} \neq 0$, subtract the resulting $r$-th row multiplied by $\overline{a}_{rs}$. 
\end{enumerate}
It's important to note that this procedure preserves the feasibility of the solutions.

\subsection{Tableau representation}
A system with non-negativity constraints can be represented as a table in the following manner:
\begin{table}[H]
    \centering
    \begin{tabular}{cc}
                              & $x_1 \dots x_n$            \\ \hline
    \multicolumn{1}{|c|}{0}   & \multicolumn{1}{c|}{$c^T$} \\ \hline
    \multicolumn{1}{|c|}{$b$} & \multicolumn{1}{c|}{$A$}   \\ \hline
    \end{tabular}
\end{table}
If we consider a basis $B$ and a partition $A=\left[B|N\right]$, with $0=c^Tx-z$, the representation becomes:
\begin{table}[H]
    \centering
    \begin{tabular}{cccc}
                              & $x_1 \dots x_n$              & $x_{m+1}\dots x_n$           & $z$                                                                                                                                                                      \\ \hline
    \multicolumn{1}{|c|}{0}   & \multicolumn{1}{c|}{$c^T_B$} & \multicolumn{1}{c|}{$c^T_N$} & \multicolumn{1}{c|}{$-1$}                                                                                                                                                \\ \hline
    \multicolumn{1}{|c|}{$b$} & \multicolumn{1}{c|}{$A$}     & \multicolumn{1}{c|}{$N$}     & \multicolumn{1}{c|}{$\begin{matrix}0\\\vdots\\0\end{matrix}$} \\ \hline
    \end{tabular}
\end{table}
By performing pivoting operations, we transform the tableau into canonical form with respect to $B$:
\begin{table}[H]
    \centering
    \begin{tabular}{cccc}
                                                                &                                           & $x_1 \dots x_n$                       & $x_{m+1}\dots x_n$                                \\ \cline{2-4}
    $-z$                                                        & \multicolumn{1}{|c|}{$-z_0$}              & \multicolumn{1}{c|}{$0 \dots 0$}      & \multicolumn{1}{c|}{$\overline{c}_N^T$}           \\ \cline{2-4}
    $\begin{matrix}X_{B[1]}\\\vdots\\X_{B[m]}\end{matrix}$      & \multicolumn{1}{|c|}{$\overline{b}$}      & \multicolumn{1}{c|}{$I$}              & \multicolumn{1}{c|}{$\overline{N}$}               \\ \cline{2-4}
    \end{tabular}
\end{table}
\begin{example}
    For the linear program:
    \begin{align*}
        \min                      \:&\: -x_1-x_2          \\
        \textnormal{such that }     &\: 6x_1+4x_2+x_3=24  \\
                                    &\: 3x_1-2x_2+x_4=6  \\
                                    &\: x_1,x_2,x_3,x_4 \geq 0
    \end{align*}
    The corresponding tableau is: 
    \begin{table}[H]
        \centering
        \begin{tabular}{cccccc}
                                   &                         & $x_1$ & $x_2$ & $x_3$ & $x_4$                  \\ \cline{2-6} 
        \multicolumn{1}{c|}{$-z$}  & \multicolumn{1}{c|}{0}  & -1    & -1    & 0     & \multicolumn{1}{c|}{0} \\ \cline{2-6} 
        \multicolumn{1}{c|}{$x_3$} & \multicolumn{1}{c|}{24} & 6     & 4     & 1     & \multicolumn{1}{c|}{0} \\
        \multicolumn{1}{c|}{$x_4$} & \multicolumn{1}{c|}{6}  & 3     & -2    & 0     & \multicolumn{1}{c|}{1} \\ \cline{2-6} 
        \end{tabular}
    \end{table}
    Now we pivot the tableau with respect to 3 to derive an expression for $x_1$ from the pivot row and substituting it in all other rows.
    We have that $x_1$ enters the basis and $x_4$ exits the basis. 
    The tableau after these changes looks like this:
    \begin{table}[H]
        \centering
        \begin{tabular}{cccccc}
                                   &                         & $x_1$ & $x_2$ & $x_3$ & $x_4$                    \\ \cline{2-6} 
        \multicolumn{1}{c|}{$-z$}  & \multicolumn{1}{c|}{2}  & 0     & -5/3  & 0     & \multicolumn{1}{c|}{1/3} \\ \cline{2-6} 
        \multicolumn{1}{c|}{$x_3$} & \multicolumn{1}{c|}{12} & 0     & 8     & 1     & \multicolumn{1}{c|}{-2}  \\
        \multicolumn{1}{c|}{$x_1$} & \multicolumn{1}{c|}{2}  & 1     & -2/3  & 0     & \multicolumn{1}{c|}{1/3} \\ \cline{2-6} 
        \end{tabular}
    \end{table}
    In this case, only $x_2$ can enter the basis, and $x_3$ is the only one that can exit.
    The optimal solution can be found in subsequent tableaux, and the final result is:
    \begin{table}[H]
        \centering
        \begin{tabular}{cccccc}
                                   &                            & $x_1$ & $x_2$ & $x_3$ & $x_4$                         \\ \cline{2-6} 
        \multicolumn{1}{c|}{$-z$}  & \multicolumn{1}{c|}{9/2}   & 0     & 0     & 5/24  & \multicolumn{1}{c|}{-1/12}    \\ \cline{2-6} 
        \multicolumn{1}{c|}{$x_2$} & \multicolumn{1}{c|}{3/2}   & 0     & 1     & 1/8   & \multicolumn{1}{c|}{-1/4}     \\
        \multicolumn{1}{c|}{$x_1$} & \multicolumn{1}{c|}{3}     & 1     & 0     & 1/12  & \multicolumn{1}{c|}{1/6}      \\ \cline{2-6} 
        \end{tabular}
    \end{table}
    \begin{table}[H]
        \centering
        \begin{tabular}{cccccc}
                                   &                         & $x_1$ & $x_2$ & $x_3$ & $x_4$                    \\ \cline{2-6} 
        \multicolumn{1}{c|}{$-z$}  & \multicolumn{1}{c|}{6}  & 1/2   & 0     & 1/4   & \multicolumn{1}{c|}{0}   \\ \cline{2-6} 
        \multicolumn{1}{c|}{$x_2$} & \multicolumn{1}{c|}{6}  & 3/2   & 1     & 1/4   & \multicolumn{1}{c|}{0}   \\
        \multicolumn{1}{c|}{$x_4$} & \multicolumn{1}{c|}{18} & 6     & 0     & 1/2   & \multicolumn{1}{c|}{1}   \\ \cline{2-6} 
        \end{tabular}
    \end{table}
    With all reduced costs greater than or equal to zero, the optimal basic solution is: 
    \[x^{*T}=(0,6,0,18) \textnormal{ with } z^{*}=-6\]
\end{example}

\subsection{Algorithm}
\begin{algorithm}[H]
    \caption{Simplex algorithm (LP with minimization)}
        \begin{algorithmic}[1]
            \State Let $B[1],\dots,B[m]$ be the column indices of the initial feasible basis $B$
            \State Construct the initial tableau $\overline{A}=\{\overline{a}[i,j]|0 \leq i \leq m, 0 \leq j \leq n\}$ in canonical form with respect to $B$
            \State unbounded $\leftarrow$ false
            \State optimal $\leftarrow$ false
            \While{optimal=false and unbounded=false}
                \If{$\overline{a}[i,j] \geq 0 \: \forall j=1,\dots,m$}
                    \State optimal $\leftarrow$ true
                \Else
                    \State Select a non-basic variable $x_s$ with $\overline{a}[0,s] < 0$
                    \If{$\overline{a}[i,s] \geq 0 \: \forall j=1,\dots,m$}
                        \State unbounded $\leftarrow$ true
                    \Else 
                        \State $r \leftarrow \textnormal{argmin}\left[ \dfrac{\overline{a}[i,0]}{\overline{a}[i,s]} \textnormal{ with } 1 \leq i \leq m \textnormal{ and } \overline{a}[i,s]>0\right]$
                        \State $pivot(r,s)$
                        \State $B[r] \leftarrow s$
                    \EndIf
                \EndIf
            \EndWhile
        \end{algorithmic}
\end{algorithm}

\subsection{Degenerate basic feasible solution and convergence}
\begin{definition}[\textit{Degenerate basic feasible solution}]
    A basic feasible solution $x$ is degenerate if it contains at least one basic variable $x_j=0$.
\end{definition}
This property implies that a solution $x$ with more than $n - m$ zero values corresponds to multiple distinct bases. 
When degenerate basic feasible solutions are present, changing the basis may not necessarily result in a decrease in the objective function value.
In such cases, it is possible to cycle through a sequence of degenerate bases associated with the same vertex.
To manage these scenarios, various anti-cycling rules have been proposed to select the variables that enter and exit the bases (indices $r$ and $s$). 
One such rule is Bland's rule, which dictates choosing the variable with the smallest index among all candidates to enter or exit the basis.
\begin{proposition}
    The Simplex algorithm, when using Bland's rule, terminates after a maximum of $\binom{n}{m}$ iterations.
\end{proposition}
In some exceptional cases, the number of iterations may grow exponentially with respect to the values of $n$ or $m$. 
However, in general, the Simplex algorithm is highly efficient, and extensive experimental studies have shown that the number of iterations increases linearly with respect to $m$ and very slowly with respect to $n$. 

\subsection{Two-phase simplex method}
The two-phase Simplex method differs from the standard simplex method in that it first addresses an auxiliary problem aiming to minimize the sum of artificial variables. 
Once this auxiliary problem is solved and the final tableau is reorganized, the second phase, which resembles the standard Simplex method, begins.

Consider a linear program ($P$): 
\begin{align*}
    \min                      \:&\:z=c^T            \\
    \textnormal{such that }     &\: Ax=b            \\
                                &\: x \geq 0
\end{align*}
where $b \geq 0$. 
An auxiliary linear program, equipped with artificial variables $y_i$, $i=1,\dots,m$, is formulated as ($P_A$): 
\begin{align*}
    \min                      \:&\: \sum_{i=1}^{m}y_i               \\
    \textnormal{such that }     &\: Ax+ly=b                         \\
                                &\: x \geq 0, x \geq 0
\end{align*}
The initial basic feasible solution $y=b \geq 0$ and $x=0$ is obvious. 
Two possibilities arise:
\begin{enumerate}
    \item If $v^{*}>0$, then ($P$) is infeasible.
    \item If $v^{*}=0$, where $t^{*}=0$ and $x^{*}$ represents a basic feasible solution of ($P$). 
        In this case: 
        \begin{itemize}
            \item If $y_i$ is non-basic for all $i$ within $1 \leq i \leq m$, the corresponding columns and be removed to obtain a tableau in canonical form with respect to a basis, with the row for $z$ determined by substitution.
            \item In the case of a basic $y_i$ (indicating a degenerate basic feasible solution), a pivot operation is performed, exchanging $y_i$ with any non-basic variable $x_j$.
        \end{itemize}
\end{enumerate}
\begin{example}
    Consider the following linear program: 
    \begin{align*}
        \min                      \:&\: x_1+x_2+10x_3           \\
        \textnormal{such that }     &\: x_2+4x_3=2   \\
                                    &\: -2x_1+x_2-6x_3=2        \\
                                    &\: x_1,x_2,x_3 \geq 0
    \end{align*}
    In standard form, it can be expressed as:
    \begin{align*}
        \min                      v &=  y_1+y_2                 \\
        \textnormal{such that }     &\: x_2+4x_3+y_1=2          \\
                                    &\: -2x_1+x_2-6x_3+y_2=2    \\
                                    &\: x_1,x_2,x_3,y_1,y_2     \geq 0
    \end{align*}
    To obtain the canonical form of $v=y_1+y_2$, we can express $y_1$ and $y_2$ in terms of $x_1,x_2,x_3$. 
    \begin{table}[H]
        \centering
        \begin{tabular}{ccccccc}
                                   &                         & $x_1$ & $x_2$ & $x_3$                   & $y_1$ & $y_2$                  \\ \cline{2-7} 
        \multicolumn{1}{c|}{$-v$}  & \multicolumn{1}{c|}{-4} & 2     & -2    & \multicolumn{1}{c|}{2}  & 0     & \multicolumn{1}{c|}{0} \\ \cline{2-7} 
        \multicolumn{1}{c|}{$y_1$} & \multicolumn{1}{c|}{2}  & 0     & 1     & \multicolumn{1}{c|}{4}  & 1     & \multicolumn{1}{c|}{0} \\
        \multicolumn{1}{c|}{$y_2$} & \multicolumn{1}{c|}{2}  & -2    & 1     & \multicolumn{1}{c|}{-6} & 0     & \multicolumn{1}{c|}{1} \\ \cline{2-7} 
        \end{tabular}
    \end{table}
    \begin{table}[H]
        \centering
        \begin{tabular}{ccccccc}
                                   &                        & $x_1$ & $x_2$ & $x_3$                    & $y_1$ & $y_2$                  \\ \cline{2-7} 
        \multicolumn{1}{c|}{$-v$}  & \multicolumn{1}{c|}{0} & 2     & 0     & \multicolumn{1}{c|}{10}  & 2     & \multicolumn{1}{c|}{0} \\ \cline{2-7} 
        \multicolumn{1}{c|}{$x_2$} & \multicolumn{1}{c|}{2} & 0     & 1     & \multicolumn{1}{c|}{4}   & 1     & \multicolumn{1}{c|}{0} \\
        \multicolumn{1}{c|}{$y_2$} & \multicolumn{1}{c|}{0} & -2    & 0     & \multicolumn{1}{c|}{-10} & -1    & \multicolumn{1}{c|}{1} \\ \cline{2-7} 
        \end{tabular}
    \end{table}
    Following the pivot operations, we have identified the optimal solution for this auxiliary problem:
    \[x^{*}=(0,2,0), y^{*}=(0,0)\]
    This optimal solution yields an objective value of $v^{*}=0$. 

    To achieve this result, we selected the coefficient $-2$ from the row associated with $y_2$ as the pivot.
    \begin{table}[H]
        \centering
        \begin{tabular}{ccccccc}
                                   &                        & $x_1$ & $x_2$ & $x_3$                  & $y_1$ & $y_2$                     \\ \cline{2-7} 
        \multicolumn{1}{c|}{$-v$}  & \multicolumn{1}{c|}{0} & 0     & 0     & \multicolumn{1}{c|}{0} & 1     & \multicolumn{1}{c|}{1}    \\ \cline{2-7} 
        \multicolumn{1}{c|}{$x_2$} & \multicolumn{1}{c|}{2} & 0     & 1     & \multicolumn{1}{c|}{4} & 1     & \multicolumn{1}{c|}{0}    \\
        \multicolumn{1}{c|}{$x_1$} & \multicolumn{1}{c|}{0} & 1     & 0     & \multicolumn{1}{c|}{5} & 1/2   & \multicolumn{1}{c|}{-1/2} \\ \cline{2-7} 
        \end{tabular}
    \end{table}
    Hence, this represents an optimal basic feasible solution for ($P_A$), and the corresponding values $x=(0,2,0)$ form a basic feasible solution for ($P$).
    In the context of the original linear program, the objective function is given as:
    \[z=x_1+x_2+10x_3\]
    It's important to note that this expression is not in canonical form, primarily because $x_3$ is considered a non-basic variable. 
    However, by performing the following substitution:
    \[
        \begin{cases}
            x_2=2-4x_3 \\
            x_1=-5x_3
        \end{cases}
        \implies
        z=2+x_3
    \]
    Now, we derive the tableau that corresponds to the initial basic feasible solution for ($P$):
    \begin{table}[H]
        \centering
        \begin{tabular}{ccccc}
                                   &                         & $x_1$ & $x_2$ & $x_3$                  \\ \cline{2-5} 
        \multicolumn{1}{c|}{$-z$}  & \multicolumn{1}{c|}{-2} & 0     & 0     & \multicolumn{1}{c|}{1} \\ \cline{2-5} 
        \multicolumn{1}{c|}{$x_2$} & \multicolumn{1}{c|}{2}  & 0     & 1     & \multicolumn{1}{c|}{4} \\
        \multicolumn{1}{c|}{$x_1$} & \multicolumn{1}{c|}{0}  & 1     & 0     & \multicolumn{1}{c|}{5} \\ \cline{2-5} 
        \end{tabular}
    \end{table}
    Since the basic feasible solution is already optimal, there is no need for a phase two.
\end{example}

\subsection{Polynomial-time algorithms for linear programming}
Over the years, polynomial-time algorithms for linear programming have been developed, including:
\begin{itemize}
    \item \textit{Ellipsoid method} (Khachiyan, 1979), which holds significant theoretical importance.
    \item \textit{Interior point methods} (Karmarkar, 1984), known for their highly efficient variants when dealing with certain types of problem instances. 
\end{itemize}
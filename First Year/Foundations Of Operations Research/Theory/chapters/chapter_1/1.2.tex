\section{Decision-making problems}

\begin{definition}
    \emph{Decision-making problems} entail the challenge of selecting a viable solution from an array of alternatives, guided by one or multiple criteria.
\end{definition}
For more intricate decision-making problems, a mathematical modeling approach is employed. 
These problems can be categorized as follows:
\begin{enumerate}
    \item Assignment problem: given $m$ jobs and $m$ machines, suppose that each job can be executed by any machine and that $t_{ij}$ is the execution time of job $J_i$ one
        machine $M_j$. We want to decide which job assign to each machine to minimize the total execution time. Each job must be assigned to exactly one machine, and each 
        machine to exactly one job. The number of feasible solution is equal to $m!$. 
    \item Network design: we want to decide how to connect $n$ cities via a collection of possible links to minimize the total link cost. 
        Given a graph $G=(N,E)$ with a node $i \in N$ for each city and an edge $\{i,j\} \in E$ of cost $c_ij$, select a subset of edges of minimum total cost, guaranteeing that 
        all pairs of nodes are connected. The number of feasible solution is equal to $2^{\left\lvert E \right\rvert}$. 
    \item Shortest path: given a direct graph that represents a road network with distances (traveling times) for each arc, determine the shortest path between two points (nodes).
    \item Personnel scheduling: determine the week schedule for the hospital personnel, to minimize the number of people involved while meeting the daily requirements.
    \item Service management: determine how many desks to open at a given time of the day so that the average customer waiting time does not exceed a certain value. 
    \item Multi-criteria problem: decide which laptop to buy considering the price, the weight and the performance. 
    \item Maximum clique (community detection in social networks): determine the complete sub-graph of a graph, with the maximum number of vertices.
\end{enumerate}
\section{Introduction}

\begin{definition}[\textit{Integer linear programming problem}]
    An Integer linear programming (ILP) problem is an optimization problem having form:
    \begin{align*}
        \min                      \:&\: c^Tx           \\
        \textnormal{such that }     &\: Ax = b         \\
                                    &\: x \in \mathbb{Z}^n
    \end{align*}  
\end{definition}
Furthermore:
\begin{itemize}
  \item If $x_j \in \left\{ 0, 1 \right\}$ $\forall, j$, the  problem is called binary linear programming.
  \item If $\exists, i \ \text{such that}$ $x_i \notin \mathbb{Z}^n$, then the problem is called mixed integer linear programming.
\end{itemize}
Note that the integrality condition $x_i \in \mathbb{Z}$ is non-linear, since it can be expressed as $\sin(\pi x_j) = 0$.
  
\begin{definition}[\textit{Linear relaxation}]
    Let $ILP$ be an ILP problem:
    \begin{align*}
        z_{ILP}:=\min                      \:&\: c^Tx           \\
        \textnormal{such that }     &\: Ax \leq b               \\
                                    &\: x \in \mathbb{Z}^n      \\
                                    &\: x \leq 0
    \end{align*}  
    Then the LP problem:
    \begin{align*}
        z_{LP}:=\max                      \:&\: c^Tx                    \\
        \textnormal{such that }             &\: Ax \leq b               \\
                                            &\: x \leq 0
    \end{align*}  
    is the linear (or continuous) relaxation of $ILP$.
\end{definition}

\begin{property}[Bounds of ILP solutions]  
    For any ILP with $\max$, the optimal solution is bounded by the optimal solution of the LP relaxation:
    \[ z_{ILP} \leq z_{LP} \]
  
    For any ILP with $\min$, the optimal solution is bounded by the optimal solution of the LP relaxation:
    \[ z_{ILP} \geq z_{LP} \]
\end{property}
The feasible region of any ILP is a lattice of points, either finite or infinite according to the type of problem;
By deleting the integrality constraint, the ILP problem becomes an LP problem, and the optimal solution of the ILP problem isn't always the optimal solution of the LP problem;
\begin{figure}[H]
    \centering
    \begin{subfigure}[b]{0.495\textwidth}
        \centering
        \includegraphics[width=0.25\linewidth]{images/ilp.png}
        \caption{Lattice of integer points}
    \end{subfigure}
    \begin{subfigure}[b]{0.495\textwidth}
        \centering
        \includegraphics[width=0.25\linewidth]{images/ilp1.png}
        \caption{Relaxed Lattice of integer points}
    \end{subfigure}
    \caption{Lattice of integer points}
\end{figure}
  
\subsection{Solutions of the ILP problem}

\subparagraph*{Solutions as relaxation of the LP problem}
A plausible way to find a solution to the ILP problem would be to find a solution to the LP problem and then round it to the nearest integer point.
If an optimal solution of the LP problem is an integer, then it is also an optimal solution of the ILP problem.
However, often the rounded optimal solutions of the LP are either infeasible or:
\begin{itemize}
    \item \textit{Infeasible} solutions for the ILP.
    \item \textit{Useless} solutions for the ILP, as they are very different from an optimal solution of the ILP.
\end{itemize}
\begin{figure}[H]
    \centering
    \includegraphics[width=0.25\linewidth]{images/ilp2.png}
    \caption{Solution of the LP relaxation of the ILP problem}
\end{figure}

\subsection{Solution of Assignment and Transportation Problems}
Two important ILP problems, namely the Assignment and Transportation problems, have a solution that is the optimal solution of the LP relaxation;

\paragraph*{Assignment problem}
Given:
\begin{itemize}
    \item $m$ machines, $i = 1, \dots, m$
    \item $n$ jobs, $j = 1, \dots, n$, $n < m$
    \item $c_{ij}$ cost of assigning job $j$ to machine $i$
\end{itemize}
determine an assignment of jobs to the machines as to minimize the total cost, while assigning at least one job per machine and at most one machine for each job.
Variables $x_{ij}$ have value:
\[ x_{ij} = \begin{cases}
    1 \quad & \text{if job } j \text{ is assigned to machine } i \\
    0 \quad & \text{otherwise}
\end{cases} \]
The assignment problem is the following ILP problem:
\begin{align*}
\min        & \quad \sum_{i=1}^m \sum_{j=1}^n c_{ij} x_{ij}                                                      \\
\text{s.t.} & \quad \sum_{i=1}^{m} x_{ij} = 1 \quad \forall \, j \quad \text{at most one machine for each job}   \\
            & \quad \sum_{j=1}^{n} x_{ij} = 1 \quad \forall \, i  \quad \text{at least one job for each machine} \\
            & \quad x_{ij} \in \left\{ 0, 1 \right\} \quad \forall \, i, j
\end{align*}

\paragraph*{Transportation problem}
Given:
\begin{itemize}
\item $m$ productions plant, $i = 1, \dots, m$
\item $n$ clients, $j = 1, \dots, n$, $n > m$ by assumption
\item $c_{ij}$ cost of shipping one unit from plant $i$ to client $j$
\item $p_i$ production capacity of plant $i$
\item $d_j$ demand of client $j$
\item $q_{ij} \geq 0$ quantity shipped from plant $i$ to client $j$
\end{itemize}
determine a transportation plan that minimizes the total costs while satisfying the production and demand constraints.
Assumption: $\displaystyle \sum_{i=1}^m p_i = \geq \sum_{j=1}^{n} d_j$
Variables: $x_{ij}$ quantity shipped from plant $i$ to client $j$

The Assignment problem example contains a forcing constraint, while the Transportation problem adds a limit on the active number of variables;
\begin{definition}[\textit{Forcing constraint}]
    A constraint in the form: 
    \[ \displaystyle x \leq y\]
    is called a forcing constraint if both $x$ and $y$ are binary variables.
\end{definition}
\begin{definition}[\textit{Constraint on binary variables}]
    A constraint in the form:
    \[ \displaystyle \sum_{i=1}^n x_i \leq 1 \]
    where all $x_i$ are binary variables, implying that at most one of the variables $x_i$ can be one.
    Similarly, a constraint in the form: 
    \[ \displaystyle \sum_{i=1}^n x_i = 1 \]
    where all $x_i$ are binary variables, implies that exactly one variable $x_i$ must be one.
\end{definition}

The transportation-problem shows that the optimal solution of the LP relaxation of the transportation problem is also the optimal solution of the ILP problem.
\begin{theorem}[Solution of the Transportation Problem]
    If in a transportation problem $p_i, \ d_{ij}, \ q_{ij}$ are all integers, all the basic feasible solutions (vertices) of its linear relaxation are integers.
\end{theorem}

\begin{proof}
    Let $A$ be an integer constraint matrix of size $\left( mn + n + m \right) \times \left( mn \right)$, where $a_{ij} \in \left\{ -1, 0, 1 \right\}$.
    The right-hand side vector $b$ is composed of integer elements;
    the optimal solution for the linear relaxation is:
    \[ x^\ast = \begin{bmatrix}
        B^{-1} b \\ 0
    \end{bmatrix}
    \qquad
    B^{-1} = \dfrac{1}{|B|}
    \begin{bmatrix}
        \alpha_{11} & \dots  & \alpha_{1n} \\
        \ldots      & \ldots & \ldots      \\
        \alpha_{m1} & \dots  & \alpha_{mn}
    \end{bmatrix}
    \]
    where $\alpha_{ij} = (-1)^{i+j} \det\left( M_{ij} \right)$, $M_{ij}$ is the square submatrix obtained from $B$ by deleting the $i$-th row and the $j$-th column.
    Then:
    \begin{itemize}
    \item $B$ integer $\Rightarrow$ $\alpha_{ij}$ integer
    \item $\det\left( B \right) = \pm 1 \ \Rightarrow \ B^{-1} \ \text{is integer} \ \Rightarrow \ x^\ast \ \text{is integer}$
    \item It can be shown that $A$ is totally unimodular, $\Rightarrow \ \det\left( Q \right) = \left\{ -1, 0, 1 \right\}$ for any square submatrix $Q$ of $A$
    \end{itemize}
\end{proof}

\paragraph{Complexity of the ILP problem}
Most of the ILP problems are $\mathcal{NP}$-hard:
an algorithm able to solve them and prove that the solution is correct in polynomial time does not exist.
Different methods to find the optimal solutions exist, divided in:
\begin{itemize}
    \item \textit{Implicit enumeration} methods: provide an exact solution (a global optimum). 
        Branch and bound and dynamic programming methods are part of this category. 
    \item \textit{Cutting planes} methods: provide an exact solution (a global optimum). 
    \item \textit{Heuristic} algorithms: provide an approximate solution (a local optimum). 
        Greedy and local search algorithms are part of this category. 
\end{itemize}
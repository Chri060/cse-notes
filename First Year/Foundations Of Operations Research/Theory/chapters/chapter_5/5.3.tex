\section{Cutting plane method and Gomory fractional cuts}

By considering the ILP problem:
\begin{gather*}
    \begin{aligned}
    \min \quad        & c^T x                \\
    \text{s.t.} \quad & Ax \geq b            \\
                        & x \in \N, \ x \geq 0
    \end{aligned}
\end{gather*}

Having a feasible region:
\[ X = \left\{ x \in \Z \mid Ax \geq b,x \geq 0 \right\} \]
Under the assumption that $a_{ij} \in N$, $b_i \in N$, $c_i \in N$, the feasible region can be described by different sets of constraints that could be either weaker or tighter.
All infinite formulations are equivalent, but the optimal solution of the linear relaxations of the ILP ($x^\ast_{LP}$) may vary greatly.
\begin{definition}[\textit{Ideal formulation}]
    The ideal formulation is the formulation describing the convex hull of $X$, called $\conv(X)$, representing the smallest convex subset containing $X$
\end{definition}
Since all vertices have integer coordinates, for any $c \in \conv(X)$, the following relation holds:
\[ z^\ast_{LP} = z^\ast_{ILP} \]
and the optimum of the LP is also the optimum of the ILP.

\begin{theorem}[Ideal Formulation]
    For any feasible region $X$ of an ILP (bounded or unbounded), there exists an ideal formulation (a description of $\conv(X)$ involving a finite number of linear constraints) but the number of the constraints can be very large;
    the number of constraints grows exponentially with respect to the size of the original formulation.
\end{theorem}
Via this theorem, the solution of any ILP can be reduced to that of a single LP;
however, the ideal formulation is often either very large or very difficult to determine.

\subsection{Cutting plane method}
A full description of $\conv(X)$ is not required, as just a good description of the neighborhood of the optimal solution is needed.
\begin{definition}[\textit{Cutting plane}]
    A Cutting plane is an inequality $a^T x \leq b$ that is not satisfied by $x^\ast_{LP}$ but is satisfied by all the feasible solutions of the ILP.
\end{definition}


\subsection{Cutting plane methods and Gomory fractional cuts}
Given an initial formulation, iteratively add cutting planes as long as the linear relaxation does not provide an optimal integer solution.

Let $x^\ast_{LP}$ be the optimal solution of the linear relaxation of the ILP problem formulated as:
\[ \min \left\{ c^T x \mid Ax = b, x \geq 0 \right\} \]
And let $x^\ast_{B{r}}$ be the fractional basic variable.
The corresponding row of the optimal tableau is:
\[ x_{B[r]} + \displaystyle \sum_{j \vert x_j \in \N} \overline{a}_{rj} x_j = \overline{b}_r \]
\begin{definition}[\textit{Gomory cut}]
    The Gomory cut with respect to the fractional basic variable $x_{B{r}}$ is the inequality:
    \[ \displaystyle \sum_{j \vert x_j \in \N} \left( \overline{a}_{rj} - \lfloor \overline{a}_{rj} \rfloor \right) x_j \geq \overline{b}_r - \lfloor \overline{b}_r \rfloor \]
\end{definition}

\begin{property}
    The \textbf{integer} form:
    \[ \displaystyle x_{B[r]} + \sum_{j \vert x_j \in \N} \lfloor\overline{a}_{rj}\rfloor x_j \leq \lfloor\overline{b}_r\rfloor \]
    and the \textbf{fractional} form:
    \[ \displaystyle x_{B[r]} + \sum_{j \vert x_j \in \N} \left( \overline{a}_{rj} - \lfloor \overline{a}_{rj} \rfloor \right) x_j \geq \overline{b}_r - \lfloor \overline{b}_r \rfloor \]
    Are equivalent.
\end{property}

\paragraph*{Cutting plane method Algorithm}

\begin{lstlisting}
solve the linear relaxation of the ILP problem $\min\left\{c^T x \vert Ax = b, x \geq 0 \right\}$
let $x^\ast_{LP}$ be the optimal solution of the linear relaxation
while $x^\ast_{LP}$ is not integer do:
    select a basic variable $x_{B[r]}$ with fractional value
    generate the corresponding Gomory cut
    add constraint to the optimal tableau of the linear relaxation
    perform one iteration of the dual simplex method
end
\end{lstlisting}

\begin{theorem}
    If the ILP has a finite number of optimal solutions, the Cutting plane method with Gomory cuts is guaranteed to find an optimal solution.
\end{theorem}
The number of iterations that the algorithm needs to perform, however, is not known in advance and it's often very large.

\subsection{Above and Beyond the Cutting Plane Method}
\paragraph*{Other cutting planes}
Other types of generic cutting planes exists, including a large number of classes of cutting planes for specific problems.
The thorough study of the combinatorial structure of various problems led to:
\begin{itemize}
    \item the characterization of entire classes of facets.
    \item efficient procedures for generating the facets.
\end{itemize}

\paragraph*{Branch-and-Cut}
The combined approach of \textbf{branch-and-cut} aims at overcoming the disadvantages of the \textbf{branch-and-bound} method and pure \textbf{cutting plane methods}.
For each subproblem, the \textbf{branch-and-cut} method generates a set of \textbf{cuts} that are valid for the subproblem and adds them to the formulation of the subproblem.
Whenever the cutting planes become less effective, cut generation is stopped and a branching operation is performed.
The cuts then to strengthen the formulation of the various subproblems; the long series of cuts without sensible improvements are interrupted by branching operations.
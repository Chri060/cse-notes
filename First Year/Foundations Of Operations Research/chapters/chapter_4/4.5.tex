\section{Linear programming duality}

A closely related maximization (minimization) linear program can be associated with any minimization (maximization) linear program, sharing the same parameters.
Despite differing spaces and objective functions, the optimal values of the objective functions generally coincide. 
This duality is valuable because determining the best lower bound (maximum) is challenging, while finding the best upper bound is relatively simpler.

The general strategy involves linearly combining the constraints with non-negative multiplicative factors.
\begin{example}
    Consider the following original problem:
    \begin{align*}
        \max                      \:&\: 4x_1+x_2+5x_3+3x_4          \\
        \textnormal{such that }     &\: x_1-x_2-x_3+3x_4 \leq 1     \\
                                    &\: 5x_1+x_2+3x_3+8x_4 \leq 55  \\
                                    &\: -x_1+2x_2+3x_3-5x_4 \leq 3  \\
                                    &\: x_1,x_2,x_3,x_4 \geq 0      
    \end{align*}
    The dual problem can be constructed for easier solving:
    \begin{align*}
        \min                      \:&\: y_1+55y_2+3y_3              \\
        \textnormal{such that }     &\: y_1+5y_2-y_3 \geq 4         \\
                                    &\: -y_1+y_2+2y_3 \geq 1        \\
                                    &\: -y_1+3y_2+3y_3 \geq 5       \\
                                    &\: 3y_1+8y_2-5y_3 \geq 3       \\   
                                    &\: y_1,y_2,y_3 \geq 3          \\
    \end{align*}
\end{example}
\begin{property}
    The dual of the dual problem coincides with the primal problem.
\end{property}
\begin{table}[H]
    \centering
    \begin{tabular}{c|c|c|c}
        \textbf{Primal} & \textit{min} & \textit{max} & \textbf{Dual} \\ \hline
                    & $\geq b_j$    & $\geq b_j$            & \\ 
        Constraints & $\leq b_j$    & $\leq b_j$            & Variables\\ 
                    & $= b_j$       & $b_j$ free            & \\ \hline
                    & $\geq b_j$    & $\leq b_j$            & \\ 
        Variables   & $\leq b_j$    & $\geq b_j$            & Constraints \\ 
                    & $= b_j$       & $b_j$ free            & \\ 
    \end{tabular}
    \caption{Rules to transform a problem into its dual}
\end{table}

\subsection{Weak duality theorem}
\begin{theorem}[Weak duality theorem]
    Given the primal problem: 
    \begin{align*}
        \min                      \:&\: z=c^T            \\
        \textnormal{such that }     &\: Ax\geq b         \\
                                    &\: x \geq 0
    \end{align*}
    And its dual: 
    \begin{align*}
        \max                      \:&\: w=b^Ty              \\
        \textnormal{such that }     &\: A^ty\leq c          \\
                                    &\: y \geq 0
    \end{align*}
    With $X=\{x|Ax \geq b, x \geq 0\} \neq \varnothing$ and $Y=\{y|A^Ty \leq c, y \geq 0\} \neq \varnothing$.
    For every feasible solution $x \in X$ of the primal problem and every feasible solution $y \in Y$ of the dual problem, we have: 
    \[b^Ty \leq c^Tx\]
\end{theorem}
\begin{proof}
    For every pair $x \in X$ and $y \in Y$, we have $Ax \geq b$, $x \geq O$ and $A^Ty \leq c$, $y \geq 0$, which imply that: 
    \[b^Ty \leq x^TA^Ty \leq x^Tc=c^Tx\]
\end{proof}
As a result, if $x$ is a feasible solution of $(P)$ $(x \in X)$, $y$ is a feasible solution of $(D)$ $(y \in Y)$, and the values of the respective objective functions coincide, $c^Tx=b^Ty$, then $x$ is optimal for $(P)$ and $y$ is optimal for $(D)$.
The optimal solutions are denoted by $x^{*}$ and $y^{*}$. 

\subsection{Strong duality theorem}
\begin{theorem}[Strong duality theorem]
    If $X=\{x|Ax \geq b, b,x \geq 0\}$ and $\min\{c^Tx|x \in X\}$ is finite, there exist $x^{*} \in X$ and $y^{*} \in Y$ such that $c^Tx^{*}=b^Ty^{*}$. 
\end{theorem}
This is equivalent to saying:
\[\min\{c^Tx|x \in X\}=\max\{b^Ty|y \in Y\}\]
\begin{proof}
    Derive an optimal solution of $(D)$ from one of $(P)$.
    Given the primal problem: 
    \begin{align*}
        \min                      \:&\: z=c^T            \\
        \textnormal{such that }     &\: Ax = b         \\
                                    &\: x \geq 0
    \end{align*}
    And its dual: 
    \begin{align*}
        \max                      \:&\: w=b^Ty              \\
        \textnormal{such that }     &\: y^tA \leq c^T       \\
                                    &\: y \in \mathbb{R}^m
    \end{align*}
    And: 
    \[x^{*}=\begin{bmatrix}
        x^{*}_B \\ 
        x^{*}_N
    \end{bmatrix}\]
    Here, $x^{*}_B=B^{-1}b$ and $x^{*}_N=0$ an optimal feasible solution of $(P)$. 
    This solution is obtained after a finite number of iterations using the Simplex algorithm with Bland's rule.  

    Consider $\overline{y}^T=c_B^TB^{-1}$.
    Verify that $\overline{y}$ is a feasible solution of $(D)$. 
    For the non-basic variables: 
    \[\overline{c}_N^T=c_N^T-(c_B^TB^{-1})N=c_N^T-\overline{y}^TN \geq 0^T \implies \overline{y}^TN \geq c_N^T\]
    For the basic variables: 
    \[\overline{c}_B^T=c_B^T-(c_B^TB^{-1})B=c_B^T-\overline{y}^TB \geq 0^T \implies \overline{y}^TB \geq c_B^T\]
    
    According to weak duality, $\overline{y}$ is an optimal solution for $(D)$: 
    \[\overline{y}^Tb?(c_B^TB^{-1})b=c_B^T(B^{-1}b)=c_B^Tx_B^{*}=c^Tx^{*}\]
    Thus, $\overline{y}=y^{*}$
\end{proof}

\paragraph*{Consequences}
In linear programming problems, only one of the three cases can occur:
\begin{enumerate}
    \item There exists an optimal solution. 
    \item The problem is unbounded: the optimal cost is $+\infty$ for maximization problems and $-\infty$ for minimization problems. 
    \item The problem is infeasible. 
\end{enumerate}
\begin{table}[H]
    \centering
    \begin{tabular}{ccccc}
                                                     &                         & \multicolumn{3}{c}{\textbf{Dual}}                                  \\
                                                     &                         & \textit{Finite optimum} & \textit{Unbounded} & \textit{Infeasible} \\
    \multirow{3}{*}{\rotatebox{90}{\textbf{Primal}}} & \textit{Finite optimum} & \checkmark              & \tikzxmark         & \tikzxmark          \\
                                                     & \textit{Unbounded}      & \tikzxmark              & \tikzxmark         & \checkmark          \\
                                                     & \textit{Infeasible}     & \tikzxmark              & \checkmark         & \checkmark                   
    \end{tabular}
\end{table}

\subsection{Complementary slackness}
\begin{theorem}[Complementary slackness]
    Let $x \in X^{*}$ and $y \in Y^{*}$ be feasible solutions to the primal and dual problem, respectively.
    Then $x$ and $p$ are optimal for their respective problems if and only if:
    \[y_i^{*} \left( a^T_i x - b_i \right) = 0 \quad  \forall, i \in \left\{ i = 1, \ldots, m \right\}\]
    \[\left( c_j - y^T A_j \right) x_j^{*} = 0 \quad  \forall, j \in \left\{ j = 1, \ldots, n \right\}\]
    Where $a_i$ denotes the $i$-th row of $A$ and $A_j$ the $j$-th column of $A$.
\end{theorem}

At optimality, the product of each variable with the corresponding slack variable in the constraint of the dual problem must be zero.
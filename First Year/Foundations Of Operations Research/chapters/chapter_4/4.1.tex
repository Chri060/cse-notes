\section{Introduction}

\begin{definition}
    A \emph{linear programming problem} is an optimization problem defined as follows:
    \begin{align*}
        \min                      \:&\:f(x)           \\
        \textnormal{such that }     &\: x \in x \subseteq \mathbb{R}^n \rightarrow \mathbb{R} 
    \end{align*}
    Where: 
    \begin{itemize}
        \item The objective function, denoted as $f:x \rightarrow \mathbb{R}$, is linear. 
        \item The feasible region $x$ is described as:
            \[x=\{x\in\mathbb{R}^n|g_i(x)r_i0\land i\in(1,m)\}\]
            Where $r_i$ takes values from $\{=,\geq,\leq\}$, and $g_i:\mathbb{R}^n\rightarrow\mathbb{R}$ represents linear functions for $i=1,\dots,m$.
    \end{itemize}

    An \emph{optimal solution}, denoted as $x^{*} \in \mathbb{R}^n$, of the linear program is defined as satisfying $f(x^{*}) \leq f(x)$ for all $x\in x$
\end{definition}
The general form of a linear programming problem can be expressed as follows:
\[\min{z}=c^Tx\]
Such that $Ax \geq b$ and $x \geq 0$. 
In this notation, $x$ represents the vector of decision variables.
It's important to note that the inequality for the matrix $A$ can be transformed into an equality, and the constraints on the variable values can be relaxed in some instances.
It's often helpful to convert all linear programming problems into the general form.
This transformation can be achieved using the following rules:
\begin{itemize}
    \item To convert a maximization problem into a minimization problem, you can use: 
        \[\max{(c^Tx)}=\min{(-c^Tx)}\]
    \item For $a^Tx \leq b$, you can introduce a slack variable $s$ as: 
        \[a^Tx \leq b \implies
        \begin{cases}
            a^Tx+s=b \\
            s \geq 0
        \end{cases}
        \]
    \item For $a^Tx \geq b$, you can introduce a surplus variable $s$ as: 
        \[a^Tx \geq b \implies 
        \begin{cases}
            a^Tx-s=b \\
            s \geq 0
        \end{cases}
        \]
    \item If a variable $x_{ij}$ is unrestricted in sign, you can represent it as:
        \[\begin{cases}
            x_j=x_j^{+}-x_j^{-} \\
            x_j=x_j^{+},x_j^{-} \geq 0
        \end{cases}
        \]
        After substituting $x_j$ with $x_j^{+}-x_j^{-}$, you can remove $x_j$ from the problem. 
    \item To change the direction of an inequality, you can use the following equivalences:
        \[a^Tx \leq b \textnormal{ is equivalent to } -a^Tx \geq -b\]
        \[a^Tx \geq b \textnormal{ is equivalent to } -a^Tx \leq -b\]
    \item For an equality constraint $a^Tx = b$, you can represent it as:
        \[
        \begin{cases}
            a^Tx \geq b \\
            a^Tx \leq b
        \end{cases}
        \]
\end{itemize}
\begin{example}
    Given the following linear programming model, we'll rewrite it in the normal form:
    \begin{align*}
        \max{f(x_1,x_2)}           =&\: 2x_1-3x_2          \\
        \textnormal{such that }     &\: 4x_1-7x_2 \leq 5  \\
                                    &\: 6x_1-2x_2 \geq 4  \\
                                    &\: x_1 \geq 0, x_2 \in \mathbb{R}
    \end{align*}
    We can add two new variables, $x_2=x_3-x_4$, with $x_3,x_4 \geq 0$, and we obtain:
    \begin{align*}
        \max{f(x_1,x_2)}           =&\: 2x_1-3x_2          \\
        \textnormal{such that }     &\: 4x_1-7x_3+7x_4 \leq 5  \\
                                    &\: 6x_1-2x_3+2x_4 \geq 4  \\
                                    &\: x_1,x_2,x_4 \geq 0
    \end{align*}
    Next, we introduce slack and surplus variables $x_5$ and $x_6$, resulting in:
    \begin{align*}
        \max{f(x_1,x_2)}           =&\: 2x_1-3x_2          \\
        \textnormal{such that }     &\: 4x_1-7x_3+7x_4+x_5 = 5  \\
                                    &\: 6x_1-2x_3+2x_4-x_6 = 4  \\
                                    &\: x_1,x_2,x_4,x_5,x_6 \geq 0
    \end{align*}
    Finally, we need to change the sign of the objective function:
    \begin{align*}
        \min{f(x_1,x_2)}           =&\: -2x_1+3x_2          \\
        \textnormal{such that }     &\: 4x_1-7x_3+7x_4+x_5 = 5  \\
                                    &\: 6x_1-2x_3+2x_4-x_6 = 4  \\
                                    &\: x_1,x_2,x_4,x_5,x_6 \geq 0
    \end{align*}
\end{example}
The assumptions underlying linear programming models can be summarized as follows:
\begin{enumerate}
    \item Linearity of the objective function and constraints. 
    \item Proportionality of the objective function and constraints: the contribution of each variable is scaled by a constant. 
    \item Additivity of the objective function and constraints: the overall contribution of all variables is the sum of their individual contributions.
    \item Divisibility: variables are allowed to take rational values, and fractional values are acceptable.
    \item Parameters are considered as constant which can be estimated with a sufficient degree of accuracy. 
\end{enumerate}
The linear programming sensitivity analysis is a technique used to assess how an optimal solution reacts to small alterations in parameter values.
\section{Quick documentation}

\paragraph*{Model}
The model declaration is done with the following syntax: 
\begin{lstlisting}[style=Python]
model = mip.Model(name = "", sense = mip.MINIMIZE, solver_name = mip.CBC)
\end{lstlisting}
Here, \texttt{name} is the model name, \texttt{sense} is the model sense, and \texttt{solver\_name} is the solver name. 
The possible senses are: 
\begin{itemize}
  \item \textit{Minimize}: \texttt{mip.MINIMIZE} or \texttt{min}.
  \item \textit{Maximize}: \texttt{mip.MAXIMIZE} or \texttt{max}.
  \item \textit{Knapsack}: \texttt{mip.KNAPSACK} or \texttt{knap}.
\end{itemize}

\paragraph*{Variables}
The variable declaration is done with the following syntax: 
\begin{lstlisting}[style=Python]
model.add_var(name = "", lb = 0.0, ub = inf, obj = 0.0, var_type = mip.CONTINUOUS)
\end{lstlisting}
Here, \texttt{name} is the variable name, \texttt{lb} is the lower bound, \texttt{ub} is the upper bound, \texttt{obj} is the objective coefficient, and \texttt{var\_type} is the variable type.
The possible types (also for constraints) are: 
\begin{itemize}
    \item \textit{Binary variable}: \texttt{mip.BINARY} or \texttt{B}.
    \item \textit{Integer variable}: \texttt{mip.INTEGER} or \texttt{I}.
    \item \textit{Continuous variable}: \texttt{mip.CONTINUOUS} or \texttt{C}.
\end{itemize}

\paragraph*{Constraints}
The constraint declaration is done with the following syntax: 
\begin{lstlisting}[style=Python]
model.add_constr(lin_expr, name = "", priority = None)
\end{lstlisting}
Here, \texttt{lin\_expr} is the linear expression, \texttt{name} is the constraint name, and \texttt{priority} is the constraint priority.
The possible types (also for constraints) are: 
\begin{itemize}
    \item \textit{Binary constraint}: \texttt{mip.BINARY} or \texttt{B}.
    \item \textit{Integer constraint}: \texttt{mip.INTEGER} or \texttt{I}.
    \item \textit{Continuous constraint}: \texttt{mip.CONTINUOUS} or \texttt{C}.
\end{itemize}

\paragraph*{Linear expressions}
The summation of linear expression is done with the following syntax: 
\begin{lstlisting}[style=Python]
mip.xsum(terms)
\end{lstlisting}
Here, \texttt{terms} is the list of terms. 
This function is used to create a linear expression from a summation of terms.
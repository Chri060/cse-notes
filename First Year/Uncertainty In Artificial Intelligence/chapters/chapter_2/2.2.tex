\section{Fuzzy membership function}

A crisp set is defined by a boolean membership function on some property of the considered elements. 
In contrast, a fuzzy set is a set whose membership function ranges between zero and one.
\begin{definition}[\textit{Fuzzy membership function}]
    A membership function defines a set by specifying the degree of membership of an element from the universe of discourse to the set. 
\end{definition}
A label is assigned to the set to provide a reference. 
Fuzzy sets can also be defined with a variable having discrete values.
\begin{figure}[H]
    \centering
    \includegraphics[width=0.3\linewidth]{images/function.png}
    \caption{Example of a membership function}
\end{figure}

\paragraph*{Membership function definition}
To define a membership function, we need to consider the following steps based on the purpose of the model and the available data:
\begin{enumerate}
    \item Select a variable on which the membership function will be defined.
    \item Define the range of the variable.
    \item Identify the fuzzy sets needed for the application and define the labels.
    \item Identify characteristic points for the membership function for each fuzzy set.
    \item Define the shape of the membership function.
    \item Verify the correctness of the membership function.
\end{enumerate}
The shapes of the membership function can be chosen arbitrarily. 
The choice of shape affects the smoothness of the transition between two labels (e.g., a horizontal shape results in an immediate transition within intervals).
\begin{figure}[H]
    \centering
    \includegraphics[width=0.75\linewidth]{images/shape.png}
    \caption{Possible shapes for a membership function}
\end{figure}
\begin{definition}[\textit{Frame of cognition}]
    A set of fuzzy sets that fully covers the universe of discourse is called a frame of cognition. 
\end{definition}
A frame of cognition has the following properties:
\begin{itemize}
    \item \textit{Coverage}: each element of the universe of discourse is assigned to at least one granule with membership greater than or equal to zero.
    \item \textit{Uni-modality of fuzzy sets}: there is a unique set of values for each granule with maximum membership. 
\end{itemize}
\begin{definition}[\textit{Fuzzy partition}]
    A frame of cognition for which the sum of the membership values of each value of the base variable is equal to one is called a fuzzy partition.
\end{definition}
\begin{definition}[\textit{Function's $\alpha$-cut}]
    The $\alpha$-cut of a fuzzy set is the crisp set of values of $x$ such that $\mu(x) \geq \alpha$:
    \[\alpha_\mu(x)=\{x \mid \mu(x) \geq \alpha\}\]
\end{definition}
\begin{figure}[H]
    \centering
    \includegraphics[width=0.3\linewidth]{images/alpha.png}
    \caption{Alpha-cut of a membership function}
\end{figure}
\begin{definition}[\textit{Function's support}]
    The support of a fuzzy set is the crisp set of values $x$ such that $\mu_f(x)>0$. 
\end{definition}
\begin{figure}[H]
    \centering
    \includegraphics[width=0.3\linewidth]{images/support.png}
    \caption{Support of a membership function}
\end{figure}
\begin{definition}[\textit{Function's height}]
    The height $h_f$ of a fuzzy set $f$ on the universe $X$ is the highest membership degree of an element of $X$ in the fuzzy set:
    \[h_f(X)=\max_{x \in X}\mu_f(x)\]
    A fuzzy set is considered normal if, and only if, $h_f(X)=1$.
\end{definition}
\begin{figure}[H]
    \centering
    \includegraphics[width=0.3\linewidth]{images/height.png}
    \caption{Height of a membership function}
\end{figure}
\begin{definition}[\textit{Convex set}]
    A fuzzy set is convex if and only if 
    \[\mu[\lambda x_1+(1-\lambda)x_2] \geq \min [\mu(x_1),\mu(x_2)]\]
    for any $(x_1,x_2) \in \mathbb{R}$ and any $\lambda \in [0,1]$.
\end{definition}
\begin{figure}[H]
    \centering
    \includegraphics[width=0.5\linewidth]{images/convex.png}
    \caption{Graphical difference between a convex and a not convex set}
\end{figure}
The available operations on fuzzy sets encompass:
\begin{itemize}
    \item \textit{Complement}: $\mu_{\bar{f}}(x)=1-\mu_f(x)$.
    \item \textit{Union}: $\mu_{f_1 \cup f_2}(x)=\max [\mu_{f_1}(x),\mu_{f_2}(x)]$.
    \item \textit{Intersection}: $\mu_{f_1 \cap f_2}(x)=\min [\mu_{f_1}(x),\mu_{f_2}(x)]$.
\end{itemize}
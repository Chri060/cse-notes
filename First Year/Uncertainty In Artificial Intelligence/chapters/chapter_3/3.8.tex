\section{Fuzzy system design}

The structured approach to develop a fuzzy system for solving specific problems consist of: 
\begin{enumerate}
    \item Problem definition.
    \item Parametrization of the model: concepts.
    \item Mapping definition: rules.
    \item Implementation.
    \item Testing.
\end{enumerate}
In the problem definition phase of designing a fuzzy system, it's essential to choose the input and output variables and clearly define the goal of the model. 
Input variables are typically numerical or ordinal variables, making it possible to define fuzzy sets on them. 
These variables can be categorized as follows:

\begin{itemize}
    \item Perceived values: these variables come directly from sensors, collected data, or user inputs.
    \item Computed variables: these are derived from perceived variables through calculations or processing.
\end{itemize}
The choice of input variables is a design decision, and there aren't inherently best or worst input variables to select. 
Output variables, on the other hand, depend on the specific needs of the modeler and are the result of the fuzzy model's computations. 
The goals of the fuzzy model should be defined in advance and should guide the design process.

\paragraph*{System parametrization}
During the system parametrization phase, several key decisions need to be made:
\begin{itemize}
    \item \textit{Selection of membership functions}: this involves choosing appropriate membership functions for all variables. 
        Membership functions can be defined by a single expert based on objective evaluation or interviews, by multiple experts for increased reliability, or even by automatic systems working on data (e.g., using Neural Networks). 
        The number of membership functions for each variable typically ranges from three to seven. 
        It's important to ensure that every point within the range of input variables is covered by at least one fuzzy set that participates in at least one rule. 
        The boundaries should be covered with the maximum value to avoid any gaps in coverage.
    \item \textit{Selection of inferential mechanism}: the inferential engine depends on the operators selected for different parts of the rule-based system:
        \begin{itemize}
            \item AND: the choice between using the minimum (where the worst degree of matching is the most relevant) or the product (where all degrees of matching are relevant).
            \item OR: this involves combining the degree of truth with the rule weight. 
                You can choose between using the maximum or the probabilistic sum.
            \item Aggregation of degrees of the same consequent: you can choose between using the maximum (where the best degree is the most relevant) or the probabilistic sum (where all knowledge is considered).
                You must choose the same implementation used for the OR operator. 
        \end{itemize}
    \item \textit{Selection of fuzzification and defuzzification}: this step involves deciding whether to apply fuzzification (converting crisp input values into fuzzy values) and defuzzification (converting fuzzy output values back into crisp values) in your system.
\end{itemize}
These decisions are crucial to the performance and behavior of your fuzzy system and should align with the goals and requirements defined in the problem definition phase.
The design and definition of fuzzy rules can be carried out using various approaches, and the testing phase helps ensure the effectiveness and reliability of the system. 

\paragraph*{Rule definition}
Here are some methods for rule definition: 
\begin{itemize}
    \item \textit{From experience}: rules can be formulated based on the domain knowledge and expertise of human operators or experts who understand the system. 
        These rules are often based on their experience and intuition.
    \item \textit{From another model}: in some cases, existing models, such as mathematical models or traditional control systems, can be used as a basis for defining fuzzy rules.
        These models can be adapted into fuzzy rule sets.
    \item \textit{Machine Learning}: Machine learning techniques, including supervised learning methods, can be used to derive fuzzy rules from data. 
        Algorithms like decision trees, genetic algorithms, or neural networks can automatically generate fuzzy rule sets from training data.
    \item \textit{Self-tuning techniques}: some systems employ self-tuning mechanisms, like Neural Networks or reinforcement learning algorithms, to adapt and modify fuzzy rules based on ongoing system feedback. 
        These methods can optimize rule sets over time.
\end{itemize}

\paragraph*{Testing}
Testing can be conducted using:
\begin{itemize}
    \item \textit{Dynamic simulation}: dynamic simulation involves running the fuzzy system in a simulation environment to assess its performance under various conditions and scenarios. 
        This allows for a thorough evaluation of how the system responds to different inputs and situations.
    \item \textit{Static simulation}: static simulation involves testing the fuzzy system with predefined inputs and observing its outputs. 
        This can help evaluate the system's consistency and ensure that it adheres to expected behavior without the need for a real-time dynamic simulation.
    \item \textit{Direct testing on the process}: in some cases, fuzzy systems can be tested directly on the real process, typically under safe or controlled conditions. 
        This testing approach validates the system's performance in the actual operational environment, which is valuable for systems that control physical processes.
\end{itemize}
The choice of rule definition and testing methods depends on the specific application, available data, and the complexity of the system. 
Additionally, it's essential to validate and refine the fuzzy rule set to achieve the desired system behavior and performance.
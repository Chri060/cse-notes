\section{Uncertainty classification}

Uncertainty can be categorized into two primary types:
\begin{itemize}
    \item Epistemic uncertainty: this arises from factors that, in theory, could be known but are not known in practice. 
        This can result from inaccurate measurements, overlooked model effects, or intentional data concealment. 
        Epistemic uncertainty, also termed systematic uncertainty, is, in principle, reducible by enhancing the model.   
    \item Aleatoric uncertainty: this pertains to unknown unknowns that vary each time the same experiment is conducted. 
        Aleatoric uncertainty, also referred to as statistical uncertainty, can only be described using statistical information. 
        It may also be influenced by data acquisition and processing methods. 
        In general, it is present when the model lacks comprehensive coverage.
\end{itemize}
The sources of uncertainty can be classified into several categories:
\begin{itemize}
    \item Parameter uncertainty: arising from model parameters with exact values unknown to experimentalists and beyond experimental control, or whose values cannot be inferred through statistical methods.
    \item Parametric variability: stemming from the variability in input variables of the model. 
    \item Structural uncertainty: also referred to as model inadequacy, model bias, or model discrepancy, this uncertainty arises from a lack of knowledge about the problem.
    \item Algorithmic uncertainty: alternatively known as numerical uncertainty or discrete uncertainty, this type results from numerical errors and approximations in the implementation of the computer model.
    \item Experimental uncertainty: commonly known as observation error, this uncertainty arises from variations in experimental measurements.
    \item Interpolation uncertainty: originating from a lack of variable data collected from computer model simulations and/or experimental measurements.
\end{itemize}
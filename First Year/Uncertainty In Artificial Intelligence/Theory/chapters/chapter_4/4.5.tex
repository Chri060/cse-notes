\section{Fuzziness measure}

The fuzziness measures provide the degree of fuzziness of a fuzzy set. A fuzziness measure is often quantified as the entropy of a fuzzy set.
\begin{definition}
    Given a fuzzy set $A=\{x,\mu_A(x)\}$, the fuzziness measure (entropy) is defined as:
    \[d(A)=K \sum_{i=1}^{n}S(\mu_A(x_i))\]
    where $S(x)$ is the Shannon's function: 
    \[S(x)=-x \ln(x)-(1-x)\ln(1-x)\]
\end{definition}
\begin{example}
    Let's define set A as the set of integers close to ten. We have the following data:
    \begin{center}
        \begin{tabular}{|c|c|c|c|c|c|c|c|} 
        \hline
            $x$ & 7 & 8 & 9 & 10 & 11 & 12 & 13 \\ \hline
            $\mu_{\textnormal{A}(x)}$ & 0.1 & 0.5 & 0.8 & 1 & 0.8 & 0.5 & 0.1 \\ \hline
        \end{tabular}
    \end{center}
    The entropy of set A is calculated as:
    \[d(A)=0.325+0.693+0.673+0.501+0+0.501+0.693+0.325=3.711\]
    Now, let's define set B as the set of integers quite close to ten. We have the following data:
    \begin{center}
        \begin{tabular}{|c|c|c|c|c|c|c|c|c|c|} 
            \hline
            $x$         & 6     & 7     & 8     & 9     & 10    & 11    & 12    & 13    & 14 \\ \hline
            $\mu_A(x)$  & 0.1   & 0.3   & 0.4   & 0.7   & 1     & 0.8   & 0.5   & 0.3   & 0.1 \\ \hline
        \end{tabular}
    \end{center}
    The entropy of set B is calculated as:
    \[d(A)=4.35\]
    It's worth noting that set B exhibits more fuzziness compared to set A, as indicated by the fact that $d(B)>d(A)$.
\end{example}
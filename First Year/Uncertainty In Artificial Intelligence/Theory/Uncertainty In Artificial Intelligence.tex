\documentclass[12pt, a4paper]{report}
\usepackage{graphicx, array, amsthm, amssymb, amsmath, algorithm, algpseudocode, float, xcolor, thmtools, thmbox}
\usepackage[english]{babel}

\makeatletter
\renewcommand\thmbox@headstyle[2]{\bfseries #1}
\makeatother
\newtheorem[style=M,bodystyle=\normalfont]{theorem}{Theorem}
\newtheorem[style=M,bodystyle=\normalfont]{corollary}{Corollary}
\newtheorem[style=M,bodystyle=\normalfont]{lemma}{Lemma}
\newtheorem[style=M,bodystyle=\normalfont]{definition}{Definition}


\title{Uncertainty In Artificial Intelligence \\ \textit{Theory}}
\author{Christian Rossi}
\date{Academic Year 2023-2024}

\begin{document}

\maketitle

\newpage

\begin{abstract}
    The topics of the course are:
    \begin{itemize}
        \item Uncertainty sources that affect models: typology, issues, and modeling approaches.
        \item Measure-based uncertainty modeling.
        \item Logic-based uncertainty modeling.
        \item Fuzzy models: fuzzy sets, fuzzy logic, fuzzy rules, motivations for fuzzy modeling, tools for fuzzy systems, design 
            of fuzzy systems, applications.
        \item Bayesian networks: basics, design, learning, evaluation, applications.
        \item Hidden Markov Models: basics, design, learning, evaluation, applications.
        \item Applications: motivations, choices, models, case studies.
    \end{itemize}
\end{abstract}

\newpage

\tableofcontents

\newpage

\chapter{Introduction}
    \section{Definition}
    \begin{definition}[Uncertainty]
        \emph{Uncertainty} refers to epistemic situations involving imperfect or unknown information. It applies to predictions 
        of future events, to physical measurements that are already made, or to the unknown. 
    \end{definition}
    Uncertainty arises in partially observable or stochastic environments, as well as due to ignorance, indolence, or both. It arises 
    in any number of fields, including insurance, philosophy, physics, statistics, economics, finance, medicine, psychology, sociology, 
    engineering, metrology, meteorology, ecology and information science.
    
    The lack of certainty, a state of limited knowledge where it is impossible to exactly describe the existing state, a future outcome,
    or more than one possible outcome. This puts in evidence that uncertainty is related to the need of describing a piece of reality.

    \section{Modelling}
    Modelling is at the base of our life: the way we interact with the world is through models that interpret data coming from sensors
    and generate knowledge and actions. Modelling is also the way we may represent entities in a computer and possibly making it reasoning
    on them.
    \begin{definition}[Model]
        A \emph{model} is a representation of some entity, defined for a specific purpose. A model captures only the aspects of the entity
        modelled that are relevant for the purpose. A model is necessarily different from the modelled entity. So, intrinsic to modelling
        are all sort of uncertainties.
    \end{definition}
    All Artificial Intelligence applications are based on models, either defined by somebody or learned. These models are represented In
    different ways, but share uncertainty issue mainly on inputs. 

    \section{Uncertainty classification}
    The uncertainty can be of two main types: 
    \begin{itemize}
        \item Epistemic uncertainty: it is due to things one could in principle know but does not in practice. This may because a 
            measurement is not accurate, because the model neglects certain effects, or because particular data have been deliberately
            hidden. It is also known as systematic uncertainty and can in principle be reduced by enriching the model.      
        \item Aleatoric uncertainty: it is representative of unknown unknowns that differ each time we run the same experiment. 
            Aleatoric uncertainty is also known as statistical uncertainty, since only statistical information can describe it. This may
            also depend on the way we get and elaborate data. In general, it is present when the model is missing some aspects.
    \end{itemize}
    The sources of uncertainty can be: 
    \begin{itemize}
        \item Parameter: it comes from the model parameters, whose exact values are unknown to experimentalists and cannot be controlled
            in experiments, or whose values cannot be inferred by statistical methods. 
        \item Parametric variability: it comes from the variability of input variables of the model. 
        \item Structural: also known as model inadequacy, model bias, or model discrepancy, this comes from the lack of knowledge of the
            problem.
        \item Algorithmic: also known as numerical uncertainty, or discrete uncertainty. This type comes from numerical errors and
            numerical approximations in the implementation of the computer model. 
        \item Experimental: also known as observation error, this comes from the variability of experimental measurements.
        \item Interpolation: this comes from a lack of variable data collected from computer model simulations and/or experimental 
            measurements. 
    \end{itemize}

    \section{Uncertainty modelling}
    The type of uncertainty model depends on the type of uncertainty, its sources and the information we have in uncertainty and mostly
    has to do with qualification and quantification of uncertainty. The possible models for uncertainty are: statistical, logical and
    cognitive.

    Artificial Intelligence and Machine Learning technologies are based on models that include uncertainty models of these sorts, essential
    not only for the implementation of effective models, but also to define learning models able to cope with complex situations, and 
    to evaluate the quality of learned/developed models. 
    There are two major types of problems in uncertainty quantification: 
    \begin{itemize}
        \item Forward propagation of uncertainty: the various sources of uncertainty are propagated through the model to predict the overall
            uncertainty in the system response:
            \begin{itemize}
                \item To evaluate low-order moments of the outputs (mean and variance).
                \item To evaluate the reliability of the outputs.
                \item To assess the complete probability distribution of the outputs. 
            \end{itemize}
            This is what is done also in Bayesian networks and graphical models. 
        \item Inverse assessment of model uncertainty and parameter uncertainty, where the model parameters are calibrated simultaneously
            using test data: given some experimental measurements of a system and some results from its mathematical model, inverse 
            uncertainty quantification estimates the discrepancy between the experiment and the mathematical model (bias correction) and
            estimates the values of unknown parameters in the model if there are any (parameter calibration).
    \end{itemize}
    
\end{document}
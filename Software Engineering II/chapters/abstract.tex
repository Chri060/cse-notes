\begin{abstract}
    This course covers essential aspects of the software process, starting with the software lifecycle, project management principles, and techniques for cost estimation. 
    It introduces notations and specification methods, particularly focusing on Alloy, a language used to specify and analyze software models.

    A significant portion of the course is dedicated to problem definition and requirements engineering, emphasizing the importance of clear requirements and exploring the Jackson-Zave approach, which defines requirements, specifications, and domains distinctly. 
    Students will learn to utilize modeling tools like Alloy and UML to effectively support requirement modeling activities.
    
    In addition to requirements, the course delves into methodologies and technologies for product development, exploring software architectures and various architectural styles, along with middleware and software components essential to modern software systems.
    
    Verification and validation form a core part of the course, covering both the process and techniques for ensuring software correctness. 
    This includes a range of analysis techniques and testing methods designed to validate and verify software systems effectively.
\end{abstract}
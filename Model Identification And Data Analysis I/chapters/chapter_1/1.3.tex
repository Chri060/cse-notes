\subsection{Classification}

\paragraph*{Static and dynamic}
A system can be categorized as follows:
\begin{itemize}
    \item \textit{Static system}: in this type of system, knowledge of the input variables alone is adequate to determine the output value.
    \item \textit{Dynamic system}: this refers to a system with memory, wherein the past behavior of the output impacts its current value.
\end{itemize}

\paragraph*{Discrete and continuous}
Systems can be further categorized based on their time description, which can be either discrete or continuous.
Natural and physical phenomena are inherently continuous and are often mathematically described using ordinary differential equations.
On the other hand, discrete systems are mathematically described using difference equations.

However, a computer can only handle a limited amount of data. 
This necessitates the sampling of signals at discrete intervals with a sampling time $T_s$. 
This ensures that only a finite amount of data is stored at discrete time points $t \cdot T_s$, where $t=1,\dots,N$: 
\[y(t)=y(t \cdot T_s)\]
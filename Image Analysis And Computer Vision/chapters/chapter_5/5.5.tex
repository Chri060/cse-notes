\section{Localization}

The localization of a known planar shape from a calibrated image consists of determining the relative position and orientation of the object with respect to the camera. 
This requires estimating the homography $\mathbf{H}$ that maps points from the plane $\boldsymbol{\pi}$ to their corresponding image points.

The plane $\boldsymbol{\pi}$ has a relative pose with respect to the camera, which is described by a rotation $\mathbf{R}_{\boldsymbol{\pi}}$ and a translation $\mathbf{0}_{\boldsymbol{\pi}}$. 
Assuming the world reference frame is aligned with the camera frame, the projection matrix is $\mathbf{P}=\begin{bmatrix}\mathbf{K} & \mathbf{0} \end{bmatrix}$. 
From this, the object's pose relative to the camera can be determined using the homography:
\[\begin{bmatrix} \mathbf{r}_1 & \mathbf{r}_2 & \mathbf{r}_3 \end{bmatrix}=\mathbf{K}^{-1}\mathbf{H}\]
Here $\mathbf{r}_1$ and $\mathbf{r}_2$ are the first two columns of the rotation matrix, and the third column is given by:
\[\mathbf{r}_3=\mathbf{r}_1\times \mathbf{r}_2\]
If the coordinates of the points $\mathbf{x}_{\boldsymbol{\pi}\mathbf{j}}$ on the plane are known only up to scale, the translation vector $\mathbf{o}_{\boldsymbol{\pi}}$ can also only be determined up to scale. 
This means that its direction is known, but its magnitude remains uncertain.

Knowing the shape of the object allows estimation of its image orientation and the viewing direction.
Knowing both the shape and size of the object enables full determination of its image position, orientation, and viewing direction.
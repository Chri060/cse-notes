\section{Calibrated camera rectification}

Consider a calibrated camera capturing a single image of a planar object. 
If two vanishing points can be identified, the line at infinity $\mathbf{l}_{\infty}^{\prime}$ passing through them can also be determined.
The plane-to-image homography $\mathbf{H}$ describes the mapping of points from the plane to the image. 
The inverse transformation maps image points back onto the plane:
\[\mathbf{H}_R=\mathbf{H}^{-1}\]
Given that the homography is expressed as:
\[\mathbf{H}=\mathbf{K}\begin{bmatrix}\mathbf{r}_{\boldsymbol{\pi}\mathbf{1}} & \mathbf{r}_{\boldsymbol{\pi}\mathbf{2}} & \mathbf{o}_{\boldsymbol{\pi}}\end{bmatrix}\]
It follows that $\mathbf{H}$ depends on both the camera calibration matrix $\mathbf{K}$ and the relative pose of the plane $\boldsymbol{\pi}$ with respect to the camera.

Since the only constrained element is the normal direction $\mathbf{n}_{\boldsymbol{\pi}}$ of the plane, we are free to define a convenient reference frame on $\boldsymbol{\pi}$.
The normal is given by:
\[\mathbf{n}_{\boldsymbol{\pi}}=\mathbf{K}^T\mathbf{l}_{\infty}^{\prime}\]
To simplify rectification, we choose $\mathbf{r}_{\boldsymbol{\pi}\mathbf{1}}$ and $\mathbf{r}_{\boldsymbol{\pi}\mathbf{2}}$ as orthogonal vectors, both perpendicular to $\mathbf{n}_{\boldsymbol{\pi}}$ and normalize them.
Setting $\mathbf{o}_{\boldsymbol{\pi}}=\mathbf{n}_{\boldsymbol{\pi}}$, the rectifying homography becomes:
\[\mathbf{H}_R=\mathbf{R}_{\boldsymbol{\pi}}^T\mathbf{K}^{-1}\]

\subsection{Rectification with unknown planar scene}
In the case of reconstructing an unknown planar scene from two calibrated images, image-to-image homographies are utilized. 
The following constraints hold:
\[\begin{cases}\mathbf{I}^{\prime T}\boldsymbol{\omega}\mathbf{I}^{\prime}=\mathbf{0} \\ \mathbf{I}^{\prime T}\mathbf{H}^{\prime T}\boldsymbol{\omega}\mathbf{H}^{\prime}\mathbf{I}^{\prime}=\mathbf{0}\end{cases}\]
Here, $\boldsymbol{\omega}$ is known, and the unknowns are the complex coordinates $\mathbf{I}^{\prime}$. 
At least two images are required, as each equation provides two constraints (real and imaginary parts).
Geometrically, this corresponds to the intersection of two conics, resulting in two pairs of imaged circular points. 
The selection of the correct solution can be guided by reprojection error minimization or an additional third image.

\noindent The steps for the rectification are:
\begin{enumerate}
    \item Extract the image of circular points and compute the conic dual to the circular points:
        \[\mathbf{C}_{\infty}^{\ast\prime}=\mathbf{I}^{\prime}\mathbf{J}^{\prime T}+\mathbf{J}^{\prime}\mathbf{I}^{\prime T}\]
    \item Apply Singular Value Decomposition to $\mathbf{C}_{\infty}^{\ast\prime}$:
        \[\text{SVD}(\mathbf{C}_{\infty}^{\ast\prime})=\mathbf{U}\begin{bmatrix} a & 0 & 0 \\ 0 & b & 0 \\ 0 & 0 & 0 \end{bmatrix}\mathbf{U}^T\]
    \item Construct the rectification matrix from SVD output $\mathbf{U}$:
        \[\mathbf{H}_{SR}=\begin{bmatrix} \frac{1}{\sqrt{a}} & 0 & 0 \\ 0 & \frac{1}{\sqrt{b}} & 0 \\ 0 & 0 & 1 \end{bmatrix}\mathbf{U}^T\]
    \item Apply the rectification transformation to the image: 
        \[\mathbf{M}_S = \mathbf{H}_{SR} \cdot \text{image}\]
\end{enumerate}
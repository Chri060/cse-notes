\section{Quadrics}

\begin{definition}[\textit{Quadric}]
    A point $\mathbf{X}$ lies on a quadric $\mathbf{Q}$ if it satisfies a homogeneous quadratic equation, given by:
    \[\mathbf{X}^T\mathbf{Q} \mathbf{X} = \mathbf{0}\]
    Here, $\mathbf{Q}$ is a $4\times 4$ symmetric matrix.
\end{definition}

The matrix $\mathbf{Q}$ is homogeneous, meaning that for any scalar $\lambda$, $\lambda\mathbf{Q}$ represents the same quadric as $\mathbf{Q}$
This implies that the matrix $\mathbf{Q}$ has 9 degrees of freedom. 
In general, 9 points in a general position are sufficient to define a unique quadric surface.

\paragraph*{Intersection with a plane}
A quadric can intersect with a plane in various ways, depending on the geometric properties of both. One important concept in projective geometry is the absolute conic.
\begin{proposition}
    The absolute conic $\mathbf{\Omega}_{\infty}$ contains the circular points $\mathbf{I}_{\infty},\mathbf{J}_{\infty}$ of any plane $\boldsymbol{\pi}$. 
\end{proposition}
These circular points are a fundamental extension in the context of projective geometry, connecting the concept of a plane's geometry with the quadric surfaces.

\subsection{Degenrate quadrics}
When the matrix $\mathbf{Q}$ is degenrate, the quadric takes on special forms. Specifically, the rank of $\mathbf{Q}$ reveals the type of degeneration:
\begin{enumerate}
    \item If $\text{rank}(\mathbf{Q})=1$, the quadric corresponds to a repeated plane, and $\mathbf{Q}$ can be factored as $\mathbf{Q}=\mathbf{AA}^T$, where $\mathbf{A}$ is a vector defining the plane.
    \item If $\text{rank}(\mathbf{Q})=2$, the quadric is the intersection of two distinct planes, and $\mathbf{Q}$ can be written as $\mathbf{Q}=\mathbf{AB}^T+\mathbf{BA}^T$, where $\mathbf{A}$ and $\mathbf{B}$ define the two planes.
    \item If $\text{rank}(\mathbf{Q})=3$, the quadric is a cone. The vertex of the cone can be found as the right null space (RNS) of $\mathbf{Q}$. 
\end{enumerate}
A special case of a cone occurs when the vertex is at infinity, leading to the geometric object known as a cylinder.
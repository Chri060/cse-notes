\section{Machine Learning operations}

A feature store is a centralized repository designed to store, manage, and serve features for Machine Learning models.
Features are the input variables that describe the data objects used by models to make predictions. 
The main goal of a feature store is to simplify the process of creating, sharing, and using these features across various Machine Learning models. 
This ensures that features are easily accessible, consistent, and reusable, making it easier to develop, maintain, and deploy Machine Learning systems. 
Feature stores are a key component of Machine Learning operations.

Machine Learning operations refers to a set of practices and tools that streamline the development, deployment, and maintenance of Machine Learning models in production. 
It extends DevOps principles to the Machine Learning world, focusing on the unique challenges of building, deploying, and monitoring Machine Learning systems at scale. 
Machine Learning operations brings together data scientists, Machine Learning engineers, and operations teams, enabling them to collaborate effectively to ensure reliable and efficient deployment of Machine Learning models.
The Machine Learning operations lifecycle includes the following: 
\begin{enumerate}
    \item \textit{Development}: data scientists experiment with models, algorithms, and feature sets. 
        Automation ensures proper versioning of code, models, and datasets.
    \item \textit{Training and validation}: the model is trained and validated using test datasets. 
        Machine Learning operations automates this process, ensuring reproducibility.
    \item \textit{Deployment}: once validated, the model is deployed into production using automated practices.
    \item \textit{Monitoring}: post-deployment, the model's performance is continuously monitored for accuracy, data drift, and concept drift.
    \item \textit{Retraining}: when performance drops or new data is available, models are retrained automatically to stay current.
\end{enumerate}
\noindent Machine Learning operations is essential for transitioning Machine Learning from research to scalable production systems. 
By automating workflows, ensuring consistency, and enhancing collaboration, Machine Learning operations helps organizations efficiently manage and operationalize Machine Learning models at scale.

\subsection{Feature stores}
A feature store is a centralized repository used to manage, store, and serve features for Machine Learning models. Features are the input variables that help Machine Learning models make predictions. 
The feature store's main goal is to streamline the process of creating, sharing, and using these features across various models, ensuring they are consistent, reusable, and easily accessible.
This is particularly important for maintaining high-quality data and enabling collaboration between data scientists, Machine Learning engineers, and data engineers.

A feature store offers consistency, ensuring that features used for training and inference are aligned, which helps improve model accuracy. 
It also boosts efficiency by enabling feature reuse across models and teams, reducing redundant work and accelerating model development.

\begin{definition}[\textit{Feature engineering}]
    Feature engineering is the process of transforming raw data into meaningful features that improve the performance of Machine Learning models. 
\end{definition}
\noindent This can range from simple transformations, like aggregations, to more complex methods, such as Machine Learning-generated features like word embeddings. 
Feature engineering's ultimate goal is to create a better dataset that enhances the performance of Machine Learning algorithms.

The role of a feature store is to act as a centralized repository where curated features are stored. 
It serves as a data management layer, enabling collaboration between teams, and provides an interface that converts raw data into features used in model training and inference. 
By ensuring consistency across features used in both training and deployment phases, a feature store eliminates discrepancies and reduces the risk of errors. 
It also prevents duplication of code, making the process more efficient and maintaining alignment between model development and deployment.

Feature stores are essential when deploying models at scale and in production, as they provide consistency and enable teams to collaborate and reuse features. 
However, for small teams or proof-of-concept projects, the overhead of using a feature store might not be necessary, as simpler workflows can often suffice.

\subsection{Feast}
Feast is an open-source feature store that provides isolation for feature stores at the infrastructure level using resource namespacing. 
This means that different projects or teams can work with their own feature stores without interfering with each other, allowing for more organized and secure management of features.

For offline use cases that only rely on batch data, Feast does not need to ingest data itself.
Instead, it can query existing data directly, making it easy to integrate with other data sources. 
For online use cases, Feast supports a process called materialization, where features from batch sources are ingested and made available for real-time use. 
It also supports pushing streaming features to make them available both offline and online.

In Feast, an entity is a group of semantically related features, typically defined to map to the specific domain of a use case. 
Each feature view is a collection of features, and the entity key is the collection of entities for that view. 
It is important to reuse entities across feature views to ensure consistency and avoid duplication.

A data source in Feast refers to the raw data that users own, such as a table in BigQuery. 
Feast does not manage the raw data itself but rather loads this data and performs various operations to retrieve and serve features. 
Feast uses a time-series data model to interpret data in the underlying data sources. This model is used to build training datasets or materialize features into an online store for real-time use.

Feast utilizes a registry to store all the applied Feast objects, such as feature views and entities.
The registry offers methods to manage these objects, including applying, listing, retrieving, and deleting them. 
This centralized registry allows for better organization and management of features across different projects.
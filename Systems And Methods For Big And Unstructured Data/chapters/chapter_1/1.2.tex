\section{Relational database}

Relational databases are structured systems designed to organize, store, and manage data efficiently.
The design of a database typically occurs at three distinct levels:
\begin{itemize}
    \item \textit{Conceptual database design}: information model that is independent of physical implementation details, providing a high-level representation of the enterprise's data requirements.
    \item \textit{Logical database design}: organizational database based on a specific data model ensuring logical structure and consistency.
    \item \textit{Physical database design}: implementation of the database using specific data storage structures and access methods.
\end{itemize}

\subsection{Characteristics}
\paragraph*{Entity}
In an entity-relationship database model, an entity represents a distinct, real-world object that can be uniquely identified. 
Entities are characterized by attributes, which describe their properties.
Entities that share the same attributes are grouped into an entity set, and each set is distinguished by a primary key. 
The attributes of an entity belong to defined domains, which specify the range of valid values for each attribute.

\paragraph*{Relationship}
Relationships between entities form another core aspect of relational databases. 
A relationship is an association between two or more entities, and all relationships of the same type are grouped into relationship sets. 
These relationships can include descriptive attributes, which provide additional context, and are defined by the participating entities. 
The concept of cardinality further specifies the number of connections that can exist between entities in a relationship, such as one-to-one or one-to-many associations.

\paragraph*{Hierarchies and aggregations}
To enhance the flexibility of the model, relational databases often incorporate hierarchies and aggregations. 
ISA hierarchies allow for subclassing, where specific subclasses of an entity set can include unique attributes that further describe the entities within them. 
Aggregation offers a way to model more complex scenarios where a relationship itself needs to participate in another relationship. 
By treating a relationship set as though it were an entity set, it becomes possible to incorporate that relationship into higher-level associations.

\subsection{Design}
A critical aspect of database design is determining whether a concept should be modeled as an entity or a relationship.
This decision significantly impacts the structure and clarity of the model.

The entity-relationship model is a powerful tool for capturing the semantics of data, offering a rich representation of relationships and constraints.
However, not all constraints can be expressed within entity-relationship diagrams. Certain advanced constraints may need to be implemented in other layers.

\subsection{Definitions}
Relational databases revolutionized data management by formalizing the concept of structured data organization. 
SQL, based on this relational model, became commercially available in Database Management Systems in 1981.

The relational model relies on the mathematical notion of a relation, which is naturally represented as a table.
Each relation is defined by a Cartesian product, which forms the foundation of the relational structure.
\begin{definition}[\textit{Cartesian product}]
    Given $n$ sets $D_1, D_2, \dots, D_n$, the Cartesian product, denoted as $D_1 \times D_2 \times \dots \times D_n$, is the set of all ordered $n$-tuples $(d_1, d_2, \dots, d_n)$ such that $d_1 \in D_1, d_2 \in D_2, \dots, d_n \in D_n$. 
\end{definition}
\begin{definition}[\textit{Mathematical relation}]
    A mathematical relation on $D_1, D_2, \dots, D_n$ is a subset of the Cartesian product $D_1 \times D_2 \times \dots \times D_n$. 
\end{definition}
\begin{definition}[\textit{Relation}]
    A relation is a set of ordered $n$-tuples $(d_1, d_2, \dots, d_n)$, where $d_1 \in D_1, d_2 \in D_2, \dots, d_n \in D_n$. 
\end{definition}
\begin{definition}[\textit{Relation domain}]
    The sets $D_1, D_2, \dots, D_n$ are referred to as the domains of the relation.
\end{definition}
\begin{definition}[\textit{Relation degree}]
    The number $n$ is called the degree of the relation.
\end{definition}
\begin{definition}[\textit{Relation cardinality}]
    The number of $n$-tuples in a relation is termed its cardinality.
\end{definition}

\subsubsection{Schema and instances}
\begin{definition}[\textit{Relation schema}]
    A relation schema consists of a relation's name $R$ and a set of attributes $A_1, \dots, A_n$, and it is denoted as $R(A_1, \dots, A_n)$. 
\end{definition}
\begin{definition}[\textit{Database schema}]
    A database schema is a collection of relation schemas with unique names, denoted as $R = \{R_1(X_1), \dots, R_n(X_n)\}$. 
\end{definition}
\begin{definition}[\textit{Relation's instance}]
    A relation instance on a schema $R(X)$ is a set $r$ of tuples on $X$.
\end{definition}
\begin{definition}[\textit{Database's instance}]
    A database instance for a schema $R = \{R_1(X_1), \dots\}$ is a set of relations $r = \{r_1, \dots\}$.
\end{definition}
\noindent The relational model enforces a strict structure on data to ensure consistency and logical organization. 
In this model, all information is represented as tuples, which must conform to predefined relation schemas. 
This rigid framework provides clarity but also imposes certain limitations on how data is modeled and stored.

\paragraph*{Null values}
Null values are used to represent missing or inapplicable information in relational databases. 
These nulls can be categorized into three distinct types. 
Despite these distinctions, most database management systems do not differentiate between these types of nulls. 
Instead, they implicitly treat all nulls as"no-information values.

\paragraph*{Integrity constraint}
An integrity constraint is a condition that must hold true for all valid database instances. 
These constraints act as predicates, ensuring that a database instance is legal only if it satisfies all specified conditions.

\subsubsection{Keys}
A key is a fundamental concept in the relational model, representing a set of attributes that uniquely identifies tuples in a relation.
\begin{definition}[\textit{Superkey}]
    A set of attributes $K$ is a superkey for a relation $r$ if no two distinct tuples $t_1$ and $t_2$ in $r$ share the same values for $K$.
\end{definition}
\begin{definition}[\textit{Key}]
    The set of attributes $K$ is a key for $r$ if it is a minimal superkey, meaning no proper subset of $K$ is itself a superkey.
\end{definition}
\noindent The keys can be classified as: 
\begin{itemize}
    \item \textit{Primary key}: to maintain data integrity, null values are not permitted in primary keys. 
        Each relation must have a designated primary key, which uniquely identifies its tuples. 
        In database schema notation, primary key attributes are underlined for clarity.
        Primary keys also play a critical role in establishing references between relations. 
        Through these references, data consistency is maintained across the database.
    \item \textit{Foreign key}: enable relationships between tuples in different relations. 
        A foreign key in one relation references the primary key in another, establishing a link between the two.
        To ensure consistency, referential integrity constraints are imposed. 
        These constraints guarantee that every foreign key value corresponds to an existing primary key value in the referenced relation. 
        Violating these constraints would result in an inconsistent database state.
\end{itemize}

\subsection{Schema transformation}
Transforming an entity-relationship diagram into a relational database schema involves a systematic approach to map entities and relationships into relational tables. 
The process consists of the following key steps:
\begin{enumerate}
    \item \textit{Mapping entities to tables}: each entity in the ER diagram is represented as a separate table in the relational schema. 
        The attributes of the entity become the columns of the table, while each instance of the entity set corresponds to a row in the table.
    \item \textit{Handle relationships}: relationships between entities are incorporated into the schema based on their cardinality and complexity. 
\end{enumerate}
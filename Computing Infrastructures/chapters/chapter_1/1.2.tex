\section{Edge computing systems}

Edge computing is a distributed computing model where data processing occurs as close as possible to the data generation source, improving response times and saving on bandwidth. 
Processing data near its point of origin offers significant advantages, including reduced processing latency, decreased data traffic, and enhanced resilience during data connection interruptions.

Edge computing systems can be categorized into the following types:
\begin{itemize}
    \item \textit{Cloud}: provides virtualized computing, storage, and network resources with highly elastic capacity.
    \item \textit{Edge servers}: utilizes on-premises hardware resources for more computationally intensive data processing.
    \item \textit{IoT and AI-Enabled edge sensors}: facilitates data acquisition and partial processing at the network's edge.
\end{itemize}

Edge computing offers several benefits, including high computational capacity, distributed computing capabilities, enhanced privacy and security, and reduced latency in decision-making. 
However, it also has drawbacks, such as the need for a power connection and reliance on cloud connectivity.

\subsection{Embedded PCs}
An embedded system refers to a computer system comprising a processor, memory, and input and output peripheral devices, all serving a specific function within a larger mechanical or electronic system. 
Advantages of embedded PCs include their ubiquity in computing, high performance, availability of development boards, ease of programming similar to personal computers, and support from a large community. 
However, they also have disadvantages, such as relatively high power consumption and the necessity for some hardware design work.

\subsection{Internet of Things}
The Internet of Things (IoT) includes devices equipped with sensors, processing capabilities, software, and other technologies designed to connect and exchange data with other devices and systems over the internet or other communication networks. 
Advantages of IoT devices include their widespread presence, wireless connectivity, battery-powered operation, low costs, and ability to sense and actuate. 
However, these devices also have several disadvantages, such as limited computing power, energy constraints, limited memory, and programming challenges.
\section{Dependability principles}

Dependability is crucial both during the design phase and throughout runtime operations.
During the design phase, it is essential to:
\begin{itemize}
    \item Analyze the system under development.
    \item Evaluate and measure its dependability properties.
    \item Make necessary modifications to enhance dependability.
\end{itemize}
During runtime, the focus shifts to:
\begin{itemize}
    \item Detecting any malfunctions or failures.
    \item Investigating and understanding the root causes of these issues.
    \item Implementing appropriate measures to mitigate the impact of these malfunctions.
\end{itemize}
Failures are common in both development and operational stages. 
While development-stage failures should be avoided, operational failures are inevitable due to the nature of system components and must be managed effectively. 
Additionally, the effects of these failures should be predictable and deterministic rather than catastrophic.

Historically, dependability was primarily a concern for safety-critical and mission-critical applications, such as space exploration, nuclear facilities, and avionics, due to the high costs associated with ensuring dependability. 
This high cost was justified only in situations where it was absolutely necessary.

In non-critical systems, operational failures can lead to economic losses and reputational damage, as seen in consumer products. 
However, in mission-critical systems, operational failures can have serious or irreversible consequences for the mission. 
In safety-critical systems, failures pose a direct threat to human life if they occur during operation.
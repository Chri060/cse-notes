\section{Containers}

Containers are lightweight virtualization solutions, particularly significant in DevOps contexts.
They encapsulate pre-configured packages containing all necessary components for code execution.
Their primary advantage lies in predictable, repeatable, and immutable behavior.
When duplicating a master container onto another server, its execution remains consistent and error-free across environments.

Contrasting containers with VMs, the former offer virtualization at the OS level, sharing the host system kernel with other containers. 
In contrast, VMs provide hardware virtualization, with applications dependent on guest OSs.
Containers possess several key characteristics:
\begin{itemize}
    \item \textit{Flexibility}: they can containerize even complex applications.
    \item \textit{Lightweight}: containers leverage and share the host kernel, minimizing resource usage.
    \item \textit{Interchangeable}: updates can be seamlessly distributed without disruption.
    \item \textit{Portable}: they can be created locally, deployed in the cloud, and run anywhere.
    \item \textit{Scalable}: containers enable automatic replication and distribution of replicas.
    \item \textit{Stackable}: they can be vertically stacked and deployed on the fly.
\end{itemize}
Containers streamline application deployment, enhance scalability, and promote modular application development, where modules remain independent and uncoupled.

\subsection{Docker}
Docker simplifies software deployment by utilizing containers. 
It is an open-source platform that facilitates the building, shipping, and running of applications across diverse environments.
Docker operates based on DockerFiles, which are text files containing commands to assemble application images via the command line. 
When executed with the Docker build command, they create immutable Docker images—snapshots of the application. 
Containers created with Docker can be run on various platforms. 

\subsection{Kubernetes}
Kubernetes, an open-source project from Google, is well-suited for managing medium to large clusters and complex applications. 
It offers a comprehensive and customizable solution for efficiently coordinating large-scale node clusters in production. 
Kubernetes enables running containers across diverse machines within a cluster. 
It allows for scaling the performance of applications by adjusting the number of containers dynamically. 
In the event of machine failure, Kubernetes can automatically initiate new containers on alternative machines, ensuring continuous operation.
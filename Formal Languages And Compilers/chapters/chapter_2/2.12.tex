\section{Comparison of regular and context-free languages}

Regular languages constitute a specific category within free languages, distinguished by stringent constraints on rule formulation.
These constraints result in unavoidable repetitions within the sentences of regular languages. 
The transformation rules employed to convert a regular expression into a grammar generating the same regular language are as follows:
\begin{table}[H]
    \centering
    \begin{tabular}{cc}
    \hline
    \textbf{Regular expression}         & \textbf{Corresponding grammar}                \\ \hline
    $r=r_1r_2\dots r_k$                 & $E=E_1E_2 \dots E_k$                          \\ 
    $E=r_1\cup r_2\cup \dots\cup r_k$   & $E=E_1 \cup E_2 \cup \dots \cup E_k$          \\
    $r=(r_1)^{\ast}$                       & $E=EE_1\mid \varepsilon$ or $E=E_1E\mid \varepsilon$  \\
    $r=(r_1)^{+}$                       & $E=EE_1\mid E_1$ or $E=E_1E\mid E_1$                  \\
    $r = b \in \Sigma$                  & $E=b$                                         \\
    $r=\varepsilon$                     & $E=\varepsilon$                               \\ \hline
    \end{tabular}
\end{table}
In general, regular expressions form a subset of context-free languages:
\[\textnormal{REG} \subset \textnormal{CF}\] 
\begin{definition}[\textit{Unilinear grammar}]
    A grammar is unilinear if and only if its rules are either all right-linear or all left-linear.
\end{definition}
We can impose certain constraints on a unilinear grammar:
\begin{itemize}
    \item Strictly unilinear rules: each rule contains at most one terminal, expressed as $A \rightarrow aB$ with $A \in (\Sigma \cup \varepsilon)$ and $B \in (V \cup \varepsilon)$.
    \item All terminal rules are empty. 
\end{itemize}
Thus, we can assume rules of the form $A \rightarrow aB\mid \varepsilon$ for the right case and $A \rightarrow Ba\mid \varepsilon$ for the left case. 

\paragraph*{Strictly unilinear grammars}
It is demonstrable that regular expressions can be translated into strictly unilinear grammars, establishing the regular language set as a subset of unilinear grammars:
\[\textnormal{REG} \subseteq \textnormal{UNILIN}\]
Conversely, from any unilinear grammar, an equivalent regular expression can be obtained:
\[\textnormal{UNILIN} \subseteq \textnormal{REG}\]
Consequently, it holds that
\[\textnormal{UNILIN}=\textnormal{REG}\]
This property allows viewing the rules of the unilinear right grammar as equations, where the unknowns represent the languages generated by each nonterminal.
Let $G$ be a strictly unilinear right grammar with all terminal rules empty.
A string $x \in \Sigma^{\ast}$ is in $L_A$ in the following cases:
\begin{enumerate}
    \item $x$ is the empty string: governed by a rule $P: A \rightarrow \varepsilon$.
    \item $x=ay$: controlled by a rule $P: A \rightarrow aB$ and $y \in L_B$.
\end{enumerate}
For every nonterminal $A_0$ defined by $A_0 \rightarrow a_1A_1\mid a_2A_2\mid \dots\mid a_kA_k\mid \varepsilon$ we have $L_A=a_1L_{a_1} \cup a_2L_{a_2} \cup \dots \cup a_kL_{a_k} \cup \varepsilon$. 
This yields a system of $n=\left\lvert V \right\rvert$ equations in $n$ unknowns, solvable through the method of substitution and the application of the Arden identity.
\begin{definition}[\textit{Arden identity}]
    An equation $KX \cup L$, with $K$ being a nonempty language and $L$ any language, possesses a unique solution: 
    \[X=K^{\ast}L=KK^{\ast}L \cup L\]
\end{definition}
\begin{example}
    Consider the grammar: 
    \[\begin{cases}
        S \rightarrow sS \mid  eA \\
        A \rightarrow sS \mid  \varepsilon
    \end{cases}\]
    Transforming this grammar into a system of equations, we get:
    \[\begin{cases}
        L_S \rightarrow sL_S \cup eL_A \\
        L_A \rightarrow sL_S \cup \varepsilon
    \end{cases}\]
    Substituting the second equation into the first one and applying the concatenation operation, we obtain:
    \[\begin{cases}
        L_S \rightarrow (s \cup es)L_S \cup e \\
        A \rightarrow sL_S \cup \varepsilon
    \end{cases}\]
    Applying the Arden identity, we arrive at:
    \[\begin{cases}
        L_S \rightarrow (s \cup es)^{\ast}e \\
        A \rightarrow s(s \cup es)^{\ast}e \cup \varepsilon
    \end{cases}\]
\end{example}
It is noteworthy that regular languages exhibit inevitable repetitions.
\begin{property}
    Let $G$ be a unilinear grammar. 
    Every sufficiently long sentence $x$ (i.e., longer than a grammar-dependent constant $k$) can be factorized as $x=tuv$ (with $u$ non-empty), such that, for all $i \geq 1$, the string $tu^nv \in L(G)$. 
\end{property}
In simpler terms, the sentence can be pumped by injecting string $u$ an arbitrary number of times.
\begin{proof}
    Consider a strictly right-linear grammar $G$ with $k$ nonterminal symbols. 
    In the derivation of a sentence $x$ whose length is $k$ or more, there is necessarily a nonterminal $A$ that appears at least two times.
    Consequently, it is possible to derive $tv$, $tuv$, $tuuv$, and so on.
\end{proof}
This property aids in determining whether a grammar generates a regular language or not.
A grammar generates a regular language only if it lacks self-nested derivations.
Notably, the inverse is not necessarily true: a regular language may be generated by a grammar with self-nested derivations. 
The absence of self-nested derivations facilitates solving language equations of unilinear grammars.

In the context-free languages, all sufficiently long sentences necessarily contain two substrings that can be repeated arbitrarily many times, giving rise to self-nested structures.
This impedes the derivation of strings with three or more parts that are repeated the same number of times.
Consequently, the language of three or more powers is not context-free. 
Therefore, the language of copies is also not context-free.

\subsection{Closure properties}
Regular languages exhibit closure under operations such as reverse, star, complement, union, and intersection.
Conversely, context-free languages demonstrate closure under reverse, star, and union operators. 
It can be established that the intersection of a context-free language with a regular language remains within the realm of context-free languages.
Introducing a filter through a regular language can enhance the selectivity of a grammar, and notably, the outcome of this filtration consistently falls within the category of context-free languages.
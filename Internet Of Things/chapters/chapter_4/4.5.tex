\section{RFID}

A primer on radio frequency identification
To Enhance the concept of “bar-codes” for faster identification of assets
(goods, people, animals)
Ingredients:
• Transition to electronic bar codes with wireless communication
capabilities
• Transition form optical to wireless “readers”

\subsection{History}
WW2 –
First passive RFID system:“roll plane” by Lufwaffe
First active RFID system: “Friend or Foe” systems by RAF
Theremin Bug (1945)
H. Stockman Paper (1948) – Communication by means of reflected power
60s/70s: commercial deployment starts, Electronic Article Surveillance
(EAS) in retail, LF identification of animals
First RFID Patent - 1973
80s: mass deployment starts, first read/write active systems with
processors
90s: first passive systems with EEPROM, automatic toll systems
Auto-ID center founded at MIT – 1999: Standardization effort taken over
by EPC Global (Electronic Product Code)
2000: conductive inks, low-power processors
2004: Walmart use case

\subsection{Building blocks}
Reader 
Memory
Processing Unit
Control Logic/Collision Arbitration
Mechanism
Battery or Power Supply
Other Interfaces (Ethernet, WiFi)

Tag
Battery Scavenging circuitry
(active or passive)
(E2PROM to store ID)
(Control Logic/Collision
Arbitration Mechanism)
(sensors)
(processing unit)
(RF antennas)

Types of Tags by power supply
Passive
Operational power scavenged from
reader radiated power Short range (<1m), Low cost
Self sustaining
Semi-passive
Operational power provided by battery Medium range (tens of meters),
need battery, avarage cost,
long life
ùActive
Operational power provided by battery -
transmitter built into tag high range (hundreds of
meteres), need battery, Limited
lifetime

Types of Tags by power storage
1-bit tags: used in EAS systems
• realized with magnetic material
• when tag gets close to reader, current variation is
perceived at the reader
• multple detection impossible
Tags with storage space
• Read only
• Read/write
RX/TX/modulation/demodulation circuitry
Unique Identifier

\subsection{Implementation}
Tags can be attached to
anything:
• pallets or cases of product
• Vehicles
• company assets or personnel
• People or animals
• Electronic appliances
The Tags
10
Implementation challenges
Effective Energy Scavenging
Miniaturization/customization
Cost
Reader must deliver enough power from RF field to power the tag
Reader must discriminate backscatter modulation in presence of carrier at
same frequency
High magnitude difference between transmitted and received signals
Integration with enterprise solutions
RFID Backend Middleware solutions to:
• manage high data volume produced by readers
• filter the data produced by the readers (remove redundancy, eliminate
unwanted data, etc.)
• Store the data in a way that is meaningful for the specific application
• Let different readers be interoperable

\subsection{Performance measures}
Reading range
«how far we can read»
Throughput
«how fast we can read»
Robustness
«how robust the reading process»
Performance depends on:
• Carrier frequency
• Emitted power
• Environment (propagation conditions)
• Concurrency (# of tags to be read simultaneously)
Standard Different standards available in the flields of:
• animal identification
• cards and personal identification
• containers ID
Standards dedicated to item management application developed by
ISO/IEC
GS1 – EPCGlobal initiative, UHF C1 Gen 2 (900MHz) ref standards (ISO
18000-6)


RFID: Physical Communication  Near Field Model (HF) Far Field Model (UHF)
RFID HF – Inductive coupling
Inductive Coupling between two circuits (reader and tag)
Frequency Range 125kHz o 13,56 MHz
Reading range comparable to coil diameter
Functioning modes:
Duplex (concurrent charging and transmission)
sequential (charging and transmission decoupled)

Tag modulates its resistive load with the info to be sent
Voltage changes (modulated by the tag information) are triggered at the
reade

Tag activation voltage (reading range) depends on
• Size of coil antennas (the larger the better)
• Tag/reader orientation
• Tag/reader chip hardware
In LF/HF magnetic field is scarcely affected by dielectric materials
RFID HF – Qualitative performance overview
26
Limited reading range
Very sensitive to orientation
Almost immune to the environment
Moderate cost

RFID UHF/SHF – Electromagnetic coupling
Far field operation model
Energy/power is scavanged by the tag through EM waves emitted by
reader
Transmission happens via backscattering by modulating the impedence
Tens of meters of read range
Bipolar antennas (few centimeters)
High reading range
High bit rate
High impact of the environment


Tag Arbitration Peculiarities
Multiple Answers: Arbitration Required Similar to classical Access Control but:
• Fixed unknown population size
• Tags cannot implement complex protocols (E.g., carrier sense is out)
• Often reader-driven algorithms
Collision Arbitration Mechanisms: A Classification
Vertical Classification
ALOHA-like access mechanism
• Slotted ALOHA
• Dynamic Frame ALOHA
Tree-based access mechanisms
• Binary Tree
Horizontal Classification
• Centralized/Distributed
• Type of Channel Feedback (S,C,0)
The efficiency is commonly defined as the tag population size, N, over the
length of the arbitration period L(N)
\[\eta=\dfrac{N}{L_N}\]

The Frame ALOHA Extension of the ALOHA protocol where nodes are allowed to transmit
once every frame
Frame composed of r slots
Every tag chooses a slot in the frame
If transmission is failed, retry at next frame

Frame ALOHA: Single Frame
The average throughput is:
\[E[S]=n\left(1-\dfrac{1}{r}\right)^{n-1}\]
Thus, the efficiency is:
\[\eta=\dfrac{E[S]}{r}=\dfrac{n}{r}\left(1-\dfrac{1}{r}\right)^{n-1}\]
Which is maximum for: r=n
Frame Aloha: Multiple frames
The FA efficiency depends on the initial tag population (N), the current
backlog (n) and the frame size (r).
Current Frame size r is dynamically set to the current backlog n ->
Dynamic Frame Alohaefficiency: 
\[\eta=\dfrac{N}{L_N}\]
The average tag resolution process can be recursively calculated as:
\[L_n=r+\sum_{i=0}^{n-1}\Pr(S=i)L_{n-i}\]
which leads to: 
\[L_n=\dfrac{r+\sum_{i=0}^{n-1}\Pr(S=i)L_{n-i}}{1-\Pr(S=0)}\]
Problem
Initial population N and backlogs n are not known
Tag arbitration is actually composed of two modules:
Backlog Estimation Module: to provide and estimate of the backlog nest
Collision Resolution: run Frame Aloha with r= nest
Schoute’Estimate
Assume that any procedure is able to keep the frame size r equal to the
current backlog n
Under this assumption, the number of terminals transmitting in a slot is
approximated by a Poisson process with intensity 1 [terminal/slot]
The average number of terminals in a collided slot can be consequently
calculated as: 
\[H=\dfrac{1-e^{-1}}{1-2e^{-1}}=2.39\]
The backlog is estimated as:
nest=round(Hc), being c the number of collided slots

\paragraph*{Binary tree}
Random Numbers are used to partition the set of colliding tags
Tags have counters set to 1
The reader broadcasts
Trigger command: sent at the beginning and after successful/empty slots
tags decrease their counter and transmit if counter is 0
Split commands: sent after collided slots
tags with counter equal to 0 randomly choose a new counter value in [0,1]
Tags with counter greater than 0 increase their counter
Optimizations
More refined feedbacks can be used to steer the splitting
Leverage tag population estimates to steer splitting
In some slots collisions are certain, use Split command other than Trigger
one
\section{Radio Frequency Identification}

Radio Frequency Identification (RFID) is an advanced evolution of traditional bar codes, designed to enable faster and more efficient identification of assets.
Unlike optical bar codes, RFID transitions to electronic bar codes with wireless communication capabilities and replaces optical readers with wireless ones. 
This technology allows for contactless identification, making it highly versatile and applicable across various industries.

\subsection{History}
The origins of RFID can be traced back to World War II, where the first passive RFID system was developed by the Luftwaffe.
Around the same time, the RAF introduced the first active RFID system, the "Friend or Foe" identification system.
In 1945, the Theremin Bug emerged as an early example of RFID-like technology, followed by H. Stockman's seminal paper in 1948, which introduced the concept of communication through reflected power.

The commercial deployment of RFID began in the 1960s and 1970s, with applications such as Electronic Article Surveillance (EAS) in retail and low-frequency (LF) animal identification. 
The first RFID patent was granted in 1973. By the 1980s, mass deployment of RFID systems began, including the introduction of active read/write systems with embedded processors. 
The 1990s saw the development of passive systems with EEPROM memory and automatic toll systems. 
The Auto-ID Center at MIT was founded in 1999, leading to standardization efforts by EPC Global (Electronic Product Code). 
Advances in conductive inks and low-power processors in the 2000s further propelled RFID adoption, with Walmart emerging as a major use case by 2004.

\subsection{Architecture}
An RFID system consists of two main components:
\begin{itemize}
    \item \textit{Reader}: the reader is responsible for transmitting radio waves to the tag and receiving data from it. 
        It includes memory, a processing unit, control logic for collision arbitration, and interfaces such as Ethernet or Wi-Fi. 
        It may also have its own power supply.
    \item \textit{Tag}: the tag is attached to the object being identified and contains a battery scavenging circuitry (in active or semi-passive tags), an EEPROM to store the unique identifier, control logic, and RF antennas. 
        Some advanced tags may include sensors or additional processing units.
        The tags can be: 
        \begin{itemize}
            \item \textit{Passive tags}: these tags derive their operational power from the reader's radiated energy. 
                They are low-cost but have a short range and are self-sustaining.
            \item \textit{Semi-passive tags}: these tags use a battery to power their internal circuits but rely on the reader for communication. 
                They offer a medium range, average cost, and long life.
            \item \textit{Active tags}: these tags have their own power source and built-in transmitters, enabling high-range communication. 
                However, they have a limited lifetime due to battery constraints.
        \end{itemize}
        The tags may have 1 bit storage or more, based on the usage. 
\end{itemize}

\subsection{Physical communication}
RFID systems operate using either near-field or far-field models:
\begin{itemize}
    \item \textit{Near field} (HF): based on inductive coupling between the reader and tag, typically operating at 125 kHz or 13.56 MHz. 
        The reading range is comparable to the coil diameter, and functioning modes can be duplex (concurrent charging and transmission) or sequential (charging and transmission decoupled). 
        HF systems are relatively immune to environmental interference but have limited range and are sensitive to orientation.
    \item \textit{Far field} (UHF/SHF): based on electromagnetic coupling, where the tag scavenges energy from EM waves emitted by the reader.
        Transmission occurs via back-scattering. 
        UHF systems offer tens of meters of read range, high bit rates, and are more affected by environmental conditions.
\end{itemize}

\subsection{Tag arbitration}
When multiple tags respond simultaneously, collision arbitration mechanisms are required. These mechanisms can be classified into vertical and horizontal categories:
\begin{itemize}
    \item \textit{Vertical classification}: these includes ALOHA-like mechanisms (includes Slotted ALOHA and Dynamic Frame ALOHA, where tags transmit in predefined slots and retransmit if collisions occur) and tree-based mechanisms (includes Binary Tree algorithms, where colliding tags are partitioned recursively).
    \item \textit{Horizontal classification}: differentiates between centralized and distributed approaches and types of channel feedback.
\end{itemize}
\noindent The efficiency of arbitration is defined as the ratio of the tag population size to the length of the arbitration period:
\[\eta=\dfrac{N}{L_N}\]

\subsubsection{Frame ALOHA}
Frame ALOHA is an extension of the ALOHA protocol, where tags are allowed to transmit once per frame. 
Each frame consists of $r$ slots, and every tag selects a slot randomly for transmission. 
If a transmission fails due to a collision, the tag retries in the next frame.

The average throughput $\mathbb{E}[S]$ in a single frame is given by:
\[\mathbb{E}[S]=n\left(1-\dfrac{1}{r}\right)^{n-1}\]
\noindent Here, $n$ is the number of tags and r is the number of slots in the frame. 
The efficiency $\eta$ is then calculated as:
\[\eta=\dfrac{\mathbb{E}[S]}{r}=\dfrac{n}{r}\left(1-\dfrac{1}{r}\right)^{n-1}\]
\noindent The efficiency is maximized when the number of slots equals the number of tags. 

For multiple frames, the efficiency depends on the initial tag population $N$, the current backlog $n$, and the frame size $r$. 
In Dynamic Frame ALOHA, the frame size $r$ is dynamically adjusted to match the current backlog $n$. 
The efficiency is expressed as:
\[\eta=\dfrac{N}{L_N}\]
\noindent The average tag resolution process $L_n$ can be recursively calculated as:
\[L_n=r+\sum_{i=0}^{n-1}\Pr(S=i)L_{n-i}\]

A key challenge is that the initial population $N$ and the backlog $n$ are typically unknown. 
To address this, tag arbitration involves two modules:
\begin{itemize}
    \item \textit{Backlog Estimation Module}: estimates the current backlog $n$.
    \item \textit{Collision Resolution Module}: runs Frame ALOHA with the estimated backlog $n$.
\end{itemize}
\noindent Schoute's method assumes that the frame size $r$ is kept equal to the current backlog $n$. 
Under this assumption, the number of terminals transmitting in a slot follows a Poisson process with intensity 1 terminal per slot. 
The average number of terminals involved in a collided slot can be approximated as:
\[H=\dfrac{1-e^{-1}}{1-2e^{-1}}=2.39\]
\noindent The backlog $n$ is then estimated as:
\[n_{\text{est}}=\left\lfloor H\cdot c\right\rceil \] 
\noindent Here, $c$ is the number of collided slots.

\subsubsection{Binary tree}
The Binary Tree algorithm uses random numbers to partition colliding tags into smaller groups. 
Tags maintain counters initialized to 1, and the reader broadcasts commands to manage the process:
\begin{itemize}
    \item \textit{Trigger command}: sent at the start or after successful/empty slots. 
        Tags decrement their counters and transmit if their counter reaches 0.
    \item \textit{Split command}: sent after a collision. 
        Tags with a counter of 0 randomly choose a new counter value from [0, 1]. 
        Tags with a counter greater than 0 increment their counter.
\end{itemize}
\section{Bluetooth}

1. Bluetooth is an industrial specification for
WPANs
2. The WG 802.15.1 adapted the industrial
specifications of Bluetooth for the levels 1
and 2
3. ’96-’97: Ericsson internal project
4. ’98: Bluetooth SIG created (Ericsson, IBM,
Intel, Toshiba, Nokia)
5. ’99: new members join the SIG (3Com,
Lucent Technologies, Microsoft, Motorola)


BluetoothTM
o Radio technology
o Low cost
o Small range (10-20 m)
o Low complexity
o Small size
o ISM 2.4 GHz band
o Created by an industrial
consortium
o Only the first two levels
have then been
standardized by IEEE
802.15.1

\subsection{Physical layer}
o ISM band at 2.4 GHz
o 79 (23 in France and Japan) channels
spaced of 1 MHz (2402-2480 MHz)
o Modulation G-FSK (1 Mb/s)
o Device classes
Frequency Hopping (FH)
o 1600 hops/s (625 µs per hop)
o The FH sequence is pseudo random
and determined by the clock and the
address of the ‘master’ device that
regulates the access to the channel
o The other devices are ‘slaves’ and
follows the sequence fk defined by the
master
It is possible to transmit packet
with duration of 1, 3 or 5 intervals

\subsection{Piconet}
The simplest network architecture defined in
Bluetooth is called piconet
o The piconet is an ad hoc network composed
of 2 or more devices
o A device acts as master and the other as
slaves
o Communication can take place only between
master and slave and not directly between
slaves
o Up to 7 slaves can be active in a piconet
o The others can be in
n Stand-by (not part of the piconet)
n Parked (part of the piconet but not active, up to a
maximum of 256 devices)
Addresses
n MAC address of 48 bits
n AMA (Active Member Address) 3 bits
n PMA (Parked Member Address) 8 bits

\subsection{Connections}
Bluetooth considers two types of connections
o SCO (Synchronous Connection Oriented)
n Fixed rate bi-directional connection (circuit)
n FEC for improving quality
n Rate of 64 Kbit/s
o ACL (Asynchronous ConnectionLess)
n Packet switched connection shared between
master and active slaves based on a polling access
scheme
n Several options for packet formats and physical
layer codes (1, 3, 5 slots)
n Rate up to 433.9 Kbit/s symmetric (using 5-slot
packets in both directions) and 723.2/57.6 Kbit/s
asymmetric (using 5-slot packets in one direction and 1-slot packets in the other)

\subsection{Architecture}
o Original protocol
architecture not
compliant with IEEE
802 structure
o Later adapted by
IEEE for 802.15.1
specifications
o RF + Baseband
equivalent to PHY +
MAC
o Control plane for the
network and
RF connections creation

\subsection{Packet format}
BT packet includes three parts:
\begin{figure}[H]
    \centering
    \includegraphics[width=0.5\linewidth]{images/blue.png}
    \caption{Packet format}
\end{figure}
n The access code used for the synchronization
and the piconet identification
n The header used for the Link Control (LC) which
includes also the retransmission scheme
n The payload whose format depends on the type
of connection and the type of packet (number of
slots, PHY protection, etc.)
Access code:
n There are three types of access code
n Channel Access Code (CAC): It defines the piconet,
the synchronization word is derived from the MAC
address of the master
n Device Access Code (DAC): it is used to address a
specific device during the page procedure, it also
derives from the device MAC address
n Inquiry Access Code (IAC): it is used to discover all
the device in range during the inquiry procedure
Header:
n Active Member Address (AMA)
n Type of packet: there are 16 types of packets
which differs based on length, type of physical
layer protection, and connection
n Flow: flow control
n ARQ: retransmission
n SQN: sequence number
n HEC: checksum
states
o Stand-by: the devices is not active and the
radio is off
o Connection: the device is connected with
other devices. This state includes other
sub-states
o Inquiry: the device is looking for other devices in range
o Inquiry Scan: the device is listening for inquiry requests during small time intervals
(low duty cycle).
Page: the device is trying to create a piconet
connecting to another specific device with
known address
o Page Scan: the device is listening to the
channel for page requests for small internal of
time
\begin{figure}[H]
    \centering
    \includegraphics[width=0.5\linewidth]{images/blue1.png}
    \caption{Link controller states}
\end{figure}

\subsection{Page procedure}
o If a device wants to connect to another
device of which it knows the address, it runs
the page procedure
o From the address, it calculate the Device
Access Code (DAC)
o A stand-by device enters periodically into a page scan mode e starts listening for its DAC
on the channels
o Due to ISM band usage rule, the page
procedure cannot be executed on a fixed
channel
o The device in page scan mode follows a
pseudo random sequence on 32 channels
(frequencies)
In order to limit the energy consumption, the
page scan is executed for 10 ms on a channel
and then the device enters sleep mode for a
period of the order of few seconds (from
1.28 to 3.85 s)
o At each new scan period, a new channel is
selected according to the pseudo random
sequence
o The device in page can calculate the sequence
but it usually does not know the phase (clock)
o Therefore, it transmits the DAC sequentially on
all channels
In 10 ms the device in page can transmit on 16 out of 32 channels
o The transmission on the 16 channels is repeated
until a reply is received
o If after a sleep period no reply is received, the
other 16 channels are used for the sequential
transmissions
The reply is actually the transmission of the DAC
itself
o In most of the cases the connection is established
within 2 sleep periods
o A third packet is sent by the device in page
o It is the FHS packet which includes all the
information on the device, including the clock
o The connection is now active
o The device that was paging takes the role of
master and the device that was scanning takes
the role of Slave
o An AMA is assigned to the Slave

\subsection{Inquiry}
The inquiry procedure is used to discover
other devices in range
o It is very similar to the page procedure, but
the access code used is a universal one
named Inquiry Access Code (IAC)
o Also the inquiry scan sequence is pseudo
random
o The reply to a inquiry request is a FHS packet
o There may be a collision in the reply
o If after an inquiry, a device goes to page
mode it can quickly calculate the page scan
sequence and its phase for all the other
devices. Therefore the connection delay is
usually very short

\subsection{Bluetooth low power}
In the connection state, a slave device can
enter into low power modes
o Hold: in this state the slave stops listening to the channel for a period of time negotiated
with the master (the AMA is kept)
o Sniff: in this state the slave listens to the
channel at regular internals (the AMA is
kept)
o Park: in this state the AMA is released and a
PMA is assigned. The slave device listens to
the channel regularly (very low duty cycle)
and it comes active again after receiving an
unpark message from the master



Protocols: Link Management
o The link management protocol is in
charge of connection setup messages,
security and connection control
o Creation of ACL and SCO connections
o Management of security procedures
o Adding and removal of slaves from a
piconet
o LMP messages have priority over all
the others
Protocols: L2CAP
o Logical Link Control and Adaptation Protocol
(L2CAP)
o Adaptation functions (segmentation and
reassembly) and multiplexing

\subsection{Profiles}
Profiles
o Basic operation modes characteristic of different
applications
o Used to guaranteed interoperability at application
layer






\subsection{Bluetooth new versions}
Bluetooth v2.0
o v2.0 in 2004, v2.1 in 2007
o Adaptive Frequency Hopping (AFH) –
v1.2
o extended Synchronous Connections
(eSCO)
o Multicast/Broadcast
o Enhanced Data Rate (EDR) – rates up
to 3 Mb/s using Differential encoded
Phase Shift Keying (DPSK) with 4 and
8 symbols (same band)

Bluetooth v3.0
o v3.0 in 2009
o Rate up to 24 Mb/s
o … but using a different MAC/PHY,
which is actually WiFi
o BT is basically used only for the
negotiation of the connection
parameters among the two devices

Bluetooth v4.0
o v4.0 in 2010
o New revised version of
n Classic BT (Basic Rate, BR)
n BT high speed (EDR)
n And new BT low energy (BLE)
o It’s a new stack that incorporates the work
of the WiBree working group; New
commercial name is Bluetooth Smart
o Low rate (260 kb/s), short range, low
power for sensor and small devices
o Potential competitor of ZigBee





\section{Bluetooth Low Energy}
 (BLE)
o Not replacing, but sitting alongside
Bluetooth BR/EDR
o Designed to be energy efficient
o Tailored to IoT applications

BLE stack
o “Simpler” than Bluetooth

BLE Physical Layer
o Same ISM band used in BR/EDR, but:
n 40 2MHz channels
o 3 channels for advertisement/broadcast
o 37 for data transfer via adaptive frequency
hopping
o Not compatible with BR/EDR!
n GFSK modulation
n Three PHY types:
o LE 1M (1Mbps), no FEC, mandatory
o LE 2M (2Mbps), no FEC, optional
o LE Coded (0.5/0.125Mbps), FEC, optional.
Used for extending range 2x or 4x

BLE Link Layer
o Governed by a state machine
o Duties:
n Manage packet types
n Addressing
o Public / Random
n Channel management
o Adaptive Frequency hopping
n Logical link management
o LE ACL
o Packet ordering/ACK through SN/NESN

BLE error control
o SN = sequence number
o NESN = next expected sequence
number
Retransmissions
Central/Peripheral nodes
o Similar concept to master/slave
o Peripheral nodes transmit
advertisments periodically (20ms –
10s)
o Centrals scan the adv channels for
advertisements
n scan window: for how long to scan
n scan interval: how often to scan
BLE AFH
o BLE uses Adaptive Frequency Hopping
on the 37 data channels
o Central node selects:
n Channels to use (avoiding the ones with
interference). Channel map is shared
with peripheral node.
n Hopping algorithm
o Algorithm #1: sequential selection with a
fixed hop interval passed during connection
o Algorithm #2: semi-random sequence
calculated from address/internal counters
Advertising types and
connections
o Directed/undirected: accepts
connections only from known
devices/all
o Connectable/not-Connectable: allows
connections or not
o Scannable/non-Scannable: allows to
receive Scan requests and reply
BLE Connections
o A central can issue a Connection
Request to a peripheral sending
connectable advertisements
o The peripheral replies with a
Connection Reply. Now the two BLE
devices are connected and can
exchange data
BLE Data exchange
o Two connected devices (client/server)
use the Attribute Protocol (ATT)
o The ATT protocol allows to:
n Discovery attributes (resources)
available on the server
n Read / Write operations on such
attributes

Bluetooth v5.0
o Main improvements over BLE
n 4x range (up to 200m) using lower
transmission rates (250 or 125 kbps)
n Max speed 2Mbps
n Mesh topology
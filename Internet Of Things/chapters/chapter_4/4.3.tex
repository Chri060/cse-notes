\section{IP-based networking}

A low-power and lowlatency wireless mesh
networking protocol built
using open and proven
standards.
\begin{figure}[H]
    \centering
    \includegraphics[width=0.5\linewidth]{images/lowpan.png}
    \caption{6LowPAN protocol stack}
\end{figure}

\subsection{Thread}
THREAD Devices – types and roles
Full Thread Device
n Routing full thread DEVIces
o Router: Routers also provide joining and security services for devices trying to join
the network. Routers are not designed to sleep. Routers can downgrade their
functionality and become REEDs (Router-eligible End Devices).
o Leader elected role of one Router, which takes certain decisions in the Thread
network such as allowing REEDs to upgrade to Routers. If the Leader of a Thread
Network fails, another Router will be dynamically selected to resume the role. All
Routers have the required Thread Network Data to seamlessly assume this role.
n Non-routing full thread devices
o Routing-eligible End Device (REED)
o Full End Device (FED)
o Minimal Thread Device
n Non routing minimal end device
o Minimal End Device (MED)
o Sleepy End Device
o Synchronized Sleepy End Device
o Bluetooth End Device
o Boarder Router
\begin{figure}[H]
    \centering
    \includegraphics[width=0.5\linewidth]{images/lowpan1.png}
    \caption{Thread network structure}
\end{figure}
ACKs 
Slide/Figures Sources
n IPSO Alliance Webinar “6LowPAN for IP Smart
Objects”
n 6LoWPAN: The Wireless Embedded Internet,
Shelby & Bormann, ISBN: 978-0-470-74799-5,
(c) 2009 John Wiley & Sons Ltd

\subsection{6LowPAN}
What is 6LoWPAN?
An Adaptation Layer to fit IPv6 over
Low-Power wireless Area Networks
\begin{figure}[H]
    \centering
    \includegraphics[width=0.5\linewidth]{images/lowpan2.png}
    \caption{6LowPAN architecture}
\end{figure}
LoWPANs are stub networks
o Simple LoWPAN
n Single Edge Router
o Extended LoWPAN
n Multiple Edge Routers with common backbone link
o Ad-hoc LoWPAN
n No route outside the LoWPAN
o Internet Integration issues
n Maximum transmission unit
n Application protocols
n IPv4 interconnectivity
n Firewalls and NATs
n Security
Efficient header compression
n IPv6 base and extension headers, UDP header
o Fragmentation
n 1280 byte IPv6 MTU -> 127 byte 802.15.4
frames
Support for e.g. 64-bit and 16-bit 802.15.4
addressing
o Useful with low-power link layers such as IEEE
802.15.4, narrowband ISM and power-line
communications
o Network auto-configuration using neighbor
discovery
o Unicast, multicast and broadcast support
n Multicast is compressed and mapped to broadcast
o Support for IP routing (e.g. IETF RPL)
o Support for use of link-layer mesh (e.g. 802.15.5)
IPv6 (RFC 2460) = the next generation Internet Protocol
n Complete redesign of IP addressing
n Hierarchical 128-bit address with decoupled host identifier
n Stateless auto-configuration
n Simple routing and address management
o Majority of traffic not yet IPv6 but...
n Most PC operating systems already have IPv6
n Governments are starting to require IPv6
n Most routers already have IPv6 support
n So the IPv6 transition is coming
o 1400% annual growth in IPv6 traffic (2009)
The 6LoWPAN Format

6LoWPAN is an adaptation header format
n Enables the use of IPv6 over low-power wireless links
n IPv6 header compression
n UDP header compression
o Format initially defined in RFC4944
o Updated by draft-ietf-6lowpan-hc (work in progress)

\paragraph*{Header compression}
Stateless compression
o Flow-independent compression
o Simple tricks on IPv6/UDP
header
n Common values for header
fields => compact forms
n Version is always 6
n Traffic Class and Flow Label
are zero
n Payload Length always derived
from L2 header
n Source and Destination
Addresses can be elided (link- local) and/or compressed
depending on the “context” of
the transmission

\paragraph*{Addressing}
IPv6 addresses are compressed in 6LoWPAN
o Prefix
n Addresses within 6LoWPAN typically contain common prefix
n Nodes typically communicate with one or few central devices
n Establish state (contexts) for such prefixes – only state
maintenance
n Support for up to 16 contexts 6LoWPAN compresses IPv6
addresses by
o Interface ID
n Typically derived from L2 addr during autoconfiguration
n Elide when Interface Identifier can be derived from L2 header

\subsection{Ideal routing protocols for WSN}
o Shortest-path routes
o Avoids overlap
o Minimum energy consumption
o Needs global topology information
Algorithm classes
n Distance-vector
Links are associated with cost, used to find the shortest
route. Each router along the path store local next-hop
information about its route table.
n Link-state
Each node acquires complete information about the
network, typically by flooding. Each node calculated a
shortest-path tree calculated to each destination.
o Types of Signaling
n Proactive
Routing information acquired before it is needed.
n Reactive
Routing information discovered dynamically when needed.
o Route metrics are an important factor

\paragraph*{6LowPAN}
Here we consider IP routing (at
layer 3)
o Routing in a LoWPAN
n Single-interface routing
n Flat address space (exact-match)
n Stub network (no transit routing)
Routing Over Low power and Lossy networks (ROLL)
n Working group at the IETF
o Standardizing a routing algorithm for embedded
apps
o Application specific requirements
n Home automation
n Commercial building automation
n Industrial automation
n Urban environments
o Analyzed all existing protocols
o Solution must work over IPv6 and 6LoWPAN
o Protocol in-progress called RPL “Ripple”
n Proactive distance-vector approach
Requirements/Objectives
o Unicast/Anycast/Multicast
o Adaptive Routing
n Different metrics
o Constraint-Based Routing
n Parallel Paths
o Scalability
o Goal: RPL builds up a Destination-Oriented
Directed Acyclic Graph (DODAG)
o What DODAG depends on the specific
Objective Function (OF)
Metric: scalar to capture the link/path
performance (reliability, interference,
throughput, etc.)
o Constraint: criteria to eliminate links from a
DODAG
o Routing OF combines metric and constraints
n Es: “find the path with the maximum reliability
that does not traverse any non-encrypted link”
The Most Common Routing Metric
o Node
n Residual Energy (node)
n CPU, Storage, WorkLoad, Battery/Mains
o Link
n Throughput (local/global metric)
n Latency (local/global metric)
n Reliability (local/global metric)
o Expected Transmission Count (ETX)
o Link Quality Level (LQL)
o Hop Count
The ETX “The average number of packet
transmissions to successfully transmit a
packet”

\paragraph*{RPL Messages}

DODAG Information Object (DIO)
n Link Local Messages to Advertise/Build Up DODAG
n Contains all the information to exchange
metrics/set constraints
o Destination Advertisement Object (DAO)
n Used to propagate information on destination
(prefixes)
n Support to ptp and ptmp traffic
o DODAG Information Solicitation (DIS)
n Used to solicit DIOs
Multiple Routing OF calls for
Multiple DODAGs
The very same network may support different
applications
n Es: network with battery- and main-operated
devices, high and low bandwidth links, and two
applications (telemetry and alarming)
n One “time sensitive path” for alarms (low latency,
high reliability, no constraints on energy)
n One “non-time-sensitive path” for telemetry
energy optimized (do not traverse batteryoperated nodes)
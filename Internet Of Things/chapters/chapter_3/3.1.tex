\section{Introduction}

Fieldbus technology is designed to connect control systems with field devices, enabling frequent exchange of short messages.
A hallmark of fieldbus systems is their demand for low latency and deterministic communication, ensuring predictable and timely message delivery.

One of the central challenges in fieldbus networks lies in managing access to a shared communication medium among multiple devices. 
Two main strategies address this:
\begin{enumerate}
    \item \textit{Scheduled access}: devices transmit in a predefined order, coordinated either through centralized scheduling or polling mechanisms. 
        This guarantees collision-free communication, predictable delays, and consistent throughput. 
        However, it introduces system complexity due to the need for synchronization or a central authority.
    \item \textit{Random access}: Devices independently decide when to transmit. 
        While easier to implement and more flexible, this approach only offers statistical performance guarantees and can suffer under heavy traffic due to increased collision rates.
\end{enumerate}

\subsection{Controller Area Network bus}
The Controller Area Network (CAN) bus uses a shared physical medium where all connected devices receive all transmitted messages. 
It employs Carrier Sense Multiple Access (CSMA): devices sense the bus before transmitting. 

To resolve contention, messages are assigned priorities based on their identifier fields.
During transmission, each device monitors the bus. 
If it detects a higher-priority message while transmitting, it stops and defers to the higher-priority sender.

\subsection{Ethernet}
Originally introduced in 1976, Ethernet began as a shared-bus network using coaxial cables.
Through the 1990s, it shifted to star topologies using hubs and twisted-pair cabling. 
Since the 2000s, Ethernet has evolved to fully switched, full-duplex systems capable of high-speed communication.

Ethernet manages traffic using techniques like multiplexing, frame filtering, and multiple access protocols. 
A typical Ethernet frame includes a synchronization preamble and a frame delimiter to mark the start of the transmission.

\paragraph*{MAC address}
Each network interface has a unique MAC address, composed of a manufacturer identifier (first 3 bytes) and a device-specific identifier (last 3 bytes). 
A MAC address of all ones is used for broadcast communication.

\paragraph*{Collision detection}
Legacy Ethernet relied on Carrier Sense Multiple Access with Collision Detection (CSMA/CD). 
Devices monitored the channel before sending data. 
If a collision occurred, all senders aborted and waited a random interval before retrying. 
While effective, this method could introduce delays, especially under load.

Early Ethernet networks used hubs, where all connected devices shared the same bandwidth and collisions were common. 
In contrast, network switches enable full-duplex, collision-free communication by establishing dedicated links between devices and the switch. 
As a result, CSMA/CD is no longer necessary in modern switched Ethernet networks.

\paragraph*{Industry 4.0}
As part of Industry 4.0, Ethernet has been tailored to meet a range of time-sensitive industrial needs. 
It is classified into three categories based on maximum cycle time:
\begin{itemize}
    \item \textit{Class A}: non-critical applications with relaxed timing constraints.
    \item \textit{Class B}: time-sensitive tasks requiring faster responses.
    \item \textit{Class C}: high-priority, ultra-low-latency communication for critical operations.
\end{itemize}

\subsection{Wireless}
The expansion of wireless networking has been fueled by the proliferation of smart objects—connected devices ranging from sensors to everyday electronics, especially within the IoT ecosystem. 
Modern wireless systems support diverse use cases, including: mobile radio networks, cellular IoT operators, low power long range technology, and capillary multi-hop networks.
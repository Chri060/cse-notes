\section{Narrowband Internet of Things}

NB-IoT is a specialized technology designed to support the growing demand for efficient and scalable IoT deployments. 
It meets several critical requirements for IoT applications, including optimized signaling overhead, enhanced security based on 3GPP standards, extended battery life, and the ability to handle both IP and non-IP data, as well as SMS messages. 
Operating within licensed spectrum, NB-IoT benefits from a robust and interference-free connection, making it an ideal solution for various IoT use cases. 
Additionally, it can leverage existing mobile network infrastructure, simplifying integration into current systems.

\paragraph*{Power saving mode}
A key feature of NB-IoT is its Power Saving Mode which is designed to reduce energy consumption even further than traditional idle modes.
This mode allows devices to remain registered with the network without needing to re-establish packet data network connections, thus avoiding energy-intensive re-attachments. 
Devices in Power Saving Mode are not immediately reachable for mobile-terminated services, which helps conserve power without sacrificing network connectivity. 

\paragraph*{Random access procedure}
The random access procedure defines the process through which a device can initially connect to the network, ensuring an efficient use of available resources. 
Alongside this, the mechanism for message repetition ensures that critical messages are reliably delivered to devices, even in poor coverage areas, with the possibility of repeating messages up to 2048 times to maintain signal integrity.

\paragraph*{Coverage enhancements}
Coverage in NB-IoT is adaptable to varying radio conditions through the use of Coverage Enhancement levels. 
In areas of good coverage, a device operates at CE-Level 0, requiring minimal signal boosting. 
As coverage weakens, the network can boost the signal, moving to CE-Level 1 and then CE-Level 2 for areas with poor signal reception, where the network increases the number of repetitions of downlink messages and uplink preambles to ensure reliable communication. 
This enhanced coverage capability significantly improves the network's reach, even in challenging environments, by increasing the maximum coupling loss from 144 dB to 164 dB.

\paragraph*{Cost}
To reduce implementation costs, NB-IoT simplifies LTE features, focusing on lower-complexity user equipment and network configurations. 
For instance, NB-IoT devices use significantly smaller transport block sizes for both downlink and uplink communication. 
They support only single-stream transmissions and require only a single antenna. 
The reduced complexity also means that there is no need for turbo decoders in the UE, as only Turbo Block Code is used for downlink channels. 
Additionally, the system simplifies mobility measurements, with the device only needing to perform mobility measurements during idle mode, reducing operational overhead. 
This streamlined architecture also limits the system to half-duplex frequency-division duplexing operation and eliminates the need for parallel processing in the physical layer.

\paragraph*{In-band operation}
In-band operation is another key feature of NB-IoT. 
By introducing NB-IoT carriers within existing LTE carriers, this approach allows for scalable deployments by adding more carriers as needed. 
However, this can introduce near-far interference with non-upgraded LTE base stations, so a network-wide deployment is recommended to avoid performance issues. 
While this in-band deployment may reduce the capacity and maximum speed of the LTE carrier, solutions such as boosting the NB-IoT output power or dynamically sharing baseband capacity with LTE can mitigate these effects.

\paragraph*{Enhanced discontinuous reception}
One of the standout features for power-saving in NB-IoT is Enhanced Discontinuous Reception.
This feature introduces longer sleep cycles, which drastically reduce the frequency at which devices must remain awake to monitor the network. 
Additionally, the eDRX feature enables a long paging cycle of up to 2.92 hours, which can be negotiated between the UE and the network, further optimizing battery life for devices that need to stay connected without frequent wake-ups.

\subsection{Summary}
NB-IoT represents a natural evolution of LTE networks, providing an efficient solution for cellular IoT. 
Notable improvements in Release 12 significantly reduced modem complexity, offering a 50\% reduction compared to Category 1 devices and ensuring over 10 years of battery life for delay-tolerant traffic. 
Release 13 further optimized the system with a 75\% reduction in modem complexity, a 10+ year battery life for other IoT use cases, and a 15-20 dB coverage enhancement. 
While narrower LTE system bandwidths, such as 200 kHz, could be introduced for further optimization, this requires substantial efforts compared to the simpler enhancements already provided.

As the world moves toward 5G, the evolution of NB-IoT continues. 
5G promises to unlock even more industrial IoT applications through enhanced features like millimeter-wave access, massive MIMO, network densification, slicing, and mobile edge computing. 
These advancements will enable a wide range of applications, from immersive and virtual reality to autonomous vehicles and collaborative robotics, expanding the potential of IoT across various industries.
\section{Message Queuing Telemetry Transport}

Message Queuing Telemetry Transport (MQTT) is a client-server, publish and subscribe messaging protocol designed for lightweight and efficient communication. 
In MQTT, clients do not communicate directly with each other. 
Instead, they publish messages to topics, and other clients subscribe to those topics.
A single client can publish a message, and all clients subscribed to that topic will receive the message.
Unlike CoAP's pull-based request and response model, MQTT uses a push model where updates are automatically sent to clients when new information is available.

\paragraph*{Connection}
In MQTT, each client establishes a single connection (which supports push capabilities) to the MQTT broker for communication. 
MQTT can also work even through firewalls or NAT devices, thanks to its use of TCP and well-defined protocols.
When a client connects to the broker, it sends a \texttt{CONNECT} message with several fields.
When the broker responds to the connect message, it sends a \texttt{CONNACK} message which indicates wether the client's session was retained and the result of the connection attempt. 

\subsection{Publisher}
In MQTT, the \texttt{PUBLISH} message is used by the publisher to send data to the broker. 
The Quality of Service (QoS) of a publisher can be: 
\begin{itemize}
    \item \textit{QoS 0}: the message is delivered at most once, with no guarantee of delivery. 
    \item \textit{QoS 1}: the message is guaranteed to be delivered, but it may be delivered multiple times. 
        The client stores the message and retransmits it until the broker acknowledges receipt.
        Once the broker receives the publish message, it sends a \texttt{PUBACK} message back to the client to acknowledge receipt. 
    \item \textit{QoS 2}: ensures that the message is delivered exactly once, and involves a 4-step handshake:
        \begin{enumerate}
            \item \textit{Publish reception} (broker): the MQTT broker processes the packet and sends a \texttt{PUBREC} message back. 
            \item \textit{Reception} (client): upon receiving the \texttt{PUBREC} message, the client discards the original packet and sends a \texttt{PUBREL} message to the broker.
            \item \textit{Acknowledgment} (broker): the broker clears any current state and sends a \texttt{PUBCOMP} message to confirm the delivery of the message.
            \item \textit{Completion} (broker): the client receives the \texttt{PUBCOMP} message.
        \end{enumerate}
\end{itemize}

\subsection{Subscriber}
In MQTT, the \texttt{SUBSCRIBE} message is used by the client to subscribe to one or more topics. 
Once the broker receives the \texttt{SUBSCRIBE} message, it responds with a \texttt{SUBACK} message.

To unsubscribe from topics, the client sends an \texttt{UNSUBSCRIBE} message. 
\noindent When the broker processes the \texttt{UNSUBSCRIBE} request, it sends an \texttt{UNSUBACK} message.

\subsection{Session}
In MQTT, the session management behavior depends on the type of session: 
\begin{itemize}
    \item \textit{Non-persistent session}: when a client disconnects, all client-related information is cleared from the broker. 
        This means that when the client reconnects, it needs to re-subscribe to topics and will not receive messages sent during its disconnection.
    \item \textit{Persistent session}: the client and the broker maintain the session state even if the client disconnects. 
        This ensures that the client can resume its communication without losing its subscription information or missing messages. 
        With a persistent session, even if the client is offline, the broker will hold the state and deliver any relevant messages when the client reconnects. 
\end{itemize}

\subsection{Messages}
In MQTT, publishing and subscribing are asynchronous operations. 
Retained messages are \texttt{PUBLISH} messages where the \texttt{retained} flag is set to one. 
These messages are stored by the broker, and whenever a new client subscribes to the topic, the broker immediately sends the last retained message on that topic to the subscribing client. 
This ensures that subscribers receive the latest available message, even if no new message has been published.

\paragraph*{Last will}
The Last Will and Testament (LWT) message in MQTT allows a client to notify other clients about an unexpected disconnect. 
When a client connects to the broker, it can specify a LWT message.
The broker stores the LWT message and only sends it if it detects that the client has disconnected unexpectedly.
When the client experiences a hard disconnection, the broker sends the stored LWT message to all subscribed clients on the specified topic.
If the client disconnects gracefully, the broker will discard the LWT message.

\paragraph{Keep alive}
It is the client's responsibility to maintain an active MQTT connection. 
If no other interactions occur within the specified keep-alive interval, the client sends a ping request to the broker. 
The broker, upon receiving the ping, responds with a ping response to confirm the connection is still active. 
This mechanism ensures the connection remains stable and prevents unnecessary disconnections due to inactivity.

\subsection{MQTT for Sensor Networks}
MQTT for Sensor Networks (MQTT-SN) is a variation of the standard MQTT protocol designed to better suit the constraints of sensor networks and environments with limited resources. 
Here are the main differences compared to the standard MQTT protocol:
\begin{itemize}
    \item Extended architecture with gateways and forwarders.
    \item New gateway discovery procedures and messages. 
    \item Some messages are more.
    \item Extended keep alive procedures to support sleeping clients.
\end{itemize}
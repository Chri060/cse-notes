\section{Introduction}

\begin{definition}[\textit{Application}]
    An application refers to a program running on a host that can communicate with another program over a network.
\end{definition}
\noindent Applications enable machines, sensors, and control systems to exchange data, forming the foundation of smart manufacturing and Industry 4.0.

However, communication technologies in Industry 4.0 remain highly fragmented. 
Different protocols and languages coexist within factories, often creating compatibility challenges. 
Many field-level and factory-level technologies do not natively support Internet-based communication, making seamless data exchange difficult.

A practical solution to this fragmentation is the adoption of Service-Oriented Architectures, which facilitate standardized and scalable communication. 
SOA-based communication in Industry 4.0 generally follows two key models:
\begin{itemize}
    \item \textit{Client and server model}: a client requests data from a server, much like how web browsers access information. 
        Common protocols include HTTP and CoAP.
    \item \textit{Publish and subscribe model}: instead of direct requests, devices publish data to a central broker, which then distributes updates to subscribed clients. 
        This approach is more efficient for real-time and event-driven communication, with protocol such as MQTT being widely used.
\end{itemize}
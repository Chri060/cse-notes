\section{Multi-Armed Bandits}

The Multi-Armed Bandit problem is a simplified framework of Reinforcement Learning, often viewed as a special case of a Markov Decision Process. 
It can be described as follows:
\begin{itemize}
    \item \textit{State space} ($\mathcal{S}$): a single state, representing the lack of contextual information.
    \item \textit{Action space} ($\mathcal{A}$): a set of $N$ actions, or arms. 
    \item \textit{Transition probabilities} ($P$): state transitions are trivial, and the probability of remaining in the same state is always 1. 
    \item \textit{Reward function} ($R$): rewards depend solely on the action
    \item \textit{Discount factor} ($\gamma$): fixed to 1. 
    \item \textit{Initial probabilities} ($\mu_0$): fixed to 1. 
\end{itemize}
\noindent The only component left to define is the reward structure, which can vary depending on the nature of the problem:
\begin{itemize}
    \item \textit{Deterministic}: each arm yields a fixed reward. 
        This is trivial to solve.
    \item \textit{Stochastic}: rewards are drawn from a stationary probability distribution.
    \item \textit{Adversarial}: rewards are chosen strategically by an adversary who knows the agent's algorithm.
\end{itemize}
In essence, the $k$-Armed Bandit problem requires an agent to select from $k$ actions and receive rewards based on those choices, aiming to maximize the total reward over time. 
Unlike typical Markov Decision Processes, the Multi-Armed Bandit setting involves decisions made in isolation, without a broader state context.

\subsection{Expexted regret}
The agent interacts with the environment as follows:
\begin{itemize}
    \item At each round $ t $, the agent selects an arm $ a_{i} $.
    \item The environment generates a reward $ r $ from the distribution $ \mathcal{R}(a_{i}) $.
    \item The agent updates its knowledge based on the history $ h $ (previous actions and rewards).
\end{itemize}
The optimal arm $ a^\ast $ has the highest expected reward. 
When the agent selects arm $ a_{i_t} $, it incurs an instantaneous regret:
\[\Delta_{i}=R(a^\ast)-R(a_i)\]
Here, $ \Delta_i $ is known as the suboptimality gap for arm $ a_i $. 
The goal is to minimize the expected cumulative regret over $ T $ rounds:
\[L_T = T \mu^\ast - \sum_{i}^I t\mu_i\]
Here, $\mu^\ast$ is the maximum reward found, and $\mu_i$ are the rewards for each arm $i$. 

Minimizing cumulative regret is equivalent to maximizing cumulative reward. 
A good algorithm satisfies the no-regret property, where:
\[\frac{L_T}{T} \to 0  \text{ as } T \to \infty\]
The difficulty of the problem depends on the similarity of the arms: the closer their rewards, the harder it is to identify the best one.
\begin{theorem}
    For stochastic Multi-Armed Bandit problems, any algorithm satisfies the regret bound:
    \[L_T \geq \log T \sum_{i} \frac{\Delta_i}{KL(\mathcal{R}(a_i), \mathcal{R}(a^\ast))}\]
    as $ T \to \infty $, where $ KL(\mathcal{R}(a_i), \mathcal{R}(a^\ast)) $ is the Kullback-Leibler divergence between the reward distributions of arm $ a_i $ and the optimal arm $ a^\ast $. High regret arises when the rewards of different arms are very similar, making it challenging to distinguish the best arm.
\end{theorem}
The Kullback-Leibler divergence is defined as: 
\[KL(\mu_i, \mu^\ast)=\mu_i\ln\left(\dfrac{\mu_i}{\mu^\ast}\right)+(1-\mu_i)\ln\left(\dfrac{1-\mu_i}{1-\mu^\ast}\right)\]

\paragraph*{Action values}
In Multi-Armed Bandits problems, the agent uses action-value estimates $ Q_n $ to guide decision-making. The update rule depends on whether the problem is stationary or non-stationary:
\begin{itemize}
    \item \textit{Non-stationary problems}:
    \[Q_{n+1} = Q_n + \alpha \left(R_n - Q_n\right)\]
    Here, $ \alpha $ is a step-size parameter that adjusts over time.
    \item \textit{Stationary problems}: $ \alpha $ is typically set to $ \frac{1}{n} $, the reciprocal of the number of times the action has been taken.
\end{itemize}

\paragraph*{Initialization}
Action-value estimates are traditionally initialized to zero, but alternative initialization strategies can influence exploration. 
In optimistic initialization we set high initial values encourages exploration by making all arms initially appealing. 
However, this approach is less effective in non-stationary problems, where the environment evolves over time.

The challenge is choosing appropriate optimistic values is non-trivial, as they must balance exploration and exploitation effectively.

\subsection{Arm selection}
Selecting the action with the highest value is not always optimal, as it may fail to balance immediate rewards with long-term benefits. 
Multi-Armed Bandit authomatically balance between exploration and exploitation. 

Two main formulations exist for Multi-Armed Bandit algorithms:
\begin{itemize}
    \item \textit{Frequentist}: the rewards are unknown parameters. 
        A policy selects an arm at each time step based on the observation history.
    \item \textit{Bayesian}: the rewards are treated as random variables with prior distributions. 
        The policy selects an arm based on the observation history and the prior information.
\end{itemize}

\paragraph*{Upper Confidence Bound}
The Upper Confidence Bound algorithm is a frequentist approach to the stochastic Multi-Armed Bandit problem. 
Instead of relying solely on empirical estimates, it computes an upper bound $U(a_i)$ for the expected reward $R(a_i)$ with high probability:
\[U(a_i) = \hat{R}_t(a_i) + B_t(a_i) =  Q_t(a) + c \sqrt{\frac{\ln(t)}{N_t(a)}} \]
Here, $N_t(a_i)$ represents the number of times arm has been pulled, $Q_t(a)$ represents the exploitation term, $c$ is a user-defined coefficient, and $\sqrt{\frac{\ln(t)}{N_t(a)}}$ accounts for exploration. 
The selected action at each time step is:
\[A_t = \argmax_a \left[ Q_t(a) + c \sqrt{\frac{\ln(t)}{N_t(a)}} \right]\]
\begin{theorem}
    At a finite time horizon $T$, the expected cumulative regret of the Upper Confidence Bound one algorithm for a stochastic Multi-Armed Bandit problem satisfies:
    \[L_T \leq 8 \log T \sum_{i} \left[ \frac{1}{\Delta_i} + \left( 1 + \frac{\pi^2}{3} \right)\right] \Delta_i\]
\end{theorem}

\paragraph*{Thompson sampling}
Thompson sampling is a Bayesian approach to stochastic Multi-Armed Bandit problems. 
It uses Bayesian priors for each arm's reward distribution $f_1, \dots, f_N$ and updates these distributions based on observed outcomes:
\begin{itemize}
    \item At each round $t$, sample a reward estimate $\hat{r}_1, \dots, \hat{r}_N$ from the current posterior distributions.
    \item Select the arm with the highest sampled value:
        \[a_{i_t} = \argmax_i \hat{r}_i\]
    \item Update the prior distribution for the selected arm based on the observed reward.
\end{itemize}

\begin{theorem}
    At time $T$, the expected cumulative regret of Thompson sampling for a stochastic Multi-Armed Bandit problem satisfies:
    \[L_T \leq \left( \log T + \log\log T \right) \sum_{i} \frac{\Delta_i}{KL(\mathcal{R}(a_i), \mathcal{R}(a^\ast))}\]
    Here, $KL(\mathcal{R}(a_i), \mathcal{R}(a^\ast))$ is the Kullback-Leibler divergence between the reward distributions of arm $a_i$ and the optimal arm $a^\ast$.
\end{theorem}
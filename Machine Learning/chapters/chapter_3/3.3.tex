\section{Model selection}

Adding additional features increases the dimensionality of the input space exponentially. 
This growth not only increases computational cost but also requires more data and may introduce high variance in the model.
Our objective is to select a model that minimizes prediction error by reducing variance. 
Achieving this requires methods that balance complexity with performance, such as:
\begin{itemize}
    \item \textit{Feature selection}: selecting a subset of the most relevant features to avoid unnecessary complexity.
    \item \textit{Dimensionality reduction}: transforming features into a lower-dimensional space while retaining essential information.
    \item \textit{Regularization}: adding penalty terms to the loss function to discourage complex models, helping to prevent overfitting.
\end{itemize}
These approaches can be combined to improve model performance, as they address complementary aspects of the model selection process.

\subsection{Feature selection}
The simplest approach to feature selection is to evaluate all possible combinations of features. 
However, given $M$ features, the number of models with exactly $k$ features to evaluate is $\binom{M}{k}$ for each subset. 
This exhaustive search quickly becomes computationally infeasible as $M$ grows.

In practice, feature selection is often adapted to the type of model being used, and there are three main methods to perform feature selection: filter, embedded, and wrapper methods.

\paragraph*{Filter methods}
Filter methods evaluate each feature independently, using statistical measures to assess its relevance to the target variable. 
The most relevant $k$ features are then selected based on these metrics. 
While filter methods are computationally efficient, they may overlook interactions between features because they assess each feature individually, independent of the model.

One example of a filter method is the Pearson correlation coefficient, which measures the linear association between each feature $x_j$ and a target $y$: 
\[\hat{\rho}(x_j,y)=\dfrac{\sum_{n=1}^{N}(x_{j,n}-\bar{x}_j)(y_n-\bar{y})}{\sqrt{\sum_{n=1}^{N}(x_{j,n}-\bar{x}_j)^2}\sqrt{\sum_{n=1}^{N}(y_n-\bar{y})^2}}\]
Here, $\bar{x}_j$ and $\bar{y}$ are the mean of all the $x_j$ and $y$ respectively.
Features with higher correlation coefficients are prioritized. 
Filter methods typically capture only linear relationships, though there are also extensions to detect nonlinear associations.

\paragraph*{Embedded methods}
Embedded methods incorporate feature selection within the model training process itself. 
This approach is often used with regularized models, such as Lasso regression, which automatically drives irrelevant feature weights toward zero, effectively eliminating them.
Although embedded methods are computationally efficient, they are specific to the chosen model and may not generalize to other algorithms.

\paragraph*{Wrapper methods}
Wrapper methods utilize a search algorithm to find optimal feature subsets by iteratively training models on different subsets and evaluating their performance.
Unlike filter methods, wrapper methods consider interactions between features.
Common strategies for searching subsets include greedy algorithms such as forward selection (starting with no features and adding one at a time) and backward elimination (starting with all features and removing one at a time). 
Though potentially more accurate than filter methods, wrapper methods are often computationally intensive.

\subsection{Dimensionality reduction}
Dimensionality reduction seeks to reduce the number of features in the input space, but it differs from feature selection in two important ways: it utilizes all available features and projects them into a lower-dimensional space, rather than selecting a subset of the original features. 
Additionally, dimensionality reduction is generally an unsupervised approach, as it does not rely on labeled data.

Several popular methods for dimensionality reduction, each with distinct strengths and use cases, include Principal Component Analysis (PCA), Independent Component Analysis (ICA), self-organizing maps, autoencoders, ISOMAP, and t-SNE.

\paragraph*{Principal Component Analysis}
PCA is an unsupervised dimensionality reduction technique that performs a linear transformation on the original data to extract lower-dimensional features.
The core idea of PCA is to find a set of orthogonal directions, or principal components, that capture the maximum variance in the data. 
The principal components are ranked so that the first component accounts for the highest variance, the second component for the next highest, and so on.

The steps for performing PCA are as follows:
\begin{enumerate}
    \item Translate the original data $\mathbf{x}$ to $\tilde{\mathbf{x}}$ to ensure it has zero mean.
    \item Compute the covariance matrix of $\tilde{\mathbf{x}}$: 
        \[\mathbf{C}=\tilde{\mathbf{x}}^T\tilde{\mathbf{x}}\]
    \item The eigenvectors of $\mathbf{C}$, which correspond to the principal components of the data.
    \item The eigenvectors can be computed using Singular Value Decomposition. 
\end{enumerate}
There are various methods for determining the number of principal components to retain:
\begin{itemize} 
    \item Retain components until the cumulative variance reaches 90\%-95\%. 
        Cumulative variance is calculated as the fraction of each eigenvalue $\lambda_i$ relative to the sum of all eigenvalues. 
    \item Retain components that individually explain more than a specified percentage (e.g., 5\%) of the variance, discarding only those with very low variance. 
    \item Identify the elbow in the cumulative variance plot, where the marginal gain in explained variance begins to diminish significantly. 
\end{itemize}
PCA is commonly used for feature extraction, data compression, and visualization in lower dimensions.

\subsection{Regularization}
Regularization is another key method for model selection, primarily aimed at reducing model complexity and preventing overfitting by penalizing larger model coefficients. 
Regularization techniques such as Lasso, Ridge, and Elastic Net are typically applied to linear regression models, but they can also be adapted to other types of models: 
\begin{itemize} 
    \item \textit{Lasso} (L1 regularization): Adds a penalty equal to the absolute value of the coefficients. 
        Lasso performs feature selection by forcing some coefficients to zero, effectively removing certain features from the model. 
    \item \textit{Ridge} (L2 regularization): adds a penalty equal to the square of the coefficients, which helps to shrink all coefficients but does not eliminate any entirely. 
        This method is particularly useful for multicollinearity. 
    \item \textit{Elastic Net}: combines L1 and L2 regularization, balancing feature selection and coefficient shrinkage. 
        Elastic Net is effective in scenarios where there are many correlated features. 
\end{itemize}
Regularization not only aids in model selection by balancing model complexity and predictive accuracy but also improves generalization, particularly in high-dimensional datasets where overfitting is a concern.
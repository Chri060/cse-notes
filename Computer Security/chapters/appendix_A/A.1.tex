\section{Introduction}

The Instruction Set Architecture (ISA) serves as the abstract blueprint for a computer architecture, outlining its logical structure. 
It encompasses essential programming elements like instructions, registers, interrupts, and memory architecture. 
Importantly, the ISA may deviate from the physical microarchitecture of the computer system in practice.

\subsection{History}
The x86 ISA originated in 1978 as a 16-bit ISA with the Intel 8086 processor. 
Over time, it transitioned into a 32-bit ISA with the Intel 80386 in 1985. 
Finally, in 2003, it advanced to a 64-bit ISA with the AMD Opteron processor.

Characterized by its Complex Instruction Set Computing (CISC) design, the x86 ISA retains numerous legacy features from its earlier iterations.

\subsection{Von Neumann architecture}
Von Neumann architecture, named after mathematician and physicist John von Neumann, is a conceptual framework for designing and implementing digital computers. 
It consists of four main components:
\begin{enumerate}
    \item \textit{Central Processing Unit} (CPU): this is the brain of the computer, responsible for executing instructions. 
        It contains an arithmetic logic unit (ALU) for performing arithmetic and logical operations, and a control unit that fetches instructions from memory, decodes them, and controls the flow of data within the CPU.
    \item \textit{Memory}: Von Neumann computers have a single memory space that stores both data and instructions. 
        This memory is divided into cells, each containing a unique address. 
        Programs and data are stored in memory, and the CPU accesses them as needed during program execution.
    \item \textit{Input/Output} (I/O) devices: these devices allow the computer to interact with the external world. 
        Examples include keyboards, monitors, disk drives, and network interfaces. 
        Data is transferred between the CPU and I/O devices through input and output operations.
    \item \textit{Bus}: the bus is a communication system that allows data to be transferred between the CPU, memory, and I/O devices. 
        It consists of multiple wires or pathways along which data travels in the form of electrical signals.
\end{enumerate}
In Von Neumann architecture, programs and data are stored in the same memory space, and instructions are fetched from memory and executed sequentially by the CPU. 
This architecture is widely used in modern computers and forms the basis for most general-purpose computing devices. 
However, it has some limitations, such as the Von Neumann bottleneck, where the CPU is often waiting for data to be fetched from memory, leading to inefficiencies in performance.

The memory is structured into cells, with each cell capable of holding a numerical value ranging from -128 to 127.
\begin{abstract}
    This course provides a comprehensive introduction to information security, beginning with foundational concepts in understanding what information security is, alongside definitions of vulnerabilities, risks, exploits, and attackers, with a focus on managing security as a risk-based practice.

The course then covers cryptography, providing a brief history and discussing paradigm shifts, perfect and computational confidentiality, data integrity, Message Authentication Codes (MACs), cryptographic hash functions, and asymmetric cryptographic tools like key agreement, key exchange, and digital signatures. 
Students will explore Public Key Infrastructure (PKI) and critically assess digital signature schemes and PKI engineering pitfalls.

Authentication is discussed in terms of the three factors of authentication, multifactor methods, and evaluation of authentication technologies, including ways to bypass these controls. 
This leads into authorization and access control, covering discretionary (DAC) and mandatory (MAC) access control, multilevel security, and applications in sensitive contexts like military information management.

Next, the course examines software vulnerabilities, addressing design, implementation, and configuration errors, memory issues like buffer overflows, web application security, and code-injection vulnerabilities such as cross-site scripting and SQL injections.

The course also addresses secure networking architectures, exploring network protocol attacks, firewall technologies, secure network setups (like DMZ and multi-zone networks), VPNs, and secure protocols like SSL/TLS.

Finally, malicious software and its evolution are discussed, from early threats like the Morris worm to modern malware, botnets, and the underground economy. 
Techniques in malware analysis, antimalware strategies, and rootkit detection and mitigation are also covered.
\end{abstract}
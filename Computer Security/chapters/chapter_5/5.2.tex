\section{Cross-Site Scripting}

Consider a simple blog application where users can post comments. 
If the application displays comments directly without filtering, an attacker could insert malicious code, such as:
\begin{verbatim}
<script> alert('JavaScript Executed'); </script>
\end{verbatim}
If this input is not properly sanitized, the script will execute on the screens of other visitors. 
This vulnerability is known as Cross-Site Scripting (XSS).

XSS is a security flaw where an attacker can inject client-side code into a web page. 
There are three primary types of XSS attacks:
\begin{enumerate}
    \item \textit{Stored XSS}: the malicious input is saved on the target server, such as in a database. 
        When a victim accesses the affected content, the malicious code is retrieved and executed in the victim's browser.
    \item \textit{Reflected XSS}: the malicious input is sent to the Web Application in a request. 
        The application includes this input in its response without sanitizing it, causing the malicious script to execute in the client's browser.
    \item \textit{DOM-based XSS}: the attack occurs entirely within the victim's browser. 
        Malicious scripts are executed by client-side JavaScript code without the need to send data back to the server.
\end{enumerate}
XSS can lead to severe consequences, including:
\begin{itemize}
    \item \textit{Cookie theft}: stealing session cookies or hijacking user sessions.
    \item \textit{Session manipulation}: altering sessions and performing unauthorized actions.
    \item \textit{Data theft}: eavesdropping on sensitive information.
    \item \textit{Drive-by downloads}: initiating unwanted software downloads.
    \item \textit{Bypassing Same Origin Policy}: overcoming restrictions intended to isolate content from different origins.
\end{itemize}

\paragraph*{Same Origin Policy}
The Same Origin Policy (SOP) is a fundamental security feature enforced by web browsers. 
SOP ensures that client-side code loaded from one origin (domain) can only interact with data from the same origin. 
This policy helps prevent malicious scripts from accessing sensitive data on other origins. 
However, modern web technologies, such as Cross-Origin Resource Sharing (CORS) and various client-side extensions, can sometimes bypass these restrictions, leading to potential security concerns.

\subsection{Content Security Policy}
Content Security Policy (CSP) is a W3C specification designed to enhance web security by instructing browsers on which content sources are considered trustworthy. 
It acts as an extension of the Same-Origin Policy, providing more control and flexibility in defining what content can be loaded and executed. 
CSP is implemented through a set of directives sent from the server to the client via HTTP response headers.

CSP includes a variety of directives to control different aspects of web content, such as:
\begin{itemize}
    \item \texttt{script-src}: specifies the sources from which JavaScript can be loaded.
    \item \texttt{form-action}: defines the valid endpoints to which forms can be submitted.
    \item \texttt{frame-ancestors}: lists the sources allowed to embed the page within frames or iframes.
    \item \texttt{img-src}: specifies the allowed origins for loading images.
    \item \texttt{style-src}: controls the sources from which stylesheets can be loaded.
\end{itemize}
The effectiveness of these directives depends on the browser's implementation, and while CSP is increasingly adopted, it comes with its own set of challenges:
\begin{itemize}
    \item \textit{Policy creation}: identifying who is responsible for writing and managing CSP rules.
    \item \textit{Manual process}: the process of creating and maintaining CSP policies is largely manual and complex.
    \item \textit{Limited automation}: there are few automated tools available to assist with policy creation, leaving much of the task to manual efforts.
    \item \textit{Ongoing maintenance}: keeping CSP policies up-to-date can be difficult, especially as web pages often load resources from multiple sources, and both pages and resources can change dynamically.
\end{itemize}
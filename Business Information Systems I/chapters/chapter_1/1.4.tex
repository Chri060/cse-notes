\section{Agency theory}

Agency theory challenges the traditional view that markets and hierarchies are entirely separate coordination mechanisms. 
Instead, it suggests a continuum between the two, recognizing that market-like coordination mechanisms exist even within organizations. 
By applying these mechanisms effectively, organizations can improve efficiency.

The key concepts of agency theory are: 
\begin{itemize}
    \item Organizations function as networks of contracts between individuals.
    \item Internal coordination is not solely based on command and control but also involves transactional exchanges.
    \item Just like external markets, organizations incur transaction costs, referred to as agency costs.
    \item Agency costs arise whenever decision-making responsibilities are delegated to lower levels of the hierarchy.
\end{itemize}

\subsection{Agency cost}
Delegation within an organization mirrors market transactions, creating an internal market with its own coordination expenses, known as agency costs. 
These costs include:
\begin{itemize}
    \item \textit{Control costs}: expenses related to monitoring and ensuring compliance.
    \item \textit{Warranty costs}: costs associated with guaranteeing performance.
    \item \textit{Residual loss}: inefficiencies that arise despite control measures.
\end{itemize}

\subsection{Hierarchical control}
In a perfectly competitive market, customers have no direct control over their suppliers (transactions are based entirely on trust and delegation). 
However, in imperfect markets, customers (or suppliers) may exert some level of control over their counterparts. 
This control can take the form of visibility into production processes or even hierarchical oversight, where suppliers operate under certain constraints imposed by their customers.

As a result, the distinction between internal markets and hierarchical coordination is not always clear-cut; instead, there is a spectrum of overlap between the two.

\subsection{Limitations}
The main limitations of agency theory are: 
\begin{enumerate}
    \item Hierarchical mechanisms exist within market transactions, blurring the boundaries between markets and organizations.
    \item Agency theory overlooks task-related uncertainty, which affects the efficiency of coordination mechanisms.
    \item The role of technology is task-dependent—technical innovation influences organizational structures and can shift the cost balance between market-based and hierarchical coordination.
\end{enumerate}
\noindent To address these gaps, information systems theory explores how technology can enhance coordination and reshape organizational structures.
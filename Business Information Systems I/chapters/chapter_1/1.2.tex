\section{Information processing}

Emerging in the 1960s and 1970s, the information processing perspective transformed how organizations viewed technology. 
As IT became widespread within businesses, it led to a fundamental shift in traditional beliefs about the impact of technical innovation.
Key changes included:
\begin{itemize}
    \item A radical shift in management principles, as technology was no longer just a tool for efficiency but a driver of decision-making and strategy.
    \item Unlike earlier views, IT wasn't just about automation (it processed information, the most critical resource for managerial processes).
        Since managerial processes shape decision, it processed information, the most critical resource for managerial processes.
    \item This shift created both virtuous and vicious cycles: when information systems were well-integrated, they improved decision-making, coordination, and adaptability.
        However, poor implementation or information overload could lead to inefficiencies, miscommunication, and bureaucratic bottlenecks.
\end{itemize}
As organizations embraced IT and information processing became central to management, three major theoretical approaches emerged: decision theory, transaction cost economics, and agency theory.

\subsection{Decision theory}
Decision theory (1973-1977) by Galbraith is based on three key rules: 
\begin{enumerate}
    \item Organizations as open systems.
    \item Uncertainty as the variable describing the environment in which organizations operate. 
        Uncertainty measures the ability of an organization to predict market demand. 
    \item Several determinants of uncertainty such as market dynamism, number of suppliers in the market, variety and variability of market requirements, and degree of innovation. 
\end{enumerate}

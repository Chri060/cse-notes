\section{Decision theory}

Galbraith (1973-1977) introduced the decision theory which is based on the idea that organizations function as open systems, constantly interacting with their environment. 
A key challenge they face is uncertainty, which defines the conditions in which they operate and reflects their ability to predict market demand. 
Several factors contribute to uncertainty, including market dynamism, the number of suppliers, variations in market requirements, and the level of innovation.

\subsection{Bounded rationality}
Bounded rationality refers to the cognitive limitations of individuals in processing information. 
Since no single person can handle all the necessary data for decision-making, cooperation becomes essential. 
Through cooperation, individuals and organizational units develop specialized roles, which, in turn, create interdependencies in information flow. 
To function effectively, organizations must manage these interdependencies, as coordination is crucial for overcoming individual cognitive constraints. 
This need for coordination is the fundamental reason organizations exist. 
IT plays a vital role in this process, serving as a tool for organizing and managing information beyond individual capabilities.

\subsection{Hierarchy}
Hierarchy is a coordination mechanism based on command and control, where decision-making authority is centralized rather than delegated. 
It forms the foundation of many companies and institutions, ensuring the structured flow of information within an organization. 
To try to reduce uncertainty effectively, hierarchies rely on two main types of information systems: 
\begin{itemize}
    \item \textit{Vertical information systems}: manage the flow of information along hierarchical lines, reinforcing structured decision-making. 
        However, they have limitations when dealing with environmental uncertainty. 
        As uncertainty increases, exceptions arise, creating the need for more planning and control mechanisms. 
        These exceptions lead to additional information processing demands, often requiring information to flow upward toward higher hierarchical levels for resolution.
    \item \textit{Horizontal information systems}: facilitate direct communication between units at the same hierarchical level.
        These systems improve coordination by enabling decision-making at lower levels, reducing the reliance on top-down control. 
        With a higher degree of delegation, horizontal systems enhance flexibility and responsiveness in dynamic environments.
\end{itemize}

\subsection{Summary}
Organizations can address environmental uncertainty in two main ways: 
\begin{enumerate}
    \item They can increase their information processing capacity by implementing vertical and horizontal information systems.
    \item They can increase slack resources, such as maintaining warehouses or creating independent organizational units.
\end{enumerate}
\noindent However, it assumes that hierarchies are the only coordination mechanism, overlooking market-based coordination as a viable alternative when hierarchies become inefficient. 
Additionally, it considers environmental uncertainty as the primary challenge, ignoring behavioral uncertainty caused by opportunistic individual behavior, which can also undermine hierarchical effectiveness. 
\section{Transaction cost economics}

Williamson (1975) introduced the concept of transaction cost economics, which examines the costs associated with coordinating economic exchanges. 
In its simplest form, a transaction occurs when a customer receives a product or service from a supplier in exchange for payment. 
Transactions represent one of the oldest and most fundamental ways for individuals and organizations to cooperate, as they enable objectives that go beyond individual or organizational rationality.

\paragraph*{Market systems}
A key function of transactions is to reduce behavioral uncertainty by mitigating opportunism. 
In market systems, individuals produce goods and services for themselves and maximize the benefits of their own efficiency. 
However, achieving coordination often requires executing transactions, which come with an associated transaction cost.

The total cost of a coordination mechanism is the sum of production costs and transaction costs. 
Market systems tend to have low production costs because individuals and firms operate efficiently. 
However, transaction costs remain low only under conditions of perfect competition, where market frictions such as information asymmetry, bargaining difficulties, and enforcement issues are minimized.

\paragraph*{Economic transaction}
An economic transaction typically unfolds in four key phases:
\begin{enumerate}
    \item \textit{Matchmaking}: this stage involves identifying potential suppliers based on initial requirements. 
        The outcome is a list of candidates that meet the specified criteria.
    \item \textit{Negotiation}: from the set of potential suppliers, one is selected through discussions that refine the requirements. 
        The result is a formal agreement, often documented in a contract with defined service-level agreements.
    \item \textit{Execution}: the transaction is carried out according to the contract. 
        The expected output includes the delivery of the product or service, along with any deviations or exceptions from the agreed SLAs.
    \item \textit{Post settlement}: if exceptions or issues arise, this phase involves managing them through established procedures to resolve disputes, enforce agreements, or make necessary adjustments.
\end{enumerate}

\subsection{Price system}
The price system serves as the market's primary coordination mechanism, conveying crucial information about supply and demand. 
Prices are influenced not only by production costs but also by market dynamics. 
When the market functions efficiently, prices remain close to production costs and serve as a reliable indicator of product quality.

\paragraph*{Market systems}
According to Williamson (1975), several factors can disrupt market efficiency:
\begin{enumerate}
    \item \textit{Shortages}: when supply fails to meet demand.
    \item \textit{Complexity}: when goods or services are too intricate for standard pricing.
    \item \textit{Customization}: when products require personalization, limiting standardization.
    \item \textit{Uncertainty and information asymmetry}: when buyers and sellers have unequal access to relevant information.
    \item \textit{Negotiation power imbalance}: when either buyers or sellers dominate price-setting.
    \item \textit{Transaction frequency}: when repeated transactions influence cost-efficiency.
    
\end{enumerate}
\noindent When markets fail, businesses often resort to hierarchical coordination, such as in-house production, rather than relying on external suppliers. 
The decision between market-based transactions and hierarchical structures is primarily driven by cost considerations

\subsection{Information technology role}
Information technology (IT) acts as an organizational tool that reduces coordination costs. 
By improving information flow and transaction efficiency, IT strengthens market systems, leading to smaller, more numerous companies and reducing reliance on hierarchical structures.

\subsection{Limitations}
Viewing markets and hierarchies as mutually exclusive coordination mechanisms overlooks hybrid models.
Traditional theories ignore behavioral uncertainty within organizations.
Agency theory addresses these gaps by considering the complexities of decision-making and incentives within firms.
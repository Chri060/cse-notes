\section{Transaction cost economics}

Williamson (1975) introduced the concept of transaction cost economics, which examines the costs associated with coordinating economic exchanges. 
In its simplest form, a transaction occurs when a customer receives a product or service from a supplier in exchange for payment. 
Transactions represent one of the oldest and most fundamental ways for individuals and organizations to cooperate, as they enable objectives that go beyond individual or organizational rationality.

A key function of transactions is to reduce behavioral uncertainty by mitigating opportunism. 
In market systems, individuals produce goods and services for themselves and maximize the benefits of their own efficiency. 
However, achieving coordination often requires executing transactions, which come with an associated transaction cost.

The total cost of a coordination mechanism is the sum of production costs and transaction costs. 
Market systems tend to have low production costs because individuals and firms operate efficiently. 
However, transaction costs remain low only under conditions of perfect competition, where market frictions such as information asymmetry, bargaining difficulties, and enforcement issues are minimized.
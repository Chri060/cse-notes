\section{Introduction}

\begin{definition}[\textit{Technology}]
    A technology represents a process that a given organization can perform, together with all the resources needed to perform the process. 
\end{definition}
\begin{definition}[\textit{Techincal system}]
    A technical system represents a set of machines supporting a given technology.
\end{definition}
\begin{definition}[\textit{Information system}]
    An information system is a set of coordinated processes producing an information output and executing information processing activities. 
\end{definition}
\begin{definition}[\textit{Information Technology architecture}]
    An Information Technology (IT) architecture is a technical system supporting a given information system.
\end{definition}

\noindent For a long time, there has been an ongoing debate about how technical innovation influences organizations. 
A well-established set of beliefs links technological advancements to organizational change, shaping how companies adapt and evolve:
\begin{enumerate}
    \item \textit{Efficiency over effectiveness}: technological innovation primarily enhances efficiency rather than improving overall effectiveness. 
        It streamlines processes but doesn't necessarily guarantee better decision-making or outcomes.
    \item \textit{Economies of scale}: as technology advances, businesses can scale operations more efficiently, reducing costs per unit as production increases.
    \item \textit{Larger optimal size}: the minimum viable size of an organization tends to grow with technological progress, as larger entities can better leverage new systems.
    \item \textit{Increased specialization}: automation and sophisticated systems often lead to a workforce that is more specialized, with employees focusing on narrower, highly technical roles.
    \item \textit{Tayloristic perspective}: the traditional view, inspired by Taylorism, assumes that an optimal organizational structure exists.
    \item \textit{Limited focus on group work}: early studies largely ignored the impact of technology on teamwork and collaboration, focusing instead on individual efficiency.
    \item \textit{Greater bureaucracy and formalization}: as technical systems evolve, so do organizational rules, procedures, and levels of bureaucracy, making work more structured but also more rigid.
    \item \textit{More complex management}: with increased technology comes greater managerial complexity, requiring leaders to navigate intricate systems, regulations, and workflows.
\end{enumerate}

\subsection{Information processing}
Emerging in the 1970s, the information processing perspective transformed how organizations viewed technology. 
As IT became widespread within businesses, it led to a fundamental shift in traditional beliefs about the impact of technical innovation.
A radical shift in management principles, as technology was no longer just a tool for efficiency but a driver of decision-making and strategy.
When information systems were well-integrated, they improved decision-making, coordination, and adaptability.
However, poor implementation or information overload could lead to inefficiencies, miscommunication, and bureaucratic bottlenecks.
As organizations embraced IT and information processing became central to management, three major theoretical approaches emerged: decision theory, transaction cost economics, and agency theory.
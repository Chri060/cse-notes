\section{Project management}

The main Key Performance Indictors (KPI) for a project are: 
\begin{itemize}
    \item \textit{Revenue}: total income generated by the company from its consulting services before any expenses are deducted.
        It is a key indicator of overall sales performance and business growth. 
        In a consulting firm, revenue typically comes from client contracts and project fees.
    \item \textit{Earning before tax}: financial metric that measures a company's profitability before accounting for income tax expenses.
        Earning before tax reflects the profit generated from core operations and other activities, such as investments or interest income, before taxes are deducted.
    \item \textit{Cost on revenues}: ratio of costs directly associated with generating revenue, expressed as a percentage of total revenue.
        This metric helps assess how much of the revenue is consumed by costs such as consultant salaries, software tools, and travel expenses. 
        A high cost-to-revenue ratio may indicate inefficiencies in service delivery or pricing strategies.
    \item \textit{Unallocation}: refers to staff who are not directly assigned to revenue-generating activities. 
        Monitoring unallocated costs is crucial for identifying inefficiencies and ensuring that expenses are properly distributed across projects and services.
\end{itemize}



\subsection{Project development}
The two main approaches used in project development are waterfall and agile, each with its own strengths and limitations.

\paragraph*{Waterfall}
The Waterfall model follows a sequential process, where each phase—analysis, design, development, testing, and implementation. 
Once a phase begins, changes to requirements are difficult to implement. 
This approach is best suited for projects with well-defined requirements from the start.
In consulting, waterfall is ideal for projects with stable and predetermined requirements, particularly in industries where compliance, documentation, and structured processes are essential.
\renewcommand*{\arraystretch}{1.5}
\begin{table}[!ht]
    \centering
    \begin{tabular}{|c|p{10cm}|}
    \hline
    \multicolumn{2}{|c|}{\textbf{Advantages}} \\ \hline
    \textit{Clear requirements}              & A well-defined project scope ensures a structured development process  \\ \hline
    \textit{Predictability}                  & Fixed timelines and structured phases make planning and resource allocation more manageable \\ \hline
    \textit{Comprehensive documentation}     & Each phase includes detailed documentation, providing a thorough project record \\ \hline
    \multicolumn{2}{|c|}{\textbf{Disadvantages}} \\ \hline
    \textit{Limited flexibility}             & Adapting to changes mid-project is challenging, making it less suitable for evolving requirements \\ \hline
    \textit{Delayed feedback}                & Since testing happens at the end, user feedback may come too late, requiring costly revisions \\ \hline
    \textit{Minimal client involvement}      & Limited collaboration during development can lead to misaligned expectations \\ \hline
\end{tabular}
\end{table}
\renewcommand*{\arraystretch}{1}

\paragraph*{Agile}
Agile follows an iterative and incremental approach, allowing for greater flexibility. 
Work is organized into sprints, each delivering a working product increment. 
Agile encourages continuous client collaboration and adapts easily to changing requirements.
In consulting, agile is well-suited for projects where requirements may evolve, or when quick, tangible results are needed.
\renewcommand*{\arraystretch}{1.5}
\begin{table}[!ht]
    \centering
    \begin{tabular}{|c|p{10cm}|}
    \hline
    \multicolumn{2}{|c|}{\textbf{Advantages}} \\ \hline
    \textit{Adaptability}                    & Changes can be accommodated at any stage, making Agile ideal for dynamic projects  \\ \hline
    \textit{Continuous feedback}             & Regular iterations ensure alignment with user needs and expectations \\ \hline
    \textit{Client collaboration}            & Ongoing client involvement fosters a more interactive and responsive development process\\ \hline
    \multicolumn{2}{|c|}{\textbf{Disadvantages}} \\ \hline
    \textit{Uncertain timeline}              &  Iterative cycles can introduce unpredictability, making planning and resource management more complex \\ \hline
    \textit{Minimal documentation}           & Agile prioritizes working software over documentation, which may be a drawback in highly regulated industries\\ \hline
    \textit{Scope creep}                     & Frequent changes and added features can lead to uncontrolled project expansion if not properly managed\\ \hline
\end{tabular}
\end{table}
\renewcommand*{\arraystretch}{1}
\section{Design}

The design of an executive information system involves a structured process to ensure that the system meets organizational needs and provides actionable insights for decision-makers. 
Below is a detailed breakdown of the key steps involved in designing an executive information system:
\begin{itemize}
    \item \textit{Define business requirements}: start by identifying the KPIs that align with the organization's strategic goals. 
        These KPIs will serve as the foundation for the system, ensuring it provides relevant and meaningful insights to executives.
    \item \textit{Identify information sources}: data required to populate the executive information system typically comes from various operational databases and systems.
        These data comes from ERP, CRM, operational information from legacy systems and customer, and administrative information. 
    \item \textit{Process the information}: once the data sources are identified, the next step is to process the data to ensure accuracy, consistency, and usability: selection, cleaning, integration, and aggregation. 
    \item \textit{Store the processed data}: efficient storage is crucial for maintaining performance and accessibility in warehouses and marts with a proper schema. 
        Fact tables store the actual values of the KPIs, while key tables describe the dimensions that provide context to the facts.
    \item \textit{Presentation and processing}: delivering insights to users through intuitive interfaces and advanced analytics tools. 
\end{itemize}

\subsection{Critical Success Factor}
Critical Success Factor (CSF) refers to a business decision variable that is essential for the success of an organization.
These factors serve as the backbone of the requirements analysis and specification process for designing an executive information system.

CSFs are high-level, abstract ideas that represent strategic priorities.
Each CSF is not a single metric but rather a complex concept that corresponds to multiple KPIs. 
The steps needed for this approach are:
\begin{enumerate}
    \item \textit{Pre-definition}: conduct an initial analysis to identify potential CSFs based on industry standards, benchmarks, and existing documentation.
    \item \textit{Top managers interview}: engage in discussions with senior executives and decision-makers to pinpoint the CSFs most critical to the organization's success.
    \item \textit{Robustness analysis}: evaluate and select the appropriate KPIs that best represent each CSF. 
        This involves assessing the relevance, reliability, and feasibility of potential KPIs.
        The KPis are chosen based on cost of information, significance, frequency and quantitatively.
    \item \textit{Refinement and documentation}: present the identified CSFs and KPIs to stakeholders for feedback and refinement.
        Finalize the documentation, ensuring it is clear, informal, and accessible to all relevant parties
\end{enumerate}
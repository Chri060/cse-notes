\section{Information Technology integration}

The integration of Information Technology (IT) within organizations has evolved significantly over time. 
Initially, IT functionalities were developed independently for each organizational function, often without a comprehensive view of processes. 
The focus was on automating existing activities rather than supporting or reengineering them to improve performance. 
Each function operated with separate data, and objectives were often misaligned, resulting in inefficiencies.

The traditional approach involved information being created at the start of a cycle and used later. 
However, to truly optimize organizational performance, a more proactive approach is needed. 
This involves using information at the executive level and integrating the various functions within an organization to create a unified view that enhances decision-making and operations.

There are two key approaches to IT integration:
\begin{itemize}
    \item \textit{Horizontal integration}: this refers to the integration of systems along the operating processes of an organization, specifically those that align with Porter's primary processes.
        These processes include everything from product design and production to marketing and service delivery, all of which benefit from being streamlined and interconnected.
    \item \textit{Vertical integration}: this focuses on connecting the operational portfolio with the executive portfolio. 
        Vertical integration ensures that data and insights flow seamlessly between the front-line operations and the decision-making levels of an organization, enabling a more strategic approach to day-to-day activities.
\end{itemize}
\noindent Some core systems that facilitate integration include:
\begin{itemize}
    \item \textit{Computer Integrated Manufacturing} (CIM): a system that supports the integration of manufacturing processes, combining scheduling, machine control, workforce management, and quality control into a cohesive framework.
    \item \textit{Material Requirements Planning} (MRP): a system that ensures materials are available for production at the right time, helping optimize the production process and minimize waste.
\end{itemize}

\subsection{Computer Integrated Manufacturing}
CIM integrates various manufacturing processes, improving scheduling, machine control, workforce management, and quality management. Its scope covers several key areas:
\begin{itemize}
    \item \textit{Transformation processes}: scheduling transformation activities, controlling machines, and measuring lead times.
    \item \textit{Workforce management}: allocating specialists to production activities, managing work shifts, and monitoring workforce performance.
    \item \textit{Plant management}: tracing the functional states of machinery, scheduling maintenance, and managing alarms for potential issues.
    \item \textit{Materials management}: tracking outgoing orders, deliveries, and internal logistics to ensure smooth material flow.
    \item \textit{Quality management}: conducting quality control, aggregating data, and analyzing results to ensure product quality.
\end{itemize}
\noindent CIM enhances the efficiency of production by integrating these various functions and automating many manual tasks, ensuring that information flows smoothly throughout the manufacturing process.

\subsection{materials requirements planning}
MRP is a production planning and inventory control system designed to manage manufacturing processes and ensure the availability of materials for production.
It emerged in the 1970s and 1980s with the aim of achieving flexibility and economies of scale through optimal planning. 
The main idea behind MRP is to integrate the product structure and production processes, enabling:
\begin{itemize}
    \item \textit{Concurrent engineering}: designing products in parallel with the manufacturing process, ensuring that production aligns with market demands.
    \item \textit{Inside-out production processes}: streamlining internal production processes to reduce inefficiencies and improve response times.
\end{itemize}
\noindent MRP helps organizations achieve greater effectiveness by allowing them to respond more quickly to market demands while simultaneously benefiting from scale economies. 
The system integrates key planning functions, including:
\begin{itemize}
    \item \textit{Master production plan}: outlines the overall production schedule.
    \item \textit{Materials requirements plan}: details the materials needed to support production.
    \item \textit{Operational activity plan}: aligns production activities with available resources.
    \item \textit{Sales objectives and actual sales}: integrates sales forecasts and actual performance to adjust production schedules accordingly.
    \item \textit{ Product structure}: ensures that materials and components are available when needed to produce the final product.
\end{itemize}
\noindent MRP ultimately allows companies to be more responsive to changing market conditions, reducing waste and optimizing inventory management while maintaining flexibility in production.
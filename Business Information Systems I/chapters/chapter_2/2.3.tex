\section{Classes of information in the operational database}
Operational databases are organized to store various types of information that support the flow of activities within an organization. 
These can be categorized into three primary types: transaction information, operations planning information, and catalog information. 
Each type serves a unique role and has distinct characteristics.

\subsection{Transaction information}
Transaction information tracks the flow of operational activities, focusing on exchanges between different organizational units and external parties. This type of information includes activities such as:
\begin{itemize}
    \item Contracts with customers and suppliers.
    \item The status of production activities.
    \item Transfers of materials and semi-finished goods between departments.
    \item The certification of events like the inclusion of a new product in the catalog or the certification of a new supplier.
\end{itemize}
\noindent This type of information is crucial because it reflects the real-time operations and the movement of resources, making it vital for daily decision-making and process monitoring.

\subsection{Operations planning information}
Operations planning information details the objectives and expected results of operational activities. 
Essentially, it encompasses the production program, providing a roadmap for what needs to be achieved and the timelines for these goals. 
This type of information connects operational activities with strategic objectives and helps in setting clear targets for performance and resource allocation.

\subsection{Catalog information}
Catalog information refers to basic, static knowledge that exists independently of the flow of production activities. It is essential for maintaining an organization's foundational knowledge, but unlike transaction information, it doesn't change frequently. 
Examples of catalog information include: product catalog, customer directories, supplier directories, workforce directories, and product structure data.
Although catalog information may seem static, it is quite complex in terms of the number of attributes it includes and requires continuous updates and maintenance. 
This information plays a key role in organizational learning, ensuring continuity even in the face of personnel turnover. Its management directly impacts the ability of the organization to learn and adapt over time.

\subsection{Product structure}
The product structure defines the hierarchical arrangement of components that make up a finished product. 
It ranges from individual components to larger product parts, outlining the relationships and dependencies between them.
This level of information is particularly critical for manufacturing processes, where a clear understanding of how different parts come together is necessary for efficient production planning and inventory management.

\subsection{Information value}
Operational information has intrinsic value as an organizational asset.
Its usefulness extends beyond internal operations, as it can sometimes be monetized or sold, particularly in cases where the information has wide-reaching commercial applications. 
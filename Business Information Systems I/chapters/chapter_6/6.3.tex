\section{Operational Customer Relationship Management}

Operational CRM focuses on streamlining customer-facing processes to enhance efficiency, improve customer satisfaction, and support business growth. 
It encompasses tools and functionalities that enable seamless interactions across various channels, including sales, marketing, and customer service.

\subsection{Sales Force Automation}
Sales Force Automation (SFA) refers to the implementation of CRM functionalities designed to empower the sales force in both Business-to-Consumer (B2C) and Business-to-Business (B2B) environments.
SFA offers significant advantages for organizations from both a management and sales perspective. 
Provides governance over the sales force, ensuring alignment with organizational goals.
Enhances sales capabilities by equipping teams with the necessary tools and insights to perform effectively.

The primary operating objectives of SFA include: reducing the costs associated with customer acquisition, increasing customer retention rates, minimizing bureaucratic processes to improve responsiveness during the sales cycle, and enhancing the overall effectiveness of the sales force.

In physical sales channels, CRM systems play a critical role by providing data that supports the design of individual incentive systems. 
Additionally, the sales force requires comprehensive access to customer information, including invoicing details, to ensure personalized and efficient service delivery.
While the physical channel is often more expensive to operate, it is particularly well-suited for high-value products, especially in B2B contexts. 
However, one challenge is that the sales force may tend to own customer data.

\paragraph*{Performance}
The performance of physical sales channels is typically evaluated using a set of KPIs, which are categorized as: effectiveness, efficiency, and service level. 

\subsection{Document}
In operational CRM, documents serve as critical tools for communication and record-keeping. 
A typical document is composed of two main sections:
\begin{itemize}
    \item \textit{Static Sections}: fixed content that remains consistent across all instances of the document.
    \item \textit{Dynamic Sections}: content that is updated in real-time based on inputs from configuration tools, order management systems, and other sources.
\end{itemize}




\subsection{Call centers}
Call centers are a critical component of operational CRM, integrating telephone and computer technologies to provide efficient customer support and service. 
They serve as a central hub for managing inbound and outbound customer interactions across various channels.

A call center's infrastructure is built on several key technical components:
\begin{itemize}
    \item \textit{Client phone or terminal}: devices used by customers to initiate contact, including traditional fixed or mobile phones, smartphones, and personal computers with web access.
    \item \textit{Telecommunication network}: the network that connects customers to call center operators, ensuring seamless communication.
    \item \textit{Computer architecture of the call center}: includes personal computers used by operators, server machines, and Interactive Voice Response (IVR) systems to manage and process customer interactions.
    \item \textit{Integration technologies}: these technologies bridge the telephone network with the call center's computer network. 
        They also enable the integration of multiple geographically dispersed call centers into a single logical entity, enhancing operational efficiency.
\end{itemize}

\paragraph*{Architecture}
The architecture of a call center can vary based on its maturity and level of integration with the organization's information systems. 
The three main types of architectures are:
\begin{itemize}
    \item \textit{Basic architecture}: represents the simplest and least mature type of call center.
        Utilizes hardware components like Private Branch Exchange (PABX) to distribute calls to operators.
        Lacks integration with the company's information system, limiting access to customer data during interactions.
    \item \textit{Integrated architecture}: integrates the call center with the company's information system through Computer Telephony Integration (CTI) servers.
        Automatically identifies callers and associates them with relevant customer information, which is displayed in real-time on the operator's screen.
        Enhances service quality by enabling personalized and efficient customer interactions.
    \item \textit{Contact center}: incorporates advanced automated systems such as Interactive Voice Response (IVR) and Automatic Speech Recognition (ASR).
        IVR allows customers to interact with the system using keypad inputs or voice commands, while ASR recognizes speech patterns and iteratively refines responses.
        Integrates multiple communication channels into a unified platform.
        Operators handle inbound requests from all channels and manage outbound communications for campaigns.
        While this architecture is more costly to implement, it significantly reduces operating costs and improves service levels.
\end{itemize}

\paragraph*{Performance}
The performance of a call center is measured based on its ability to handle customer interactions efficiently and effectively. 
Key factors influencing performance include:
\begin{itemize}
    \item \textit{Number of calls per time unit}: a higher volume of calls requires more resources to maintain service levels.
    \item \textit{Target level of service}: defined in a SLA, this specifies acceptable waiting times and effectiveness in resolving customer issues.
\end{itemize}
\noindent Common performance indicators include:
\begin{itemize}
    \item \textit{First Call Resolution} (FCR): measures the percentage of issues resolved during the first interaction, indicating effectiveness.
    \item \textit{Percentage of calls with waiting time lower than x seconds}: tracks how many calls are answered within a specified time frame.
    \item \textit{Average call time}: evaluates the efficiency of operators by measuring the average duration of customer interactions.
\end{itemize}

\paragraph*{Sizing}
Proper sizing of a call center ensures optimal resource allocation and adherence to SLAs. 
Need to determine the ideal number of operators and their schedules based on activity volumes and SLA requirements with the inbound call volumes are unevenly distributed over time, requiring dynamic adjustments.
To perform a correct sizing we need:
\begin{itemize}
    \item \textit{Medium-term planning}: focuses on sizing infrastructure and staffing levels to meet anticipated demand.
    \item \textit{Short-term planning}: drives the scheduling of outbound activities and revises operator schedules in real-time to optimize performance.
    \item \textit{Resource allocation}: each operator represents a resource that is dynamically allocated based on demand fluctuations.
\end{itemize}
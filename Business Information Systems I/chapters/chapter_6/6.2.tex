\section{Analytical Customer Relationship Management}

Analytical CRM focuses on leveraging customer data to derive actionable insights, enabling businesses to make informed decisions and enhance customer relationships. 
It encompasses various techniques and processes aimed at understanding customer behavior, predicting trends, and optimizing marketing strategies.

\subsection{Data mining}
Data mining involves the systematic analysis of large datasets to uncover hidden patterns, correlations, and insights that are critical for strategic business management. 
This process can be executed using a variety of techniques, ranging from traditional methods such as descriptive statistics, data visualization, and statistical correlations to more advanced approaches like machine learning algorithms.

These patterns must be presented in an intuitive and interpretable manner to ensure they can be effectively utilized by decision-makers. 
With the growing volume of operational data stored in corporate databases, data mining has become an essential tool for knowledge management, helping organizations unlock valuable insights embedded within their data.

\subsection{Customer profiling}
Customer profiling involves segmenting customers based on identifiable characteristics and behaviors, often facilitated by tools like loyalty cards or transaction histories. 
Customers can be categorized along multiple dimensions, including:
\begin{itemize}
    \item \textit{Loyalty}: analyzing purchasing habits to determine how frequently and consistently customers engage with the brand.
    \item \textit{Price sensitivity}: assessing how responsive customers are to pricing changes or discounts.
    \item \textit{Lifestyle}: evaluating behavioral orientations across various dimensions, such as preferences, interests, and demographic factors.
\end{itemize}
\noindent Customer segmentation serves as the foundation for targeted marketing efforts. 
By identifying specific segments or combinations of segments, businesses can tailor promotions to align with the average characteristics of each group.
This approach not only enhances the effectiveness of marketing campaigns but also enriches the understanding of individual customer preferences.

\subsection{Campaign management}
Campaign management refers to the set of functionalities designed to support the planning, execution, and evaluation of marketing campaigns. 
A typical campaign lifecycle consists of four key phases, each with its associated tasks and objectives:
\begin{enumerate}
    \item \textit{Planning and budgeting}: defining campaign goals, target audiences, resource allocation, and financial constraints.
    \item \textit{Design}: developing campaign materials, messaging, and strategies to engage the intended audience effectively.
    \item \textit{Execution}: implementing the campaign across selected channels.
    \item \textit{Evaluation}: measuring campaign performance through key metrics.
\end{enumerate}
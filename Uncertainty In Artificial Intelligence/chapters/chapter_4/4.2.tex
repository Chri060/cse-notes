\section{Fuzzy measure and probability assignment}

\begin{definition}[\textit{Borel field}]
    A field is considered a Borel field if it possesses the property that when all the $A_n$ sets belong to the field, the union and intersection of these sets also belong to the field. 
\end{definition}

\paragraph*{Fuzzy measure}
A function $g$ defined on a Borel field B within the universe of discourse $X$ is referred to as a fuzzy measure if it satisfies the following properties:
\begin{enumerate}
    \item $g(\varnothing)=0$ and $g(X)=1$.
    \item If $A,B \in B$ and $A \subseteq B$, then $g(A) \leq g(B)$.
    \item If $A_n \in B$ and $A_1 \subseteq A_2 \subseteq A_n$ then $\lim_{n \to \infty}g(A_n)=g\left(\lim_{n \to \infty}A_n\right)$.
\end{enumerate}
The concept of a fuzzy measure differs from a classical measure, as it relaxes the requirement of additivity.

\paragraph*{Basic probabilistic assignment}
The basic probabilistic assignment is defined as follows:
\begin{itemize}
    \item $m:\wp(X) \rightarrow [0,1]$.
    \item $m(\varnothing)=0$.
    \item $\sum_{A \in \wp(X)}m(A)=1$.
\end{itemize}
Here, $m$ provides, for any set A belonging to the power set of X$(\wp(\textnormal{X}))$, an indication of how much the available and relevant evidence supports the notion that a given element belongs to set A.
It's important to note that there is no requirement for $m(X)$ to be equal to 1, no necessity for $m(A) \leq m(B)$ when $A\subseteq B$, and no inherent relationship between $m(A)$ and $m(\lnot A)$.
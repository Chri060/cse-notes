\section{Many-valued logics}

Aristotle raised questions about the suitability of classical logic as a knowledge representation tool.
For example, classical logic struggles to determine the truth value of a proposition in the context of the future.
To address this, a third value (e.g., 0.5) can be introduced to represent undefined situations, leading to the concept of three-valued logic. 
This concept can be extended to infinite-value logics that consider a continuum of truth values between zero and one.
\begin{example}[Logic L1, Łukasiewicz(1930)]
    In this type of infinite-value logic, the main rules include:
    \begin{itemize}
        \item $\textnormal{T}(\lnot a)=1-\textnormal{T}(a)$.
        \item $\textnormal{T}(a \land b)=\min (\textnormal{T}(a),\textnormal{T}(b))$.
        \item $\textnormal{T}(a \lor b)=\max (\textnormal{T}(a),\textnormal{T}(b))$.
        \item $\textnormal{T}(a \implies b)=\min (1, 1+\textnormal{T}(b)-\textnormal{T}(a))$.
        \item $\textnormal{T}(a \Leftrightarrow b)=1-\left\lvert \textnormal{T}(a)-\textnormal{T}(b) \right\rvert$.
    \end{itemize}
\end{example}
These innovations brought about a shift in society, where things are no longer categorically stated as true or false. 
Probability (Kolmogorov, 1929) and stochasticity (Markov, 1906) became the preferred ways to represent this new approach to science and life.

\paragraph*{Differences with classical logic}
The differences between classical logic (L2) and many-valued logic (L1) include:
\begin{itemize}
    \item L1 being isomorphic to fuzzy set theory, with standard operators, while classical logic L2 is isomorphic to set theory.
    \item Tautologies, which are true by definition and used to prove theorems in classical logic L2, may not be valid in L1. 
        For example, the third excluded law ($\textnormal{T}(a \lor \lnot a)=1$) and the non-contradiction law ($\textnormal{T}(a \land \lnot a)=0$) are not valid in L1.
\end{itemize}
In classical logic, the sentence "I'm a liar" would be considered a paradox if we assign meaning to the term "liar," as no formula can have the same truth value as its negation. 
However, this may not be the case in many-valued logics. In Łukasiewicz logic, for instance, it's possible for a sentence to have a truth value of 0.5, and its negation to also have a truth value of 0.5, making the proposition consistent with the axioms and not a paradox.
\section{Syntax and semantic}

A Hoare triple $\{\phi_1\}P\{\phi_2\}$ is a formal notation used to express program correctness, where $\phi_1$ and $\phi_2$ are predicates, and $P$ is a program.
This notation captures the relationship between the program's initial and final states.

\begin{definition}[\textit{Hoare partial correctness}]
    Partial correctness means that if execution starts in a state $s$ satisfying $\phi_1$, and if $P$ terminates, then the resulting state $s^\prime$ must satisfy $\phi_2$.
\end{definition}

\begin{definition}[\textit{Hoare total correctness}]
    Total correctness strengthens this by requiring that, in addition to satisfying $\phi_2$ upon termination, $P$ must always terminate when started in a state satisfying $\phi_1$.
\end{definition}
\noindent For programs that do not contain loops, partial and total correctness are equivalent since termination is already guaranteed.

\paragraph*{Strongest post conditions}
The strongest post condition of a program $P$ with respect to a precondition $\phi_1$ is the most precise description of the states that $P$ can reach when starting from $\phi_1$. 
Formally, if $\{\phi_1\}P\{\phi_2\}$ holds and for any $\phi_2^\prime$ where $\{\phi_1\}P\{\phi_2^\prime\}$ also holds, it follows that $\phi_2\implies\phi_2^\prime$, then $\phi_2$ is the strongest post condition.

\paragraph*{Weakest preconditions}
The weakest precondition of a program $P$ with respect to a post condition $\phi_2$ is the least restrictive condition on the initial state that guarantees $P$ will achieve $\phi_2$.
Formally, if $\{\phi_1\}P\{\phi_2\}$ holds and for any $\phi_1^\prime$ where $\{\phi_1^\prime\}P\{\phi_2\}$ also holds, it follows that $\phi_1^\prime\implies\phi_1$, then $\phi_1$ is the weakest precondition.
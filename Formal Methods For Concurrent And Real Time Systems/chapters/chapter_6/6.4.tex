\section{Structural rules}

The structural rules of Hoare logic define how logical connectives interact with Hoare triples, allowing us to reason about complex program properties systematically:
\begin{itemize}
    \item \textit{Conjunction rule}: if executing $P$ ensures $\psi_1$ when starting from $\phi_1$, and ensures $\psi_2$ when starting from $\phi_2$, then $P$ also ensures both conditions simultaneously when started from the conjunction of the two preconditions:
        \[\dfrac{\{\phi_1\}P\{\psi_1\}\{\phi_2\}P\{\psi_2\}}{\{\phi_1\land\phi_2\}P\{\psi_1\land\psi_2\}}\]
    \item \textit{Disjunction rule}: if $P$ guarantees $\psi_1$ when starting from $\phi_1$, and guarantees $\psi_2$ when starting from $\phi_2$, then if either $\phi_1$ or $\phi_2$ holds initially, executing $P$ will ensure either $\psi_1$ or $\psi_2$: 
        \[\dfrac{\{\phi_1\}P\{\psi_1\}\{\phi_2\}P\{\psi_2\}}{\{\phi_1\lor\phi_2\}P\{\psi_1\lor\psi_2\}}\]
    \item \textit{Existential quantification rule}: if a Hoare triple holds for all values of $v$, then it also holds for any specific instance of $v$, meaning we can introduce a universal quantifier over $v$ in both preconditions and post conditions: 
        \[\dfrac{\{\phi\}P\{\psi\}}{\{\forall v,\phi\}P\{\forall v,\psi\}}\]
    \item \textit{Universal quantification}: if a Hoare triple holds for some value of $v$, then it must hold for at least one such value, allowing us to introduce existential quantification:
        \[\dfrac{\{\phi\}P\{\psi\}}{\{\exists v,\phi\}P\{\exists v,\psi\}}\]
\end{itemize}
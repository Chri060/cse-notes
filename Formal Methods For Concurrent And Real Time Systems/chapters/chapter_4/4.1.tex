\section{Equivalence}

Trace equivalence and bisimulation
Transitions systems TS and TS' are trace equivalent wrt to a set AP if
TracesAP(TS)=TracesAP(TS’)
\begin{theorem}
    Let P be a linear-time property (over AP). If TracesAP(TS)ÍTracesAP(TS') then TS ⊨ P ® TS'⊨ P
\end{theorem}
\begin{corollary}
    Trace-equivalents transition systems satisfy the same LT
property
\end{corollary}
Trace-equivalence can be inadequate for reactive (interactive, concurrent)
systems, as shown by the example.
• Reason: it totally ignores the internal branching structure
– This is what we want for parsers and compilers, where we only care for language
(i.e., trace) equivalence
• Idea of simulation among machines: a transition system TS' can simulate
transition system TS if every step of TS can be matched by one (or more)
steps in TS.
• Bisimulation equivalence denotes the possibility of mutual, stepwise
simulation. 
Let Post(s) be the set of states reachable in one step from s
\begin{definition}[\textit{Bisimulation equivalent}]
    Let $\text{TS}_i=\left\langle S_i,\text{Act}_i,\rightarrow_i,I_i,\text{AP},L_i\right\rangle$ with $i=1,2$ be transition systems over $\text{AP}$. 
    A bisimulation for $(\text{TS}_1,\text{TS}_2)$ is a binary relation $\mathcal{R}\subseteq S_1\times S_2$ such that: 
    \begin{enumerate}
        \item $\forall s_1\in I_1 (\exists s_2\in I_2 (s_1,s_2)\in \mathcal{R})$ and $\forall s_2\in I_2 (\exists s_1\in I_1 (s_1,s_2)\in \mathcal{R})$. 
        \item For all $(s_1,s_2)\in \mathcal{R}$ it holds: 
            \begin{enumerate}
                \item $L_1(s_1)=L_2(s_2)$
                \item If $s_1^\prime\in\text{post}(s_1)$ then there exists $s_2^\prime\in\text{post}(s_2)$ with $(s_1^\prime,s_2\prime)\in\mathcal{R}$. 
                \item If $s_2^\prime\in\text{post}(s_2)$ then there exists $s_1^\prime\in\text{post}(s_1)$ with $(s_1^\prime,s_2\prime)\in\mathcal{R}$. 
            \end{enumerate}
    \end{enumerate}
    \text{TS}_1 and \text{TS}_2 are bisimulation equivalent (bisimilar) denoted \text{TS}_1\sim \text{TS}_2 if there exists a bisimulation $\mathcal{R}$ for $(\text{TS}_1,\text{TS}_2)$. 
\end{definition}
Condition (1) asserts that every initial state of TS1 is
related to an initial state of TS2, and vice versa. 
(s1 and s2 are equally labeled, ensuring their “local” equivalence)
(every outgoing transition of s1 is matched by an outgoing transition of s2, and viceversa)

\paragraph*{Properties}
Bisimulation equivalence implies trace-equivalence
– As a consequence, LTL formulae cannot distinguish two bisimulation-equivalent TS
• Reflexivity, Transitivity, and Symmetry of ∼:
– For a fixed set AP of atomic propositions, the relation ∼ is an equivalence
relation.
• Advantage of bisimulation: given a TS, if we can find a TS', much smaller
than TS, such that TS ∼ TS', then we can verify a property on TS' rather
than on TS.
– For instance, TS can be infinite-state and TS' can be finite-state
– We need some additional definition

Bisimulation equivalence as relation on states
• Bisimulation can be considered as a relation between states of a single
transition system.
• The goal is «minimization» of the number of states relevant to prove a
certain property
\begin{definition}[\textit{Bisimulation equivalent}]
    Let $\text{TS}=\left\langle S,\text{Act},\rightarrow,I,\text{AP},L\right\rangle$ be a transition system. 
    A bisimulation for TS is a binary relation on $\mathcal{R}$ such that for all $(\text{TS}_1,\text{TS}_2)\in\mathcal{R}$: 
    \begin{enumerate}
        \item $L_1(s_1)=L_2(s_2)$
        \item If $s_1^\prime\in\text{post}(s_1)$ then there exists $s_2^\prime\in\text{post}(s_2)$ with $(s_1^\prime,s_2\prime)\in\mathcal{R}$. 
        \item If $s_2^\prime\in\text{post}(s_2)$ then there exists $s_1^\prime\in\text{post}(s_1)$ with $(s_1^\prime,s_2\prime)\in\mathcal{R}$. 
    \end{enumerate}
    States $s_1$ amd $s_2$ are bisimulation equivalent or bisimilar denoted $s_1\sim_{\text{TS}}s_2$ if there exists a bisimulation $\mathcal{R}$ for TS with $(\text{TS}_1,\text{TS}_2)\in\mathcal{R}$. 
\end{definition}
\noindent For transition system $\text{TS}=\left\langle S,\text{Act},\rightarrow,I,\text{AP},L\right\rangle$ it holds that: 
\begin{enumerate}
    \item $\sim_{\text{TS}}$ is an equivalence relation on $S$.
    \item $\sim_{\text{TS}}$ is a bisimulation for TS. 
    \item $\sim_{\text{TS}}$ is the coarsest bisimulation for TS. 
\end{enumerate}

\subsection{Quotient transition system}
An equivalence relation can be used to define the quotient of a given set S (the set of
equivalence classes of elements of S. We can then define the quotient TS
\begin{definition}[\textit{Quotient transition}]
    For transition system $\text{TS}=\left\langle S,\text{Act},\rightarrow,I,\text{AP},L\right\rangle$ amd bisimulation $\sim_{\text{TS}}$, the quotien transition system $\text{TS}\setminus\sim_{\text{TS}}$ is defined by: 
    \[\text{TS}\setminus\sim_{\text{TS}}=\left\langle S\setminus\sim_{\text{TS}},\left\{\tau\right\},\rightarrow^\prime,I^\prime,\text{AP},L^\prime\right\rangle\]
    Here, $I^\prime=\left\{[s]_{\sim}\mid s \in I\right\}$, $\rightarrow^\prime$ is defined by $\frac{s\overset{\alpha}{\rightarrow}s^\prime}{[s]_{\sim}\overset{\tau}{\rightarrow}[s^\prime]_{\sim}}$, and $L^\prime([s]_{\sim})=L(s)$.
\end{definition}
\begin{theorem}
    For any transition system TS it holds that $\text{TS}\sim\text{TS}\setminus\sim$. 
\end{theorem}
\noindent Therefore, we can prove LTL properties on the quotient TS rather than on the original one.

\subsection{CTL equivalence}
We need to define CTL equivalence and compare it with bisimulation
equivalence
In general, states in a transition system are equivalent with respect to a logic
whenever these states cannot be distinguished by the truth value of any formulae of
the logic.
• Let TS, TS1, and TS2 be transition systems over AP without terminal states (infinite
paths only): 
\begin{itemize}
    \item States s1, s2 in TS are CTL-equivalent, denoted s1 \equiv CTL s2, if s1 \models \phi iff s2 \models \phi for all CTL formulae over AP.
    \item TS1 and TS2 are CTL-equivalent, denoted TS1 \equiv CTL TS2, if TS1 \models \phi iff TS2 \models \phi for all CTL formulae over AP.
\end{itemize}
\noindent For a finite transition system TS without terminal states: 
\[\sim_{\text{TS}}=\equiv_{\text{CTL}}\]
Similarly for finite transition systems TS1, TS2 (over AP) without terminal states, the
following two statements are equivalent:
\begin{itemize}
    \item TS1 ∼ TS2
    \item TS1 ≡CTL TS2 , i.e., TS1 and TS2 satisfy the same CTL formulae.
    
\end{itemize}
NB: finiteness assumption is necessary. However, for an infinite-state TS, bisimulation
equivalence implies CTL equivalence (so we can still prove CTL formulae on the quotient TS).

\begin{theorem}
    Theorem: The bisimulation quotient of a finite transition system TS = (S, Act, →, I, AP, L) can be computed in time O (|S|·|AP| + M · log |S|)---where M denotes the number of edges in the state graph
\end{theorem}
\section{Equivalence}

In the context of TS, trace equivalence and bisimulation are two ways to compare systems based on their behaviors and properties.

\subsection{Trace equivalence}
Two TS are considered trace equivalent with respect to a set of atomic propositions if they generate the same set of traces over $\text{AP}$. 
This is denoted as:
\[\text{traces}_{\text{AP}}(\text{TS}_1)=\text{traces}_{\text{AP}}(\text{TS}_2)\]
\noindent In simple terms, two systems are trace equivalent if they can produce the same sequences of observable actions (traces), even if their internal structures differ.
\begin{theorem}
    For a linear-time property $P$ over $\text{AP}$, if $\text{traces}_{\text{AP}}(\text{TS}_1)\subseteq\text{traces}_{\text{AP}}(\text{TS}_2)$, then: 
    \[\text{TS}_1\models P \implies \text{TS}_2\models P\]
\end{theorem}
\begin{corollary}
    Trace-equivalent TS satisfy the same linear-time properties.
\end{corollary}
\noindent Trace equivalence may not always be sufficient for reactive or concurrent systems, as it ignores the internal branching structure of the system. 
While this is useful for parsers and compilers where we care about the observable behavior (language equivalence), reactive systems may require a more nuanced comparison.

\subsection{Bisimulation}
Bisimulation provides a more detailed notion of equivalence, focusing on the internal structure of the systems.
A $\text{TS}_2$  is said to simulate $\text{TS}_1$ if every transition in $\text{TS}_1$ can be matched by one or more transitions in $\text{TS}_2$.
\begin{definition}[\textit{Bisimulation equivalent}]
    Let $\text{TS}_1=\left\langle S_1,\text{Act}_1,\rightarrow_1,I_1,\text{AP},L_1\right\rangle$ and \\ 
    $\text{TS}_2=\left\langle S_2,\text{Act}_2,\rightarrow_2,I_2,\text{AP},L_2\right\rangle$ be two TS.
    A bisimulation between them is a binary relation $\mathcal{R}\subseteq S_1\times S_2$ satisfying:
    \begin{enumerate}
        \item For each initial state $s_1\in I_1$, there exists an initial state $s_2\in I_2$ such that $(s_1,s_2)\in \mathcal{R}$, and viceversa. 
        \item For all pairs $(s_1,s_2)\in \mathcal{R}$, the following conditions hold: 
            \begin{enumerate}
                \item $L_1(s_1)=L_2(s_2)$ (the labeling of the states must be identical). 
                \item If $s_1^\prime\in\text{post}(s_1)$, then there exists $s_2^\prime\in\text{post}(s_2)$ such that $(s_1^\prime,s_2\prime)\in\mathcal{R}$. 
                \item If $s_2^\prime\in\text{post}(s_2)$, then there exists $s_1^\prime\in\text{post}(s_1)$ such that $(s_1^\prime,s_2\prime)\in\mathcal{R}$. 
            \end{enumerate}
    \end{enumerate}
    If such a bisimulation exists, we say that $\text{TS}_1$ and $\text{TS}_2$ are bisimulation equivalent ($\text{TS}_1\sim \text{TS}_2)$. 
\end{definition}

\paragraph*{Properties}
The key properties of bisimulation equivalence are: 
\begin{itemize}
    \item \textit{Bisimulation implies trace equivalence}: if two systems are bisimulation equivalent, they are also trace equivalent.
    \item \textit{Symmetry, transitivity, and reflexivity}: the bisimulation relation $\sim$ is an equivalence relation, meaning it satisfies these properties.
    \item \textit{Minimization}: bisimulation allows us to minimize the number of states needed to represent a system while preserving its behavior. 
        If $\text{TS}_1\sim\text{TS}_2$, and $\text{TS}_2$ is smaller than $\text{TS}_1$, we can verify properties on the reduced system $\text{TS}_2$ rather than the larger system $\text{TS}_1$, which can be especially useful for infinite-state systems.
\end{itemize}

\paragraph*{Bisimulation relation}
Bisimulation can be considered as a relation between states of a TS. 
The goal is to minimize the number of states required to prove a certain property.
\begin{definition}[\textit{Bisimulation equivalent}]
    Let $\text{TS}=\left\langle S,\text{Act},\rightarrow,I,\text{AP},L\right\rangle$ be a TS. 
    A bisimulation for $\text{TS}$ is a binary relation on $\mathcal{R}\subseteq S\times S$ such that:
    \begin{enumerate}
        \item $L_1(s_1)=L_2(s_2)$ for all $(\text{TS}_1,\text{TS}_2)\in\mathcal{R}$: 
        \item If $s_1^\prime\in\text{post}(s_1)$, then there exists $s_2^\prime\in\text{post}(s_2)$ such that $(s_1^\prime,s_2\prime)\in\mathcal{R}$. 
        \item If $s_2^\prime\in\text{post}(s_2)$, then there exists $s_1^\prime\in\text{post}(s_1)$ such that $(s_1^\prime,s_2\prime)\in\mathcal{R}$. 
    \end{enumerate}
    States $s_1$ and $s_2$ are said to be bisimulation equivalent (denoted$s_1\sim_{\text{TS}}s_2$) if there exists a bisimulation $\mathcal{R}$ for $\text{TS}$ such that $(\text{TS}_1,\text{TS}_2)\in\mathcal{R}$. 
\end{definition}
\noindent The properties of the bisimulation relation are: 
\begin{enumerate}
    \item $\sim_{\text{TS}}$ is an equivalence relation on the set of states $S$.
    \item $\sim_{\text{TS}}$ is a bisimulation for the TS. 
    \item $\sim_{\text{TS}}$ is the coarsest bisimulation for $\text{TS}$. 
\end{enumerate}

\subsection{Quotient Transition System}
A quotient TS is defined using an equivalence relation. 
Specifically, if we have a bisimulation relation $\sim_{\text{TS}}$ on a TS, we can define a quotient system $\text{TS}\setminus\sim_{\text{TS}}$, which groups states that are bisimulation equivalent into equivalence classes.
\begin{definition}[\textit{Quotient transition}]
    For TS $\left\langle S,\text{Act},\rightarrow,I,\text{AP},L\right\rangle$ and a bisimulation $\sim_{\text{TS}}$, the quotient TS $\text{TS}\setminus\sim_{\text{TS}}$ is defined as:
    \[\text{TS}\setminus\sim_{\text{TS}}=\left\langle S\setminus\sim_{\text{TS}},\left\{\tau\right\},\rightarrow^\prime,I^\prime,\text{AP},L^\prime\right\rangle\]
    Here, $I^\prime=\left\{[s]_{\sim}\mid s \in I\right\}$ is the set of equivalence classes of initial state, $\rightarrow^\prime$ is defined by $\frac{s\xrightarrow{\alpha}s^\prime}{[s]_{\sim}\xrightarrow{\tau}[s^\prime]_{\sim}}$, and $L^\prime([s]_{\sim})=L(s)$.
\end{definition}
\begin{theorem}
    For any TS, it holds that $\text{TS}\sim\text{TS}\setminus\sim$. 
\end{theorem}
\noindent This means that we can prove properties in the quotient system rather than the original one.

\subsection{Computation Tree Logic equivalence}
CTL equivalence is a way to compare TS based on the set of CTL formulas they satisfy. 
States in a TS are CTL-equivalent if they satisfy the same CTL formulas. 

\begin{definition}[\textit{CTL state equivalence}]
    States $s_1$ and $s_2$ in a TS CTL-equivalent, denoted $s_1 \equiv_{\text{CTL}} s_2$, if for all CTL formulas $\phi$ over $\text{AP}$, $s_1 \models \phi$ if and only if $s_2 \models \phi$.
\end{definition}
\begin{definition}[\textit{CTL system equivalence}]
    TS $\text{TS}_1$ and $\text{TS}_2$ are CTL-equivalent, denoted $s_1 \equiv_{\text{CTL}} s_2$, if for all CTL formulas $\phi$, $\text{TS}_1 \models \phi$ if and only if $\text{TS}_2 \models \phi$.
\end{definition}

\noindent For finite TS without terminal states, bisimulation equivalence and CTL equivalence are equivalent:
\[\sim_{\text{TS}}=\equiv_{\text{CTL}}\]
\noindent For infinite-state systems, bisimulation equivalence implies CTL equivalence, allowing us to prove CTL properties on the quotient system.
\begin{theorem}
    The bisimulation quotient of a finite TS can be computed in time:
    \[\mathcal{O} \left(\left\lvert S\right\rvert \cdot \left\lvert \text{AP}\right\rvert  + M  \log \left\lvert S\right\rvert \right)\] 
    Here, $M$ denotes the number of edges in the state graph.
\end{theorem}
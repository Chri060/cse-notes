\section{Simulation and abstraction}

Abstraction and Refinement
• Transition systems can model software or hardware at various abstraction levels.
– The lower the abstraction level, the more implementation details are present; at high
abstraction levels, such details are deliberately left unspecified.
• As in the Bakery example, we may start from a «detailed» TS and want to define
a suitable more abstract TS
– possibily preserving properties of interest, but easier to manage.
– This process is called abstraction
• We may instead start from a more abstract model and want to add more
implementation details
– possibily preserving properties of interest, but closer to real system
– This process is called refinement
\begin{definition}[\textit{Implementation relation}]
    An implementation relation is a binary relation between two TS at different
abstraction levels. 
\end{definition}
\noindent When two TS,TS’ are related by an implementation relation, one model is said to
be refined by the other; the second is said to be an abstraction of the first.
• If the implementation relation is an equivalence, then the two TS cannot be
distinguished (same observable properties at the relevant abstraction level).
• Since it is possible to define many different implementation relations, there are
different concepts of abstraction and refinement

The fixed set AP plays a crucial role in comparing transition systems using
bisimulation for checking the implementation relation.
• The set AP used in a bisimulation stands for the set of all “relevant” atomic
propositions.
– All other atomic propositions are understood as negligible and are ignored in the comparison.
• If TS is a refinement of TS' (e.g., it incorporates some implementation details),
then usually the set AP of TS is a proper superset of the set AP' of TS' .
– To compare TS and TS’, the set of common atomic propositions, AP’, is a reasonable choice.
– In this way, it is possible to check whether the branching structure of TS agrees with that of
TS’ when considering all observable information in AP’ .
• For checking the equivalence of TS and TS' wrt the satisfaction of a temporal
logic formula \phi, it suffices to consider as AP the atomic propositions of \phi.
Bisimulation relations are equivalences requiring two bisimilar states to exhibit
identical stepwise behavior.
• Simulation relations only require that whenever a state s’ simulates state s, then
s’ can mimic all stepwise behavior of s, but the reverse is not guaranteed
– s’ may perform transitions that cannot be matched by s.
– hence, every successor of s has a corresponding successor of s’, but the reverse does
not necessarily hold.
• Simulation relations often used for:
– showing that one system correctly implements another, more abstract system.
– finding a smaller abstract model preserving at least some properties of interest

\begin{definition}[\textit{Simulation}]
    Let $\text{TS}_i=\left\langle S_i,\text{Act}_i,\rightarrow_i,I_i,\text{AP},L_i\right\rangle$ with $i=1,2$ be transition systems over $\text{AP}$. 
    A simulation for $(\text{TS}_1,\text{TS}_2)$ is a binary relation $\mathcal{R}\subseteq S_1\times S_2$ such that: 
    \begin{enumerate}
        \item $\forall s_1\in I_1 (\exists s_2\in I_2 (s_1,s_2)\in \mathcal{R})$. 
        \item For all $(s_1,s_2)\in \mathcal{R}$ it holds: 
            \begin{enumerate}
                \item $L_1(s_1)=L_2(s_2)$
                \item If $s_1^\prime\in\text{post}(s_1)$ then there exists $s_2^\prime\in\text{post}(s_2)$ with $(s_1^\prime,s_2\prime)\in\mathcal{R}$. 
            \end{enumerate}
    \end{enumerate}
    \text{TS}_1 si simulated by \text{TS}_2 denoted \text{TS}_1 \preceq \text{TS}_2 if there exists a bisimulation $\mathcal{R}$ for $(\text{TS}_1,\text{TS}_2)$. 
\end{definition}
The first condition requires that all initial states in TS1 are related to an initial state of TS2 (but, there might be initial states
of TS2 that are not matched by an initial state of TS1 ).
The secobd that are as for bisimulations, but the symmetric counterpart of (B.2) is not required.

\subsection{Abstraction}
Refinement and abstraction vs. simulations
1. If TS1 is obtained from TS2 by deleting transitions of TS2 (e.g., replacing nondeterministic
choices in TS2 with only one alternative) then TS1 is simulated by TS2
– TS1 is thus a refinement of TS2, since TS1 resolves some nondeterminism in TS2.
2. If TS2 is obtained from TS1 with some “abstraction” then TS1 is simulated by TS2.
• Need to define that T2 is an abstraction of TS1 if:
– There is a common set AP of atomic propositions.
– States of TS2 are a set of “abstract states”.
– There is an abstraction function f associating each (concrete) state s of TS1 with the “abstract state” f(s) of
TS2 (and respecting the label in AP)
• Abstractions differ in the choice of the abstract states, the abstraction function f, and the
relevant propositions AP. 
TS2 is an abstraction of TS1 if:
– There is a common set AP of atomic propositions.
– States of TS2 are a set of “abstract states”.
– There is an abstraction function f associating each (concrete) state s of TS1 with the
“abstract state” f(s) of TS2 (and respecting the label in AP)
• Abstractions differ in the choice of the abstract states, the abstraction
function f, and the relevant propositions AP. 

\subsection{Safety property}
Simulation Preserves Safety Properties
\begin{property}
    Let $P_{\text{safe}}$ be a safety LT property and $\text{TS}_1$ and $\text{TS}_2$ transition systems (all over AP), then: 
    \[\text{TS}_1\preceq \text{TS}_2\land\text{TS}_2\models P_{\text{safe}}\implies\text{TS}_1\models P_{\text{safe}} \]
\end{property}
Simulation is not an equivalence, so if TS2 is not «safe» then still TS1 might be.
• Safety properties are preserved because finite paths fragments are preserved.

\paragraph*{Path fragments}
A finite path fragment p1 from s1 is simulated by a path fragment p2 from s2;
However, if p1 ends in a terminal state and p2 does not, then p1 is a path from s2 but p2 is NOT a path from s2.
• Simulation preserves the set of all finite path fragments (from initial states), but not the set of paths ending in a
terminal state
• Finite traces are defined as the traces corresponding to finite paths fragments: finite traces are preserved as well

\paragraph*{No terminal states}
Terminal states in the simulated program TS1 are the problem. If TS1 has no
terminal states trace inclusion is preserved for all traces
As a corollary, for transition systems without terminal states, simulation
preserves all LT properties and not just the safety properties.

Many abstraction techniques (e.g. SLAM, CBMC) based on building a P2
simulating P1, and then only verifying P2
• Refinement approaches build and verify P2 (i.e., a model), before implementing
an actual program P1
• For sequential, non-reactive, programs we are only interested in safety properties
– If implementation P1 is simulated by a more abstract P2 then safety properties proved on
P2 («for the same AP») are preserved in P1
• For concurrent, reactive program we are also interested in liveness property (e.g.,
no deadlock, no starvation)
– Typically we can consider the program P1 as nonterminating
– Liveness properties proved on P2 are then preserved in P1 (trace inclusion!) 

Simulation relation is transitive and reflexive but not symmetric
– if TS1 simulated by TS2, TS1\leq TS2, then it may not be the case that TS2\leq TS1 the
vice versa)
• It may however happen then TS2 can actually be simulated by TS1, so
TS2\leq TS1.
• TS1 and TS2 are simulation equivalent, TS1\cong TS2, if both TS1\leq TS2 and
TS2\leq TS1.
– Advantage: TS1 and TS2 verify the same safety properties

As it is the case for bisimulation, a simulation relation R can be defined on the
states a single transition system TS.
– Details are obivious
• State s1 of TS is simulated by s2, s1\leq_{\text{TS}}s2, if there is a simulation R for a TS
such (s1,s2) \in R.
• States s1 and s2 are simulation-equivalent, s1\cong TS s2, if s1\leq_{\text{TS}}s2 and s2\leq_{\text{TS}}s1
– NB: The simulation relation may be different in the two cases

\subsection{Simulation quotient}
As we did for bisimulation, given a transition system TS and a simulation
equivalence \cong_{\text{TS}} on states we can define a «quotient» transition system TS_{\setminus\cong}
• For any transition system TS it holds that TS \cong_{\text{TS}} TS_{\setminus\cong}
• We can then reduce the size of a system using this quotient.
• But what is the relation with bisimulation equivalence?

Bisimulation is Strictly Finer than Simulation

AP-determinismè simulation equiv. = bisim. equiv.
\begin{definition}[\textit{AP determinism}]
    A transition system $\text{TS}=\left\langle S,\text{Act},\rightarrow,I,\text{AP},L\right\rangle$ is AP-deterministic uf: 
    \begin{enumerate}
        \item for $A\subseteq\text{AP}$ we have that $\left\lvert I\cap\left\{s\mid L(s)=A\right\}\right\rvert\leq 1$
        \item For $s\in S$ we have that if $s\overset{\alpha}{\rightarrow}s^\prime$ and $s\overset{\alpha}{\rightarrow}s^{\prime\prime}$ and $L(s^\prime)=L(s^{\prime\prime})$, then$s^\prime=s^{\prime\prime}$.
    \end{enumerate}
\end{definition}
\begin{theorem}
    IF TS1 amd TS2 are AP-determinstic, Then TS1\sim TS2 if and only if TS1\simeq  TS2
\end{theorem}

\subsection{Action based bisimulation}
Our definition considered only state labels and ignored actions!
• This is consistent with our interest in model checking, where labels of transition are
irrelevant.
– For instance, in a temporal logic formula we only talk of set AP.
– An external event triggering a transition can be «stored» in the state as part of the state label.
– Actions are used mainly for communicating processes
• It is possible to define an «action-based» bisimulation.
• The two notions are of course strictly related.
• Action-based bsimulations are studied extensively in concurrency theory («process
algebras» such as CSP, labeled transition systems).
\begin{definition}[\textit{Action based bisimulation equivalence}]
    
\end{definition}
All results and concepts presented for ∼ can be adapted for ∼Act in a straightforward manner. For instance,
∼Act is an equivalence and can be adapted to an equivalence ∼Act
TS also for the states of a single transition
system TS.
Definition can be used also for defining bisim. for NFA (adding condition that final states are not equivalent
to nonfinal states).




TABLE to be put Here
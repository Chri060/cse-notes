\section{Concurrent systems}

When transitioning from sequential to concurrent or parallel systems, fundamental shifts occur in how we define and model computation:
\begin{itemize}
    \item Usually, the traditional problem formulation changes significantly.
    \item The rise of networked and interactive systems demands new models focused on interactions rather than just algorithmic transformations.
    \item Many modern systems do not have a clear beginning and end but instead involve continuous, ongoing computations.
        This requires us to consider infinite sequences (infinite words), leading to a whole branch of formal language theory designed for such systems.
    \item We must account for interleaved signals flowing through different channels.
\end{itemize}
\begin{definition}[\textit{System}]
    A system is a collection of abstract machines, often referred to as processes.
\end{definition}
In some cases, we can construct a global state by combining the local states of individual processes. 
However, with concurrent systems, this is often inconvenient or even impossible:
\begin{itemize}
    \item Each process evolves independently, synchronizing only occasionally.
    \item Asynchronous systems do not have a globally synchronized state.
    \item Finite State Machines capture interleaving semantics but differ fundamentally from asynchronous models.
\end{itemize}
\noindent In distributed systems, components are physically separated and communicate via signals.
As system components operate at speeds approaching the speed of light, it becomes meaningless to assume a well-defined global state at any given moment.

\subsection{Time formalization}
When time becomes a factor in computation, things become significantly more complex. 
Unlike traditional engineering disciplines computer science often abstracts away from time, treating it separately in areas like complexity analysis and performance evaluation.

While this abstraction is sufficient for many applications, it is inadequate for real-time systems, where correctness explicitly depends on time behavior. 
In such systems, we must consider:
\begin{enumerate}
    \item The occurrence and order of events.
    \item The duration of actions and states.
    \item Interdependencies between time and data.
\end{enumerate}
Over the years, time has been integrated into formal models in various ways.

\paragraph*{Operational formalism}
These approaches incorporate time directly into system execution models: timed transitions, timed Petri networks, and time as a system variable. 

\paragraph*{Descriptive formalism}
These approaches focus on reasoning about time without explicitly simulating execution: temporal logic (treats time as an abstract concept, focusing on event ordering rather than durations), and metric temporal logics (extensions of temporal logic introduce time constraints).
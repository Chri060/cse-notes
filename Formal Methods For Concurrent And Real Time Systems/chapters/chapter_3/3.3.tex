\section{Temporal logic}

Temporal logic is widely used to specify and verify properties of Transition Systems. 
It allows expressing conditions over time, ensuring that a system behaves as expected in different scenarios.

\paragraph*{Taxonomy}
Temporal logics can be classified based on different criteria:
\begin{itemize}
    \item \textit{Time representation}: discrete-time or ontinuous-time.
    \item \textit{Metric constraints}: metric or non-metric.
    \item \textit{Computation structure}: linear or branching.
\end{itemize}
\noindent Among these, two primary temporal logics are commonly used in system verification:
\begin{itemize}
    \item \textit{Linear Temporal Logic}: focuses on linear sequences of states.
    \item \textit{Computation Tree Logic}: explores branching structures of state transitions.
\end{itemize}
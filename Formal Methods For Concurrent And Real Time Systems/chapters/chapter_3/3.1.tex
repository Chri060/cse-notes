\section{Safety property}

Safety properties ensure that a system never reaches an undesirable state. 
Regular safety properties can be characterized using a nondeterministic finite automaton that recognizes finite words over the power set of atomic propositions, denoted as $\left(2^{\text{AP}}\right)^\ast$.

\begin{definition}[\textit{Safety property model checking}]
    Given a regular safety property $P_{\text{safe}}$ over the atomic propositions $\text{AP}$ and a finite Transition System $\text{TS}$ (without terminal states), model checking verifies whether:
    \[\text{TS}\models P_{\text{safe}}\]
\end{definition}
\noindent 
To achieve this, we use an nondeterministic finite automaton $\mathcal{A}$ that recognizes the minimal bad prefixes of $P_{\text{safe}}$.
This allows us to define an invariant property:
\[P_{\text{inv}(\mathcal{A})}=\bigwedge_{q\in\mathcal{F}}\lnot q\]
Here, $\mathcal{A}$ must not reach a final state.

\subsection{Invariant checking}
Verification of the safety property can be reduced to checking an invariant by following these steps:
\begin{enumerate}
    \item Construct the product of the Transition System and the nondeterministic finite automaton $\text{TS}\otimes \mathcal{A}$. 
        This operation is similar to the synchronous composition of two Nondeterministic Finite Automata.
    \item The following conditions are equivalent:
        \begin{itemize}
            \item The Transition System satisfies the safety property: 
                \[\text{TS} \models P_{\text{safe}}\]
            \item The set of finite traces of the Transition System does not intersect the language of:
                \[\mathcal{A}: \text{traces}_{\text{fin}}(\text{TS}) \cap L(\mathcal{A}) = \varnothing\]
            \item The product system satisfies the invariant:
                \[\text{TS}\otimes \mathcal{A} \models P_{\text{inv}(\mathcal{A})}\]
        \end{itemize}
\end{enumerate}
\noindent Thus, checking a safety property reduces to verifying an invariant in the product system.

\subsection{Algorithm}
Given a finite Transition System $\text{TS}$ and a regular safety property $P_{\text{safe}}$, the algorithm returns either true ($\text{TS}\models P_{\text{safe}}$) or false ($\text{TS}\not\models P_{\text{safe}}$), with a counterexample. 
\begin{algorithm}[H]
    \caption{Safety property model checking}
        \begin{algorithmic}[1]
            \State Let nondeterministic finite automaton $\mathcal{A}$ (with accept states $F$) be such that $\mathcal{L}(\mathcal{A})$ are the bad prefixes of $P_{\text{safe}}$ 
            \State Construct the product Transition System $\text{TS}\otimes\mathcal{A}$
            \State Check the invariant $P_{\text{inv}(\mathcal{A})}$ with proposition $\lnot F=\wedge_{q \in F}\lnot q$ on $\text{TS}\otimes\mathcal{A}$
            \If{$\text{TS}\otimes\mathcal{A}$}
                \State \Return $\text{true}$
            \Else 
                \State Determine an initial path fragment $\left\langle s_0,q_1\right\rangle,\dots,\left\langle s_n,q_{n+1}\right\rangle$ of $\text{TS}\otimes\mathcal{A}$ with $q_{n+1}\in F$
                \State \Return $(\text{false}, s_0s_1\dots s_n)$
            \EndIf
        \end{algorithmic}
\end{algorithm}
\noindent The time and space complexity of this approach is:
\[\mathcal{O}(\left\lvert\text{TS}\right\rvert\cdot\left\lvert\mathcal{A}\right\rvert)\]
Here, $\left\lvert\text{TS}\right\rvert$ and $\left\lvert\mathcal{A}\right\rvert$ denote the number of states and transitions in the Transition System and the nondeterministic finite automaton, respectively.
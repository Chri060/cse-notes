\section{Symbolic model checking}

When dealing with systems with an extremely large number of states, explicitly representing each state becomes impractical. 
Instead, symbolic model checking reformulates the process by representing entire sets of states and transitions rather than individual elements. 
This shift allows for more efficient computation, as operations on states are replaced by set operations, which can often be computed much more effectively.

A common approach to symbolic model checking relies on boolean encoding of the state space. 
Each state is mapped to a binary representation, where a characteristic function defines any subset of states. 
The transition relation, instead of being stored explicitly, is represented as a boolean function that maps a pair of states to a truth value. 
By encoding these functions efficiently, the complexity of the model-checking procedure can be significantly reduced. 

One of the primary advantages of symbolic model checking is its scalability. 
Systems with an astronomical number of states can still be analyzed using symbolic techniques. 
The use of OBDDs and similar structures enables operations on entire sets of states, rather than iterating through them one by one. 
However, efficiency heavily depends on choosing an appropriate variable ordering, which can greatly impact the size of the OBDD representation. 
Despite these optimizations, the worst-case complexity of model checking remains unchanged, meaning that certain problems remain computationally difficult even with symbolic methods.

Symbolic model checking has been particularly successful in hardware verification, where state spaces tend to be structured in a way that benefits from OBDD-based representations. 
In contrast, software verification often favors Bounded Model Checking, which leverages SAT solvers and can sometimes be more effective in handling complex program structures. 
Tools like Nu-SMV incorporate both symbolic and bounded model checking approaches, making them versatile for different verification tasks. 
While symbolic techniques have led to significant advancements, their effectiveness ultimately depends on the nature of the system being analyzed and the efficiency of the underlying boolean function representation.
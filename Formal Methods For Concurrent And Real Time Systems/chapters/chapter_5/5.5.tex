\section{Timed languages}

A timed $\omega$-word (or simply timed word) is an infinite sequence of states, where each state is associated with a real-valued timestamp.
Formally, a timed word is a sequence:
\[(\sigma_1,\tau_1),(\sigma_2,\tau_2),(\sigma_3,\tau_3),\dots\]
Here, each symbol $\sigma_i$ is a symbol from a finite alphabet, and $\tau_i$ is a real-valued timestamp indicating when $\sigma_i$ occurs.
\begin{definition}[\textit{Timed sequence}]
    A timed sequence is an infinite sequence of time values:
    \[\tau=\tau_1,\tau_2,\tau_3,\dots\]
    Here, each $\tau_i$ satisfies: 
    \begin{itemize}
        \item \textit{Monotonicity}: $\tau_i\leq\tau_{i+1}$ for all $i\geq 1$. 
            If $\tau_i<\tau_{i+1}$ for all $i\geq 1$ the sequence is strictly monotonic.
        \item \textit{Progress}: for every real number $t\in\mathbb{R}_{\geq 0}$, there exists some $i\geq 1$ such that $\tau_i> t$. 
    \end{itemize}
\end{definition}
\begin{definition}[\textit{Timed word}]
    A timed word over an alphabet $\mathcal{A}$ (a finite set of symbols) is a pair:
    \[\sigma,\tau\]
    Here, $\sigma=\sigma_1\sigma_2\sigma_3\dots$ is an $\omega$-word (an infinite sequence of symbols from $\mathcal{A}$), and $\tau$ is a timed sequence providing the timestamps associated with each symbol.
\end{definition}
\begin{definition}[\textit{Timed language}]
    A timed language over an alphabet $\mathcal{A}$ is a set of timed words over $\mathcal{A}$.
\end{definition}

\paragraph*{Untime operation}
The untime operation on a timed language discards the time values associated with symbols.

\begin{definition}
    For a timed language $\mathcal{L}$ over $\mathcal{A}$:
    \[\text{untime}(\mathcal{L}) = \{ \sigma \in \mathcal{A}^\omega \mid \text{there exists a } \tau \text{ such that } (\sigma, \tau) \in \mathcal{L} \}\]
\end{definition}
\noindent $\text{untime}(\mathcal{L})$ is a set of $\omega$-words.

\subsection{Timed Automata}
A timed automato is defined over an alphabet of actions, called the input alphabet. 
It is represented as a tuple:
\[\mathcal{A} = \langle L, \Sigma, C, E, l_0, J, F \rangle\]
\noindent Here:
\begin{itemize}
    \item $L$ is a finite set of locations.
    \item $\Sigma$ is a finite set of input symbols.
    \item $C$ is a finite set of clocks.
    \item $l_0 \in L$ is the initial location.
    \item $E \subseteq L \times B(C) \times \Sigma \times 2^C \times L$ is a set of edges, where $B(C)$ denotes the set of clock constraints.
    \item $J: L \to B(C)$ assigns invariants to locations.
    \item $F \subseteq L$ is the set of final locations.
\end{itemize}
\noindent A timed word is accepted if there exists a path leading infinitely often to a final location.

\begin{definition}[\textit{Run}]
    A run $r_{(\sigma, \tau)}$ over a timed word $(\sigma, \tau)$, where $\sigma = \sigma_1 \sigma_2 \dots$ and $\tau = \tau_1 \tau_2 \dots$ is an infinite sequence:
    \[r_{(\sigma, \tau)} = \langle l_0, \eta_0 \rangle \xrightarrow{\sigma_1, \tau_1} \langle l_1, \eta_1 \rangle \xrightarrow{\sigma_2, \tau_2} \langle l_2, \eta_2 \rangle \dots\]
\end{definition}

Here, $l_i \in L$ and $\eta_i$ are clock valuations satisfying:
\begin{itemize}
    \item \textit{Initialization}: $\eta_0(x) = 0$ for all $x \in C$.
    \item \textit{Transition condition}: for all $i \geq 1$, there exists an edge $e \in E$ such that:
        \[e = \langle l_{i-1}, \gamma_i, \sigma_i, \hat{C}_i, l_i \rangle\]
        Here, $(\eta_{i-1} + (\tau_i - \tau_{i-1}))$ satisfies both $\gamma_i$ and $J(l_{i-1})$, and the clock update is given by:
        \[\eta_i = [\hat{C}_i \mapsto  0](\eta_{i-1} + (\tau_i - \tau_{i-1}))\]
    \item $\eta_i$ satisfies the invariant $J(l_i)$.
\end{itemize}
\noindent The set of locations visited infinitely often in the run is denoted by $\inf(r_{(\sigma, \tau)})$.
A timed word $(\sigma, \tau)$ is accepted by a Timed Automaton $\mathcal{A}$ if:
\[\inf(r(\sigma, \tau)) \cap F \neq \emptyset\]

\subsection{Timed regular languages}
\begin{definition}[\textit{Timed regular language}]
    A timed regular language is a language accepted by some Timed Automaton.
\end{definition}
\noindent If $\mathcal{L} \subseteq \mathcal{A}^\omega$ is an $\omega$-regular language, then $\mathcal{L}_t = \{ (\sigma, \tau) \mid \sigma \in \mathcal{L} \}$ is a timed regular language.

\paragraph*{Untimed languages}
Given a Timed Automata that accepts a language $\mathcal{L}$, there exists a Büchi automaton (BA) that accepts $\text{untime}(\mathcal{L})$.
That is, for any timed regular language $\mathcal{L}$, its untimed version $\text{untime}(\mathcal{L})$ is $\omega$-regular.
However, simply removing timing constraints is insufficient. 


\subsection{Timed Automata properties}
Timed Automata are closed under union and intersection but not under complementation.
Timed Automata cannot be determinized.
Non-closure under complement limits automata-based model checking.
Language inclusion and equivalence are undecidable.
Emptiness is decidable by verifying the emptiness of the Region Automaton (similar to the Region Transition System).

\paragraph*{Continuous time semantics}
Timed Automata semantics can be defined over signals instead of timed words.
The standard Transition System semantics is based on timed words.
Linear Temporal Logics are more naturally interpreted over signals.
Decision problems become harder in continuous-time semantics.

\subsection{Model Checking Timed Automata for Linear Temporal Logic}
Model checking Timed Automata for Linear Temporal Logic (both pointwise and continuous semantics) is PSPACE-complete.
The steps are: 
\begin{enumerate}
    \item Construct a Büchi automaton equivalent to the negation of a formula.
    \item Build the product of this Büchi automaton with the Timed Automata.
    \item Verify the emptiness of the resulting automaton.
\end{enumerate}
\noindent Since Timed Automata are closed under intersection, the result is still a Timed Automata.
However, most metric extensions of Linear Temporal Logic are undecidable.
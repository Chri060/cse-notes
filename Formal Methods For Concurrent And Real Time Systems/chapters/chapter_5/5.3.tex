\section{Timed Automata analysis}

To verify properties of a system modeled as a timed automaton, we need to explore all possible states of the automaton.

A state in a Timed Automata is represented as $\left\langle \text{location},\text{clock evaluation}\right\rangle$.
The set of possible clock evaluations, denoted $\text{eval}(C)$, consists of real-valued clock assignments.
Since clock values come from $\mathbb{R}_{\geq 0}$, this results in an infinite state space.

\subsection{Clock regions}
To make exhaustive exploration feasible, we construct a finite abstraction of the state space: clock regions provide a way to group equivalent clock valuations into a finite number of equivalence classes.
Two clock valuations $\eta_1$ and $\eta_2$ belong to the same region if and only if:
\begin{enumerate}
    \item They agree on the integer parts of all clock values.
        For each  $x\in C$, the integer parts of $\eta_1(x)$ and $\eta_2(x)$ must be identical.
    \item They agree on the ordering of fractional parts of all clock values.
        If two clocks $x$ and $y$ in $\eta_1$ will reach their next integer values in a specific order, they must reach them in the same order in $\eta_2$.
\end{enumerate}
\noindent For each clock $x\in C$, define $x_c$ as the largest integer appearing in constraints: if a constraint $x \leq c$ or $c\leq x$ appears in the system, set $c_x=\max(c)$.

\begin{definition}[\textit{Clock valuation equivalence}]
    Clock valuations $\eta$ and $\eta^\prime$  are equivalent (i.e., belong to the same clock region) if and only if:
    \begin{itemize}
        \item If $\eta(x)>c_x$ and $\eta^\prime(x)>c_x$ for every clock $x$, then they are considered equivalent.
        \item Otherwise, for all $x,y\in C$ with $\eta(x), \eta^\prime(x)\leq c_x$ and $\eta(y), \eta^\prime(y)\leq c_y$: 
            \begin{itemize}
                \item They have the same integer part: 
                    \[\left\lfloor \eta(x)\right\rfloor=\left\lfloor \eta^\prime(x)\right\rfloor\]
                \item Their fractional parts agree in ordering: 
                    \[\text{frac}(\eta(x))\leq\text{frac}(\eta(y))\Leftrightarrow\text{frac}(\eta^\prime(x))\leq\text{frac}(\eta^\prime(y))\]
                \item $\text{frac}(\eta(x))=0$ if and only if $\text{frac}(\eta^\prime(x))=0$
            \end{itemize}
    \end{itemize}
\end{definition}
\begin{definition}[\textit{Clock region}]
    A clock region is defined as an equivalence class of clock evaluations under the relation $\cong$. 
\end{definition}
\noindent Since there are only finitely many ways to assign integer values and compare fractional parts, the number of regions is finite.
The region abstraction enables efficient verification of properties in timed automata by ensuring a finite number of explored states.

\subsection{Region based automata}
By using clock regions, we can construct a finite representation of all possible states in the timed Transition System $\text{TS}(\text{TA})$.
state in $\text{TS}(\text{TA})$ is represented as $\left\langle \ell_i,\right\rangle$ where $\ll_i$ is a location and $\eta_i$ is a clock valuation.
The key idea is to group clock valuations into equivalence classes (regions) to define a finite region-based Transition System.
Since there is a maximum constant in the clock constraints, the number of clock regions is finite.

Thus, for a given timed automaton, we build a region-based Transition System where states are pairs:
\[\left\langle \ell_i,\right\rangle\]

To define transitions in the region-based Transition System, we need to determine time successors of clock regions.
\begin{definition}[\textit{Time successor}]
    A clock region $\alpha^\prime$ is a time-successor of a clock region $\alpha$ if:
    \[\forall\eta\in\alpha\qquad \exists t\in\mathbb{R}_{\geq 0}\mid \eta+t\in\alpha^\prime\]
\end{definition}
\noindent This means that every reachable clock region along a diagonal transition is a time-successor.
If two clock valuations are in the same region, the automaton can take the same transitions.
However, the exact delays might need to be adjusted to ensure they reach corresponding successor regions.

\subsection{Time abstracted bisimulation}
Clock equivalence can be seen as a special case of time-abstracted bisimulation.
\begin{definition}[\textit{Time abstracted bisimulation}]
    A relation $R$ on the states of a timed automata is a time based bisimulation if $(\ell_1,\upsilon_1)R(\ell_2,\upsilon_2)$ and $(\ell_1,\upsilon_1)\xrightarrow{d_1,a}(\ell_1^\prime,\upsilon_1^\prime)$ for some $d_1 \in\mathbb{R}_{\geq 0}$ and $a\in\Sigma$ imply $(\ell_2,\upsilon_2)\xrightarrow{d_2,a}(\ell_2^\prime,\upsilon_2^\prime)$ for some $d_2 \in\mathbb{R}_{\geq 0}$ with $(\ell_1^\prime,\upsilon_1\prime)R(\ell_2\prime,\upsilon_2\prime)$ and viceversa. 
\end{definition}
\begin{theorem}
    Clock equivalence is a bisimulation equivalence.
\end{theorem}
\noindent Clock equivalence defines a bisimulation over atomic propositions. 
Since clock equivalence is a finite-index relation, we obtain a finite quotient Transition System:
\[\text{RTS}(\text{TA})\] 
\noindent From this, we can derive that: 
\begin{enumerate}
    \item Every path in the infinite $\text{TS}(\text{TA})$ has a corresponding path in the finite $\text{RTS}(\text{TA})$.
    \item Every path in $\text{RTS}(\text{TA})$ corresponds to an infinite set of paths in $\text{TS}(\text{TA})$.
    \item Computation Tree Logic-style logical properties (used in model checking) are preserved in $\text{RTS}(\text{TA})$.
\end{enumerate}
\noindent Thus, region-based automata allow efficient analysis of timed systems while maintaining correctness guarantees.
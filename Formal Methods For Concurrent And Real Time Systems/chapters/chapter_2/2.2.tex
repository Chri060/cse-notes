\section{Program graphs}

A common transformation in system modeling is moving external inputs into state labels. 
This approach simplifies definitions and system analysis by leaving only internal communications as actual inputs.

When dealing with variables, TS are referred to as Program Graph (PG). 
A PG consists of:
\begin{itemize}
    \item A set of variables, where each variable has a value assigned in every state by an evaluation function.
    \item Transitions that may include conditions based on variable values.
    \item An effect function, which describes how inputs modify variable values.
    \item States, which are typically called locations in the context of PGs.
\end{itemize}

\paragraph*{Transformation}
PGs can always be converted into a (potentially infinite) TS.
However, TS do not inherently include guards or variables. 
Instead:
\begin{itemize}
    \item Guards can be represented as symbols in a set of atomic propositions.
    \item The atomic proposition set must also include all locations from the PG.
    \item While this transformation results in a very large atomic proposition set, in practice, only a small portion is usually relevant for analyzing system properties.
\end{itemize}
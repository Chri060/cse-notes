\section{Fairness}

To ensure realistic system behavior, fairness constraints must be introduced. 
These constraints prevent unrealistic execution patterns by guaranteeing that processes are given a fair chance to execute. 
Fairness is especially relevant in concurrent systems where multiple processes compete for execution.

Fairness can be classified into three main types:
\begin{itemize}
    \item \textit{Unconditional fairness}: every process gets a chance to execute infinitely often, regardless of other conditions.
    \item \textit{Strong fairness}: if a process is enabled infinitely often, it must eventually execute infinitely often.
    \item \textit{Weak fairness}: if a process remains continuously enabled from a certain point onward, it must eventually execute infinitely often.
\end{itemize}
\noindent These fairness levels follow a logical hierarchy:
\[\text{unconditional fairness}\implies\text{strong fairness}\implies\text{weak fairness}\]

\begin{definition}[\textit{Fairness constraint}]
    A fairness constraint defines a set of actions that must occur under a given fairness assumption (unconditional, strong, or weak).
\end{definition}
\noindent These constraints play a crucial role in ensuring liveness properties, which guarantee that something will eventually happen.
Fairness constraints can be efficiently expressed using Büchi automata or Linear Temporal Logic (LTL). 
However, incorporating fairness into TS requires careful handling to ensure correctness.

\subsection{Fairness formalization}
\begin{definition}[\textit{Fairness}]
    Let $\text{TS}=\left\langle S,\text{Act},\rightarrow,I,\text{AP},L\right\rangle$. 
    The enabled actions at a state $s$ are given by $\text{Act}(s)=\left\{\alpha\in\text{Act}\mid\exists s^\prime \in s, s\xrightarrow{\alpha}s^\prime\right\}$
    For an infinite execution fragment $\rho=s_0\xrightarrow{\alpha_0}s_1\xrightarrow{\alpha_1}\dots$ we define the fairness conditions:
    \begin{itemize}
        \item \textit{Unconditional fairness}: $\rho$ is unconditionally $A$-fair if there exists infinitely many $j$ for all $\alpha_i\in A$. 
        \item \textit{Strong fairness}: $\rho$ is strongly $A$-fair if: 
            \[\left(\exists \text{ infinitely many }j\mid\text{Act}(s_j)\cap A\neq\varnothing\right)\implies\left(\exists \text{ infinitely many }j\mid\alpha_j\in A\right)\]
        \item \textit{Weak fairness}: $\rho$ is weakly $A$-fair if: 
            \[\left(\forall \text{ sufficiently large }j\mid\text{Act}(s_j)\cap A\neq\varnothing\right)\implies\left(\exists \text{ infinitely many }j\mid\alpha_j\in A\right)\]
    \end{itemize}
\end{definition}
\begin{definition}[\textit{Fairness assumption}]
    A fairness assumption $\mathcal{F}$  is defined as a triple: 
    \[\mathcal{F}=\left\langle \mathcal{F}_{\text{uncond}},\mathcal{F}_{\text{strong}},\mathcal{F}_{\text{weak}}\right\rangle\] 
    Here, $\mathcal{F}_{\text{uncond}},\mathcal{F}_{\text{strong}},\mathcal{F}_{\text{weak}}\subseteq 2^{\text{Act}}$. 
\end{definition}
\noindent An execution $\rho$ is $\mathcal{F}$-fair if: 
\begin{enumerate}
    \item It is unconditionally $A$-fair for all $A \in \mathcal{F}_{\text{uncond}}$.
    \item It is strongly $A$-fair for all $A \in \mathcal{F}_{\text{strong}}$.
    \item It is weakly $A$-fair for all $A \in \mathcal{F}_{\text{weak}}$.
\end{enumerate}

\paragraph*{Fair traces}
Traces that satisfy a fairness constraint F are called Fair-Traces.
\begin{definition}[\textit{Fair traces}]
    Let $P$ be a linear time property over $\text{AP}$ and $\mathcal{F}$ a fairness assumption over $\text{Act}$. 
    A TS fairly satisfies $P$, denoted $\text{TS} \models_{\mathcal{F}} P$ if and only if: 
    \[\text{fairTraces}_{\mathcal{F}}(\text{TS})\subseteq P\]

\end{definition}
\noindent This follows the hierarchy: 
\[\text{TS} \models_{\mathcal{F}_{\text{weak}}} P \implies \text{TS} \models_{\mathcal{F}_{\text{strong}}} P \implies \text{TS} \models_{\mathcal{F}_{\text{uncond}}} P\]

\subsection{Fairness and safety}
\begin{definition}[\textit{Realizable fairness assumption}]
    A fairness assumption $\mathcal{F}$ for a TS is called realizable if every reachable state $s$  satisfies:
    \[\text{fairPaths}_\mathcal{F} (s) \neq \varnothing\]
\end{definition}
\begin{theorem}
    Let $\text{TS}$ be a TS with set of propositions $\text{AP}$, $\mathcal{F}$ a realizable fairness assumption, and $P_{\text{safe}}$ a safety property. 
    Then:
    \[\text{TS}\models P_{\text{safe}} \Leftrightarrow \text{TS}\models_{\mathcal{F}} P_{\text{safe}}\]
\end{theorem}
\noindent This theorem establishes that safety properties are independent of fairness assumptions, meaning that if a system satisfies a safety property, it does so regardless of fairness constraints.
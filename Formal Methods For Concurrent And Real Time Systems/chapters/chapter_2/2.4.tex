\section{Nano Promela}

TS provide a mathematical foundation for modeling and verifying reactive systems. 
However, in practice, we need more user-friendly specification languages.

One such language is Promela, designed for the SPIN model checker to describe TS. 
We will focus on a simplified subset of Promela called Nano-Promela.

\subsection{Syntax}
A Promela program consists of a set of interleaving processes that communicate either synchronously or through finite first in first out channels.
The syntax of statements in Nano-Promela is as follows:
\begin{verbatim}
stmt ::= skip | x := expr | c?x | c!expr |
         stmt1; stmt2 | atomic{assignments} |
         if :: g1 => stmt1 ... :: gn => stmtn fi |
         do :: g1 => stmt1 ... :: gn => stmtn do
\end{verbatim}
Here: 
\begin{itemize}
    \item \texttt{expr} represents an expression.
    \item \texttt{skip} represents a process that terminates in one step, without modifying any variables or channels.
    \item \texttt{stmt1; stmt2} denotes sequential execution: \texttt{stmt1} runs first, followed by \texttt{stmt2}.
    \item \texttt{atomic{assignments}} defines an atomic region, meaning \texttt{stmt} executes as a single, indivisible step. 
        This prevents interference from other processes and helps reduce verification complexity by avoiding unnecessary interleaving.
\end{itemize}

\paragraph*{Conditional statement}
The conditional statement is expressed as: 
\begin{verbatim}
if :: g1 => stmt1 ... :: gn => stmtn fi
\end{verbatim}
This represents a nondeterministic choice between multiple guarded statements.
The system chooses one of the \texttt{stmti} for which \texttt{gi} holds in the current state.
The selection and the first execution step are performed atomically, meaning no other process can interfere.
If none of the guards hold, the process blocks.
However, other processes may unblock it by changing shared variables, causing one of the guards to become true.

\paragraph*{Loop}
The loop is expressed as: 
\begin{verbatim}
do :: g1 => stmt1 ... :: gn => stmtn do
\end{verbatim}
This represents a loop that repeatedly executes a nondeterministic choice among the guarded statements.
If a guard \texttt{gi} holds, the corresponding \texttt{stmti} executes.
Unlike \texttt{if-fi}, \texttt{do-od} does not block when all guards fail; instead, the loop simply terminates.

\subsection{Features}
Nano-Promela can be formally defined using PGs, but full Promela provides additional powerful features, including: more complex atomic regions (beyond just assignments), arrays and richer data types, and dynamic process creation.
\section{Concurrency}

Given two TS: 
\[\text{TS}_1=\left\langle S_1, \text{Act}_1, \rightarrow_1, I_1,\text{AP}_1, L_1\right\rangle \qquad \text{TS}_2=\left\langle S_2, \text{Act}_2, \rightarrow_2, I_2,\text{AP}_2, L_2\right\rangle\]    
Their interleaving is defined as: 
\[\text{TS}_1\:|||\:\text{TS}_2=\left\langle S_1\times S_2, \text{Act}_1 \cup \text{Act}_2, \rightarrow_1, I_1\times I_2,\text{AP}_1\cup \text{AP}_2, L\right\rangle\]
Here,  $L(\left\langle s_1,s_2\right\rangle )= L(s_1) \cup L(s_2)$ and the transition relation $\rightarrow$ is:

\[\dfrac{s_1\xrightarrow{\alpha}s_1^\prime}{\left\langle s_1,s_2\right\rangle \xrightarrow{\alpha}\left\langle s_1^\prime,s_2\right\rangle} \land \dfrac{s_2\xrightarrow{\alpha}s_2^\prime}{\left\langle s_1,s_2\right\rangle \xrightarrow{\alpha}\left\langle s_1,s_2^\prime\right\rangle}\]

\noindent In practice, the two TS proceed independently, but only one at a time is considered to be “active”. 
Similar to non-synchronizing threads: all possible interleaving are allowed.

\paragraph*{No shared variables}
When two PGs, $\text{PG}_1$ and $\text{PG}_2$, do not share variables, their interleaving can be naturally defined as:
\[\text{TS}_1(\text{PG}_1)\:|||\: \text{TS}_2(\text{PG}_2)\] 
This straightforward composition allows both TS to operate independently.

\paragraph*{Shared variables} 
If $\text{PG}_1$ and $\text{PG}_2$ share variables, the simple interleaving:
\[\text{TS}_1(\text{PG}_1)\:|||\:\text{TS}_2(\text{PG}_2)\] 
May not be valid, as some locations might become inconsistent. 
This happens because both PGs access shared critical variables, leading to potential conflicts.

\paragraph*{Constraint synchronization}
To ensure consistency, components must coordinate by imposing constraints on shared variables.
Execution progresses only when the conditions are satisfied in both TS. 
This synchronization mechanism ensures that shared variables remain valid across all transitions.

\subsection{Handshaking}
In parallel composition with handshaking, two TS synchronize on a set of shared actions $H$, which is a subset of their common actions:
\[\text{TS}_1 \parallel_H \text{TS}_2\]    
They evolve independently (interleaving) for actions outside $H$.
This is similar to firing a transition in Petri nets.
To synchronize, processes must shake hands, a concept also known as Synchronous Message Passing.

If there are no shared actions $\text{Act}_1\cap \text{Act}_2$, handshaking reduces to standard interleaving:
\[\text{TS}_1\parallel_{\varnothing}\text{TS}_2=\text{TS}_1\:|||\:\text{TS}_2\]

If $H$ includes all common actions, we simply write:
\[\text{TS}_1\parallel\text{TS}_2\]

Given two TS:
\[\text{TS}_1=\left\langle S_1, \text{Act}_1, \rightarrow_1, I_1,\text{AP}_1, L_1\right\rangle \qquad \text{TS}_2=\left\langle S_2, \text{Act}_2, \rightarrow_2, I_2,\text{AP}_2, L_2\right\rangle\]    
Their handshaking synchronization is defined as: 
\[\dfrac{s_1\xrightarrow{\alpha}_1s_1^\prime\land s_2\xrightarrow{\alpha}_2s_2^\prime}{\left\langle s_1,s_2\right\rangle \xrightarrow{\alpha}\left\langle s_1^\prime,s_2^\prime\right\rangle}\]
This means that both systems must simultaneously perform the shared action $\alpha$ to transition together.

\paragraph*{Broadcasting}
If a fixed set of handshake actions $H$ exists such that: 
\[\text{Act}_1\cap \text{Act}_2 \dots \cap \text{Act}_n\]
Then all processes can synchronize on these actions.
In this case, the handshaking operator $\parallel_H$ is associative, meaning we can compose multiple TS as:
\[\text{TS}=\text{TS}_1\parallel_H\text{TS}_2\parallel_H\dots \parallel_H\text{TS}_n\]
This allows for synchronized execution across multiple processes.

\subsection{Channel system}
Handshaking synchronization does not inherently introduce a direction for message exchange. 
In other words, it lacks a cause-effect relationship between components during synchronization.

However, in many real-world scenarios, directionality is natural (one component sends a message, and another receives it). 
To model this, we use first in first out channels, which explicitly define the direction of communication.
Now, transitions in a system include:
\[\rightarrow\subseteq S \times (\text{Act}\cup C! \cup C?) \times S\]
Here, $C!$ represents sending operations, with messages of the form $c!x$ (sending $x$ through channel $c$) and $C?$ represents receiving operations, with messages of the form $c?x$ (receiving $x$ from channel $c$).

\paragraph*{Channel capacity}
The capacity of a first in first out channel determines how many events (messages) can be stored in its buffer at a time: 
\begin{itemize}
    \item If $\text{capacity}(c) = 0$, the sender and receiver must synchronize instantly (just like standard handshaking), but with a different syntax.
    \item If $\text{capacity}(c) > 0$, the sender can execute $c!x$ without waiting for a receiver, as long as the buffer isn't full.
        If the channel is full, the sender is blocked until space becomes available.
        A receiver performing $c?x$ is blocked until $x$ reaches the front of the queue.
\end{itemize}
\section{Gray box system identification}

In gray-box system identification, we begin with a model constructed using a white-box approach, which relies on first principles:
\[\mathcal{S}:\begin{cases}
    x(t+1)=f(x(t),u(t),\vartheta)+v_1(t) \\
    y(t)=h(x(t),\vartheta)+v_2(t)
\end{cases}\]
Here, $f(\cdot)$ and $h(\cdot)$ represent known equations with some parameters unknown, denoted by the parameter vector $\vartheta$ having physical significance.

The objective is to estimate the parameters of $\vartheta$ from a given dataset. This task can be tackled using the Kalman Filter technique along with a state extension trick. 
Consequently, the new system becomes:
\[\mathcal{S}:\begin{cases}
    x(t+1)=f(x(t),u(t),\vartheta(t))+v_1(t) \\
    \vartheta(t+1)=\vartheta(t)+v_{\vartheta}(t) \\
    y(t)=h(x(t),\vartheta(t))+v_2(t)
\end{cases}\]
The extended state vector $x_E(t)$ comprises the previous states along with the value of $\vartheta(t)$: 
\[x_E(t)=\begin{bmatrix} x(t) \\ \vartheta(t) \end{bmatrix}\]

\subsection{The new equation} 
The equation $\vartheta(t+1)=\vartheta(t)+v_{\vartheta}(t)$ is a fictitious equation necessitated by the state extension.
The fundamental dynamical relationship remains $\vartheta(t+1)=\vartheta(t)$, representing the equation of a constant quantity.
This is appropriate since our objective is to estimate a constant parameter.
Notably, this equation represents a trivially unstable system (non-asymptotically stable). 
However, this instability poses no issue as the Kalman Filter is capable of handling non-asymptotically stable systems.

The inclusion of the noise $v_{\vartheta}(t)$ is crucial; without it, the equation becomes steady-state, and the Kalman Filter does not alter the initial condition.
By introducing $v_{\vartheta}(t)$, we instruct the Kalman Filter to iteratively converge to the correct value of $\vartheta$, without overly relying on the initial condition.

\paragraph*{Noise definition}
The primary challenge lies in defining the noise. 
Typically, the following assumption is made:
\[v_\vartheta\sim WN(0,V_\vartheta)\]
This noise is assumed to be independent of other noises.

A common simplifying assumption is often made:
\[V_{\vartheta}=\begin{bmatrix} \lambda_\vartheta^2 & 0 & 0 & 0 \\ 0 & \lambda_\vartheta^2 & 0 & 0 \\ \vdots &  \vdots & \ddots & \vdots \\ 0 & 0 & 0 & \lambda_\vartheta^2 \end{bmatrix}\]
In practice, all noise characteristics are condensed into a single parameter $\lambda_\vartheta^2$. 
This parameter is empirically tuned, serving as a tuning parameter.
A large value of  $\lambda_\vartheta^2$ yields rapid convergence but substantial post-convergence oscillation, while a small value results in slower convergence but minimal post-convergence oscillation.

This approach is particularly useful when $\vartheta$ varies, as the algorithm remains active. 
Applying the Kalman Filter to the extended system facilitates simultaneous estimation of the state vector (software sensing) and the unknown parameters (system identification).
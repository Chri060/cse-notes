\section{Introduction}

In mathematical analysis and optimization, understanding the fundamental properties of sets in Euclidean space is crucial. 
This section introduces key definitions and properties that form the foundation of nonlinear optimization.

\subsection{Definitions}
Let $S \subseteq \mathbb{R}^n$ be a subset of Euclidean space.
\begin{definition}[\textit{Interior point}]
    A point $\mathbf{x}\in S \subseteq\mathbb{R}^n$ is called an interior point of $S$ if there exists $\varepsilon>0$ such that the open ball:
    \[B_{\varepsilon}(\mathbf{x})=\left\{\mathbf{y}\in\mathbb{R}^n\mid\left\lVert \mathbf{y}-\mathbf{x}\right\rVert <\varepsilon\right\}\subseteq S\]
\end{definition}

\begin{definition}[\textit{Boundary point}]
    A point $\mathbf{x}\in\mathbb{R}^n$ is a boundary point of $S$ if, for every $\varepsilon > 0$, the open ball $B_{\varepsilon}(\mathbf{x})$ contains at least one point of $S$ and at least one point of $\mathbb{R}^n\setminus S$.
\end{definition}

\begin{definition}[\textit{Set interior}]
    The interior of $S$, denoted $\text{int}(S)$, is the set of all interior points of $S$.
\end{definition}

\begin{definition}[\textit{Set boundary}]
    The boundary of $S$, denoted $\partial S$, is the set of all boundary points of $S$.
\end{definition}

\begin{definition}[\textit{Open set}]
    A set $S \subseteq \mathbb{R}^n$ is said to be open if $S = \text{int}(S)$.
\end{definition}

\begin{definition}[\textit{Closed set}]
    A set $S \subseteq \mathbb{R}^n$ is said to be closed if its complement $\mathbb{R}^n \setminus S$ is open.
\end{definition}
\noindent Intuitively, a closed set contains all its boundary points.

\begin{definition}[\textit{Bounded set}]
    A set $S \subseteq \mathbb{R}^n$ is bounded if there exists $M > 0$ such that $\left\lVert\mathbf{x}\right\rVert \leq M$ for all $\mathbf{x} \in S$.
\end{definition}

\begin{definition}[\textit{Compact set}]
    A set is compact if it is both closed and bounded.
\end{definition}

\subsection{Properties}
\begin{property} 
    A set $S \subseteq \mathbb{R}^n$ is closed if and only if every convergent sequence $\{\mathbf{x}_i\}_{i\in\mathbb{N}} \subseteq S$ has its limit in $\mathbf{x} \in S$.
\end{property}

\begin{property} 
    A set $S \subseteq \mathbb{R}^n$ is compact if and only if every sequence $\{\mathbf{x}_i\}_{i\in\mathbb{N}} \subseteq S$ has a convergent subsequence with its limit in $\mathbf{x} \in S$.
\end{property}

\subsection{Optimal solutions existence}
In general, for a function $f : S \subseteq \mathbb{R}^n \rightarrow \mathbb{R}$ we can always determine a greatest lower bound (infimum) given by:
\[\inf_{\mathbf{x}\in S}f (\mathbf{x})\]
However, achieving the minimum value within $S$ depends on additional conditions.
\begin{theorem}
    Let $S \subseteq \mathbb{R}^n$ be a nonempty and compact set, and let $f : S \rightarrow \mathbb{R}$ be continuous. 
    Then there exists $\mathbf{x}^\ast \in S$ such that: 
    \[f (\mathbf{x}^\ast) \leq f (\mathbf{x}) \qquad\forall \mathbf{x} \in S\]
\end{theorem}
\noindent This result does not hold if $S$ is not closed, $S$ is not bounded, or $f(\mathbf{x})$ is not continuous on $S$. 
When a minimizing point $\mathbf{x}^\ast \in S$ exists, we denote it as $\min_{\mathbf{x} \in S} f(\mathbf{x})$.

\begin{definition}[\textit{Cone}]
    Given a set $S\subset \mathbb{R}^n$, the conic hull of $S$, denoted as $\text{cone}(S)$, is the set of all conic combinations of points in $S$, i.e., all points of the form: 
    \[\mathbf{x} = \sum_{i=1}^{m} \alpha_i \mathbf{x}_i\qquad \forall i, 1 \leq i \leq m\]
    Here, $\mathbf{x}_1, \dots, \mathbf{x}_m \in S$, and $\alpha_i \geq 0$.
\end{definition}

\begin{definition}[\textit{Affine subspace}]
    The affine hull, denoted as $\text{aff}(S)$, is the smallest affine subspace containing $S$.
\end{definition}
It coincides with the set of all affine combinations of points in $S$, i.e., all points of the form: 
\[\mathbf{x} = \sum_{i=1}^{m} \alpha_i \mathbf{x}_i\qquad\forall i, 1 \leq i \leq m\]  
Here, $\mathbf{x}_1, \dots, \mathbf{x}_m \in S$, $\sum_{i=1}^{m} \alpha_i = 1$, and $\alpha_i \in \mathbb{R}$.
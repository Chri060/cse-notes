\section{Convex analysis}

\begin{definition}[\textit{Convex set}]
    A set $C\subset\mathbb{R}^n$ is convex if, for any two points $\mathbf{x}_1, \mathbf{x}_2 \in C$ and any $\alpha\in [0, 1]$, the following condition holds:
    \[\alpha \mathbf{x}_1 + (1 - \alpha )\mathbf{x}_2 \in C\]
\end{definition}
\noindent This means that for any two points in the set, the entire line segment connecting them also lies within the set.

\begin{definition}[\textit{Convex combination}]
    A point $\mathbf{x}\in\mathbb{R}^n$ is a convex combination of points $\mathbf{x}_1,\dots,\mathbf{x}_m\in\mathbb{R}^n$ if:
    \[\mathbf{x}=\sum_{i=1}^{m}\alpha_i\mathbf{x}_i\qquad\forall i, 1\leq i \leq m\]
    Here, the coefficients satisfy: $\sum_{i=1}^m\alpha_i=1$, and $\alpha_1\geq 0$.
\end{definition}
\noindent In other words, $\mathbf{x}$ is a weighted sum of the given points, with non-negative weights that sum to one.

\begin{property}
    If $C_1,C-2,\dots,C_k$ are convex sets, then their intersection $\bigcap_{i=1}^{k}C_i$ is also convex.
\end{property}

\begin{definition}[\textit{Polyedron}]
    A polyhedron is the intersection of a finite number of closed half-spaces.
\end{definition}
\noindent In the context of linear programming, the set of all optimal solutions forms a polyhedron.

\begin{definition}[\textit{Convex hull}]
    The convex hull of a set $S \subseteq \mathbb{R}^n$, denoted by $\text{conv}(S)$, is the smallest convex set that contains $S$.
\end{definition}
\noindent Alternatively, it can be characterized as the set of all convex combinations of points in $S$.

\begin{definition}[\textit{Extreme point}]
    A point $x \in  C$ is an extreme point of a convex set $C$ if it cannot be written as a convex combination of two other distinct points in $C$.
    That is, if:
    \[\mathbf{x} = \alpha \mathbf{x}_1 + (1 - \alpha )\mathbf{x}_2\]
    Here, $\mathbf{x}_1,\mathbf{x}_2\in  C$ and $\alpha\in (0, 1)$.
\end{definition}
\noindent Then it must be that $\mathbf{x}_1 = \mathbf{x}_2$, meaning $\mathbf{x}$ is not an interior point of any segment within $C$.

\begin{lemma}
    Let $C \subseteq \mathbb{R}^n$ be a nonempty, closed, and convex set.
    Then, for any point $\mathbf{y} \notin C$, there exists a unique point $\mathbf{x}^\prime\in  C$ that is closest to $\mathbf{y}$.
    Moreover, $\mathbf{x}^\prime$ is the closest point if and only if:
    \[(\mathbf{y}- \mathbf{x}^\prime)^T(\mathbf{x} - \mathbf{x}^\prime) \leq 0 \qquad \forall \mathbf{x} \in  C\]
\end{lemma}

\begin{definition}[\textit{Projection}]
    The point $\mathbf{x}^\prime$  is called the projection of $\mathbf{y}$ onto $C$.
\end{definition}

\begin{definition}[\textit{Supporting hyperplane}]
    Let $S\subset\mathbb{R}^n$ be a nonempty set, and let $\overline{\mathbf{x}}\in\partial(S)$  (the boundary relative to the affine hull of $S$). 
    A supporting hyperplane of $S$ at $\overline{\mathbf{x}}$ is defined as:
    \[H=\{\mathbf{x}\in\mathbb{R}^N\mid\mathbf{p}^T(\mathbf{x}-\overline{\mathbf{x}})=0\}\]
    Such that either $S\subseteq H^{-}$ or $S\subseteq H^{+}$.  
\end{definition}

\subsection{Separation theorem}
\begin{theorem}
    Let $C \subset\mathbb{R}^n$ be a nonempty, closed, and convex set, and let $\mathbf{y} \notin  C$. 
    Then, there exists a vector $\mathbf{p} \in \mathbb{R}^n$ such that:
    \[\mathbf{p}^T\mathbf{x} < \mathbf{p}^T\mathbf{y} \qquad\forall\mathbf{x} \in  C\]
    This means that there exists a hyperplane: 
    \[H = \{\mathbf{x} \in  \mathbb{R}^n\mid \mathbf{p}^T\mathbf{x} = \beta\}\] 
    With $\mathbf{p} \neq 0$ that separates $\mathbf{y}$ from $C$. 
    That is:
    \[C \subseteq H^{-} = \left\{\mathbf{x} \in  \mathbb{R}^n\mid \mathbf{p}^T\mathbf{x} \leq \beta\right\} \land \mathbf{y} \notin  H^{-}\]
    Here, the condition $\mathbf{y} \notin  H^{-}$ imples that $\mathbf{p}^T\mathbf{y} > \beta$. 
\end{theorem}
\begin{proof}
    By the previous lemma, we know that for the closest point $\mathbf{x}^\prime\in C$ to $\mathbf{y}$, we have:
    \[(\mathbf{y} - \mathbf{x}^\prime)^T(\mathbf{x} - \mathbf{x}^\prime) \leq 0\qquad \forall \mathbf{x} \in  C\]
    Defining $\mathbf{p} = (\mathbf{y}-\mathbf{x}^\prime) \neq 0$  and setting $\beta = (\mathbf{y}- \mathbf{x}^\prime)^T\mathbf{x}^\prime$ we obtain: 
    \[\mathbf{p}^T\mathbf{x}\leq\beta\qquad\forall\mathbf{x} \in  C\]
    Moreover: 
    \[\mathbf{p}^T\mathbf{y} -\beta = (\mathbf{y} - \mathbf{x}^\prime)^T(\mathbf{y}-\mathbf{x}^\prime) = {\left\lVert \mathbf{y} -\mathbf{x}^\prime\right\rVert}^2 > 0\]  
    Since $\mathbf{y} \notin  C$.
    This proves the existence of a separating hyperplane.
\end{proof}
The separation theorem has three main consequences. 

\paragraph*{Convex set as intersection}
Any nonempty, closed convex set is the intersection of all closed half-spaces containing it.

\paragraph*{Existence of a supporting hyperplane}
If $C$ is a nonempty convex set, then for every boundary point $\overline{\mathbf{x}}\in\partial(S)$, there exists at least one supporting hyperplane $H$ at $\overline{\mathbf{x}}$. 
That is, there exists $\mathbf{p}\neq 0$ such that: 
\[\mathbf{p}^T(\mathbf{x}-\overline{\mathbf{x}})\leq 0\qquad\forall\mathbf{x}\in C\]

\paragraph*{Farkas lemma}
Farkas' Lemma is a fundamental result in linear algebra and optimization, playing a key role in proving optimality conditions for nonlinear programming. 
It provides an alternative statement for the feasibility of a linear system, offering a way to certify infeasibility.
\begin{lemma}
    Let $\mathbf{A}\in\mathbb{R}^{m\times n}$ and $\mathbf{b}\in\mathbb{R}^m$. 
    Exactly one of the following two statements is true:
\end{lemma}
\begin{enumerate}
    \item \textit{There exists $\mathbf{x}\in\mathbb{R}^n$ such that:}
        \[\mathbf{A}\mathbf{x}=\mathbf{b}\qquad\forall\mathbf{x}\geq \mathbf{0}\]
    \item \textit{There exists $\mathbf{y}\in\mathbb{R}^m$ such that:}
        \[\mathbf{y}^T\mathbf{A}\leq\mathbf{0}^T\qquad\mathbf{y}^T\mathbf{b}\geq\mathbf{0}^T\]
\end{enumerate}
Thus, either $\mathbf{b}$ belongs to the cone of $\mathbf{A}$, or it is strictly separated by some hyperplane.

\begin{proof}[Proof (forward direction)]
    Suppose there exists $\tilde{\mathbf{x}}\geq 0$ such that $\mathbf{A}\tilde{\mathbf{x}}=\mathbf{b}$. 
    Then, for any $\mathbf{y}$ satisfying  $\mathbf{y}^T\mathbf{A}\leq \mathbf{0}$, we have:
    \[\mathbf{y}^T\mathbf{b}=\mathbf{y}^T\mathbf{A}\tilde{\mathbf{x}}\leq \mathbf{0}\]
\end{proof}
\begin{proof}[Proof (backward direction)]
    Suppose $\mathbf{A}\mathbf{x}=\mathbf{b}$, $\mathbf{x}\geq 0$  is infeasible, meaning $\mathbf{b}\notin\text{cone}(\mathbf{A})$. 
    Since $\text{cone}(\mathbf{A})$ is a nonempty, closed, convex set, and $\mathbf{b}\notin\text{cone}(A)$, by the separating hyperplane theorem, there exists $\mathbf{p}\in\mathbb{R}^n$ and $\beta\in\mathbb{R}$ such that: 
    \[\mathbf{p}^T\mathbf{b}>\beta \quad\mathbf{p}^T\mathbf{z}\leq \beta \qquad\forall\mathbf{z}\in\text{cone}(\mathbf{A})\]
    Since $\mathbf{0}\in\text{cone}(\mathbf{A})$, it follows that $\beta\geq 0$, implying $\mathbf{p}^T\mathbf{b}>0$.
    Furthermore, since $\mathbf{z}\in\text{cone}(\mathbf{A})$, we get:
    \[\mathbf{p}^T\mathbf{A}\mathbf{x}\leq\beta\qquad\forall\mathbf{x}\geq 0\]
    Since $\mathbf{x}\geq \mathbf{0}$, this implies $\mathbf{p}^T\mathbf{A}\leq 0$.
    Thus, we have found $\mathbf{y}=\mathbf{p}$ satisfying the alternative system, proving the lemma.
\end{proof}
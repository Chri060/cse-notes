\section{Impartial combinatorial games}

\begin{definition}[\textit{Impartial combinatorial game}]
    An impartial combinatorial game is defined by the following characteristics:
\end{definition}
\begin{enumerate}
    \item There are two players who alternate turns.
    \item The game consists of a finite number of positions.
    \item Both players adhere to the same set of rules.
    \item The game concludes when no further moves can be made.
    \item The outcome of the game is not influenced by chance.
    \item In the classical version of the game, the winner is the player who leaves the opponent with no available moves; in the misère version, the objective is reversed.
\end{enumerate}
\begin{example}
    Several examples illustrate impartial combinatorial games:
    \begin{itemize}
        \item $k$ piles of cards: each player, on their turn, can take any number of cards (at least one) from a single pile.
        \item $k$ piles of cards with restrictions: each player can take any number of cards (at least one) from no more than $j < k$ piles during their turn.
        \item $k$ cards in a row: players can take $j_l$ cards on their turn.
    \end{itemize}
    In all these variations, the player who is left without cards loses. 
    In the first two examples, the positions can be represented as $(n_1, \cdots , n_k)$, where each  $n_i$ is a non-negative integer corresponding to the number of cards in each pile.
    In the third example, the positions are characterized by all non-negative integers less than or equal to $k$.
\end{example}
To solve impartial combinatorial games, we begin by partitioning the set of all possible positions (which are finite in number) into two distinct categories:
\begin{enumerate}
    \item $P$-positions: these are positions where the previous player has a winning strategy, meaning they are losing positions for the player who is about to move.
    \item $N$-positions: these are positions where the next player has a winning strategy, indicating they are winning positions for the player who is about to move.
\end{enumerate}
It is important to note that the current state of the game is what matters, rather than which player is designated to move.

\paragraph*{Partition rules}
The rules for the partitions are:
\begin{itemize}
    \item The terminal position $(0, 0, \cdots, 0)$ is classified as a $P$-position. 
        This is a losing position because the player has no cards left to play.
    \item From any $P$-position, only $N$-positions can be reached. 
        This means that the next player is guaranteed to have a winning strategy.
    \item From any $N$-position it is possible (but not obligatory) to move to a $P$-position. 
        The player in an $N$-position can make a move that leads their opponent to a losing position.
\end{itemize}
Therefore, the player who starts from an $N$-position is assured of a victory, given that they play optimally. 

\subsection{Nim game}
The Nim game is characterized by a tuple $(n_1, \cdots, n_k)$, where each $n_i$ is a positive integer.
During their turn, each player must choose one pile $n_i$ and replace it with $\hat{n}_i$, ensuring that $\hat{n}_i < n_i$. 
The player who reduces the position to $(0, \dots, 0)$ wins. 

Therefore, each player's action involves removing cards from a single pile with the objective of clearing the entire table.

\begin{theorem}[Bouton]
    A position $(n_1, n_2, \dots, n_k)$ in the Nim game is a $P$-position if and only if:
    \[n_1 \oplus n_2 \oplus \dots \oplus n_k = 0\]
\end{theorem}
\begin{proof}
    The terminal position $(0, 0, \dots , 0)$ is a $P$-position corresponding to a Nim-sum of zero.

    If the Nim-sum $n_1 \oplus n_2 \oplus \cdots \oplus n_k = 0$, any subsequent position will have a non-zero Nim-sum. 
    Assume the next position is $(\hat{n}_1, n_2, \ldots, n_k)$ such that $\hat{n}_1 \oplus n_2 \oplus \cdots \oplus n_k = 0$. 
    Then we would have:
    \[n_1 \oplus n_2 \oplus \cdots \oplus n_k = 0\]
    Which, by the cancellation law, implies $\hat{n}_1 = n_1$. 
    This is a contradiction, as the game rules stipulate that $\hat{n}_1 < n_1$.
    
    Conversely, if $n_1 \oplus n_2 \oplus \cdots \oplus n_k \neq 0$, it is possible to move to a position with a zero Nim-sum. 
    Let $z = n_1 \oplus n_2 \oplus \cdots \oplus n_k \neq 0$.
    Identify a pile where the binary representation of $z$ has a 1 in its leftmost column. 
    Change that digit to 0 and adjust the digits to the right, leaving unchanged the digits that correspond to 0. 
    This operation produces a new number that is smaller than the original. 
\end{proof}
\begin{example}
    Consider the configuration with a non-zero Nim-sum: $4,6,5$. 
    By removing a card from the first pile, we can reach the configuration $3,6,5$, which has a zero Nim-sum. 
    Thus, there are three initial advantageous moves, one available for each pile. 
\end{example}

\subsection{Conclusions}
Games with perfect information can typically be resolved through backward induction. 
However, this method is primarily effective for relatively simple games due to the constraints of limited rationality. 
Depending on the specifics of the game, we may arrive at varying degrees of solutions.
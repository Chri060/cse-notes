\section{Bimatrices}

Conventionally, Player 1 selects a row, while Player 2 selects a column. 
This results in a pair of values that represent the utilities for Player 1 and Player 2, respectively. 
These options can be conveniently displayed in a bimatrix.
\begin{example}
   Consider the following bimatrix:
    \[\begin{pmatrix} \begin{pmatrix} 8 & 8 \end{pmatrix} & \begin{pmatrix} 2 & 7 \end{pmatrix} \\ \begin{pmatrix} 7 & 2 \end{pmatrix} & \begin{pmatrix} 0 & 0 \end{pmatrix} \end{pmatrix}\]
    In this example, Player 1's utilities are given by:
    \[\begin{pmatrix} 8 & 2 \\ 7 & 0 \end{pmatrix}\]
    Since the second row is strictly dominated by the first (i.e., Player 1's utility in the first row is higher for any choice by Player 2), Player 1 will rationally choose the first row. 
    Similarly, Player 2 will select the first column, as it strictly dominates the second column.
\end{example}
While the principle of eliminating strictly dominated strategies may seem simplistic, it can lead to surprisingly powerful insights and outcomes.
\begin{example}
    Consider the following two games:
    \[\begin{pmatrix} \begin{pmatrix} 10 & 10 \end{pmatrix} & \begin{pmatrix} 3 & 15 \end{pmatrix} \\ \begin{pmatrix} 15 & 3 \end{pmatrix} & \begin{pmatrix} 5 & 5 \end{pmatrix} \end{pmatrix}\]
    \[\begin{pmatrix} \begin{pmatrix} 8 & 8 \end{pmatrix} & \begin{pmatrix} 2 & 7 \end{pmatrix} \\ \begin{pmatrix} 7 & 2 \end{pmatrix} & \begin{pmatrix} 0 & 0 \end{pmatrix} \end{pmatrix}\]
    Note that in the first game, players have outcomes like $\begin{pmatrix} 10 & 10 \end{pmatrix}$ and $\begin{pmatrix} 15 & 3 \end{pmatrix}$, which individually seem to offer higher utilities than most outcomes in the second game.
    However, applying rational decision-making principles leads to a surprising result.

    According to the principle of elimination of dominated strategies, players will end up choosing the outcome pair $\begin{pmatrix} 8 & 8 \end{pmatrix}$ in the second game because it dominates other available outcomes.
    This leads them to prefer the second game over the first game, despite the fact that the first game contains outcomes with higher individual utilities, like $\begin{pmatrix} 10 & 10 \end{pmatrix}$ and $\begin{pmatrix} 15 & 3 \end{pmatrix}$.

    Now, consider the expanded form of the first game, which contains even more outcomes:
    \[\begin{pmatrix} \begin{pmatrix} 1 & 1 \end{pmatrix} & \begin{pmatrix} 11 & 0 \end{pmatrix} & \begin{pmatrix} 4 & 0 \end{pmatrix} \\ \begin{pmatrix} 0 & 11 \end{pmatrix} & \begin{pmatrix} 8 & 8 \end{pmatrix} & \begin{pmatrix} 2 & 7 \end{pmatrix} \\ \begin{pmatrix} 0 & 4 \end{pmatrix} & \begin{pmatrix} 7 & 2 \end{pmatrix} & \begin{pmatrix} 0 & 0 \end{pmatrix} \end{pmatrix}\]
    This expanded version of the first game includes all the outcomes from the second game, plus some additional options. 
    However, rationality axioms suggest that in the first game, players should choose the outcome $\begin{pmatrix} 10 & 10 \end{pmatrix}$, which dominates the other possibilities.

    Interestingly, in the second game, where fewer options are available, the players end up selecting $\begin{pmatrix} 8 & 8 \end{pmatrix}$.
    This leads to a paradoxical outcome: having fewer available actions can actually make players better off by simplifying the decision-making process and avoiding suboptimal choices.
\end{example}
\begin{example}
    Consider the rational outcomes of the following game. 
    \[\begin{pmatrix} \begin{pmatrix} 0 & 0 \end{pmatrix} & \begin{pmatrix} 1 & 1 \end{pmatrix} \\ \begin{pmatrix} 1 & 1 \end{pmatrix} & \begin{pmatrix} 0 & 0 \end{pmatrix} \end{pmatrix}\]

    While we may not know the rational outcomes formally, it is clear that the preferred outcome for both players is $\begin{pmatrix} 1 & 1 \end{pmatrix}$. 
    However, this leads to a coordination problem.

    Both pairs of actions result in the same outcome $\begin{pmatrix} 1 & 1 \end{pmatrix}$, but there is no clear way for the players to distinguish between these two strategies.
    As a result, while the rational outcome is obvious, the players face difficulty coordinating on which specific actions to take to achieve it.
\end{example}
\begin{example}
    Consider a voting game with three players, each having the following preferences:
    \begin{enumerate}
        \item Player 1: $A \precneqq B \precneqq C$
        \item Player 2: $B \precneqq C \precneqq A$
        \item Player 3: $C \precneqq A \precneqq B$
    \end{enumerate}
    Here, the notation $A \precneqq B$ indicates that Player 1 prefers $B \preceq A$, but not vice versa. 
    The winner is determined by the alternative that receives the most votes. 
    However, if there is a tie among three different votes, the alternative chosen by Player 1 will win.

    Let's now analyse the rational outcome of the game through the elimination of dominated actions:
    \begin{itemize}
        \item Alternative A is a weakly dominant strategy for Player 1.
        \item Players 2 and 3 have their least preferred choice as a weakly dominated strategy.
    \end{itemize}
    To avoid their worst outcome, Player 2 retains options $B$ and $C$ (ordered in rows), while Player 3 keeps $C$ and $A$ (ordered in columns). 
    Player 1 will consistently choose $A$. 
    This simplifies the game to a $2 \times 2$ table with the following outcomes:
    \begin{table}[H]
        \centering
        \begin{tabular}{l|ll|}
        \cline{2-3}
        \textbf{}                        & \textbf{C} & \textbf{A} \\ \hline
        \multicolumn{1}{|l|}{\textbf{B}} & A          & A          \\
        \multicolumn{1}{|l|}{\textbf{C}} & C          & A          \\ \hline
        \end{tabular}
    \end{table}
    Since $C \precneqq A$ for both Players 2 and 3, they will choose the outcome in the second row and first column, leading to the final result being $C$, which is the worst outcome for Player 1.
\end{example}
\begin{example}
    Consider the classic Chicken Game. 
    Two cars are driving toward each other on a narrow road, and there isn't enough room for both cars to pass. 
    If both cars keep going straight, they will crash, but if at least one deviates, the crash is avoided.
    The payoff bimatrix for the game is:
    \[\begin{pmatrix} \begin{pmatrix} -1 & -1 \end{pmatrix} & \begin{pmatrix} 1 & 10 \end{pmatrix} \\ \begin{pmatrix} 10 & 1 \end{pmatrix} & \begin{pmatrix} -10 & -10 \end{pmatrix} \end{pmatrix}\]
    The payoffs are as follows:
    \begin{itemize}
        \item Both players receive a utility of $-1$ if they both deviate.
        \item A player earns $10$ for going straight when the other deviates, while the deviating player receives $1$.
        \item If both go straight and crash, they both receive $-10$.
    \end{itemize}

    The best outcomes are either $\begin{pmatrix} 10 & 1 \end{pmatrix}$ and $\begin{pmatrix} 1 & 10 \end{pmatrix}$.
    However, there is no way to decisively determine which of these two outcomes will occur, as both are equally viable equilibria.
\end{example}
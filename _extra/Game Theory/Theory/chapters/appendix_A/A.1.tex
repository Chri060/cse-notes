\section{Binary sum}

We define a binary operation $\oplus$ on the set of natural numbers $\mathbb{N}=\{0, 1, 2, \dots, n, \dots \}$ as follows.
For any two natural numbers $n_1,n_2 \in \mathbb{N}$:
\begin{enumerate}
    \item Convert $n_1$ and $_n2$ into their binary representations, denoted as $[n_1]_2$ and $[n_2]_2$. 
    \item Perform the binary addition of  $[n_1]_2$ and $[n_2]_2$ using the standard addition method, but without carrying over.
        This means if the addition of two bits results in 2 (i.e., $1 + 1$), it should be represented as 0 in that position with no carry to the next higher bit.
    \item The result of the operation $\oplus$ is then represented in binary form, which corresponds to the sum computed in step 2.
\end{enumerate}

\begin{example}
    Let's apply the $\oplus$ operation to the numbers $1,2,4$, and $1$: 
    \begin{enumerate}
        \item Convert the numbers to binary: 
            \begin{itemize}
                \item $1$ in binary: $[1]_2=001$
                \item $2$ in binary: $[2]_2=010$
                \item $4$ in binary: $[4]_2=100$
                \item $1$ in binary: $[1]_2=001$
            \end{itemize}
        \item Perform the $\oplus$ operation:
            \begin{table}[H]
                \centering
                \begin{tabular}{l}
                    $[1]_2=001$ $+$ \\ 
                    $[2]_2=010$ $+$ \\ 
                    $[4]_2=100$ $+$\\
                    $[1]_2=001=$ \\
                    \hline
                    $[6]_2=110$
                \end{tabular}
            \end{table}
        \item The result of the operation $1\oplus 2\oplus 4\oplus 1$ in decimal form is 6.
    \end{enumerate}
\end{example}
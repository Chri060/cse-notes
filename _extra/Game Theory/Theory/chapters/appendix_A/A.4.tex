\section{Linear programming}

\begin{definition}[\textit{Duality first form}]
    The following two linear programs are said to be in duality: 
    \[\begin{cases}
        \min(\mathbf{c},\mathbf{x}) \\
        \mathbf{Ax} \geq \mathbf{b} \\
        \mathbf{x} \geq 0
    \end{cases} \qquad \begin{cases}
        \max(\mathbf{b},\mathbf{y}) \\
        \mathbf{A}^T\mathbf{y} \leq \mathbf{c} \\
        \mathbf{y} \geq 0
    \end{cases}\]
    Here the matrix $A\in\mathbb{R^{m\times n}}$ and the vectors $\mathbf{c},\mathbf{x}\in\mathbb{R}^n$, $\mathbf{b},\mathbf{y}\in\mathbb{R}^m$.
\end{definition}
The minimization problem is called primal problem and the maximization is called dual problem. 
\begin{definition}[\textit{Duality second form}]
    The following two linear programs are said to be in duality: 
    \[\begin{cases}
        \min(\mathbf{c},\mathbf{x}) \\
        \mathbf{Ax} \geq \mathbf{b} 
    \end{cases} \qquad \begin{cases}
        \max(\mathbf{b},\mathbf{y}) \\
        \mathbf{A}^T\mathbf{y} \leq \mathbf{c} \\
        \mathbf{y} \geq 0
    \end{cases}\]
\end{definition}
The minimization problem in the second form can be written in an equivalent way in the first form; dualizing this shows that the dual is equivalent to the dueal of the second form, in the sense that the solution is the same. 

Given two problems in duality, there are three options: 
\begin{enumerate}
    \item Both can be feasible.
    \item Only one can be feasible. 
    \item They can both be infeasible.
\end{enumerate}
\begin{example}
    Consider the following problem: 
    \[\begin{cases}
        \min x_1+x_2 \\
        x_1+2x_2\geq 1 \\
        x_1,x_2 \geq 0
    \end{cases}\]
    In this case we have: 
    \[\mathbf{c}=\begin{bmatrix} 1 & 1 \end{bmatrix} \qquad \mathbf{A}=\begin{bmatrix} 1 & 2 \end{bmatrix} \qquad b=1\]
    The corresponding dual problem is: 
    \[\begin{cases}
        \max y \\
        y \leq 1 \\
        2y \leq 1 \\
        y \geq 0
    \end{cases}\]
    The solution for the first problem is $\begin{pmatrix} 0 & \frac{1}{2} \end{pmatrix}$, and for the second is $\frac{1}{2}$. 
    Therefore, they are both feasible. 
\end{example}
\begin{example}
    Consider the following problem: 
    \[\begin{cases}
        \min x_1-x_2 \\
        x_1+x_2 \geq 2 \\
        -x_1-x_2 \geq -1 \\
        x_1,x_2 \geq 0
    \end{cases}\]
    In this case we have: 
    \[\mathbf{c}=\begin{bmatrix} 1 & -1 \end{bmatrix} \qquad \mathbf{A}=\begin{bmatrix} 1 & 1 \\ -1 & -1 \end{bmatrix} \qquad b=\begin{pmatrix} 2 \\ -1 \end{pmatrix}\]
    The corresponding dual problem is: 
    \[\begin{cases}
        \max 2y_1-y_2 \\
        y_1-y_2 \leq 1 \\
        y_1-y_2 \leq -1 \\
        y_1,y_2 \geq 0
    \end{cases}\]
    The primal problem is infeasible, while the dual problem has a solution $\begin{pmatrix} 0 & 1 \end{pmatrix}$
\end{example}

\subsection{Duality theorems}
\begin{theorem}[\textit{Weak duality}]
    Let $v$ be the value of the primal minimization problem and $V$ the value of the dual maximization problem. 
    Then: 
    \[v\geq V\]
\end{theorem}
\begin{proof}
    In the first form we have: 
    \[(\mathbf{c},\mathbf{x})\geq(\mathbf{A}^T\mathbf{y},\mathbf{x})=(\mathbf{y},\mathbf{Ax})\geq(\mathbf{y},\mathbf{b})\]
    In the second form we have: 
    \[(\mathbf{c},\mathbf{x})=(\mathbf{A}^T\mathbf{y},\mathbf{x})=(\mathbf{y},\mathbf{Ax})\geq(\mathbf{y},\mathbf{b})\]
\end{proof}

\begin{theorem}[\textit{Strong duality}]
    If the primal and the dual problems are feasible, then both problems have optimal solutions $\bar{\mathbf{x}}$ and $\bar{\mathbf{y}}$ and the optimal values coincide, thai is: 
    \[v=(\mathbf{c},\bar{\mathbf{x}})=(\mathbf{b},\bar{\mathbf{y}})=V\]

    If the primal is feasible and the dual is infeasible, then $v=V=-\infty$. 

    If the primal is infeasible and the dual is feasible, then $v=V=+\infty$. 

    If both the primal and the dual are infeasible, then $v=+\infty>V=-\infty$.
\end{theorem}
\begin{corollary}
    If one problem is feasible and has an optimal solution, then also the dual problem is feasible and has solution. 
    Morovere, there is no duality gap. 
\end{corollary}

\subsection{Complementarity}
\begin{theorem}[Complementarity condition first form]
    Let $\bar{\mathbf{x}}$ and $\bar{\mathbf{y}}$ be primal and dual feasible. 
    Then $\bar{\mathbf{x}}$ and $\bar{\mathbf{y}}$ are simultaneously optimal if and only if: 
    \[\begin{cases}
        \forall i \bar{x}_i>0 \implies\sum_{k=1}^ma_{ik}\bar{y}_k=c_i \\
        \forall i \bar{y}_i>0 \implies\sum_{k=1}^na_{kj}\bar{x}_k=b_i 
    \end{cases}\]
\end{theorem}
\begin{proof}
    Recall that $(\mathbf{c},\mathbf{x})\geq(\mathbf{A}^T\mathbf{y},\mathbf{x})\geq(\mathbf{y},\mathbf{Ax})\geq(\mathbf{b},\mathbf{y})$.
    So, $\bar{\mathbf{x}}$ and $\bar{\mathbf{y}}$ are optimal if and only if: 
    \[(\mathbf{c},\bar{\mathbf{x}})=(\mathbf{A}^T\bar{\mathbf{y}},\bar{\mathbf{x}})=(\bar{\mathbf{y}},\mathbf{A}\bar{\mathbf{x}})=(\mathbf{b},\bar{\mathbf{y}})\]
    This is equivalent to: 
    \[(\mathbf{A}^T\bar{\mathbf{y}}-\mathbf{c},\bar{\mathbf{x}})=0 \qquad (\mathbf{A}\bar{\mathbf{y}}-\mathbf{b},\bar{\mathbf{y}})=0 \]
    Since $\bar{\mathbf{x}}, \bar{\mathbf{y}}\geq 0$, $\mathbf{A}\bar{\mathbf{x}}\geq \mathbf{b}$, and $\mathbf{A}^T\bar{\mathbf{y}}\leq\mathbf{c}$ the latter equations are equivalent to the complementary conditions states by the theorem.
\end{proof}
\begin{example}
    Consider the following linear programming problem: 
    \[\begin{cases}
        \min x_1+x_2 \\
        2x_1+x_2 \geq 2 \\ 
        x_1 + 2x_2 \leq 2 \\
        x_1,x_2 \geq 0
    \end{cases}\]
    The corresponding dual is: 
    \[\begin{cases}
        \max 2y_1-2y_2 \\
        2y_1 - y_2 \leq 1 \\
        y_1-2y_2 \leq 1 \\
        y_1,y_2 \geq 0
    \end{cases}\]
    We have that $v=1$, and $\bar{\mathbf{x}}=\begin{bmatrix} 1 & 0 \end{bmatrix}$. 
    We have that $V=1$, and $\bar{\mathbf{x}}=\begin{bmatrix} \frac{1}{2} & 1 \end{bmatrix}$.

    We may now check for the complementary conditions: 
    \[\bar{y}_1=\dfrac{1}{2}>0\implies2\bar{x}_1-\bar{x}_2=2 \qquad \bar{x}_1=1>0 \implies 2y_1+y_2=1\]
\end{example}

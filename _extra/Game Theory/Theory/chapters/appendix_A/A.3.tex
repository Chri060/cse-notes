\section{Convexity}

\begin{definition}[\textit{Convex set}]
    A set $C \subset \mathbb{R}^n$ is called convex if for any points $x, y \in C$ and for any $\lambda \in [0, 1]$ the point $\lambda x + (1 - \lambda)y \in C$.
\end{definition}
This means that the line segment connecting any two points in $C$ is entirely contained within $C$.
The properties of a convex set are:
\begin{itemize}
    \item The intersection of an arbitrary family of convex sets is convex.
    \item A closed convex set with a nonempty interior coincides with the closure of its internal points.
\end{itemize}

\begin{definition}[\textit{Convex combination}]
    A convex combination of elements $x_1, \dots, x_n$ is any vector $x$ of the form:
    \[x = \lambda_1x_1 + \dots + \lambda_nx_n\]
    where $\lambda_1 \geq 0, \dots, \lambda_n \geq 0$ and $\sum_{i=1}^{n} \lambda_i = 1$.
\end{definition}
\begin{proposition}
    A set $C$ is convex if and only if for every $\lambda_1 \geq 0, \dots, \lambda_n \geq 0$ such that $\sum_{i=1}^{n}\lambda_i = 1$, and for every $c_1, \dots, c_n \in C$, we have 
    \[\sum^n_{i=1} \lambda_i c_i \in C\]
\end{proposition}
\begin{definition}[\textit{Convex hull}]
    The convex hull of a set $C$, denoted by $\text{co }C$, is the smallest convex set containing $C$. 
    It is defined as:
    \[\text{co }C =\bigcap_{A\in\mathcal{C}}A\]
    where $\mathcal{C} = \left\{A | C \subset A \text{ and } A \text{ is convex}\right\}$. 
\end{definition}
\begin{proposition}
    The convex hull of a set $C$ can be expressed as:
    \[\text{co }C=\left\{\sum_{i=1}^{n}\lambda_ic_i|\lambda_i\geq 0,\sum_{i=1}^{n}\lambda_i=1,c_i\in C\quad\forall i, n\in \mathbb{N}\right\}\]
\end{proposition}
The convex hull of a set $C$ consists of all convex combinations of points in $C$.
When $C$ is a finite set, the convex hull is called a polytope.
\begin{theorem}
    Given a closed convex set $C$ and a point $x$ outside $C$, there exists a unique point $p \in C$ such that for all $c \in C$:
    \[\left\lVert p - x\right\rVert\leq\left\lVert c-x\right\rVert\]
\end{theorem}
The projection $p$ is the point in $C$ closest to $x$ and satisfies the following:
\begin{enumerate}
    \item $p \in C$.
    \item $(x - p, c - p) \leq 0 \text{ for all } c \in C$.
\end{enumerate}
\begin{theorem}
    Let $C$ be a convex subset of $\mathbb{R}^l$, and assume $\bar{x} \in \text{cl } C^c$ (the closure of the complement of $C$).
    Then, there exists a nonzero $x^\ast \in \mathbb{R}^l$ such that for all $c \in C$:
    \[(x^\ast, c) \geq (x^\ast, \bar{x})\]
\end{theorem}
This result provides a criterion to distinguish points outside of $C$ from those inside.
\begin{proof}
    Assume that $\bar{x} \notin \text{cl } C$ and let $p$ be its projection onto $\text{cl } C$. 
    By the previous theorem, we have:
    \[(\bar{x} - p, c - p) \leq 0 \qquad \forall c \in C\]
    Now, define $x^\ast = p - \bar{x} \neq 0$, then the inequality becomes:
    \[(-x^\ast, c - \bar{x} - x^\ast) =(-x^\ast, -x^\ast) + (-x^\ast, c - \bar{x}) \leq 0\]
    This implies that:
    \[(x^\ast, c - \bar{x}) \geq \left\lVert x^\ast\right\rVert^2\]
    Since $\left\lVert x^\ast\right\rVert^2> 0$, by linearity we obtain:
    \[(x^\ast, c) \geq (x^\ast, \bar{x}) \qquad \forall c \in C\]
    Since $x^\ast$ appears on both sides of the inequality, we can renormalize and choose $\left\lVert x^\ast\right\rVert = 1$. 

    If $\bar{x} \in \text{cl }C \setminus C$, take a sequence $\{x_n\} \subset C^c$ such that $x_n \rightarrow \bar{x}$. 
    From the first part of the proof, there exists a sequence of vectors $x^\ast_n$, each of norm 1, such that:
    \[(x^\ast_n, c) \geq (x^\ast_n, x_n) \qquad \forall c \in C\]
    By taking the limit along a subsequence where $x^\ast_n \rightarrow x^\ast$, we obtain:
    \[(x^\ast, c) \geq (x^\ast, \bar{x}) \qquad \forall c \in C\]
\end{proof}
\begin{corollary}
    For any closed convex set $C$ in Euclidean space, and any point $x$ on the boundary of $C$, there exists a hyperplane that contains $x$ and leaves all points in $C$ on one side of the hyperplane.
\end{corollary}
This hyperplane is called a supporting hyperplane for $C$ at $x$.
\begin{corollary}
    Any closed convex set $C$ in Euclidean space can be represented as the intersection of all half-spaces that contain it.
\end{corollary}
\begin{theorem}
    Let $A$ and $C$ be closed convex subsets of $\mathbb{R}^l$, with $\text{int } A \neq\varnothing$ and $\text{int }A \cap C = \varnothing$. 
    Then, there exists a nonzero vector $x^\ast$ and a scalar $b \in \mathbb{R}$ such that for all $a \in A$ and $c \in C$:
    \[(x^\ast, a) \geq b \geq (x^\ast, c)\]
\end{theorem}
This provides a criterion to determine whether a point lies in $A$ or $C$.
\begin{proof}
    Since $0=\bar{x} \in (\text{int}A - C)^c$, by the previous separation theorem, there exists $x^\ast \neq 0$ such that:
    \[(x^\ast, x) \geq 0 \qquad \forall x \in \text{int}A - C\]
    By linearity, for $x = a - c$, we obtain:
    \[(x^\ast, a) \geq (x^\ast, c) \qquad \forall a \in \text{int }A , \forall c \in C\]
    Extending this inequality to the closure of $\text{int }A$, we have:
    \[(x^\ast, a) \geq (x^\ast, c) \qquad \forall a \in \text{cl } \text{int }A = A , \forall c \in C\]
\end{proof}
The hyperplane $H = \{x : (x^\ast, x) = b\}$  is the separating hyperplane, with $A$ and $C$ located in different half-spaces defined by $H$.
\section{Introduction}

\begin{definition}[\textit{Zero sum game}]
    A two-player zero-sum game in strategic form can be described as a triplet $(X, Y , f : X \times Y \rightarrow \mathbb{R})$, where:
    \begin{itemize}
        \item $X$ is the strategy space of Player 1.
        \item $Y$ is the strategy space of Player 2.
        \item $f(x, y)$ represents the payoff Player 1 receives from Player 2 when they play strategies $x$ and $y$, respectively.
    \end{itemize}
\end{definition}
\noindent Since this is a zero-sum game, Player 2's utility function $g$ is defined as the negative of Player 1's utility function:
\[g(x,y)=-f(x,y)\]
\noindent In the case where the strategy spaces are finite the game can be represented by a payoff matrix $P$. 
In this matrix, Player 1 chooses a row $i$, and Player 2 chooses a column $j$:
\[\begin{pmatrix} p_{11} & \cdots & p_{1m} \\ \cdots & p_{ij} & \cdots \\ p_{n1} & \cdots & p_{nm} \end{pmatrix}\]
\noindent Here, $p_{ij}$ denotes the payment Player 2 makes to Player 1 when they select strategies $i$ and $j$, respectively.

To determine the optimal strategy, both players can employ conservative reasoning: 
\begin{itemize} 
    \item Player 1 can ensure a minimum payoff of $v_1 = \max_i \min_j p_{ij}$. 
    \item Player 2 can limit their losses to at most $v_2 = \min_j \max_i p_{ij}$. 
\end{itemize}
\noindent These values, $v_1$ and $v_2$, are known as the conservative values for Player 1 and Player 2, respectively.

In more general cases where the strategy spaces $X$ and $Y$ are not finite, a similar reasoning applies. 
Let $(X, Y, f:X\times Y\rightarrow\mathbb{R})$ describe the game, where $X$ and $Y$ are arbitrary strategy sets. 
The conservative values can be defined as follows: 
\begin{itemize} 
    \item Player 1: $v_1 = \sup_x \inf_y f(x, y)$.
    \item Player 2: $v_2 = \inf_y \sup_x f(x, y)$. 
\end{itemize}
\noindent These values, $v_1$ and $v_2$, are known as the conservative values for Player 1 and Player 2, respectively. 
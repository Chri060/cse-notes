\section{Combinatorial games}

\begin{definition}[\textit{Impartial combinatorial game}]
    An impartial combinatorial game is defined by the following characteristics:
    \begin{enumerate}
        \item There are two players in alternate order.
        \item There is a finite number of positions in the game.
        \item The players follow the same rules.
        \item The game concludes when no further moves are possible.
        \item The game does not involve chance. 
        \item In the classical version, the winner is the player leaving the other player with no available moves.
    \end{enumerate}
\end{definition}
\noindent To solve impartial combinatorial games, we begin by partitioning the set of all possible positions (which are finite in number) into two distinct categories:
\begin{enumerate}
    \item \textit{$P$-positions}: loosing.
    \item \textit{$N$-positions}: winning.
\end{enumerate}
\noindent It is important to note that the current state of the game is what matters, rather than which player is designated to move.

We have the following rules:
\begin{itemize}
    \item The terminal position $(0, 0, \cdots, 0)$ is classified as a $P$-position. 
        This is a losing position because the player has no cards left to play.
    \item From any $P$-position, only $N$-positions can be reached. 
    \item From any $N$-position it is possible (but not obligatory) to move to a $P$-position. 
\end{itemize}
\noindent Therefore, the player who starts from an $N$-position is assured of a victory, given that they play optimally. 

\subsection{Nim game}
The Nim game is characterized by a tuple $(n_1, \cdots, n_k)$, where each $n_i$ is a positive integer.
During their turn, each player must choose one pile $n_i$ and replace it with $\hat{n}_i$, ensuring that $\hat{n}_i < n_i$. 
The player who reduces the position to $(0, \dots, 0)$ wins. 
Therefore, each player's action involves removing cards from a single pile with the objective of clearing the entire table.

\begin{theorem}[Bouton]
    A position $(n_1, n_2, \dots, n_k)$ in the Nim game is a $P$-position if and only if:
    \[n_1 \oplus n_2 \oplus \dots \oplus n_k = 0\]
\end{theorem}
\begin{proof}
    The terminal position $(0, 0, \dots , 0)$ is a $P$-position corresponding to a Nim-sum of zero.

    If the Nim-sum $n_1 \oplus n_2 \oplus \cdots \oplus n_k = 0$, any subsequent position will have a non-zero Nim-sum. 
    Assume the next position is $(\hat{n}_1, n_2, \ldots, n_k)$ such that $\hat{n}_1 \oplus n_2 \oplus \cdots \oplus n_k = 0$. 
    Then we would have:
    \[n_1 \oplus n_2 \oplus \cdots \oplus n_k = 0\]
    Which, by the cancellation law, implies $\hat{n}_1 = n_1$. 
    This is a contradiction, as the game rules stipulate that $\hat{n}_1 < n_1$.
    
    Conversely, if $n_1 \oplus n_2 \oplus \cdots \oplus n_k \neq 0$, it is possible to move to a position with a zero Nim-sum. 
    Let $z = n_1 \oplus n_2 \oplus \cdots \oplus n_k \neq 0$.
    Identify a pile where the binary representation of $z$ has a 1 in its leftmost column. 
    Change that digit to 0 and adjust the digits to the right, leaving unchanged the digits that correspond to 0. 
    This operation produces a new number that is smaller than the original. 
\end{proof}
\noindent Games with perfect information can typically be resolved through backward induction. 
However, this method is primarily effective for relatively simple games due to the constraints of limited rationality. 
Depending on the specifics of the game, we may arrive at varying degrees of solutions.
\section{Shapley value}

Let $\phi:\mathcal{G}(N)\rightarrow\mathbb{R}^n$ be a one point solution. 
\begin{property}
    Efficiency: $\sum_{i\in N}\phi_i(v)=v(N)$ for every $v\in\mathcal{G}(N)$. 
\end{property}
\begin{property}
    Symmetry:  symmetric players must get the same value. 
    If $v\in\mathcal{G}(N)$ is a game such that $v(A\cup\{i\})=v(A\cup\{j\})$ for every $A$ not containing $i$ and $j$, then $\phi_i(v)=\phi_j(v)$. 
\end{property}
\begin{property}
    Null player property: a player contributing nothing to any coalition cannot get anything. 
    If $v\in\mathcal{G}(N)$ and $i\in N$ are such that $v(A)=v(A\cup\{i\})$ for all $A$, then $\phi_i(v)=0$. 
\end{property}
\begin{property}
    Additivity: $\phi(v+w)=\phi(v)+\phi(w)$ for every $v,w\in\mathcal{G}(N)$. 
\end{property}
\begin{theorem}
    Let $\sigma:\mathcal{G}(N)\rightarrow\mathbb{R}^n$ be defined by: 
    \[\sigma_i(v)=\sum_{S\in 2^{N\setminus\{i\}}}\dfrac{s!(n-s-1)!}{n!}\left[v(S\cup\{i\})-v(S)\right]\]
    Then $\sigma$ satisfies the properties of efficiency, symmetry, null player and additivity. 
    Conversely, if $\tilde{\sigma}$ is a one point solution satisfying the property of efficiency, symmetry, null player and additivity, then $\tilde{\sigma}=\sigma$. 
\end{theorem}
\noindent In other words, there exits one point solution satisfying the properties of efficiency, symmetry, null player and additivity.
We call it the Shapley value. 
The term $m_i(v,S)=v(S\cup\{i\})-v(S)$ is called the marginal contribution of player $i$ to coalition $S\cup\{i\}$. 
\begin{proof}[Efficiency proof]
    Consider a generic term $v(S\cup\{i\})-v(S)$. 
    The term $v(N)$ appears $n$ times, once for every player, when $S=N\setminus\{i\}$. 
    As $s=n-1$ its coefficient is $\frac{(n-1)!(n-n)!}{n!}=\frac{1}{n}$. 
    Consider now any other coalition $T\neq N$. 
    The term $v(T)$ appears both with positive and negative coefficients; 
    \begin{itemize}
        \item The positive coefficient $\frac{(t-1)!(n-t)!}{n!}$ appears $t$ times, once for every player $i\in S$ when $S=T\setminus\{i\}$: hence the distribution is $\frac{t!(n-t)!}{n!}$.
        \item The negative coefficient $-\frac{(t-1)!(n-t)!}{n!}$ appears $n-t$ times, once for every player $i\notin T$ when $S=T$: hence the distribution is $-\frac{t!(n-t)!}{n!}$.
    \end{itemize}
    Thus in the sum: 
    \[\sum_{i=1}^{n}\sum_{S\in 2^{N\setminus\{i\}}}\dfrac{s!(n-s-1)!}{n!}\left[v(S\cup\{i\})-v(S)\right]\]
    $v(N)$ appears with coefficient one and every $A\neq N$ appears with null coefficient. 
\end{proof}
\begin{proof}[Symmetry proof]
    We have that: 
    \begin{align*}
        \sigma_i(v)=&\sum_{S\in 2^{N\setminus\{i\cup j\}}}\dfrac{s!(n-s-1)!}{n!}\left[v(S\cup\{i\})-v(S)\right]+ \\
                    &\sum_{S\in 2^{N\setminus\{i\cup j\}}}\dfrac{(s+1)!(n-s-2)!}{n!}\left[v(S\cup\{i\cup j\})-v(S\cup\{j\})\right]
    \end{align*}
    \begin{align*}
        \sigma_j(v)=&\sum_{S\in 2^{N\setminus\{i\cup j\}}}\dfrac{s!(n-s-1)!}{n!}\left[v(S\cup\{j\})-v(S)\right]+ \\
                    &\sum_{S\in 2^{N\setminus\{i\cup j\}}}\dfrac{(s+1)!(n-s-2)!}{n!}\left[v(S\cup\{i\cup j\})-v(S\cup\{i\})\right]
    \end{align*}
    The terms in the sums are thus equal for symmetric players. 
\end{proof}
\begin{proof}[Uniqueness proof]
    Consider a unanimity game $u_A$: 
    \begin{itemize}
        \item Players not belonging to $A$ are null players: thus $\sigma$ assigns zero to them. 
        \item Players belonging to $A$ are symmetric, and so $\sigma$ must assign the same value to both. 
        \item $\sigma$ is efficient. 
        \item Then $\sigma_i(u_A)=\frac{1}{\left\lvert A\right\rvert}$ if $i\in A$, $\sigma_i(u_A)=0$ otherwise. 
    \end{itemize}
    Then $\phi$ is uniquely determined by the basis of $\mathcal{G}(N)$ give in terms of the unanimity games. 
    The same argument applies to the game $c\cdot u_A$, for $c \in \mathbb{R}$. 
    Because of the additivity axiom, at most once function satisfies the properties. 
\end{proof}

\subsection{Simple games}
In the case of simple games, the Shapley value becomes: 
\[\sigma_i=\sum_{A\in\mathcal{A}_i}\dfrac{a!(n-a-1)!}{n!}\]
\noindent Here, $\mathcal{A}_i$ is the set of the coalitions $A$ such that $i \notin A$, $A$ is not winning, and $A\cup\{i\}$ is winning. 

Alternatively, it can be written: 
\[\sigma_i=\sum_{A\in\mathcal{W}_i}\dfrac{(a-1)!(n-a)!}{n!}\]
\noindent Here, $\mathcal{W}_i$ is the set of the coalitions $A$ such that $i \in A$, $A$ is winning, and $A\setminus\{i\}$ is not winning. 

In simple games the Shapley value measures the fraction of power of every player. 
In order to measure the relative power of the players in a simple game, the requirement of efficiency is not mandatory, hence coalitions could even form in a different way from the case of the Shapley value. 
\begin{definition}[\textit{Probabilistic power index}]
    A probabilistic power index $\psi$ one the set of simple games is: 
    \[\psi_i(v)=\sum_{S\in2^{N\setminus\{i\}}}\Pr_i(S)m_i(v,S)\]
    Here, $\Pr_i$ is a probability measure on $2^{N\setminus\{i\}}$.
\end{definition}
\begin{definition}[\textit{Semi-value}]
    A probabilistic power index $\psi$ one the set of simple games is a semi-value if there exists a vector $(\Pr_0,\dots,\Pr_{n-1})$ such that: 
    \[\psi_i(v)=\sum_{S\in2^{N\setminus\{i\}}}\Pr_sm_i(v,S)\]
\end{definition}
\noindent Since the index is probabilistic, the two conditions must hold: $\Pr_s\geq 0$ and $\sum_{n=0}^{n-1}\binom{n-1}{s}\Pr_s=1$.
\begin{definition}[\textit{Regular semi-value}]
    If $\Pr_s>0$ for all $s$, the semi-value is called regular. 
\end{definition}
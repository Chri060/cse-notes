\section{Imputation}

\begin{definition}[\textit{Imputation}]
    The solution $I:\mathcal{G}(N)\rightarrow\mathbb{R}^n$ such that $x\in I(v)$ if: 
    \begin{enumerate}
        \item $x_i\geq v(\{i\})$ for all $i$. 
        \item $\sum_{i=1}^{n}x_i=v(N)$. 
    \end{enumerate}
    is called imputation. 
\end{definition}
\noindent The first condition states that player $i$ will not paretcipate if the solution assigns him a value $x_i$ less than what he would be able to earn by himself. 
The second condition can be split in two inequalities: 
\begin{itemize}
    \item \textit{Feasibility}: $\sum_{i=1}^{n}x_i\leq v(N)$ assures that if the grand coalition is formed the amount available to the player is $v(N)$. 
    \item \textit{Efficiency}: $\sum_{i=1}^{n}x_i\geq v(N)$ says that the overall amount will be effectively distributed among all the players. 
\end{itemize}
\noindent If a game satisfies $v(N)\geq\sum_iv(\{i\})$, then the imputation is nonempty. 
If $v$ is additive, them $I(v)=\{(v(1,\dots,v(n)))\}$. 
\begin{proposition}
    The imputation set $I(v)$ is a polytope. 
\end{proposition}
\noindent Efficiency is a mandatory requirements in cooperative games. 
The imputation set is non empty if the game is superadditive and it reduces to a singleton if it is additive. 
The imputation set is the intersection of the half spaces defined by efficiency and feasibility constraints. 
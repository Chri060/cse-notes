\section{Nucleolus}

\begin{definition}
    The excess of a coalition $A$ over the imputation $x$ is: 
    \[e(A,x)=v(A)-\sum_{i\in A}x_i\]
\end{definition}
\noindent $e(A,x)$ is a measure of the dissatisfaction of the coalition $A$ with respect to the assignment of the imputation $x$. 
An imputation $x$ of the game $v$ belongs to $C(v)$ if and only if $e(A,x)\leq 0$ for all $A$. 

\begin{definition}[\textit{Lexicographic vector}]
    The lexicographic vector attached to the imputation $x$ is the $(2^{n}-1)^{\text{th}}$ dimensional vector $\theta(x)$ such that: 
    \begin{enumerate}
        \item $\theta_i(x)=e(A,x)$ for some $A\subseteq N$.
        \item $\theta_1(x)\geq \dots \geq \theta_{2^n-1}(x)$
    \end{enumerate}
\end{definition}
\noindent It arranges the excess of coalition over the imputation $x$ in decreasing order. 
\begin{definition}[\textit{Nucleolus}]
    The nucleolus solution is the solution $\nu:\mathcal{G}(N)\rightarrow\mathbb{R}^n$ such that $\nu(v)$ is the set of imputations $x$ such that $\theta(x)\leq_L\theta(y)$, for all imputations $y$ of the game $v$. 
\end{definition}
\noindent $x\leq_Ly$ if $x=y$ or there exists $j$ such that $x_i=y_i$ for all $i<j$, and $x_i<y_j$. 
$x\leq_Ly$ defines a total order in any Euclidean space. 
That is, the nucleolus minimizes the excess. 
\begin{theorem}
    For every transferable utility game $v$ with nonempty imputation set, the nucleolus $\nu(v)$ is a singleton.
\end{theorem}
\begin{proposition}
    Suppose $v$ is such that $C(v)\neq\varnothing$. 
    Then $\nu(v)\in C(v)$.
\end{proposition}
\begin{proof}
    For all $x\in C(v)$, by definition od core $\theta_1(x)\leq 0$. 
    Since the nucleolus minimizes the excess, we have $\theta_i(\nu(v))\leq 0$. 
    Then, $\nu(v)$ is in the core. 
\end{proof}
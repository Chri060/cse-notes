\section{Core}
\begin{definition}[\textit{Core}]
    The core is the solution $C:\mathcal{G}(N)\rightarrow\mathbb{R}^n$ of a game $v$ such that: 
    \[C(v)=\left\{x\in\mathbb{R}^n\mid\sum_{i=1}^{n}x_i=v(N)\land\sum_{i\in S}x_i\geq v(S)\quad\forall S\subseteq N\right\}\]
\end{definition}
\noindent The core is a subset of the imputations. 
Imputations are efficient distributions of utilities accepted by all individual players, while core vectors are efficient distributions of utilities accepted by all coalitions. 
\begin{proposition}
    The core $C(v)$ is a polytope.
\end{proposition}
\noindent The core reduces to the singleton $(v(\{1\}),\dots,v(\{n\}))$ if $v$ is additive. 
But superadditive games can have an empty core. 
\begin{definition}[\textit{Veto player}]
    In a game $v$, a player $i$ is a veto player if $v(A)=0$ for all $A$ such that $i\notin A$. 
\end{definition}
\begin{theorem}
    Let $v$ be a simple game. 
    Then $C(v)\neq\varnothing$ if and only if there is at least one veto player. 
    When a veto player exists, the core is the closed convex polytope with the vectors $(0,\dots,1,\dots,0)$ as extreme points, where $1$ corresponds to the veto player. 
\end{theorem}
\begin{proof}
    If there is no veto player, then for every $i$ there is $A_i$ such that $i\notin A_i$ and $v(A_i)=1$. 
    Suppose that $\mathbf{x}\in C(v)$, then it follows that: 
    \[\sum_{j\neq i}x_j\geq\sum_{j\in A_i}=1\qquad\forall i\]
    However, by summing up the above inequalities from $1$ to $n$ one obtains: 
    \[(n-1)\sum_{j=1}^{n}x_j=n\]
    Which yields a contradiction with $\sum_{j=1}^{n}x_j=1$.

    Conversely, and imputation assigning zero to the non-veto players must be in the core. 
\end{proof}
\begin{theorem}
    The Linear programming problem: 
    \begin{align*} 
        \min\:&\sum_{i=1}^{n}x_i \\
        \text{such that}    \:&\sum_{i\in S}x_i\geq v(S)\quad\forall S \subseteq N
    \end{align*}
    Has always a nonempty set of solutions $C$. 
    The core $C(v)$ is nonempty and $C(v)=C$ if and only if the optimal value of the problem is $v(N)$. 
\end{theorem}
\noindent The value $V$ of the Linear Programming problem is $V\geq v(N)$, due to the constraint $\sum_ix_i\geq v(N)$. 
Thus, for every $x$ fulfilling the constraint one has $\sum_{i=1}^{n}x_i\geq v(N)$. 
\begin{theorem}
    $C(v)\neq\varnothing$ if and only if every vector $(\lambda_S)_{S\subset N}$ such that: 
    \[\lambda_S\geq 0 \qquad \forall S\subseteq N\]
    \[\sum_{S\mid i \in S\subseteq N}\lambda_S=1\qquad \forall i\]
    Satisfies also the following inequality: 
    \[\sum_{S\subseteq N}\lambda_Sv(S)\leq v(N)\]
\end{theorem}
\noindent Note that the coefficients $\lambda_S$ can be interpreted as indicating how much a given coalition $S$ represents the players. 
Therefore, the theorem suggests that, no matter what quota the players contribute to the coalition, the weighted values must not exceed the overall amount of utility.
\begin{proof}
    The first Linear Programming problem has the following matrix form: 
    \begin{align*} 
        \min\:&\left\langle c,x \right\rangle  \\
        \text{such that}    \:&Ax\geq b
    \end{align*}
    Here $c=1_n, b=(v(\{1\}),\dots,v(N))$ and $A$ si a $(2^n-1)\times n$ matrix, whose rows are coalitions and columns are players:  $A_{ij}=1$ if player $j$ is in the coalition $i$, $A_{ij}=0$ otherwise. 
    The dual of the problem takes the form: 
    \begin{align*} 
        \max\:&\sum_{S\subseteq N}\lambda_Sv(S) \\
        \text{such that}    \:&\sum_{S\mid i\in S\subseteq N}\lambda_S=1 \qquad\forall i \\
        \:&\lambda_S\geq0
    \end{align*}
    Since the primal has at least one finite solution, the fundamental duality theorem states that this is true also for the dual and there is no duality gap. 
    So, the core $C(v)$ is nonempty if and only if the value $V$ of the dual problem is such that $V\leq v(N)$. 
\end{proof}

\subsection{Balanced coalitions}
\begin{definition}[\textit{Balanced family}]
    A family $(S_1,\dots,S_m)$ of coalitions is called balanced in case there exists $\lambda=(\lambda_1,\dots,\lambda_m)$ such that $\lambda_j>0$ for all $j\in [1,m]$ and for every $i\in N$: 
    \[\sum_{k\mid i\in S_k}\lambda_k=1\]
    $\lambda$ is called balancing vector. 
\end{definition}
\noindent The set of $\lambda_S$ is a convex polytope with a finite number of extreme points. 
\begin{definition}[\textit{Minimal balancing family}]
    A minimal balancing family is a balancing family for which there is no sub-family that is balanced.
\end{definition}
\begin{lemma}
    A balanced family is minimal if and only if its balancing vector is unique. 
\end{lemma}
\begin{theorem}
    The positive coefficient of the extreme points of the constraint set of the dual problem are the balancing vectors of the minimal balanced coalitions.  
\end{theorem}
\noindent To find the extreme points of the dual constraint set it is enough to find balanced families with unique balancing vector. 
The partitions of $N$ are minimal balanced families.
The relevant condition: 
\[\sum_{S\subseteq N}\lambda_Sv(S)\leq v(N)\]
is automatically fulfilled if the game is superadditive. 
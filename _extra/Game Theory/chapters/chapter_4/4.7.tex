\section{Repeated game}

When a game is repeated many times, collaboration between players, eben if dominated in the one shot game, can be based on rationality. 
The common strategy of the Nash equilibria profile has a weakness: it is based on mutual threat of the players, which is not completely credible since by pushing the player who deviates from the agreement the other will also damage himself. 
In general, the number of Nash equilibria profile in the repetition of the game is very large. 

\begin{definition}[\textit{Stage game}]
    A stage game is played with infinite horizon by the players.
\end{definition}
\noindent We need to define the strategy and the payoff. 

\subsubsection{Strategy}
Assume that at each stage $\tau$, the player know which outcome has been selected at stage $\tau-1$.
Thus the strategy for a player is: 
\[s=s(\tau),\tau=0,\dots\]
\noindent Here, for each $\tau$, $s(\tau)$ is a specification of the moves of the stage game, which is in general a function of the past choices of the players.

\subsubsection{Payoff}
In general, it is not possible to sum payoffs obtained at each stage since the sum will be infinite for $\tau=\infty$.
There are different possible choices to construct the payoff function.
One standard choice is to use a discount factor $\delta$, where $0<\delta < 1$. 
So, the utility function becomes: 
\[u_i(s,t)=(1-\delta)\sum_{\tau=0}^{\infty}\delta^\tau u_i(s(\tau),t(\tau))\]
Here,  $u_i(s(\tau),t(\tau))$ is the stage-game payoff of the player $i$ at time $\tau$ given strategy profile $(s(\tau),t(\tau))$. 

\begin{definition}[\textit{Threat value}]
    For the bi-matrix game $(A,B)$ representing the stage game: 
    \[\underbar{v}_1=\min_j\max_i a_{ij}\qquad \underbar{v}_2=\min_i\max_j b_{ij}\]
    Are called threat values of Player 1 and Player 2, respectively. 
\end{definition}
\noindent Note that $v_1$ and $v_2$ are not the conservative values of the two players. 
\begin{theorem}
    For every feasible payoff vector $v=(v_1,v_2)=(a_{\bar{ij}},b_{\bar{ij}})$ such that $v_i>\underbar{v}_i$ where $i=1,2$, there exists $\bar{\delta}<1$ such that for all $\delta>\bar{\delta}$ there is a Nash equilibrium of the repeated game with discounting factor $\delta$, which yields payoffs $v$. 
\end{theorem}
\begin{proof}
    $v=(v_1,v_2)=(a_{\bar{ij}},b_{\bar{ij}})$ such that $v_i>\underbar{v}_i$. 
    Define the following strategy $s$: play the strategy yielding $v$ at any stage, unless the opponents deviates at time $t$. 
    In the latter case play the threat strategy form the stage $t+1$ onwards. 
    We need to prove that $s$ provides utility vector $v$ and $s$ is a Nash equilibrium for all $\delta$ close to $1$. 

    At time $\tau=t$ player $i$ could gain at most $\max_{i,j}a_{ij}$. 
    Denote by $s_t$ the strategy of deviating at time $t$. 
    So, if the Player 1 deviates, after $t$ he will gain at most $\underbar{v}_1$. 
    Hence, the payoff is such that: 
    \begin{align*}
        u_1(s_t)    &\leq(1-\delta)\left(\sum_{\tau=0}^{t-1}\delta^\tau v_1+\delta^t\max_{i,j}a_{ij}+\sum_{\tau=t+1}^{\infty}\delta^\tau\underbar{v}_1\right) 
                    &(1-\delta^t)v_1+(1-\delta)\delta^\tau\max_{i,j}a_{ij}+(\delta^{t+1})\underbar{v}_1
    \end{align*}
    Instead with strategy $s$ the payoff is: 
    \[u_1(s)=(1-\delta)\sum_{\tau=0}^{\infty}\delta^\tau v_1=v_1\]
    Then: 
    \[u_1(s)=v_1\geq u_1(s_t)=(1-\delta^t)v_1+(1-\delta)\delta^\tau\max_{i,j}a_{ij}+(\delta^{t+1})\underbar{v}_1\]
    If and only if: 
    \begin{align*}
        (1-\delta^t)v_1+(1-\delta)\delta^\tau\max_{i,j}a_{ij}+(\delta^{t+1})\underbar{v}_1  &\leq v_1 \\
        (1-\delta)\delta^\tau\max_{i,j}a_{ij}+(\delta^{t+1})\underbar{v}_1                  &\leq \delta^t v_1 \\
        (1-\delta)\max_{i,j}a_{ij}+\delta\underbar{v}_1                                     &\leq  v_1 \\
        \delta(\max_{i,j}a_{ij}-\underbar{v}_1)                                             &\geq  \max_{i,j}a_{ij}-v_1
    \end{align*}
    By properly setting $\boldsymbol{\delta}_{i}=\frac{\max_{i,j}a_{ij}-v_i}{\max_{i,j}a_{ij}-\underbar{v}_{i}}<1$ we have: 
    \[\boldsymbol{\delta}=\max_{i=1,2}\boldsymbol{\delta}_i\]
\end{proof}

\subsection{Correlated equilibrium}
Given a game $(A,B)$ with $n$ strategies for Player 1 and $m$ strategies for Player 2. 
Let $I=\left\{1,\dots,n\right\}$, $J=\left\{1,\dots,m\right\}$, and $X=I\times J$. 
\begin{definition}[\textit{Correlated equilibrium}]
    A correlated equilibrium is a probability distribution $P=(p_{ij})$ on $X$ such that for all $\bar{i}\in I$: 
    \[\sum_{j=1}^{m}p_{\bar{i}j}a_{\bar{i}j}\geq\sum_{j=1}^{m}p_{\bar{i}j}a_{ij}\qquad \forall i\in I\]
    And for all $\bar{j}\in J$: 
    \[\sum_{j=1}^{m}p_{i\bar{j}}b_{i\bar{j}}\geq\sum_{j=1}^{m}p_{i\bar{j}}b_{ij}\qquad \forall j\in J\]
\end{definition}

\subsubsection{Existence}
The set of correlated equilibria of a finite game is nonempty. 
\begin{theorem}
    A Nash equilibrium profile generates a correlated equilibrium. 
\end{theorem}
\noindent Given the Nash equilibrium profile $(\bar{x},\bar{y})$, the probability distribution of the outcome matrix is $p$, where each element is such that $p_{ij}=\bar{x}_i\bar{y}_j$
\begin{proof}
    We have to prove that: 
    \[\sum_{j=1}^{m}\bar{x}_{\bar{i}}\bar{y}_ja_{\bar{i}j}\geq \sum_{j=1}^{m}\bar{x}_{\bar{i}}\bar{y}_ja_{ij}\forall i \in I\]
    That is obvious for $\bar{x}_{\bar{i}}=0$. 
    If $\bar{x}_{\bar{i}}>0$ we need to show that: 
    \[\sum_{j=1}^{m}\bar{y}_{j}a_{\bar{i}j}\geq\sum_{j=1}^{m}\bar{y}_{j}a_{ij}\qquad \forall i\in I\]
    The left (right) hand side is the expected utility of Player 1 is he chooses row $\bar{i}$ ($i$) given that Player 2 chooses his equilibrium strategy $\bar{y}$. 
    The inequality holds since the pure strategy $\bar{i}$ is played with positive probability, hence $\bar{i}$ must be (one of) the best reaction($s$) to $\bar{y}$. 
\end{proof}

\begin{theorem}
    The set of the correlated equilibria of a finite game is a nonempty convex polytope.
\end{theorem}
\begin{proof}
    Remember that a convex polytope is the smallest convex set containing a finite number of points. 
    The set of the correlated equilibria is the solution set of a system $n^2+m^2$ linear inequalities called incentive constraints, plus the condition of being a probability distribution.
\end{proof}
\begin{proposition}
    If a row $\bar{i}$ is strictly dominated, then $P_{\bar{i}j}$ for every $j$. 
\end{proposition}
\begin{proof}
    Suppose $\bar{i}$ is strictly dominated by $i$. 
    Since: 
    \[\sum_{j=1}^{m}p_{\bar{i}j}(a_{\bar{i}j}-a_{ij})\geq 0\]
    It must be $p_{\bar{i}j}$ for every $j$. 
\end{proof}
\noindent The most important conclusion we can draw is that there is essentially a unique rationality paradigm in the whole theory: the idea of best reaction. 
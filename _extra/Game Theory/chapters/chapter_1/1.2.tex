\section{Players}

Players are supposed to be selfish and rational. 

\subsection{Selfish player}
The player only car about their own preferences wrt to the outcome of the game. 
This is a mathematical assumption to define the meaning of rational choice. 

\subsection{Rational player}
\begin{definition}[\textit{Preference relation}]
    A preference relation on a set $X$ is a binary relation $\succeq$ that satisfies the following properties for all $x,y,z\in X$: 
    \begin{itemize}
        \item \textit{Reflexive}: $x \succeq x$.
        \item \textit{Complete}: $x \succeq y$ or $y \succeq x$.
        \item \textit{Transitive}: if $x \succeq y$ and $y \succeq z$, then $x \succeq z$.
    \end{itemize}
\end{definition}

\begin{definition}[\textit{Utility function}]
    Given a preference relation $\succeq$ over a set $X$, a utility function representing $\succeq$ is a function $u:X\rightarrow\mathbb{R}$ such that: 
    \[u(x)\geq u(y)\Leftrightarrow x \succeq y\]
\end{definition}
\noindent While a utility function may not always exist in specific cases, it does exist in general settings, particularly when $X$ is finite.
If a utility function does exist, there are infinitely many such functions, differing by any strictly increasing transformation of the original function.
Each player $i$ is assigned a set $X_i$, representing all the choices available to them.
Therefore, the set $X=xX_i$ over which the utility function $u$ is defined represents the combined choices of all players.

\subsubsection{Rationality assumptions}
The following assumptions define the rational behavior of players:
\begin{enumerate}
    \item The players are able to provide a preference relation over the outcomes of the game, and the order must be consistent. 
    \item The players are able to provide a utility function representing their preferences relations, whenever it is necessary. 
    \item The players use consistently the laws of probability. 
    \item The players are able to understand the consequences of all their actions. 
    \item The players are able to use decision theory, whenever it is possible
\end{enumerate}
\noindent Therefore, given a set of alternatives $X$ and a utility function $u$, each player seeks $\bar{x}\in X$ such that: 
\[u(\bar{x}) \geq u(x)\qquad\forall x \in X\]
From the axioms we can derive the principle of elimination of strictly dominated strategies: a player does not take an action if he has another action providing him a strictly better result, no matter what the other players do. 

\subsection{Actions}
The set of actions can be represented as a set of pair of values that represent the utilities for Player 1 and Player 2, respectively. 
These options can be conveniently displayed in a bi-matrix.
Conventionally, Player 1 selects a row, while Player 2 selects a column. 
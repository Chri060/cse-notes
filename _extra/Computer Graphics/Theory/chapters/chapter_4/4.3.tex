\section{Index primitives}

While triangle strips can offer significant savings, often up to two-thirds compared to triangle lists, many scenarios still entail repeated vertices. 
Due to the inability to encode certain primitives with a single triangle strip, vertices may be shared among different strips. 
Indexed primitives mitigate this issue by reducing the overhead of duplicating vertices across lists or strips.

Consider a sphere, typically composed of multiple strips where each vertex is shared by at least four triangles. 
Although triangle strips can decrease the space needed for each band, the same vertex is reiterated across various bands.

Indexed primitives employ two arrays: the vertex array containing vertex definitions (positions), and the index array specifying triangles indirectly. 
Triangles are drawn based on their indices, with vertex coordinates retrieved from the vertex list according to the index's position.

Indexing achieves substantial byte savings without the complexity of determining correct ordering. 
Consequently, many file interchange formats exclusively support meshes encoded with indexed triangle lists to streamline their architecture.

\paragraph*{Restart}
In Vulkan, the restart feature enables additional space reduction in band-like structures. 
Specifically, a negative index restarts the strip, optimizing the encoding process.

\paragraph*{Lines}
When creating 2D plots and charts, lines are typically employed instead of triangles.

\paragraph*{Wireframe}
Lines are also utilized to generate Wireframe views, showcasing only the outlines of objects by connecting their vertices with lines. 
These views are valuable in numerous scenarios, particularly for debugging 3D applications.

The two primary types of wireframe mesh encoding are:
\begin{itemize}
    \item \textit{Line lists}: each segment is encoded as a pair of separate vertices. 
        Encoding $N$ segments necessitates $2N$ vertices.
    \item \textit{Line strips}: a path of connected vertices is encoded. 
        To represent $N$ segments, $N+1$ vertices are required.
\end{itemize}
Wireframe primitives can also be indexed. 
Additionally, Vulkan supports drawing standard objects solely using the contour of their triangles. 
However, this approach is simply a means to simplify creating a wireframe object corresponding to the mesh's wireframe.
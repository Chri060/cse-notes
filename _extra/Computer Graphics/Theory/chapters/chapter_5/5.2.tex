\section{Back-face culling}

Back-face culling effectively removes faces on the rear side of a mesh by evaluating the vertex order of triangles concerning the viewer, determining if they're arranged clockwise or counterclockwise.

This evaluation can take place before projection or using normalized screen coordinates.
Vulkan's implementation of back-face culling operates within normalized screen coordinates, hence our focus on this approach.

Suppose all triangles of a mesh are consistently encoded with a specific orientation, such as clockwise.
When projected onto the screen, front faces maintain their clockwise order, while back faces are ordered oppositely.

To compute the orientation of triangle vertices in normalized screen coordinates, a straightforward cross-product can be employed.
If the resultant vector points toward the viewer (with a positive z component), the vertices are ordered clockwise. 
Because only the sign of the $z$ component matters, this test can be executed efficiently:
\[\begin{bmatrix}
    u_x & u_y & u_z
\end{bmatrix} \times \begin{bmatrix}
    v_x & v_y & v_z
\end{bmatrix}=\begin{bmatrix}
    u_y{v}_z - u_z{v}_y & u_z{v}_x - u_x{v}_z & u_x{v}_y -u_y{v}_x
\end{bmatrix}\]

\paragraph*{Remarks}
There are transformations that can alter the vertex ordering, thus affecting the effectiveness of back-face culling. 
Additionally, not all 3D asset creation software follows the same convention for defining front faces. 
Therefore, users must have the ability to specify whether faces with vertices ordered clockwise or counterclockwise should be displayed.

While back-face culling can enhance application performance, there are potential issues that need consideration. 
Firstly, if a world matrix involves scaling with an odd number of negative coefficients (either one or all three), the acceptance test must be inverted, indicating that back-face culling may not always be applicable. 
Moreover, in non-2-manifold objects, those with holes or thin faces, certain polygons belonging to the back faces that should be visible might inadvertently be deleted.

If a camera is positioned inside an object where back-face culling is enabled, its boundaries become invisible since their vertices are oriented in the opposite direction. 
To address this, an object delineating the borders of the 3D world area (e.g., a room or a skybox) should be created with vertices ordered in the opposite direction.
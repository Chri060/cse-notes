\section{Area lights}

Many real-world light sources do not emit light from a single point. Instead, they have a finite area from which light emanates. 
Area lights are designed to capture the shape and extent of these light sources within a scene. 
However, incorporating area lights introduces complexity, as individual sources cannot be treated independently anymore. 
In scan-line rendering, this necessitates the use of a full integral rather than simple point calculations.

Current methods for simulating area lights rely on approximations to this integral, as accounting for the complete light shape is computationally intensive. 
Moreover, these approximations are tightly coupled with the Bidirectional Reflectance Distribution Function of surfaces, making it challenging to separate the effects of area lights from surface properties.
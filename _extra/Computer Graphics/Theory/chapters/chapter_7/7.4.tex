\section{Spot light model}

Spotlights are specialized projectors designed to illuminate specific objects or areas within a scene.
They emit light in a conical pattern defined by a direction $\mathbf{d}$ and a position $\mathbf{p}$, with the light starting from point $\mathbf{p}$.
The spotlight's illumination is divided into three zones by two angles, $\alpha_{IN}$ and $\alpha_{OUT}$: constant (inside $\alpha_{IN}$), decay (between $\alpha_{IN}$ and $\alpha_{OUT}$) and absent (outside $\alpha_{OUT}$).
Within the decay zone, light intensity decreases linearly from the inner to the outer angle.
These angles, $\alpha_{IN}$ and $\alpha_{OUT}$, allow for sizing the light to focus its effect on a specific subject. 
Typically, the cosine of half-angles of the inner and outer cones, $c_{in}$ and $c_{out}$, are used for implementation.

The cosine of the angle between the light direction vector $\overrightarrow{lx}$ and the spot direction $\mathbf{d}$ can be calculated by the dot product between the two:
\[\cos\alpha=\overrightarrow{lx}\cdot\mathbf{d}\]
The cone dimming effect is determined as:
\[\text{clamp}\left(\dfrac{\cos\alpha-c_{OUT}}{c_{IN}-c_{OUT}}\right)\]
Where, 
\[\text{clamp}(y)=\begin{cases}
    0 \qquad y<0 \\
    y \qquad y\in[0,1] \\
    1 \qquad y>1
\end{cases}\]
Spotlights modulate other light sources with the introduced dimming term. 
They inherit the light direction $\overrightarrow{lx}_0$ from their parent model and adjust their color $L_0(l, \overrightarrow{lx})$ with the dimming term:
\[L(l,\overrightarrow{lx})=L_0(l,\overrightarrow{lx})\cdot\text{clamp}\left(\dfrac{\frac{\mathbf{p}-\mathbf{x}}{\left\lvert \mathbf{p}-\mathbf{x}\right\rvert}\cdot\mathbf{d}-c_{OUT}}{c_{IN}-c_{OUT}}\right)\]
The most common implementation of spotlights integrates the dimming factor with point lights. 
Here, spotlights are also characterized by a decay factor $\beta$, a target distance $g$, and a color vector $\mathbf{l}$. 
The light direction is computed similarly to point lights.
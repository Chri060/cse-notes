\section{Planes}

In homogeneous coordinates, planes in space can be represented by the vector:
\[\boldsymbol{\pi}=\begin{bmatrix} a \\ b \\c \\d \end{bmatrix}\]
Here, $(a, b, c)$ defines the direction normal to the plane. 
The perpendicular distance from the origin to the plane is given by:
\[\textnormal{distance}=-\dfrac{d}{\sqrt{a^2+b^2+c^2}}\]
Similar to points, plane coordinates in homogeneous form exhibit the property of homogeneity.
Thus, any non-zero scalar multiple $\lambda \boldsymbol{\pi}$, where $\lambda \neq 0$, represents the same plane.
This makes $a, b, c,$ and $d$ the homogeneous parameters of the plane.
As with points, a single plane has infinitely many equivalent representations, all scaled versions of its normal vector. 
The null vector, however, does not represent any plane, and if $d=0$, the plane passes through the origin.

To determine whether a point $\mathbf{X}$ lies on a plane $\boldsymbol{\pi}$ or if a plane passes through a point, the following equation must be satisfied:
\[\begin{cases} ax+by+cz+dw=0 \\ \boldsymbol{\pi}^T\mathbf{X}=\mathbf{X}^T\boldsymbol{\pi}=0 \end{cases}\]
\begin{definition}[\textit{Plane at the infinity}]
    The plane at infinity, $\boldsymbol{\pi}_{\infty}$, is defined as:
    \[\begin{bmatrix} 0 & 0 & 0 & 1 \end{bmatrix} \begin{bmatrix} x \\ y \\ z \\ w \end{bmatrix}=w=0\] 
\end{definition}
This plane has an undefined normal direction and represents points at infinity.
\begin{theorem}[Duality principle]
    For any true statement involving terms like point, plane, is on, and goes through, there exists a corresponding dual statement that is also true, derived by making the following substitutions:
\end{theorem}
\begin{itemize}
    \item \textit{Point $\Leftrightarrow$ plane.}
    \item \textit{Is on $\Leftrightarrow$ goes through.}
\end{itemize}

\paragraph*{Point as intersection of planes}
A point can be described as the intersection of three distinct planes:
\[\begin{cases}
    \boldsymbol{\pi}_1^T\mathbf{X}=0 \\
    \boldsymbol{\pi}_2^T\mathbf{X}=0 \\
    \boldsymbol{\pi}_3^T\mathbf{X}=0 
\end{cases}\implies\begin{bmatrix} \boldsymbol{\pi}_1^T \\ \boldsymbol{\pi}_2^T \\ \boldsymbol{\pi}_3^T \end{bmatrix}\mathbf{X}=\begin{bmatrix} 0 \\ 0 \\ 0 \end{bmatrix}\]
This setup finds the Right Null Space of the matrix of planes:
\[\mathbf{X}=\text{RNS}\left(\begin{bmatrix} \boldsymbol{\pi}_1^T \\ \boldsymbol{\pi}_2^T \\ \boldsymbol{\pi}_3^T \end{bmatrix}\right)\]
Yielding a solution vector representing the intersection point, along with all of its scalar multiples.

\paragraph*{Plane trough tree points}
A plane passing through three points $\mathbf{X}_1^T$, $\mathbf{X}_2^T$, and $\mathbf{X}_3^T$ can be determined by finding the Right Null Space of the matrix of points:
\[\mathbf{X}=\text{RNS}\left(\begin{bmatrix} \mathbf{X}_1^T \\ \mathbf{X}_2^T \\ \mathbf{X}_3^T \end{bmatrix}\right)\]
\begin{property}
    The plane $\boldsymbol{\pi}$, defined by the linear combination $\boldsymbol{\pi} = \alpha \boldsymbol{\pi}_1 + \beta\boldsymbol{\pi}_2$ of two planes $\boldsymbol{\pi}_1$ and $\boldsymbol{\pi}_2$, passes through the line $\mathbf{l}^\ast$, which lies on both $\boldsymbol{\pi}_1$ and $\boldsymbol{\pi}_2$. 
\end{property}
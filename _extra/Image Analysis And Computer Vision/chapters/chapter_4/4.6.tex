\section{Transformations}

\begin{definition}[\textit{Projective Mapping}]
    A projective mapping between a projective space $\mathbb{P}^3$ and another projective space $\mathbb{P}^{\prime 3}$ is an invertible mapping that preserves collinearity of points.
\end{definition}

\begin{theorem}
    A mapping $h : \mathbb{P}^3 \rightarrow \mathbb{P}^{\prime 3}$ is projective if and only if there exists an invertible $4 \times 4$ matrix $\mathbf{H}$ such that for any point $\mathbf{X}$ in $\mathbb{P}^3$, we have:
    \[h(\mathbf{X}) = \mathbf{H}\mathbf{X}\]
\end{theorem}

Projective mappings are linear in homogeneous coordinates, but not linear in Cartesian coordinates. 
From the theorem, we have the relation:
\[h(\mathbf{X}) = \mathbf{X}^\prime = \mathbf{H} \mathbf{X}\]
Thus, if we multiply the matrix $\mathbf{H}$ by any nonzero scalar $\lambda$, the relation still holds for the same points:
\[\mathbf{X}^\prime = \lambda \mathbf{H} \mathbf{X}\]
Therefore, any nonzero scalar multiple of $\mathbf{H}$ represents the same projective mapping as $\mathbf{H}$. 
This implies that $\mathbf{H}$ is a homogeneous matrix: despite having 16 entries, $\mathbf{H}$ only has 15 degrees of freedom, corresponding to the ratios between its elements.

A homography transforms each point $\mathbf{X}$ into a point $\mathbf{X}^\prime$ such that:
\[\mathbf{X}^\prime = \mathbf{H} \mathbf{X}\]

Additionally, a homography transforms each plane $\boldsymbol{\pi}$ into a plane $\boldsymbol{\pi}^\prime$ such that:
\[\boldsymbol{\pi}^\prime = \mathbf{H}^{-T} \boldsymbol{\pi}\]

A homography transforms each quadric $\mathbf{Q}$ into a quadric $\mathbf{Q}^\prime$ such that:
\[\mathbf{Q}^\prime = \mathbf{H}^{-T} \mathbf{Q} \mathbf{H}^{-1}\]

Similarly, a homography transforms each dual quadric $\mathbf{Q}^*$ into a dual quadric $\mathbf{Q}^{*^\prime}$ such that:
\[\mathbf{Q}^{*^\prime} = \mathbf{H} \mathbf{Q}^* \mathbf{H}^T\]

Cross-ratios are invariant under projective mappings.

\subsection{Isometries}
An isometry is a transformation that preserves distances. 
The matrix representation of an isometry $\mathbf{H}_I$ is given by:
\[\mathbf{H}_I = \begin{bmatrix}
    \mathbf{R}_{\perp} & & \mathbf{t} & \\
    0 & 0 & 0 & 1
\end{bmatrix}\]
where $\mathbf{R}_{\perp}$ is a $3 \times 3$ orthogonal matrix, meaning that:
\[\mathbf{R}_{\perp}^{-1} = \mathbf{R}_{\perp}^T \quad \text{and} \quad \det(\mathbf{R}_{\perp}) = \pm 1\]
The sign of the determinant indicates whether the transformation is a rigid displacement ($\det(\mathbf{R}_{\perp}) = 1$) or includes a reflection ($\det(\mathbf{R}_{\perp}) = -1$).

Isometries have 6 degrees of freedom: 3 for translation ($\mathbf{t}$) and 3 for rotation (Euler angles $\theta$, $\phi$, $\psi$).

Invariants of isometries include lengths, distances, and areas, which means that the shape and size of objects are preserved, as well as their relative positions.

\subsection{Similarities}
A similarity transformation includes both rigid motion and scaling. 
The matrix representation of a similarity $\mathbf{H}_S$ is:
\[\mathbf{H}_S = \begin{bmatrix}
    s \mathbf{R}_{\perp} & & \mathbf{t} & \\
    0 & 0 & 0 & 1
\end{bmatrix}\]
where $s$ is a scalar that represents the scale factor and $\mathbf{R}_{\perp}$ is a $3 \times 3$ orthogonal matrix. 
The matrix $\mathbf{R}_{\perp}$ satisfies:
\[\mathbf{R}_{\perp}^{-1} = \mathbf{R}_{\perp}^T \quad \text{and} \quad \det(\mathbf{R}_{\perp}) = \pm 1\]
Similarities have 7 degrees of freedom: 3 for rigid displacement (translation and rotation) and 1 for scaling.

Invariants of similarities include the ratio of lengths and angles, which preserve the shape (but not the size) of objects. 
Additionally, the absolute conic $\Omega_{\infty}$ and the absolute dual quadric $\mathbf{Q}^*_{\infty}$ remain invariant under similarity transformations.

\subsection{Affinities}

An affinity is a transformation that preserves parallelism and the ratio of distances along parallel lines. 
The matrix representation of an affinity $\mathbf{H}_A$ is:
\[\mathbf{H}_A = \begin{bmatrix}
    \mathbf{A} & & \mathbf{t} & \\
    0 & 0 & 0 & 1
\end{bmatrix}\]
where $\mathbf{A}$ is any invertible $3 \times 3$ matrix, and $\mathbf{t}$ is a translation vector. 
Affinities have 12 degrees of freedom: 9 for the matrix $\mathbf{A}$ and 3 for the translation vector $\mathbf{t}$.

Invariants of affinities include parallelism, the ratio of lengths along parallel lines, and the ratio of areas. 
A key invariant is the plane at infinity $\boldsymbol{\pi}_{\infty}$, which remains unchanged under affine transformations.

\subsection{Projectivities}
A projectivity is a transformation that preserves collinearity and incidence relations. 
The matrix representation of a projectivity $\mathbf{H}_P$ is:
\[\mathbf{H}_P = \begin{bmatrix}
    \mathbf{A} & \mathbf{t} \\
    \mathbf{v}^T & 1
\end{bmatrix}\]
where $\mathbf{A}$ is any invertible $3 \times 3$ matrix, $\mathbf{t}$ is a translation vector, and $\mathbf{v}$ is a vector that represents the relation between the homogeneous coordinates of the points.

Projectivities have 15 degrees of freedom: 9 for the matrix $\mathbf{A}$, 3 for the translation $\mathbf{t}$, and 3 for the vector $\mathbf{v}$.

Invariants of projectivities include collinearity, incidence, and the order of contact (such as crossing, tangency, and inflections). Additionally, the 1D, 2D, and 3D cross ratios are invariant under projective transformations.

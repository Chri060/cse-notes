\section{ISO definitions}
\begin{definition}[\textit{Robot}]
    A robot is an actuated mechanism programmable in two or more axes with a degree of autonomy, moving within its environment, to perform intended tasks. 
    Autonomy in this context means the ability to perform intended tasks based on current state and sensing, without human intervention.
\end{definition}

\begin{definition}[\textit{Service robot}]
    A service robot is a robot that performs useful tasks for humans or equipment excluding industrial automation application. 
\end{definition}
\begin{definition}[\textit{Personal service robot}]
    A personal service robot or a service robot for personal use is a service robot used for a noncommercial task, usually by lay persons. 
\end{definition}
Examples of personal service robots include domestic servant robots, automated wheelchairs, personal mobility assist robots, and pet exercising robots.
\begin{definition}[\textit{Professional service robot}]
    A professional service robot or a service robot for professional use is a service robot used for a commercial task, usually operated by a properly trained operator.
\end{definition}
Examples of professional service robots encompass cleaning robots for public spaces, delivery robots in offices or hospitals, fire-fighting robots, rehabilitation robots, and surgical robots in hospitals.
In this context, an operator is an individual designated to initiate, oversee, and terminate the intended operation of a robot or robot system.

\paragraph*{Notes}
A robot system is defined as a system comprising robots, end-effectors, and any machinery, equipment, or sensors that support the robot in performing its tasks.

According to this definition, service robots require a degree of autonomy, which can range from partial autonomy involving human-robot interaction to full autonomy without active human intervention. 
Human-robot interaction involves information and action exchanges between humans and robots via a user interface to accomplish tasks.

Industrial robots, whether fixed or mobile, can also be considered service robots if they are utilized in non-manufacturing operations. 
Service robots may or may not feature an arm structure, which is common in industrial robots. 
Additionally, service robots are often mobile, but this is not always the case.

Some service robots consist of a mobile platform with one or several arms attached, controlled similarly to industrial robot arms. 
Unlike their industrial counterparts, service robots do not necessarily need to be fully automatic or autonomous. 
Many of these machines may assist a human user or operate via teleoperation.
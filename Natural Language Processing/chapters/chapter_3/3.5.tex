\section{Ranking}

In web search, search engines rely on a variety of indicative features to determine the most relevant results for a user query.
Search engines combine hundreds of signals to generate the most relevant search results. 
To do this efficiently, rank learning offers an automated and coherent method of combining diverse signals into a single retrieval score. 
It optimizes this score based on metrics that are important to users.

\subsection{Re-ranking}
The re-ranking process follows these steps: 
\begin{enumerate}
    \item \textit{Start with a query}: the user enters a search query.
    \item \textit{Generate initial ranking}: use keyword-based ranking to retrieve an initial set of results.
    \item \textit{Truncate ranking}: limit the ranking to a manageable number of candidates for further evaluation.
    \item \textit{Calculate feature values}: compute relevant features for each candidate document.
    \item \textit{Normalize features}: normalize each feature at the query level to make the comparison between documents more consistent.
    \item \textit{Training}: use ground truth relevance labels to train the model.
\end{enumerate}

\subsection{Ranking metrics}
There are several metrics used to evaluate the performance of search engine rankings. 
Here's a breakdown:
\begin{itemize}
    \item \textit{Precision at depth $k$}: the percentage of relevant documents in the top $k$ results: 
        \[\text{precision}=\dfrac{\text{number of relevant documents in top }k}{k}\]
    \item \textit{Recall at depth $k$}: the percentage of all relevant documents that are found in the top k results.
        \[\text{recall}=\dfrac{\text{number of relevant documents in top }k}{\text{total relevant documents}}\]
    \item \textit{F-measure at depth $k$}: a combination of precision and recall, providing a single score that balances both: 
        \[\text{F1}=2\dfrac{\text{precision}\cdot\text{recall}}{\text{precision}+\text{recall}}\]
    \item \textit{Mean Average Precision}: this is the average of the precision values at each rank position where a relevant document appears. 
        It estimates the area under the precision-recall curve: 
        \[\text{MAP}=\dfrac{\sum_{k=1}^n(\text{precision}(k)\cdot\text{rel}(k))}{\text{total number of relevant documents}}\]
    \item \textit{Normalized Discounted Cumulative Gain}: this metric is more faithful to the user experience, as it gives lower ranks less importance, aligning with the natural way users interact with search results. 
        It's normalized at the query level:
        \[\text{NDCG}(Q,k)=\dfrac{1}{\left\lvert Q \right\rvert}\sum_{j=1}^{\left\lvert Q \right\rvert}Z_{kj}\sum_{m=1}^k\dfrac{2^{R(j,m)}-1}{\log_2(1+m)}\]
\end{itemize}

\subsection{Regression re-ranking}
Re-ranking can be treated as a regression problem, where the goal is to predict the relevance of a document based on various features. 
The model can then use standard regression techniques to combine these features and predict relevance scores.
During training, the loss function can be defined in three different ways:
\begin{itemize}
    \item \textit{Point-wise}: this approach calculates the loss based on individual query-document pairs (Mean Squared Error).
    \item \textit{Pairwise}: this method considers the relative ordering of document pairs (number of incorrectly ordered pairs).
    \item \textit{List-wise}: this considers the entire ranking list (optimizing NDCG at the query level).
\end{itemize}
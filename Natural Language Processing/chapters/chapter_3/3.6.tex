\section{Text clustering}

Text clustering is the process of grouping documents into coherent subgroups based on similarities in their content. 
These subgroups can be based on various criteria.

The key challenge in text clustering is to accurately measure the similarity between documents. 
This similarity is typically determined by analyzing the content of the documents.
Traditionally, documents are represented as tf-idf vectors, which are scaled bag-of-words representations.
In this approach:
\begin{itemize}
    \item \textit{Sparse representation}: most of the values in the vector are zero, meaning that each document is represented by a vector with many empty or zero entries.
    \item \textit{Similarity measure}: the similarity between documents is usually based on the number of shared words, with the importance of each word being scaled by its rarity.
\end{itemize}

\subsection{Clustering algorithms}
There are several popular clustering algorithms used for grouping documents based on similarity.

\subsubsection{k-means}
$k$-means searches for exactly $k$ clusters, each represented by a centroid, in the following way: 
\begin{enumerate}
    \item Randomly initialize $k$ centroids.
    \item Assign each data point to the nearest centroid.
    \item Recompute the centroids by averaging the data points in each cluster.
    \item epeat steps 2 and 3 until convergence.
\end{enumerate}
\noindent It scales well to large datasets and does not need pairwise distance calculations needed, which makes it faster.
However, it assumes clusters are globular, which may not be ideal for text data, relies on the Euclidean distance metric, which might not be the best choice for all types of data.

The number of clusters must be specified in advance.
Choosing the right value of $k$ can be tricky, though the elbow method can help.
The algorithm can converge on a local minimum; running it multiple times can help mitigate this issue.
Scaling of the data affects clustering results, especially for text, where tf-idf weighting and document length normalization are important.

\subsubsection{k-metoids}
$k$-medoids is similar to $k$-means, but it uses medoids instead of centroids.
A medoid is a point from the dataset itself that is closest to all other points in the cluster.
In each iteration, the algorithm reassigns data points to the cluster with the closest medoid and then recomputes the medoids.

It is flexible because it works with various distance metrics. 
Since a medoid is an actual data point (not a mean), it provides a more realistic representation of the cluster.
However, it has higher computational complexity compared to $k$-means because it requires calculating distances between pairs of points, which is an $\mathcal{O}(n^2)$ operations.

\subsubsection{Agglomerative hierarchical clustering}
Agglomerative hierarchical clustering is a hierarchical clustering builds a hierarchy of clusters, known as a dendrogram.
Agglomerative hierarchical clustering works by merging smaller clusters bottom-up in the following way: 
\begin{enumerate}
    \item Assign each document to its own group.
    \item Merge the two most similar groups.
    \item Repeat until only one group remains.
\end{enumerate}
\noindent To merge clusters, hierarchical clustering computes distances between them using various linkage criteria:
\begin{itemize}
    \item \textit{Complete-linkage}: maximum distance between points in two groups.
    \item \textit{Single-linkage}: minimum distance across groups.
    \item \textit{Average-linkage}: average distance across groups.
\end{itemize}
\noindent The choice of linkage criteria affects the shape and tightness of the clusters. 
Complete or Average-linkage tends to create tight, globular clusters.
Single-linkage can lead to long, thin clusters.

One advantage is that it works with any distance metric or similarity function.
Thus, the dendrogram provides useful insight into the structure of the data.
However, it has an high time complexity makes it unsuitable for large datasets.

\subsubsection{Density-Based Spatial Clustering of Applications with Noise}
DBScan is a density-based clustering algorithm that does not require the number of clusters to be specified in advance.
It has two key parameters, namely $\varepsilon$ (adius of the neighborhood around each point) and $\text{minPoints}$ (minimum number of points required to form a cluster). 
The algorithm classifies points as: 
\begin{itemize}
    \item \textit{Core points}: points with at least $\text{minPoints}$ within their $\varepsilon$-neighborhood.
    \item \textit{Border points}: points that are not core points but lie within the $\varepsilon$-neighborhood of a core point.
    \item \textit{Noise points}: points that are neither core nor border points.
\end{itemize}
\noindent In this algorithm we do not need need to specify the number of clusters upfront, it is robust to noise and outliers, and can find arbitrary-shaped clusters, making it highly flexible.
However, the performance depends on the chosen parameters and may not work well when clusters have different densities.

\subsection{Topic modelling}
Topic modeling is a soft clustering technique for documents, meaning that each document can belong to multiple clusters (or topics) to varying degrees. 
It is a way to extract the underlying themes or topics from a collection of documents.

Topic modeling reduces the dimensionality of documents by representing them in a lower-dimensional space compared to the high-dimensional vocabulary space.
Each topic is described by a probability distribution over words, where the terms should ideally reflect a single theme. 
Similarly, documents are represented as probability distributions over topics.

\paragraph*{Matrix decomposition}
Topic modeling is often approached as a form of matrix decomposition. 
The general idea is to decompose a term-document matrix (which represents word counts or tf-idf scores for each term in each document) into smaller matrices representing $\text{terms}\cdot\text{topics}$ and $\text{topics}\cdot\text{documents}$.
Instead of dealing with a large $V\times D$ matrix (where $V$ is the vocabulary size and $D$ is the number of documents), we decompose it into two much smaller matrices: 
\[V\times T\quad  (\text{terms}\cdot\text{topics}) \qquad T\times D\quad (\text{topics}\cdot\text{documents})\]
Here, $V$ is the vocabulary size, $T$ the number of topics, and $D$ the documents count. 
This decomposition reduces the number of parameters that need to be estimated, making the topic modeling process more efficient.
The most used modeling techniques are: 
\begin{itemize}
    \item \textit{Latent Dirichlet Allocation}: LDA is the most famous technique in topic modeling. 
        It uses a Dirichlet prior to estimate the parameters.
        In LDA, each document is assumed to be a mixture of topics, and each topic is a mixture of words.
    \item \textit{Non-negative Matrix Factorization}: a related technique to LDA, which factorizes the document-term matrix into two non-negative matrices representing term-to-topic and topic-to-document relations.
    \item \textit{Latent Semantic Indexing}: LSI applies Singular Value Decomposition to a TF-IDF matrix to uncover latent semantic structure.
\end{itemize}

\paragraph*{Applications}
Topic modeling helps address several challenges:
\begin{itemize}
    \item \textit{Polysemy}: it can disambiguate words with multiple meanings.
    \item \textit{Synonymy}: it identifies synonyms that might be used interchangeably.
    \item \textit{Short documents}: it improves the representation of documents that may have limited vocabulary due to their short length.
\end{itemize}
\noindent In addition to its utility in improving document representation, topic modeling is also useful for dimensionality reduction. 
After modeling topics, further dimensionality reduction techniques can help visualize collections of documents in a more interpretable way.

\subsubsection{Generative model}
The Generative Model for Latent Dirichlet Allocation is a probabilistic process that describes how the words in a document are generated from topics. 
Here's how it works:
\begin{enumerate}
    \item \textit{Choose word proportions for each topic}: each topic is associated with a probability distribution over words.
    \item \textit{Choose topic proportions for each document}: each document is modeled as a distribution over topics.
    \item \textit{For each word in a document}: choose a topic based on the document's topic proportions and choose a word based on the topic's word distribution.
\end{enumerate}
\noindent Estimating the parameters of the topic model involves updating the topic and word distributions iteratively.
This is typically done using Bayesian priors to avoid overfitting, ensuring that the model doesn't just memorize the data but rather generalizes well.
Gibbs sampling or other sampling techniques are used to avoid local maxima during parameter optimization.

The hyperparameters of this method are: 
\begin{itemize}
    \item $\alpha$: the prior on the topic distribution for each document (controls how concentrated the topic distribution is).
    \item $\beta$: the prior on the word distribution for each topic (controls how concentrated the word distribution is).
\end{itemize}
By iteratively updating these parameters and sampling from the distribution, the model learns the underlying structure of topics in a collection of documents.
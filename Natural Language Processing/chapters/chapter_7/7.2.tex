\section{Usage}

LLMs can be designed to interact with external tools by following a few key mechanisms:
\begin{itemize}
    \item \textit{Instruction tuning}: modern LLMs are often fine-tuned with instructions that teach them how to use tools. 
        During training, they are exposed to examples that demonstrate tool invocation and response integration.
    \item \textit{Declaring available tools}: at the beginning of an interaction, the available tools can be declared explicitly. 
        The LLM is informed of each tool's purpose and usage pattern, enabling it to make appropriate decisions during the conversation.
    \item \textit{Specialized syntax for tool use}: interactions with tools are formatted using special tokens or syntactic conventions, allowing the LLM to differentiate between user inputs, tool calls, and tool outputs. 
        Tools are treated as participants in the dialogue, enabling a seamless multi-turn workflow.
    \item \textit{Backend parsing and execution}: while the LLM generates text to indicate a tool call, the underlying system interprets this output, executes the corresponding tool or API call, and feeds the result back into the conversation.
\end{itemize}

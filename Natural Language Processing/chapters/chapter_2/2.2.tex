\section{Text classification}

Text classification involves training a model to assign documents to specific categories. 
It's a widely-used task across various fields. Classification problems can take different forms, including:
\begin{itemize}
    \item \textit{Binary classification}: where the output is either one of two possible labels.
    \item \textit{Ordinal regression}: where the output is an ordered value, representing categories with a natural rank.
    \item \textit{Multi-class classification}: where the output corresponds to one category from a set of predefined options.
    \item \textit{Multi-label classification}: where the output is a set of categories that can overlap or be chosen simultaneously.
\end{itemize}

\subsection{Feature extraction}
Text can be arbitrarily long, so it can't be fed directly into a model.
To make text usable for ML, we first need to extract meaningful features.
Features are the useful signals in the document that help predict its category. 
To do this, we need to convert text data into a vector of features that a classifier can process.

When training data is limited (with only a few documents available), some approaches to feature extraction include:
\begin{itemize}
    \item \textit{Syntax-based features}: such as the number of capitalized words.
    \item \textit{Part-of-speech features}: like the count of verbs versus proper nouns.
    \item \textit{Reading difficulty features}: such as average word or sentence length
\end{itemize}
\noindent However, the most common and effective features to extract are the words themselves. 
The vocabulary of the document provides significant signals.
The frequency of word occurrences offers additional context.

One popular method is the Bag-of-Words (BoW) model.
This model represents documents as vectors of word counts.
It results in a sparse representation (long vectors with many zero entries).

\paragraph*{One-hot encoding}
One-hot encoding could be an option to create fixed-dimensional feature vectors. 
However, there are practical limitations: represents each word as a binary feature, creating a vector where each dimension corresponds to a word in the vocabulary.
To solve this, we often sum the one-hot encodings. 
This reduces the number of features to the size of the vocabulary.
While this discards word order, it retains the critical information about the vocabulary and word occurrences.

\subsection{Word frequencies}
In text data, certain statistical laws describe how term frequencies behave across documents and collections:
\begin{itemize}
    \item \textit{Heap's law}: this law states that the vocabulary size grows with the square root of the document or collection length:
        \[V(l)\propto l^{\beta}\]
        Here, $\beta \approx 0.5$. 
        This means the number of unique words in a document or collection increases slowly as the length of the document or the size of the collection grows.
    \item \textit{Zipf's law}: this law describes the frequency of a token being inversely proportional to its rank:
        \[\text{ctf}_t\propto\dfrac{1}{\text{rank}(t)^s}\]
        Here, $s\approx 1$. 
        In simple terms, a small number of words (the most frequent ones) appear very often, while a large number of words appear very rarely. 
\end{itemize}
\noindent Heap's law is derived from Zipf's law and can be understood through models like random typing, showing how the vocabulary of a document or collection grows more slowly compared to its length.

\paragraph*{Bag of Words}
The Bag of Words model represents a document as a collection of its terms, ignoring grammar and word order but keeping track of the frequency of each word. 
In this model:
\begin{itemize}
    \item The vocabulary of a document is much smaller than the vocabulary of the entire collection, so the terms in the document generally give a good representation of its content.
    \item The BoW representation typically includes the count of occurrences of each term, although a binary representation (indicating presence or absence of words) can also be used with minimal loss of information.
\end{itemize}
\noindent However, the BoW model has limitations. 
It produces a sparse representation of the text, meaning most of the values in the vector are zero.
The model completely ignores word order, which means it cannot capture the sequence or context in which words appear.

To improve this, we can extend BoW to include $n$-grams, which capture sequences of words. 
This can enhance performance, but it significantly increases the number of features, requiring more data to avoid overfitting.

\subsection{Preprocessing}
In NLP, preprocessing is a critical step to prepare the text data for ML. 
Common preprocessing tasks include tokenization, spelling correction, and other forms of cleaning the text.

\paragraph*{Spelling correction}
When dealing with text data, misspellings can often occur, especially with user-generated content. 
Correcting these misspellings can improve model performance. One way to approach spelling correction is by using a probabilistic model.

If we had an enormous corpus of misspellings and their correct versions, we could estimate the probability of a misspelling being corrected in a certain way by using string edit distance, which measures how much one string differs from another.
The edit distance counts the minimum number of operations (insertions, deletions, substitutions, or transpositions) needed to convert one string into another.

To apply this to spelling correction, we use Bayes' Rule to reverse the conditional probability:
\begin{itemize}
    \item The likelihood of a misspelling being corrected to a particular word can be computed using the edit distance between the misspelled word and the candidate correction.
    \item The prior $\Pr(\text{correct})$ represents the popularity or frequency of the word in a large corpus. 
        This can be estimated by counting how often the candidate correction appears in a corpus.
        \item The likelihood $\Pr(\text{observed}\mid\text{correct})$ represents the probability of observing the misspelled word given the correct word.
\end{itemize}
In practical terms, the prior shows how often a word appears in a large corpus, while the likelihood shows how likely it is that the observed misspelling corresponds to a particular correction, based on string edit distance.
Since the denominator in Bayes' Rule is the same for all candidate corrections, we can ignore it during the calculation and normalize probabilities later.

To improve spelling correction, we can incorporate contextual information.
This can be done by considering bi-grams (pairs of consecutive words) instead of just individual words (uni-grams).
By calculating the bi-gram probabilities $\Pr(\text{bigram})$, the model can use both the misspelled word and the previous word in the sentence as features.
This approach turns the spelling correction process into a Naïve Bayes model with two features: the misspelled word and the previous word in the sentence.
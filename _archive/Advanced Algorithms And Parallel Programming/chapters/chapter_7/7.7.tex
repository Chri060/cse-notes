\section{Gather pattern}

The gather pattern is a mechanism for creating a new collection of data by extracting elements from an existing source collection based on a set of indices. 
This operation is particularly useful in parallel and distributed computing for data reorganization and preparation tasks.

Given a collection of ordered indices, the gather pattern reads data from the source collection at each specified index, and wrtites the retrieved data to the output collection, maintaining the order of indices.
The resulting output collection has the same element type as the source collection, and the same dimensionality and shape as the index collection.

\subsection{Implementation}
The main gather patterns include the following:
\begin{itemize}
    \item \textit{Zip}: combines multiple collections into a single collection of tuples, preserving corresponding elements across input arrays. 
    \item \textit{Unzip}: separate a collection of tuples into individual sub-collections.
    \item \textit{Shift}: move data within a collection by fixed offsets. 
\end{itemize}
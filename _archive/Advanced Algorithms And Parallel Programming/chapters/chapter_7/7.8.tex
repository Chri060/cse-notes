\section{Scatter pattern}

The scatter pattern involves distributing data from an input collection to specific locations in an output collection based on a set of write locations. 
The output collection size may need to be larger to accommodate all write locations without overlap.

\paragraph*{Scatter-gather conversion}
Scatter operations are generally more expensive than gather operations due to cache-related challenges. 
These issues can be mitigated by converting scatter operations into gather operations when the write addresses are known in advance. 
This allows optimizations to be applied, particularly when the scatter pattern is repeated, amortizing the conversion cost.

\subsection{Collisions}
Writes to the same location are possible, leading to potential collisions.
Race conditions occur when two or more values are written to the same location in the output collection. 
This results in undefined behavior unless specific rules are enforced to resolve these collisions.

\paragraph*{Collision rules}
Several strategies can be applied to resolve collisions in scatter operations:
\begin{itemize}
    \item \textit{Atomic}: a non-deterministic approach where one and only one of the colliding values is written.
    \item \textit{Permutation}: collisions are detected in advance. 
    \item \textit{Merge}: associative and commutative operators are used to combine colliding elements.
    \item \textit{Priority}: each input element is assigned a priority based on its position.
\end{itemize}
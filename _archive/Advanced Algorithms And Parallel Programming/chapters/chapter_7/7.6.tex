\section{Pack pattern}

The pack operation is used to eliminate unused elements from a collection and consolidate retained elements into contiguous memory locations. 
This reorganization improves performance by enhancing data locality and reducing cache misses.
The pack operation follows these steps:
\begin{enumerate}
    \item Convert the input array into Booleans, where 1 represents retained elements and 0 represents discarded ones.
    \item Perform an exclusive scan on this integer array using the addition operation to compute offsets.
    \item Write retained values to the output array based on the computed offsets.
\end{enumerate}

\subsection{Implementation}
The main pack patterns include the following:
\begin{itemize}
  \item \textit{Unpack}: restores the original structure of the data by spreading retained elements back to their initial locations. 
  \item \textit{Split}: generalizes the pack operation by moving elements into distinct regions of the output collection based on a classification state.
    Unlike pack, split retains all input data and does not discard any information.
  \item \textit{Unsplit}: reconstruct the original input collection from the separated data. 
  \item \textit{Bin}: extends the split operation to support more than two categories. 
  \item \textit{Expand}: elements meeting specific criteria are retained and packed together.
\end{itemize}
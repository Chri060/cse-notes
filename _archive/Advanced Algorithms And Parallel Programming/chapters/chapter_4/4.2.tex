\section{Parallel Random Access Machine}

A Parallel Random Access Machine (PRAM) is an abstract machine designed to model algorithms for parallel computing. 
\begin{definition}[\textit{Parallel Random Access Machine}]
    A Parallel Random Access Machine (PRAM) is defined as a system $M^\prime = \langle M, X, Y, A \rangle$, where:
\end{definition}
\begin{itemize}
    \item $M$ represent an infinite collection of identical RAM processors without memory. 
    \item $X$ represent the system's input.
    \item $Y$ represent the system's output.
    \item $A$ are shared memory cells between processors.
\end{itemize}
The set of RAMs $M$ contains an unbounded collection of processors $P$, that have unbounded registers for internal storage. 
The set of shared memory cells $A$ is unbounded and can be accessed in constant time.
This set is used by the processors $P$ to communicate with each other. 

\subsection{Computation}
The computation in a PRAM consists of five phases, carried out in parallel by all processors. 
Each processor performs the following actions:
\begin{enumerate}
    \item Reads a value from one of the input cells $X_i$.
    \item Reads from one of the shared memory cells $A_i$.
    \item Performs some internal computation.
    \item May write to one of the output cells $Y_i$.
    \item May write to one of the shared memory cells $A_i$.
\end{enumerate}
Some processors may remain idle during computation. 

\paragraph*{Conflicts}
Conflicts can arise in the following scenarios:
\begin{itemize}
    \item \textit{Read conflicts}: two or more processors may simultaneously attempt to read from the same memory cell.
    \item \textit{Write conflicts}: two or more processors attempt to write simultaneously to the same memory cell.
\end{itemize}
PRAM models are classified based on their ability to handle read/write conflicts, offering both practical and realistic classifications:
\renewcommand*{\arraystretch}{2}
\begin{table}[H]
    \centering
    \begin{tabular}{|l|l|}
    \hline
    \multicolumn{1}{|c|}{\textbf{PRAM model}} & \multicolumn{1}{c|}{\textbf{Operation}} \\ \hline
    Exclusive Read                            & Read from distinct memory locations     \\
    Exclusive Write                           & Write to distinct memory locations      \\
    Concurrent Read                           & Read from the same memory locations     \\
    Concurrent Write                          & Write to the same memory locations      \\ \hline
    \end{tabular}
\end{table}
\renewcommand*{\arraystretch}{1}
When a write conflict occurs, the final value written depends on the conflict resolution strategy:
\begin{itemize}
    \item \textit{Priority CW}: processors are assigned priorities, and the value from the processor with the highest priority is written.
    \item \textit{Common CW}: all processors are allowed to complete their write only if all values to be written are equal.
    \item \textit{Arbitrary CW}: a randomly chosen processor is allowed to complete its write operation.
\end{itemize}

\subsection{Conclusion}
The PRAM model is both attractive and important for parallel algorithm designers for several reasons:
\begin{itemize}
    \item \textit{Natural}: the number of operations executed per cycle on $P$ processors is at most $P$.
    \item \textit{Strong}: any processor can access and read/write any shared memory cell in constant time.
    \item \textit{Simple}: it abstracts away communication or synchronization overhead.
    \item \textit{Benchmark}: if a problem does not have an efficient solution on a PRAM, it is unlikely to have an efficient solution on any other parallel machine.
\end{itemize}
Some possible variants of the PRAM machine model are: 
\begin{itemize}
    \item \textit{Bounded number of shared memory cells}: when the input data set exceeds the capacity of the shared memory, values can be distributed evenly among the processors.
    \item \textit{Bounded number of processors}: if the number of execution threads is higher than the number of processors, processors may interleave several threads to handle the workload.
    \item \textit{Bounded size of a machine word}: limits the size of data elements that can be processed in a single operation.
    \item \textit{Handling access conflicts}: constraints on simultaneous access to shared memory cells must be considered.
\end{itemize}
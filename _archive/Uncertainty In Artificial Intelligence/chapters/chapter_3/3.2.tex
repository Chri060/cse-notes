\section{Propositional logic}

Propositional logics revolve around propositional operators that can be applied to one or more propositions to generate new propositions. 
The primary focus lies on the truth value of propositions and how these truth values are combined.
\begin{definition}[\textit{Truth functional logic}]
    A logic is considered truth functional if the truth value of a compound sentence depends solely on the truth values of the individual atomic sentences, without regard to their meaning or structure. 
    For such a logic, the critical question regarding propositions is the range of truth values they may assume.
\end{definition}
In classical, Boolean, or two-valued logic, each proposition is either true or false, and no other characteristic of the proposition is considered relevant.
The fundamental operators in propositional logics include conjunction ($\land$), disjunction ($\lor$), and negation ($\lnot$).
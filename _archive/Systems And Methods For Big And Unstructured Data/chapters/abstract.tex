\begin{abstract}
    The course is structured around three main parts. 
    The first part focuses on approaches to Big Data management, addressing various challenges and dimensions associated with it.
    Key topics include the data engineering and data science pipeline, enterprise-scale data management, and the trade-offs between scalability, persistency, and volatility. 
    It also covers issues related to cross-source data integration, the implications of the CAP theorem, the evolution of transactional properties from ACID to BASE, as well as data sharding, replication, and cloud-based scalable data processing.

    The second part delves into systems and models for handling Big and unstructured data. It examines different types of databases, such as graph, semantic, columnar, document-oriented, key-value, and IR-based databases. 
    Each type is analyzed across five dimensions: data model, query languages, data distribution, non-functional aspects, and architectural solutions.
    
    The final part explores methods for designing applications that utilize unstructured data.
    It covers modeling languages and methodologies within the data engineering pipeline, along with schema-less, implicit-schema, and schema-on-read approaches to application design.
\end{abstract}
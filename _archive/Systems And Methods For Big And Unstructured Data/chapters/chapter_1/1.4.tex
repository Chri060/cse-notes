\section{NoSQL databases}

NoSQL databases emerged as an alternative to traditional relational databases, offering greater flexibility and scalability.

\begin{chronology}[10]{1960}{2020}{0.9\textwidth}
    \event{1965}{Multivalue databases}
    \event{1979}{DBM database}
    \event{2000}{Modern NoSQL databases}
    \event{2009}{NoSQL cathegorization}
\end{chronology}

\noindent NoSQL databases are designed to manage dynamic and unstructured data, often without requiring an explicit schema.
This flexibility allows NoSQL databases to efficiently handle large-scale, constantly changing datasets.

The rise of Big Data has influenced a major shift in database design. 
NoSQL databases adopt a schema on read approach, meaning that data can be ingested without a predefined structure. 
The schema is only applied when the data is read or queried, offering more adaptability and scalability compared to traditional systems.

\paragraph*{Object-relational mapping}
In traditional databases, object-relational mapping is used to bridge the gap between object-oriented programming languages and relational databases. 
Object-relational mapping helps overcome the impedance mismatch that arises when translating objects in code to rows in a relational database. 
NoSQL databases mitigate or even eliminate this issue by storing data in formats that align more naturally with the objects in application code.

\paragraph*{Data lake}
NoSQL databases are often at the core of data lakes, which serve as large, centralized repositories for raw, unstructured, and structured data. 
Data lakes are designed to store data in its native format, allowing for future analysis without requiring immediate transformation into a rigid schema. 
This approach makes NoSQL databases highly compatible with data lakes, as they can efficiently store and manage vast amounts of diverse data that can be used for analysis at a later time.

\paragraph*{Scalability}
One of the key strengths of NoSQL databases is their ability to scale horizontally. 
This approach is particularly effective for handling the massive datasets and high-throughput demands often found in Big Data applications. 
By scaling out across multiple machines, NoSQL databases can meet the increasing demands of modern applications without sacrificing performance or reliability.

\subsection{CAP theorem}
The CAP theorem describes the inherent trade-offs faced by distributed systems.
\begin{theorem}
    A distributed system cannot simultaneously guarantee all three of the following properties:
\end{theorem}
\begin{itemize}
    \item \textit{Consistency: every node in the system has the same view of data at the same time.}
    \item \textit{Availability: every request receives a response, regardless of success or failure.}
    \item \textit{Partition tolerance: the system remains operational even if communication between nodes is disrupted due to network failures.}
\end{itemize}
In practice, distributed systems must make compromises. 
NoSQL databases often trade either consistency or availability depending on their specific use case. 
Based on the CAP theorem, systems are typically classified as follows:
\begin{itemize}
    \item \textit{CP}: these systems ensure data accuracy and integrity but may sacrifice availability during network failures.
    \item \textit{AP}: these systems prioritize availability, allowing responses even if the data might be stale or inconsistent during partition events.
\end{itemize}
Choosing the right balance of these properties is essential for designing systems that align with performance, reliability, and scalability goals.

\paragraph*{BASE properties}
Traditional databases adhere to the ACID principles (atomicity, consistency, isolation, durability) to ensure strict data reliability. 
In contrast, many NoSQL databases follow the BASE model, which offers a more flexible approach to consistency:
\begin{itemize}
    \item \textit{Basically Available}: the system prioritizes availability, even if it temporarily sacrifices consistency.
    \item \textit{Soft state}: the system's state may evolve over time without new input, reflecting eventual changes.
    \item \textit{Eventual consistency}: the system guarantees consistency over time, but intermediate states may remain inconsistent.
\end{itemize}
The BASE model is particularly suited to scenarios where high availability and scalability are critical. 
It enables systems to handle network partitions and high traffic volumes effectively while still converging to a consistent state.

\subsection{NoSQL taxonomy}
NoSQL databases can be classified into several types, each designed to address different data management needs, scalability, and performance requirements:
\begin{itemize}
    \item \textit{Graph databases}: these databases model data as nodes and relationships (edges), which is optimal for scenarios that involve complex relationships.
    \item \textit{Documental databases}: data is stored as documents, typically in flexible formats like JSON, making them suitable for semi-structured or unstructured data. 
        This type is ideal for applications requiring rich data models.
    \item \textit{Key-value databases}: data is stored as simple key-value pairs, making these databases highly efficient for lookups based on keys.
    \item \textit{Column stores}: data is stored in columns rather than rows, which is particularly useful for analytical queries and big data applications. 
        Column stores excel at reading and processing large datasets quickly.
\end{itemize}
\section{Free grammars extended with regular expressions}


The EBNF class is instrumental in constructing grammars that enhance readability through the use of star, cross, and union operators. 
These grammars also facilitate the creation of syntax diagrams, serving as a blueprint for the syntax analyzer's flowchart.
It's important to note that since the context-free family is closed under all regular operations, the generative power of EBNF is equivalent to that of BNF.
\begin{definition}[\textit{EBNF grammar}]
    An EBNF grammar is defined as a four-tuple $\{V, \Sigma, P, S\}$, where there are exactly $\left\lvert V \right\rvert$ rules in the form $A \rightarrow \eta$ with $\eta$ being a regular expression over $\Sigma \cup V$.
\end{definition}
Compared to BNF, an EBNF grammar is typically shorter and more readable. 
Additionally, the use of nonterminal symbols in EBNF allows for more intuitive and meaningful names.

\paragraph*{Derivation}
The derivation relation in EBNF is defined by considering an equivalent BNF with infinite rules. 
\begin{definition}[\textit{Derived string}]
    Given strings $\eta_1$ and $\eta_2$ within $(\Sigma \cup V)^{\ast}$. 
    The string $\eta_2$ is said to be derived immediately in $G$ from $\eta_1$, denoted as $\eta_1 \implies \eta_2$, if the two strings can be factorized as: 
    \[\eta_1=\alpha A \gamma\qquad\eta_2=\alpha \vartheta \gamma\]
    and there exists a rule: 
    \[A \rightarrow e\]
    such that the regular expression $e$ admits the derivation $e \overset{\ast}{\implies} \vartheta$. 
\end{definition}
Note that $\eta_1$ and $\eta_2$ do not incorporate operators or parentheses typically found in regular expressions.
The only element that qualifies as a regular expression is the string $e$, and it only appears in the derivation if it is terminal.
When using EBNF, the node degree is unbounded, leading to a generally wider tree with reduced depth.
\section{Ambiguity}

\begin{definition}[\textit{Ambiguous grammar}]
    A sentence $x$ defined by grammar $G$ is considered ambiguous if it possesses multiple distinct syntax trees. 
    In such instances, we characterize the grammar $G$ as ambiguous.
\end{definition}
\begin{definition}[\textit{Degree of ambiguity}]
    The degree of ambiguity of a sentence $x$ in a language $L(G)$ is defined as the count of distinct syntax trees compatible with $G$.
    For a grammar, the degree of ambiguity is the maximum among the degrees of ambiguity for its sentences.
\end{definition}
The problem of determining whether a grammar is ambiguous is undecidable because there exists no general algorithm that, for any context-free grammar, can guarantee termination with the correct answer in a finite number of steps.
Consequently, establishing the absence of ambiguity in a specific grammar often requires a manual, case-by-case analysis through inductive reasoning, involving the examination of a finite number of cases.

To demonstrate the ambiguity of a grammar, one can provide a witness, which is an example of an ambiguous sentence generated by the grammar.
Therefore, it is advisable to design grammars with the aim of being unambiguous from the outset to avoid potential issues related to ambiguity. 
Ambiguity can be classified into various categories, as outlined below.

\subsection{Ambiguity from bilateral recursion}
Bilateral recursion occurs when a non-terminal symbol $A$ displays both left and right recursion.
\begin{example}
    Consider grammar $G_1$:
    \[G_1= E \rightarrow E+E\mid i\]
    This grammar can generate the string $i+i+i$ in two distinct ways. 
    Notably, the language generated by $L(G_1)=i(+i)^{\ast}$ is regular.
    Hence, it's possible to create simpler, unambiguous grammars, such as:
    \begin{itemize}
        \item A right-recursive grammar: $E \rightarrow i+E\mid i$.
        \item A left-recursive grammar: $E \rightarrow E+i\mid i$.
    \end{itemize}

    Let's now consider the grammar $G_2$:
    \[G_2= A \rightarrow aA\mid Ab\mid c\]
    The language generated by $G_2$, $L(G_2) = a^{\ast}cb^{\ast}$, is regular. 
    However, grammar $G_2$ allows derivations where the $a$ and $b$ characters in a sentence can be obtained in any order, making it ambiguous.
    To resolve this ambiguity, two nonambiguous grammars can be constructed:
    \begin{enumerate}
        \item Generate $a$'s and $b$'s separately using distinct rules:
            \[G_2=\begin{cases}
                S \rightarrow AcB               \\
                A \rightarrow aA\mid\varepsilon    \\
                B \rightarrow bB\mid\varepsilon 
            \end{cases}\]
        \item First generate the $a$'s then the $b$'s:
            \[G_2=\begin{cases}
                S \rightarrow aS\mid X              \\
                X \rightarrow Xb\mid c   
            \end{cases}\]
    \end{enumerate} 
\end{example}

\subsection{Ambiguity from language union}
If languages $L_1=L(G_1)$ and $L_2=L(G_2)$ share some sentences, constructing a grammar $G$ for their union language introduces ambiguity.
For any sentence $x \in L_1 \cap L_2$, it permits two distinct derivations: one following the rules of $G_1$ and the other following the rules of $G_2$.
This ambiguity persists when utilizing a single grammar $G$ that amalgamates all the rules.
However, sentences exclusively belonging to $L_1 \setminus L_2$ and $L_2 \setminus L_1$ are nonambiguous. 
To resolve this ambiguity, a solution is to provide separate sets of rules for $L_1 \cap L_2$, $L_1 \setminus L_2$, and $L_2 \setminus L_1$.

\subsection{Inherent ambiguity}
A language is deemed inherently ambiguous when every grammar defining it is ambiguous.
\begin{example}
    Consider the language $L=\{a^ib^jc^k\mid i=j \lor j=k\}=\{a^ib^ic^{\ast}\mid i \geq 0\} \cup \{a^{\ast}b^ic^i\mid i \geq 0\}$. 
    This language is characterized by two grammars:
    \[G_1=\begin{cases}
        S_1 \rightarrow XC \\
        X \rightarrow aXb\mid\varepsilon \\
        C \rightarrow cC\mid\varepsilon
    \end{cases}    
    \qquad G_2=\begin{cases}
        S_2 \rightarrow AY \\
        Y \rightarrow bYc\mid\varepsilon \\
        A \rightarrow aA\mid\varepsilon
    \end{cases}\]
    The union grammar of these two grammars is ambiguous.
    This observation suggests the intuitive conclusion that any grammar for the language $L$ is also ambiguous, reflecting the inherent ambiguity of the language itself.
\end{example}

\subsection{Ambiguity from concatenation of languages}
Ambiguity can arise in the concatenation of languages when a suffix of a sentence in the first language also acts as a prefix of a sentence in the second language.
To eliminate this ambiguity, one should avoid situations where a substring from the end of a sentence in the first language seamlessly connects to the beginning of a sentence in the second language.
A practical solution involves introducing a new terminal symbol, acting as a separator, which does not belong to either of the two alphabets.
\begin{example}
    Given two languages, $L_1$ and $L_2$, if concatenation introduces ambiguity, the issue can be resolved by adding a new terminal symbol, denoted as $\#$.
    The axiomatic rule can then be transformed as follows:
    \[S \rightarrow S_1 \# S_2\]
    It is crucial to note that this modification also alters the language itself.
\end{example}

\subsection{Other cases of ambiguity}
There are additional, less significant instances of ambiguity, including:
\begin{itemize}
    \item \textit{Ambiguity in regular expressions}: Resolve this by eliminating redundant productions from the rule.
    \item \textit{Lack of order in derivations}: Address this issue by introducing a new rule that enforces the desired order.
\end{itemize}
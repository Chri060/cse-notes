\section{Spectral density}

The spectral density of a stochastic stationary process $y(t)$ is expressed as follows:
\[\Gamma_y(\omega)=\sum_{\tau=-\infty}^{+\infty}\gamma_y(\tau)e^{-j\omega\tau}=\mathcal{F}\left\{ \gamma_y(\tau) \right\}\]
In simpler terms, it is the Fourier transform of the auto-covariance function $\gamma_y(\tau)$.

This spectral density possesses several key properties:
\begin{itemize}
    \item It is a real function of a real variable $\omega$:
        \[Im(\Gamma_y(\omega))=0\quad\forall\omega\in\mathbb{R}\]
    \item It is always non-negative:
        \[\Gamma_y(\omega) \geq 0\quad\forall\omega\in\mathbb{R}\]
    \item It exhibits symmetry (even function):
        \[\Gamma_y(\omega)=\Gamma_y(-\omega)\quad\forall\omega\in\mathbb{R}\] 
    \item It follows a 2$\pi$ periodic pattern: 
        \[\Gamma_y(\omega)=\Gamma_y(\omega+2\pi k)\quad\forall\omega\in\mathbb{R},k\in\mathbb{Z}\]
\end{itemize}
The Nyquist frequency is denoted as $\pi$, implying that $\pi$ equals the maximum value of $\omega$. 
Consequently,
\[\omega=\dfrac{2\pi}{T} \leq \pi \rightarrow T_{\min}\leq 2\]
This signifies that a periodic signal must be represented with a minimum period of two samples.

\subsection{Inverse transformation}
It's feasible to derive the covariance function from the spectral density:
\[\gamma_y(\omega)=\mathcal{F}^{-1}\left(\gamma_y(\omega)\right)=\dfrac{1}{2\pi}\int_{-\pi}^{\pi}\Gamma_y(\omega)e^{j\omega\tau}\,d\omega\]
This transformation is valid because both $\gamma_y(\omega)$ and $\Gamma_y(\omega)$ encapsulate identical information about the process $y(t)$.

It's worth noting that the variance determined when $\tau=0$ is equal to:
\[\gamma_y(0)=\dfrac{1}{2\pi}\int_{-\pi}^{\pi}\Gamma_y(\omega)\,d\omega\]
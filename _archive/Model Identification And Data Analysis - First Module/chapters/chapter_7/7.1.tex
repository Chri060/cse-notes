\section{Introduction}

If our interest lies in certain features of the stochastic process $y(t)$, such as $\mathbb{E}[y(t)]$ or $\gamma_y(\tau)$, it is unnecessary to identify the complete model $M(\vartheta)$ and estimate these features directly from it.

For a stationary stochastic process $y(t)$, data-driven estimation is feasible for estimating $\mathbb{E}[y(t)]$, $\gamma_y(\tau)$, and $\Gamma_y(\omega)$.

\begin{definition}[\textit{Estimator correctness}]
    An estimator $\hat{Q}_N$ of $Q$ is considered correct if:
    \[\mathbb{E}\left[\hat{Q}_N\right]=Q\]
\end{definition}
This property is also known as unbiased estimation.
\begin{definition}[\textit{Estimator consistency}]
    An estimator $\hat{Q}_N$ of $Q$ is regarded as consistent if:
    \[\mathbb{E}\left[\left(\hat{Q}_N-Q\right)^2\right]\rightarrow 0\]
    as the number of samples $N$ tends to infinity.
\end{definition}
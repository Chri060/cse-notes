\section{NarrowBand Internet of Things}

NarrowBand Internet of Things (NB-IoT) is a cellular technology purpose-built to address the unique requirements of large-scale IoT deployments. 
It is designed for low-power, low-throughput applications and offers several advantages. 
Operating within the licensed spectrum, NB-IoT ensures robust, interference-free connectivity, making it especially suitable for industrial, urban, and rural IoT use cases. 
Moreover, it benefits from seamless integration with existing LTE infrastructure, allowing mobile operators to deploy it cost-effectively.

Power Saving Mode (PSM) allows devices to stay registered with the network without needing to re-initiate connections. 
This avoids the energy-intensive signaling associated with re-attachments, significantly reducing power consumption. 
While in PSM, devices are unreachable for mobile-terminated traffic, but remain logically connected, enabling long-term sleep periods and dramatically extending battery life without compromising network accessibility.

The random access procedure governs how a device initiates communication with the network. 
This mechanism ensures efficient resource allocation, especially important when a large number of devices attempt to connect simultaneously. 
To maintain reliability even in weak signal conditions, NB-IoT supports extensive message repetition ensuring that critical messages are delivered even in deep indoor or remote environments.

\subsection{Coverage}
NB-IoT offers robust coverage capabilities through Coverage Enhancement (CE) levels, allowing reliable operation across diverse signal environments. 
Devices in strong signal areas operate at CE Level 0, while those in medium and poor signal areas transition to CE Levels 1 and 2, respectively. 
These levels determine the number of repetitions for both uplink and downlink transmissions, increasing signal robustness. 
This adaptability allows NB-IoT to achieve up to 164 dB of maximum coupling loss, significantly extending coverage into basements, tunnels, and remote outdoor locations.

\subsection{Features}
NB-IoT minimizes device and deployment costs by simplifying traditional LTE features. 
It reduces modem complexity through the use of smaller transport block sizes, single-stream transmissions, and single-antenna support. 
Additionally, turbo decoding is eliminated from the device (UE) side, as only Turbo Block Coding (TBC) is used for downlink channels. 

NB-IoT supports in-band operation by embedding its carriers within existing LTE carriers. 
This approach enables efficient reuse of spectrum and infrastructure, allowing flexible scaling by adding more NB-IoT carriers as needed. 
However, care must be taken to manage near-far interference, especially with legacy LTE base stations that don't support NB-IoT. 
A network-wide upgrade is typically required to ensure optimal coexistence. 
Although in-band deployment may slightly reduce LTE capacity, strategies like increasing NB-IoT transmit power or dynamic base band resource sharing can mitigate performance degradation.

To further optimize energy efficiency, NB-IoT employs Enhanced Discontinuous Reception (eDRX). 
This mechanism allows devices to remain dormant for extended intervals significantly reducing the frequency at which they must wake up to monitor paging messages. 
The eDRX configuration is negotiated between the device and the network, enabling flexible trade-offs between responsiveness and battery life. 
This feature is particularly beneficial for delay-tolerant applications like smart meters and environmental sensors.
\begin{abstract}
    The course covers several key topics in computer architecture. 
    It begins with a review of fundamental concepts, such as the RISC approach, pipelining, and the memory hierarchy. 
    Students will learn about performance evaluation metrics for computer architectures and techniques for optimizing performance in both processors and memory. 
    The course delves into instruction-level parallelism, discussing static and dynamic scheduling, superscalar architectures, their principles and challenges, and VLIW architectures, with examples from various architecture families. 
    Additionally, the course covers thread-level parallelism and explores multiprocessor and multicore systems, including their taxonomy, topologies, communication management, memory management, and cache coherency protocols, with examples of different architectures. 
    Finally, the course examines stream processors, vector processors, graphics processors, GP-GPUs, and heterogeneous architectures.
\end{abstract}
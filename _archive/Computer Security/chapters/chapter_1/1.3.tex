\section{Ethical hacking}

White hats, also known as security professionals or ethical hackers, are tasked with:
\begin{itemize}
    \item \textit{Identifying vulnerabilities}.
    \item \textit{Developing exploits}.
    \item \textit{Creating attack-detection methods}.
    \item \textit{Designing countermeasures against attacks}.
    \item \textit{Engineering security solutions}.
\end{itemize}

Since no system is invulnerable, it's crucial to assess its risk level.
This involves evaluating the potential damage due to vulnerabilities and threats through the concept of risk:
\begin{definition}[\textit{Risk}]
    Risk is a statistical and economic evaluation of potential damage resulting from the presence of vulnerabilities and threats:
    \[\text{Risk}=\text{Asset} \times \text{Vulnerabilities} \times \text{Threats}\]
\end{definition}
Assets and vulnerabilities can be managed, but threats are independent variables.

To ensure system security, a balance must be struck between cost and reducing vulnerabilities and containing damage.
The costs of securing a system can be categorized as direct and indirect.
Direct costs include management, operational, and equipment expenses, while indirect costs, which often form the larger portion, stem from:
\begin{itemize}
    \item \textit{Reduced usability}. 
    \item \textit{Slower performance}. 
    \item \textit{Decreased privacy} (due to security controls). 
    \item \textit{Lower productivity} (as users may be slower). 
\end{itemize}
It's important to note that simply spending more money on security may not always resolve the issue.

In real-world systems, setting boundaries is essential, meaning that a portion of the system must be assumed as secure. 
These secure parts consist of trusted elements determined by the system developer or maintainer. 
For example, the level of trust in a particular system can be determined at the software, compiler, or hardware level.
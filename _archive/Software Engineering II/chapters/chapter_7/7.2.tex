\section{Requirement Analysis and Specifications Document}

The Requirements and Specifications Document (RASD) serves several purposes:
\begin{itemize}
    \item \textit{Communication}: it conveys an understanding of the requirements, encompassing the application domain and the system under development.
    \item \textit{Contractual}: it can be legally binding, serving as a formal agreement between stakeholders.
    \item \textit{Baseline for project planning and estimation}: it provides a foundation for project planning and estimation, covering aspects like size, cost, and schedule.
    \item \textit{Baseline for software evaluation}: it supports system testing, verification, and validation activities. 
        It contains the information necessary to verify if the delivered system aligns with the requirements.
    \item \textit{Baseline for change control}: it establishes a foundation for managing changes in requirements as the software evolves.
\end{itemize}
The RASD document is utilized by various stakeholders, including:
\begin{itemize}
    \item \textit{Customers and users}: they are interested in a high-level description of system functionalities and requirements.
    \item \textit{System analyst and requirement analysts}: these individuals use the RASD to specify how the system interacts with other systems.
    \item \textit{Developers and programmers}: they refer to the RASD for implementation details.
    \item \textit{Testers}: they use the RASD to check if the system meets its requirements.
    \item \textit{Project managers}: they rely on the RASD to control the development process.
\end{itemize}
\begin{figure}[H]
    \centering
    \includegraphics[width=1\linewidth]{images/RASD.png}
    \caption{IEEE standard for RASD}
\end{figure}
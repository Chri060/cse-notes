\section{Executing, monitoring and controlling}

\paragraph*{Executing}
The project's execution is delineated through five primary phases:
\begin{itemize}
\item Initiate the project: commence with a kick-off meeting.
\item Assemble and oversee the project team: engage both internal and external resources.
\item Procure necessary equipment, materials, and external services.
\item Execute the tasks outlined in the schedule.
\item Conduct monitoring and controlling activities.
\end{itemize}

\paragraph*{Monitoring}
Monitoring involves gathering data about the project's current status, as projects often deviate from their initial plans due to changes, challenges, and other factors.

\paragraph*{Controlling}
Control involves implementing corrections to realign your project and bring it back on course.

\subsection{Monitoring process}
\paragraph*{Data gathering}
Collecting up-to-date data is a crucial aspect of project management. 
This includes obtaining information on actual dates, time spent on tasks, and the expenditures incurred. 
The process involves monitoring the start dates of tasks, recording actual work hours or durations, keeping track of remaining work or duration, investigating any additional costs, and updating the project schedule. 
The original schedule serves as the baseline against which these updates are made to ensure accurate tracking and management of the project's progress.

\paragraph*{Monitoring schedule}
Monitoring the schedule consists in evaluate your current project schedule in comparison to the initially established baseline schedule.
To do so we have to search for the following warning signs: 
\begin{itemize}
    \item Tasks running late.
    \item Tasks not started that should be.
    \item Haven't completed as much work as planned.
    \item Compare actual costs to assigned budget.
\end{itemize}

\paragraph*{Risks}
Continuously monitor potential risks, and should they materialize, implement the mitigation strategy outlined in the risk management plan. 
Adjust the mitigation strategy as necessary through regular revisions.

\paragraph*{Earned value analysis}
he earned value analysis is a methodology used to evaluate project performance and progress. 
The scope, time, and cost variables associated with a project's activities are inherently different and cannot be directly compared.
Earned Value represents the financial translation of these three project variables, streamlining them into monetary values for monitoring and comparison purposes. 
The key components include:
\begin{itemize}
    \item Budget at completion (BAC): the total budget allocated for the entire project.
    \item Planned value (PV): the budgeted cost of the planned work.
    \item Earned value (EV): the budgeted cost of the work that has been performed.
    \item Actual cost (AC): the actual cost incurred for completed work.
\end{itemize}
This approach facilitates the monitoring and comparison of costs and work completed up to the current date.
The key indices utilized for this analysis include:
\begin{itemize}
    \item Schedule Variance (SV):
        \[\text{SV} = \text{EV} - \text{PV}\]
    \item Schedule performance index (SPI):
        \[\text{SPI} = \dfrac{\text{EV}}{\text{PV}}\]
    \item Cost variance (CV):
        \[\text{CV} = \text{EV} - \text{AC}\]
    \item Cost performance index (CPI):
        \[\text{CPI} = \dfrac{\text{EV}}{\text{AC}}\]
\end{itemize}
The first two indices pertain to the schedule perspective, while the latter two relate to the cost viewpoint.
These indices facilitate the estimation of the budget at completion (EAC) under different assumptions:
\begin{enumerate}
    \item Continuing to spend at the same rate (same CPI):
        \[\text{EAC} = \dfrac{\text{BAC}}{\text{CPI}}\]
    \item Continuing to spend at the baseline rate:
        \[\text{EAC} =\text{AC}+\left(\text{BAC}-\text{EV}\right)\]
    \item Maintaining the same cost and schedule performance index (same CPI and SPI):
        \[\text{EAC} =\dfrac{\text{AC}+\left(\text{BAC}-\text{EV}\right)}{\text{CPI}\cdot\text{SPI}}\]
\end{enumerate}

\subsection{Controlling process}
The control process is employed to harmonize the aspects of scope, time, cost, resources, and quality within a project. 
When prioritizing schedule, techniques such as fast-tracking and crashing can be applied. 
If financial considerations take precedence, costs can be trimmed by reducing resources or overhead costs. 
In cases where schedule, finances, and resources are non-negotiable, the option is to curtail scope by eliminating associated tasks. 
However, this choice introduces a risk factor and relies on the priorities defined by stakeholders.

\paragraph*{Fast tracking}
The fast-tracking technique involves advancing task start times by introducing negative lag time if necessary.
This can only be implemented if the identified tasks can be effectively parallelized.
While there is no associated cost increase, the approach introduces higher risk.

\paragraph*{Crashing}
Crashing is a technique that involves shortening tasks on the critical path by introducing additional resources. 
This results in a cost increase and is feasible only if the addition of new resources proves beneficial.

\paragraph*{Closing}
The closing phase encompasses several key activities:
\begin{itemize}
    \item \textit{Ensure project acceptance}.
    \item \textit{Track project performance}.
    \item \textit{Lessons learned}.
    \item \textit{Close contracts}.
    \item \textit{Release resources}.
\end{itemize}
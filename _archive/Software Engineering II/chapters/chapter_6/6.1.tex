\section{Introduction}

\begin{definition}[\textit{Project}]
    A project is a temporary organization (time, resources) that is created for the purpose of delivering one or more business products (scope) according to an agreed Business Case (scope, cost, time, quality, risks).
\end{definition}
Key variables integral to a project include scope, quality, schedule (time), budget (cost), resources, and risks.
Project management assumes a pivotal role in the strategic planning, monitoring, and control of these variables, thereby contributing to the project's success.
It is imperative to recognize that even a well-conceived project may face failure, while a poorly-directed one is almost certain to fail.
The project management process unfolds through distinct phases:
\begin{enumerate}
    \item \textit{Initiating}: primarily concerned with obtaining commitment to initiate the project, this phase involves defining the project scope and strategy.
    \item \textit{Planning}: involves the definition of the schedule, identification of stakeholders and risks, estimation of costs and resources, and outlining the project management processes essential for execution, monitoring, and control.
    \item \textit{Executing}: encompasses the active implementation of the project plan.
    \item \textit{Monitoring and controlling}: involves ongoing oversight and adjustment of project variables to ensure alignment with the project plan.
    \item \textit{Closing}: concludes the project, addressing necessary finalization tasks.
\end{enumerate}
Recognizing the critical nature of delivering the right product within stipulated timeframes and budget constraints, most software organizations hinge their success on effective project management practices, which are seamlessly integrated into their software processes.
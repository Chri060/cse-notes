\section{Formal methods}

Informal methods often suffer from several major issues:
\begin{itemize}
    \item \textit{Lack of precision}: ambiguous definitions and specifications can lead to misunderstandings and errors in interpretation.
    \item \textit{Unreliable verification}: traditional testing methods have well-known limitations, making it difficult to ensure correctness.
    \item \textit{Safety and security risks}: if a flawed program were part of a critical system, it could result in serious consequences.
    \item \textit{Economic impact}: errors in software can lead to financial losses.
    \item \textit{Limited generality and reusability}: informal approaches often produce software that is difficult to reuse, adapt, or port to different environments.
    \item \textit{Overall poor quality}: the lack of rigorous foundations can lead to unreliable and suboptimal software.
\end{itemize}
\noindent Formal methods offer a structured, mathematical approach to software and system development. 
Ideally, they provide a comprehensive formalization (every aspect of the system is modeled mathematically), and mathematical reasoning and verification (analysis is performed using formal proofs and supported by specialized tools).
By applying formal methods, we can achieve greater precision, reliability, and confidence in complex systems.
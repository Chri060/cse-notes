\section{Introduction}

In many systems, timing is crucial, and they must operate within strict time constraints. 
These time-critical systems require careful handling of time to ensure proper functionality.

In classical Finite State Machines, time is often not explicitly modeled. 
Time is treated as a discrete, implicit metric, with an infinite number of states under a Büchi acceptance condition. 
This means time is ignored in the traditional sense, where each transition represents one time step, assuming a fixed underlying clock rate.

However, this approach may not be sufficient in more complex systems.
When Finite State Machines are composed to form more intricate models, the temporal relationships between events depend heavily on the method of composition. 
In complex system models, where multiple Finite State Machines are often composed, the rigid notion of one transition is one time step becomes inadequate, especially when some components operate asynchronously.

To address this, we need more advanced mechanisms to capture the progression of time, especially when dealing with asynchronous components. 
Several alternatives to the traditional Finite State Machine model can better handle time in complex systems:
\begin{enumerate}
    \item \textit{Transition-based time advancement}: in some models, time advances only when specific transitions are taken or when certain conditions are met. 
        This approach remains fundamentally transition-based and is effective for discrete-time systems.
    \item \textit{Independent time progression}: in models like Timed and Hybrid Automata, time can advance independently of transitions. 
        Here, transitions may include time-based guards, allowing them to be triggered only when certain timing conditions are met. 
        This allows for the modeling of continuous or dense time, capturing a broader range of behaviors in time-critical systems.
\end{enumerate}
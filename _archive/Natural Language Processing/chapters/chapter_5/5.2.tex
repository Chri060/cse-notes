\section{Automatic Speech Recognition}

Automatic Speech Recognition (ASR) is the task of converting spoken language into written text. 
Traditionally, this problem was tackled using hand-engineered features and statistical models:
\begin{itemize}
    \item \textit{Feature extraction}: audio signals were first transformed into Mel-frequency cepstral coefficients (MFCCs) derived from Mel spectrograms.
    \item \textit{Modeling}: these features were then fed into Hidden Markov Models (HMMs) combined with Gaussian Mixture Models (GMMs) for sequence prediction.
\end{itemize}
\noindent However, the advent of DL has significantly changed the landscape of ASR.

A common approach using Convolutional Neural Networks (CNNs) involves classifying phonemes from raw audio or Mel spectrograms. This method, however, introduces several issues:
\begin{itemize}
    \item CNNs make predictions over fixed-size windows, but phonemes and words vary in duration.
    \item There's a need to determine how many windows correspond to each linguistic unit.
\end{itemize}
\noindent These limitations highlight the need for sequence models that can align input and output of different lengths.
A breakthrough in ASR came with encoder-decoder architectures, which can handle variable-length input and output sequences and learn temporal alignments without requiring fixed-size segmentation.

\paragraph*{Wav2vec}
Wav2Vec represents a significant advancement in ASR. 
It is a transformer-based model that operates directly on raw audio waveforms.
A convolutional frontend first extracts latent representations from the audio, which are then fed into a transformer encoder to capture temporal dependencies and contextual information. 
Importantly, Wav2Vec is trained in a self-supervised manner, enabling the use of large, unlabeled audio datasets.

\paragraph*{Whisper}
More recently, the Whisper model has achieved state-of-the-art performance in ASR. 
Unlike Wav2Vec, Whisper uses Mel spectrograms as its input representation and adopts an architecture inspired by Vision Transformers. 
Its training paradigm is weakly supervised and leverages vast amounts of diverse, multilingual data. 
This approach supports multiple languages and tasks, such as transcription, translation, and language detection, within a single unified model.

\subsection{Evaluation}
The performance of ASR systems is typically evaluated using two main metrics: Word Error Rate (WER) and Sentence Error Rate (SER).
WER quantifies the number of word-level substitutions, deletions, and insertions needed to transform the predicted transcription into the correct one, normalized by the number of words in the reference. 
SER, on the other hand, measures the proportion of sentences that contain at least one error, offering a complementary view of system performance at the sentence level.

\subsection{Advanced Automatic Speech Recognition}
Beyond general-purpose transcription, ASR systems can be adapted to improve accuracy for individual speakers. 
Speaker-specific variation, such as vocal tract length and pitch, can significantly affect recognition performance. 
Techniques such as vocal tract length normalization, which warps the frequency axis of the speech spectrum, help account for anatomical differences between speakers. 
Similarly, pitch normalization—particularly useful for recognizing children's speech. 
Another approach involves adapting a pre-trained acoustic model using a small amount of data from a new speaker to fine-tune the system for improved personalization.

An additional challenge in ASR is handling non-verbal and non-word sounds, which frequently occur in real-world audio. 
These include sounds like coughing, sighing, and environmental noises such as telephone rings or door slams. 
To accommodate these, special phonetic units can be defined, and corresponding placeholder words added to the model's lexicon and language model. 
Training data must include annotations for these special tokens to ensure the model can recognize and appropriately represent such sounds in the transcript.
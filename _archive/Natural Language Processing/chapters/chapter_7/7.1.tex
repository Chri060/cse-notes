\section{Introduction}

Agentic AI Design Patterns refer to commonly used architectural strategies for deploying Large Language Models (LLMs) in autonomous or semi-autonomous workflows. 
These patterns enable LLMs to reason, act, and collaborate in ways that mirror intelligent agent behavior. 
The following are some of the most widely adopted patterns:
\begin{itemize}
    \item \textit{Reflection pattern}: the LLM evaluates and critiques its own outputs. 
        This can be used to enforce behavioral constraints.
    \item \textit{Tool use pattern}: the LLM interacts with external tools and incorporates their outputs into its reasoning. 
        Tools may include calculators, search engines, or APIs, enabling the model to go beyond its static knowledge.
    \item \textit{Reasoning and acting pattern}: the LLM alternates between reasoning steps and tool use to iteratively achieve a goal. 
    \item \textit{Planning pattern}: the LLM creates a multi-step plan to accomplish a complex task. 
        It monitors the execution of subtasks, handles failures, and adjusts the plan dynamically as needed to ensure the desired outcome.
    \item \textit{Multi-agent pattern}: multiple LLMs function as distinct agents, each with specialized roles or capabilities. 
        These agents communicate and collaborate to solve problems collectively, often outperforming a single-agent approach for complex tasks.
\end{itemize}

\subsection{LangChain}
LangChain is an open-source framework designed to streamline the development of LLM-powered applications. 
LangChain supports modular design, making it easy to integrate tools, prompts, and memory.
\section{Computer vision}

Computer vision is an interdisciplinary field that focuses on enabling computers to interpret and understand the visual world using digital images or videos. 
Initially, many computer vision techniques relied on mathematical models and statistical analysis of images. 
However, with the rise of Machine Learning modern approaches have shifted toward data-driven methods. 
These methods have significantly enhanced the effectiveness and adaptability of algorithms for solving complex visual tasks.

\subsection{Digital images}
A digital color image is typically represented using three separate matrices, each corresponding to one of the primary colors: red, green, and blue (RGB). 
Each matrix element represents the intensity of a pixel and is usually encoded with values ranging from 0 to 255.

Although images are often stored in compressed formats to save disk space, they must be decompressed for processing in memory. 
This decompression increases the size of the data considerably, which poses challenges in terms of memory management and processing power. 
The complexity of managing large amounts of image data becomes even more pronounced when working with video files, as they contain a vast number of frames that must be processed sequentially.
When using Neural Networks for image processing tasks, raw image data needs to be efficiently handled, which can result in large computational and memory demands.

\subsection{Local transformations}
Local transformations involve modifying each pixel in an image based on the values of its neighboring pixels within a defined neighborhood $U$. 
The transformation can be expressed as:
\[\mathbf{G}(r,c)=T_U[\mathbf{I}](r,c)\]
Here, $\mathbf{I}$ is the input image, $\mathbf{G}$ is the output image, $U$ defines the neighborhood around the pixel, and $T_U$ is a spatial transformation function. 
This function can be either linear or non-linear, depending on the transformation.

For a pixel at coordinates $(r, c)$, the neighborhood $U$  is typically a square region centered at the pixel and is defined as:
\[\{\mathbf{I}(u,v)\mid(u-r,v-c)\in U\}\]
Here, $(u, v)$ represents the displacement relative to the center of the neighborhood $(r, c)$.
The transformation function $T_U$ is applied repeatedly across all pixels in the image, making the transformation spatially invariant.

\paragraph*{Local linear filters}
In the case of linear spatial transformations, the output at a given pixel $(r,c)$, denoted as $T_U[\mathbf{I}](r,c)$, is a linear combination of the pixel values within the neighborhood $U$. 
This can be expressed as:
\[T_U[\mathbf{I}](r,c)=\sum_{(u,v)\in U}\mathbf{w}_i(u,v)\mathbf{I}(r+u,c+v)\]
Here, $w(u, v)$ represents the weights associated with each pixel in the neighborhood $U$.
These weights can be interpreted as defining a filter which is applied uniformly across the entire image. 
This approach allows the same operation to be applied to all pixels, making it a simple and efficient method for image processing.

This approach applies to both grayscale and color (RGB) images, where each color channel is processed separately:
\[T_U[\mathbf{I}](r,c)=\sum_i\sum_{(u,v)\in U}\mathbf{w}_i(u,v,i)\mathbf{I}(r+u,c+v,i)\]

The correlation between a filter $\mathbf{w}$ and an image $\mathbf{I}$ can be computed using:
\[(\mathbf{I}\otimes \mathbf{w})(r,c)=\sum_{u=-L}^L\sum_{v=-L}^L\mathbf{w}(u,v)\mathbf{I}(r+c,c+v)\]
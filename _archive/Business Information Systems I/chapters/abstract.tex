\begin{abstract}
    This course offers a methodology to align IT design decisions with business goals. 
    It introduces the concept of IT architecture, classifies key IT design choices, and examines how these choices impact IT architecture from both a software and infrastructure perspective. 
    The course explores IT architecture within the manufacturing, utilities, and financial services sectors, focusing on both internal and external organizational processes along the industry value chain (e-business). 
    It also equips students with the tools to analyze organizational requirements, with a particular emphasis on executive information systems, including the use of Key Performance Indicators.

    Building on the concept of IT architecture, the course outlines a functional map of Enterprise Resource Planning systems, distinguishing between core and extended functionalities. 
    It traces the evolution of information systems over time and highlights how ERPs have emerged through an ongoing process of functional integration. 
    The course begins with a review of organizational theory from an information perspective, providing a framework to understand the organizational changes driven by ERP implementations. 
    It then delves into the core functional areas of ERP systems, such as accounting and finance, operations, and management and control. 
\end{abstract}
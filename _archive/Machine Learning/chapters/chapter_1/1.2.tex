\section{Supervised Learning}

Supervised Learning encompasses several distinct tasks:
\begin{itemize}
    \item \textit{Classification}: this involves assigning predefined categories or labels to data points based on their features. 
        The model is trained on labeled data, learning patterns to predict the class labels of new data points.
    \item \textit{Regression}: the goal here is to predict continuous numerical values based on input features, as opposed to discrete class labels in classification. 
        The model learns a function mapping input features to output values.
    \item \textit{Probability estimation}: this task predicts the likelihood of certain events or outcomes occurring, often used to gauge the confidence of model predictions.
\end{itemize}
Formally, in Supervised Learning, a model learns from data to map known inputs to known outputs. 
The training set is denoted as $\mathcal{D}=\left\{\left\langle x,t \right\rangle\right\}$, where $t = f(x)$, with $f$ representing the unknown function to be determined using Supervised Learning techniques.

Various techniques can be employed for Supervised Learning, including linear models, artificial neural networks, Support Vector Machines, and decision trees.
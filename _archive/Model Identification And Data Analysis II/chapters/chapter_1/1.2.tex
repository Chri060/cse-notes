\section{Transformations}

There are six possible transformations between the representations discussed earlier.

\subsection{State space to transfer function}
Given a state space representation, we can derive the transfer function representation using the formula:
\[W(z)=\mathbf{H}{\left(z\mathbf{I}-\mathbf{F}\right)}^{-1}\mathbf{G}-\mathbf{D}\]
Alternatively, we can directly transform the state equations using the delay operator.

\subsection{Transfer function to state space}
The transformation from transfer function to state space is known as realization. 
The main challenge of this method is that the state space representation is not unique, leading to infinitely many equivalent realizations of a transfer function into a state space model. 
Assuming a monic denominator, the transfer function is given by:
\[W(z)=\dfrac{b_0z^{n-1}+b_1z^{n-2}+\cdots+ b_{n-1}}{z^n+a_1z^{n-1}+a_2z^{n-2}+\cdots+a_n}\]
With this representation, we can find the state space representation as follows:
\[ \mathbf{F}=\begin{bmatrix} 0 & 1 & \cdots & 0 \\ \vdots & \ddots  & \ddots & \vdots \\ 0 & 0 & 0 & 1 \\ -a_n & -a_{n-1} & \cdots & -a_1 \end{bmatrix} \quad  \mathbf{G}=\begin{bmatrix} 0 \\ 0 \\ \vdots \\ 1 \end{bmatrix} \quad  \mathbf{H}=\begin{bmatrix} b_{n-1} \\ b_{n-2} \\ \cdots \\ b_0 \end{bmatrix} \quad  \mathbf{D}=\begin{bmatrix} 0 \end{bmatrix}\]

\subsection{Transfer function to impulse response}
We can transform the transfer function into impulse response representation by performing a long polynomial division between the numerator and the denominator of the transfer function.
From the quotient of the long division, we obtain $\omega(t)$ for the impulse response, and then we can express the representation in terms of the impulse response. 

\subsection{Impulse response to transfer function}
Given a discrete-time signal $s(t)$, where $s(t)=0$ if $t<0$, the Z-transform of the signal $s(t)$ is defined as: 
\[\mathcal{Z}[s(t)]=\sum_{t=0}^{+\infty}s(t)z^{-t}\]
It can be shown that:
\[W(z)=\mathcal{Z}[\omega(t)]=\sum_{t=0}^{+\infty}\omega(t)z^{-t}\]
However, in practice, this formula cannot be directly used due to the requirement of having infinite values of the impulse response and the necessity for the impulse response to be noise-free.

\subsection{State space to impulse response}
The impulse response can be determined starting from the state space representation with the initial conditions:
\[\begin{cases}
    x(0)=0 \\
    y(0)=0
\end{cases}\]
Then, the system simulation is run starting from $t=1$ onwards. 
The generic pattern for the impulse response calculation is:
\[\omega(t)=\begin{cases}
    \mathbf{D} \qquad\qquad\quad\text{if } t=0 \\
    \mathbf{HF}^{t-1} \mathbf{G}   \qquad \, \text{otherwise}
\end{cases}\]

\subsection{Impulse response to state space}
The transformation from impulse response representation to state space is a crucial task in subspace-based state space system identification methods, such as the 4SID method. 
This method is a black-box system identification approach.
\section{Software sensing methodologies comparison}

White-box models require a physical system model and, in practice, a training dataset. 
They offer interpretability, allow retuning, and provide good software sensor accuracy, including sensing unmeasurable states. 
In contrast, black-box models do not need a physical model but require training data. 
They lack interpretability and retuning capability but achieve very good software sensor accuracy without sensing unmeasurable states.
\renewcommand*{\arraystretch}{2}
\begin{table}[H]
    \centering
    \begin{tabular}{l|l|l|}
    \hline
    \multicolumn{1}{|l|}{\textbf{Feature}}                                                       & \textbf{White box}                       & \textbf{Black box} \\ \hline
    \multicolumn{1}{|l|}{\textit{Physical model of the system}}            & Required                                   & Not required       \\
    \multicolumn{1}{|l|}{\textit{Training dataset}}                        & Required in practice                       & Required           \\
    \multicolumn{1}{|l|}{\textit{Interpretability}}                        & Yes                                        & No                 \\
    \multicolumn{1}{|l|}{\textit{Retuning}}                                & Yes                                        & No                 \\
    \multicolumn{1}{|l|}{\textit{Software sensor accuracy}}                & Good                                       & Very good          \\
    \multicolumn{1}{|l|}{\textit{Software sensing of unmeasurable states}} & Yes                                        & No                 \\ \hline
    \end{tabular}
\end{table}
\renewcommand*{\arraystretch}{1}
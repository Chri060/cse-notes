\section{Introduction}

The method at hand is straightforward and intuitive, making it a staple in control applications. 
It follows a classic parametric approach:
\begin{enumerate}
    \item Experiment design, data collection, and preprocessing.
    \item Selection of parametric model family $\mathcal{M}(\boldsymbol{\theta})$. 
    \item Definition of performance index $J(\boldsymbol{\theta})$. 
    \item Optimization: $\hat{\boldsymbol{\theta}}=\argmin_{\boldsymbol{\theta}}{J(\boldsymbol{\theta})}$.  
\end{enumerate}
In frequency domain system identification, the overarching approach involves:
\begin{itemize}
    \item Conducting a series of experiments with single-sinusoid excitations.
    \item Extracting a single data point representing the system's frequency response from each experiment.
    \item Employing optimization techniques to match the measured frequency response with the modeled frequency response, thereby determining the system parameters.
\end{itemize}
\section{Offline grey box identification}

In many applications, we have some prior knowledge of a possible model for a system, denoted as $\mathcal{M}(\boldsymbol{\theta})$, which depends on certain physical parameters. 
The model, expressed in state-space representation, is given by:
\[\mathcal{M}(\boldsymbol{\theta}):\begin{cases}
    \dot{\mathbf{x}}(t)=\mathbf{F}(\boldsymbol{\theta})\mathbf{x}(t)+\mathbf{G}(\boldsymbol{\theta})\mathbf{u}(t) \\
    \mathbf{y}(t)=\mathbf{H}(\boldsymbol{\theta})\mathbf{x}(t)+\mathbf{D}(\boldsymbol{\theta})\mathbf{u}(t)
\end{cases}\]
Our objective is to estimate the parameters $\boldsymbol{\theta}$ based on input and output data.

\subsection{Simulation Error Method}
One approach to parameter estimation is the Simulation Error Method. 
We select an initial estimate $\bar{\boldsymbol{\theta}}$ and simulate the model $\mathcal{M}\left(\bar{\boldsymbol{\theta}}\right)$. 
The accuracy of the estimate is assessed by comparing the simulated output $\mathbf{y}_{\text{sim}}(t)$ with the actual measured output $\mathbf{y}(t)$. 
This comparison is formulated mathematically as:
\[\boldsymbol{\theta}_{\text{opt}}=\argmin_{\boldsymbol{\theta}\in\boldsymbol{\Theta}}\text{RMS}(\mathbf{y}(t)-\mathbf{y}_{\text{sim}}(t,\boldsymbol{\theta}))=\argmin_{\boldsymbol{\theta}\in\boldsymbol{\Theta}}\text{RMS}(\mathbf{y}(t)-\mathcal{M}(\boldsymbol{\theta})u(t))\]
Here, $\text{RMS}(\cdot)$ denotes the Root Mean Square error, and $\boldsymbol{\Theta}$ represents the set of all possible parameter values.
Since the optimization problem is generally non-convex and does not have a linear relationship with the parameter vector $\boldsymbol{\theta}$, finding an optimal solution requires specialized numerical techniques.

\paragraph*{Data quality}
For effective identification, the dataset $\mathcal{D}$ must be informative. 
The length of the dataset is less important than ensuring it excites all relevant system dynamics. 
The quality of the identified parameters directly depends on the richness of the data.

Common signal types used for system identification include:
\begin{itemize}
    \item \textit{Sine sweep or multiple sine experiments}: these signals cover a broad frequency range, providing useful information in both time and frequency domains. 
        The frequency sweeps from an initial value $f_{\text{init}}$ to a final value $f_{\text{end}}$.
    \item \textit{Pseudo-Random Binary Signal}: a two-level signal that switches values at random intervals.
        PRBS provides a nearly flat frequency spectrum up to a specified cutoff frequency $f_{\text{max}}$, making it effective for system identification.
\end{itemize}
Typically, the optimization process is performed on at least one dataset and validated on a separate dataset to ensure robustness. 
The validation must be conducted in both the time and frequency domains.

\paragraph*{Optimization}
The grey-box identification optimization follows these steps:
\begin{algorithm}
    \caption{Grey-box optimization}
    \begin{algorithmic}[1]
        \State $\mathbf{F}=f(\boldsymbol{\theta})$
        \State $\mathbf{G}=f(\boldsymbol{\theta})$
        \State $\mathbf{H}=f(\boldsymbol{\theta})$
        \State $\mathbf{D}=f(\boldsymbol{\theta})$
        \State $\text{sys}=\text{ss}(\textbf{F},\textbf{G},\textbf{H},\textbf{D})$ \Comment{Construct the state-space system}
        \State $\text{y}\_\text{sim}=\text{lsim}(\text{sys}, \text{id\_data.u}, \text{id\_data.time})$ \Comment{Simulate system response}
        \State $\text{cost} = \text{rms}(\text{id}\_\text{data}\_\text{y}-\text{y}\_\text{sim})$ \Comment{Compute cost function}
        \State \Return $\text{cost}$
    \end{algorithmic}
\end{algorithm}

\noindent Since the optimization is non-convex, it requires specialized solvers.
We implement the optimization in MATLAB, using two different solvers: \texttt{fmincon} (a gradient-based solver) and \texttt{particleswarm} (a global optimization method). 
These solvers offer different advantages, and their performance is compared to determine the best approach for parameter estimation.
\section{Data representation}

\begin{definition}[\textit{Instance}]
    An instance an atomic element of information from a dataset. 
\end{definition}
\begin{definition}[\textit{Attribute}]
    An attribute measures aspects of an instance.
\end{definition}
\noindent Each instance is composed of a certain number of attributes. 
\begin{definition}[\textit{Concept}]
    A concept is a special content inside the data and is a thing that can be learned. 
\end{definition}

\subsection{Attribute types}
The attributes can be: 
\begin{itemize}
    \item \textit{Numeric}: Real-valued or integer-valued domain. 
        Can be interval-scaled (differences are meaningful) or ratio-scaled (differences and ratios are meaningful). 
        Not only ordered but measured in fixed and equal units. Characteristics
        – Difference of two values makes sense
        –Sum or product doesn’t make sense
        – Zero point is not defined
        Sometimes they are divided into “discrete” and “continuous”
    \item \textit{Categorical}: Set-valued domain composed of a set of symbols. 
        Can be nominal (only equality is meaningful) or ordinal (both equality and inequality are meaningful).
        Values are distinct symbols that serve only as labels or names. 
        Characteristics
        – No relation is implied among nominal values
        – No ordering
        – No distance measure
        – Only equality tests can be performed
        The order of attributes can be sequential, diverging or cycling. 
    \item \textit{Binary}: Represented by just two values 0/1
\end{itemize}
\noindent Some attributes may have an internal hierarchical structure

Data type nfluence the type of statistical analyses
and visualization we can perform
Some algorithms and functions
fit some specific data types best

\subsection{Encodings}
Categorical to numerical using encoder and viceversa through discretization. 
The main encoding are: 
\begin{itemize}
    \item Label Encoder: 
Map a categorical variable described by n values
into a numerical variables with values from 0 to n-1
For example, attribute Outlook would be replaced by a numerical
variables with values 0, 1, and 2. By replacing a label with a number might
influence the process in unexpected ways. If we apply label encoding, we should store the mapping used
for each attribute to be able to map the encoded data into the original o
    \item One hot encoding: Map each categorical attribute with
n values into n binary 0/1 variables
Each one describing one
specific attribute values
. One hot encoding assigns the same
numerical value (1) to all the labels
But it can generate a massive amount of variables
when applied to categorical variables with many values
    \item Categorical embeddings: Apply deep learning to map categorical variables
in into Euclidean spaces
Similar values are mapped close to each other
in the embedding space thus revealing the intrinsic
properties of the categorical variables

\end{itemize}
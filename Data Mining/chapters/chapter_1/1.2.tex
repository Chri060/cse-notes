\section{Data Mining}

\begin{definition}[\textit{Data Mining}]
    Data Mining is the non-trivial process of identifying valid, novel, potentially useful, and understandable patterns in data. 
\end{definition}
\noindent The general idea of Data Mining is to build computer programs that navigate through databases automatically, seeking patterns.
However, most patterns will be uninteresting, most patterns are spurious, inexact, or contingent on accidental coincidences in the data, and real data is imperfect, some parts will be garbled, and some will be missing.
Algorithms need to be robust enough to cope with imperfect data and to extract regularities that are inexact but useful.

\begin{definition}[\textit{Descriptive model}]
    Are the models built for gaining insight? (about what already happened)
\end{definition}
\begin{definition}[\textit{Predictive model}]
    Or are they built for accurate prediction? (about what might happen)
\end{definition}
\begin{definition}[\textit{Prescriptive model}]
    Apply descriptive and predictive mining to recommend a course of action.
\end{definition}

\subsection{Process}
The main steps of a Data Mining application are: 
\begin{enumerate}
    \item \textit{Selection}: what are data we need to answer the posed question?
    \item \textit{Cleaning}: are there any errors or inconsistencies in the data we need to eliminate?
    \item \textit{Transformation}: some variables might be eliminated because equivalent to others, some variables might be elaborated to create new variables.
    \item \textit{Mining}: select the mining approach and choose and apply the mining algorithms.
    \item \textit{Validation}: are the patterns we discovered sound? According to what criteria? are the criteria sound? Can we explain the result?
    \item \textit{Presentation}: what did we learn? Is there a story to tell? A take-home message?
\end{enumerate}

\subsection{Tasks}
The main tasks related to Data Mining are: 
\begin{itemize}
    \item \textit{Regression}: given a set of data points we aim at predicting a continuous value. 
    \item \textit{Classification}: given a set of data points we aim at predicting a discrete value such as a label. 
        We may have: logistic regression (straight line dividing the two classes), $k$-Nearest Neighbors or decision trees
    \item \textit{Clustering}: given a set of data points divide them in $k$ subsets without a predefined classification labels, grouping them by some similarity. 
    \item \textit{Associations}: find interesting data associations between elements. 
        Each association is defined with a support (how likely it is to happen), a confidence and a lift measure (how much is unexpected). 
\end{itemize}
\noindent Other than that we also have Outlier analysis
– An outlier is a data object that does not comply with the general behavior of the data
– It can be considered as noise or exception but is quite useful in rare events analysis
• Trend and evolution analysis
– Trend and deviation: regression analysis
– Sequential pattern mining, periodicity analysis
– Similarity-based analysis
• Text Mining, Topic Modeling, Graph Mining, Data Streams
• Sentiment Analysis, Opinion Mining, etc.
• Other pattern-directed or statistical analyses

\subsection{Issues}
Are all the “Discovered” Patterns Interesting?
Interestingness measures
– Data Mining may generate thousands of patterns, but typically not all of them are interesting.
– A pattern is interesting if it is easily understood by humans, valid on new or test data with
some degree of certainty, potentially useful, novel, or validates some hypothesis that a user
seeks to confirm
• Objective vs. subjective interestingness measures
– Objective measures are based on statistics and structures of patterns
–Subjective measures are based on user’s belief in the data, e.g., unexpectedness, novelty, etc.

Can We Find All and Only Interesting Patterns?
Completeness
–Find all the interesting patterns
– Can a data mining system find all
the interesting patterns?
– Association vs. classification vs. clustering
• Optimization
–Search for only interesting patterns:
– Can a data mining system find only
the interesting patterns?
– Two approaches: (1) first general all the patterns and then filter out the uninteresting ones;
(2) generate only the interesting patterns—mining query optimization
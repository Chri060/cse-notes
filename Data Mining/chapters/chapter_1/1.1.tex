\section{Introduction}

Data mining become important in the mid 1990s due to the eplosive growth of data and the pressing needs for the automated analysis of massive data. 

\paragraph*{Big Data}
Gigantic quantities of data are being collected in every domain imaginable. 
Exponential increases in computing power has led to ever decreasing costs of instrumentalization and storage. 
The quantity of data is so large that it facilitates predictions not previously possible, and it necessitates special processes to deal with. 

\paragraph*{Machine Learning}
\begin{definition}[\textit{Machine Learning}]
    A computer program is said to learn from experience $E$ with respect to some class of task $T$ and a performance measure $P$, if its performance at tasks in $T$, as measured by $P$, improves because of experience $E$.
\end{definition}
\noindent Suppose we have the experience $E$ encoded as a dataset $\mathcal{D}=x_1,\dots,x_N$, we may cathegorize Machine Learning in three main paradigms: 
\begin{itemize}
    \item \textit{Supervised Learning}: given the desired outputs $t_1,\dots,t_N$ learns to produce the correct output given a new set of input. 
    \item \textit{Unsupervised Learning}: exploits regularities in $\mathcal{D}$ to build a representation to be used for reasoning or prediction. 
    \item \textit{Reinforcement Learning}: producing actions $a_1,\dots,a_N$ which affect the environment, and receiving rewards $r_1,\dots,r_N$ learn to act in order to maximize rewards in the long term.
\end{itemize}

\paragraph*{Data science}
Data science focuses of the entire data analysis and elaboration comprising mining, statistics, interpreting and leveraging. 
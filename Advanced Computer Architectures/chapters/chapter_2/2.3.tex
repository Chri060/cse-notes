\section{Cache performance}

\begin{definition}[\textit{Memory stall cycles}]
    Memory stall cycles represent the number of cycles during which the CPU is idle, waiting for memory access to complete.
\end{definition}
We assume that the cycle time includes the time necessary to manage a cache hit and that during a cache miss, the CPU is stalled. 
We can compute:
\[\text{CPU execution time}=\left(\text{CPU clock cycles}+\text{memory stall cycles} \right) \cdot \text{clock cycle time}\]
\[\text{memory stall cycles}=\text{number of misses} \cdot \text{miss penalty}\]
Simplifying by averaging reads and writes:
\[\text{memory stall cycles} = \text{IC} \cdot \left(\dfrac{\text{memory accesses}}{\text{instruction}}\right)\cdot \text{miss rate} \cdot \text{miss penalty}\]
This value is architecture-dependent but hardware-independent.

The average access time is computed as:
\[T_{\text{access}} = H_{\text{time}} + M_{\text{rate}} \cdot M_{\text{penalty}}\]

\paragraph*{Optimization}
The access time for a cache can be optimized by:
\begin{itemize}
    \item Reducing $M_{\text{rate}}$: achieved through larger block size, larger cache size, or higher associativity.
    \item Reducing $M_{\text{penalty}}$: achieved by using multilevel caches.
    \item Reducing $H_{\text{time}}$: achieved by giving reads priority over writes.
\end{itemize}

\subsection{Cache design}
In cache design, factors such as cache size, block size, associativity, replacement policy, and the choice between write-through and write-back mechanisms play significant roles.
The best option involves balancing access patterns (workload and usage) and technological expenses. 
Often, simplicity emerges as the preferred solution.

\paragraph*{SRAM}
Static RAM (SRAM) memory requires low power to retain data and uses six transistors for each bit.

\paragraph*{DRAM}
Dynamic RAM (DRAM) must be rewritten after each read and periodically refreshed (each row every eight milliseconds). 
It requires only one transistor per bit. 
The address lines are multiplexed into upper and lower half addresses.

\paragraph*{Flash memory}
Flash memory is a type of EEPROM, requiring block erasure prior to overwrite. 
It retains data without power, classifying it as non-volatile storage. 
It has a finite number of write cycles and falls in price between SDRAM and disk storage.
Although slower than SRAM, it outpaces traditional disk speeds.

\paragraph*{Optimizations}
According to Amdahl's Law, memory capacity should grow linearly with processor speed.
However, memory capacity and speed have not kept pace with processors.
To address this issue, several optimizations are possible:
\begin{itemize}
    \item Multiple accesses to the same row.
    \item Synchronous DRAM: Adds a clock to the DRAM interface and supports burst mode with critical word first.
    \item Wider interfaces.
    \item Double Data Rate (DDR) memory.
    \item Multiple banks on each DRAM device.
\end{itemize}
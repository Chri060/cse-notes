\section{Tiny Deep Learning architectures}

Deep Learning models designed for resource-constrained devices must balance computational efficiency with accuracy.
These architectures are optimized to reduce memory usage, computation overhead, and energy consumption while maintaining competitive performance.
In this context, three prominent families of architectures have emerged: SqueezeNet, MobileNet, and EfficientNet.

\subsection{SqueezeNet}
Introduced in 2016, SqueezeNet was the first CNN architecture explicitly designed for deployment on edge devices.
It achieves a remarkable reduction in model size (up to 50 to 500 times fewer parameters than AlexNet) while maintaining comparable accuracy.
This makes SqueezeNet highly suitable for distributed training environments, where communication overhead is proportional to model size.
Additionally, its smaller footprint reduces the cost of exporting updated models to clients.

The objectives of SqueezeNet are to design a CNN with fewer parameters but equivalent accuracy to AlexNet.
To achieve this goal, SqueezeNet employs several innovative strategies:
\begin{enumerate}
    \item \textit{Replace $3\times 3$ filters with $1\times 1$ filters}: using $1\times 1$ filters instead of $3\times 3$ filters reduces the number of parameters by a factor of 9. 
        This also decreases the number of MACs. 
        However, not all $3\times 3$ filters are replaced, as they are essential for capturing spatial patterns.
    \item \textit{Reduce input channels to the $3\times 3$ filters}: to further minimize the number of parameters, SqueezeNet introduces squeeze layers that reduce the number of input channels to $3\times 3$ filters. 
        This results in fewer $3\times 3$ filters overall. 
        This is done using squeeze layers. 
    \item \textit{Downsample late in the network}: delaying downsampling preserves larger activation maps, which can improve accuracy despite using fewer weights.
        While this increases memory usage, it prevents the loss of critical features in early stages due to overly aggressive dimensionality reduction.
\end{enumerate}

\paragraph*{Fire module}
At the heart of SqueezeNet is the fire module, a novel convolutional layer designed to drastically reduce memory and computation demands while preserving accuracy. 
Each Fire module consists of two stages:
\begin{enumerate}
    \item \textit{Squeeze layer}: applies only $1\times 1$ convolutional filters. 
        Reduces the number of input channels to $s$, a hyperparameter chosen during design.
        The output dimensions is $H\times W \times s$. 
    \item \textit{Expand layer}: takes the reduced activation map as input.
        Applies $e$ $3\times 3$ filters and $e$ $1\times 1$, producing an output with $2e$ channels. 
        Need $s<2e$ to maintain computational efficiency.
        The output dimensions is $H\times W \times 2e$. 
\end{enumerate}

\subsubsection{Architecture}
The final example architecture is composed by: 
\begin{enumerate}
    \item \textit{Initial convolutional layer}: processes the input image and prepares it for subsequent Fire modules.
    \item \textit{Eight fire modules}: each module progressively increases the number of filters to capture more complex features.
        Max-pooling is applied after Fire modules 4 and 8 to downsample the feature maps.
    \item \textit{Global Average Pooling and softmax}: instead of fully connected layers, SqueezeNet uses GAP followed by a softmax layer for classification.
        This eliminates the need for dense layers, further reducing the parameter count.
\end{enumerate}
\noindent Further techniques, such as sparsity (pruning unnecessary weights) and quantization (reducing precision), can be applied to SqueezeNet to enhance efficiency without significant accuracy loss. 
The base SqueezeNet model achieves AlexNet-level accuracy while requiring far fewer parameters and MACs.

\subsubsection{Variants}
Several variants of SqueezeNet have been proposed to improve accuracy with minimal additional complexity:
\begin{enumerate}
    \item \textit{Simple bypass architecture}: adds skip connections around specific fire modules.
        These connections allow the modules to learn residual functions, improving gradient flow and accuracy.
        No extra parameters are introduced, making this variant highly efficient.
    \item \textit{Complex bypass architecture}: incorporates $1 \times 1$ convolutional layers within the skip connections.
        The number of filters in these convolutions equals the number of output channels, introducing additional trainable parameters.
        While slightly more accurate, this variant increases the model's size and computational cost.
\end{enumerate}
\noindent Among the variants, the simple bypass architecture provides the best trade-off, offering modest accuracy improvements without adding extra weights.

\subsection{MobileNet}
MobileNet is a family of efficient NN architectures designed specifically for mobile and embedded vision applications.
It achieves lightweight computation through the use of depth-wise separable convolutions, a streamlined approach that significantly reduces the number of parameters and computational cost while maintaining competitive accuracy.
This architecture has become a cornerstone for deploying Deep Learning models on resource-constrained devices.

\paragraph*{Depth-wise scalable convolution}
The key innovation in MobileNet is the use of depth-wise separable convolutions, which decompose a standard convolution into two simpler operations:
\begin{enumerate}
    \item \textit{Depth-wise convolution}: applies a single $R\times S$ filter to each input channel independently.
        Unlike a standard convolution, this step does not combine information across channels.
        The output dimensions is $H\times W\times C$, where $C$ is the number of input channels.
    \item \textit{Point-wise convolution}: combines the outputs of the depth-wise convolution across channels using $M$ $1\times 1$ filters.
        This step aggregates channel-wise information, producing the final output feature map.
\end{enumerate}

\begin{table}[h!]
    \centering
    \begin{tabular}{|l|c|c|}
    \hline
    \textbf{Convolution} & \textbf{MAC operations} & \textbf{Weights} \\ \hline
    Standard & $R \times S \times C \times E \times F \times M$ & $R \times S \times C \times M$ \\
    Depth-wise & $R \times S \times C \times E \times F + C \times M \times E \times F$ & $R \times S \times C + C \times M$ \\ \hline
    \end{tabular}
\end{table}
By decoupling spatial and channel-wise computations, depth-wise separable convolutions achieve significant reductions in both MAC operations and the number of weights compared to standard convolutions.

\subsubsection{Architecture}
Each block in MobileNet follows a consistent structure:
\begin{enumerate}
    \item \textit{Depth-wise convolution}: applies a $3\times 3$ filter to each input channel independently.
        Reduces computational cost by avoiding cross-channel interactions at this stage.
    \item \textit{Batch normalization}: normalizes the outputs of the depth-wise convolution to stabilize training.
    \item \textit{ReLU activation}: introduces non-linearity to the model.
    \item \textit{Point-wise convolution}: combines information across channels using $1\times 1$ convolutions.
    \item \textit{Batch normalization}: normalizes the outputs of the point-wise convolution.
    \item \textit{ReLU activation}: applies another non-linear transformation.
\end{enumerate}

\subsection{EfficientNet}
EfficientNet represents a breakthrough in the design of efficient CNNs by introducing a novel method for scaling up models. 
Unlike traditional approaches that focus on increasing either depth (layers), width (filters), or resolution individually, EfficientNet employs compound scaling, which scales all three dimensions simultaneously while maintaining a balanced relationship between them.

The key idea behind compound scaling is to scale depth ($\alpha^\phi$), width ($\beta^\phi$), and resolution ($\gamma^\phi$) with constant ratios: 
\[\alpha\beta^2\gamma^2\approx 2 \qquad \alpha,\beta,\gamma \geq 1\]
Here, $\alpha$, $\beta$, and $\gamma$ are constants that control how each dimension is scaled, and $\gamma$ is a user-specified coefficient that determines the extent of scaling (how many more resources are available). 
EfficientNet's compound scaling is implemented in two steps:
\begin{enumerate}
    \item \textit{Determine scaling coefficients ($\alpha, \beta, \gamma$)}: start with a baseline network and assume twice the available computational resources (double $\phi$).
        Perform a small grid search to find optimal values of $\alpha$, $\beta$, and $\gamma$, subject to the constraint $\alpha\beta^2\gamma^2\approx 2$.
        Fix these coefficients as constants for subsequent scaling.
    \item \textit{Scale the baseline network}: use the fixed scaling coefficients to scale the baseline network to different sizes by varying $\phi$.
        Increasing $\phi$ corresponds to enlarge the model, achieving higher accuracy at the cost of increased computational resources.
\end{enumerate}
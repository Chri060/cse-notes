\section{Artificial Intelligence neutrality}

While AI is often discussed in technical terms, it is shaped by a series of social, ethical, and political decisions. 
From the initial stages of data collection to the implementation of classification mechanisms, each step in the development and deployment of AI systems involves value-laden choices.

AI systems—especially at the edge—are not merely computational tools. 
They are social-technical systems (Johnson and Verdicchio, 2017), shaped by the interplay between technological artifacts, human behavior, institutional arrangements, and social meanings.

These systems embody several important choices:
\begin{itemize}
    \item How care is delivered and organized.
   \item How health systems integrate AI
   \item Whether medicine becomes increasingly personalized.
   \item How data are treated (as assets, liabilities, or public goods).
\end{itemize}
\noindent However, these systems also raise concerns:
\begin{itemize}
    \item High computational demands for model training and maintenance.
    \item Simplifications in measurement.
    \item Tensions between technical feasibility and clinical or contextual relevance.
    \item Questions of professional judgment, accountability, and ethical oversight.
\end{itemize}

These considerations challenge the notion that AI systems are neutral or purely objective. 
As Johnson and Verdicchio (2017) argue, AI systems should be understood as embedded within complex social-technical environments:
\begin{itemize}
    \item They embody political, philosophical, and technical choices
    \item They consist of interactions between artifacts, human users, and institutional settings
    \item Their function depends not only on technical accuracy but also on meaning—how people interpret and respond to the system's outputs
\end{itemize}
\noindent In short, AI systems are never value-free. 
They are deeply shaped by—and in turn help shape (the social contexts in which they operate). 
Recognizing this non-neutrality is essential for responsible development, deployment, and governance of AI, especially in sensitive areas such as health, education, and public infrastructure.
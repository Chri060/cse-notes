\section{Introduction}

We want to address the level siz of TinyML in which we have both training and inference entirely on end device. 
We want to adapt at various levels: 
\begin{enumerate}
    \item \textit{Environmental}: we may have to analyze data coming from different situations and weather so the data may change. 
        Therefore, we need to adapt the data acquisition and preprocessing based on the actual state of the environment. 
    \item \textit{System}: the system itself may operate under different hardware constraints, network conditions, or computational capabilities. 
        Thus, it is essential to adapt resource allocation, communication protocols, and processing strategies to ensure robustness, efficiency, and real-time performance across heterogeneous setups.  
    \item \textit{Task}: different tasks may require different models, processing pipelines, or decision-making criteria.
        Adapting to the task means selecting the appropriate algorithms, models, or data representations dynamically.  
    \item \textit{User}: users may have different preferences, goals, or behaviors. 
        Therefore, the system should adapt. 
\end{enumerate}
\noindent The problem with adaptation is that the system may change its behavior or not function properly.

\subsection{Technological perspective}
The most significant technological challenge lies in computational constraints, particularly memory limitations (for storing data and model weights in RAM).
Training is considerably more demanding than inference in terms of computational time.
Since the new data lacks labels, an alternative approach is needed to classify the incoming data samples.
Validation also becomes more complex, as it is increasingly difficult to compare different solutions effectively.
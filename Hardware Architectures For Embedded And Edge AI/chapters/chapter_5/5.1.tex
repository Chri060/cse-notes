\section{Datafication}

Data plays a fundamental role in enabling current forms of AI, including applications at the edge. However, the acquisition and use of data come with significant challenges. 
These include scarcity, high costs, and ethical or regulatory limitations. 
In addition, growing concerns about the energy consumption and environmental impact of data-centric technologies have been raised (UN, 2024).

We are witnessing an unprecedented expansion in the presence and production of data (a trend driven by the increasing importance of datafication).
\begin{definition}[\textit{Datafication}]
    Datafication refers to the process of transforming objects, behaviors, and activities into digital data.
\end{definition}
\noindent This is not a purely technical operation but one that is embedded in broader social, economic, and political contexts.
For datafication to occur, several key conditions and actors must be present:
\begin{enumerate}
    \item \textit{Community}: a network of users, professionals, researchers, hackers, and innovators who interact with and contribute to the data ecosystem.
    \item \textit{Care}: the values, incentives, practices, and regulatory standards that ensure the responsible handling of data and maintain trust in data-driven systems.
    \item \textit{Capacities}: the technical infrastructure required for data production and processing, including sensors, platforms, cloud computing, storage, and analytical tools.
    \item \textit{Data}: the actual outputs of datafication which includes digital traces, sensor measurements, content, metadata, and more.
\end{enumerate}
Importantly, data do not exist or carry meaning on their own. 
Their utility and legitimacy depend on the interplay between these four elements.

Thus, datafication is not merely a technological process; it is deeply social and political. 
It requires alignment between infrastructure, stakeholders, and norms. 
When these elements are not in sync—when there is resistance, lack of care, or misalignment of interests—the processes of datafication can falter or generate conflict.
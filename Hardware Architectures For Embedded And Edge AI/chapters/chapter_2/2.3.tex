\section{Sensors and signals}

Sensors are used to acquire measurements from the environment or from human interactions. 
They generate continuous streams of data, which can be used for various AI-driven applications. 
In addition to sensor-generated data, other sources such as digital device logs, network packets, and radio transmissions can also provide valuable information. 
Sensors can output data in different formats depending on their purpose and design.

\paragraph*{Data storage}
Data values can be stored in various formats, depending on precision and memory constraints. 
Boolean values (1 bit) represent binary states with two possible values. 
An 8-bit integer can store up to 256 distinct values, while a 16-bit integer extends this range to 65,536 possible values. 
A 32-bit floating point number can represent a wide range of values with up to seven decimal places, reaching a maximum of approximately $3.4\times 10^{38}$.
Quantization techniques help optimize memory usage by reducing the required storage for each value while maintaining sufficient precision for AI computations.
\section{Sensors and signals}

Sensors are used to acquire measurements from the environment or from human interactions. 
They generate continuous streams of data, which can be used for various AI-driven applications. 
In addition to sensor-generated data, other sources such as digital device logs, network packets, and radio transmissions can also provide valuable information. 
Sensors can output data in different formats depending on their purpose and design.

\paragraph*{Data storage}
Data values can be stored in various formats, depending on precision and memory constraints. 
Boolean values (1 bit) represent binary states with two possible values. 
An 8-bit integer can store up to 256 distinct values, while a 16-bit integer extends this range to 65,536 possible values. 
A 32-bit floating point number can represent a wide range of values with up to seven decimal places, reaching a maximum of approximately $3.4\times 10^{38}$.
Quantization techniques help optimize memory usage by reducing the required storage for each value while maintaining sufficient precision for AI computations.

\subsection{Data}
\paragraph*{Time series data}
A time series is a sequence of data points recorded in chronological order:
\[X=(x_1,x_2,\dots,x_N)\] 
\noindent Essentially, it represents observations collected at consistent time intervals.
Key factors to consider include:
\begin{itemize}
    \item \textit{Sampling period}: the time gap between consecutive data points.
    \item \textit{Bit depth} ($n$): the number of bits used to represent each value.
    \item \textit{Memory usage}: each sample requires $n$ bits of storage.
\end{itemize}

\paragraph*{Audio data}
Audio data is a specific type of time series, representing sound wave oscillations as they travel through air.
Key parameters are:
\begin{itemize}
    \item \textit{Sampling rate} (Hz): the number of samples taken per second.
    \item \textit{Quantization} (bit depth): the precision of each sample.
    \item \textit{Signal duration} (s): the total length of the recording.
    \item \textit{Number of channels}: mono (single channel) or stereo (two channels).
\end{itemize}
\noindent Memory consumption is calculated as:
\[\text{length}\times\text{sampling}\times\text{bit}\times\text{channel}\]

\paragraph*{Image data}
Images capture visual information as a grid of pixels, where each pixel represents a specific property of the scene.
Key characteristics are:
\begin{itemize}
    \item \textit{Dimensions} ($W \times H$): width and height of the image.
    \item \textit{Bit depth} ($N$): the number of bits used to store each pixel.
    \item \textit{Number of channels}: typically 1 (grayscale) or 3 (RGB).
\end{itemize}
\noindent Memory usage is given by:
\[W \times K \times N \times \text{channels}\]

\paragraph{Video data}
Videos are sequences of images displayed rapidly to create motion. 
They share the same structure as images but add an extra dimension: time.
Critical parameters:
\begin{itemize}
    \item \textit{Resolution} ($W \times H$): width and height of each frame.
    \item \textit{Bit depth} ($N$): bits per pixel.
    \item \textit{Number of channels}: defines color representation.
    \item \textit{Frame rate} (fps): number of frames per second.
    \item \textit{Duration} (s): total length of the video.
\end{itemize}
\noindent Memory requirements are determined by:
\[W \times K \times N \times \text{channels} \times \text{frame rate} \times \text{length}\]

\subsection{Sensors}
There are thousands of different types of sensors available, each designed to capture specific kinds of data. 
In the context of embedded and edge AI, sensor technologies can be categorized into six main families:
\begin{enumerate}
    \item \textit{Acoustic and vibration}: detects sound and mechanical vibrations.
    \item \textit{Visual and scene}: captures images, video, and environmental light data.
    \item \textit{Motion and position}: measures movement, acceleration, and spatial positioning.
    \item \textit{Force and tactile}: detects pressure, touch, and force.
    \item \textit{Optical, electromagnetic, and radiation}: measures light, radio waves, and radiation levels.
    \item \textit{Environmental and chemical}: monitors temperature, humidity, gases, and other environmental factors.
\end{enumerate}

\paragraph*{Acoustic and vibration}
Detecting vibrations is a crucial capability in embedded and edge AI. 
These sensors allow systems to perceive movement, structural vibrations, and even communication signals from humans and animals at a distance.
Acoustic sensors measure vibrations traveling through different media: air (microphones), water (hydrophones), and ground (geophones and seismometers).
Since acoustic data is distributed across different frequencies, the sampling frequency plays a key role in ensuring accurate representation for a given application. 
These sensors typically produce audio data as their output.

\paragraph*{Visual and scene}
Visual sensors capture information about the environment without direct contact. 
These range from tiny, low-power cameras to high-resolution multi-megapixel sensors.
Key characteristics of image sensors: color channels, spectral response (infrared sensors), pixel size, resolution, and frame rate.
The output of these sensors can be 2D or 3D images or video data, depending on the application.

\paragraph*{Motion and position}
Motion and position sensors track movement and spatial positioning in various ways:
\begin{itemize}
    \item \textit{Tilt sensors}: simple mechanical switches that detect orientation changes.
    \item \textit{Accelerometers}: measure acceleration along one or more axes.
    \item \textit{Gyroscopes}: detect rotational movement.
    \item \textit{Time-of-flight sensors}: emit light or radio waves to measure distances to objects.
    \item \textit{Real-time locating systems}: use multiple transceivers placed around a space to track object positions.
    \item \textit{Global Positioning System}: uses satellites to determine an object's precise location.
\end{itemize}
\noindent  These sensors typically generate time-series data, tracking movement and positioning over time.
\section{Sensors and signals}

Sensors are used to acquire measurements from the environment or from human interactions. 
They generate continuous streams of data, which can be used for various AI-driven applications. 
In addition to sensor data, other sources. 
Sensors can output data in different formats depending on their purpose and design.

\subsection{Data}
\paragraph*{Time series}
A time series is a sequence of data points recorded in chronological order. 
Essentially, it represents observations collected at consistent time intervals.
Key factors to consider include the sampling period (time gap between consecutive data points) and bit depth (number of bits used to represent each value).

\paragraph*{Audio}
Audio is a specific type of time series, representing sound wave oscillations as they travel through air.
Key parameters are: the sampling rate (number of samples taken per second), quantization (bit depth), length (duration of the recording), and the number of channels (mono or stereo).
Memory consumption is calculated as:
\[\text{length}\times\text{sampling}\times n \times\text{channel}\]

\paragraph*{Image}
Images capture visual information as a grid of pixels, where each pixel represents a specific property of the scene.
Key characteristics are: resolution (width and height of the image), bit depth, and channels (RGB or gray-scale).
Memory usage is given by:
\[H \times W \times n \times \text{channels}\]

\paragraph{Video}
Videos are sequences of images displayed rapidly to create motion. 
They share the same structure as images but add the time.
The important parameters are the same as the ones of the images with also the frame rate and duration. 
Memory requirements are determined by:
\[H \times W \times n \times \text{channels} \times \text{frame rate} \times \text{length}\]

\subsection{Sensors}
There are thousands of different types of sensors available, each designed to capture specific kinds of data. 
In the context of embedded and edge AI, sensor technologies can be categorized into six main families:
\begin{enumerate}
    \item \textit{Acoustic and vibration}: detects sound and mechanical vibrations.
    \item \textit{Visual and scene}: captures images, video, and environmental light data.
    \item \textit{Motion and position}: measures movement, acceleration, and spatial positioning.
    \item \textit{Force and tactile}: detects pressure, touch, and force.
    \item \textit{Optical, electromagnetic, and radiation}: measures light, radio waves, and radiation levels.
    \item \textit{Environmental and chemical}: monitors temperature, humidity, gases, and other environmental factors.
\end{enumerate}

\paragraph*{Acoustic and vibration}
Detecting vibrations is a crucial capability in embedded and edge AI. 
These sensors allow systems to perceive movement, structural vibrations, and even communication signals from humans and animals at a distance.
Acoustic sensors measure vibrations traveling through different media: air (microphones), water (hydrophones), and ground (geophones and seismometers).
Since acoustic data is distributed across different frequencies, the sampling frequency plays a key role in ensuring accurate representation for a given application. 
These sensors typically produce audio data as their output.

\paragraph*{Visual and scene}
Visual sensors capture information about the environment without direct contact. 
These range from tiny, low-power cameras to high-resolution multi-megapixel sensors.
Key characteristics of image sensors: color channels, spectral response (infrared sensors), pixel size, resolution, and frame rate.
The output of these sensors can be 2D or 3D images or video data, depending on the application.

\paragraph*{Motion and position}
Motion and position sensors track movement and spatial positioning in various ways. 
Examples includes: tilt, accelerometers, gyroscopes, time-of-flight and Global Navigation Satellite System (GNSS). 
These sensors typically generate time-series data, tracking movement and positioning over time.

\paragraph*{Force and tactile}
These sensors help users interact with devices, understand fluid and gas flow, or measure mechanical strain on objects. 
Example includes buttons, capacitive touch sensors, strain gauges, load cells, flow and pressure sensors. 

\paragraph*{Optical, electromagnetic, and radiation}
These sensors detect electromagnetic radiation, magnetic fields, and electrical properties. 
Examples includes: photo-sensors, color sensor, spectroscopy, magnetometers, proximity sensor, electromagnetic field meters and current sensors. 

\paragraph*{Environmental, biological, and chemical}
These sensors track environmental conditions, biological signals, and chemical presence. 
Examples includes: temperature, gas, bio signals and chemical sensors. 
Most of these measurements are typically recorded as time-series data, allowing for trend analysis and predictive insights.
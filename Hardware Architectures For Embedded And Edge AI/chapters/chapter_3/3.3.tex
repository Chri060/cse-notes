\section{Machine and Deep Learning algorithms}

\paragraph*{Functionality}
Machine and Deep Learning algorithms can be categorized based on their functionality:  
\begin{itemize}
    \item \textit{Classification}: assigns input data to predefined categories, such as identifying spam emails or recognizing handwritten digits.
    \item \textit{Regression}: predicts continuous values based on input data, such as forecasting stock prices or estimating house prices.
    \item \textit{Object detection}: identifies and locates objects within an image or video, commonly used in autonomous driving and security surveillance.
    \item \textit{Segmentation}: divides an image into meaningful regions, such as medical image analysis or self-driving car lane detection.
    \item \textit{Anomaly detection}: identifies rare or unusual patterns in data, useful in fraud detection and predictive maintenance.
    \item \textit{Prediction}: forecasts future outcomes based on historical data, such as weather prediction or demand forecasting.
    \item \textit{Feature reduction}: reduces the dimensionality of data while preserving essential information, improving efficiency in Machine Learning models.
\end{itemize}

\paragraph*{Implementation}
Machine and deep learning algorithms can be categorized based on their implementation approach:  
\begin{itemize}
    \item \textit{Conditional logic}: rule-based systems that use explicit if-then conditions to make decisions, commonly found in expert systems and traditional automation.
        The advantages of conditional logic are: determinism, efficiency, and there is no need for training data. 
    \item \textit{Machine Learning}: algorithms that learn patterns from data and make predictions without explicit programming.
    \item \textit{Deep Learning}: a subset of machine learning that utilizes multi-layered neural networks to model complex patterns.
\end{itemize}

\subsection{Tiny Machine Learning}
Tiny Machine Learning focuses on bringing machine learning and deep learning capabilities to small, low-power devices, particularly in the Internet of Things. 
This enables real-time data processing directly on the device, reducing reliance on cloud computing and improving efficiency, privacy, and responsiveness.  

The key steps in deploying TinyML applications are:  
\begin{enumerate}
    \item \textit{Set up the hardware}: choose a suitable microcontroller or edge device with sufficient processing power and energy efficiency.
    \item \textit{Install the software}: set up the necessary development tools, frameworks, and libraries.
    \item \textit{Collect data}: gather relevant sensor data to train the model.
    \item \textit{Train the model}: develop and optimize a lightweight machine learning model tailored for low-power devices.
    \item \textit{Build the application}: integrate the trained model into an application that interacts with the device.
    \item \textit{Optimize and compile}: convert the application into a format suitable for deployment on the target hardware.
    \item \textit{Deploy to the device}: flash the compiled application onto the microcontroller or embedded system.
    \item \textit{Test and monitor}: evaluate the performance of the deployed model in real-world conditions and refine as needed.
\end{enumerate}

%\begin{tabular}{|l|p{4cm}|p{4cm}|p{4cm}|}

\begin{table}[h]
    \centering
    \begin{tabular}{|c|l|}
        \hline
        \textbf{Level} & \textbf{Category} \\  
        \hline
        \textbf{6} & End device training and inference \\  
        \hline
        \textit{5} & Edge training and inference \\  
        \hline
        \textit{4} & Cloud-edge co-training and inference \\  
        \hline
        \textbf{3} & Cloud training, on-device inference \\  
        \hline
        \textit{2} & Cloud training, edge inference \\  
        \hline
        \textit{1} & Cloud training, cloud and edge inference \\  
        \hline
                   & Cloud training and inference \\  
        \hline
    \end{tabular}
\end{table}
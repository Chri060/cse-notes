\section{Data preprocessing and feature extraction}

The goal of data preprocessing and feature extraction  is to convert raw signals into structured data that can be processed by Machine Learning and Deep Learning algorithms.
This process involves three key steps:
\begin{enumerate}
    \item \textit{Signal chucking}: breaking the continuous signal into smaller, manageable chunks of data.
    \item \textit{Data processing}: applying digital signal processing techniques to reduce noise and enhance the most relevant parts of the signal.
    \item \textit{Feature extraction}: identifying and extracting meaningful patterns or characteristics from the processed signal chunks to be used in model training.
\end{enumerate}

\subsection{Data segmentation}
Sensors generate continuous streams of data, which must be divided into smaller segments (windows) for processing. 
Each window represents a chunk of data that is analyzed by an algorithm to produce meaningful results.
\begin{definition}[\textit{Latency}]
    Latency refers to the time required to process a single chunk of data.
\end{definition}
\noindent Latency plays a crucial role in determining how efficiently an embedded system can process data. 
Lower latency allows for a higher number of processed chunks per unit of time. However, there is a trade-off:
\begin{itemize}
    \item Larger windows increase latency but provide more information, often leading to improved accuracy.
    \item Smaller windows reduce latency but may capture less useful data, potentially affecting performance.
\end{itemize}
\noindent Windows can be overlapping (ensuring no information is lost from the signal), non-overlapping (which may be more computationally efficient but risks missing important details).

\paragraph*{Frame rate}
The frame rate defines how frequently the system can acquire and process data, similar to how frame rate applies to image and video streaming.

High latency can hinder real-time analysis by preventing the system from processing new incoming data while still handling previous chunks, potentially leading to data loss. 
Optimizing both latency and frame rate is essential for effective data segmentation and real-time performance.

\subsection{Data processing}
Data preprocessing using digital processing algorithms typically involves three key steps:
\begin{enumerate}
    \item \textit{Reconstruction of missing data}:
        \begin{itemize}
            \item \textit{Global filling methods}: filling in missing data based on patterns and trends observed across the entire dataset.
            \item \textit{Local filling methods}: estimating missing values using nearby data points. Examples include forward fill, moving averages, and local interpolation techniques.
            \item \textit{Deletion of affected time periods}: in some cases, time periods with missing data are removed entirely to avoid inaccuracies in analysis.
        \end{itemize}
    \item \textit{Resampling}:
        \begin{itemize}
            \item \textit{Time series resampling}: handling time series data with varying sampling frequencies.
                In upsampling we increase the sampling rate by replicating or interpolating between timestamps.
                In downsampling we reduce the sampling rate through sub-sampling.
                Be cautious of aliasing when changing the sampling frequency.
            \item \textit{Image resampling}: adjusting the spatial resolution (pixels per image).
                In downsampling we reduce the image resolution by decreasing the number of pixels.
                In interpolation we increase the image resolution by adding pixels based on existing data.
            \item \textit{Shape modification for images}: in cropping we trim parts of the image.
                In resizing we adjust the image dimensions while maintaining or altering aspect ratios.
        \end{itemize}
    \item \textit{Filtering}: we can use different types of filter: 
        \begin{itemize}
            \item \textit{Low-pass filter}: retains low frequencies, removing high-frequency noise. Pay attention to the cutoff frequency and frequency response.
            \item \textit{High-pass filter}: retains high frequencies, removing low-frequency components. The cutoff frequency and response are important here as well.
            \item \textit{Band-pass filter}: keeps a specific range of frequencies, removing those outside the band.
        \end{itemize}
\end{enumerate}

\subsection{Feature extraction} 
Feature extraction can be applied across various domains and types of data:
\begin{itemize}
    \item \textit{Time domain}: features like mean, PCA eigenvalues, amplitude, signal-to-noise ratio (SNR), peak decay, and energy can be extracted.
    \item \textit{Frequency domain}: features such as maximum amplitude, dominant frequency, and peak variance are commonly extracted.
    \item \textit{Images}: features like edges, corners, blobs, and ridges (curves) are often detected.
\end{itemize}

\paragraph*{Sensor fusion}
In sensor fusion, the goal is to combine data from multiple sensors rather than relying on a single sensor. 
Each sensor provides unique perspectives and data, and by combining these diverse inputs, we create a more comprehensive and accurate representation of the environment or system.

The features extracted from each sensor are merged to enhance the robustness, accuracy, and reliability of the analysis. 
These fused features are then used in Machine Learning or Deep Learning models for further processing, ultimately enabling more precise and informed decision-making.

\paragraph*{Normalization and standardization}
For more efficient training, data should be normalized or standardized. 
Features with different scales can negatively impact the performance of Machine Learning models, potentially leading to underfitting.
\begin{itemize}
    \item \textit{Normalization}: scales data to a range of $[0,1]$:
        \[x^\prime=\dfrac{x-x_{\min}}{x_{\max}-x_{\min}}\]
    \item \textit{Standardization}: centers data around zero mean and with a unitary standard deviation, but does not limit it to a specific range:
        \[x^\prime=\dfrac{x-\mu}{\sigma}\]
        Here, $\mu$ is the mean and $\sigma$ is the standard deviation. 
\end{itemize}
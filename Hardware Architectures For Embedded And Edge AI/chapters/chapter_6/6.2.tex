\section{Classification}

AI systems, particularly those deployed at the edge, function fundamentally as classification machines. 
These systems are designed to categorize and interpret input data to perform tasks such as activity recognition, automated measurement and analysis, and generating recommendations and indications. 

At the core of these capabilities lies the act of classification.
As Bowker and Star (2000) argue, classification is not merely a technical activity but a foundational operation in scientific inquiry and social organization. 
It involves:
\begin{itemize}
    \item The development of categories and kinds.
    \item Structuring the ways we understand, predict, and explain the world.
    \item Enabling inference, explanation, and decision-making.
\end{itemize}
\noindent However, classification is not a neutral or purely objective process. It carries with it the power to naturalize particular views of the world. That is:
\begin{itemize}
    \item Categories created by AI systems are not necessarily natural or inevitable.
    \item To classify phenomena is to reify the categories.
    \item Once institutionalized, these categories can stabilize.
\end{itemize}
This means that data and classification schemes actively shape the world rather than merely describing it. 
They define what is seen, what is ignored, and how meaning is constructed.

Classifying people or behaviors can reinforce specific interpretations and value systems.
These classifications include certain identities or actions while excluding others.
Over time, they not only assign meaning to entities but also lend authority to the categories themselves.
Thus, AI systems do not just use data; they participate in the making of social and material realities.
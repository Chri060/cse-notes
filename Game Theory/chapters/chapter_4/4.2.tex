\section{Nash equilibrium}

\begin{definition}[\textit{Nash equilibrium}]
    A Nash equilibrium profile for $(X,Y,f:X\times Y\rightarrow\mathbb{R},g:X\times Y\rightarrow\mathbb{R})$ is a pair $(\bar{x},\bar{y})\in X \times Y$ such that $f(\bar{x},\bar{y})\geq f(x,\bar{y})$ for all $x\in X$ and $f(\bar{x},\bar{y})\geq f(\bar{x},y)$ for all $y\in Y$.
\end{definition}
\noindent A Nash equilibrium profile is a joint combination of strategies which is stable with respect to unilateral deviations of any individual player. 
At equilibrium, neither player can improve their utilities by changing strategy. 
In fact, it is not even convenient for the players to change, given that each one takes for granted that the other one will play the selected strategy. 

The main ideas of the Nash model can be seen with two player: having more players does not add complexity to the concept (except for the notation). 
Let us consider a  $n$-player game with strategy sets $X_i$ for each player and payoffs $u_i:X\rightarrow\mathbb{R}$ with $X=\prod_{i=1}^{n}X_i$. 
Let $x=(x_1,\dots,x_{i-1},x_i,x_{i+1},\dots,x_n)$ be a strategic profile $x_{-i}$ denotes the vector $x_{-i}=(x_1,\dots,x_{i-1},x_{i+1},\dots,x_n)$ and write also $x=(x_i,x_{-i})$ to emphasize the role of $x_i$. 
Then, $\bar{x}=(\bar{x}_i)_{i=1}^{n}$ is a Nash equilibrium profile if for every $i$, for every $x\in X_i$: 
\[u_i(\bar{x})\geq u_i(x,\bar{x}_{-i})\]

The notion of Nash equilibrium provides a new definition of rationality. 
We have to see the connection with dominant strategies, backward induction, and optimal strategies in zero sum games. 

\subsection{Dominant strategies}
Suppose $\bar{x}$ is a weakly dominant strategy for Player 1: 
\[f(\bar{x},y)\geq f(x,y)\qquad \forall x,y\]
\noindent If $\bar{y}$ maximizes the function $y\rightarrowtail g(\bar{x},y)$ for Player 2, then $(\bar{x},\bar{y})$ is a Nash equilibrium profile. 
In fact, for $\bar{x}$ weakly dominant it is true, in particular, that $f(\bar{x},\bar{y})\geq f(x,\bar{y})$ for all $x\in X$, thereby satisfying Nash condition on the utility function $f$ of player 1. 
Then, the maximization requirement that $\bar{y}$ be such that $g(\bar{x},\bar{y})\geq g(\bar{x},y)$ for all $y\in Y$ naturally satisfies the Nash condition on the utility function $g$ of Player 2. 

\paragraph*{Non uniqueness}
Let us suppose $\bar{y}$ maximizes the function $y\rightarrowtail g(\bar{x},y)$: 
\begin{itemize}
    \item If $\bar{x}$ is a weakly dominant strategy for Player 1, then other Nash equilibria beyond $(\bar{x},\bar{y})$ can exists.
    \item If $\bar{x}$ is a strictly dominant strategy for Player 1, then no other Nash equilibria exists different from the above ones. 
\end{itemize}
\begin{proof}
    Assume that there us another Nash equilibrium $(x_i,y_i)$ different than $(\bar{x},\bar{y})$. 
    By definition, it implies the fact that $f(x_i,y_i)\geq f(\bar{x},y_j)$.
    However: 
    \begin{itemize}
        \item If $\bar{x}$ is weakly dominant, since $f(\bar{x},y)\geq f(x,y)$ for all $x$ and all $y$ it follows in particular that such inequality holds for $y_j$, that is $f(\bar{x},y_i)\geq f(x_i,y_i)$, which is consistent with the above fact. 
            Hence, $(x_i,y_i)$ can be a Nash equilibrium. 
        \item If $\bar{x}$ is strictly dominant, since $f(\bar{x},y)>f(x,y)$ for all $x$ and all $y$ it follows in particular that such inequality holds for $y_j$, that is $f(\bar{x},y_i)> f(x_i,y_i)$, but that is in contradiction with the above fact. 
            Hence, no pair $(x_i,y_i)$ other than $(\bar{x},\bar{y})$ can be a Nash equilibrium. 
    \end{itemize}
\end{proof}

\subsection{Backward induction}
Backward induction provides a Nash equilibrium for a game of perfect information, since players systematically make an optimal choice in every part of the tree of the game. 
It is possible that in games of perfect information there are more equilibria than the ones provided by backward induction. 

\subsection{Zero sum games}
\begin{theorem}
    Let $X,Y$ be nonempty sets and $f:X\times Y \rightarrow \mathbb{R}$ a function (so that in zero sum games $g(x,y)=-f(x,y)$). 
    Then, the following are equivalent: 
    \begin{enumerate}
        \item The pair $(\bar{x},\bar{y})$ fulfills: 
            \[f(x,\bar{y})\geq f(\bar{x},\bar{y})\geq f(\bar{x},y)\qquad\forall x,y \]
        \item The following conditions are satisfied: 
        \begin{align*}
            &\inf_y\sup_x f(x,y)=\sup_x\inf_y f(x,y) \\
            &\inf_y f(\bar{x},y)=\sup_x\inf_y f(x,y) \\
            &\sup_x f(\bar{x},y)=\inf_y\sup_x f(x,y)
        \end{align*}
    \end{enumerate}
\end{theorem}
\noindent By the first condition, the equilibrium $(\bar{x},\bar{y})$ yields conservative values for both players. 
By the second condition, the conservative value agree; moreover the players must solve independent problems: $\bar{x}$ maximizes $f(\cdot,y)$ and $\bar{y}$ minimizes $f(x,\cdot)$.
\begin{proof}
    Starting from the first we have: 
    \[v_2=\inf_y\sup_x f(x,y)\leq \sup_x f(x, \bar{y})=f(\bar{x},\bar{y})=\inf_y f(\bar{x},y)\leq \sup_x\inf_y f(x,y)=v_1\]
    Since $v_1\leq v_2$ always holds, all above inequalities are equalities. 

    Suppose now that the second condition holds, we have that: 
    \[\inf_y\sup_x f(x,y)=\sup_x f(x,\bar{y})\geq f(\bar{x},\bar{y})\geq \inf_y f(\bar{x},y)=\sup_x\inf_y f(x,y)\]
    Because of the first consequence of the second condition, all inequalities are equalities. 
\end{proof}
\noindent As a consequence, given a general zero sum game $X,Y,f:X\times Y\rightarrow\mathbb{R}$:
\begin{itemize}
    \item Any Nash equilibrium $(\bar{x},\bar{y})$ provides optimal strategies for the players. 
        $f(\bar{x},\bar{y})=v$ is the value of the game. 
    \item Any pair of optimal strategies $\bar{x}$ for the Player 1 ant $\bar{y}$ for Player 2 are such that $(\bar{x},\bar{y})$ is a Nash equilibrium profile of the game and $f(\bar{x},\bar{y})=v$. 
\end{itemize}
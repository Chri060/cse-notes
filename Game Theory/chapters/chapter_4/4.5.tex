\section{Potential games}

Consider a finite game with strategy sets $X_i$ and suppose that all the players have the same payoff $p:Z\rightarrow\mathbb{R}$, that is for all $i$, the utility function are: 
\[u_i(x_1,\dots,x_n)=p(x_1,\dots,x_n)\]
\noindent If $\bar{x}=(\bar{x}_1,\dots,\bar{x}_n)\in X$ is a strategy profile such that $p(\bar{x})\geq p(x)$ for all strategy profiles $x\in X$, then $\bar{x}$ is a Nash equilibrium in pure strategies. 
Note that there might be other Nash equilibria in pure o mixed strategies. 
However, $\bar{x}$ is the best strategy for all players. 

Consider the following payoff-improving procedure: 
\begin{enumerate}
    \item Start from an arbitrary strategy profile $(x_1,\dots,x_n)\in X$. 
    \item Ask if any player has a better strategy $x_i^\prime$ that strictly increases her payoff. 
        If yes, replace $x_i$ with $x_i^\prime$ and repeat. 
        Otherwise stop, ew have found a pure Nash equilibrium profile. 
\end{enumerate}
\noindent Each iteration strictly increases the value $p(x)$, so that no strategy profile $x\in X$ can be visited twice. 
Since $X$ is a finite set, the procedure must reach a pure Nash equilibrium after at most $\left\lvert X \right\rvert$ steps.
Therefore, this procedure guarantees to reach the global minimum $\bar{x}$. 

Consider now an arbitrary finite game with payoffs $u_i:X\rightarrow\mathbb{R}$. 
We can add a constant $c_i$ to the payoff of player $i$: 
\[\tilde{u}_i(x_1,\dots,x_n)=u_i(x_1,\dots,x_n)+c_i\]
If, instead, $c_i$ depends only on $x_{-i}$ and not on $x_i$, the best responses and equilibria remain the same. 
\begin{definition}
    The payoffs $\tilde{u}_i$ and $u_i$ are said diff-equivalent for player $i$, if the difference: 
    \[\tilde{u}_i(x_1,\dots,x_n)-u_i(x_1,\dots,x_n)=c_i(x_{-i})\]
    does not depend on her decision $x_i$ but only on the strategies of the other players.
\end{definition}
\begin{theorem}
    Finite games with diff-equivalent payoffs have the same pure Nash equilibria. 
\end{theorem}
\begin{proof}
    The best reaction multi-function, for every player $i$, is the same when considering two diff-equivalent payoffs $u_i$ and $\tilde{u}_i$, no matter how different from each other the latter functions are. 
\end{proof}

\begin{definition}[\textit{Potential game}]
    A finite game with strategy set $X_i$ and payoffs $u_i:X\rightarrow\mathbb{R}$ is called a potential game, if it is diff-equivalent to a game with common payoffs. 

    That is, there exists a potential function $p:XX\rightarrow \mathbb{R}$ such that for each $i$, for every $x_{-i}\in X_{-i}$, and all $x_i^\prime,x_i\in X_i$ we have: 
    \[\Delta u_i(x_i^\prime,x_i,x_{-i})=\Delta p(x_i^\prime,x_i,x_{-i})\]
    here, $\Delta p(x_i^\prime,x_i,x_{-i})=p(x_i^\prime,x_{-i})-(x_i,x_{-i})$
\end{definition}
\begin{corollary}
    Every finite potential game has at least one pure Nash equilibrium. 
\end{corollary}
\begin{corollary}
    In a finite potential game every best response iteration reaches a pure Nash equilibrium in finitely many steps. 
\end{corollary}

\subsection{Potential search}
A potential $p:X\rightarrow\mathbb{R}$ is characterized by: 
\[\Delta u_i(x_i^\prime,x_i,x_{-i})=\Delta p(x_i^\prime,x_i,x_{-i})\]
\noindent Adding a constant to $p(\cdot)$ provides a new potential. 
Now, the potential $p(\cdot)$ is determined uniquely: 
\[p(x_1,\dots,x_n)=\sum_{i=1}^{n}\left[u_i(\bar{x}_1,\dots\bar{x}_{i-1},x_i,\dots,x_n)u_i(\bar{x}_1,\dots\bar{x}_{i-1},\bar{x}_i,\dots,x_n)\right]\]

\paragraph*{Existence}
If a game admits a potential the sum on the right hand side of the previous equation is independent of the particular order used. 
The converse is also true. 
However, checking that all these orders yield the same answer is impractical for more than two or three players. 
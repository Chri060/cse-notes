\section{Combinatorial game}

\begin{definition}[\textit{Impartial combinatorial game}]
    An impartial combinatorial game is characterized by the following properties:
    \begin{enumerate}
        \item Two players take turns alternately.
        \item The game has a finite number of distinct positions.
        \item The set of allowed moves from a position is the same for both players.
        \item The game ends when no legal moves remain.
        \item There is no element of chance.
        \item In the classical version, the player who makes the last move wins.
    \end{enumerate}
\end{definition}
\noindent To analyze such games, we categorize all possible positions into two types winning ($P$) and losing ($N$). 
Importantly, whether a position is a $P$- or $N$-position depends solely on the configuration of the game, not on which player is about to move.

The terminal position $(0, 0, \cdots, 0)$ is a $P$-position, as no moves are available.
Any move from a $P$-position must lead to an $N$-position. 
From any $N$-position, there exists at least one move that leads to a $P$-position. 

\subsection{Nim game}
The classical game of Nim is represented by a tuple $(n_1, \cdots, n_k)$, where each $n_i \in \mathbb{N}$ represents the size of pile $i$.
On each turn, a player selects a pile $n_i>0$ and reduces it to a smaller value $\hat{n}_i<n_i$. 
The objective is to force the opponent into the terminal position $(0, \dots, 0)$, thereby winning the game.

\begin{theorem}[Bouton]
    A position $(n_1, n_2, \dots, n_k)$ in Nim game is a $P$-position if and only if:
    \[n_1 \oplus n_2 \oplus \dots \oplus n_k = 0\]
\end{theorem}
\begin{proof}
    Consider the position $(0, 0, \dots , 0)$ with a Nim-sum of 0 ($P$-position).
    Suppose a player chooses to modify pile $n_i$ to $\hat{n}_i<n_i$.
    Then the new Nim-sum becomes:
    \[\hat{n}_i \oplus \hat{n}_1 \oplus n_2 \oplus \cdots \oplus n_k\]
    Then we would have:
    \[\hat{n}_i \oplus 0 \neq 0\]
    The resulting position is an $N$-position.
    
    Conversely, suppose the current position has Nim-sum $n_1 \oplus n_2 \oplus \cdots \oplus n_k \neq 0$.  
    Then there exists some pile $n_j$ such that changing it to $\hat{n}_j$ yields:
    \[\hat{n}_i \oplus \hat{n}_1 \oplus n_2 \oplus \cdots \oplus n_k = 0\]
    The resulting position is an $P$-position.
    Thus, from any position with non-zero Nim-sum, the player can always move to a position with zero Nim-sum.
\end{proof}
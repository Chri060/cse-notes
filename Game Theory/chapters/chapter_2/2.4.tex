\section{Strategy game}

In the context of extensive-form games, backward induction requires that a specific action be determined at every decision point for each player. 
Let $P_i$ denote the set of all decision nodes where Player $i$ must choose an action.
\begin{definition}[\textit{Pure strategy}]
    A pure strategy for Player $i$ is a function $s_i:P_i\rightarrow V$, where for each node $v \in P_i$, the function selects a child node $x$ of $v$. 
\end{definition}
\begin{definition}[\textit{Mixed strategy}]
    A mixed strategy for Player $i$ is a probability distribution over the set of all pure strategies available to Player $i$. 
\end{definition}
\noindent If Player $i$ has $n$ distinct pure strategies, the set of all mixed strategies corresponds to the standard simplex in $\mathbb{R}^n$
\[\sum_n=\left\{p=(p_1,\dots,p_n)|p_i\geq 0 \text{ and }\sum{p_i}=1\right\}\]
This simplex represents all convex combinations of the pure strategies (each point in $\Sigma_n$ encodes a possible mixed strategy).

\begin{theorem}[Von Neumann]
    In the game of chess, exactly one of the following outcomes holds:
    \begin{enumerate}
        \item White has a strategy that guarantees a win, regardless of Black's moves.
        \item Black has a strategy that guarantees a win, regardless of White's moves.
        \item Both players have strategies that ensure at least a draw, regardless of the opponent's play.
    \end{enumerate}
\end{theorem}
\noindent The outcome of the game depends on the structure of the payoff matrix or tree:
\begin{enumerate}
    \item A row of all wins for White implies a winning strategy for White.
    \item A column of all wins for Black implies a winning strategy for Black.
    \item If neither of these occurs, the best both players can secure is a draw.
\end{enumerate}
\section{Extensive game with perfect information}

An extensive game with perfect information is a formal model used to represent sequential interactions between players, where each player is fully aware of all previous actions taken.
\begin{definition}[\textit{Extensive Game with Perfect Information}]
    An extensive game with perfect information consists of the following components:
    \begin{enumerate}
        \item A finite set $N = \{1,\dots,n\}$ of players. 
        \item A game tree $(V,E,x_0)$.
        \item A partition of the non-leaf vertices into sets ${P_1, P_2, \dots , P_{n+1}}$.
        \item A probability distribution for each vertex in $P_{n+1}$, defined on the edges from that vertex to its children.
        \item A $n$-dimensional vector attached to each leaf (list of possible outcomes). 
    \end{enumerate}
\end{definition}
\noindent If $P_{n+1}$ is empty, the game contains no random events and is entirely deterministic.

\subsection{Solution}
To determine the rational outcomes of an extensive game, we apply the axioms of rationality recursively to the structure of the game tree.
\begin{definition}[\textit{Length}]
    The length of a game is the length of the longest path from the root to a leaf in the game tree.
\end{definition}
\noindent The fifth rationality assumption allows players to solve games of length 1. The fourth assumption ensures that if players can solve all games of length at most $i$, they can also solve games of length $i+1$. 
This recursive process leads to the concept of backward induction.
\begin{theorem}[First rationality theorem]
    The rational outcomes of a finite extensive game with perfect information are exactly those determined by the backward induction procedure.
\end{theorem}
Backward induction is applicable because each node $v$ in the tree defines a sub-game rooted at $v$, which can be solved independently using the same principles.

\subsection{Possible outcomes}
Extensive games may admit multiple possible outcomes depending on the strategies chosen by the players.
\begin{theorem}[Von Neumann]
    In the game of chess, exactly one of the following holds:
    \begin{enumerate}
        \item White has a winning strategy, regardless of Black's moves.
        \item Black has a winning strategy, regardless of White's moves.
        \item Both players can force at least a draw, no matter what the opponent does.
    \end{enumerate}
\end{theorem}
\begin{proof}
    Assume the game has a finite maximum length of $2k$ moves. 
    Let $a_i$ denote White's $i$-th move and $b_i$ denote Black's $i$-th move.
    
    The first possibility can be expressed formally as:
    \[\exists a_1 \mid \forall b_1 \exists a_2 \mid \forall b_2 \dots \exists a_k \mid \forall b_k \implies \text{white wins}\]
    
    If this statement is false, its negation must be true:
    \[\forall a_1 \exists b_1 \mid \forall a_2 \mid \exists b_2 \mid \dots \forall a_k \mid \exists b_k \implies \text{white does not win}\]
    Which implies that Black has a strategy to prevent White from winning, ensuring at least a draw. 
    The same reasoning applies symmetrically to the second possibility.
\end{proof}
\begin{corollary}
    In a finite, perfect-information, two-player game where outcomes are limited to a win for one player or a tie, exactly one of the following must be true:
    \begin{enumerate}
        \item Player 1 has a winning strategy, no matter what Player 2 does.
        \item Player 2 has a winning strategy, no matter what Player 1 does.
    \end{enumerate}
\end{corollary}
\noindent This leads to a classification of solution types in game theory:
\begin{itemize}
    \item \textit{Very weak solution}: the game has a rational outcome, but it is inaccessible.
    \item \textit{Weak solution}: the outcome is known to exist, but the procedure to obtain it is unknown.
    \item \textit{Solution}: there exists an explicit, constructive algorithm to determine the outcome.
\end{itemize}
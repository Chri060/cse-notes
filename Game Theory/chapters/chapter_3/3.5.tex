\section{Symmetric games}

\begin{definition}[\textit{Antisymmetric}]
    A $n \times n$ matrix $P$ with elements $(p_{ij})$ is said to be antisymmetric provided that $p_{ij}=-p_{ji}$ for all $i,j=1,\cdots,n$. 
\end{definition}
\begin{definition}[\textit{Fair game}]
    A finite zero sum game is fair if the associated matrix is antisymmetric.
\end{definition}
In fair games there is no favorite plater: in fact, their role can be exchanged. 
\begin{proposition}
    If $P=(p_{ij})$ is antisymmetric the conservative value $v=0$ and $\bar{x}$ is an optimal strategy for Player 1 if and only if it is optimal for Player 2.
\end{proposition}
\begin{proof}
    Recall that $P^T=-P$ if $P$ is an antisymmetric matrix. 
    Then, since: 
    \[(Px.x)=(x,P^Tx)=-(x,Px)=-(Px,x)\]
    one has $f(x,x)=0$ for all $x$. 
    This implies $v_1\leq 0,v_2\geq 0$
    If $\bar{x}$ is optimal for the first player, then: 
    \[(\bar{x},Py)\geq 0 \qquad \forall y \in \Sigma_n\]
    so that $(P^T\bar{x},y)\geq 0$, which by the fact that $P$ is antisymmetric becomes: 
    \[(P\bar{x},y)\leq 0 \qquad \forall y \in \Sigma_n\]
    Therefore $\bar{x}$ is optimal also for the second player, and conversely.
\end{proof}

\subsection{Optimal strategies in fair games}
In order to find optimal strategies in fair games, we need to solve the system of inequalities: 
\[\begin{cases}
    x_1p_{11}+\cdots+x_np_{n1}\geq 0 \\
    \cdots \\
    x_1p_{1j}+\cdots+x_np_{nj}\geq 0 \\
    \cdots \\
    x_1p_{1n}+\cdots+x_np_{nn}\geq 0
\end{cases}\]
With extra conditions $x_i\geq 0$ and $\sum_{i=1}^nx_i=1$. 
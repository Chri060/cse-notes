\section{Linear programming}

\begin{definition}[\textit{Duality first form}]
    The following two linear programs are said to be in duality: 
    \[\begin{cases}
        \min(\mathbf{c},\mathbf{x}) \\
        \mathbf{Ax} \geq \mathbf{b} \\
        \mathbf{x} \geq 0
    \end{cases} \qquad \begin{cases}
        \max(\mathbf{b},\mathbf{y}) \\
        \mathbf{A}^T\mathbf{y} \leq \mathbf{c} \\
        \mathbf{y} \geq 0
    \end{cases}\]
    Here the matrix $A\in\mathbb{R^{m\times n}}$ and the vectors $\mathbf{c},\mathbf{x}\in\mathbb{R}^n$, $\mathbf{b},\mathbf{y}\in\mathbb{R}^m$.
\end{definition}
\noindent The minimization problem is called primal problem and the maximization is called dual problem. 

\begin{definition}[\textit{Duality second form}]
    The following two linear programs are said to be in duality: 
    \[\begin{cases}
        \min(\mathbf{c},\mathbf{x}) \\
        \mathbf{Ax} \geq \mathbf{b} 
    \end{cases} \qquad \begin{cases}
        \max(\mathbf{b},\mathbf{y}) \\
        \mathbf{A}^T\mathbf{y} \leq \mathbf{c} \\
        \mathbf{y} \geq 0
    \end{cases}\]
\end{definition}
\noindent The minimization problem in the second form can be written in an equivalent way in the first form; dualizing this shows that the dual is equivalent to the dual of the second form, in the sense that the solution is the same. 

Given two problems in duality, there are three options: 
\begin{enumerate}
    \item Both can be feasible.
    \item Only one can be feasible. 
    \item They can both be infeasible.
\end{enumerate}

\subsection{Duality theorems}
\begin{theorem}[\textit{Weak duality}]
    Let $v$ be the value of the primal minimization problem and $V$ the value of the dual maximization problem. 
    Then: 
    \[v\geq V\]
\end{theorem}

\begin{theorem}[\textit{Strong duality}]
    If the primal and the dual problems are feasible, then both problems have optimal solutions $\bar{\mathbf{x}}$ and $\bar{\mathbf{y}}$ and the optimal values coincide, that is: 
    \[v=(\mathbf{c},\bar{\mathbf{x}})=(\mathbf{b},\bar{\mathbf{y}})=V\]

    If the primal is feasible and the dual is infeasible, then $v=V=-\infty$. 

    If the primal is infeasible and the dual is feasible, then $v=V=+\infty$. 

    If both the primal and the dual are infeasible, then $v=+\infty>V=-\infty$.
\end{theorem}
\begin{corollary}
    If one problem is feasible and has an optimal solution, then also the dual problem is feasible and has solution. 
    Moreover, there is no duality gap. 
\end{corollary}

\subsection{Complementarity}
\begin{theorem}[Complementarity condition first form]
    Let $\bar{\mathbf{x}}$ and $\bar{\mathbf{y}}$ be primal and dual feasible. 
    Then $\bar{\mathbf{x}}$ and $\bar{\mathbf{y}}$ are simultaneously optimal if and only if: 
    \[\begin{cases}
        \forall i \bar{x}_i>0 \implies\sum_{k=1}^ma_{ik}\bar{y}_k=c_i \\
        \forall i \bar{y}_i>0 \implies\sum_{k=1}^na_{kj}\bar{x}_k=b_i 
    \end{cases}\]
\end{theorem}
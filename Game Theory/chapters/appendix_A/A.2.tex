\section{Group}

\begin{definition}[\textit{Group}]
    A group is defined as a nonempty set $A$ equipped with a binary operation $\cdot$ such that the following conditions hold:
    \begin{enumerate}
        \item \textit{Closure}: for any elements $a, b \in A$, the result of the operation $a \cdot b$ is also an element of $A$.
        \item \textit{Associativity}: the operation $\cdot$ is associative, meaning that for all $a,b,c\in A$, it holds that $(a \cdot b) \cdot c = a \cdot (b \cdot c)$.
        \item \textit{Identity element}: there exists a unique element $e$ known as the identity element, such that for every $a\in A$, the following holds: $a \cdot e = e \cdot a = a$.
        \item \textit{Inverse element}: for every element $a \in A$, there exists a unique element $b \in A$ (denoted as $a^{-1}$) such that $a \cdot b = b \cdot a = e$. 
            This element $b$ is called the inverse of $a$.
    \end{enumerate}
\end{definition}

\begin{definition}[\textit{Abelian group}]
    A group $A$ is termed an Abelian group (or commutative group) if the operation is commutative; that is, for all $a, b \in A$, the equation $\cdot$ $a \cdot b = b \cdot a$ holds true.
\end{definition}

\begin{proposition}
    Let $(A, \cdot)$ be a group, then the cancellation law holds:
    \[a \cdot b = a \cdot c \implies b = c\]
\end{proposition}

\begin{proposition}
    The set of natural numbers with the operation $\oplus$ forms an Abelian group.
\end{proposition}
\section{Convexity}

\begin{definition}[\textit{Convex set}]
    A set $C \subset \mathbb{R}^n$ is called convex if for any points $x, y \in C$ and for any $\lambda \in [0, 1]$ the point $\lambda x + (1 - \lambda)y \in C$.
\end{definition}
This means that the line segment connecting any two points in $C$ is entirely contained within $C$.
The properties of a convex set are:
\begin{itemize}
    \item The intersection of an arbitrary family of convex sets is convex.
    \item A closed convex set with a nonempty interior coincides with the closure of its internal points.
\end{itemize}

\begin{definition}[\textit{Convex combination}]
    A convex combination of elements $x_1, \dots, x_n$ is any vector $x$ of the form:
    \[x = \lambda_1x_1 + \dots + \lambda_nx_n\]
    where $\lambda_1 \geq 0, \dots, \lambda_n \geq 0$ and $\sum_{i=1}^{n} \lambda_i = 1$.
\end{definition}
\begin{proposition}
    A set $C$ is convex if and only if for every $\lambda_1 \geq 0, \dots, \lambda_n \geq 0$ such that $\sum_{i=1}^{n}\lambda_i = 1$, and for every $c_1, \dots, c_n \in C$, we have 
    \[\sum^n_{i=1} \lambda_i c_i \in C\]
\end{proposition}
\begin{definition}[\textit{Convex hull}]
    The convex hull of a set $C$, denoted by $\text{co }C$, is the smallest convex set containing $C$. 
    It is defined as:
    \[\text{co }C =\bigcap_{A\in\mathcal{C}}A\]
    where $\mathcal{C} = \left\{A | C \subset A \text{ and } A \text{ is convex}\right\}$. 
\end{definition}
\begin{proposition}
    The convex hull of a set $C$ can be expressed as:
    \[\text{co }C=\left\{\sum_{i=1}^{n}\lambda_ic_i|\lambda_i\geq 0,\sum_{i=1}^{n}\lambda_i=1,c_i\in C\quad\forall i, n\in \mathbb{N}\right\}\]
\end{proposition}
The convex hull of a set $C$ consists of all convex combinations of points in $C$.
When $C$ is a finite set, the convex hull is called a polytope.
\begin{theorem}
    Given a closed convex set $C$ and a point $x$ outside $C$, there exists a unique point $p \in C$ such that for all $c \in C$:
    \[\left\lVert p - x\right\rVert\leq\left\lVert c-x\right\rVert\]
\end{theorem}
The projection $p$ is the point in $C$ closest to $x$ and satisfies the following:
\begin{enumerate}
    \item $p \in C$.
    \item $(x - p, c - p) \leq 0 \text{ for all } c \in C$.
\end{enumerate}
\begin{theorem}
    Let $C$ be a convex subset of $\mathbb{R}^l$, and assume $\bar{x} \in \text{cl } C^c$ (the closure of the complement of $C$).
    Then, there exists a nonzero $x^\ast \in \mathbb{R}^l$ such that for all $c \in C$:
    \[(x^\ast, c) \geq (x^\ast, \bar{x})\]
\end{theorem}
This result provides a criterion to distinguish points outside of $C$ from those inside.
\begin{corollary}
    For any closed convex set $C$ in Euclidean space, and any point $x$ on the boundary of $C$, there exists a hyperplane that contains $x$ and leaves all points in $C$ on one side of the hyperplane.
\end{corollary}
This hyperplane is called a supporting hyperplane for $C$ at $x$.
\begin{corollary}
    Any closed convex set $C$ in Euclidean space can be represented as the intersection of all half-spaces that contain it.
\end{corollary}
\begin{theorem}
    Let $A$ and $C$ be closed convex subsets of $\mathbb{R}^l$, with $\text{int } A \neq\varnothing$ and $\text{int }A \cap C = \varnothing$. 
    Then, there exists a nonzero vector $x^\ast$ and a scalar $b \in \mathbb{R}$ such that for all $a \in A$ and $c \in C$:
    \[(x^\ast, a) \geq b \geq (x^\ast, c)\]
\end{theorem}
This provides a criterion to determine whether a point lies in $A$ or $C$.
The hyperplane $H = \{x : (x^\ast, x) = b\}$  is the separating hyperplane, with $A$ and $C$ located in different half-spaces defined by $H$.

\begin{definition}[\textit{Quasi concavity}]
    Quasi concavity for a real valued function $h$ means that the sets: 
    \[h_a=\left\{z\mid h(z)\geq a\right\}\]
    are convex for all $a\in\mathbb{R}$. 
\end{definition}
\section{Players}

In game theory, players are assumed to be both selfish and rational. 
These assumptions form the foundation for analyzing strategic interactions.

\subsection{Selfish player}
A selfish player is one who is solely concerned with the outcomes that affect their own preferences. 
That is, they evaluate the game purely based on personal benefit, without regard for the preferences or well-being of others.

\subsection{Rational player}
\begin{definition}[\textit{Preference relation}]
    A preference relation on a set $X$ is a binary relation $\succeq$ satisfying the following properties for all $x, y, z \in X$:
    \begin{itemize}
        \item \textit{Reflexivity}: $x \succeq x$.
        \item \textit{Completeness}: either $x \succeq y$ or $y \succeq x$.
        \item \textit{Transitivity}: if $x \succeq y$ and $y \succeq z$, then $x \succeq z$.
    \end{itemize}
\end{definition}

\begin{definition}[\textit{Utility function}]
    Given a preference relation $\succeq$ over a set $X$, a utility function is a function $u:X\rightarrow\mathbb{R}$ that represents $\succeq$ such that:
    \[u(x)\geq u(y)\Leftrightarrow x \succeq y\]
\end{definition}
\noindent While a utility function may not exist for all possible preference structures, it is guaranteed to exist in many common cases (particularly when the outcome set $X$ is finite).
Moreover, if a utility function exists, there are infinitely many such functions, each differing by a strictly increasing transformation of the original.

Each player $i$ is associated with a choice set $X_i$ representing all available actions.
The global outcome space consists of the joint actions of all players.
Utility functions are then defined over this combined set.

\subsubsection{Rationality assumptions}
Rational behavior in game theory is guided by the following assumptions:
\begin{enumerate}
    \item Players can define a consistent preference relation over the possible outcomes.
    \item Players can represent their preferences using a utility function when needed.
    \item Players apply probability theory consistently when dealing with uncertainty.
    \item Players can comprehend the consequences of all possible actions.
    \item Players apply principles of decision theory whenever applicable.
\end{enumerate}
\noindent Given a set of alternatives $X$ and a utility function $u$, a rational player chooses an alternative $\bar{x} \in X$ such that:
\[u(\bar{x}) \geq u(x)\qquad\forall x \in X\]
One direct implication of these assumptions is the elimination of strictly dominated strategies: a rational player will never choose an action that yields a strictly worse outcome than another, regardless of the other players' choices.

\subsection{Actions}
The available actions and corresponding outcomes can be represented as pairs of utilities (one for each player). 
In two-player games, this data is often organized in a bi-matrix, where Player 1 selects a row and Player 2 selects a column.
Each cell in the matrix represents a pair of payoffs corresponding to the selected strategies of both players.
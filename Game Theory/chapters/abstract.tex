\begin{abstract}
    The theory begins by examining the main assumptions that distinguish decision theory from interactive decision theory.
    While decision theory focuses on individual decision-making in isolation, interactive decision theory explores how multiple decision-makers interact, considering each other's potential actions.

    In the context of non-cooperative games, the discussion extends to games represented in extensive form, where players make decisions at various points, and games with perfect information, where all players are fully informed of prior moves. 
    The technique of backward induction is key in solving such games. 
    Additionally, combinatorial games are explored, emphasizing their strategic complexity.

    Zero-sum games are analyzed in terms of conservative values, where each player seeks to minimize potential losses.
    The concept of equilibrium in pure strategies is introduced, and this is extended to mixed strategies in finite games, invoking von Neumann's theorem. 
    Finding optimal strategies and determining the value of finite games is achieved through the use of linear programming techniques.

    The Nash non-cooperative model plays a central role in understanding strategic interactions. 
    Nash equilibrium is discussed, focusing on the existence of equilibria in both pure and mixed strategies within finite games. 
    Examples of potential games are provided, along with methods for identifying potential functions. 
    Notable examples include congestion games, routing games, network games, and location games. 
    Concepts such as the price of stability, price of anarchy, and correlated equilibria are explored to analyze the efficiency and stability of these systems.

    Finally, the discussion shifts to cooperative games, defining key concepts such as the core, nucleolus, Shapley value, and power indices. Examples of cooperative scenarios illustrate how these concepts help to determine fair outcomes and power distribution among players.
\end{abstract}
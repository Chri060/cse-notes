\section{Constructive noise-free 4SID algorithm}

The 4SID algorithm is utilized to derive an estimated state-space model based on a truncated impulse response acquired from system measurements.

Beginning with a finite dataset of impulse response values, the algorithm operates under the assumption of a flawless, noise-free measurement of the impulse response. 
While the initial discussion centers on the algorithm's application to impulse response experiments for clarity, it's important to note its versatility to accommodate various types of input excitations.

The algorithm operates as follows:
\begin{enumerate}
    \item Construct the Hankel matrix in increasing order and assess its rank. 
        Cease increasing the order when encountering the first matrix lacking full rank.
        At this point,  the rank value denotes the system's order, denoted as $n$.
    \item Select $\mathbf{H}_{n+1}$ (the first non-full rank matrix with a rank of $n$). 
        Decompose $\mathbf{H}_{n+1}$  into two rectangular matrices of dimensions $(n+1) \times n$ and $n \times (n+1)$.
        The former matrix becomes $\mathbf{O}_{n+1}$, while the latter becomes $\mathbf{R}_{n+1}$, signifying the extended $(n+1)$ observability and reachability matrices, respectively.
    \item Derive $\hat{\mathbf{H}}$, $\hat{\mathbf{G}}$, and $\hat{\mathbf{F}}$ from $\mathbf{O}_{n+1}$ and $\mathbf{R}_{n+1}$: 
        \[\mathbf{O}_{n+1}=\begin{bmatrix} \mathbf{H} \\ \mathbf{HF} \\ \mathbf{HF}^2 \\ \vdots \\ \mathbf{HF}^{n-1} \\ \mathbf{HF}^{n} \end{bmatrix} \qquad \mathbf{R}_{n+1}=\begin{bmatrix} \mathbf{G} \\ \mathbf{FG} \\ \mathbf{F}^2\mathbf{G} \\ \vdots \\ \mathbf{F}^{n-1}\mathbf{G} \\ \mathbf{F}^{n}\mathbf{G} \end{bmatrix}\]
        Form this, $\mathbf{H}$ and $\mathbf{G}$ are directly extracted as the first row of $\mathbf{O}_{n+1}$ and the first column of $\mathbf{R}_{n+1}$, respectively. 
        To estimate $\hat{\mathbf{F}}$, focus on $\mathbf{O}_{n+1}$ (the same can be applied to $\mathbf{R}_{n+1}$). 
        Formulate two submatrices, $\mathbf{O}_1$ and $\mathbf{O}_2$, by excluding the last and first rows of $\mathbf{O}$, respectively:
        \[\mathbf{O}_{1}=\begin{bmatrix} \mathbf{H} \\ \mathbf{HF} \\ \mathbf{HF}^2 \\ \vdots \\ \mathbf{HF}^{n-1} \end{bmatrix} \qquad \mathbf{O}_{2}=\begin{bmatrix} \mathbf{HF} \\ \mathbf{HF}^2 \\ \vdots \\ \mathbf{HF}^{n-1} \\ \mathbf{HF}^{n} \end{bmatrix}\]
        Notably, $\mathbf{O}_1$ and $\mathbf{O}_2$ are square matrices owing to the shift invariance property.
        Consequently, $\mathbf{O}_2=\mathbf{O}_1\cdot \mathbf{F}$.
        As $\mathbf{O}_1$ is square and invertible (due to the full rank of $\mathbf{H}_n$ with a row removed), $\hat{\mathbf{F}}$ is determined as:
        \[\hat{\mathbf{F}}=(\mathbf{O}_1)^{-1}(\mathbf{O}_2)\]
\end{enumerate}
It's important to note that we've derived the state space matrices by commencing with a noise-free impulse response experiment.
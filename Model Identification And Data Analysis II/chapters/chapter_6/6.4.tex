\section{Generalized Minimum Variance Control}

The primary limitations of Minimum Variance Control are:
\begin{itemize}
    \item It can only be applied to minimum phase systems.
    \item There is no moderation of the effort and intensity of the control action $u(t)$. 
    \item t lacks the capability to design a specific behavior or relationship between $y^{0}(t)$ and $y(t)$. 
\end{itemize}
To address these limitations, Generalized Minimum Variance Control extends Minimum Variance Control.

\subsection{Performance indexes comparison}
For Minimum Variance Control, the performance index is:
\[J=\mathbb{E}\left[\left(y(t)-y^{0}(t)\right)^2\right]\]
However, for Generalized Minimum Variance Control, the performance index is more comprehensive:
\[J=\mathbb{E}\left[\left(y(t)-P(z)y^{0}(t)+Q(z)u(t)\right)^2\right]\]
Here, the additional term $Q(z)u(t)$ penalizes the usage of control effort, and $P(z)$ establishes a reference model between $y^{0}(t)$ and $y(t)$. 
It's noteworthy that Generalized Minimum Variance Control reduces to simple Minimum Variance Control when $P(z)=1$ and $Q(z)=0$. 

\paragraph*{Reference model matrix}
The matrix $P(z)$ serves as the reference model for the closed-loop system's desired behavior. 
In general, we envision the ideal scenario as:
It's important to note that in some systems, the optimal behavior isn't always achieved when $P(z)=1$. 
In such cases, a low-pass filter $P(z)$ might be employed to establish a new target for the closed loop.
\section{Introduction}

The process of designing a control system typically unfolds in three sequential steps:
\begin{enumerate}
    \item \textit{System modeling}: this involves creating a mathematical representation of the system under consideration.
    \item \textit{Estimation of unmeasurable variables} (software sensing): certain variables within the system may not be directly measurable and need to be estimated through software-based techniques.
    \item \textit{Control algorithm design}: here, the actual control algorithm that governs the system's behavior is developed.
\end{enumerate}
In the realm of minimum variance control, the mathematical techniques employed bear strong resemblance to those utilized in system identification and software sensing. 
Minimum variance control, essentially an optimization endeavor centered around an ARMAX system, shares mathematical commonalities with these areas.

Indeed, Minimum Variance Control can be viewed as a versatile tool for optimizing stochastic feedback systems, offering a broad spectrum of applications in the realm of control engineering.

\paragraph*{Problem}
The problem of Minimum Variance Control is set around a standard ARMAX representation of the system:
\[y(t)=\dfrac{B(z)}{A(z)}u(t-k)+\dfrac{C(z)}{A(z)}e(t)\]
Here: 
\[\begin{cases}
    A(z)=1+a_1z^{-1}+\dots+a_nz^{-n} \\
    B(z)=b_0+b_1z^{-1}+\dots+b_pz^{-p} \\
    C(z)=1+c_1z^{-1}+\dots+c_mz^{-m} \\
\end{cases}\]
We assume that the transfer function $\frac{C(z)}{A(z)}$ is in canonical form, and $k$ is the pure delay from $u(t)$ to $y(t)$ ($k \geq 1$). 
Also, we assume $b_0\neq 0$ to prevent changes in the actual delay. 
Additionally, we assume that $\frac{B(z)}{A(z)}$ is minimum phase. 

Controlling a non-minimum phase system is challenging because immediate reactions may lead to incorrect decisions. 
Designing control systems for non-minimum phase systems requires specific design tools. 
Minimum variance control cannot be used for non-minimum phase systems. 
Instead, there exists a Generalized Minimum Variance Control that can be employed, even for minimum phase systems.

The objective of the control system is to track a desired output reference signal:
\begin{figure}[H]
    \centering
    \includegraphics[width=0.75\linewidth]{images/cs.png}
    \caption{Control system}
\end{figure}
Ideally, $y(t)$ should exactly follow $y^{0}(t)$. 

\paragraph*{Formal description}
In a formal manner, we can define the problem as an optimal control problem: find the signal $u(t)$ such that the performance index $J$ is minimized: 
\[\min{J}=\mathbb{E}\left[\left(y(t)-y^{0}(t)\right)^2\right]\]
That is the variance of the tracking error between $y(t)$ and $y^{0}(t)$. 

We attempt to solve this problem with the following technical assumptions:
\begin{enumerate}
    \item $y^{0}(t)$ and $e(t)$ are uncorrelated. 
    \item $y^{0}(t)$ is unpredictable (worst-case assumption). 
        This implies that we have no preview of the future desired output behavior. 
        The best prediction we can make at time $t$ is:
        \[\hat{y}^\circ(t+k\mid t)=y^{0}(t)\]
\end{enumerate}
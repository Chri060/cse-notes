\section{Introduction}

The history of mobile devices began in 1973 when Martin Cooper at Motorola made the first-ever mobile phone call using a prototype, marking a pivotal moment in telecommunications. 
Over the decades, mobile devices have evolved from simple communication tools to highly advanced, multifunctional systems equipped with a wide array of sensors. 
Modern smartphones now integrate technologies such as accelerometers, gyroscopes, digital compasses, GPS, barometers, ambient light sensors, and proximity sensors, enabling a range of applications from navigation to augmented reality.

Alongside hardware advancements, mobile programming has grown to support a variety of platforms, each with its own preferred development languages. 
Key programming languages used in mobile app development include: 
\begin{itemize} 
    \item \textit{Objective-C and Swift}: primarily for iOS development. 
    \item \textit{Java and Kotlin}: widely used for Android development. 
    \item \textit{C\#}: common for cross-platform development, particularly with frameworks like Xamarin.
    \item \textit{HTML5 and JavaScript}: popular for web-based and cross-platform apps, especially with frameworks like React Native.
    \item \textit{Python and other languages}: increasingly used for cross-platform solutions through frameworks such as Kivy or BeeWare.
\end{itemize}
This diverse ecosystem of languages supports the ever-expanding capabilities of mobile devices and the broad range of applications they power.
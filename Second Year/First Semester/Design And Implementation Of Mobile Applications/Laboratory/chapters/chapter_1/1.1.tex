\section{Dart basics}

The \texttt{main.dart} file contains the following code: 
\begin{lstlisting}[style=Dart]
import 'package:flutter/material.dart';

void main() {
    runApp(const MyApp());
}
\end{lstlisting}
In this case we have that when the application is started the main will call an immutable class \texttt{MyApp} (that becomes a Widget) that shows something on the screen.

An example of called class could be: 
\begin{lstlisting}[style=Dart]
class MyApp extends StatelessWidget {
    const MyApp({super.key}); 

    @override
    Widget build(BuildContext context) {
        return MaterialApp();
    }
}
\end{lstlisting}
In this case we have a constructor with the same name of the class, and an override method taht returns a Widget that becomes the root of our application. 

The \texttt{MaterialApp} needs a title and a home as follows: 
\begin{lstlisting}[style=Dart]
MaterialApp (
    title: "Hello World!",
    home: Scaffold(
        appBar: AppBar(
            title: const Text("Hello World!"),
            style: TextStyle(color: Colors.red),
        )
    ),
)
\end{lstlisting}
In this case we use a \texttt{Scaffold}, that is the biggest element on the screen, and we set the top bar with the string \texttt{Hello World!} coloured in red. 
\texttt{Scaffold} also has an argument called \texttt{body} to inser directly a widget inside it. 

To center an element we may use \texttt{Center}: 
\begin{lstlisting}[style=Dart]
Center( 
    child: HelloWorld(),
)
\end{lstlisting}

The constructor may have null elemebts, in this case we add a question mark after the variable type.
An example of constructor: 
\begin{lstlisting}[style=Dart]
class HelloWorlds extends StatelessWidget {
    const HelloWorld({Key? key}) : siper(key: key);
}

class HelloWorldPlus extends StatelessWidget {
    final int number; 
    final Color color; 

    const HelloWorldPlus(this.number, this.color);
}
\end{lstlisting}
To add a variable in a string we can simply use \$ befor the variable name: 
\begin{lstlisting}[style=Dart]
@override
Widget build(BuildContext context) {
    return Text("Hello World! $number",)
}                               
\end{lstlisting}
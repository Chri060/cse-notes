\section{Introduction}

\begin{definition}[\textit{Zero sum game}]
    A two player zero sum game in strategic form is the triplet $(X, Y , f : X \times Y \rightarrow \mathbb{R})$. 
    Here: 
    \begin{itemize}
        \item $X$ is the strategy space of Player 1.
        \item $Y$ the strategy space of Player 2.
        \item $f (x, y)$ is what Player 1 gets from Player 2, when they play $x, y$ respectively.
    \end{itemize}
\end{definition}
Given that $f$ is the utility function of Player 1, by definition of zero sum games the utility function $g$ of Player 2 must be: 
\[g=-f\]

In the finite case $X = \{1, 2, \dots, n\}$, $Y = \{1, 2, \dots, m\}$ the game is described by a payoff matrix $P$, wherein Player 1 selects row $i$ while Player 2 selects column $j$:
\[\begin{pmatrix} p_{11} & \cdots & p_{1m} \\ \cdots & p_{ij} & \cdots \\ p_{n1} & \cdots & p_{nm} \end{pmatrix}\]
Here, $p_{ij}$ is the payment of Player 2 to Player 1 when they play $i, j$ respectively.
In order to choose the optimal strategy, each player can reason as follows:
\begin{itemize}
    \item Player 1 can guarantee herself to get at least $v_1 = \max_i \min_j p_{ij}$. 
    \item Player 2 can guarantee himself to pay at most $v_2 = \min_j max_i p_{ij}$. 
\end{itemize}
$v_1$ and $v_2$ are said to be the conservative values of Player 1 and Player 2, respectively.
\begin{example}
    Consider the game: 
    \[\begin{pmatrix} 4 & 3 & 1 \\ 7 & 5 & 8 \\ 8 & 2 & 0 \end{pmatrix}\]
    For the player 1 we pick the minimum for each row, that is : $\begin{pmatrix} 1 & 5 & 0 \end{pmatrix}$ and then we choose the maximum between them.
    Thus, the conservative value for the player 1 is $v_1=5$. 

    For the player 2 we pick the maximum for each column, that is : $\begin{pmatrix} 8 & 5 & 8 \end{pmatrix}$ and then we choose the mimimum between them.
    Thus, the conservative value for the player 2 is $v_2=5$. 

    Accordingly, the rational outcome is $5$ and the rational behavior is $( \bar{i} = 2, \bar{j} = 2 )$.
\end{example}

\subsection{Rationality in zero sum games}
Let us suppose the following: 
\begin{itemize}
    \item $v_1 = v_2 := v$,
    \item $\bar{i}$ the row such that $p_{\bar{i}\bar{j}} \geq v_1 = v$ for all $j$
    \item $\bar{j}$ the column such that $p_{\bar{i}\bar{j}} \leq v_2 = v$ for all $i$
\end{itemize}

Then $p_{\bar{i}\bar{j}} = v$ and $p_{\bar{i}\bar{j}} = v$ is the rational outcome of the game

$\bar{i}$ is an optimal strategy for Pl1, because she cannot get more than $v_2$, since $v_2$ is the conservative value of the second player
$\bar{j}$ is an optimal strategy for Pl2, because he cannot pay less than $v_1$, since $v_1$ is the conservative value of the first player
$\bar{i}$ maximizes the function $\alpha(i) = minj p_{ij}$ ; $\bar{j}$ minimizes the function $\beta(j) = \max_i p_{ij}$.

\subsection{Extension to arbitrary games}
Let the triplet $(X, Y , f : X \times Y \rightarrow \mathbb{R})$ represent a game between two players, wherein their respective strategy spaces $X$ and $Y$ may not be finite sets.
For the sake of rational behaviour, the players can guarantee to themselves the following outcomes:
\begin{itemize}
    \item Player 1: $v_1 = \sup_x\inf_y f (x, y)$
    \item Player 2: $v_2 = \inf_y \sup_xf (x, y)$
\end{itemize}

The outcomes $v_1$, $v_2$ are the conservative values of the players.
If $v_1 = v_2$, we set $v = v_1 = v_2$ and we say that the game has value $v$.

\paragraph*{Optimality}
Let $X$ and $Y$ be arbitrary sets. 
Suppose:
\begin{enumerate}
    \item $v_1 = v_2 := v$
    \item there exists strategy $\bar{x}$ such that $f (\bar{x}, y) \geq v$ for all $y \in Y$
    \item there exists strategy $\bar{y}$ such that $f (x, \bar{y}) \leq v$ for all $x \in X$
\end{enumerate}

(the last two conditions are needed if the sets are infinite and not compact). 
Then:
\begin{itemize}
    \item $v$ is the rational outcome of the game.
    \item $\bar{x}$ is an optimal strategy for Pl1.
    \item $\bar{y}$ is an optimal strategy for Pl2.
\end{itemize} 
Observe
$\bar{x}$ is optimal for Pl1 since it maximizes the function $\alpha(x) = \inf_y f (x, y)$
$\bar{y}$ is optimal for Pl2 since it minimizes the function $\beta(y) = \sup_xf (x, y)$
where $\alpha(x)$ is the value of the optimal choice of Pl2 if he knows that Pl1 plays $x$,
and $\beta(y)$ is the value of the optimal choice of Pl1 if she knows that Pl2 plays $y$.

\begin{proposition}
    Let $X, Y$ be nonempty sets and let $f : X \times Y \rightarrow \mathbb{R}$ be an arbitrary real valued function. 
    Then: 
    \[v_1 = \sup_x\inf_yf (x, y) \leq \inf_y\sup_xf (x, y) = v_2\]
\end{proposition}
\begin{proof}
    By definition, for all $x, y$:
    \[\inf_yf(x, y) \leq f (x, y) \leq \sup_xf (x, y)\]
    Thus:
    \[\alpha(x) = \inf_yf (x, y) \leq \sup_xf (x, y) = \beta(y)\]
    Since for all $x \in X$ and $y \in Y$ it holds that
    \[\alpha(x) \leq \beta(y)\]
    it follows that 
    \[\sup_x\alpha(x) \leq \inf_y\beta(y)\]
\end{proof}
As a consequence, in every game $v_1 \leq v_2$.

\begin{example}
    Consider the rock, scissors, and paper game: 
    \[\begin{pmatrix} 0 & 1 & -1 \\ -1 & 0 & 1 \\ 1 & -1 & 0 \end{pmatrix}\]
    The conservative values are not the same: in fact, $v_1 = -1$ and $v_2 = 1$. 
\end{example}
There is no winning strategy since each player always plays randomly. 
But if the game is repeated many times, the rational solution for both players is to play each option one-third of the times, so that in the long run their expected utility is zero.
By extending the game with mixed strategies, both conservative values become 0. 

\paragraph*{Case $v_1<v_2$}
Consider the case of mixed strategies with a gmae with an $n\times m$ utility matrix $P$. 
In this case the strategy spaces are defined as: 
\[\sum_k=\left\{x=(x_1,\dots,x_k)|x_i\geq 0 \text{ and }\sum_{i=1}^{k}x_i=1\right\}\]
with $k = n$ for Pl1 and $k = m$ for Pl2. 
The utility function is: 
\[f(x,y)=\sum_{i=1,\dots,n,j=i,\dots,m}x_iy_jp_{ij}=(x,Py)\]
Here $p_{ij}$ is an element of $P$ corresponding to the utility of Pl1 when she plays
row $i$ and Pl2 plays column $j$ (of course the utility of Pl2 is just the opposite). 
Thus, the mixed extension of the initial game is 
\[\left(\sum_n, \sum_m, f (x, y) = (x, Py)\right)\]



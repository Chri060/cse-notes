\section{Introduction}

\begin{definition}[\textit{Zero sum game}]
    A two-player zero-sum game in strategic form can be described as a triplet $(X, Y , f : X \times Y \rightarrow \mathbb{R})$, where:
\end{definition}
\begin{itemize}
    \item $X$ is the strategy space of Player 1.
    \item $Y$ is the strategy space of Player 2.
    \item $f(x, y)$ represents the payoff Player 1 receives from Player 2 when they play strategies $x$ and $y$, respectively.
\end{itemize}
Since this is a zero-sum game, Player 2's utility function $g$ is defined as the negative of Player 1's utility function:
\[g=-f\]
In the case where the strategy spaces are finite, i.e., $X = \{1, 2, \dots, n\}$, $Y = \{1, 2, \dots, m\}$, the game can be represented by a payoff matrix $P$. 
In this matrix, Player 1 chooses a row $i$, and Player 2 chooses a column $j$:
\[\begin{pmatrix} p_{11} & \cdots & p_{1m} \\ \cdots & p_{ij} & \cdots \\ p_{n1} & \cdots & p_{nm} \end{pmatrix}\]
Here, $p_{ij}$ denotes the payment Player 2 makes to Player 1 when they select strategies $i$ and $j$, respectively.

To determine the optimal strategy, both players can employ conservative reasoning: 
\begin{itemize} 
    \item Player 1 can ensure a minimum payoff of $v_1 = \max_i \min_j p_{ij}$. 
    \item Player 2 can limit their losses to at most $v_2 = \min_j \max_i p_{ij}$. 
\end{itemize}
These values, $v_1$ and $v_2$, are known as the conservative values for Player 1 and Player 2, respectively.
\begin{example}
    Consider the following game with the payoff matrix:
    \[\begin{pmatrix} 4 & 3 & 1 \\ 7 & 5 & 8 \\ 8 & 2 & 0 \end{pmatrix}\]
    For Player 1, the minimum values for each row are $\begin{pmatrix} 1 & 5 & 0 \end{pmatrix}$ and then we choose the maximum between them.
    The maximum of these is $v_1 = 5$, so the conservative value for Player 1 is 5.

    For Player 2, the maximum values for each column are $\begin{pmatrix} 8 & 5 & 8 \end{pmatrix}$ and then we choose the mimimum between them.
    The minimum of these is $v_2 = 5$, so the conservative value for Player 2 is also 5.

    Thus, the rational outcome of the game is 5, with Player 1 choosing row $\bar{i} = 2$ and Player 2 choosing column $\bar{j} = 2$.
\end{example}

\subsection{Generalization}
In more general cases where the strategy spaces $X$ and $Y$ are not finite, a similar reasoning applies. 
Let $(X, Y, f)$ describe the game, where $X$ and $Y$ are arbitrary strategy sets. 
The conservative values can be defined as follows: 
\begin{itemize} 
    \item Player 1: $v_1 = \sup_x \inf_y f(x, y)$.
    \item Player 2: $v_2 = \inf_y \sup_x f(x, y)$. 
\end{itemize}
If $v_1 = v_2$, the game has a value $v = v_1 = v_2$.
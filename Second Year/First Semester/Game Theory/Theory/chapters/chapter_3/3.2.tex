\section{Rationality}

Now, suppose the following holds:
\begin{itemize}
    \item $v_1 = v_2 = v$.
    \item There exists a row $\bar{i}$ such that $p_{\bar{i}\bar{j}} \geq v_1 = v$ for all $j$.
    \item There exists a column $\bar{j}$ such that $p_{\bar{i}\bar{j}} \leq v_2 = v$ for all $i$.
\end{itemize}
In this case, $p_{\bar{i}\bar{j}} = v$, and this value represents the rational outcome of the game.

The strategies $\bar{i}$ and $\bar{j}$ are optimal because: 
\begin{itemize} 
    \item Player 1 cannot guarantee more than $v_2$, the conservative value of Player 2. 
    \item Player 2 cannot pay less than $v_1$, the conservative value of Player 1. 
\end{itemize}
Thus, $\bar{i}$ maximizes the function $\alpha(i) = \min_j p_{ij}$, and $\bar{j}$ minimizes the function $\beta(j) = \max_i p_{ij}$.

\subsection{Existence of a rational outcome}
To demonstrate the existence of a rational outcome in a zero-sum game, we need to establish the following:
\begin{enumerate}
    \item \textit{Equality of conservative values}: the conservative values of both players agree, i.e., $v_1 = v_2$.
    \item \textit{Existence of an optimal strategy for Player 1}: there exists a strategy $\bar{x}$ such that:
        \[v_1 = \inf_y f (\bar{x}, y)\]
        This ensures that $\bar{x}$ is an optimal strategy for Player 1. 
    \item \textit{Existence of an optimal strategy for Player 2}: there exists a strategy $\bar{y}$ such that:
        \[v_2 = \sup_x f (x, \bar{y})\]
        This ensures that $\bar{y}$ is an optimal strategy for Player 2.
\end{enumerate}
In the case where the strategy spaces are finite, such optimal strategies $\bar{x}$ and $\bar{y}$ always exist. 
Therefore, proving the existence of a rational outcome is equivalent to demonstrating the equality of the conservative values, i.e., $v_1 = v_2$.

\begin{theorem}[Von Neumann]
    There always exists a rational outcome for a finite two-player zero-sum game, as described by its payoff matrix $P$.
\end{theorem}
This fundamental result, known as the Minimax theorem, guarantees that in every finite zero-sum game, the conservative values for both players coincide, and optimal strategies exist for both players, leading to a rational outcome.
\begin{proof}
    Suppose without loss of generality that all $p_{ij}$ in the matrix $P$ are positive. 
    Take the column vectors $p_1, \dots, p_m$ of $\mathbb{R}^n$, and call $C$ their convex hull. 
    Define 
    \[Q_t = \{x \in \mathbb{R}^n : x_i \leq t \}\]
    and 
    \[v = \sup \{t \geq 0 : Q_t \cap C = \emptyset\}\]
    Since $\text{int } Q_v \cap C = \emptyset$, the sets $Q_v$ and $C$ can be separated by a hyperplane; thus, there are coefficients $\bar{x}_1, \dots, \bar{x}_n$, with some $\bar{x}_i \neq 0$, and $b \in \mathbb{R}$ such that
    \[(\bar{x}, u) = \sum_{i=1}^{n} \bar{x}_i u_i \leq b \leq \sum_{i=1}^{n} \bar{x}_i w_i = (\bar{x}, w)\]
    for all $u = (u_1, \dots, u_n) \in Q_v$, $w = (w_1, \dots, w_n) \in C$. 
    Then it follows that:
    \begin{enumerate}
        \item All $\bar{x}_i$'s must be non-negative, and hence we can assume $\sum \bar{x}_i = 1$.
        \item $b = v$: since $\bar{v} := (v, \dots, v) \in Q_v$, from $(\bar{x}, \bar{v}) = \sum_i \bar{x}_i v = v \sum_i \bar{x}_i = v$, we get $b \geq v$; but, if $b > v$, by taking a small $a > 0$ such that $b \geq v + a$, then we have $\sup \{\sum_{i=1}^{n} \bar{x}_i u_i : u \in Q_{v + a} \} < b$, which implies $Q_{v + a} \cap C = \emptyset$, against the definition of $v$.
        \item $Q_v \cap C \neq \emptyset$: Let $\bar{w} = \sum_{j=1}^{m} \bar{y}_j p_j$ (given that $C$ is convex) for some $\bar{y} = (\bar{y}_1, \dots, \bar{y}_m) \in \Sigma_m$. Since $\bar{w} \in Q_v$, it follows $\bar{w}_i \leq v$ for all $i$.
    \end{enumerate}
    We now prove that $\bar{x}$ is optimal for the first player, that $\bar{y}$ is optimal for the second player, and that $v$ is the value of the game.
    \begin{itemize}
        \item As for the first player: since $(\bar{x}, w) \geq v$ for every $w \in C$, by the separation result, and since obviously every column $p_{\cdot j} \in C$, then
            \[(\bar{x}, p_{\cdot j}) \geq v, \quad \text{for all } j\]
        \item Now consider $\sum_{j=1}^{m} \bar{y}_j p_j = w \in Q_v \cap C$ as before. 
            Then $w_i = \bar{y} p_{i \cdot}$. Since $w \in Q_v$, then $w_i \leq v$ for every $i$ and
            \[v \geq w_i = \bar{y} p_{i \cdot}\]
    \end{itemize}
\end{proof}
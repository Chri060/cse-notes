\section{Optimality}

Let $X$ and $Y$ be arbitrary sets. 
Suppose:
\begin{enumerate}
    \item $v_1 = v_2 := v$.
    \item There exists a strategy $\bar{x}$ such that $f(\bar{x}, y) \geq v$ for all $y \in Y$.
    \item There exists a strategy $\bar{y}$ such that $f(x, \bar{y}) \leq v$ for all $x \in X$. 
\end{enumerate}
Then: 
\begin{itemize} 
    \item $v$ is the rational outcome of the game. 
    \item $\bar{x}$ is an optimal strategy for Player 1. 
    \item $\bar{y}$ is an optimal strategy for Player 2. 
\end{itemize}
It follows that $\bar{x}$ is optimal for Player 1 since it maximizes $\alpha(x) = \inf_y f(x, y)$, while $\bar{y}$ is optimal for Player 2 since it minimizes $\beta(y) = \sup_x f(x, y)$. 
The values $\alpha(x)$ and $\beta(y)$ represent the best responses for the players if they knew the opponent's strategy.

\subsection{Conservative values different or equal}
\begin{proposition}
    Let $X$ and $Y$ be nonempty sets, and let $f : X \times Y \rightarrow \mathbb{R}$ be an arbitrary real-valued function.
    Then: 
    \[v_1 = \sup_x\inf_yf (x, y) \leq \inf_y\sup_xf (x, y) = v_2\]
\end{proposition}
\begin{proof}
    By definition, for all $x \in X$ and $y \in Y$:
    \[\inf_yf(x, y) \leq f (x, y) \leq \sup_xf (x, y)\]
    Thus, for all $x$ and $y$, it holds that:
    \[\alpha(x) = \inf_yf (x, y) \leq \sup_xf (x, y) = \beta(y)\]
    Taking the supremum over $x$ and the infimum over $y$, we conclude:
    \[\sup_x\alpha(x) \leq \inf_y\beta(y)\]
\end{proof}
As a result, it follows that for any game, $v_1 \leq v_2$.
\begin{example}
    Consider the game of rock-paper-scissors, represented by the following matrix:
    \[\begin{pmatrix} 0 & 1 & -1 \\ -1 & 0 & 1 \\ 1 & -1 & 0 \end{pmatrix}\]
    The conservative values are not the same: in fact, $v_1 = -1$ and $v_2 = 1$. 

    Here, $v_1 = -1$ and $v_2 = 1$, indicating the conservative values are not equal. 
    Therefore, no single deterministic strategy guarantees a win. 
    However, in a repeated game with mixed strategies, both players should play each option with equal probability (one-third of the time), resulting in an expected utility of zero for both players.
\end{example}

\subsection{Conservative values not equal}
When the conservative values differ, mixed strategies must be considered. 
In this case, the strategy spaces for both players are probability distributions:
\[\sum_k=\left\{x=(x_1,\dots,x_k)|x_i\geq 0 \text{ and }\sum_{i=1}^{k}x_i=1\right\}\]
Here, $k = n$ for Player 1 and $k = m$ for Player 2. 
The utility function is extended to:
\[f(x,y)=\sum_{i=1,\dots,n,j=i,\dots,m}x_iy_jp_{ij}=(x,Py)\]
Thus, the mixed extension of the original game is given by:
\[\left(\sum_n, \sum_m, f (x, y) = (x, Py)\right)\]

\subsection{Pure strategies optimality}
\begin{theorem}
    If a player knows the strategy being used by the opposing player, they can always adopt a pure strategy to achieve the best possible outcome.
\end{theorem}
This means that once one player's choice is fixed, the optimization problem reduces to a linear problem over a simplex, given that the utility function in such a game is bilinear.
\begin{proof}
    Consider Player 2, who knows that Player 1 is using a mixed strategy $\bar{x}$. 
    Player 2's task is then to minimize the function:
    \[f (\bar{x}, y) = (\bar{x}, Py)\]
    over the simplex $\sum_m$ (the set of mixed strategies for Player 2). 
    The optimal value will be attained at one of the vertices $e_j$ of the simplex, which corresponds to a pure strategy.
    Thus, Player 2 can use a pure strategy to achieve the optimal outcome.
\end{proof}
Given a payoff matrix  $P$, let the column vector corresponding to the $j$-th pure strategy be denoted as $p_{\cdot j}$, and the row vector corresponding to the $i$-th pure strategy as $p_{i\cdot}$, respectively. 
The payoff of the first player in the mixed extension of the game is given by:
\[f(x,y)=(x,Py)\]
The previous theorem implies that, to verify the existence of a rational outcome for the game, we need to show the existence of mixed strategies $\bar{x}$ and $\bar{y}$, as well as a value $v$, such that: 
\begin{itemize}
    \item $(\bar{x},P_{e_j})=(\bar{x},p_{\cdot j})$ for every column $j$. 
    \item $(e_i,p_{i\cdot}\bar{y})\leq v$ for every row $i$.
\end{itemize}
Here, $e_j$ is the $j$-th strategy of Player 2, and $e_i$ is the $i$-th strategy of Player 1.
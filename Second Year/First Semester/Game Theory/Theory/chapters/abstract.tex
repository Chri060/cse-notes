\begin{abstract}
    The course begins by exploring the fundamental assumptions of decision theory and comparing it with interactive decision theory, highlighting the key differences between these approaches.

    It then moves on to the study of non-cooperative games, including those represented in extensive form and games with perfect information, where backward induction and combinatorial games are discussed.
    
    The focus shifts to zero-sum games, analyzing conservative values, equilibrium in pure strategies, and the extension of finite games to mixed strategies. 
    It covers the von Neumann theorem and the application of linear programming to find optimal strategies and the value of finite games.
    
    The course also delves into the Nash non-cooperative model, discussing Nash equilibrium and the existence of mixed equilibria in finite games. 
    Examples include potential games, where students learn to identify potential functions, and other types such as congestion, routing, network, and location games. 
    The concepts of the price of stability and the price of anarchy are introduced, along with correlated equilibria.
    
    Finally, the course covers cooperative games, providing definitions and examples. 
    It examines key concepts such as the core, the nucleolus, the Shapley value, and power indices.
\end{abstract}
\section{Impartial combinatorial games}

\begin{definition}[\textit{Impartial combinatorial game}]
    An impartial combinatorial game is a game such that:
    \begin{enumerate}
        \item There are two players moving in alternate order.
        \item There is a finite number of positions in the game.
        \item The players follow the same rules.
        \item The game ends when no further moves are possible.
        \item The game does not involve chance.
        \item In the classical version, the winner is the player leaving the other player with no available moves, in the mis`ere version the opposite.
    \end{enumerate}
\end{definition}
\begin{example}
    Examples of impartial combinatorial games are: 
    \begin{itemize}
        \item $k$ piles of cards. At her turn each player takes as many cards as she wants (at least one!) from one and only one pile. 
        \item $k$ piles of cards. At her turn each player takes as many cards as she wants (at least one) from no more than $j < k$ piles. 
        \item $k$ cards in a row. At her turn each player takes either $j_1$ or $\vdots$ or $j_l$ cards. 
    \end{itemize}
    In all these variants of the game, the player remaining without cards loses.
    In the first two cases the positions are $(n_1, \cdots , n_k)$ where $n_i$ is a non negative integer for all $i$. 
    In the last example positions can be seen as all non negative integers smaller or equal to $k$.
\end{example}
To solve this type of games we start by partitioning the set of all possible (finitely many) positions into two sets:
\begin{enumerate}
    \item $P$-positions (i.e. previous player has a winning strategy): loosing.
    \item $N$-positions (i.e. next player has a winning strategy): winning.
\end{enumerate}
Note: it is the state of the game that matters, and not who is called to move.
Rules for the partition (for the classical version):
\begin{itemize}
    \item Terminal position $(0, 0, \cdots, 0)$ is a $P$-position (it is a losing position, as the player does not have any card left)
    \item From a $P$-position only $N$-positions are available
    \item From a $N$-position it is possible, yet not necessary, to go to a $P$-position. 
\end{itemize}
Therefore, the player starting from a $N$-position wins. 

\subsection{Nim game}
The Nim game is defined as $(n_1, \cdots, n_k)$ where $n_i$ is a positive integer for all $i$.
At her turn any player is supposed to take one (and only one) $n_i$ and substitute it with $\hat{n}_i < n_i$. 
The winner is the player who arrives at the position $(0, \dots, 0)$. 
\begin{itemize}
    \item Actions: taking away cards from one pile.
    \item Goal: to clear the whole table.
\end{itemize}

\begin{theorem}[Bouton]
    In the Nim game the position $(n_1, n_2, \dots, n_k)$ is a $P$-position if and only if:
    \[n_1 \oplus n_2 \oplus \dots \oplus n_k = 0\]
\end{theorem}
\begin{proof}
    Terminal position $(0, 0, \dots , 0)$ is a $P$-position, with zero Nim-sum.

    Positions with $n_1 \oplus n_2 \oplus \dots \oplus n_N = 0$ go only to positions with non-zero Nim-sum. 
    For, suppose that the next position $( \hat{n}_1, n_2, \dots , n_N )$ is such that $\hat{n}_1 \oplus n_2 \oplus \dots \oplus n_N = 0 = n_1 \oplus n_2 \oplus \dots \oplus n_N$ : then, by the cancelation law one has $\hat{n}_1 = n_1$, which is impossible since the game requires $\hat{n}_1 < n_1$.

    Positions with $n_1 \oplus n_2 \oplus \dots \oplus n_N \neq 0$ can go to positions with zero Nim-sum
    Let $z := n_1 \oplus n_2 \oplus \dots \oplus n_N\neq 0$. Take a pile having 1 in the first column on the left of the expansion of $z$ and put 0 there; then go right, leaving unchanged digits corresponding to 0 and changing them otherwise. 
    Provably, the result is smaller than the original number.
\end{proof}
\begin{example}
    From the following arrangement yielding non-zero Nim-sum: $4, 6,5$. 
    by taking a card out of the first row we go to the next one with zero Nim-sum  $3, 6,5$. 
    So, there are three initial good moves, one for each row.
\end{example}

\subsection{Conclusions}
Games with perfect information can be solved by using backward induction. 
However backward induction is a concrete solution method only for very simple games, because of limited rationality.
Depending on the game, we can reach different levels of solutions.
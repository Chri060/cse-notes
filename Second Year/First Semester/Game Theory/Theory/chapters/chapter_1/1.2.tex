\section{Game theory assumptions}

Game theory assumes that the players are supposed to be: 
\begin{enumerate}
    \item \textit{Selfish}. 
    \item \textit{Rational}. 
\end{enumerate}

\subsection{Selfishness}
The players only care about their own preferences with respect to the outcomes of the game.
This is not an ethical problem, but a mathematical assumption. 
In fact, we need it to define what is the meaning of a ratinal choice.
    



\subsection{Rationality}
\begin{definition}[\textit{Preference relation}]
    Let $X$ be a set. 
    A preference relation on $X$ is a binary relation $\preceq$ such that for all $x,y,z\in X$: 
    \begin{itemize}
        \item Reflexive: $x\preceq x$. 
        \item Complete: $x \preceq y$ or $y \preceq x$ or both. 
        \item Transitive: if $x \preceq y$ and $y \preceq z$, then $x \preceq z$. 
    \end{itemize}
\end{definition}
The transitive property is useful to have a consistent ranking. 

\paragraph*{First rationality assumption}
The first rationality assumption is that the players are able to provide a preference relation over the outcomes of the game, and the order must be consistent. 
\begin{definition}[\textit{Utility function}]
    Let $\preceq$ be a preference relation over $X$.
    A utility function representing $\preceq$ is a function $u:X\rightarrow\mathbb{R}$ such that: 
    \[u(x)\geq u(y)\Leftrightarrow x \preceq y\]
\end{definition}
A utility function may not exists in particular cases, however it exists in the general setting, specifically when $X$ is a finite set. 
If a utility function exists, then there exist infinitely many utility functions, given by any strictly increasing transformation of the former. 

To player $i$ there is assigned a set $X_i$, repesenting all the choices available to her.
Hence, the set $X=xX_i$ over which $u$ is defined comprises the possible choices of all players. 

\paragraph*{Second rationality assumption}
The second rationality assumption states that the agents are able to provide a utility function representing their preference relations, whenever it is necessary. 

\paragraph*{Third rationality assumption}
The third rationality assumption states that the players use consistently the laws of probability. 
In particular, they are consistent with the computation of the expected utilities, they are able to update probabilities according to Bayes rule. 

\paragraph*{Fourth rationality assumption}
The fourth rationality assumption states that the players are able to understand the consequences of all their actions, the consequences of this information on any other player, the consequences of the consequences and so on.

\paragraph*{Fifth rationality assumption}
The fifth rationality assumption states that the players are able to use decision theory, whenever it is possible.

That is, given a set of alternatives $X$, and a utility function $u$ on $X$, each player  seeks a $\bar{x}\in X$ such that: 
\[u(\bar{x}) \geq u(x)\qquad\forall x \in X\]

An important consequence of the previous axioms is the principle of elimination of strictly dominated strategies. . 
A player does not take an action $a$ it she has available an action $b$ providing her a strictly better result, no matter what the other players do.











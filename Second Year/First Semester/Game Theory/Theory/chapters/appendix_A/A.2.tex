\section{Group}
\begin{definition}[\textit{Group}]
    A nonempty set $A$ with a binary operation $\cdot$ defined on it is called a group provided that:
    \begin{enumerate}
        \item for $a, b \in A$ the element $a \cdot b \in A$
        \item $\cdot$ is associative: $(a \cdot b) \cdot c = a \cdot (b \cdot c)$
        \item there is a (unique) element $e$, called identity, such that $a \cdot e = e \cdot a = a$ for all  $a \in A$
        \item for every $a \in A$ there is $b \in A$ such that $a \cdot b = b \cdot a = e$: such an element isunique and called inverse of $a$
    \end{enumerate}
    If $a \cdot b = b \cdot a$ for all $a, b \in A$ the group is called abelian.
\end{definition}

\begin{example}
    Examples of abelian groups:
The integers $\mathbb{Z}$, equipped with the usual sum
The real numbers excluded $0$, equipped with the usual product

Examples of non-abelian groups:
The $n \times n$ matrices with non-zero determinant, equipped with the usual product
\end{example}
\begin{proposition}
    Let $(A, \cdot)$ be a group. 
    Then the cancellation law holds:
\[a \cdot b = a \cdot c \implies b = c\]
\end{proposition}
\begin{proof}
    By multiplying by $a^{-1}$ both sides of the equation $a \cdot b = a \cdot c$, one obtains $a^{-1}a \cdot b = a^{-1} a \cdot c$. 
    Then, insofar $a^{-1}$ is the inverse of $a$, this expression reduces to $e \cdot b = e \cdot c$, which by the property of the identity $e$ is exactly equal to $b = c$.
\end{proof}

\begin{proposition}
    The set of the natural numbers with operation $\oplus$ is an abelian group.
\end{proposition}
\begin{proof}
    The identity element is of course $0$. 
    The inverse of $n$ is $n$ itself.
    Associativity and commutativity of $\oplus$ are easy to show.
\end{proof}
Therefore, the cancellation law holds: $n_1 \oplus n_2 = n_1 \oplus n_3 \implies n_2 = n_3$.
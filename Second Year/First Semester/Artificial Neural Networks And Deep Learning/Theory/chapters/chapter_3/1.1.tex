\caption{Image classification}

A digital colored image in a pc is saved using three matrces: one for green, one for red, and one for blue. 
Each element of the matrix is composed by 8 bits (value range from 0 and 255)
The diaganal of the cube with the three colors contains all the tonalities of gray. 

In Python we can divide the three sub-matrices in the following way: 
\begin{verbatim}
# Read the image
I = imread('image.jpg')

# Extract the color channels
R = I[:, :, 0]
G = I[:, :, 1]
B = I[:, :, 2]
\end{verbatim}
When loaded in memory, image sizes are much larger than on the disk where images are typically compressed. 

The images must be decompressed to be analyzed, so the images are big data to be manages. 
For videos we hame multiple images per second, augmenting the memory to be used. 

Without compression: 1Byte per color per pixel
1 frame in full HD: R = 1080, C= 1920 ≈ 6 MB
1 sec in full HD (24fps) ≈ 150 MB

Fortunately, visual data are very redundant, thus compressible.
This has to be taken into account when you design a Machine learning algorithm for images or vides.


\subsection{Local transformations}







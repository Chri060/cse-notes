\section{Data analysis}

We start our analysis by importing the well-known Iris dataset and printing a description of it. 
\begin{lstlisting}[style=Python]
data = load_iris()
print(data.DESCR)
\end{lstlisting}
Instead of obtaining a general description of the dataset, we may want to inspect a table with the features of the elements in the dataset.
We could do this as follows: 
\begin{lstlisting}[style=Python]
iris_dataset = pd.DataFrame(data.data, columns=data.feature_names)
print('Iris dataset shape', iris_dataset.shape)
iris_dataset.head(10)
\end{lstlisting}
We can now print all the statistical data of the given dataset with the following command: 
\begin{lstlisting}[style=Python]
print('Iris dataset shape', iris_dataset.shape)
iris_dataset.describe()
\end{lstlisting}
To find how the dataset has beed divided and classified, and also retrieve the labales of the elements in the dataset we can use the following command: 
\begin{lstlisting}[style=Python]
target = data.target
print('Target shape', target.shape)
unique, count = np.unique(target, return_counts=True)
print('Target labels:', unique)
for u in unique:
    print(f'Class {unique[u]} has {count[u]} samples')
\end{lstlisting}
Lastly, we can check the distribution of the samples over the space. 
\begin{lstlisting}[style=Python]
plot_dataset = iris_dataset.copy()
plot_dataset["Species"] = target
sns.pairplot(plot_dataset, hue="Species", palette="tab10", markers=["o", "s", "D"])
plt.show()
del plot_dataset
\end{lstlisting}
In particular, in this case we can see that there is a linear decision boundary in many dimensions. 
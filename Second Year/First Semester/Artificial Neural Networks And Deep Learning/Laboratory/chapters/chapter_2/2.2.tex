\section{Data analysis}

We start our analysis by importing the California housing dataset and printing a description of it. 
\begin{lstlisting}[style=Python]
data = fetch_california_housing(as_frame=True)
print(data.DESCR)
\end{lstlisting}
Instead of obtaining a general description of the dataset, we may want to inspect a table with the features of the elements in the dataset.
We could do this as follows: 
\begin{lstlisting}[style=Python]
housing_dataset = data.frame
print('California Housing dataset shape', housing_dataset.shape)
housing_dataset.head(10)
\end{lstlisting}
We can now print all the statistical data of the given dataset with the following command: 
\begin{lstlisting}[style=Python]
print('California Housing dataset shape', housing_dataset.shape)
housing_dataset.describe()
\end{lstlisting}




We can check the structure of the datase:
\begin{lstlisting}[style=Python]
# Set Seaborn theme with white grid style
sns.set_theme(style="whitegrid")

# Define a harmonious colour palette
palette = sns.color_palette("viridis", n_colors=9)

# Generate the figure and axes for subplots
fig, axes = plt.subplots(nrows=3, ncols=3, figsize=(16, 10), sharex=False, sharey='row')

# Flatten the axes array for easier access
axes = axes.flatten()

# List the columns of the dataset
columns = housing_dataset.columns

# Create histograms with customised colours for each feature
for i, column in enumerate(columns):
    if i < len(axes):  # Ensure the number of subplots is not exceeded
        sns.histplot(
            data=housing_dataset,
            x=column,
            kde=True,
            ax=axes[i],
            color=palette[i],
            edgecolor='white',
            alpha=0.7
        )
        axes[i].set_title(column.capitalize(), fontsize=14, fontweight='bold', color='#333333')
        axes[i].set_xlabel('')  # Remove x-axis labels for all subplots
        if i % 3 != 0:
            axes[i].set_ylabel('')  # Remove y-axis labels except for the first column
        axes[i].tick_params(axis='both', which='major', labelsize=10)

        # Improve label readability by limiting the number of ticks
        axes[i].xaxis.set_major_locator(ticker.MaxNLocator(6))
        axes[i].yaxis.set_major_locator(ticker.MaxNLocator(6))

# Remove any extra subplots not used
for i in range(len(columns), len(axes)):
    fig.delaxes(axes[i])

# Enhance layout with proper padding
plt.tight_layout(pad=3.0, h_pad=2.5, w_pad=2.0)
fig.suptitle('Distribution of Housing Dataset Features', fontsize=20, fontweight='bold', y=1.02, color='#444444')
plt.show()
\end{lstlisting}
Lastly, we can check the distribution of the samples over the space. 
\begin{lstlisting}[style=Python]
housing_dataset.plot(
    kind="scatter",
    x="Longitude",
    y="Latitude",
    c="MedHouseVal",
    cmap="jet",
    colorbar=True,
    legend=True,
    sharex=False,
    figsize=(10,7),
    s=housing_dataset['Population']/100,
    label="Population",
    alpha=0.7
    )
plt.show()
\end{lstlisting}
\section{Perspective and vanishing point}

When considering all lines parallel to the direction parameters $\begin{bmatrix} \alpha & \beta & 1 \end{bmatrix}$, we can establish the following system of equations:
\[\begin{cases} X = X_0 + \alpha Z \\ Y = Y_0 + \beta Z \\ Z = 1 \cdot Z \end{cases}\]
To project these lines onto the 2D image using the triangle equality, we derive the following expressions:
\[x=f \dfrac{X}{Z} = f \dfrac{X_0 + \alpha Z}{Z} = f\alpha + \dfrac{X_0}{Z}\]
\[y=f \dfrac{Y}{Z} = f \dfrac{Y_0 + \beta Z}{Z}  = f\beta  + \dfrac{Y_0}{Z}\]
Next, we find the image of the point at infinity along these lines, which results in the point:
\[\begin{bmatrix} f\alpha & f\beta \end{bmatrix}\]
Remarkably, this image point is independent of the values of $X_0$ and $Y_0$. 
Therefore, all parallel lines share the same image of their points at infinity.
\begin{definition}[\textit{Vanishing point}]
    The image of the point at infinity of the lines is known as the vanishing point. 
\end{definition}
Consequently, we observe that all parallel lines in the real world project onto converging lines in the image.
\begin{figure}[H]
    \centering
    \includegraphics[width=0.4\linewidth]{images/vanishingpoint.png}
    \caption{Vanishing point}
\end{figure}
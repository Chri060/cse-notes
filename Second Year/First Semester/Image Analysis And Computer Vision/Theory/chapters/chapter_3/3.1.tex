\section{Introduction}

Recovering a model of an unknown planar scene from a single image, where the image is a projective transformation of the scene, presents a challenging problem.
This transformation is represented by the equation $\mathbf{x}_i^\prime=\mathbf{Hx}_i$, where $\mathbf{x}_i$ denotes the scene points and $\mathbf{H}$ is the transformation matrix.
The difficulty arises from the fact that while the scene points $\mathbf{x}_i$ are known, the transformation matrix $\mathbf{H}$ is unknown, making a direct inversion of the mapping impossible.

The complexity of this task stems from the large number of unknown variables in the transformation matrix $\mathbf{H}$. 
In its general form, this problem is underdetermined, meaning that without additional constraints or simplifications, it cannot be uniquely solved. 
To address this challenge, two primary strategies are typically employed:
\begin{enumerate}
    \item \textit{Reducing the number of unknowns}: in many practical cases, the goal is not to recover the exact original configuration of the scene but rather to retrieve its overall geometric structure, a process known as shape reconstruction.
        By focusing on the shape rather than the full projective transformation, the number of unknowns can be reduced from eight to four. In this case, the transformation matrix $\mathbf{H}$ simplifies to:
        \[\mathbf{H}= \begin{bmatrix} s\cos \vartheta & -s\sin \vartheta & t_x \\ s\sin \vartheta & s\cos \vartheta & t_y \\  0 & 0 & 1 \end{bmatrix}\]
    \item \textit{Adding constraints}: another approach is to incorporate additional information to constrain the reconstruction.
        This extra information often involves parameters that remain invariant under the specific type of transformation being sought, but vary under more general classes of transformations.
        By leveraging these invariants, we can narrow down the possible solutions and recover a more accurate model of the scene.
\end{enumerate}
Reconstruction methods can be broadly classified into two categories:
\begin{itemize}
    \item \textit{Affine reconstruction}: in this approach, the reconstructed scene is related to the original through an affine transformation, preserving parallelism but not necessarily angles or distances.
    \item \textit{Shape reconstruction}: this method seeks to recover the overall shape of the scene using a similarity transformation, which preserves both angles and relative distances, providing a more faithful representation of the scene's geometry while simplifying the problem.
\end{itemize}
\section{Focalization of parallel light rays}

\begin{figure}[H]
    \centering
    \includegraphics[width=0.4\linewidth]{images/focalization.png}
\end{figure}
In the image, we observe two rays: one passing through the center of the lens and another that remains parallel to the first ray but passes through a different point. 
From this, we can make the following observations:
\begin{itemize}
    \item When $Y = 0$, the ray experiences no deviation and continues straight without deflection.
    \item Using the relationship $Y=f\cdot\delta\theta$, we can express the focal length of the lens as follows:
        \[f=\dfrac{1}{(n-1)\left(\dfrac{1}{\rho_1}+\dfrac{1}{\rho_2}\right)}\] 
\end{itemize}
This indicates that all parallel rays converge at a common point known as the focal point, denoted as $Z$. 
The distance from the focal point to the $y$ axis is given by:
\[Z=-f\]
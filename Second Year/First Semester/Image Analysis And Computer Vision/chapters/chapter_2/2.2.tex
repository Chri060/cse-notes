\section{Points}

To define points in Cartesian coordinates, we establish a Euclidean plane with a designated origin. 
Each point is uniquely represented by a pair of Cartesian coordinates, $\begin{bmatrix} x & y \end{bmatrix}$.

For image analysis, it is advantageous to use homogeneous coordinates.
This involves constructing a 3D space with axes labeled $\begin{bmatrix} x & y & w \end{bmatrix}$. 
To represent a point, we assign three values, which allows for an infinite number of representations by varying the value of $w$.
The relationship between Cartesian and homogeneous coordinates can be expressed as follows:
\[\mathbf{x} = \begin{bmatrix} x \\ y \\ w \end{bmatrix} = w \begin{bmatrix} X \\ Y \\ 1 \end{bmatrix}\]

Consequently, a vector $\mathbf{x} = {\begin{bmatrix} x & y & w \end{bmatrix}}^T$ and all its nonzero multiples, including ${\begin{bmatrix} \frac{x}{w} & \frac{y}{w} & 1 \end{bmatrix}}^T$, represent the same point in Cartesian coordinates ${\begin{bmatrix} X & Y \end{bmatrix}}^T={\begin{bmatrix}  \frac{x}{w} &  \frac{y}{w} \end{bmatrix}}^T$ on the Euclidean plane. 
\begin{property}[Homogeneity]
    Any vector $\mathbf{x}$ is equivalent to all its nonzero multiples $\lambda \mathbf{x}$, where $\lambda \neq 0$, as they denote the same point.
\end{property}
The null vector does not represent any point.

\begin{definition}[\textit{Projective plane}]
    We define the projective plane as:
    \[\mathbb{P}^2=\left\{{\begin{bmatrix} x & y & w \end{bmatrix}}^T \in \mathbb{R}^3\right\}\setminus\left\{{\begin{bmatrix} 0 & 0 & 0 \end{bmatrix}}^T\right\}\]
\end{definition}
\begin{example}
    The origin of the plane is defined as: 
    \[{\begin{bmatrix} 0 & 0 & 1 \end{bmatrix}}^T\]
    A generic point in homogeneous coordinates can easily be transformed into Cartesian coordinates by simple division.
    For instance, the point:
    \[{\begin{bmatrix} 0 & 8 & 4 \end{bmatrix}}^T\]
    in Cartesian coordinates is: 
    \[{\begin{bmatrix} \dfrac{x}{w} & \dfrac{y}{w} \end{bmatrix}}^T=\begin{bmatrix} \dfrac{0}{4} & \dfrac{8}{4} \end{bmatrix}=\begin{bmatrix} 0 & 4 \end{bmatrix}\]
\end{example}
Consider a point $\mathbf{x}={\begin{bmatrix} x & y & w \end{bmatrix}}^T$, and let $w$ slowly decrease from $w=1$. 
As $w$ decreases, the point moves in a constant direction $\begin{bmatrix} x & y \end{bmatrix}$, distancing itself from the origin.
As $w$ approaches $0$, the point tends toward infinity along the direction $\begin{bmatrix} x & y \end{bmatrix}$. 
\begin{definition}[\textit{Point at the infinity along the direction}]
    We define the point at the infinity along the direction $\begin{bmatrix} x & y \end{bmatrix}$ as: 
    \[\mathbf{x} = \begin{bmatrix} x \\ y \\ w \end{bmatrix}\]
\end{definition}
Points at infinity, representing directions, exist outside the Euclidean plane and are well-defined within the projective plane. 
Thus, the projective plane encompasses not only the Euclidean plane but also these points at infinity.
\section{Points}

In spatial geometry, points are typically represented in a Euclidean coordinate system by defining a reference origin and specifying each point's location with three Cartesian coordinates $(x, y, z)$.
This system provides an unambiguous means of defining each point's position within three-dimensional space.

For image analysis, however, it is often more convenient to use homogeneous coordinates. 
Homogeneous coordinates introduce an additional coordinate, $w$, allowing points to be represented as follows: 
\[\mathbf{X} = \begin{bmatrix} x \\ y \\ z \\ w \end{bmatrix} = w\begin{bmatrix} X \\ Y \\ Z \\ 1 \end{bmatrix}\]
In this form, any non-zero scalar multiple of $\mathbf{X}$, denoted as $\lambda\mathbf{X}$ for $\lambda \neq 0$, represents the same point.
This property, known as homogeneity, is foundational in projective geometry as it enables the representation of points at infinity, which cannot be represented in a traditional Cartesian system.
Notably, the null vector does not correspond to any point.
\begin{definition}[\textit{Projective space}]
    The projective space $\mathbb{P}^3$ is defined as:
    \[\mathbb{P}^3=\{{\begin{bmatrix} x & y & z & w \end{bmatrix}}^T \in \mathbb{R}^4\}-\{{\begin{bmatrix} 0 & 0 & 0 & 0 \end{bmatrix}}^T\}\]
\end{definition}
\begin{property}
    A point $\mathbf{X}$, formed by a linear combination $\mathbf{X}=\alpha\mathbf{X}_1+\beta\mathbf{X}_2$ of two points $\mathbf{X}_1$ and $\mathbf{X}_2$, lies on the line $\mathbf{l}$ that passes through both $\mathbf{X}_1$ and $\mathbf{X}_2$.
\end{property}
\section{Lines}

In space geometry, lines act as intermediate entities between points and planes. 
While they are considered primitive in planar geometry, in three-dimensional space they are defined in terms of points and planes.
Lines are self-dual elements, meaning that statements about lines retain their validity when duality is applied, reflecting the symmetry in their relationships with points and planes.

Lines can be defined in various ways:
\begin{itemize}
    \item \textit{Intersection of two planes}: a line can be obtained as the intersection of two distinct planes, $\boldsymbol{\pi}_1^T$ and $\boldsymbol{\pi}_2^T$. 
        This yields: 
        \[\mathbf{X}=\text{RNS}\left(\begin{bmatrix} \boldsymbol{\pi}_1^T \\ \boldsymbol{\pi}_2^T \end{bmatrix}\right)\]
        Here, $\mathbf{X}$ represents a 2D solution set of points within both planes, reduced to a 1D set of points due to homogeneity, parameterized by a single variable along the line.
    \item \textit{Passing throug two points}: a line can also be defined by two distinct points, $\mathbf{X}_1$ and $\mathbf{X}_2$. 
        This produces:
        \[\mathbf{X}=\text{RNS}\left(\begin{bmatrix} \mathbf{X}_1^T \\ \mathbf{X}_2^T \end{bmatrix}\right)\]
        This results in a 2D set of solution vectors (representing all planes containing the line), which reduces to a 1D set due to homogeneity, where each plane in the set is determined by a rotation angle.
    \item \textit{Linear combination of two points}: A line $\mathbf{l}$ can also be represented as the set of all points that are linear combinations of two given points $\mathbf{X}_1$ and $\mathbf{X}_2$:
        \[\mathbf{X}=\alpha\mathbf{X}_1+\beta\mathbf{X}_2\]
        Here, $\alpha$ and $\beta$ are scalars. 
        This defines the line in terms of its point-based representation.
    \item \textit{Linear combination of two planes}: alternatively, a line $\mathbf{l}^\ast$ can be defined as the set of all planes that are linear combinations of two given planes $\boldsymbol{\pi}_1$ and $\boldsymbol{\pi}_2$:
        \[\boldsymbol{\pi} = \alpha \boldsymbol{\pi}_1 + \beta\boldsymbol{\pi}_2\] 
        Here, $\alpha$ and $\beta$ are scalars. 
        This representation defines the line by the set of planes containing it.
\end{itemize}
\begin{theorem}[Duality Principle]
    For any true statement involving the terms point, line, plane, is on, and goes through, there exists a corresponding dual statement that is also true, derived by making the following substitutions:
\end{theorem}
\begin{itemize}
    \item \textit{Point $\Leftrightarrow$ plane.}
    \item \textit{Is on $\Leftrightarrow$ goes through.}
    \item \textit{Line $\Leftrightarrow$ line.}
\end{itemize}
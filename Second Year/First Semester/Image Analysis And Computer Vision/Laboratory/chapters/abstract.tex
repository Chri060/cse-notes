\begin{abstract}
    The course begins with an introduction to camera sensors, including their transduction, optics, geometry, and distortion characteristics. 
    It then covers the basics of projective geometry, focusing on modeling fundamental primitives such as points, lines, planes, conic sections, and quadric surfaces, as well as understanding projective spatial transformations and projections.

    The course continues with an exploration of camera geometry and single-view analysis, addressing topics like calibration, image rectification, and the localization of 3D models. 
    This is followed by a study of multi-view analysis techniques, which includes 3D shape reconstruction, self-calibration, and 3D scene understanding.
    
    Students will also learn about linear filters and convolutions, including space-invariant filters, the Fourier Transform, and issues related to sampling and aliasing. 
    Nonlinear filters are discussed as well, with a focus on image morphology and operations such as dilation, erosion, opening, and closing, as well as median filters.
    
    The course further explores edge detection and feature detection techniques, along with feature matching and tracking in image sequences. 
    It addresses methods for inferring parametric models from noisy data and outliers, including contour segmentation, clustering, the Hough Transform, and RANSAC (random sample consensus).
    
    Finally, the course applies these concepts to practical problems such as object tracking, recognition, and classification.
\end{abstract}
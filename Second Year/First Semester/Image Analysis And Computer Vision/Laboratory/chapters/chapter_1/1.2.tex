\section{Image processing}

In MATLAB, images are represented as matrices, where pixel intensities correspond to matrix elements. 
Here, we'll cover how to load, display, modify, and process images.

\subsection{Image loading}
To import and display an image: 
\begin{lstlisting}[style=MATLAB] 
im = imread('E1_data/image_Lena512.png'); % Load an image 
imshow(im); % Display the image 
imagesc(im), colormap spring; % Display with arbitrary colormap 
\end{lstlisting}
When converting an image to \texttt{double} format, rescale the pixel intensities to the [0, 1] range: 
\begin{lstlisting}[style=MATLAB] 
imshow(double(im) / 255); % Convert to double in the range [0, 1] 
\end{lstlisting}

\subsection{Image histogram}
An image histogram shows the distribution of intensity values. You can compute and display it as follows: 
\begin{lstlisting}[style=MATLAB] 
h = hist(im(:), 0:255); % Histogram of intensity values 
figure; 
stairs(0:255, h, 'b', 'LineWidth', 3); % Plot histogram 
title('Intensity Histogram'); 
xlabel('Intensity'); 
ylabel('Frequency'); 
axis tight; 
\end{lstlisting}
Histograms represent the probability density function (PDF) of pixel intensities. 
Histogram equalization adjusts these intensities to make the distribution more uniform: 
\begin{lstlisting}[style=MATLAB] 
cdf_im = cumsum(h); % Cumulative distribution function 
cdf_im = cdf_im / cdf_im(end); % Normalize to range [0, 1] 
figure; 
stairs(0:255, cdf_im, 'b', 'LineWidth', 3); 
title('Cumulative Distribution Function'); 
xlabel('Intensity'); 
ylabel('CDF');
\end{lstlisting}
To map pixel intensities based on the CDF: 
\begin{lstlisting}[style=MATLAB] 
map = floor(255 * cdf_im); % Mapping based on CDF 
im_eq = map(im + 1); % Equalized image 
hist_im_eq = hist(im_eq(:), 0:255); % Histogram of equalized image
figure; 
subplot(2, 2, 1); 
imshow(im); 
title('Original Image'); 
subplot(2, 2, 2); 
imshow(im_eq / 255); 
title('Equalized Image'); 
subplot(2, 2, 3);
bar(h); 
title('Original Histogram'); 
subplot(2, 2, 4); 
bar(hist_im_eq); 
title('Equalized Histogram'); 
\end{lstlisting}

\subsection{Image Modification}
You can adjust image brightness by adding a fixed value to each pixel. 
For example: 
\begin{lstlisting}[style=MATLAB] 
figure; 
imshow([im, im + 50]); 
title('Brightness Increased by 50 Levels'); 
\end{lstlisting}
Contrast adjustments can also be controlled by setting intensity limits in \texttt{imshow}: 
\begin{lstlisting}[style=MATLAB] 
figure; 
subplot(2,2,1); 
imshow(im, [-100, 156]); 
title('Brightened, Same Contrast'); 
subplot(2,2,2); 
imshow(im, [0, 156]); 
title('Brightened and Higher Contrast'); 
subplot(2,2,3); 
imshow(im, [100, 256]); 
title('Dimmed, Higher Contrast'); 
subplot(2,2,4); 
imshow(im, [100, 356]); 
title('Dimmed, Same Contrast'); 
\end{lstlisting}

\subsection{RGB Channels}
Color images are represented as 3D matrices, with three layers corresponding to the Red, Green, and Blue channels. 
You can isolate and display each channel: 
\begin{lstlisting}[style=MATLAB] 
imr = im(:, :, 1); 
imshow(imr); 
title('Red Channel'); 
img = im(:, :, 2); 
imshow(img); 
title('Green Channel'); 
imb = im(:, :, 3); 
imshow(imb); 
title('Blue Channel');
\end{lstlisting}

\subsection{Logical Operations on images}
Logical operations can be applied to images to create binary masks: 
\begin{lstlisting}[style=MATLAB] 
l = imb > (1.3 * imr); % Pixels where blue is 30% stronger than red 
imshow(l); 
title('Blue 30% Stronger than Red');
\end{lstlisting}

\subsection{Gamam correction}
Gamma correction adjusts brightness non-linearly. 
Here's a visualization of different gamma transformations: 
\begin{lstlisting}[style=MATLAB] 
x = 0:0.001:1; 
figure;
hold on; 
for gamma = [0.04, 0.1, 0.2, 0.4, 0.7, 1, 1.5, 2.5, 5, 10, 25] 
    y = x .^ gamma; 
    plot(x, y, 'DisplayName', sprintf('\gamma = %.2f', gamma), 'LineWidth', 3); 
    text(x(round(end / 2)), y(round(end / 2)), sprintf('\gamma = %.2f', gamma)); 
    end 
xlabel('Input Intensity'); 
ylabel('Output Intensity'); 
title('Gamma Correction Curves'); 
legend('Location', 'eastoutside'); 
grid on; 
hold off; 
\end{lstlisting}
\section{MATLAB basics}

To begin working in MATLAB, it's helpful to clear the workspace, close all figures, and clear the command window. 
This ensures a clean environment for running code.
\begin{lstlisting}[style=MATLAB] 
close all % Close all figures 
clear % Clear all variables from the workspace 
clc % Clear the command window 
\end{lstlisting}

\subsection{Variables}
In MATLAB, all variables are stored in matrix form, regardless of their type or dimensionality. 
Scalars are represented as $1\times 1$ matrices. 
Variables are created through assignments, and their size reflects the dimensions of the matrix.
\begin{lstlisting}[style=MATLAB] 
v = 12; % Define a scalar 
c = 'c'; % Define a character 
size(v) % Check the size of a variable 
whos v % Display detailed information about a variable 
\end{lstlisting}
The most commonly used variable types in MATLAB include:
\begin{itemize}
    \item \texttt{double} (double-precision floating point),
    \item \texttt{uint8} (8-bit unsigned integer, ranges from 0 to 255),
    \item \texttt{logical} (boolean type, for true/false values).
\end{itemize}

\subsection{Arrays and matrices}
Arrays and matrices in MATLAB can be defined as either row or column vectors. 
MATLAB indices start at 1 (not 0 as in many other programming languages).
\begin{lstlisting}[style=MATLAB] 
    r = [1 2 3 4]; % Row vector 
    c = [1; 2; 3; 4]; % Column vector 
\end{lstlisting}
Vectors can be defined with specific increments:
\begin{lstlisting}[style=MATLAB] 
a = [1 : 2 : 10]; % Vector from 1 to 10 with a step of 2 
\end{lstlisting}
Matrices can also be created directly: 
\begin{lstlisting}[style=MATLAB] 
v = [1 2; 3 4]; 
\end{lstlisting}
You can concatenate matrices or vectors, as long as their dimensions align. 
The apostrophe \texttt{'} symbol is used for matrix transposition.
\begin{lstlisting}[style=MATLAB] 
B = [v', v']; % Concatenate the transpose of v 
num_rows = size(B,1); % Number of rows in B 
num_cols = size(B,2); % Number of columns in B 
\end{lstlisting}
To concatenate arrays along a specified dimension, use the \texttt{cat} function: 
\begin{lstlisting}[style=MATLAB] 
my_matrix = cat(1, my_vec1, my_vec2); % Concatenate along rows 
\end{lstlisting}
o access elements in a matrix: 
\begin{lstlisting}[style=MATLAB] 
my_matrix(2,3); % Element in the 2nd row, 3rd column 
my_matrix(:,2); % All elements in the 2nd column 
my_matrix(1,:); % All elements in the 1st row 
B = my_matrix(:, [1, 3]); % Elements in 1st and 3rd columns 
\end{lstlisting}
MATLAB allows expanding matrices by assigning values outside the current bounds: 
\begin{lstlisting}[style=MATLAB]
my_matrix(:,5) = 1; % Adds a 5th column with all values set to 1 
\end{lstlisting}
The \texttt{end} keyword can be used to reference the last row or column: 
\begin{lstlisting}[style=MATLAB] 
my_matrix(1, [2 4]); % All rows and columns 2 and 4 
my_matrix(:, end-1); % All rows, second-to-last column 
\end{lstlisting}
Flatten a matrix into a single column vector: 
\begin{lstlisting}[style=MATLAB]
my_vector = my_matrix(:); 
\end{lstlisting}

\subsection{Operations}
MATLAB supports a variety of linear algebra operations on vectors and matrices. 
Note the difference between matrix and element-wise operations.
Matrix operations: 
\begin{lstlisting}[style=MATLAB] 
v = [1 2 3]; 
v * v'; % Matrix multiplication 
v' * v; 
\end{lstlisting}
Element-wise operations:
\begin{lstlisting}[style=MATLAB] 
[1 2 3] .* [4 5 6]; % Element-wise multiplication 
[1 2 3] + 5; % Add 5 to each element 
[1 2 3] ./ 2; % Divide each element by 2 
[1 2 3] .^ 2; % Square each element
\end{lstlisting}
Common rounding functions: 
\begin{lstlisting}[style=MATLAB] 
    ceil(10.56); % Round up 
    floor(10.56); % Round down 
    round(10.56); % Round to nearest integer 
\end{lstlisting}
Basic arithmetic functions: 
\begin{lstlisting}[style=MATLAB] 
sum([1 2 3 4]); % Sum of vector elements 
my_matrix = [1:4; 5:8]; 
sum(sum(my_matrix)); % Sum of all elements in a matrix 
sum(my_matrix(:)); % Sum all elements after flattening 
\end{lstlisting}
Using logical operators, you can create vectors of logical values: 
\begin{lstlisting}[style=MATLAB] 
b = v > 2; % Logical vector where each element of v > 2
\end{lstlisting}
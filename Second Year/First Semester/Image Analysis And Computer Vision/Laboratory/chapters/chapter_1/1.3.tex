\section{Homogeneous coordinates in MATLAB}

To illustrate homogeneous coordinates, we begin by analyzing a 5x5 checkerboard image and selecting points manually. 
Let's start by loading the image:
\begin{lstlisting}[style=MATLAB] 
% Load the checkerboard image
I = imread('E1_data/checkerboard.png');
figure(1), imshow(I);
hold on;
[x, y] = getpts();
\end{lstlisting}
After selecting four points on the checkerboard, we store them as homogeneous coordinates by setting the third component to 1:
\begin{lstlisting}[style=MATLAB] 
% Define points in homogeneous coordinates
a = [x(1); y(1); 1];
b = [x(2); y(2); 1];
c = [x(3); y(3); 1];
d = [x(4); y(4); 1];
\end{lstlisting}
To visualize the selected points, we add labels with different colors at each point's location:
\begin{lstlisting}[style=MATLAB] 
% Display points on the image
text(a(1), a(2), 'a', 'FontSize', 12, 'Color', 'r');
text(b(1), b(2), 'b', 'FontSize', 12, 'Color', 'b');
text(c(1), c(2), 'c', 'FontSize', 12, 'Color', 'g');
text(d(1), d(2), 'd', 'FontSize', 12, 'Color', 'c');
\end{lstlisting}
Lines in homogeneous coordinates are represented by 3D vectors, where each vector contains the coefficients of the line equation. 
Using the cross product, we compute several lines based on the selected points:
\begin{lstlisting}[style=MATLAB] 
% Compute lines from pairs of points
lab = cross(a, b); % Line through points a and b (reference line)
lad = cross(a, d); % Line orthogonal to lab
lac = cross(a, c); % Line at 45 degrees with lab
lcd = cross(c, d); % Line parallel to lab
\end{lstlisting}
To verify that points lie on certain lines, we can compute the incidence relations. For instance, checking if a point a lies on the line lab:
\begin{lstlisting}[style=MATLAB] 
% Check incidence relations
a' * lab; % Should be close to zero if a in lab
c' * lcd; % Should also be zero if c in lcd
\end{lstlisting}
We define the borders of the image as lines in homogeneous coordinates and find intersections with our computed lines.
\begin{lstlisting}[style=MATLAB] 
% Define the image border lines
c1 = [1; 0; -1];     % Left-most column
c500 = [1; 0; -500];  % Right-most column
r1 = [0; 1; -1];      % Top-most row
r500 = [0; 1; -500];  % Bottom-most row

% Compute the intersection between lab and the left-most column
x1 = cross(c1, lab);
x1 = x1 / x1(3); % Convert to Cartesian coordinates for plotting
text(x1(1), x1(2), 'x1', 'FontSize', 12, 'Color', 'b');

% Similarly, for the right-most column
x500 = cross(c500, lab);
x500 = x500 / x500(3);
text(x500(1), x500(2), 'x500', 'FontSize', 12, 'Color', 'b');

% Plot line segment between the intersections
plot([x1(1), x500(1)], [x1(2), x500(2)], 'LineWidth', 3);
\end{lstlisting}
To find the angle between two lines, we use the dot product and calculate the angle $\vartheta$:
\begin{lstlisting}[style=MATLAB] 
% Calculate the angle between lab and lac
l = lab;
m = lac;
cosTheta = (l(1) * m(1) + l(2) * m(2)) / sqrt(sum(l(1:2).^2) * sum(m(1:2).^2));
theta = acosd(cosTheta);
\end{lstlisting}
We verify that any linear combination of points a and b lies on lab:
\begin{lstlisting}[style=MATLAB] 
% Linear combination of points
lambda = rand(1);
mu = 1 - lambda;
p = lambda * a + mu * b;
p' * lab; % Should be zero if p in lab
p = p / p(3); % Convert to Cartesian coordinates
text(p(1), p(2), 'p', 'FontSize', 12, 'Color', 'b');
\end{lstlisting}
Similarly, any linear combination of two lines passes through their intersection:
\begin{lstlisting}[style=MATLAB] 
% Intersection of lines lab and lad
x0 = cross(lab, lad);
lambda = rand(1);
mu = 1 - lambda;
l = lambda * lab + mu * lad;
x0' * l; % Should be zero if x0 in l
\end{lstlisting}
When intersecting parallel lines, the result is a point at infinity.
Here, we check intersections of lines that are theoretically parallel in homogeneous coordinates:
\begin{lstlisting}[style=MATLAB] 
% Intersections of parallel lines lab and lcd
vac = cross(lab, lcd);
vac = vac / vac(3); % Point at infinity

% Intersections of top and bottom borders
vab = cross(r1, r500);
vad = cross(c1, c500);
\end{lstlisting}
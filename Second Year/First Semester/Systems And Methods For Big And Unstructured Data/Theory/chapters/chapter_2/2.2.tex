\section{Documental database}

n traditional relational databases, data is distributed across multiple tables. 
However, in business applications, it is often beneficial to structure the data into a single, cohesive document that pulls together relevant information from various sources. 
This approach provides a more intuitive representation of the data, simplifying query processes by avoiding the need to join multiple tables. 
Additionally, document-oriented databases are highly flexible, allowing for the easy addition of attributes, which makes handling schema changes much more straightforward. 
The document model also closely aligns with object-oriented programming paradigms, effectively resolving the impedance mismatch problem that arises when trying to map objects to relational tables.

\subsection{MongoDB}
AMongoDB is a widely-used, open-source, document-oriented database. 
It stores data in flexible, JSON-like documents, offering developers agility and scalability. With MongoDB, the schema is dynamic, allowing for flexible and evolving data models. 
Furthermore, it supports automatic data sharding, enabling seamless horizontal scaling.
Key advantages of MongoDB include: 
\begin{itemize} 
    \item \textit{General-purpose}: MongoDB offers a rich data model, full-featured indexes, and a sophisticated query language that can handle a wide variety of use cases. 
    \item \textit{Ease of use}: its structure allows for an easy mapping to object-oriented code, with native drivers for popular programming languages. 
        The setup and management process is simple and developer-friendly. 
    \item \textit{Performance and scalability}: MongoDB operates at in-memory speed whenever possible, and its built-in auto-sharding ensures smooth scaling without downtime. 
        Developers can dynamically add or remove capacity as needed. 
\end{itemize}

\paragraph*{Security features} 
SSL encryption between client and server, and intra-cluster communication. 
Fine-grained authorization controls at the database level, supporting read-only, read and write, and administrative roles.

\subsection{Data model}
MongoDB stores data in JSON format, which is ideal for web applications due to its readability and flexibility. 
JSON structures can be easily adapted to various needs, and MongoDB's dynamic schema approach means that new fields can be added without requiring changes to the existing schema.
Key features of the MongoDB data model: 
\begin{itemize} 
    \item MongoDB keeps frequently accessed data in memory to optimize performance. 
    \item It is highly horizontally scalable, allowing servers to be added as needed. 
    \item MongoDB organizes data in contiguous regions for better locality and access speed.
\end{itemize}
Although MongoDB lacks traditional database features such as schemas, transactions, and joins, it is highly optimized for modern application development. 
Large documents can be stored using GridFS, and the maximum document size is 16MB.

\paragraph*{Binary JSON} 
MongoDB uses BSON, a binary representation of JSON documents, to improve speed and efficiency. 
BSON supports more data types than JSON, including date and byte array types, while optimizing space and serialization speed. 
Each document must have a unique identifier, which MongoDB can automatically generate if not provided.
MongoDB also allows for embedding multiple documents within a single document, which simplifies data retrieval and reduces the need for complex joins.














\subsection{Query language}
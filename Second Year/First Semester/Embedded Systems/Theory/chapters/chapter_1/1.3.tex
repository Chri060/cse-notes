\section{Productivity and planning}

The revenue model can be visualized simply as a triangle, where the product's life is represented by a span of 2W, peaking at W. 
The time of market entry defines the triangle, representing market penetration, with the area of the triangle corresponding to total revenue.
Any delay in market entry results in a loss, which is the difference between the areas of the on-time and delayed triangles.

The productivity gap, however, reveals a more challenging situation. 
In theory, increasing the number of designers on a team should reduce project completion time. 
In practice, though, productivity per designer tends to decrease due to the complexities of team management and communication, a phenomenon famously referred to as the mythical man-month (Brooks 1975). 
At a certain point, adding more designers can even extend project timelines, a classic case of too many cooks spoiling the broth.

\subsection{Platform-based design}
To enhance productivity, the design methodology must support reuse, particularly at higher abstraction levels, and this should be backed by standardization. 
As integrated systems increasingly require both digital and non-digital functionalities, this dual trend is captured by the International Technology Roadmap for Semiconductors. 
It emphasizes the miniaturization of digital functions, often referred to as More Moore, and functional diversification, known as More-than-Moore.

\subsection{Technology trends}
Two significant trends in technology development are System-on-Chip (SoC) and System-in-Package (SiP). 
SoC focuses on full integration and achieving the lowest cost per transistor, while SiP focuses on lowering the cost per function for the entire system. 
These architectures are complementary rather than competitive, each requiring distinct industrial approaches and advanced research and design knowledge. 
SoC emphasizes miniaturization, whereas SiP centers on integrating multiple components, necessitating different manufacturing competencies for each.

\paragraph*{MEMS}
Micro-Electro-Mechanical Systems (MEMS) involve the creation of 3D structures using integrated circuit fabrication technologies and specialized micromachining processes, typically on silicon or glass wafers. 
MEMS devices include transducers, microsensors, microactuators, and other mechanically functional microstructures. 
Applications range from microfluidics (valves, pumps, and flow channels) to microengines (gears, turbines, combustion engines).
Integrated microsystems combine circuitry and transducers to perform tasks autonomously or with the assistance of a host computer.
MEMS components bridge the gap between the electrical and non-electrical world, where sensors receive inputs from non-electronic events and actuators output to them.

\paragraph*{Energy scavenging}
Energy scavenging involves capturing energy from objects with temperature gradients. 
Another source of scavenging is vibrations, such as self-winding watches, which generate around 5 microwatts on average when worn and up to 1 milliwatt when shaken vigorously.

\paragraph*{Wireless sensor nodes}
Wireless sensor nodes are small, battery-powered devices that monitor local conditions.
These devices typically have limited resources and form nodes within a wireless network that covers a region or object of interest. 
Wireless sensor nodes enable new applications by collecting, fusing, reasoning, and responding to sensor data. 
These applications can lead to smarter systems in fields ranging from environmental monitoring to industrial automation.
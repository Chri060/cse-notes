\section{Introduction}

Microprocessors are a part of the embedded systems applications implementi in software their algorithms
the main features are: 
\begin{itemize}
    \item \textit{Flexibility}: it is divided into mainteinability and evolution of the application: sofware develoopment is less complex, faster, verification is less critical and software designers are available in volume.
    \item \textit{Time To Market} (TTM): time needed to have a product placeable to the customer.
    \item \textit{Easy upgrade}.
    \item \textit{Cost}: the cost depends on the volume. 
        The design cost is one time only, and the production relies only on the cost of the silicon.
\end{itemize}
This approach is cost effective since usually we dont't need all the power of a processor, but we use only a fraction of it. 
Hence, microprocessors keeps only the needed functions and power, reducing the production cost and the consumption of material. 

Typically an equivalent hardware solution has a better performance than software: microprocessors arechitecture is optimized for flexibility but it is generic, not specialized.
In fact, performance is not only in computing speed, but also in energy, power, memory footprint, chip area, and so on.
Usually, the software implementation is easier than the hardware one, but in case of new requirements it is more simple to modify the hardware than the software and the performance will remain better on harware. 

Designing a software solution requires a deep knowledge of the architecture of the microprocessors available on the market. 
The haracteristics of the problems (algorithm) should drive the designer, beside non functional constraints (performance, cost, development time, power and so on).
Before to select a specific processer, the initial step is the choice of the class of processor to be used and in which form it has to be purchased.




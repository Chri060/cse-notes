\section{Introduction}

Microprocessors play a crucial role in embedded systems by implementing software algorithms that govern their functionality. 
The key features of microprocessor-based applications include:
\begin{itemize}
    \item \textit{Flexibility}: this encompasses maintainability and the ability to evolve the application. 
        Software development becomes less complex and faster, verification processes are less critical, and a larger pool of software designers is available.
    \item \textit{Time To Market} (TTM): this refers to the duration required to bring a product to the customer, emphasizing the importance of efficient development cycles.
    \item \textit{Easy upgrade}: software solutions can be updated and improved with minimal disruption.
    \item \textit{Cost}: the overall cost is volume-dependent. 
        Design costs are incurred only once, while production expenses are primarily linked to the cost of silicon.
\end{itemize}
This approach is cost-effective since embedded systems often do not require the full processing power of a microprocessor; they typically utilize only a fraction of it. 
Consequently, microprocessors retain only the necessary functions and capabilities, resulting in reduced production costs and material consumption.

While an equivalent hardware solution may generally provide better performance, microprocessor architectures are optimized for flexibility, making them more generic and less specialized. 
Performance encompasses various factors beyond computing speed, such as energy consumption, power efficiency, memory footprint, and chip area. 
While software implementations are typically easier to develop than hardware solutions, modifying hardware to meet new requirements is often simpler, with performance advantages remaining on the hardware side.

Designing a software solution necessitates a comprehensive understanding of the microprocessor architectures available in the market. 
The characteristics of the problem (algorithm) should guide the designer, alongside non-functional constraints such as performance, cost, development time, and power consumption. 
The initial step before selecting a specific processor is determining the appropriate class of processor and the form in which it will be acquired.
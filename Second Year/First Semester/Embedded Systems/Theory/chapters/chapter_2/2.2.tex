\section{Processors taxonomy}

Processors can be categorized into two main types:
\begin{itemize}
    \item \textit{Application Specific Processors} (ASP): these processors are tailored for specific classes of applications that require high performance, handling numerous operations per second.
    \item \textit{General Purpose Processors} (GPP): known as versatile four-season processors, GPPs are not optimized for any particular application, making them suitable for various types of tasks.
\end{itemize}
In typical embedded systems, a multi-processor solution often employs GPPs for supervising and controlling the activities of one or more ASPs.

\paragraph*{Processors availability}
Processors can further be classified as follows:
\begin{itemize}
    \item \textit{Components Off The Shelf} (COTS): these are standard chips purchased off-the-shelf and mounted onto a printed circuit board (PCB) along with the necessary interfaces to integrate with the rest of the system.
    \item \textit{Intellectual Property} (IP): this involves purchasing the design specifications of a microprocessor. 
        There are several abstraction levels in IP:
        \begin{itemize}
            \item \textit{Soft-macro}: the microprocessor is described using Hardware Description Language (HDL) at the register-transfer (RT) level.
            \item \textit{Hard-macro}: the description includes details down to the layout level.
        \end{itemize} 
\end{itemize}
With the increasing prevalence of Programmable Logic Devices (PLDs), the use of IP is becoming more popular.
Suppliers of Complex Programmable Logic Devices (CPLDs) and/or Field Programmable Gate Arrays (FPGAs) often provide one or more cores, sometimes even for free.

\subsection{Selection process}
The selection process is based on: 
\begin{itemize}
    \item \textit{Class}: the nature of the algorithm and the operations and data to be processed are the primary drivers for selecting a processor class.
    \item \textit{Form}: the target architecture of the system plays a crucial role. 
    \item \textit{Performance}: a key metric for performance is the average number of instructions per clock cycle (IPC/CPI), which is relative to the clock frequency and should be scaled for comparison across different architectures. 
        MIPS (Million Instructions Per Second) serves as an absolute measure of throughput but can be misleading when comparing processors with different Instruction Sets (ISAs).
        For floating-point operations, MFLOPS is commonly used, while specialized architectures like Digital Signal Processors (DSPs) often use MMACS (Million Multiply-Accumulate Operations Per Second), and Network Processors (NPs) typically measure the average number of processed packets per time unit.
    \item \textit{Power}: power efficiency is perhaps the most critical driver for embedded systems. 
        Both average power and peak power are important metrics for estimating the maximum or average power consumption of the overall system. 
        Initially, a combined measure of power and speed (related to performance) is often used.
\end{itemize}
Other important considerations include:
\begin{itemize}
    \item \textit{Memory}: the bandwidth and size requirements of the application can impose hard constraints. 
        In some cases, only internal memory may be available, making external memory utilization impractical. 
        Achieving sufficient bandwidth may require integrated memory solutions, and for complex systems, addressing space can be a critical factor.
    \item \textit{Peripherals}: embedded systems often process external signals and control physical devices. 
        Designers can either choose an architecture focused on computation and design the rest of the system accordingly or opt for a single-chip solution that integrates computing, interfacing, and peripherals. 
        Selecting the appropriate microprocessor can necessitate using a PCB or a System on Chip (SoC) that consolidates all peripherals. 
        Using a microcontroller simplifies integration challenges, albeit with fewer alternatives available.
    \item \textit{Software}: many embedded applications have limited legacy code and primarily utilize standard functions and libraries. 
        The availability of libraries can simplify both the design and validation processes, making the development of software solutions feasible that might otherwise be impractical. 
        Access to an operating system and SDK specific to the processor is also vital, serving as the foundation for software development. 
        Differences among SDKs are significant, and factors such as the software compilation flow, code quality, and flexibility offered to the designer are important. 
        The availability and reliability of analysis tools, including debuggers, as well as documentation and reference designs, contribute to the ecosystem surrounding the microprocessor.
    \item \textit{Packaging}: for COTS processors, various packages are available, differing in size, pinout, and materials.
    \item \textit{Certifications}: different certification standards exist, including consumer, industrial, aerospace, automotive, and military certifications. 
        Some certifications are tailored for specific fields, such as MISRA rules for automotive code. 
        The availability of a component with the appropriate certification can be a critical constraint for its adoption.
\end{itemize}
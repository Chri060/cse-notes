\section{Processors taxonomy}

The processors can be cathegorized into: 
\begin{itemize}
    \item \textit{Application Specific Processors} (ASP): tailored for specific classes of applications that works with a lot of operations per seconds.
    \item \textit{General Purpose Processors} (GPP): four season processor, not optimized for any specific applications.
        Versatile, it can be used for different type of applications.
\end{itemize}
The embedded systems multi-processors Typical solution is to use GPP for supervision and control of the
activities of one or more ASPs

The processors can be: 
\begin{itemize}
    \item \textit{Components Off The Shelf} (COTS): just buy a chip off-the-shelf and mount in on a board (PCB) together with the interfaces to the rest of the systems
    \item \textit{Intellectual Property} (IP): the design (description) of the micro is purchased. 
        Several abstraction levels: 
        \begin{itemize}
            \item \textit{Soft-macro}: the micro is described in HDL at RT-level.
            \item \textit{Hard-macro}: description down to the level of layout.
        \end{itemize} 
\end{itemize}
Give the diffusion of Programmable Logic Devices (PLD), the used of IP is becoming more and more popular.
Typically the suppliers of CPLD and/or FPGA make available one or more cores (even for free). 

\subsection{Selection process}
The selection process is based on: 
\begin{itemize}
    \item \textit{Class}: the main driver is the nature of the algorithm, the operations and data to be elaborated.
    \item \textit{Form}: the main driver is the target architecture of the systems. 
        But performance and power, in addition to royalties, can make the difference.
    \item \textit{Performance}: ain metric is the average number of instructions per clock cycle (IPC/CPI). 
        It is relative to the clock frequency, hence it need to be scaled to compare different architectures.
        MIPS is absolute measure of the throughput. 
        It can be misleading when processors have different Instruction Sets (ISA). 
        MFLOPS is the same but for Floating Points operations, for more specific architectures, like DSP or NP, metrics can change
        DSP, frequently it is used MMACS.
        NP, average number of processed packets in a time unit.
    \item \textit{Power}: probably the most important driver for embedded systems. 
        Average power or peak power, useful to roughly estimate the max power or the avg power of the overall systems.
        Frequently, especially at the beginning, it is used a joint measure between power and speed (related to performance).
        These information are very useful but must be properly used during the design process.
\end{itemize}
Other important elements to be considered are:
\begin{itemize}
    \item \textit{Memory}: bandwidth and size of the application can be a hard constraint.
        In some case it could be impossible to use external memory only that internal to the micro is available.
        Maybe that some bandwidth can be achieved only by integrated memories.
        For complex systems, the crucial point could be the addressing space.
    \item \textit{Peripherals}: an embedded systems typ elaborates external signals and control physical apparatus. 
        It is possible to select and architecture oriented toward the computing, and to design the rest of the system, or to look for a single chip solution integrating computing, interfacing and some peripherals
        The choice of the micro require a PCB or a SoC integrating all the peripherals under the responsibility of the designer.
        Using a microcontroller, the designer get rid of the integration problem, paid in terms of reduction in the number of alternatives.
    \item \textit{Software}: Many embedded applications have a limited legacy code, most of them exploit standard functions and libraries
        The availability of libraries simplifies both design and validation of the applications, making feasible in some cases the development of sw solutions otherwise impractical.
        The availability of an operating systems and SDK for the specific processor is important, too.
        It is the cornerstone of the software development, many differences exist among the various SDKs. 
        Important aspects to be evaluated are the software compilation flow, the quality of the code generated and the flexibility and knobs offered to the designer
        Concerning the analysis tools, the richness, their accuracy and reliability are important. 
        Debuggers should be carefully analyzed
        Documentation and reference design
        Community end ecosystem.
    \item \textit{Packaging}: if the micro is a COTS, different packages are available: size, pinout, material (e.g. plastic, metal), certification (consumer, industrial, aerospace, and so on).
    \item \textit{Certifications}: there exists several certification: Consumer, industrial, aerospace, automotive, military, and so on.
        Some are specifically tailored for narrow fields (e.g. MISRA rules for the code for automotive).
        The availability of a component with a proper certification could be a must constraint for its adoption.
\end{itemize}
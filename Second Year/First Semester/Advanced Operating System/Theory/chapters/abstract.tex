\begin{abstract}
    This course provides an overview of key topics related to operating systems, focusing on design patterns, resource management, and peripheral interaction. 
    It begins by introducing the main goals of operating systems, detailing design patterns and mechanisms for mediating and regulating access to system resources. 
    It also categorizes operating systems into different types, such as monolithic, microkernel, hybrid, and uni-kernel, with examples for each.

    The section on resource management and concurrency covers support for multi-process and multi-threaded execution, CPU scheduling for both single and multiprocessors, and modern load-balancing techniques for NUMA systems.
    Topics such as memory consistency, multi-threaded synchronization, advanced locking techniques, lockless programming, and inter-process communication (IPC) primitives are discussed in detail. 
    Additionally, asynchronous programming, deadlock, starvation, virtual memory management, and the basics of software and hardware virtualization, including containers, are explored.
    
    In peripheral and persistence management, the document addresses low-level peripheral access, communication buses like PCI, programmable interrupt controllers, and interrupt management. 
    Topics include character-based and block-based I/O, the development of device drivers, and an overview of modern storage devices (e.g., HDDs, SSDs) and file systems, emphasizing the filesystem-centric view of peripherals.
    
    Lastly, run-time support is discussed, including boot loaders, kernel initialization, device discovery mechanisms (ACPI and device trees), C runtime support, the structure of dynamic libraries, and linking. 
    Tools for the development, analysis, and profiling of embedded code are also reviewed.
\end{abstract}
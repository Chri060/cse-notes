\section{Operating systems architectures}

The design of an operating system can adopt a hybrid of different architectural approaches, including: bare metal, monolithic (with modules), micro-kernel, hybrid, and library.

\subsection{Bare metal}
Bare metal programming is typically employed in scenarios where there is a single-purpose application that demands high control of the hardware along with strict timing requirements. 
This approach is also favored when low power consumption is essential, and there is no need for abstractions such as tasks.

\subsection{Monolithic} 
In a monolithic architecture, there is a single large kernel binary. 
Device drivers and the kernel are part of the same executable and reside in the same memory area. 
Examples include Linux, Embedded Linux, AIX, HP-UX, Solaris, and *BSD.

\paragraph*{Monolithic with modules} 
This variation of the monolithic architecture includes only a subset of core components within the kernel. 
Additional services are implemented via external modules, which can be dynamically linked on demand at runtime.

\subsection{Microkernel}
All non-essential components of the kernel are implemented as processes in user space. 
A single small kernel provides minimal process and memory management, as well as communication facilities through message passing.

Due to its asynchronous nature, a crash of a system process does not necessarily result in a crash of the entire system.

Service invocation occurs between user-level client/server programs through message passing. 
Examples of such systems include SeL4, GNU Hurd, MINIX, and MkLinux.

\subsection{Hybrid}
Hybrid architectures are similar to microkernels but include some additional code in kernel space to enhance performance.
Some services, such as the network stack or filesystem, run in kernel space, while device drivers operate in user space. 
Examples include Windows NT, 2000, XP, Vista, 7, 8, 8.1, 10, and macOS.

\subsection{Library amd unikernels}
In library operating systems, services such as networking are provided in the form of libraries that are compiled with the application and configuration code. 
A unikernel is a specialized, single address space machine image that can be deployed in cloud or embedded environments (RTOSes).
Examples include FreeRTOS, IncludeOS, and MirageOS.
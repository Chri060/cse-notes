\section{Binary Decision Diagram}

The core idea behind ROBDDs is to avoid the inefficiencies of sequential sub-case exploration by instead storing sub-cases in memory, allowing for faster retrieval and processing.
This approach relies on two crucial hashing mechanisms: a unique table and a computed table. 
The unique table identifies and consolidates identical sub-cases to prevent redundancy, while the computed table stores the results of previously computed sub-cases to reduce redundant calculations.

ROBDDs represent logic functions as Directed Acyclic Graphs (DAGs), which often provide a more compact form than traditional Sum of Products (SOP) expressions. 
With careful structuring, ROBDDs can be made canonical, meaning they provide a unique representation of a function. 
This can shift the focus in Boolean reasoning from solving SAT problems to efficiently managing function representation.

Another significant advantage is the efficiency of Boolean operations on BDDs. 
Many operations, such as checking for tautology or computing complements, can be performed quickly—often in linear time relative to the result's size or even in constant time. 
However, the size of a BDD is critically influenced by variable ordering, with the right orderings resulting in significantly smaller, more manageable diagrams.

\paragraph*{Directed Acyclic Graph representation}
In an ROBDD, the logic function is represented by a Directed Acyclic Graph (DAG) with a structure rooted in a single root node and terminating in two terminal nodes, labeled 0 and 1. 
Each internal node in the graph is associated with a variable and has exactly two children. 
The DAG's structure is based on a Shannon co-factoring tree, modified to be both reduced and ordered, creating the canonical form known as ROBDD: 
\begin{itemize}
    \item \textit{Reduction}: this process simplifies the graph by eliminating redundancy: if a node has two identical children, it is removed, else if two nodes have identical subgraphs, they are merged into a single node.
    \item \textit{Ordering}: co-factoring (splitting) variables are processed in a consistent, predefined order across all paths, ensuring that each path from the root to any terminal visits variables in ascending order, typically denoted $x_1<\dots<x_n$.
\end{itemize}

An Ordered Binary Decision Diagram (OBDD) applies only the ordering rule, while a Reduced Ordered Binary Decision Diagram (ROBDD) applies both ordering and reduction. 
The ROBDD's reduction rules are:
\begin{enumerate}
    \item If a node's two children are identical, the node is removed, effectively reducing the function to $f = vf + \bar{v}f$. 
    \item If two nodes have isomorphic graphs, they are replaced by a single instance, so each node uniquely represents a distinct logic function.
\end{enumerate} 
The onset of the function represented by an ROBDD can be identified by tracing all paths leading to the 1-terminal node. 
This set of paths corresponds to a cover of pairwise disjoint cubes, providing an efficient and compact representation of the function without needing to explicitly enumerate every path.
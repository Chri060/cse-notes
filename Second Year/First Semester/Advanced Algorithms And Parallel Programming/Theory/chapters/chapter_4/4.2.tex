\section{Hiring problem}

Imagine you need to hire a new employee, and a headhunter sends you one applicant per day for $n$ days. 
If an applicant is better than the current employee, you fire the current one and hire the new applicant. 
Since both hiring and firing are costly, you are interested in minimizing these operations.

We may have two estreme cases: 
\begin{itemize}
    \item \textit{Worst-case scenario}: the headhunter sends applicants in increasing order of quality, meaning each new applicant is better than the previous one. 
        In this case, you hire and fire each applicant, resulting in $n$ hires.
    \item \textit{Best-case scenario}: the best applicant arrives on the first day, so you hire them and make no further changes. 
        The total cost is just one hire.
\end{itemize}
In the average case, the input to the hiring problem is a random ordering of $n$ applicants. 
There are $n!$ possible orderings, and we assume that each is equally likely (though other distributions could be considered). 
We want to compute the expected cost of our hiring algorithm, which in this case is the expected number of hires.

Let $X(s)$ be the random variable representing the number of applicants hired given the input sequence $s$. 
We are interested in $\mathbb{E}[X]$, the expected number of hires.
To solve this, we can break the problem down using indicator random variables. 
Instead of counting the total number of hires with a single random variable, we define $n$ indicator variables—one for each applicant.

Let $X_i$ be the indicator variable for whether applicant $i$ is hired:
\[X_i=\begin{cases}
    1\qquad\text{if applicant }i\text{ is hired} \\
    0\qquad\text{otherwise}
\end{cases}\]
The total number of hires, $X$, is the sum of these indicator variables:
\[X=X_1+X_2+\cdots+X_n\]
Now, using the linearity of expectation, we have:
\[\mathbb{E}[X]=\mathbb{E}\left[\sum X_i\right]=\sum\mathbb{E}[X_i]\]
Next, we need to compute $\mathbb{E}[X_i]$, the probability that applicant $i$ is hired. 
An applicant $i$ is hired only if they are better than all previous applicants ($i-1$ applicants). 
For a uniformly random order of applicants, the probability that applicant $i$ is better than the previous $i-1$ applicants is $\frac{1}{i}$.

Thus, the expected value for each $X_i$ is:
\[\mathbb{E}[X_i]=\Pr(\text{applicant }i\text{ is hired})=\dfrac{1}{i}\]
Finally, the expected total number of hires is:
\[\mathbb{E}[X]=\sum_{i=1}^n\dfrac{1}{i}\]
This sum is the harmonic series, which is bounded by $\ln n + 1$. 
Therefore, the average number of hires is approximately $\ln n$, which is significantly better than the worst-case scenario of $n$ hires.

\subsection{Active randomization}
The analysis above assumes that the headhunter sends applicants in a random order. 
However, if the headhunter is biased you cannot rely on this randomness. 
In such cases, if you have access to the entire list of applicants in advance, you can take control by randomizing the input yourself.

By randomly permuting the list of applicants before interviewing them, you essentially convert the hiring problem into a randomized algorithm. 
This way, the hiring process no longer depends on the headhunter's input order, and you maintain the same expected number of hires, $O(\log n)$, regardless of the original order.

In general, randomized algorithms allow for multiple possible executions on the same input, which ensures that no single input can guarantee worst-case performance. 
Instead of assuming some distribution for the inputs, you actively create your own distribution, thereby moving from passive probabilistic analysis to a more robust, actively randomized approach. 
This is especially useful when facing potentially adversarial or biased inputs.

\subsection{Summary}
In summary, the hiring problem demonstrates the effectiveness of probabilistic analysis and randomized algorithms. 
By leveraging indicator variables and the linearity of expectation, we can show that, on average, the number of hires is much lower than in the worst case. 
Moreover, through active randomization, you can control the input order and reduce the number of hires to $O(\log n)$, making the process much more efficient in practice.
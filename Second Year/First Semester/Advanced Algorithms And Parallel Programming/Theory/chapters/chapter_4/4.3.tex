\section{Randomized algorithms}

Randomized algorithms can be broadly classified into two main types: Las Vegas algorithms and Monte Carlo algorithms. 
These categories are distinguished by whether the randomness affects the correctness of the solution or the efficiency of the algorithm.

\subsection{Las Vegas algorithms}
An example of a Las Vegas algorithm is a randomized sorting algorithm, where randomness influences the algorithm's performance, but it always produces the correct solution. 
The only aspect that varies between different runs of the algorithm is its running time.
\begin{itemize}
    \item \textit{Correctness}: guaranteed to produce the correct result every time.
    \item \textit{Randomness}: affects only the running time, not the output.
\end{itemize}
For example, Randomized Quicksort is a Las Vegas algorithm. 
It randomly selects a pivot at each step to partition the array, and the sorting process continues recursively. 
Although the running time can vary depending on the choice of pivots, the final output will always be a correctly sorted array.

\subsection{Monte Carlo algorithms}
In contrast, the min-cut algorithm (such as Karger's Min-Cut Algorithm) is an example of a Monte Carlo algorithm, where randomness affects the correctness of the solution. 
The algorithm might return an incorrect answer (i.e., it may fail to find the minimum cut), but we can bound the probability of this happening. 
By running the algorithm multiple times, we can reduce the probability of error to a desired level.
\begin{itemize}
    \item \textit{Correctness}: may return an incorrect solution with a known probability.
    \item \textit{Randomness}: affects both the running time and the correctness of the output.
\end{itemize}
For instance, Karger's min-cut algorithm works by randomly contracting edges in the graph until only two vertices remain. 
The cut between these two vertices is the candidate for the minimum cut. 
However, due to the randomness involved, the algorithm might not find the actual minimum cut, though it can do so with high probability if repeated enough times.

\subsection{Comparison}
Las Vegas Algorithms always give the correct solution, but running time may vary due to randomness. 
The focus is on ensuring that the algorithm always terminates with the correct result.

Monte Carlo Algorithms may produce an incorrect result with some probability, but this error can be controlled and reduced. 
These algorithms typically have a fixed running time but sacrifice certainty for speed.

Both types of algorithms highlight different ways to incorporate randomness. 
In Las Vegas algorithms, randomness is used for performance improvement without compromising correctness, whereas in Monte Carlo algorithms, randomness is used to trade some degree of correctness for efficiency.
\section{Introduction}

In probabilistic analysis, the algorithm is deterministic, meaning that for any given fixed input, the algorithm will always produce the same result and follow the same execution path each time it runs. This analysis relies on a specific technique:
\begin{itemize}
    \item A probability distribution is assumed for the inputs.
    \item The behavior of the algorithm is then analyzed over this distribution, focusing on the item of interest (such as runtime or accuracy).
\end{itemize}
However, there are important caveats to this approach:
\begin{itemize}
    \item Certain specific inputs may result in significantly worse performance.
    \item If the assumed distribution of inputs is inaccurate, the analysis may present a misleading or overly optimistic view of the algorithm's behavior.
\end{itemize}

In contrast, randomized algorithms introduce randomness into their execution. 
For a fixed input, the outcome may vary depending on the results of internal random decisions. 
The key features of randomized algorithms are:
\begin{itemize}
    \item They generally work well with high probability for any input.
    \item There is, however, a small chance that they may fail on any given input, though this probability is low.
\end{itemize}

\subsection{Analysis tools}
Key analysis tools for randomized algorithms are: 
\begin{itemize}
    \item \textit{Indicator variables}: to analyze a random variable $X$, which represents a combination of many random events, we can break it down using indicator variables. 
        Let $X_i$ be an indicator variable that captures the outcome of an individual event. 
        The overall random variable is then expressed as the sum of these indicator variables:
        \[X = \sum X_i\]
    \item \textit{Linearity of expectation}: one powerful tool in the analysis of randomized algorithms is the linearity of expectation. 
        Suppose we have random variables $X$, $Y$, and $Z$, where $X$ is the sum of $Y$ and $Z$. 
        The expected value of $X$ is then the sum of the expected values of $Y$ and $Z$:
        \[\mathbb{E}[X] = \mathbb{E}[Y + Z] = \mathbb{E}[Y] + \mathbb{E}[Z]\]
        This holds true regardless of whether the variables are independent, making it a versatile and widely applicable tool in randomized algorithm analysis.
    \item \textit{Recurrence relations}: recurrence relations often arise in the analysis of algorithms, especially when dealing with recursive procedures. 
        These relations describe the behavior of an algorithm in terms of smaller subproblems and are fundamental in understanding the overall performance of both deterministic and randomized algorithms.
\end{itemize}
\section{Power of a number}

The problem is the computation of the value of $a^n$, where $n\in\mathbb{N}$. 
The naive approach consists in multiplying $n$ times the value of $a$, with a total complexity of $\Theta(n)$. 

We can apply a divide an conquer algorithm also to this problem, by dividing the power by two in the following way: 
\[a^n=\begin{cases}
    a^\frac{n}{2}\cdot a^\frac{n}{2}  \:\:\qquad\qquad \text{if }n\text{ is even} \\
    a^\frac{n-1}{2}\cdot a^\frac{n-1}{2} \cdot a \qquad \text{if }n\text{ is odd}
\end{cases}\]
In this scenario, we have that the divide and the combination phase have a constant complexity since can be performed with a single division or a single multiplication, respectively. 
We have half input dimension at each iteration step, and we have exactly one sub-problem to solve (since we have two equal sub-problems). 
Thus, the final recurrent formula for complexity is: 
\[T(n)=T\left(\dfrac{n}{2}\right)+\Theta(1)\]
By applying the master method, we obtain a final complexity of $\Theta(\log_2n)$, that is lower than the naive situation. 
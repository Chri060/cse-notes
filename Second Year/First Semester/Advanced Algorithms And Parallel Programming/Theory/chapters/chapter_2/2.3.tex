\section{Power of a number}

The problem at hand is to compute the value of $a^n$, where $n\in\mathbb{N}$. 
The naive approach involves multiplying $a$ by itself $n$ times, resulting in a total complexity of $\Theta(n)$. 

We can also use a divide and conquer algorithm to solve this problem by dividing the exponent by two, as follows:
\[a^n=\begin{cases} a^\frac{n}{2}\cdot a^\frac{n}{2}  \:\:\qquad\qquad \text{if }n\text{ is even} \\ a^\frac{n-1}{2}\cdot a^\frac{n-1}{2} \cdot a \qquad \text{if }n\text{ is odd} \end{cases}\]
In this approach, both the divide and combine phases have a constant complexity, as they involve a single division and a single multiplication, respectively. 
Each iteration reduces the problem size by half, and we solve one sub-problem (with two equal parts).

Thus, the recurrence relation for the complexity is:
\[T(n)=T\left(\dfrac{n}{2}\right)+\Theta(1)\]
By applying the master method, we find a final complexity of $\Theta(\log_2n)$, which is significantly more efficient than the naive approach.
\section{Introduction}

The divide and conquer design paradigm consists in the following parts: 
\begin{enumerate}
    \item Divide the problem into sub-problems. 
    \item Conquer the sub-problems by solving them recursively. 
    \item Combine the sub-problems solution. 
\end{enumerate}
This procedure allow to have a more problems with less input length that can be solved in less time. 

The divide step is constant since we have to simply split an array in two parts of equal length. 
The second step depends on the algorithm that is going to be analyzed.
The combine step depends again on the algorithm: it can either constant or take some time. 

\paragraph*{Merge sort}
The merge sort considered before is performed with the following steps: 
\begin{itemize}
    \item \textit{Divide}: the array is trivially divided in two sub-arrays. 
    \item \textit{Conquer}: the two sub-arrays are recursively sorted. 
    \item \textit{Combine}: the two sub-arrays are merged in a linear time. 
\end{itemize}
The recursive expression for the complexity of this algorithm is then divided into the following components: 
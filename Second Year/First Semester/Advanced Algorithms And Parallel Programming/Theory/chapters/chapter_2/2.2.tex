\section{Binary search}

The binary search problem involves locating an element within a sorted array. 
This can be efficiently solved using the divide and conquer approach, outlined as follows:
\begin{enumerate}
    \item \textit{Divide}: check the middle element of the array.
    \item \textit{Conquer}: recursively search within one of the sub-arrays.
    \item \textit{Combine}: if the element is found, return its index in the array.
\end{enumerate}
n this scenario, we only have one sub-problem, which is the new sub-array, and its length is half that of the original array.
Both the divide and combine steps have a constant complexity.

Thus, the final expression for the complexity is:
\[T(n)=1T\left(\dfrac{n}{2}\right)+\Theta(1)\]

Here, we have $a=1$ and $b=2$, leading to:
\[n^{\log_ba}=n^{\log_21}=n^0=1\]
We can apply the second case of the master method with $k=0$, resulting in a final complexity of:
\[T(n)=\Theta(\log n)\]
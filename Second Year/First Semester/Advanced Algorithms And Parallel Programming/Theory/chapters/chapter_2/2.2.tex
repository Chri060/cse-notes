\section{Binary search}

The problem of binary search consists in finding an element in a sorted array. 
This problem can be solved in a divide and conquer manner in the following way: 
\begin{enumerate}
    \item \textit{Divide}: check the middle element of the array.
    \item \textit{Conquer}: recursively search one sub-array. 
    \item \textit{Combine}: if the result is found, return the index of the element in the array. 
\end{enumerate}
In this case we have one sub-problem, that is the new sub-array. 
The sub-array has a length that is half the length of the original problem. 
And the complexity of the divide and combine steps is constant.
Therefore, the final expression for the complexity is: 
\[T(n)=1T\left(\dfrac{n}{2}\right)+\Theta(1)\]

In this case we have that $a=1$, and $b=2$, and so we have that: 
\[n^{\log_ba}=n^{\log_21}=n^0=1\]
We can then use the second case of the master method with $k=0$, obtaining a final complexity of $T(n)=\Theta(\log n)$. 
\section{Prefix sum}

Given a sequence of values $\{a_1,\dots,a_n\}$, the prefix sum $S_i$ up to position $i$ is defined as:
\[S_i=\sum_{j=1}^ia_j\]
In the case of prefix sums, the total computational work required by a parallel algorithm exceeds that of a serial algorithm.

For a serial algorithm, computing each prefix sum is straightforward: each element in the prefix sum can be computed in sequence, where $S_i$ simply depends on $S_{i-1}$ and $a_i$. 
This approach only requires $\mathcal{O}(n)$ operations, with each element added once.

In contrast, a parallel algorithm introduces additional overhead. 
To achieve parallelism, the algorithm needs to divide the work among processors, requiring intermediate calculations and combining steps.
Thus, the parallel prefix sum algorithm typically involves $\mathcal{O}(n\log n)$ operations, as it requires multiple rounds to propagate intermediate results across processors.
\section{Matrix multiplication}

Matrix multiplication involves taking two matrices $A$ and $B$ as input and producing a resulting matrix $C$, which is their product.
Each element of the matrix $C$ is computed as follows:
\[c_{ij}=\sum_{k=1}^{n}a_{ik}\cdot b_{kj}\]
The standard algorithm for matrix multiplication is outlined below:
\begin{algorithm}[H]
    \caption{Standard matrix multiplication}
        \begin{algorithmic}[1]
            \For{$i = 1$ \textbf{to} $n$} 
                \For{$j = 1$ \textbf{to} $n$} 
                    \State $c_{ij} = 0$
                    \For{$k = 1$ \textbf{to} $n$} 
                        \State $c_{ij} = c_{ij}+a_{ik} b_{kj}$
                    \EndFor
                \EndFor
            \EndFor
        \end{algorithmic}
\end{algorithm}
The complexity of this algorithm, due to the three nested loops, is $\Theta(n^3)$.

\paragraph*{Divide and conquer}
For the divide and conquer approach, we divide the original $n\times n$ matrix into four $\frac{n}{2}\times\frac{n}{2}$ sumatrices: 
\[\begin{bmatrix} r & s \\ t & u \end{bmatrix}=\begin{bmatrix} a & b \\ c & d \end{bmatrix} \cdot \begin{bmatrix} e & f \\ g & h \end{bmatrix}\]
This requires solving the following system:
\[\begin{cases} r = ae + bg \\ s = af + bh \\ t = ce + dg \\ u = cf + dh \end{cases}\]
This results in a total of eight multiplications and four additions of the submatrices. 
The recursive part of the algorithm involves the matrix multiplications. 
The time complexity can be expressed as $T(n)=8T\left(\frac{n}{2}\right)+\Theta(n^2)$.
Using the master method, we find that the total complexity remains 
\[\Theta(n^3)\]

\paragraph*{Strassen}
To improve efficiency, Strassen proposed a method that reduces the number of multiplications from eight to seven matrices.
This is achieved using the following factors:
\[\begin{cases} P_1=a\cdot(f-h) \\ P_2=(a+b)\cdot h \\ P_3=(c+d)\cdot e \\ P_4=d\cdot(g-e) \\ P_5=(a+d)\cdot(e+h) \\ P_6=(b-d)\cdot(g+h) \\ P_7=(a-c)\cdot(e+f) \end{cases}\]
Using these products, we can compute the elements of the resulting matrix:
\[\begin{cases} r=P_5+P_4-P_2+P_6 \\ s=P_1+P_2 \\ t=P_3+P_4 \\ u=P_5+P_1-P_3-P_7 \end{cases}\]
This approach requires seven multiplications and a total of eighteen additions and subtractions.

The divide and conquer steps are as follows:
\begin{enumerate}
    \item \textit{Divide}: partition matrices $A$ and $B$ into $\frac{n}{2}\times\frac{n}{2}$ submatrices and formulate terms for multiplication using addition and subtraction.
    \item \textit{Conquer}: recursively perform seven multiplications of $\frac{n}{2}\times\frac{n}{2}$ submatrices.
    \item \textit{Combine}: construct matrix $C$ using additions and subtractions on the $\frac{n}{2}\times\frac{n}{2}$ sumbatrices. 
\end{enumerate}
The recurrence relation for the complexity is: $T(n)=7T\left(\frac{n}{2}\right)+\Theta(n^2)$
By solving this recurrence with the master method, we obtain a complexity of:
\[\Theta\left(n^{\log_27}\right)\approx \Theta\left(n^{2.81}\right)\]

Although $2.81$ may not seem significantly smaller than $3$, the impact of this reduction in the exponent is substantial in terms of running time.
In practice, Strassen's algorithm outperforms the standard algorithm for $n \geq 32$.

The best theoretical complexity achieved so far is $\Theta\left(n^{2.37}\right)$, although this remains of theoretical interest, as no practical algorithm currently achieves this efficiency.
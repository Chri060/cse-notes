\section{Introduction}

The divide and conquer design paradigm consists of three key steps:
\begin{enumerate}
    \item Divide the problem into smaller sub-problems. 
    \item Conquer the sub-problems by solving them recursively. 
    \item Combine the solutions of the sub-problems.
\end{enumerate}
This approach enables us to tackle larger problems by breaking them down into smaller, more manageable pieces, often resulting in faster overall solutions.

The divide step is typically constant, as it involves splitting an array into two equal parts. 
The time required for the conquer step depends on the specific algorithm being analyzed.
Similarly, the combine step can either be constant or require additional time, again depending on the algorithm.

\paragraph*{Merge sort}
The merge sort algorithm, previously discussed, follows these steps:
\begin{itemize}
    \item \textit{Divide}: the array is split into two sub-arrays.
    \item \textit{Conquer}: each of the two sub-arrays is sorted recursively.
    \item \textit{Combine}: the two sorted sub-arrays are merged in linear time.
\end{itemize}
The recursive expression for the complexity of merge sort can be expressed as follows:
\[T(n)=\underbrace{2}_{\#\text{subproblems}} \underbrace{T\left(\dfrac{n}{2}\right)}_{\text{subproblem size}}+\underbrace{\Theta(n)}_\text{work dividing and combining}\]
\section{Pthreads}

Thread management
Creation
Joining
Detaching
Set/query attributes
Synchronization
Mutexes: Control access to mutual exclusion
sections
Condition variables: Control conditional interthreads communications
Once created, threads are peers, and may create
other threads: there is no implied hierarchy or
dependency between threads.
The maximum number of threads depends on the
implementation

\subsection{Thread creation}
The creation of a thread is: 
\begin{lstlisting}[style=C]
int pthread_create(pthread_t * thread, const pthread_attr_t * attr, void * (* start_routine) (void *), void *arg)
\end{lstlisting}
Here: 
\begin{itemize}
    \item thread: identifier for the new thread returned by
    the subroutine
    attr: used to set thread attributes
    Joinable/detached
    Scheduling
    Stack size
    start_routine: the C routine that the thread will
    execute once it is created.
    arg: argument passed to start_routine. It must be
    passed by address as a pointer cast of type void
\end{itemize}

\subsection{Thread elimination}
The thread returns from its starting routine
The thread makes a call to the pthread_exit
subroutine
The thread is canceled by another thread via the
pthread_cancel routine
The entire process is terminated
• (Created threads survive if main calls
pthread_exit)

\subsection{Threads join}
The joining of a thread is: 
\begin{lstlisting}[style=C]
int pthread_join(pthread_t thread, void ** retval)
\end{lstlisting}
Here: 
\begin{itemize}
    \item pthread_join blocks the calling thread until the
    specified thread terminates
    retval will contain the copy of the argument of
    pthread_exit
\end{itemize}
Threads can be created as joinable or detached
A joinable thread can become detached (not vice
versa)
Explicitly set joinable or detached attribute
Default attribute should be joinable
Not all implementations adopt this default
Detached threads may consume less resources

\paragraph*{Barriers}
We can use barriers for join operations: 
\begin{lstlisting}[style=C]
    int pthread_barrier_init(pthread_barrier_t * barrier, pthread_barrierattr_t * attr, unsigned int count)

    int pthread_barrier_wait(pthread_barrier_t * barrier)
\end{lstlisting}
pthread_join requires to wait for a specific
thread exit
Several threads can be synchronized among them
through barriers
all the threads of the group need to wait for all
the other threads, then they can resume
working
count is the number of threads to be waited
Not all implementations include barriers!

\subsection{Thread mutex}
Mutex (mutual exclusion) variables are the basic
method to protect shared data when multiple
writes occur
Only one thread can lock a mutex variable at any
given time
If several threads try to lock a mutex only one
thread will be successful
\begin{lstlisting}[style=C]
    pthread_mutex_lock(&my_lock);
    /* critical section */
    pthread_mutex_unlock(&my_lock);
\end{lstlisting}
Threads that could not acquire the mutex are
blocked
Less blocking overhead: trylock
The thread can keep on working while the lock
is not available
Multiple mutexes can lead to deadlocks
Three types of mutexes:
Normal
Recursive
Error check

\paragraph*{Condition variables}
Mutexes implement synchronization by serializing
data accesses
Condition variables allow threads to synchronize
explicitly by signaling the meeting of a condition
Without condition variables, the programmer
would need to poll to check if the condition is met
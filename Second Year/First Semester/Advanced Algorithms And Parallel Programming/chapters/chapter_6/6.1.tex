\section{Introduction}

Amortized analysis is a technique used to evaluate the average cost per operation over a sequence, ensuring that the overall performance remains efficient even if individual operations can be costly. 
Unlike probabilistic analyses, amortized analysis provides a guarantee on the average cost of each operation, even in the worst case.

The three primary methods of amortized analysis are:
\begin{itemize}
    \item \textit{Aggregate method}: provides a simple overall average but lacks precision.
    \item \textit{Accounting method}: uses a banking approach with amortized costs per operation.
    \item \textit{Potential method}: relies on a potential function to manage the amortized cost.
\end{itemize}

\paragraph*{Hash table resizing}
A well-designed hash table should balance compactness with sufficient size to minimize overflow and ensure efficient performance. 
However, determining the optimal size in advance is often impractical. 
To address this, we use a dynamic table that expands as needed. When the table reaches its capacity, a larger table is allocated, all entries are rehashed into the new table, and the memory from the old table is released. 
This dynamic resizing allows the hash table to grow as entries are added, maintaining efficiency without predefined size constraints.
\section{Amortized analysis}

Amortized analysis is a technique used to evaluate the average cost per operation over a sequence, ensuring that the overall performance remains efficient even if individual operations can be costly. 
Unlike probabilistic analyses, amortized analysis provides a guarantee on the average cost of each operation, even in the worst case.

The three primary methods of amortized analysis are:
\begin{itemize}
    \item \textit{Aggregate method}: provides a simple overall average but lacks precision.
    \item \textit{Accounting method}: uses a "banking" approach with amortized costs per operation.
    \item \textit{Potential method}: relies on a potential function to manage the amortized cost.
\end{itemize}

\subsection{Accounting method}
n the accounting method, each operation is assigned a fictitious "amortized" cost, denoted as $\hat{c}_i$. Here, each dollar (or unit) represents one unit of work:
\begin{itemize}
    \item \textit{Immediate cost}: the fee covers the immediate operation.
    \item \textit{Banked cost}: any excess units not used immediately are saved to fund future operations.
\end{itemize}
To ensure consistency, the bank balance must never be negative, i.e.,
\[\sum_{i=1}^nc_i\leq\sum_{i=1}^n\hat{c}_i \qquad\forall n\]
Thus, the total amortized cost provides an upper bound on the true total cost of the operations.

\paragraph*{Accounting for dynamic tables}
For a dynamic hash table with expanding capacity, each insertion is charged an amortized cost of $\hat{c}_i=3$:
\begin{itemize}
    \item \textit{Immediate cost}: \$1 to perform the insertion.
    \item \textit{Banked cost}: \$2 banked for future table expansions.
\end{itemize}
When the table doubles in size, the accumulated bank covers the costs: \$1 pays to reinsert each newly added item, and \$1 covers the cost of moving older items to the new table.
This setup ensures the bank balance never falls below zero, establishing an upper bound on the total actual costs through the sum of the amortized costs. 
Thus, the dynamic table remains efficient with guaranteed average performance across all operations.

\subsection{Potential method}





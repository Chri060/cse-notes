\section{Potential method}

The potential method of amortized analysis views the bank account as the potential energy of a dynamic set of operations. 
The goal is to use the potential function to account for the work done by an operation and how it affects the overall cost over time.

In this framework, we start with an initial data structure, denoted as $D_0$. 
Each operation $i$ transforms this data structure from $D_{i-1}$ to $D_i$, with a cost of $c_i$.
We now define a potential function $\Phi$ that maps each data structure status $D_i$ to a real number: 
\[\Phi: \{D_i\} \rightarrow \mathbb{R}\]
This functions has the following properties: 
\[\Phi(D_0 ) = 0 \qquad \Phi(D_i ) \geq 0\]
We can finally define the amortized cost $\hat{c}_i$ for an operation $i$ with respect to the potential function as:
\[\hat{c}_i=c_i+\Phi(D_i)-\Phi(D_{i-1})=c_i+\Delta\Phi_i\]
Here, $\Delta\Phi_i$ is termed potential difference. 

The potential difference may be positive ore negative: 
\begin{itemize}
    \item If $\Delta\Phi_i>0$, then $\hat{c}_i>c_i$, meaning the operation stores work in the data structure for future use.
    \item If $\Delta\Phi_i<0$, then $\hat{c}_i<c_i$, meaning the data structure delivers stored work to help pay for the operation.
\end{itemize}
The total amortized cost over $n$ operations is given by:
\[\sum_{i=1}^n\hat{c}_i\geq\sum_{i=1}^nc_i\]
This inequality ensures that the total amortized cost provides an upper bound on the true total cost of the operations.

\paragraph*{Hash table resizing}
To apply the potential method to the dynamic resizing of a hash table, define the potential of the table after the $i$-th insertion by: 
\[\Phi(D_i) =2i - 2^{\left\lceil \log i\right\rceil }\] 
We assume that $2^{\left\lceil \log 0\right\rceil }=0$ (this accounts for the growth of the table as it resizes). 

The amortized cost of the $i$-th insertion is:
\[\hat{c}_i=c_i+\Phi(D_i)-\Phi(D_{i-1})=c_i+(2i - 2^{\left\lceil \log i\right\rceil })-(2(i-1) - 2^{\left\lceil \log (i-1)\right\rceil })\]
Here, the true cost $c_i$ of the $i$-th insertion is given by:
\[c_i=\begin{cases}
    i \qquad \text{if }i-1\text{ is an exact power of }2 \\
    1 \qquad \text{otherwise}
\end{cases}\]
For the case in which $i-1$ is an exact power of $2$, the amortized cost is:
\[\hat{c}_i=i + 2 - 2i + 2 + i - 1=3\]
For the second case, the amortized cost is:
\[\hat{c}_i=3\]
In both cases, the amortized cost per insertion is 3, and therefore, after $n$ insertions, the total cost is $\Theta(n)$ in the worst case.
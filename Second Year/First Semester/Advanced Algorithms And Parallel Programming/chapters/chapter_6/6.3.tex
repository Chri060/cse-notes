\section{Accounting method}

In the accounting method of amortized analysis, each operation is assigned a fictitious amortized cost, denoted as $\hat{c}_i$.
This amortized cost represents an accounting balance for the operation, where the units can be either used immediately or saved for future operations. 
The two key components of the amortized cost are:
\begin{itemize}
    \item \textit{Immediate cost}: this is the actual cost of the operation performed.
    \item \textit{Banked cost}: any excess cost that is saved and banked for future operations.
\end{itemize}
The key principle behind the accounting method is that the total accumulated banked cost must never be negative, ensuring that the resources saved are always sufficient to fund future operations.
This can be expressed mathematically as:
\[\sum_{i=1}^nc_i\leq\sum_{i=1}^n\hat{c}_i \qquad\forall n\]
Here, $c_i$  is the true cost of the $i$-th operation and $\hat{c}_i$ is the amortized cost.
By ensuring the bank balance never goes negative, the total amortized cost provides an upper bound on the true total cost of the operations, ensuring efficiency.

\paragraph*{Hash table resizing}
For a dynamic hash table that expands its capacity as needed, we can apply the accounting method to model the cost of insertions and table expansions. 
In this case, each insertion is charged an amortized cost of $\hat{c}_i=3$:
\begin{itemize}
    \item \textit{Immediate cost}: the immediate cost of performing the insertion is 1 unit, which represents the cost of adding an entry to the table.
    \item \textit{Banked cost}: 2 units is banked for future table expansions, which helps cover the cost of rehashing and moving entries during a table expansion.
\end{itemize}
When the table doubles in size, the banked cost ensures that the expansion process remains efficient. Specifically:
\begin{itemize}
    \item 1 unit of the banked cost is used to reinsert the newly added items into the larger table.
    \item The remaining 1 unit banked cost is used to cover the cost of moving the existing items to the new table.
\end{itemize}
This approach ensures that the bank balance never falls below zero, and thus the total amortized cost provides an upper bound on the true costs. 
With each insertion being charged an amortized cost of 3, and with the banked cost effectively covering the expansion costs, the dynamic hash table remains efficient. 
The amortized cost guarantees that the average cost per operation stays constant over time, ensuring good performance even in the face of resizing operations.
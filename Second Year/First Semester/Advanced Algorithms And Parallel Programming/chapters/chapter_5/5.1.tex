\section{Introduction}

Traditionally, people (are used to) think sequential
Developers (are used to) think sequential
Most of the existing algorithms are sequential
but Modern architectures offer an high degree of
parallelism: they can execute different
instructions/tasks at the same time

The main advantages of parallism are: 
\begin{itemize}
    \item Time saving: parallel algorithms can be more performant than sequential ones (i.e., take less time)
    \item Money saving: a parallel architecture composed of cheap components can be less expensive than a single processor architecture composed of a costly processor
    \item Solve «Grand Challenge» problems, i.e., problems that in practice can be solved only exploiting parallelism because of their complexity
\end{itemize}

\subsection{Moore's law}
Performance increasing of single cores is slowing
Moore’s law:
The number of transistors incorporated in a chip
will approximately double every 24 months
But a single core can not exploit anymore all these
transistors
Continue increasing the frequency of processors is
not anymore possible because of power
consumption


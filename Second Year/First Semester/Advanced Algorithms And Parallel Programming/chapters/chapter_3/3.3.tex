\section{Accounting method}

In the accounting method of amortized analysis, each operation is assigned a fictitious amortized cost, denoted as $\hat{c}_i$, which represents an accounting balance for the operation. 
This balance can either be used immediately or saved to cover the cost of future operations. 
The amortized cost consists of two main components: 
\begin{itemize} 
    \item \textit{Immediate cost}: the actual cost incurred for performing the operation. 
    \item \textit{Banked cost}: any excess cost that is set aside and saved for future operations. 
\end{itemize}
The core idea of the accounting method is that the total accumulated banked cost should never be negative, ensuring that there are sufficient funds to cover the costs of future operations. 
Mathematically, this can be expressed as:
\[\sum_{i=1}^n\hat{c}_i\geq\sum_{i=1}^nc_i\]
Here, $c_i$ represents the true cost of the $i$-th operation, and $\hat{c}_i$ is the amortized cost.
By ensuring that the bank balance remains non-negative, the amortized cost provides an upper bound on the true total cost of all operations, thereby guaranteeing efficient performance.

\paragraph*{Hash table resizing}
For a dynamic hash table that adjusts its size as needed, the accounting method can be applied to model the costs associated with insertions and table expansions. 
In this case, each insertion is assigned an amortized cost of $\hat{c}_i = 3$: 
\begin{itemize}
    \item \textit{Immediate cost}: the immediate cost of inserting an element into the expanded table is one unit. 
    \item \textit{Banked cost}: two units are banked with each insertion, to cover the cost of future expansions. 
\end{itemize}
When the table doubles in size, the banked cost ensures that the expansion remains efficient.
Specifically, half of the banked units are used to insert the existing elements into the new table, while the remaining units help move the elements that were added after the last expansion. 
This strategy ensures that the bank balance never goes negative, allowing the amortized cost to provide an upper bound on the true cost of all operations.

By charging each insertion an amortized cost of three, and using the banked cost to effectively cover the resizing expenses, the dynamic hash table remains efficient. 
The amortized cost guarantees that the average cost per operation remains constant over time, even during resizing operations, thus ensuring sustained performance.
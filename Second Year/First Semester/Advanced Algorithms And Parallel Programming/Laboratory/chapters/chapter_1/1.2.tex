\section{Syntax}

The ISPC  language is a C-based programming language designed for parallel computation. 
In ISPC, each local variable is associated with one of two qualifiers:
\begin{itemize}
    \item \textit{Uniform}: this qualifier indicates that the variable is shared across all instances within a gang.
    \item \textit{Varying}: this qualifier denotes that the variable is private to each instance (this is the default behavior).
\end{itemize}

\subsection{Loops}
ISPC introduces specific statements to facilitate parallelization. 
For example, loops can be specified over potentially multi-dimensional domains of integer ranges using the following syntax:
\begin{verbatim}
    foreach(identifier = start ... end) { /* body */ }
\end{verbatim}

\subsection{Communication}
The ISPC language provides two sets of functions for inter-instance communication:
\begin{itemize}
    \item \textit{Cross-program instance operations}: these are low-level communication facilities that include operations such as \texttt{broadcast}, \texttt{rotate}, and \texttt{shuffle}.
    \item \textit{Reductions}: these are high-level communication facilities that enable operations like \texttt{any}, \texttt{reduce\_add}, and \texttt{reduce\_max}. 
\end{itemize}

The ISPC compiler implicitly defines two essential variables:
\begin{itemize}
    \item \texttt{programIndex}: identifies each instance within a gang.
    \item \texttt{programCount}: represents the size of the gang.
\end{itemize}

To exchange values between instances, ISPC relies on the shuffle function, defined as follows:
\begin{verbatim}
    int32 shuffle(int32 value, int permutation)
\end{verbatim}
To retrieve a value from a specific ISPC process, the extract function is used:
\begin{verbatim}
    uniform int32 extract(int32 value, uniform int index)
\end{verbatim}

Additionally, the ISPC compiler can utilize the PRAM model to enhance performance. 
It is recommended to use reductions whenever available; if not, the low-level communication interface should be employed.
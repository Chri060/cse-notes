\section{Scaling}

In ISPC, a task is defined as a program that executes asynchronously, with each task being executed by a gang of program instances. 
The language provides two key keywords to manage the lifecycle of tasks:
\begin{itemize}
    \item \texttt{launch[<n>]}: this keyword is used to spawn the execution of $n$ tasks.
    \item \texttt{sync}: this keyword is employed to wait for the completion of all spawned tasks.
\end{itemize}
To define a task, you use the \texttt{task} prefix in a function declaration, ensuring that the function returns \texttt{void}. 
The syntax is as follows:
\begin{verbatim} 
    task void my_task_func( /* params / ) { / body */ }
\end{verbatim}
Within the function body, two built-in variables can be utilized:
\begin{itemize}
    \item \texttt{taskIndex}: represents the index of the current task.
    \item \texttt{taskCount}: denotes the total number of tasks.
\end{itemize}
It is important to note that the mapping between system threads and tasks is determined by the user, allowing for flexibility in implementation.

To effectively scale across different cores, tasks can be employed. 
While the mapping between system threads and tasks is adaptable, it is also advisable to consider external frameworks for managing cross-core parallelism. 
Solutions such as OpenMP, Intel Threading Building Blocks (TBB), or pthreads can complement ISPC's capabilities in handling parallel execution efficiently.
\section{World coordinates}

To simplify positional and directional references, we'll establish the origin at the center of the 3D world and utilize a compass for directional definitions. 
Specifically, we'll designate the negative $z$-axis as North and the positive $x$-axis as east.

The projection matrices we've encountered assume a projection plane parallel to the $xy$-plane. 
For parallel projections, we assume the projection rays align with the negative $z$-axis. 
In perspective, the projection center is at the origin, akin to a camera facing north.

In practical applications, generating a 2D view for a plane positioned anywhere in space involves additional transformations before projection.

Though our focus is on perspective, the same principles apply to parallel projection. 
The projection plane resembles a camera sensor viewing the scene from the projection center. 
This camera is defined by its position, aiming direction, and angle around this direction. 
The projection matrices we've discussed compute the view from a camera initially at the origin, facing the negative $z$-axis. 
This camera can be conceptualized as a 3D object.

We can then define a transformation matrix $M_C$ that relocates this camera object to its target position, aligning the negative $z$-axis with the desired direction. 
This matrix is termed the camera matrix. 
By applying the inverse of $M_C$ to all scene objects, we transform them into a new 3D space where the projection plane aligns as needed.

In this transformed space, the view from the arbitrarily oriented camera can be computed by applying the projection matrices.

The inverse of matrix $M_C$, denoted as $M_V$, is known as the view matrix:
\[M_V=M_C^{-1}\]
The view matrix precedes the projection matrix $M_{prj}$, enabling the determination of normalized screen coordinates of points in space as observed by the specified camera. 
This combined matrix is often referred to as the view-projection matrix:
\[M_{VP}=M_{prj} \cdot M_V\]
It transforms a point from world coordinates (in homogeneous coordinates) to 3D normalized screen coordinates (in Cartesian coordinates) at a specific location and direction in space according to matrix $M_V$.
Various techniques exist for creating a view matrix in a user-friendly manner, with the two most popular being:
\begin{itemize}
    \item \textit{Look-in-direction} technique.
    \item \textit{Look-at} technique.
\end{itemize}

\subsection{Look-in-direction}
This technique is commonly employed in first-person applications where the user directly controls the camera's position and view direction. 
In the look-in-direction model, the camera's position $(c_x, c_y, c_z)$ is given in world coordinates. 
The direction where the camera is facing is specified using three angles: the compass direction (angle $\alpha$), the elevation (angle $\beta$), and the roll over the viewing direction (angle $\rho$). 
The roll parameter is seldom used.

Specifically, with $\alpha=0^\circ$, the camera faces north, while $\alpha=90^\circ$ implies the camera is facing west. 
South corresponds to $\alpha=180^\circ$, and East is $\alpha=270^\circ$. 
A positive angle $\beta > 0^\circ$ makes the camera look up, while a positive angle $\rho > 0^\circ$ rotates the camera counterclockwise. 
In terms of the camera object, roll corresponds to rotation around the $z$-axis, elevation (or pitch) along the $x$-axis, and direction (or yaw) around the $y$-axis. 
Rotations must follow a specific order, discussed later: roll first, then elevation, and finally direction. 
Translation is applied after rotations. 
The camera matrix is composed as follows:
\[M_C=T(c_x, c_y, c_z) \cdot R_y(\alpha) \cdot R_x(\beta) \cdot R_z(\rho)\]
The view matrix, being the inverse of the camera matrix, follows the rules for inverting a chain of transformations:
\[M_V=M_C^{-1}=R_z(-\rho) \cdot R_x(-\beta) \cdot R_y(-\alpha) \cdot T(-c_x, -c_y, -c_z)\]

\paragraph*{Look-in in GLM}
GLM does not offer specific support for constructing a look-in-direction matrix. 
However, due to its simplicity, it can be easily implemented as follows:
\begin{verbatim}
glm::mat4 Mv = glm::rotate(glm::mat4(1.0), -rho, glm::vec3(0,0,1)) *
               glm::rotate(glm::mat4(1.0), -beta, glm::vec3(1,0,0)) *
               glm::rotate(glm::mat4(1.0), -alpha, glm::vec3(0,1,0)) *
               glm::translate(glm::mat4(1.0), glm::vec3(-cx, -cy, -cz));
\end{verbatim}

\subsection{Look-at}
The look-at model is typically utilized in third-person applications, where the camera tracks a specific point or object. 
In this scenario, the camera's position $(c_x, c_y, c_z)$ is given in world coordinates. 
This technique also requires the definition of an up vector, denoted as $u=(u_x, u_y, u_z)$, perpendicular to the ground plane of the scene. 
In $y$-up coordinate systems, the up vector is commonly set to $u=(0, 1, 0)$, aligning the camera's $y$-axis perpendicular to the horizon.

The view matrix is computed by first determining the direction of its axes in world coordinates, and then constructing the camera matrix accordingly. 
Initially, we establish the transformed (negative) $z$-axis as the normalized vector originating from the point the camera is looking at and ending at the camera center:
\[v_z=\dfrac{c-a}{\left\lvert c-a \right\rvert }\]
where $a$ represents the point the camera is aimed at.

The new $x$-axis must be perpendicular to both the new $z$-axis and the up vector. 
This can be calculated using the normalized cross product of the two vectors:
\[v_x=\dfrac{u \cdot v_z}{\left\lvert u \cdot v_z \right\rvert }\]
It's important to note that the cross product yields zero if the vectors $u$ and $v_z$ are collinear, making it impossible to determine $v_x$. 
This situation arises when the viewer is perfectly vertical, hindering the alignment of the camera with the ground. 
In such cases, the simplest solution is to use the previously computed matrix or select a random orientation for the $x$-axis.

Finally, the new $y$-axis should be perpendicular to both the new $z$-axis and the new $x$-axis, and it can be computed via the cross product of the two previously obtained vectors:
\[v_y=v_z \cdot v_x\]

The camera matrix, denoted as $M_C$, is then constructed by arranging vectors $v_x$, $v_y$, and $v_z$ in the first three columns, and the position of the camera center $c$ in the fourth column:
\[M_C=\begin{bmatrix}
    v_x & v_y & v_z & c \\ 
    0 & 0 & 0 & 1
\end{bmatrix}\]
The view matrix can be obtained by inverting $M_C$.
Since the vectors are orthogonal, the inversion of a look-at camera matrix can be computed easily with a transposition and a matrix-vector product:
\[M_V=M_C^{-1}=M_C=\begin{bmatrix}
    R_C^T & -R_C^T\cdot c \\ 
    0 & 1
\end{bmatrix}\]

\paragraph*{Look-at in GLM}
GLM provides convenient functions for vector operations and matrix construction, simplifying the computation of view matrices.

To compute the cross product of two vectors, GLM offers the \texttt{cross ()} function:
\begin{verbatim}
glm::vec3 uXvz = glm::cross(u, vz);
\end{verbatim}
To normalize a vector, use the \texttt{normalize ()} function:
\begin{verbatim}
glm::vec3 vx = glm::normalize(uXvz);
\end{verbatim}
Constructing a $4 \times 4$ matrix from four column vectors can be achieved using the appropriate constructor:
\begin{verbatim}
glm::mat4 Mc = glm::mat4(vx, vy, vz, glm::vec4(c.x, c.y, c.z, 1));
\end{verbatim}
This construction can be implemented as follows:
\begin{verbatim}
glm::vec3 vz = glm::normalize(c - a);
glm::vec3 vx = glm::normalize(glm::cross(u, vz));
glm::vec3 vy = glm::cross(vz, vx);
glm::mat4 Mc = glm::mat4(glm::vec4(vx), glm::vec4(vy), 
                         glm::vec4(vz), glm::vec4(c, 1));
\end{verbatim}
Here, $c$, $a$, and $u$ are three \texttt{glm::vec3} vectors representing, respectively, the position of the camera, the target point, and the up vector. 
To obtain the view matrix, simply compute the inverse of the camera matrix:
\begin{verbatim}
glm::mat4 Mv = glm::inverse(Mc);
\end{verbatim}

Additionally, GLM provides the \texttt{lookAt ()} function to create a look-at matrix directly from three vectors representing the camera center, target point, and up vector:
\begin{verbatim}
glm::mat4 Mv = glm::lookAt(glm::vec3(cx, cy, cz), glm::vec3(ax, ay, az),
glm::vec3(ux, uy, uz));
\end{verbatim}
Or using vectors directly:
\begin{verbatim}
glm::mat4 Mv = glm::lookAt(c, a, u);
\end{verbatim}

\paragraph*{Roll in GLM}
Roll in look-at matrices can be implemented in two ways:
\begin{enumerate}
    \item Rotating the $u$ vector.
    \item Adding a rotation of an angle $\rho$ around the $z$-axis.
\end{enumerate}
For the second method, it can be implemented as follows:
\begin{verbatim}
glm::mat4 Mv = glm::rotate(glm::mat4(1.0), -Roll, glm::vec3(0,0,1)) *
               glm::lookAt(c, a, u);
\end{verbatim}

\subsection{Local coordinates and world matrix}
A fundamental aspect of 3D computer graphics is the capability to position and manipulate objects within the virtual environment. 
Object manipulation is typically achieved using a transformation matrix known as the World Matrix, denoted as $M_W$.

Each object possesses a set of local coordinates, representing the positions of its points within its creation space. 
When assembling a scene, these object points undergo a transformation from their original modeled positions to the positions where they are displayed. 
These transformed coordinates are referred to as the object's global or world coordinates.

The World Matrix $M_W$ facilitates this transformation from local to global coordinates. 
It applies a sequence of translations, rotations (and possibly scaling or shears) to the object's local coordinates. 
As discussed previously, the order of transformations is crucial as matrix multiplication is not commutative. 
While there may not be a single solution, adhering to best practices generally yields the desired outcome.

\subsection{World matrix definition}
When working with objects described in local coordinates, users typically aim to:
\begin{itemize}
    \item Position the object.
    \item Rotate the object.
    \item Scale or mirror the object.
\end{itemize}

Advanced transformations like shearing or scaling along arbitrary axes are generally avoided as they are not easily generalized. 
If such transformations are required, custom procedures must be developed to handle them.

\paragraph*{Positioning}
When positioning an object in 3D space, the user aims to define coordinates that remain consistent regardless of the object's size or orientation. 
Thus, specifying the placement at point $p=(p_x,p_y,p_z)$ necessitates applying translation $T(p_x,p_y,p_z)$ as the final transformation. 
This ensures that the object's location remains unchanged despite any prior rotations or scaling.
The parameters $p=(p_x,p_y,p_z)$ of the translation $T(p_x,p_y,p_z)$ determine the position where the origin of the object, in its local space, will be located after it has been translated in the world space.

It's crucial that the object being positioned is modeled with its origin set at the point most accurately representing its location. 
This specific point varies for each object and should be carefully chosen during the modeling process.

If the object is to have dimensions $s_x,s_y,s_z$, rotations should be applied afterward. 
Otherwise, the scaling direction would be altered. 

Therefore, scaling by $S(s_x,s_y,s_z)$ must be the initial transformation applied.
When scaling by a factor $s_x$, it means that a unitary segment in local coordinates (along the $x$-axis) becomes $s_x$ units in global coordinates. 
This principle similarly applies to scaling factors $s_y$ and $s_z$.

\paragraph*{Rotation}
Rotation should be conducted between scaling and translation operations. 
In order to establish an orientation in 3D space, the fundamental rotations along the three axes, namely $x$, $y$, and $z$, need to be combined.

There are various methods to compute the rotation matrix for an object. 
One objective in constructing the rotation component of the world matrix is to streamline the specification of parameters needed by the application to define the direction of an object. 
Euler Angles, comprising three parameters that describe an object's orientation in 3D space, are commonly referred to as Roll, Pitch, and Yaw.

Euler's angles were initially introduced for $z$-up coordinate systems, making them easier to remember in that context.
Therefore, we will initially explain them in a $z$-up coordinate system, where the World's North aligns with the positive $y$-axis.
Objects should be modeled in a way that they face along the positive $x$-axis, their side is aligned with the $y$-axis, and their vertical line is oriented along the $z$-axis.

The Roll, denoted as $\phi$, represents the rotation of the object along its facing axis (the $x$-axis).
A positive roll turns the object clockwise.

The Pitch, denoted as $\theta$, defines the elevation of the object and corresponds to a rotation around its $y$-axis (the side axis).
A positive pitch turns the head of the object facing down.

The Yaw, denoted as $\psi$, defines the direction of the object and corresponds to a rotation along the $z$-axis (the vertical axis).
It indicates the direction where the object is heading: $\psi=0^\circ$represents East, while $\psi=90^\circ$ corresponds to North.

Following this convention, rotations are performed in alphabetical order: $x$-axis, $y$-axis, and finally $z$-axis.
For different conventions, the axes must always be rotated in the roll, pitch, yaw order (RPY).
When considering a $y$-up system, where the object faces the positive $z$-axis, rotations should be performed in the $z$, $x$, and $y$ order.

\paragraph*{Final world matrix}
With this convention, positioning an object in space requires nine parameters: three for position, three for Euler angles, and three for scaling factors. 
Each parameter holds an intuitive physical significance.

The World Matrix $M_W$ is then computed by factoring the five transformations in the specified order:
\[M_W=T(p_x,p_y,p_z)\times R_y(\phi)\times R_x(\theta)\times R_z(\varphi) \times S(s_x,s_y,s_z)\]

\paragraph*{World matrix in GLM}
GLM does not provide dedicated functions for constructing a World matrix directly from basic position, scales, and rotations. 
However, within a specific package of the library, there exists a specialized function for generating a rotation matrix from Euler angles:
\begin{verbatim}
#include <glm/gtx/euler_angles.hpp>

glm::mat4 MEa = glm::eulerAngleYXZ(glm::radians(yaw),
                                    glm::radians(pitch),
                                    glm::radians(roll));
\end{verbatim}
It is typically more practical to define the entire sequence of transformations ourselves:
\begin{verbatim}
glm::mat4 Mw = glm::translate(glm::mat4(1.0), glm::vec3(px, py, pz)) *
               glm::rotate(glm::mat4(1.0), yaw , glm::vec3(0,1,0)) *
               glm::rotate(glm::mat4(1.0), pitch , glm::vec3(1,0,0)) *
               glm::rotate(glm::mat4(1.0), roll , glm::vec3(0,0,1)) *
               glm::scale(glm::mat4(1.0), glm::vec3(sx, sy, sz));
\end{verbatim}

\paragraph*{Conventions}
In a real application, modelers and developers collaborate to define the convention that should be used to create assets suitable for integration into the project. 
An adaptation transformation matrix $M_A$ can be appended before the world matrix to adapt the convention followed by third-party assets to the one required by the application:
\[M_W=T(p_x,p_y,p_z)\times R_y(\phi)\times R_x(\theta)\cdot R_z(\varphi) \times S(s_x,s_y,s_z) \cdot M_A\]
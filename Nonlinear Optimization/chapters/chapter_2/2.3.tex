\section{Convex functions}

\begin{definition}[\textit{Convex function}]
    A function $f : C \rightarrow\mathbb{R}$ defined on a convex set $C \subseteq \mathbb{R}^n$ is convex if
    \[f (\alpha x_1 + (1-\alpha )x_2) \leq \alpha f (x_1) + (1-\alpha )f (_2) \qquad\forall x_1,x_2 \in C,\forall\alpha \in[0,1]\]
\end{definition}
\begin{definition}[\textit{Strictly proper function}]
    The function $f$ is strictly convex if the inequality holds with less than for all $x_1,x_2 \in C$ with $x_1 \neq x_2$ and $\alpha\in(0,1)$.
\end{definition}
\begin{definition}[\textit{Concave function}]
    $f$ is concave if $-f$ is convex.
\end{definition}
\begin{definition}[\textit{Linear function}]
    $f$ is linear if it is both convex and concave.
\end{definition}
\begin{definition}[\textit{Epigraph}]
    The epigraph of $f : S \subseteq \mathbb{R}^n \rightarrow \mathbb{R}$, denoted by $\text{epi}(f)$, is the subset of $\mathbb{R}^{n+1}$:
    \[\text{epi}(f) = \{(\underline{x},y) \in S \times \mathbb{R} \mid f(\underline{x}) \leq y\}\]
\end{definition}
\begin{definition}[\textit{Domain}]
    Let $f : C \rightarrow\mathbb{R}$ be convex, the domain of $f$ is the subset of $\mathbb{R}^n$:
    \[\text{dom}(f) = \{\underline{x} \in C \mid f (\underline{x}) <+\infty\}\]
\end{definition}

Let $C \subseteq \mathbb{R}^n$ with $C \neq \varnothing$ and $f : C \rightarrow\mathbb{R}$ be convex.
The following properties holds: 
\begin{enumerate}
    \item For each $β \in\mathbb{R}$ (also $\beta\in + \infty$), the level sets:
        \[L_{\beta} = \{\underline{x} \in C \mid f (\underline{x}) \leq \beta\} \land \{\underline{x} \in C \mid f (\underline{x}) <\beta\}\]
        are convex subsets of $\mathbb{R}^n$.
    \item $f$ is continuous in the relative interior (with respect to $\text{aff}(C)$) of its domain.
    \item $f$ is convex if and only if $\text{epi}(f)$ is a convex subset of $\mathbb{R}^{n+1}$. 
\end{enumerate}
Consider $\min_{\underline{x}\in C \subseteq\mathbb{R}^n} f(\underline{x})$ where $C \subseteq\mathbb{R}^n$ and $f : C \rightarrow\mathbb{R}$ are convex.
Proposition:
i) If $C$ and $f$ are convex, each local minimum of $f$ on $C$ is a global minimum.
ii) If $f$ is strictly convex on $C$ , exists at most one global minimum (if not unbounded).
\begin{proof}
    proposition1: suppose $\underline{x}^\prime$ is a local minimum and there exists an $\underline{x}^\ast\in C$ such that $f(\underline{x}^\ast)< f(\underline{x}^\prime)$. 
    Since $f$ is convex: 
    \[f(\alpha \underline{x}^\prime +(1-\alpha)\underline{x}^\ast)\leq \alpha f( \underline{x}^\prime)+ (1-\alpha)\]

    proposition2: -........
\end{proof}

\subsection{Convex functions characterization}
\begin{proposition}
    $f : C \rightarrow\mathbb{R}$ of class $C^1$ with nonempty convex and open $C \subseteq\mathbb{R}^n$ is convex if and only if: 
    \[f(\underline{x}) \geq f (\underline{x}) +  \nabla ^t f (\underline{x})(\underline{x}-\underline{\overline{x}}) \qquad\forall \underline{x},\underline{\overline{x}} \in C\]
    $f$ is strictly convex if and only if inequality holds with greater than for all $\forall \underline{x},\underline{\overline{x}} \in C$ with $\underline{x}\neq\underline{\overline{x}}$.
\end{proposition}

\begin{proposition}
    $f : C \rightarrow\mathbb{R}$ of class $C^2$ with nonempty convex and open $C \subseteq\mathbb{R}^n$ is
convex if and only if the Hessian matrix $\nabla^2 f(\underline{x}) = \left( \frac{\partial^2f}{\partial x_i \partial x_j }\right)$ is positive
semidefinite at every $\underline{x}\in C$.

\end{proposition}
For $f \in C^2$, if $\nabla^2f (\underline{x})$ is positive definite $\forall\underline{x}\in C$ then $f (\underline{x})$ is strictly convex.

SUFFICEINTE CODNITION 


POSITIVE SEMIDEFINITE 


Equivalent definitions: based on the sign of the eigenvalues or principal minors of $A$ or of the diagonal coefficients of specific factorizations of $A$. 

\subsection{Subgradient of convex functions}
Convex/concave not everywhere differentiable (continuous) functions
Generalization of the concept of gradient for $C^1$ functions to piecewise $C^1$ functions.

\begin{definition}[\textit{Subgradient}]
    Let $C \subseteq\mathbb{R}^n$ and $f : C \rightarrow\mathbb{R}$ be convex.
    $\gamma\in \mathbb{R}^n$ is a subgradient of $f$ at $\underline{x}\in C$ if
    \[f (\underline{x}) \geq f (\underline{x}) + \gamma^t (\underline{x}-\underline{\overline{x}}x) \qquad \forall x \in C \]
\end{definition}
\begin{definition}[\textit{Subdifferential}]
    Let $C \subseteq\mathbb{R}^n$ and $f : C \rightarrow\mathbb{R}$ be convex.
    The subdifferential, denoted by $\partial f (\underline{x})$, is the set of all the subgradients of $f$ at $\underline{x}$.
\end{definition}
Let $C \subseteq\mathbb{R}^n$ and $f : C \rightarrow\amthbb{R}$ be convex.
The properties are as follows: 
\begin{enumerate}
    \item $f$ admits at least a subgradient at every interior point $x$ of $C$.
        In particular, if $x \in\text{int}(C)$ then $\exists\gamma\in\amthbb{R}^n$ such that
        \[H= \{(x,y ) \in\mathbb{R}^{n+1} \mid y = f (x) + \gamma^t (x-x)\}\]
        is a supporting hyperplane of $\text{epi}(f)$ at $(x,f (x))$.
        N.B.: Existence of (at least) a subgradient at every point of $\text{int}(C )$, with $C$ convex, is a necessary and sufficient condition for $f$ to be convex on $\text{int}(C)$.
    \item If $x ∈C$ , $\partial f (x)$ is a nonempty, convex, closed and bounded set.
    \item $x^\ast$ is a (global) minimum of $f$ on $C$ if and only if $0 \in\partial f (x^\ast)$.
\end{enumerate}
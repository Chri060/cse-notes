\section{Convex analysis}






the consequences of the separation theorem are the following three. 

1. Any nonempty, closed convex set $C\subseteq\mathbbb{R}^n$ is the intersection of all closed half spaces containing it. 
\begin{definition}[\textit{Supporting hyperplane}]
    Let $S\subset\mathbb{R}^n$ with $S\neq\varnothing$ and $\underline{\overline{x}}\in\partial(S)$ (boundary wrt $\text{aff}(S)$), $H=\{\underline{x}\in\mathbb{R}^N\mid\underline{p}^t(\underline{x}-\underline{\overline{x}})=0\}$ is a supporting hyperplane of $S$ at $\underline{\overline{x}}$ if $S\subseteq H^{-}$ or $S\subseteq H^{+}$.  
\end{definition}
2. If $C\neq\varnothing$ is convex then for every $\underline{\overline{x}}\in\partial(S)$ there exists at least a supporting hyperplane $H$ at $\underline{\overline{x}}$, i.e., exists $\underline{p}\neq 0$ such that $\underline{p}^t(\underline{x}-\underline{\overline{x}})\leq 0$, for each $\underline{x}\in C$. 
3. Farkas lemma, from which we derive the optimality conditions for nonlinear optimization. 
\begin{lemma}
    Let $A\in\mathbb{R}^{m\times n}$ and $\underline{b}\in\mathbb{R}^m$. 
    Then exists $\underline{x}\in\mathbb{R}^n$ such that $A\underline{x}=\underline{b}$ and $\underline{x}\geq 0$ if and only if there not exists $\underline{y}\in\mathbb{R}^m$ such that $\underline{y}^tA\leq\underline{0}^t$ and $\underline{y}^t\underline{b}\geq\underline{0}^t$.
\end{lemma}
Provides an infeasibility certificate, also known as theorem of the alternative: exactly one of $A\underline{x}=\underline{b},\underline{x}\geq 0$ and $\underline{y}^tA\leq\underline{0}^t$, $\underline{y}^t\underline{b}\geq\underline{0}^t$ is feasible. 

The geometric interpretation is that $\underline{b}$ belong to convex cone generated by the columns of $A$ ie $\text{cone}(S)=\{\underline{z}\in\mathbb{R}^m\mid\underline{z}=\sum_{j=1}^{n}x_jA_j,x_1\geq 0,\dots,x_n\geq 0\}$ if and only if no hyperplane separating $\underline{b}$ from $\text{cone}(A)$ exists.
The alternative is either $\underline{b}\in\text{cone}(A)$ or $\underline{b}\notin\text{cone}(A)$. 

\begin{proof}
    To right: Consider $\underline{\tilde{x}}\geq 0$ such that $A\underline{\tilde{x}}=\underline{b}$. 
    For all $\underline{y}$ such that $\underline{y}^tA\leq 0$ we have $\underline{y}^t\underline{b}=\underline{y}^tA\underline{\tilde{x}}\leq 0$. 

    To left: assume that $A\underline{x}=\underline{b}$, $\underline{x}\geq 0$ is infeasible, i.e., $\underline{b}\notin\text{cone}(A)$. 
    Consider $\text{cone}(A)=\{\underline{z}\in\mathbb{R}^m\mid z=\sum_{j=1}^m x_jA_j,x_j\geq 0 \forall j\}$. 
    $\text{cone}(A)\neq\varnothing$, closed convex set and $\underline{b}\notin\text{cone}(A)$. 
    Separating hyperplane result. Exists $\underline{p}\in\mathbb{R}^n$ and $\beta\in\mathbb{R}$ such that $\underline{p}^t\underline{b}>\beta$ and $\underline{p}^t\underline{z}\leq \beta \froall\underline{z}\in\text{cone}(A)$. 
    Since $0\in\text{cone}(A)$, $\beta\geq 0$ and hence $\underline{p}^t\underline{b}>0$.
    Moreover $\underline{p}^t\underline{z}\leq\beta\forall\underline{z}\in\text{cone}(A)$
    means $\underline{p}^tA\underline{x}\leq\beta\forall\underline{x}\geq 0$. 
    Since $\underline{x}\geq 0$, $\underline{p}^tA\underline{x}\leq\beta\forall\underline{x}\geq 0$ is true only if $\underline{p}^tA\leq 0$ because $x_j$ can take any negative value.
    Thus $\underline{p}\neq 0$, $\underline{p}^tA\leq 0$ and $\underline{p}^t\underline{b}>0$ ($\underline{y}=\underline{p}$). 
    So the system is feasible. 
\end{proof}
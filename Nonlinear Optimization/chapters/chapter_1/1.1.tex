\section{Introduction}

Optimization is a powerful and widely used field of applied mathematics, playing a crucial role in solving real-world problems across various domains.

Given a set $X\subseteq\mathbb{R}^n$ and a function $f:X \rightarrow \mathbb{R}^n$ hat we aim to minimize, the goal is to find an optimal solution $x^{\ast}\in X$ such that:
\[f(\underline{x}^{\ast})\leq f(\underline{x}) \qquad \forall\underline{x}\in X\]
Many decision-making problems cannot be effectively modeled using linear approaches due to their inherent nonlinearity.

\begin{example}
    We are given:
    \begin{itemize}
        \item $m$ warehouses, indexed by $i = 1,\dots, m$, each with a capacity $p_i$ and a location constraint within an area $A_i\subseteq\mathbb{R}^2$. 
        \item $n$ clients, indexed by $j = 1,\dots, n$, each located at coordinates $(a_j, b_j)$ with a demand $d_j$.
    \end{itemize}
    Our goal is to determine the optimal warehouse locations and how to distribute the product to clients while minimizing transportation costs, ensuring that warehouse capacities and client demands are met. 
    We assume that the total available supply is sufficient: $\sum_{i=1}^{m}p_i\geq\sum_{j=1}^{n}d_j$. 

    The decision variables for this problem are: 
    \begin{itemize}
        \item $(x_i,y_i)$: the coordinates of of warehouse $i$ (for $1 \leq i \leq m$).
        \item $w_{ij}$: the quantity of product transported from warehouse $i$ to client $j$ (for $1 \leq i \leq m$, $1 \leq j \leq n$).
        \item $t_{ij}$: the distance between warehouse $i$ and client $j$, given by: 
            \[t_{ij}=\sqrt{(x_{i}-a_j)^2+(y_{i}-b_j)^2}\]
    \end{itemize}

    We aim to minimize the total transportation cost:
    \[\min\sum_{i=1}^{m}\sum_{j=1}^{n}t_{ij}w_{ij}\]
    Subject to the following constraints:
    \begin{enumerate}
        \item Warehouse capacity constraints: 
            \[\sum_{j=1}^{n}w_{ij}\leq p_i \qquad \forall i\]
        \item Client demand satisfaction:
            \[\sum_{i=1}^{m}w_{ij}\geq d_j \qquad \forall j\]
        \item Warehouse location constraints: 
            \[(x_i,y_i)\in A_i\subseteq\mathbb{R}^2 \qquad \forall i\]
        \item Non-negativity constraints: 
            \[w_{ij}\geq 0, t_{ij}\geq 0\qquad\forall i,j\]
    \end{enumerate}
    This formulation ensures that all client demands are met while keeping transportation costs minimal and adhering to warehouse capacities and location constraints.
\end{example}

\begin{example}
    In computerized tomography, we analyze a 3D volume $V\subseteq\mathbb{R}^{3}$ that is subdivided into $n$ small cubes, called voxels $V_j$.
    We assume that the matter density is constant within each voxel.

    Our goal is to reconstruct a 2D slice of $V$, meaning we need to determine the density $x_j$ for each voxel $V_j$  based on measurements from $m$ X-ray beams.
    For the $i$-th beam:
    \begin{itemize}
        \item $a_{ij}$ represents the path length of the beam within voxel $V_j$. 
        \item $I_0$ is the initial X-ray intensity at the source.
        \item $I_i$ is the intensity after passing through the volume.
    \end{itemize}
    According to the Beer-Lambert law, the total log-attenuation of the beam is linearly related to the density:
    \[\sum_{j=1}^{n}a_{ij}x_j=b_i=\log\dfrac{I_0}{I_i} \qquad i=1,\dots,m\]
    We can formulate this as a linear system:
    \[Ax=b, \quad x_j\geq 0 \qquad \forall j=1,\dots,n\]
    However, this system is often infeasible due to measurement errors, non-uniformity of voxels, and other practical factors.

    To handle inconsistencies, we use a least squares formulation to minimize the reconstruction error:
    \[\min\sum_{i=1}^{m}\left(b_i-\sum_{j=1}^{n}a_{ij}x_{j}\right)^2\]
    Subject to: 
    \[x_j\geq 0 \qquad j=1,\dots,n\]
    Since $n\gg m$ (many voxels, fewer beams), the problem may have multiple optimal solutions. 
    To improve stability, we introduce a regularization term and minimize:
    \[f(x)=\sum_{i=1}^{m}\left(b_i-\sum_{j=1}^{n}a_{ij}x_{j}\right)^2+\delta\sum_{j=1}^{n}x_j\]
    Here, $\delta>0$ controls the strength of regularization.

    The function $f(x)$ can include nonlinear terms to better account for material properties, image characteristics, or a stochastic model of attenuation. 
    The number and directions of beams can also be optimized to improve reconstruction quality.
    In dynamic imaging, we can extend this to four dimensions to account for respiratory motion over time.
\end{example}

\begin{example}
    In this auction setting, bidders can place bids on combinations of discrete items, rather than bidding on individual items separately.
    Given:
    \begin{itemize}
        \item A set $N$ of $n$ bidders.
        \item A set $M$ of $m$ distinct items,
        \item For each subset $S \subseteq M$, bidder $j \in N$ is willing to pay $b_j(S)$ for that specific bundle $S$.
    \end{itemize}
    We assume that bidders may place higher bids for bundles than for individual items separately, i.e., there may be synergies in bundling items.

    Our goal is to determine the allocation of items to bidders to maximize total revenue while ensuring that no item is allocated more than once.
    The decision variables will be: 
    \begin{itemize}
        \item $b(S) = \max_{j\in N} b_j(S)$: the highest bid received for bundle.
        \item $x_S$: a binary variable indicating whether the highest bid for $S$ is accepted: 
            \[x_S=\begin{cases} 1 \qquad\text{if the highest bid on }S\text{ is accepted} \\ 0 \qquad \text{otherwise} \end{cases}\]
    \end{itemize}

    The formulation is the following: 
    \[\max\sum_{S\subseteq M}b(S)x_S\]
    Subject to: 
    \begin{enumerate}
        \item Each item can be allocated to at most one winning bundle: 
            \[\sum_{S\subseteq M\mid i\in S}x_S\leq 1\qquad \forall i \in M\]
        \item Each bundle is either selected or not: 
            \[x_S\in\{0,1\}\qquad\forall S \subseteq M\]
    \end{enumerate}
    In essence, when $x_S = 1$, bundle $S$ is awarded to a bidder willing to pay the highest price, ensuring that the total revenue is maximized while maintaining a valid allocation of items.

    This formulation has $2^{\left\lvert M \right\rvert}$ variables, making it computationally challenging for large sets of items.
    Efficient algorithms, such as branch-and-bound, linear programming relaxations, or heuristic methods, may be necessary for practical problem sizes.
\end{example}
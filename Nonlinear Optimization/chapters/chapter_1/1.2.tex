\section{Optimization problem}

The general optimization problem is formulated as follows:
\begin{align*}
    \min                      \:&\: f(\underline{x})        \\
    \text{such that }           &\: g_i(\underline{x})\leq 0 \qquad i\leq i \leq m \\
                                &\: \underline{x}\in S \subseteq \mathbb{R}^n
\end{align*}
\begin{definition}[\textit{Feasible region}]
    The feasible region consists of all points that satisfy both the set constraints and the algebraic constraints:
    \[X=S\cap\left\{\underline{x}\in\mathbb{R}^n\mid g_i(\underline{x})\leq 0,1 \leq i \leq m\right\}\]
    Here, each constraint function $g_i:S\rightarrow\mathbb{R}$ defines a restriction on the feasible set.
\end{definition}
\begin{definition}[\textit{Objective function}]
    The function to be minimized, known as the objective function, is given by:
    \[f:X\rightarrow\mathbb{R}\]
\end{definition}
\noindent It assigns a numerical value to each feasible solution $\underline{x}$, which we seek to minimize.

Without loss of generality, we assume:
\begin{itemize} 
    \item The problem is a minimization problem, since maximization can be rewritten as:
        \[\max_{\underline{x}\in X}f(\underline{x})=\min_{\underline{x}\in X}-f(\underline{x})\]
    \item All algebraic constraints are inequality constraints, since equality constraints can be rewritten as two inequalities:
        \[g(\underline{x})=0 \equiv \begin{cases} g(\underline{x})\geq 0 \\ g(\underline{x})\leq 0\end{cases}\]
\end{itemize} 

\begin{definition}[\textit{Global optimum}]
    A feasible solution $\underline{x}^\ast\in X$ is a global optimum if:
    \[f (\underline{x}^\ast) \leq f (\underline{x}) \qquad\forall\underline{x}\in X\]
\end{definition}
\begin{definition}[\textit{Local optimum}]
    A feasible solution $\underline{\overline{x}} \in X$ is a local optimum if there exists $\epsilon > 0$ such that:
    \[f (\underline{\overline{x}}) \leq f (\underline{x}) \qquad\forall\underline{x} \in X \cap \mathcal{N}_\epsilon(\underline{\overline{x}})\]
    Here, $\mathcal{N}_\epsilon(\underline{\overline{x}}) = \{\underline{x} \in X \mid \left\lVert \underline{x}-\underline{\overline{x}}\leq \epsilon\right\rVert \}$ is an epsilon-neighborhood around $\underline{\overline{x}}$.
\end{definition}
For complex optimization problems, finding a global optimum is often computationally infeasible. 
Instead, we focus on obtaining good local optima within a reasonable computation time.
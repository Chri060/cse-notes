\section{History}

\paragraph*{Filmography}
In the play "Rossum Universal Robots" from 1920, the term "robota" was introduced to refer to the first automatic robots.
Several years later, Isaac Asimov penned the renowned science fiction series "I, Robot".
Additionally, notable instances of robots in film include:

\begin{chronology}[5]{1972}{2020}{0.9\textwidth}
    \event{1977}{Star Wars}
    \event{1986}{Short circuit}
    \event{2001}{I robot}
    \event{2015}{Ex machina}
\end{chronology}

\paragraph*{Robots evolution}
The mechanical era commenced in 1700 with the advent of the first automata, initially devised as specialized dolls for specific purposes. 
Transitioning from this era, the dawn of the 1920s saw a resurgence of interest in universal-purpose robots within the realm of fiction.

By 1940, the cybernetics era took root with the creation of the first turtles and telerobots. 
Grey Walters pioneered a significant development in this era by crafting a robotic tortoise that exhibited mechanical animal tropism (movement directed by stimuli).

Two decades later, the automation era commenced with the inception of the first industrial robots, marking a shift towards mechanized processes. 
In 1961, UNIMATE, the inaugural industrial robot, initiated operations at General Motors, executing programmed tasks with precision and efficiency.
In 1968, Marvin Minsky introduced the Tentacle Arm, a groundbreaking innovation resembling the movements of an octopus. 
This hydraulic-powered arm, controlled by a PDP-6 computer, featured twelve flexible joints facilitating maneuverability around obstacles.

In 1972, Shakey pioneered mobility in robotics with the creation of the Stanford cart, heralding advancements in mobile robotics.

The year 1980 witnessed the establishment of the first comprehensive definition of a robot as a reprogrammable, multifunctional manipulator designed for diverse tasks involving material, parts, tools, or specialized devices.

The onset of the information era in 1990 saw robots evolving to possess autonomy, cooperation, and intelligence, marking a significant leap in their capabilities.

Finally, in 2012, the International Organization for Standardization (ISO) established the standard definition for robots, consolidating their diverse functionalities and characteristics into a unified framework.

\begin{chronology}[25]{1650}{2040}{0.9\textwidth}
    \event{1700}{Mechanical era}
    \event{1920}{Fiction era}
    \event{1940}{Cybernetics era}
    \event{1960}{Automation era}
    \event{1990}{Information era}
\end{chronology}
\section{Introduction}

One approach to creating a map involves measuring distances from landmarks, adjusting the position slightly, and then measuring again. 
However, the challenge lies in the inherent noise in these measurements, making it impossible to achieve a perfectly accurate map. 
Additionally, this method assumes perfect knowledge of the robot's pose, which is unattainable in practical scenarios.

Alternatively, a grid map can be employed, which discretizes the environment into a grid of cells. 
Each cell is either activated with a value of 1 if occupied or 0 if not. 
Another option is to utilize a grid map that incorporates occupancy probability.

\subsection{Occupancy from sonar return}
Within the framework of sonar-based occupancy estimation, two key components are employed:
\begin{itemize}
    \item A 2D Gaussian distribution is utilized to model occupied space.
    \item Another 2D Gaussian distribution is employed to represent free space.
\end{itemize}
However, the integration of sonar sensors introduces several challenges. 
Wide sonar cones contribute to noisy maps, while specular (multi-path) reflections generate unrealistic measurements.

\subsection{Two-dimensional occupancy grid}
A straightforward representation for maps in two dimensions involves an occupancy grid, where each cell is treated as independent.
The probability of a cell being occupied is typically estimated using Bayes' theorem: 
\[\Pr(A|B)=\dfrac{\Pr(B|A)\Pr(A)}{\Pr(B|A)\Pr(A)+\Pr(B|\sim A)\Pr(\sim A)}\]
The environment is mapped as an array of cells, with each cell representing a specific area. 
Typically, the size of each cell ranges from 5 to 50 centimeters. 
Within this array, each cell contains a probability value representing the likelihood of the cell being occupied. 
This approach is particularly useful for integrating data from various sensors and sensor modalities to create a comprehensive map.

\paragraph*{Cell occupancy}
The occupancy \texttt{occ(i,j)} of a cell with coordinates $(i,j)$ in the grid can be computed in several ways:
\begin{enumerate}
    \item \textit{Probability}: this represents the likelihood of the cell being occupied, ranging between zero and one:
        \[\Pr\left(occ\left(i,j\right)\right)\in\left[0,1\right]\]
    \item \textit{Odds}: This method calculates the ratio of the probability of the cell being occupied to the probability of it not being occupied. 
        It offers an advantage over the probability method since updates can be applied to either the numerator or the denominator independently:
        \[o(occ(i,j))=\dfrac{\Pr\left(occ\left(i,j\right)\right)}{\Pr\left(\lnot occ\left(i,j\right)\right)}\in\left[0,+\infty\right]\]
    \item \textit{Log odds}: Log odds: a variation of the odds that can take any real number as its value. 
        If it's more probable for a cell to be unoccupied, the value is negative; otherwise, it's positive:
        \[\log o(occ(i,j))\in\left[-\infty,+\infty\right]\]
\end{enumerate}
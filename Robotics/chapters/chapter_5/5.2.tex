\section{Mapping}

Initially, the log odds are set to zero, corresponding to a probability of $50\%$. 
Subsequently, the cells are recursively updated using Bayes' theorem:
\[\Pr(A|B)=\dfrac{\Pr(B|A)\Pr(A)}{\Pr(B)}\]
Here, $A$ denotes the actual occupancy value of the cell, while $B$ represents the new measurement for the cell.

The challenge with this approach arises from the fact that real-world robots lack precise localization, making it impossible to generate a usable map.

\section{Scan matching}
To address the challenge of perfect pose assumption, we can alternate between localization and mapping in the following manner:
\begin{enumerate}
    \item Correct odometry by maximizing the likelihood of pose $x_t$ based on the estimates of pose and map at time $t-1$: 
        \[\hat{x}_t=\argmax_{x_t}\left\{\Pr(z_t|X_t,\hat{m}_{t-1})\Pr(x_t|u_{t-1}\hat{x}_{t-1})\right\}\]
        Here, $\hat{m}_{t-1}$ represents the map constructed so far.
    \item Compute the map $\hat{m}_{t}$ based on mapping with known poses, leveraging the new pose and current observations.
\end{enumerate}

However, this method fails to adequately track uncertainty during the odometry correction process.
While the final map will likely be superior to those obtained with naive methods, uncertainty could lead to ambiguously defined obstacles and duplications.
\section{Introduction}

The Robot Operating System (ROS) offers a robust platform for implementing navigation capabilities in robotic systems, leveraging a wide array of pre-existing, high-quality components. 
The ROS navigation stack, in particular, stands out for its comprehensive suite of tools designed to facilitate path planning, obstacle avoidance, and localization.

The primary elements of the ROS navigation stack include:
\begin{itemize}
    \item \texttt{move\_base}: this is the central node within the ROS navigation stack, responsible for integrating all planning and control functionalities. 
        It relies on dynamically configurable plugins via ROS's  \texttt{pluginlib}, allowing for easy extension and customization.
        The core logic is built upon the \texttt{nav\_core} class. 
    \item \texttt{nav\_core}: this class serves as a central interface for navigation goals and velocity commands. 
        It supports plugins that can be swapped out at runtime, enabling flexible and dynamic updates to the navigation logic. 
        It relies on sensor data and map information provided by the map server to function effectively.
    \item \texttt{costmap}: a crucial component that processes sensor data to generate a 2D or 3D occupancy grid, where each cell represents a different cost value. 
        The costmap is divided into two types:
        \begin{itemize}
            \item \textit{Global costmap}: used for long-term path planning across the entire environment.
            \item \textit{Local costmap}: focuses on short-term planning and obstacle avoidance in the robot's immediate vicinity.
        \end{itemize}
        Both costmap have distinct configurations but share some common settings.
    \item \texttt{map\_server}: this tool is essential for publishing and saving maps. 
        It provides map data through both topics and services and can handle dynamically generated maps. 
        When combined with \texttt{costmap\_2d}, it manages multi-layered 2D maps, inflating obstacles based on sensor inputs. 
        The map consists of a YAML file for map metadata, and an image file encoding occupancy data.
    \item \texttt{amcl}: the Adaptive Monte Carlo Localization (AMCL) system is a probabilistic localization method that uses a 2D map.
        It requires laser scan data and benefits from odometry information to estimate the robot's position within the global frame. 
        AMCL transforms laser scan data to the odometry frame and publishes the transformation between the global and odometry frames.
\end{itemize}

\paragraph*{Other elements}
Certain platform-specific elements need to be implemented manually, including:
\begin{itemize}
    \item Low-level robot interaction.
    \item Sensor drivers.
    \item Sensor measurement processing.
    \item Odometry estimation.
    \item High-level task planning.
\end{itemize}
Many of these components are available as existing ROS packages, such as drivers and \texttt{robot\_pose\_ekf}.

\subsection{Requirements}
For effective operation, the ROS Navigation stack requires:
\begin{itemize}
    \item Sensors for localization and obstacle avoidance, such as \texttt{sensor\_msgs/LaserScan} or \texttt{sensor\_msgs/PointCloud}.
    \item A source of odometry, such as \texttt{nav\_msgs/Odometry}.
    \item Mechanisms to convert \texttt{geometry\_msgs/Twist} messages into motor commands.
    \item A well-formed tf tree to accurately represent the positions of sensors, the robot, and the map.
\end{itemize}
While the ROS Navigation stack is versatile and adaptable, optimal performance is achieved under certain hardware conditions:
\begin{itemize}
    \item It works best with differential drive or holonomic robots.
    \item It requires a planar laser scanner for effective scanning and localization.
    \item It performs best with robots that have square or circular shapes.
\end{itemize}
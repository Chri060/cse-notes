\section{Middleware origins and usage}

Middleware was introduced by d'Agapeyeff in 1968, the concept of a wrapper emerged in the 1980s as a bridge between legacy systems and new applications. 
Today, it is prevalent across various domains, including robotics. 
Examples outside of robotics include Android, SOAP, and Web Services.

The concept of Middleware is widely recognized in software engineering. 
It serves as a computational layer, acting as a bridge between applications and low-level details. 
It's important to note that Middleware is more than just a collection of APIs and libraries.

\paragraph*{Issues}
The challenge lies in fostering cooperation between hardware and software components. 
Robotics systems face architectural disparities that affect their integration. 
Ensuring software reusability and modularity is also a critical concern in this context.

\paragraph*{Main features}
The key attributes of middlewares include:
\begin{itemize}
    \item \textit{Portability}: offering a unified programming model irrespective of programming language or system architecture.
    \item \textit{Reliability}: middlewares undergo independent testing, enabling the development of robot controllers without the need to delve into low-level details, while leveraging robust libraries.
    \item \textit{Manage the complexity}: handling low-level aspects through internal libraries and drivers within the middleware, thereby reducing programming errors and speeding up development time.
\end{itemize}
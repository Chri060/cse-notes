\section{ActionLib}

Node A sends a request to Node B to perform a task. 
There are two primary methods for handling these requests: 
\begin{itemize}
    \item \textit{Services}: 
        \begin{itemize}
            \item Designed for tasks with small execution times.
            \item The requesting node can wait for the task to complete.
            \item  No status updates or cancellation options are available.
        \end{itemize}
    \item \textit{Actions}:
        \begin{itemize}
            \item Suitable for tasks with long execution times.
            \item The requesting node cannot wait for the task to complete.
            \item Provides status monitoring and cancellation options.
        \end{itemize}
\end{itemize}
The \texttt{actionlib} package in ROS serves as an implementation similar to threads, based on a client/server paradigm. 
It provides tools to:
\begin{itemize}
    \item Create servers that execute long-running tasks, which can be preempted.
    \item Create clients that interact with these servers.
    \item The ActionClient and ActionServer communicate using the ROS Action Protocol, which is built on top of ROS messages.
\end{itemize}
\begin{figure}[H]
\centering
\includegraphics[width=0.75\linewidth]{images/cs.png}
\caption{Client-server interaction}
\end{figure}
The main functions of the actionlib package include:
\begin{itemize}
    \item \textit{Goal}: Sends new goals to the server.
    \item \textit{Cancel}: Sends cancellation requests to the server.
    \item \textit{Status}: Notifies clients of the current state of every goal in the system.
    \item \textit{Feedback}: Sends clients periodic auxiliary information for a goal.
    \item \textit{Result}: Sends clients one-time auxiliary information upon completion of a goal.
\end{itemize}
Action templates are defined by a name and additional properties through a .action structure in ROS. 
Each instance of an action has a unique Goal ID, which provides a robust way for the action server and action client to monitor the execution of a particular instance of an action.
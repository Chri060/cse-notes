\section{ROS distributed}

ROS can function as a distributed system across multiple devices interconnected on the same network. 
Large projects typically leverage distributed systems for scalability and efficiency. 
Remote monitoring of robots becomes straightforward with a single ROS network.

To utilize ROS across multiple devices, it's essential to designate one device to run the ROS master, achieved by executing the command \texttt{roscore} on that device exclusively. 
For all other nodes, specifying the IP address of the master is necessary. 
To find your IP, use the \texttt{ifconfig} command and look for \texttt{inet addr}.

To ensure proper configuration, export all required variables, and append them to your \texttt{~/.bashrc} file to apply them automatically for every new terminal session:
\begin{verbatim}
$ gedit ~/.bashrc
\end{verbatim}
\paragraph*{Master configuration}
Begin by setting the master's IP:
\begin{verbatim}
export ROS_MASTER_URI=http://master_ip:11311
\end{verbatim}
Next, inform the ROS master of your IP:
\begin{verbatim}
export ROS_IP=master_ip
\end{verbatim}
\paragraph*{Client configuration}
Similarly, configure the client by setting the master's IP:
\begin{verbatim}
export ROS_MASTER_URI=http://master_ip:11311
\end{verbatim}
And specifying the client's IP to the ROS master:
\begin{verbatim}
export ROS_IP=master_ip
\end{verbatim}

On the master PC, initiate the ROS core with the command \texttt{roscore}. 
To verify proper functionality on the clients, open a new terminal and execute \texttt{rostopic list} without initiating \texttt{roscore} previously. 
You should observe topics on the ROS network. 
Now, all clients are interconnected and capable of communication and node initiation on the distributed ROS network.
\paragraph*{Time synchronization}
Recording high-throughput bags often necessitates splitting recordings across different ROS devices. 
To utilize these bags collectively, timestamp coherence is crucial. 
Hence, synchronizing the clocks of all devices on the ROS network is necessary.

The standard method involves employing an NTP server on the master device and configuring chrony clients on all other devices.
\begin{enumerate}
    \item Install the NTP server on the master and chrony on other nodes.
    \item Modify the chrony configuration file located at \texttt{/etc/chrony/chrony.conf}.
    \item After making changes, stop and restart chrony for them to take effect:
\begin{verbatim}
$ sudo service chrony stop
$ sudo service chrony start
\end{verbatim}
    \item Monitor synchronization progress:
\begin{verbatim}
$ chronyc tracking
\end{verbatim}
\end{enumerate}
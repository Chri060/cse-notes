\section{Introduction}

\begin{definition}[\textit{Machine Learning}]
    A computer program is said to learn from experience $E$ with respect to some class of tasks $T$ and performance measure $P$ if it improves with experience $E$. 
\end{definition}
Machine Learning allows systems to automatically improve and adapt by learning from data, enabling them to make predictions or decisions without being explicitly programmed.

\subsection{Supervised Learning}
In Supervised Learning, the model is provided with a dataset $\mathcal{D} = \{x_1, x_2, \dots, x_N\}$, where each input $x_i$ has a corresponding desired output or label $t_i$. 
The goal is for the model to learn the relationship between inputs and outputs so it can correctly predict outputs for new, unseen inputs.

Supervised Learning tasks include:
\begin{itemize} 
    \item \textit{Classification}: the model assigns a category or label to input data.
    \item \textit{Regression}: the model predicts a continuous numerical value. 
\end{itemize} 

\subsection{Unsupervised Learning}
In Unsupervised Learning, the model is given a dataset $\mathcal{D} = \{x_1, x_2, \dots, x_N\}$ without any labels or target outputs.
The goal is to identify patterns, structures, or regularities within the data.

A common task in Unsupervised Learning is clustering, in which the model groups similar data points into clusters based on their intrinsic properties.
Unsupervised Learning is often used for exploratory data analysis and dimensionality reduction.

\subsection{Reinforcement Learning}
Reinforcement learning focuses on training a model to make decisions by interacting with an environment.
At each step, the model takes an action $a_t$ based on the current state $x_t$, and the environment provides feedback in the form of a reward $r_t$. 
The objective is to learn a policy that maximizes cumulative rewards over time.
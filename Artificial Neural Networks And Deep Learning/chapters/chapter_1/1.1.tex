\section{Introduction}

\begin{definition}[\textit{Machine learning}]
    A computer program is considered to learn from experience $E$ with respect to a specific class of tasks $T$ and a performance measure $P$ if its performance at tasks in $T$, as measured by $P$, improves with experience $E$.
\end{definition}
Given a dataset $\mathcal{D} = \{x_1, x_2, \dots, x_N\}$, machine learning can be broadly categorized into three types:
\begin{itemize} 
    \item \textit{Supervised learning}: in this type of learning, the model is provided with desired outputs $\{t_1, t_2, \dots, t_N\}$ and learns to produce the correct output for new input data. 
        The primary tasks in supervised learning are: 
        \begin{itemize} 
            \item \textit{Classification}: the model is trained on a labeled dataset and returns a label for new data. 
            \item \textit{Regression}: the model is trained on a dataset with numerical values and returns a number as the output. 
        \end{itemize} 
    \item \textit{Unsupervised learning}: here, the model identifies patterns and regularities within the dataset $\mathcal{D}$ without being provided with explicit labels. 
        The main task in unsupervised learning is clustering, in which the model groups similar data elements based on inherent similarities within the dataset. 
    \item \textit{Reinforcement learning}: in this approach, the model interacts with the environment by performing actions $\{a_1, a_2, \dots, a_N\}$ and receives rewards $\{r_1, r_2, \dots, r_N\}$ in return. 
        The model learns to maximize cumulative rewards over time by adjusting its actions. 
\end{itemize}
\section{Recurrent Neural Network}

Many real-world problems involve dynamic data, where patterns unfold over time. 
To address this, various approaches can be employed depending on the need for temporal modeling: 
\begin{itemize}
    \item \textit{Memoryless models}: these models operate without retaining a true memory of past states. 
        Instead, they rely on a fixed number of past observations, often referred to as delay taps, to predict future values. 
    \item \textit{Feed Forward Neural Networks}: can extend the capabilities of traditional memoryless models by incorporating non-linear transformations. 
\end{itemize}
While these methods can handle certain types of sequential data, they are inherently limited because they lack a true memory mechanism to capture long-term dependencies.

\subsection{Models with memory}
Dynamic data often requires models capable of retaining information over time to capture long-term dependencies. 
These models rely on a hidden state that evolves dynamically, often influenced by noise and external inputs, which drive state transitions.
This hidden state, while not directly observable, must be inferred to compute outputs.

\subsubsection{Recurren Neural Networks}
Recurrent Neural Network offer a flexible and powerful approach by leveraging distributed hidden states that evolve through non-linear dynamics. 
Recurrent Neural Networks are deterministic systems designed to efficiently store and process sequential information. 
The hidden state in a Recurrent Neural Network is updated at each time step based on the current input and the prior state, allowing the network to learn complex temporal patterns through its non-linear transformation capabilities.

\paragraph*{Backpropagation}
The key strength of Recurrent Neural Networks lies in their recurrent connections, which allow information from prior time steps to persist and influence future computations. 
This architecture enables Recurrent Neural Networks to model both short-term and long-term dependencies, although training them presents unique challenges. 
One such challenge is the vanishing gradient problem, which arises during backpropagation through time.

In backpropagation through time, the network is unrolled for a fixed number of time steps, and gradients are computed for each step before being averaged to update the weights.
However, as the network propagates gradients backward through many time steps, they often shrink exponentially due to the nature of activation functions like Sigmoid and hyperbolic tangent. 
This makes it difficult for the network to learn dependencies that span a long temporal range.

\paragraph*{Vanishing gradient}
Addressing the vanishing gradient problem is crucial for enabling Recurrent Neural Networks to capture long-term dependencies effectively. 
One approach involves designing architectures like Long Short-Term Memory and Gated Recurrent Units. 
These models incorporate specialized mechanisms, such as gates and memory cells, that allow them to retain information over extended time spans while avoiding the issues of vanishing or exploding gradients. 
Another strategy is gradient clipping, which constrains gradients within a specific range to maintain stability during training.

\subsubsection{Long Short-Term Memory}
The vanishing gradient problem, which hampers the training of traditional Recurrent Neural Networks over long time sequences, was effectively addressed by Hochreiter and Schmidhuber in 1997. 
They introduced the Long Short-Term Memory architecture, which uses memory cells designed with logistic and linear units interacting multiplicatively.

A Long Short-Term Memory cell controls the flow of information through three key mechanisms:
\begin{enumerate}
    \item Information enters the cell when its input gate is activated.
    \item The information is retained in the cell as long as its forget gate remains active.
    \item Information is extracted from the cell by activating its output gate.
\end{enumerate}
These gates allow the Long Short-Term Memory to selectively store and retrieve information, providing a mechanism for maintaining long-term dependencies in a stable manner.
Importantly, the loop within the cell operates with fixed weights, making it possible to backpropagate gradients effectively through time.

The Long Short-Term Memory architecture inspired simpler variants, such as the Gated Recurrent Unit. 
Gated Recurrent Units streamline the memory cell design by merging the forget and input gates into a single update gate.
Additionally, they combine the cell state and hidden state into one, reducing complexity while retaining much of the Long Short-Term Memory's effectiveness.

In recurrent neural network architectures, computation graphs can be built with continuous transformations using recurrent layers. 
When working with full input sequences, bidirectional Recurrent Neural Networks take advantage of information in both temporal directions. 
This involves:
\begin{itemize}
    \item Having one Recurrent Neural Network process the sequence from left to right, and another process it from right to left.
    \item Concatenating the hidden layer outputs from both directions to create a richer feature representation.
\end{itemize}

Initialization plays a crucial role in training Recurrent Neural Networks effectively. 
The initial state of the Recurrent Neural Networks must be defined before processing sequences. 
There are several strategies for this:
\begin{itemize}
    \item A common approach is to initialize the states to a fixed value, such as zero.
    \item A more effective method is to treat the initial state as a set of learnable parameters. 
        In this case, the initial state values are randomly initialized and refined through training.
\end{itemize}
The training process involves backpropagating the prediction error through time, all the way to the initial state values, and computing the gradient of the error with respect to these parameters. 
Gradient descent is then used to update the initial state values, enabling the network to learn optimal starting conditions for its sequences.
\section{Modules}

Modules provide a flexible way to manage the operating system image at runtime. 
Rather than rebooting the system to load a different operating system image, modules allow the operating system's functionality to be extended dynamically. 
This approach offers a convenient method for interacting with and modifying the kernel.

\subsection{Paramters}
Parameters for modules can be declared and initialized directly within the code:
\begin{verbatim}
    static int num = 5;
// S_IRUG0: everyone can read the sysfs entry
    module_param(num, int, S_IRUG0);
\end{verbatim}
These parameters can also be specified when the module is instantiated:
\begin{verbatim}
    insmod yourmodule.ko num=10
\end{verbatim}

\subsection{Kernel crashes}
Since your module has full access to kernel code and data, it can potentially compromise the kernel's state and lead to a crash.
Common causes include:
\begin{itemize}
    \item Memory access errors (e.g., NULL pointer dereferencing, out-of-bounds access).
    \item Explicitly triggering a kernel panic on error detection (using \texttt{panic}). 
    \item Incorrect execution mode (e.g., sleeping in atomic context). 
    \item Deadlocks detected by the kernel (e.g., soft lockup or locking issues).
\end{itemize}

\paragraph*{Kernel oops}
When a kernel oops occurs, the CPU's state at the time of the fault is captured, along with: 
\begin{itemize} 
    \item The contents of registers, which might offer clues about the crash. 
    \item A backtrace of function calls that led to the crash. 
    \item Stack content (the last few bytes).
\end{itemize} 
In severe cases, the kernel may panic and halt all execution by stopping the scheduling of applications and entering a busy loop.

\subsection{Printing messages}
Messages can be printed from the kernel to the log buffer using the \texttt{pr\_ast} family of functions, such as: \texttt{pr\_emerg}, \texttt{pr\_alert}, \texttt{pr\_crit}, \texttt{pr\_err}, \texttt{pr\_warn}, \texttt{pr\_notice}, \texttt{pr\_info}, \texttt{pr\_cont},
When printing pointers, consider the following formats: 
\begin{itemize} 
    \item \texttt{\%p} : displays the hashed value of a pointer by default.
    \item \texttt{\%px} : always displays the pointer's address (use with caution for non-sensitive addresses). 
    \item \texttt{\%pK} : displays the hashed pointer value, zeros, or the actual address based on the \texttt{kptr\_restrict sysctl} setting. 
\end{itemize}
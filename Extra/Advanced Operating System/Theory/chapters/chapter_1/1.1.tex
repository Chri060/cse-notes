\section{Introduction}

The primary objectives of an Operating System (OS) include:
\begin{itemize}
    \item \textit{Resource management}: the OS allows programs to be created and executed as though they each have dedicated resources, ensuring fair and efficient use of these resources. 
        Common resources managed include the CPU, memory, and disk. 
        For the CPU, time-sharing mechanisms are employed, while memory is often divided into multiple regions for better management.
    \item \textit{Isolation and protection}: the OS ensures system reliability and security by controlling access to resources such as memory. 
        This prevents conflicts, ensures that one application doesn't interfere with another or access sensitive data, enforces data access rights, and guarantees mutual exclusion when necessary.
    \item \textit{Portability}: the OS uses interface and implementation abstractions to simplify hardware access and management for applications, effectively hiding the underlying complexity (using the facade pattern). 
        Additionally, these abstractions allow the same applications to function across systems with different physical resources, facilitating compatibility, such as running older applications on newer systems.
    \item \textit{Extensibility}: the OS employs interface and implementation abstractions to create uniform interfaces for lower layers, which enables the reuse of upper-layer components like device drivers. 
        Additionally, these abstractions help hide the complexity associated with different hardware variants, such as different peripheral models, using patterns like the bridge pattern.
\end{itemize}
\section{Isolation and protection}

To ensure isolation and protection, the operating system employs a Virtual Address Space (VAS), which encompasses all the memory locations a program can reference. 
This space is typically isolated from other processes, though certain portions may be shared in a protected manner. 
The VAS is constructed from various virtual memory areas, some of which are derived from the program's on-disk representation, others are dynamically created during execution, and some are entirely inaccessible, such as kernel space.

The usage of virtual address space does not directly correspond to the actual physical memory in use. 
Instead, it is fragmented to accommodate the most recently accessed portions. The remaining pages are stored in mass storage. 
For dynamically modified data, the swap area is utilized, while read-only data and executable code are retrieved from the original program on disk.

This system of indirection, known as paging, effectively cheats by allowing each process to operate as if it has access to a larger memory resource than physically available. 
As a result, a limited physical memory resource is perceived as abundant from the perspective of individual processes.
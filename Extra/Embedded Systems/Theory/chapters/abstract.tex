\begin{abstract}
    The course delves into embedded systems, covering their characteristics, requirements, and constraints. 
    It explores hardware architectures, including various types of software executors, communication methods, interfacing techniques, off-the-shelf components, and architectures suited for both prototyping and large-scale production.

    In terms of software architectures, the course examines abstraction levels, real-time operating systems, complex networked systems, and the tools and methodologies used for code analysis, profiling, and optimization.
    
    Students will also learn to analyze and optimize hardware/software architectures for embedded systems, focusing on managing design constraints and selecting appropriate architectures. 
    Key topics include estimating and optimizing performance and power at various abstraction levels, project management, and designing for reuse.
    
    Additionally, the course addresses run-time resource management and includes case studies to illustrate trade-offs based on application fields and system sizes.
\end{abstract}
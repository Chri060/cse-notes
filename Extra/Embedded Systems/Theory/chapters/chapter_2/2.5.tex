\section{Microcontrollers IDE}

Microcontrollers are primarily designed to host the final application, making them less suitable for code development processes compared to traditional PC-based environments. 
Given the limitations of microcontrollers, there arises a need for an Integrated Development Environment (IDE). 
An IDE is not just a simple code editor but a comprehensive suite that supports editing, versioning, cross-compilation of code, debugging, profiling, and integration with evaluation boards. 
Modern IDEs offer additional features, such as debugger integration, device libraries, support for real-time operating systems (RTOS), and code templates to simplify and accelerate development.

\subsection{MDK Professional}
MDK Professional is a comprehensive development suite that supports software development for a wide range of devices. 
It caters to both secure and non-secure applications.
MDK Professional is well-suited for Internet of Things applications requiring secure network connectivity to the cloud, as well as projects that rely on proven middleware components. 
The suite includes the Arm C/C++ Compiler, which is optimized for small code size and performance, and features highly optimized runtime libraries.
Additionally, MDK supports the installation of Software Packs, that include device support, CMSIS libraries, middleware, board support, code templates, and example projects. 

\subsection{MuVision}
The $\mu$Vision IDE integrates project management, build tools, a runtime environment, source code editing, and program debugging into a single environment to accelerate embedded software development.
It supports multi-screen configurations, allowing users to create individual window layouts on their visual workspace.
The $\mu$Vision Debugger provides a unified environment to test, verify, and optimize application code. 
It includes features such as simple and complex breakpoints, watch windows, and execution control, while offering full visibility into device peripherals. 
Software applications can be created using pre-built software components and device support from Software Packs, which may include libraries, configuration files, source code templates, and documentation.

\subsection{Parasoft C/C++}
Parasoft C/C++ test is a complete quality testing solution that enhances software development team productivity and improves the quality of C and C++ applications. 
Its key features include static code analysis, coding policy enforcement, automated code reviews with graphical progress tracking, unit and regression testing, and code coverage analysis.
Additionally, Parasoft C/C++test uses the high-speed streaming trace capabilities of the ULINKpro adapter to capture performance and code coverage data, which is then analyzed using the MDK-ARM development kit. 
It also includes advanced features such as value tracking of variables and strong typedef-based type checking to detect issues early in the development cycle.

\subsection{ULINK Debug Adapters}
The ULINK family of USB Debug Adapters connects a PC's USB port to a target system, enabling developers to debug and analyze embedded programs running on target hardware. 
The ULINKpro provides advanced features like execution profiling and code coverage through direct streaming trace to a PC. 
In addition to supporting real hardware targets, developers can use Fixed Virtual Platforms (FVPs) for software development without the need for physical hardware. 
FVPs simulate an entire Arm system, including processors, memory, and peripherals, allowing developers to start bare-metal coding and Linux application development at speeds comparable to real hardware.

\subsection{Code Composer Studio}
Code Composer Studio (CCS) is an integrated development environment designed for Texas Instruments' microcontroller and embedded processors portfolio.  
It combines the advantages of the Eclipse software framework with advanced embedded debugging capabilities, providing a feature-rich environment for embedded developers.
For high-performance processors the highly-optimizing C/C++ VLIW compiler can perform various optimization techniques, to accelerate algorithms. 
The compiler is regularly validated against industry-standard benchmarks, ensuring stability and performance across releases.

Many high-performance Texas Instruments processors offer the capability to perform processor trace, providing a detailed historical account of code execution, timing, and data access patterns. 
Trace data can be captured in dedicated on-chip memory or exported for external analysis.

\paragraph*{System analyzer}
The system analyzer tool provides visibility into the application, operating system, and hardware by correlating software and hardware instrumentation across multicore systems.
It consists of two core components:
\begin{itemize}
    \item \textit{Unified Instrumentation Architecture} (UIA): a software package for logging, runtime control, and data movement.
    \item \textit{Analysis displays}: tooling for data collection, decoding, analysis, and visualization.
\end{itemize}
System analyzer can capture data from the on-chip embedded trace buffer or stream it off-device via the system trace receiver. 
It offers real-time insight into system behavior, allowing developers to optimize and debug their applications.

\paragraph*{Linux development}
CCS supports both Linux kernel and application development.
The Linux kernel can be debugged via JTAG, while application development can be conducted using GDB. 
Additional functionality, such as Linux Trace Tools (LTTng), can be installed to enhance the development environment.

\subsection{IAR Systems}
IAR Embedded Workbench is a versatile development toolchain supporting multiple platforms and architectures, including RISC-V. 
IAR Systems offer a complete development environment encompassing all aspects of embedded software development with powerful functionality. 
IAR Embedded Workbench targets multiple architectures and offers a dedicated team of experts to assist developers.

The IAR Evaluation Kit includes a 30-day evaluation license for IAR Embedded Workbench for RISC-V, a GigaDevice Evaluation Board, and the I-jet Lite debug probe, as well as access to IAR Academy's online course for RISC-V development.
\section{Introduction}

Embedded systems are characterized by their ubiquitous presence, low power consumption, high performance, and interconnected nature.
These systems are commonly utilized in four main application contexts:
\begin{itemize}
    \item \textit{Public infrastructures}: safety is a primary concern to prevent potential attacks, and in some cases, latency is also crucial. 
        Examples include highways, bridges, and airports.
    \item \textit{Industrial systems}: reliability and safety are the predominant concerns. 
        Key industries utilizing embedded systems include automotive, aerospace, and medical.
    \item \textit{Private spaces}: this includes control systems for houses and offices.
    \item \textit{Nomadic system}: these systems involve data collection related to the health and positions of animals and people, requiring both security and low latency.
\end{itemize}
In the future, embedded systems will evolve in several key ways:
\begin{itemize}
    \item \textit{Networked}: transitioning from isolated operations to interconnected, distributed solutions.
    \item \textit{Secure}: addressing significant security challenges that impact both technical and economic viability.
    \item \textit{Complex}: enhanced by advancements in nanotechnology and communication technologies.
    \item \textit{Low power}: utilizing energy scavenging methods.
    \item \textit{Thermal and power control}: implementing runtime resource management.
\end{itemize}
Usually, we process useful data locally and send the relevant data to the cloud less frequently. 
This approach conserves energy, thereby extending battery life. 
Due to time constraints, developing a device from scratch is often infeasiblebecause the design process involves a multidisciplinary team.

\subsection{Technological problems}
Applications are expanding rapidly, pushing the need for mass-market compatibility and integrating into all aspects of life, which in turn drives up volumes. 
However, finding a balance in this progress is complicated by several factors. 
CMOS technology is reaching its physical limits, and the cost of developing new foundational technologies can often be unaffordable. 
Additionally, the power and energy demands create significant barriers, as does the exponential proliferation of data. 
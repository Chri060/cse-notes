\section{Productivity and planning}

The revenue model can be visualized simply as a triangle, where the product's life is represented by a span of $2W$, peaking at $W$. 
Any delay in market entry results in a loss, which is the difference between the areas of the on-time and delayed triangles.

In theory, increasing the number of designers on a team should reduce project completion time. 
In practice, though, productivity per designer tends to decrease due to the complexities of team management and communication. 
At a certain point, adding more designers can even extend project timelines.

To enhance productivity, the design methodology must support reuse, particularly at higher abstraction levels, and this should be backed by standardization (International Technology Roadmap for Semiconductors). 
Two nain technology development trends are: 
\begin{itemize}
    \item \textit{System on Chip} (SoC): focuses on full integration and achieving the lowest cost per transistor. 
    \item \textit{System in Package} (SiP): focuses on lowering the cost per function for the entire system.
\end{itemize}

\subsection{Micro Electro Mechanical systems}
Micro-Electro-Mechanical Systems (MEMS) are structures with integrated circuit and specialized micromachining (transducers, microsensors, and microactuators). 
Integrated microsystems combine circuitry and transducers to perform tasks autonomously or with the assistance of a host computer.

\paragraph*{Wireless sensor nodes}
Wireless sensor nodes are small, battery-powered devices that monitor local conditions.
These devices typically have limited resources and form nodes within a wireless network that covers a region or object of interest. 
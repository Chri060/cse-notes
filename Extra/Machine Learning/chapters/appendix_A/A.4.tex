\section{Confidence intervals}

We need to establish a level of confidence to determine if our estimator is sufficiently accurate. 
Since the probability of $\bar{X}=\mathbb{E}[X]$ is zero, given that the realization of the expected value is a continuous random variable itself, we need to construct intervals where we have high confidence that the true mean $\mathbb{E}[X]$ lies within.
A 95\% confidence interval implies that it works correctly 95\% of the time.

The available options for confidence intervals are:
\begin{itemize}
    \item Gaussian approximation: 
        \[\bar{X}-\dfrac{z_{\frac{\alpha}{2}\sigma}}{\sqrt{n}} \leq \mu \leq \bar{X}+\dfrac{z_{\frac{\alpha}{2}\sigma}}{\sqrt{n}}\]
    \item Chebyshev's inequality (requires $\mathbb{E}[X]=\mu<\infty$ and $\text{Var}[X]=\sigma^2<\infty$):
        \[\bar{X}-\dfrac{\sigma}{\sqrt{n}\sqrt{\alpha}} \leq \mu \leq \bar{X}+\dfrac{\sigma}{\sqrt{n}\sqrt{\alpha}}\]
    \item Hoeffding's inequality (finite support): 
        \[\bar{X}-(b-a)\sqrt{\dfrac{-\log(\frac{\alpha}{2})}{2n}} \leq \mu \leq \bar{X}+(b-a)\sqrt{\dfrac{-\log(\frac{\alpha}{2})}{2n}}\]
\end{itemize}
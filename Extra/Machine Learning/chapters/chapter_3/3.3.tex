\section{Model complexity}

Introducing an additional feature leads to an exponential growth in the volume of the input space.
This growth leads to the following problems: 
\begin{itemize}
    \item \textit{Computational cost}: the computational resources required to process and analyze the expanded input space increase significantly.
    \item \textit{Data quantity}: the amount of data needed to effectively explore and train models in the expanded input space may be substantial.
    \item \textit{Large model variance} (overfitting): with the increased complexity of the input space, there is a higher risk of models capturing noise or irrelevant patterns, leading to overfitting and decreased generalization performance.
\end{itemize}

Our goal is to choose the model with the minimal prediction error, which can be attained by decreasing the variance of the model:
\begin{itemize}
    \item \textit{Feature selection}: by carefully designing the feature space, we can choose the most impactful subset from all available features.
    \item \textit{Dimensionality reduction}: mapping the input space to a lower-dimensional representation can effectively reduce complexity and variance.
    \item \textit{Regularization}: shrinkage of parameter values towards zero helps control model complexity and mitigate overfitting.
\end{itemize}
These approaches are not mutually exclusive and can be combined to enhance model performance.

\subsection{Feature selection}
The most straightforward approach appears to be comparing all possible combinations of features.
Given $M$ features, for each $k=1,\dots,M$, we would need to train all $\binom{M}{k}=\frac{M!}{k!(M-k)!}$ models with exactly $k$ features and select the optimal one. 
However, this procedure quickly becomes computationally impractical.

In practical scenarios, feature selection is often carried out based on the specific model being utilized:
\begin{itemize}
    \item \textit{Filter}: features are assessed individually using certain evaluation metrics (e.g., correlation, variance, information gain), and the top $k$ features are selected. 
        While this method is very fast, it fails to capture any subset of mutually dependent features.
    \item \textit{Embedded}: feature selection is integrated into the Machine Learning approach itself (e.g., lasso, decision trees).
        Although this method is not computationally expensive, the features identified are specific to the chosen learning approach.
    \item \textit{Wrapper}: a search algorithm is employed to identify a subset of features by iteratively training a model with different feature subsets and evaluating their performance. 
        This method utilizes either a simpler model or a basic Machine Learning approach to evaluate the features. 
        Greedy algorithms are typically employed to search for the best feature subset.
\end{itemize}

\subsection{Dimensionality reduction}
Dimensionality reduction aims to decrease the dimensions of the input space, but it differs from feature selection in two significant ways:
\begin{itemize}
    \item It utilizes all features and transforms them into a lower-dimensional space.
    \item It is an unsupervised approach, meaning it doesn't rely on labeled data for training.
\end{itemize}
There are numerous methods for performing dimensionality reduction, including:
\begin{itemize}
    \item Principal Component Analysis (PCA).
    \item Independent Component Analysis (ICA).
    \item Self-Organizing Maps.
    \item Autoencoders.
    \item ISOMAP.\@
    \item t-SNE.\@
\end{itemize}
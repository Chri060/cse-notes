\begin{abstract}
    The course will cover several topics, starting with an introduction to basic concepts. 
    Learning theory will be explored, including the bias-variance tradeoff, Union and Chernoff/Hoeffding bounds, VC dimension, worst-case (online) learning, and practical advice on using learning algorithms effectively.

    In supervised learning, key areas of focus include the supervised learning setup, LMS, logistic regression, perceptron, the exponential family, and kernel methods such as Radial Basis Networks, Gaussian Processes, and Support Vector Machines. 
    Additionally, topics like model selection, feature selection, ensemble methods (e.g., bagging and boosting), and strategies for evaluating and debugging learning algorithms will be addressed.
    
    The course will also delve into reinforcement learning and control, examining Markov Decision Processes (MDPs), Bellman equations, value iteration, policy iteration, TD, SARSA, Q-learning, value function approximation, policy search, REINFORCE, POMDPs, and the Multi-Armed Bandit problem.
\end{abstract}
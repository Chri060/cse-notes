\section{Exercise two}

Consider a dataset comprising workers' attributes such as the number of hours spent working ($x_1$), the number of completed projects ($x_2$), and whether they received a bonus ($t$).
After applying logistic regression, we obtain the following coefficients: $w_0=-6$, $w_1=0.05$, and $w_2=1$. 
\begin{enumerate}
    \item Determine the likelihood of a worker receiving a bonus given that they worked for 40 hours and completed 3.5 projects.
    \item Calculate the number of hours a worker needs to work to have a 50\% chance of receiving a bonus.
    \item Discuss whether values of $z$ in $\sigma(z)$ lower than $-6$ are meaningful in this context, and provide reasoning.
\end{enumerate}

\subsection*{Solution}
\begin{enumerate}
    \item The logistic model yields the probability of receiving a bonus as its output, expressed by:
        \[\text{P}(t=1|\textbf{x})=\sigma(w_0+w_1x_1+w_2x_2)\]
        Given $x_1=40$ and $x_2=3.5$: 
        \[\text{P}(t=1|\textbf{x})=\sigma(-6+0.05\cdot 40+1\cdot 3.5)=\sigma(-0.5)=0.3775\]
    \item To ascertain the probability of receiving a bonus with a confidence level $\alpha\%$, we need to invert the sigmoidal function.
        However, in this instance, a 50\% chance corresponds to the sigmoidal argument being zero. 
        Hence:
        \[w_0+w_1\hat{x}+w_2x_2=0\rightarrow -6+0.05\hat{x}+3.5=0\rightarrow\hat{x}=50\]
    \item Considering that all the variables under consideration are positive definite, it is reasonable to regard predictions with values greater than -6 as meaningful.
\end{enumerate}
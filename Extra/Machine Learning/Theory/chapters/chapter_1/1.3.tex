\section{Unsupervised learning}

Unsupervised learning encompasses two main tasks:
\begin{itemize}
    \item \textit{Clustering}: in this task, the objective is to group similar data points together based on their features, without predefined labels. 
        The goal is to uncover underlying patterns or structures within the data. 
        Clustering algorithms segment the data into clusters or groups, where data points within the same cluster exhibit greater similarity compared to those in different clusters. 
    \item \textit{Dimensionality reduction}: this task involves reducing the number of input variables or features in a dataset while retaining essential information. 
        This is often done to address the curse of dimensionality, enhance computational efficiency, and mitigate overfitting risks in models. 
        Dimensionality reduction techniques aim to transform high-dimensional data into a lower-dimensional representation while preserving most relevant information.
\end{itemize}
Formally, in unsupervised learning, computers learn previously unknown patterns and efficient data representations.
The training set is defined as $\mathcal{D}=\left\{ x \right\}$, where the goal is to find a function $f$ that extracts a representation or grouping of the data.

Various techniques are used for unsupervised learning, including k-means clustering, self-organizing maps, and principal component analysis.
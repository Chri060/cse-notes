\section{Bayesian approach}

The bounds provided by traditional methods do not allow for the incorporation of information one has about the distribution parameters.
A new approach considers the expected value $\mu$ of the random variable $X$ as a random variable itself.
Bayes' formula is utilized to update this information:
\[\text{P}(a|b)=\dfrac{\text{P}(b|a)\text{P}(a)}{\text{P}(b)}\]
Considering Bayes' formula, we have:
\begin{align*}
    \text{P}(\mu|x_1,\dots,x_t)     &=\dfrac{\text{P}(x_1,\dots,x_t|\mu)\text{P}(\mu)}{\text{P}(x_1,\dots,x_t)} \\
                                    &\propto \text{P}(x_t|\mu)\text{P}(x_1,\dots,x_{t-1}|\mu)\text{P}(\mu) \\
                                    &= \text{P}(x_t|\mu)\text{P}(x_{t-1}|\mu)\text{P}(x_1,\dots,x_{t-2}|\mu)\text{P}(\mu) \\
                                    &=\text{P}(\mu)\prod_{h=1}^{t}\text{P}(x_h|\mu)
\end{align*}
We incrementally incorporate information from a prior distribution $\text{P}(\mu)$.
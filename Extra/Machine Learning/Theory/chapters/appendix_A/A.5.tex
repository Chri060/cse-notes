\section{hypothesis testing}

In hypothesis testing, we aim to demonstrate that the estimated parameter $\bar{X}$ is equal to $\mu$ and that another estimated parameter $\bar{X}^\prime$  is different from yet another estimated parameter $\bar{X}^{\prime\prime}$. 
With statistics, we define a null hypothesis $H_0$ and an alternative hypothesis $H_1$: 
\[H_0:\mu=\mu_0 \qquad \text{vs} \qquad H_1:\mu\neq\mu_0\]
and utilize the data to provide evidence supporting either hypothesis.

\subsection{Basic Gaussian test}
Given the data $\left\{ x_1,\dots,x_n \right\}$, we have: 
\[\bar{X}\sim\mathcal{N}\left(\mu,\dfrac{\sigma^2}{n}\right)\rightarrow t=\dfrac{\bar{X}-\mu}{\frac{\sigma}{\sqrt{n}}}\sim\mathcal{N}(0,1)\]
Fixing a confidence $1-\alpha$ with $\alpha \in (0,1)$, the test statistic $t$ should be close to the true mean $\mu$ with very high probability. 
Formally:
\[\text{P}(t<z_{\frac{\alpha}{2}} \lor t> z_{1-\frac{\alpha}{2}})=\alpha\]
The corresponding decision table is:
\begin{table}[H]
    \centering
    \begin{tabular}{cccc}
                                                        &                                                       & \multicolumn{2}{c}{\textit{Decision}}                                                                                               \\ \cline{3-4} 
                                                        & \multicolumn{1}{c|}{}                                 & \multicolumn{1}{c|}{\textbf{Fail to reject $\textbf{H}_\textbf{0}$}} & \multicolumn{1}{c|}{\textbf{Reject $\textbf{H}_\textbf{0}$}} \\ \cline{2-4} 
    \multicolumn{1}{c|}{\multirow{2}{*}{\textit{True}}} & \multicolumn{1}{c|}{\textbf{$\textbf{H}_\textbf{0}$}} & \multicolumn{1}{c|}{Correct}                                         & \multicolumn{1}{c|}{Type I error ($\alpha$)}                 \\ \cline{2-4} 
    \multicolumn{1}{c|}{}                               & \multicolumn{1}{c|}{\textbf{$\textbf{H}_\textbf{1}$}} & \multicolumn{1}{c|}{Type II error}                                   & \multicolumn{1}{c|}{Correct}                                 \\ \cline{2-4} 
    \end{tabular}
\end{table}

\subsection{P-value}
To avoid specifying the confidence $\alpha$, we can let the data inform us about how confident we might be about their correspondence to a specific hypothesis:
\begin{itemize}
    \item Small p-values imply that we are confident that the $H_1$ hypothesis holds. 
    \item Large p-values imply that we are not able to reject the $H_0$ hypothesis.
\end{itemize}
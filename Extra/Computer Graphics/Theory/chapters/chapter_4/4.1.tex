\section{Introduction}

Natural objects exhibit a blend of glossy, bumpy, linear, and curved surfaces. 
In the digital realm, 3D assets serve as virtual renditions of these objects, encoded digitally. 
While we've discussed techniques for determining on-screen positions of points in 3D space, objects typically aren't represented as disjointed point clouds due to the immense resources such a representation would demand to create an illusion of solidity.

Instead, solid objects are initially stored solely by their boundaries, with their internal content often disregarded. 
This initial step simplifies the storage and processing requirements for digital representations of objects.

The encoding of object geometry relies on mathematical models that articulate surfaces using a series of parameters. 
Computational geometry delves into the optimal methods for mathematically articulating these surfaces. 
Numerous methodologies have been outlined in scholarly literature, with the most prevalent being: meshes (polygonal surfaces), hermite surfaces, NURBS (Non-Uniform Rational B-Splines), HSS (Hierarchical Subdivision surfaces), and metaballs. 
3D authoring tools such as Blender typically offer users the flexibility to choose from various techniques for encoding the models they create. 
Each technique necessitates distinct sets of tools and modeling proficiencies.
Every mathematical model possesses unique characteristics and constraints. 
Nonetheless, when it comes to rendering, all models are ultimately converted into meshes, or polygonal surfaces. 
Therefore, our focus will be primarily on meshes.

\paragraph*{Polygonal surfaces}
Polygonal surfaces represent objects described by a series of connected polygons. 
With specialized rendering features, these surfaces can effectively approximate curved shapes. 
Each polygon describing a planar section of the object's surface is termed a face, while the edges denote the sides of the polygons, typically where two faces intersect. 
Vertices mark the beginning and end points of edges, with each edge being delimited by exactly two vertices. 
In a proper solid, at least three faces intersect at a vertex.

\paragraph*{Two-manyfold}
In the context of topology, surfaces where every edge is adjacent to precisely two faces exhibit a unique structure known as a 2-manyfold. 
Surfaces that deviate from this pattern usually depict non-physical objects, and if utilized, necessitate careful attention to ensure accurate rendering.

At times, non-2-manifold objects are employed to minimize the polygon count required for encoding extremely thin objects or to achieve specific visual effects.
However, algorithms such as backface-culling are incompatible with non-2-manifold objects and may not function as intended.

\paragraph*{Triangles and meshes}
Each polygon can be subdivided into a collection of triangles that share certain edges. 
This assemblage of adjacent triangles forms what is known as a mesh.

Given that three non-collinear points define a unique plane in space, and likewise three non-collinear points define a triangle, it follows that polygons with more than three vertices may not necessarily represent a single planar surface. 
Consequently, there exist various possible arrangements for connecting more than three points to delineate a surface in space. 
This necessitates the conversion of every polygon, whether planar or not, into a collection of triangles through a process known as polygon tessellation. 
Notably, polygon tessellations are not singular; multiple tessellations can be generated for each polygon with more than three sides.

In a mesh representation of an object, its surfaces are stored using the collection of polygons that form its boundaries. 
These boundary polygons are subsequently transformed into a series of interconnected triangles that share certain edges.
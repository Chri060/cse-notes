\section{Introduction}

Rendering is the process of creating realistic 3D images with accurate colors.
To achieve realistic images with filled surfaces rather than just wireframes, it's essential to accurately emulate the reflection of light.
Rendering involves replicating the effects of illumination by defining the light sources within the virtual environment and specifying the surface properties of the objects within the 3D world.

Light sources are components of the scene from which illumination originates.
These light sources emit various frequencies of the spectrum, and objects in the scene reflect a portion of this emitted light.
Photons emitted by the light sources interact with objects, bouncing off them, with some eventually reaching the viewpoint (i.e., the camera).
During rendering, the intensity and color of these photons are computed.
The amount of light reflected depends on both the input (incoming light) and output (reflected light) directions, as photons can bounce off objects in numerous ways before reaching the viewer.

\paragraph*{Radiance}
Radiance refers to the energy emitted by a surface in all directions during a specific time interval, measured in Joules ($J$).
Power, measured in Watts ($W$), represents the instantaneous emission of light energy by a surface at a given moment.
\begin{definition}[\textit{Irradiance}]
    Irradiance, denoted by $E$, quantifies the fraction of power emitted by a point on a surface within a specific time interval and is measured in $W/m^2$. 
\end{definition}
\begin{definition}[\textit{Radiance}]
    Radiance, denoted by $L$, measures the energy emitted from a point on a surface in a particular direction at a specific time instant, expressed in $W/(m^2\cdot sr)$. 
\end{definition}
Here, $sr$ represents steradians, the unit of solid-angle measure. 
Using solid angles allows radiance to remain independent of the distance at which an object is observed.

Most light sensors, including human eyes and cameras, produce readings proportional to radiance. 
In rendering, the radiance received at each point on the projection plane (i.e., each pixel on the screen) is determined based on the direction of the corresponding projection ray.
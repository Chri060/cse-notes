\section{Transformation inverse}

Transformations can be reversed to return an object to its original position, size, or orientation. 
One advantage of using matrices is that the inverse transformation can be easily computed by inverting the corresponding matrix: if point $p^\prime$ is the result of applying the transformation encoded in matrix $M$ to a point $p$, then point $p$ can be retrieved from $p^\prime$ by multiplying it with $M^{-1}$, the inverse of matrix $M$:
\[p=M^{-1} \cdot p^\prime\]
It can be shown that a transformation matrix $M$ is invertible if its sub-matrix composed of the first 3 rows and 3 columns is invertible. 
This is generally true, except in cases where:
\begin{itemize}
    \item One or two of the projected axes degenerate to zero length.
    \item Two axes perfectly overlap.
    \item One axis aligns with the plane defined by the other two.
\end{itemize}

The inverse of a general matrix $M$ can be computed as:
\[M^{-1}=\dfrac{1}{\det{(M)}}\text{adj }(M)\]
Here, $\text{adj }(M)$ denotes the adjugate of a square matrix $M$, which is the transpose of its cofactor matrix.

However, for some transformations presented earlier, their inverses can be computed using simple matrix patterns:
\begin{itemize}
    \item For translation:
        \[M=\begin{bmatrix}
            1 & 0 & 0 & d_x \\
            0 & 1 & 0 & d_y \\
            0 & 0 & 1 & d_z \\
            0 & 0 & 0 & 1   \\
        \end{bmatrix}
        \qquad 
        M^{-1}=\begin{bmatrix}
            1 & 0 & 0 & -d_x \\
            0 & 1 & 0 & -d_y \\
            0 & 0 & 1 & -d_z \\
            0 & 0 & 0 & 1   \\
        \end{bmatrix}\]
    \item For scaling:
        \[M=\begin{bmatrix}
            s_x & 0   & 0   & 0 \\
            0   & s_y & 0   & 0 \\
            0   & 0   & s_z & 0 \\
            0   & 0   & 0   & 1 \\
        \end{bmatrix}
        \qquad 
        M^{-1}=\begin{bmatrix}
            \frac{1}{s_x} & 0   & 0   & 0 \\
            0   & \frac{1}{s_y} & 0   & 0 \\
            0   & 0   & \frac{1}{s_z} & 0 \\
            0   & 0   & 0   & 1 \\
        \end{bmatrix}\]
    \item For rotation, by changing the sign of the sine function:
        \[M=\begin{bmatrix}
            1 & 0   & 0   & 0 \\
            0   & \cos \alpha & - \sin \alpha   & 0 \\
            0   & \sin \alpha   & \cos \alpha & 0 \\
            0   & 0   & 0   & 1 \\
        \end{bmatrix}
        \qquad
        M^{-1}=\begin{bmatrix}
            1 & 0   & 0   & 0 \\
            0   & \cos \alpha & \sin \alpha   & 0 \\
            0   & -\sin \alpha   & \cos \alpha & 0 \\
            0   & 0   & 0   & 1 \\
        \end{bmatrix}\] 
\end{itemize}
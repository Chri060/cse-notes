\section{Specular reflection}

We'll explore the following specular reflection models:
\begin{itemize}
    \item Phong specular reflection.
    \item Blinn specular reflection.
    \item Ward.
    \item Cook-Torrance.
\end{itemize}
A perfect mirror surface reflects light in only one direction, lying on the same plane as both the incident light and the surface's normal, but with an opposite angle. 
Consequently, light reflected from such a surface would be visible solely along this specific angle and invisible in any other direction. 
However, if a surface is rough, incoming light will also be reflected at angles close to the mirror reflection angle. 
Hence, even though at reduced intensities, the reflected ray could be visible in an area near the mirror direction.

Specular reflection can be accounted for by adding a specular term to the diffuse component. 
This term calculates the probability of mirror reflection occurring in the given viewing direction $\omega_r$.
Similar to the diffuse case, the specular component is characterized by a color $m_S$ that dictates how the RGB components of the incoming light are reflected.
In most cases, objects exhibit white specular color, i.e., $m_S = (1,1,1)$.
However, metallic objects like gold or copper have specular colors identical to their diffuse counterparts.

\subsection{Phong reflection model}
In the Phong model, the mirror reflection direction $\mathbf{r}$  is first calculated.
For parallel projection, $\omega_r$ remains constant at$\mathbf{p}$
In the case of perspective projection, $\omega_r$ can be computed as the normalized difference between the surface point $\mathbf{x}$ and the center of projection $\mathbf{c}$. 

The Phong specular reflection model takes into account the angular distance $\alpha$ between the specular direction and the observer. 
It computes the intensity of specular reflection based on $\cos\alpha$: the term reaches its maximum when the specular direction aligns with the observer and diminishes to zero as the angle exceeds $90^\circ$. 
To confine highlight regions more effectively, $\cos\alpha$ is raised to the power of $\gamma$.
A higher $\gamma$ results in smaller highlight areas, making the object appear shinier as its surface behaves more like a mirror.

To calculate the direction of the reflected ray, $n^\prime$, the projection of the light vector onto the normal vector is first computed.
This is achieved by finding its length with the dot product between the normal and light direction ($d\cdot n_x$), and then multiplying the result with the normal vector $\mathbf{n}_x$ to obtain a vector perpendicular to it.
Subsequently, subtracting $n^\prime$  from the light vector yields $d^\prime$, the perpendicular from $\mathbf{d}$ to $\mathbf{n}$. 
By adding $d^\prime$ twice to $d$, the reflected vector $\mathbf{r}$ is obtained: 
\[r=d+2d^\prime\]
In summary:
\[\mathbf{r}_{l,x}=2\left(\overrightarrow{xl}\mathbf{n}_x\right)\mathbf{n}_x-\overrightarrow{xl}\]
Note that many shading languages provide built-in functions to directly compute the reflected vector. 

The intensity of the specular reflection term can then be computed as:
\[\cos^\gamma\alpha=\text{clamp}(\omega_r\cdot\mathbf{r})^\gamma\]
Similar to the Lambert diffuse term, it is necessary to exclude cases where the cosine is negative using the \texttt{clamp()} function.

\paragraph*{Simplified parametrization}
The Blinn reflection model is an alternative to the Phong shading model that employs the half vector $\mathbf{h}$: a vector positioned midway between the viewer direction $\omega_r$ and the light $\mathbf{d}$.

\subsection{Blinn reflection model}
The angle $\alpha$ between the observer and the reflected ray is approximated by the angle $\alpha^\prime$ between the normal vector $\mathbf{n}_x$ and the half vector $\mathbf{h}$.
The half vector $\mathbf{h}$ can be computed as the normalized average of vectors $\mathbf{d}$ and $\omega_r$. 
Using the notation from the rendering equations:
\[\mathbf{h}_{l,x}=\dfrac{d+\omega_r}{\left\lvert d+\omega_r \right\rvert }=\text{normalize}(d+\omega_r)\]

The specular highlight is then computed by raising to the power of $\gamma$ the cosine of $\alpha^\prime$, expressed as the dot product of $\mathbf{n}_x$ and $\mathbf{h}_{l,x}$. 
The formula for Blinn specular reflection is:
\[f_{specular}(x,\overrightarrow{lx},\omega_r)=\mathbf{m}_S\cdot\text{clamp}(\mathbf{n}_x\cdot\mathbf{h}_{l,x})^\gamma\]

The Blinn specular model is typically slightly more computationally expensive than the Phong model (as it requires normalization, which is more complex than simple reflection). However, it is still easily achievable in real-time with current hardware.

Since the two techniques yield slightly different results, they are often both implemented and chosen by artists to achieve various effects. 
For instance, in the examples provided, the left image employs the Phong model, while the right image uses the Blinn specular computation. 
Blinn reflections generally exhibit a larger decay area compared to Phong reflections with similar parameters.

\subsection{Ward anisotropic specular model}
Some objects, such as hairs, CDs, or brushed metals, are characterized by grooves on their surfaces, leading to specular highlights oriented along these grooves. 
Such surfaces are termed anisotropic materials. The Ward specular model is significant for two key reasons:
\begin{enumerate}
    \item It is derived from physically inspired principles.
    \item It supports anisotropic reflections.
\end{enumerate}
To accommodate anisotropy, it's necessary to specify an orientation for the grooves on the surface by assigning two additional vectors alongside the normal. 
These vectors, termed tangent and bi-tangent, are denoted by $\mathbf{b}$ and $\mathbf{t}$ respectively. 
We will delve into methods for encoding or deriving these vectors in a future lesson.

Similar to the Blinn specular model, the Ward technique relies on the half-vector $\mathbf{h}$ between the light and viewer directions.
Specifically, it depends on the angle of this vector $\mathbf{h}$  with the normal (denoted with $\delta$), and the angle between its projection onto the $\mathbf{bt}$-plane and the groove direction (denoted with $\phi$).

The formula for the specular component of the Ward model incorporates two roughness parameters for the $\mathbf{h}$ and $\mathbf{bh}$ directions, denoted $\alpha_t$ and $\alpha_b$ respectively:
\[f_{\text{specular}}(\mathbf{x}, \mathbf{L}_x, \omega_r) =\dfrac{e^{-\frac{\left(\frac{\mathbf{ht}_x}{\alpha_t}\right)^2+\left(\frac{\mathbf{hb}_x}{\alpha_b}\right)^2}{\mathbf{hn}_x^2}}}{4\pi\alpha_t\alpha_b\sqrt{\frac{\omega_r\mathbf{n}_x}{\overrightarrow{lx}\mathbf{n}_x}}}\]

\subsection{The Cook-Torrance reflection model}
Realistic specular highlights on objects often display a soft falloff area. However, conventional models like Blinn and Phong yield a sharp falloff. 
Additionally, according to the Fresnel principle, objects tend to exhibit stronger specular reflection when the incident light is nearly parallel to the surface. 
These observations, among others, prompted the development of more sophisticated specular illumination models capable of accurately capturing these physical characteristics.

The Cook-Torrance Bidirectional Reflectance Distribution Function (BRDF) model aims to compute both the specular and diffuse components in a physically accurate manner.
The diffuse component adheres to the Lambert diffusion model. 
However, for achieving a physically accurate behavior, it is blended with the specular part through linear interpolation, governed by a coefficient $k$:
\[f_r(x,\overrightarrow{lx},\omega_r)=k\cdot f_{diffuse}(x,\overrightarrow{lx},\omega_r)+(1-k)f_{specular}(x,\overrightarrow{lx},\omega_r)\]
Here, $f_{diffuse}(x,\overrightarrow{lx},\omega_r)=\mathbf{m}_D\cdot\text{clamp}(\overrightarrow{lx}\mathbf{n}_x)$. 
The specular term is computed as the product of three factors:
\[f_{specular}(x,\overrightarrow{lx},\omega_r)=\mathbf{m}_S\dfrac{D\cdot F\cdot G}{4\cdot\text{clamp}(\omega_r\mathbf{n}_x)}\]
Similar to other specular models, it is characterized by a specular color $\mathbf{m}_S$. 
The $\text{clamp}(\omega_r\cdot\mathbf{n}_x)$ term is a geometric factor that normalizes the values computed by components $D$, $F$, and $G$.

The model also defines a constant $\rho$, commonly referred to as roughness, which determines the smoothness of the surface:
\begin{itemize}
    \item $\rho=0$ denotes a perfectly smooth object.
    \item $\rho=1$ represents a very rough surface.
\end{itemize}
For each of the three terms $D$, $F$, and $G$, various definitions exist, each characterized by its features and complexities. 
Many formulations will rely on the half-vector, defined for the Blinn specular model as:
\[\mathbf{h}_{l,x}=\dfrac{\overrightarrow{lx}+\omega_r}{\left\lvert \overrightarrow{lx}+\omega_r\right\rvert }=\text{normalize}(\overrightarrow{lx}+\omega_r)\]

\paragraph*{Distribution term}
The distribution term $D$ encapsulates the surface roughness. 
The Blinn version adapts the corresponding specular model to the Cook-Torrance framework. 
Specifically, it replaces the specular power $\gamma$ with the roughness parameter $\rho$ and includes a normalization factor.
\[D=\dfrac{(\mathbf{h}_{l,x}\cdot\mathbf{n}_x)^{\frac{2}{\rho^2}-2}}{\pi\rho^2}\]
The Beckmann version relies on parameter $rho$ to define the average slope of the surface at a microscopic level.
\[D=\dfrac{e^{-\left(\frac{\tan\alpha}{\rho}\right)^2}}{\pi\rho^2\cos^2\alpha}\]
Here, $\alpha=\arccos(\mathbf{h}_{l,x}\cdot\mathbf{n}_x)$
The GGX version of the distribution term $D$ employs the following definition. 
In certain studies, it has been shown to yield the most realistic outcomes while maintaining a complexity akin to the Blinn version.
\[D=\dfrac{\rho^2}{\pi(\text{clamp}(\mathbf{h}_{l,x}\cdot\mathbf{n}_x)^2(\rho^2-1)+1)^2}\]

\paragraph*{Fresnel term}
The Fresnel term $F$ is dependent on a parameter $F_0 \in [0,1]$. 
It dictates how the light response alters concerning the angle of incidence relative to the viewer and can be approximated with the following expression:
\[F=F_0+(1-F_0)\left(1-\text{clamp}(\omega_r\cdot\mathbf{h}_{l,x})\right)^5\]

\paragraph*{Geometric term}
The microfacet version of the geometric term $G$ isn't characterized by any parameters and solely relies on the angles.
\[G=\min\left(1,\dfrac{2(\mathbf{h}_{l,x}\cdot\mathbf{n}_x)(\omega_r\cdot\mathbf{n}_x)}{(\omega_r\cdot\mathbf{h}_{l,x})},,\dfrac{2(\mathbf{h}_{l,x}\cdot\mathbf{n}_x)(\overrightarrow{lx}\cdot\mathbf{n}_x)}{(\omega_r\cdot\mathbf{h}_{l,x})}\right)\]
The GGX version for the geometric term $G$ depends on the surface roughness $\rho$ and utilizes a helper function that is first invoked with the light direction and then with the viewer direction.
\[g_{GGX}(\mathbf{n},\mathbf{a})=\dfrac{2}{1+\sqrt{1+\rho^2\frac{1-(\mathbf{na}^2)}{\mathbf{na}^2}}}\]
\[G=g_{GGX}(\mathbf{n}_x,\omega_r)g_{GGX}(\mathbf{n}_x,\overrightarrow{lx})\]

\paragraph*{Summary}
Despite its complexity, the Cook-Torrance reflection model offers realistic reflections that consider numerous physical behaviors. 
Typically, it serves as a fundamental building block for various advanced rendering techniques. 
These techniques generate specialized textures, utilized as look-up tables, enabling the rendering of this reflection model in real-time.
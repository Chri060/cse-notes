\section{Bidirectional reflectance distribution function}

In scan-line rendering, the BRDF often lacks energy conservation in many cases. 
Typically, the BRDF comprises two main components:
\begin{itemize}
    \item Diffuse Reflection
    \item Specular Reflection
\end{itemize}
\[f_r(x,\overrightarrow{lx},\omega_r)=f_{diffuse}(x,\overrightarrow{lx},\omega_r)+f_{specular}(x,\overrightarrow{lx},\omega_r)\]
The diffuse component of the BRDF represents the primary color of an object. 
Shiny surfaces tend to reflect incoming light at specific angles, known as specular directions, which depend on the viewpoint $\omega_r$.
This effect is captured in the specular component of the BRDF. 
Typically, the number and shape of highlights correspond to those of the direct light sources in the scene.

In scan-line rendering, BRDF values for each color frequency and both diffuse and specular components generally fall within the range of $[0,1]$. 
However, due to the influence of individual lights, the resultant pixel color can sometimes exceed one. 
To address this, a common solution is to clamp the values within the range $[0,1]$ at the end of computation. 
Although not physically accurate, this technique creates effects similar to overexposure in photography, where areas receiving excessive light appear as white spots.

\subsection{High Dynamic Range}
Modern rendering techniques employ more sophisticated computations, allowing values beyond the traditional $[0,1]$ range. 
Instead of clamping, the final color is mapped into the $[0,1]$ range using suitable non-linear functions:
\[L(x,\omega_r)=g(L^\prime(x,\omega_r))\]
This approach enables the consideration of larger color dynamics, resulting in images with details visible in both very dark and extremely bright areas. 
However, HDR requires significantly more memory (four times as much) and greater computational power to apply the final non-linear scaling function.
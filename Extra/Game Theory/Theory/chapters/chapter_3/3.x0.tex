\subsection{Rational outcome}

To establish the existence of a rational outcome for the game, we need to prove:
\begin{enumerate}
    \item $v_1 = v_2$ (the two conservative values agree). 
    \item there exists $\bar{x}$ fulfilling
        \[v_1 = \inf_y f (\bar{x}, y)\]
        ($\bar{x}$ is optimal for Pl1)
    \item there exists \bar{y} fulfilling
        \[v_2 = \sup_x f (x, \bar{y})\]
        ($\bar{y}$ is optimal for Pl2)
\end{enumerate}
In the finite case optimal$ \bar{x}$ and $\bar{y}$ always exist; thus existence is equivalent to coincidence of the conservative values

\begin{theorem}[Von Neumann]
    There always exists a rational outcome for a a finite, zero sum game with two players, as described by a payoff matrix $P$.
\end{theorem}

\subsection{Optimality in pure strategies}

\begin{theorem}
    If a player knows the strategy played by the other player, she can always use a pure strategy to get the best outcome
\end{theorem}
That is, once the choice of one player is fixed, the optimization becomes a linear problem over a simplex (recall that the utility function is bilinear)
\begin{proof}
    Consider e.g. the second player, who knows that the first one plays a mixed strategy \bar{x}. 
    Then the second player must minimize the function
    \[f (\bar{x}, y) = (\bar{x}, Py)\]
    over the simplex $\sum_m$. 
    The sought-after value is reached in at least one vertex $e_j$.
    This corresponds to a pure strategy.
\end{proof}

\section{Rationality}

Now, suppose the following holds:
\begin{itemize}
    \item $v_1 = v_2 = v$.
    \item There exists a row $\bar{i}$ such that $p_{\bar{i}\bar{j}} \geq v_1 = v$ for all $j$.
    \item There exists a column $\bar{j}$ such that $p_{\bar{i}\bar{j}} \leq v_2 = v$ for all $i$.
\end{itemize}
In this case, $p_{\bar{i}\bar{j}} = v$, and this value represents the rational outcome of the game.

The strategies $\bar{i}$ and $\bar{j}$ are optimal because: 
\begin{itemize} 
    \item Player 1 cannot guarantee more than $v_2$, the conservative value of Player 2. 
    \item Player 2 cannot pay less than $v_1$, the conservative value of Player 1. 
\end{itemize}
Thus, $\bar{i}$ maximizes the function $\alpha(i) = \min_j p_{ij}$, and $\bar{j}$ minimizes the function $\beta(j) = \max_i p_{ij}$.

\subsection{Existence of a rational outcome}
To demonstrate the existence of a rational outcome in a zero-sum game, we need to establish the following:
\begin{enumerate}
    \item \textit{Equality of conservative values}: the conservative values of both players agree, i.e., $v_1 = v_2$.
    \item \textit{Existence of an optimal strategy for Player 1}: there exists a strategy $\bar{x}$ such that:
        \[v_1 = \inf_y f (\bar{x}, y)\]
        This ensures that $\bar{x}$ is an optimal strategy for Player 1. 
    \item \textit{Existence of an optimal strategy for Player 2}: there exists a strategy $\bar{y}$ such that:
        \[v_2 = \sup_x f (x, \bar{y})\]
        This ensures that $\bar{y}$ is an optimal strategy for Player 2.
\end{enumerate}
In the case where the strategy spaces are finite, such optimal strategies $\bar{x}$ and $\bar{y}$ always exist. 
Therefore, proving the existence of a rational outcome is equivalent to demonstrating the equality of the conservative values, i.e., $v_1 = v_2$.

\begin{theorem}[Von Neumann]
    There always exists a rational outcome for a finite two-player zero-sum game, as described by its payoff matrix $P$.
\end{theorem}
This fundamental result, known as the Minimax theorem, guarantees that in every finite zero-sum game, the conservative values for both players coincide, and optimal strategies exist for both players, leading to a rational outcome.
\begin{proof}
    Suppose, without loss of generality, that all $p_{ij}$ in the matrix $P$ are positive. 
    Consider the column vectors $p_1, \dots, p_m \in \mathbb{R}^n$, and let $C$ denote their convex hull. 
    Define the set
    \[Q_t = \{x \in \mathbb{R}^n : x_i \leq t \}\]
    and
    \[v = \sup \{t \geq 0 : Q_t \cap C = \emptyset \}\]
    
    Since $\text{int } Q_v \cap C = \emptyset$, the sets $Q_v$ and $C$ can be separated by a hyperplane. 
    Hence, there exist coefficients $\bar{x}_1, \dots, \bar{x}_n$, with some $\bar{x}_i \neq 0$, and $b \in \mathbb{R}$ such that:
    \[(\bar{x}, u) = \sum_{i=1}^{n} \bar{x}_i u_i \leq b \leq \sum_{i=1}^{n} \bar{x}_i w_i = (\bar{x}, w)\]
    for all $u = (u_1, \dots, u_n) \in Q_v$ and $w = (w_1, \dots, w_n) \in C$.
    
    Since all $\bar{x}_i$'s must be non-negative, we can assume $\sum \bar{x}_i = 1$. 
    Additionally, $b = v$, since $\bar{v} := (v, \dots, v) \in Q_v$, and
    \[(\bar{x}, \bar{v}) = \sum_i \bar{x}_i v = v \sum_i \bar{x}_i = v\]
    Therefore, $b \geq v$. If $b > v$, by choosing a small $a > 0$ such that $b \geq v + a$, we would have
    \[\sup \left\{\sum_{i=1}^{n} \bar{x}_i u_i : u \in Q_{v + a} \right\} < b\]
    which would imply $Q_{v + a} \cap C = \emptyset$, contradicting the definition of $v$.
    
    Next, since $Q_v \cap C \neq \emptyset$, let $\bar{w} = \sum_{j=1}^{m} \bar{y}_j p_j$ (as $C$ is convex) for some $\bar{y} = (\bar{y}_1, \dots, \bar{y}_m) \in \Sigma_m$. 
    Since $\bar{w} \in Q_v$, we have $\bar{w}_i \leq v$ for all $i$.
    
    We now show that $\bar{x}$ is optimal for Player 1, $\bar{y}$ is optimal for Player 2, and $v$ is the value of the game. 
    
    For Player 1, since $(\bar{x}, w) \geq v$ for every $w \in C$ by the separation result, and since each column $p_{\cdot j} \in C$, we have
    \[(\bar{x}, p_{\cdot j}) \geq v, \quad \text{for all } j\]
    
    For Player 2, consider $w = \sum_{j=1}^{m} \bar{y}_j p_j \in Q_v \cap C$ as before. 
    Then, $w_i = \bar{y} p_{i \cdot}$, and since $w \in Q_v$, it follows that $w_i \leq v$ for every $i$. Hence, we have:
    \[v \geq w_i = \bar{y} p_{i \cdot}\]
\end{proof}
Von Neumann's theorem guarantees that even when a finite zero-sum game has no solutions in pure strategies, the following holds:
\begin{itemize}
    \item For Player 1, there exists a mixed strategy, represented as a probability distribution $\mathbf{x}=\begin{pmatrix}x_1 & \dots & x_n\end{pmatrix}$, over her pure strategies. 
        For every column $j$: 
        \[(x,p_{\cdot j})=\sum_{i=1}^nx_ip_{ij}=x_1p_{1j}+x_2p_{2j}+\dots+x_np_{nj}\geq v\]
     \item For Player 2, there exists a mixed strategy, represented as a probability distribution $\mathbf{y}=\begin{pmatrix}y_1 & \dots & y_n\end{pmatrix}$, over her pure strategies.
        For every row $i$: 
        \[(y,p_{i\cdot})=\sum_{j=1}^ny_jp_{ij}=y_jp_{i1}+y_2p_{i2}+\dots+y_mp_{im}\leq v\]
\end{itemize}
The constant $v$ is the value of the game under mixed strategies. 
Player 1 aims to maximize $v$, while Player 2 seeks to minimize it.
\section{Group}

\begin{definition}[\textit{Group}]
    A group is defined as a nonempty set $A$ equipped with a binary operation $\cdot$ such that the following conditions hold:
\end{definition}
\begin{enumerate}
    \item \textit{Closure}: for any elements $a, b \in A$, the result of the operation $a \cdot b$ is also an element of $A$.
    \item \textit{Associativity}: the operation $\cdot$ is associative, meaning that for all $a,b,c\in A$, it holds that $(a \cdot b) \cdot c = a \cdot (b \cdot c)$.
    \item \textit{Identity element}: there exists a unique element $e$ known as the identity element, such that for every $a\in A$, the following holds: $a \cdot e = e \cdot a = a$.
    \item \textit{Inverse element}: for every element $a \in A$, there exists a unique element $b \in A$ (denoted as $a^{-1}$) such that $a \cdot b = b \cdot a = e$. 
        This element $b$ is called the inverse of $a$.
\end{enumerate}
\begin{definition}[\textit{Abelian group}]
    A group $A$ is termed an abelian group (or commutative group) if the operation is commutative; that is, for all $a, b \in A$, the equation $\cdot$ $a \cdot b = b \cdot a$ holds true.
\end{definition}
\begin{example}
    Examples of abelian groups:
    \begin{enumerate}
        \item \textit{The integers $\mathbb{Z}$}: the set of integers, equipped with the usual addition operation ($+$), forms an abelian group.
            This group satisfies all the group properties:
            \begin{itemize}
                \item \textit{Closure}: the sum of any two integers is an integer.
                \item \textit{Associativity}: addition is associative, i.e., $(a + b) + c = a + (b + c)$.
                \item \textit{Identity}: the identity element is $0$ since $a + 0 = a$ for any integer $a$.
                \item \textit{Inverses}: for every integer $a$, the inverse is $-a$ because $a + (-a) = 0$.
            \end{itemize}
        \item \textit{The non-zero real numbers $\mathbb{R}^*$}: the set of all real numbers except $0$, equipped with the usual multiplication operation ($\times$), is an abelian group. 
            It fulfills the following criteria:
            \begin{itemize}
                \item \textit{Closure}: the product of any two non-zero real numbers is also a non-zero real number.
                \item \textit{Associativity}: multiplication is associative, i.e., $(a \times b) \times c = a \times (b \times c)$.
                \item \textit{Identity}: the identity element is $1$ because $a \times 1 = a$ for any non-zero real number $a$.
                \item \textit{Inverses}: for every non-zero real number $a$, the inverse is $\frac{1}{a}$ since $a \times \frac{1}{a} = 1$.
            \end{itemize}
    \end{enumerate}
    Examples of non abelian groups: 
    \begin{enumerate}
        \item \textit{The group of $n \times n$ matrices with non-zero determinant}: the set of all $n \times n$ matrices with a non-zero determinant, equipped with the usual matrix multiplication, is a non-abelian group (often denoted as $\text{GL}(n, \mathbb{R})$). 
            This group satisfies the group properties as follows:
            \begin{itemize}
                \item \textit{Closure}: the product of two invertible matrices is invertible, thus remaining in the group.
                \item \textit{Associativity}: matrix multiplication is associative, i.e., $(A \cdot B) \cdot C = A \cdot (B \cdot C)$.
                \item \textit{Identity}: the identity matrix serves as the identity element.
                \item \textit{Inverses}: each invertible matrix has an inverse that is also an invertible matrix.
                \item \textit{Non-abelian}: for matrices $A$ and $B$, it is generally true that $A \cdot B \neq B \cdot A$.
            \end{itemize}
    \end{enumerate}
\end{example}
\begin{proposition}
    Let $(A, \cdot)$ be a group. 
    Then the cancellation law holds:
    \[a \cdot b = a \cdot c \implies b = c\]
\end{proposition}
\begin{proof}
    To demonstrate the cancellation law, we start with the equation $a\cdot b=a \cdot c$. 
    
    By multiplying both sides of this equation by the inverse of $a$, denoted as $a^{-1}$, we obtain: 
    \[a^{-1}a \cdot b = a^{-1} a \cdot c\] 
    Utilizing the property of inverses, this simplifies to:
    \[e \cdot b = e \cdot c\]
    Here, $e$ is the identity element of the group. 
    By the definition of the identity, we have $b = c$.
\end{proof}
\begin{proposition}
    The set of natural numbers with the operation $\oplus$ forms an abelian group.
\end{proposition}
\begin{proof}
    We verify the group properties.
    The identity element is $0$, since $n \oplus 0 = n$ for any natural number $n$.
    For any natural number $n$, the inverse with respect to $\oplus$ is $n$ itself. 
    However, since we consider the natural numbers as a set starting from $0$, the formal definition of inverses in this context might not strictly apply, but $0$ serves as an absorbing element for addition.
    The operation $\oplus$ is associative, as $(n_1 \oplus n_2) \oplus n_3 = n_1 \oplus (n_2 \oplus n_3)$ holds for all natural numbers $n_1, n_2, n_3$.
    The operation $\oplus$ is commutative since $n_1 \oplus n_2 = n_2 \oplus n_1$ for any natural numbers $n_1$ and $n_2$.

    Therefore, since all group properties are satisfied, $(\mathbb{N}, \oplus)$ is an abelian group. 
    Consequently, the cancellation law holds: 
    \[n_1 \oplus n_2 = n_1 \oplus n_3 \implies n_2 = n_3\]
\end{proof}
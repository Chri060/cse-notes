\section{Convexity}

\begin{definition}[\textit{Convex set}]
    A set $C \subset \mathbb{R}^n$ is convex just in case for any $x, y \in C$, provided $\lambda \in [0, 1]$, one has:
    \[\lambda x + (1 - \lambda)y \in C\]
\end{definition}
The intersection of an arbitrary family of convex sets is convex. 
A closed convex set with nonempty interior coincides with the closure of its internal points.
\begin{definition}[\textit{Convex combination}]
    We shall call a convex combination of elements $x_1, \dots, x_n$ any vector $x$ of the form
    \[x = \lambda_1x_1 + \dots + \lambda_nx_n\]
    with $\lambda_1 \geq 0, \dots, \lambda_n \geq 0$ and $\sum_{i=1}^{n} \lambda_i = 1$
\end{definition}
\begin{proposition}
    A set $C$ is convex if and only if for every $\lambda_1 \geq 0, \dots, \lambda_n \geq 0$ such that $\sum_{i=1}^{n}\lambda_i = 1$, for every $c_1, \dots, c_n \in C$, for all $n$, then $\sum^n_{i=1} \lambda_i c_i \in C$.
\end{proposition}
If $C$ is not convex, then there is a smallest convex set containing $C$: it is the intersection of all convex sets containing $C$. 
\begin{definition}[\textit{Convex hull}]
    The convex hull of a set $C$, denoted by co $C$, is:
    \[\text{co }C =\bigcap_{A\in\mathcal{C}}A\]
    where $\mathcal{C} = \left\{A | C \subset A \text{ and } A \text{ is convex}\right\}$. 
\end{definition}
\begin{proposition}
    Given a set $C$, then 
    \[\text{co }C=\left\{\sum_{i=1}^{n}\lambda_ic_i|\lambda_i\geq 0,\sum_{i=1}^{n}\lambda_i=1,c_i\in C\quad\forall i, n\in N\right\}\]
\end{proposition}
The convex hull of any set $C$ is the set of all convex combinations of points in $C$.
When $C$ is a finite collection of points, its convex hull co $C$ is called a polytope.
E.g. if $C$ contains three points, co $C$ is the triangle (i.e. a polygone) with such points at its angles, which includes also all the points inside
\begin{theorem}
    Given a closed convex set $C$ and a point $x$ outside $C$, there is a unique element $p \in C$ such that, for all $c \in C$:
    \[\left\lVert p - x\right\rVert\leq\left\lVert c-x\right\rVert\]

    The projection $p$ is characterized by the following properties:
    \begin{enumerate}
        \item $p \in C$
        \item $(x - p, c - p) \leq 0 \text{ for all } c \in C$
    \end{enumerate}
\end{theorem}
That is, $p$ is the closest point to $x$ belonging to the set $C$, and by the last property it forms an obtuse angle between $x$ and any $c \in C$. 

\begin{theorem}
    Let $C$ be a convex proper subset of the Euclidean space $\mathbb{R}^l$ and assume $\bar{x} \in \text{cl } C^c$.
    Then there is an element $0\neq x^\ast \in \mathbb{R}^l$ such that, $\forall c \in C$:
    \[(x^\ast, c) \geq (x^\ast, \bar{x})\]
\end{theorem}
This theorem means that a criterion to tell apart an external point from any internal point. 
\begin{proof}
    Suppose $\bar{x} \notin \text{cl } C$ and call $p$ its projection on $\text{cl } C$. 
    Based on the previous theorem, in follows that $(\bar{x} - p, c - p) \leq 0$ for all $c \in C$. 
    By setting $x^\ast := p - \bar{x} \neq 0$, the inequality becomes $(-x^\ast, c - \bar{x} - x^\ast) =(-x^\ast, -x^\ast) + (-x^\ast, c - \bar{x}) \leq 0$, that is $(x^\ast, c - \bar{x}) \geq \left\lVert x^\ast\right\rVert^2$. 
    Then, since $\left\lVert x^\ast\right\rVert^2> 0$, by linearity one obtains
    \[(x^\ast, c) \geq (x^\ast, \bar{x}) \quad \forall c \in C\]
    As $x^\ast$ appears in both sides, by renormalization we can choose $\left\lVert x^\ast\right\rVert  = 1$. 
    If $\bar{x} \in \text{cl }C \setminus C$, take a sequence $\{x_n\} \subset C^c$ such that $x_n \rightarrow \bar{x}$. 
    From the first step of the proof, we can find some norm one $x^\ast_n$ such that
    \[(x^\ast_n, c) \geq (x^\ast_n, x_n) \quad \forall c \in C\]
    So, given that for some sub-sequence one has $x^\ast_n \rightarrow x^\ast$, taking the limit of the above inequality yields
    \[(x^\ast, c) \geq (x^\ast, \bar{x}) \quad \forall c \in C\]
\end{proof}
\begin{corollary}
    Let $C$ be a closed convex set in a Euclidean space, let $x$ be on the boundary of $C$.
    Then there is a hyperplane containing $x$ and leaving all of $C$ in one of the halfspaces determined by the hyperplane.
\end{corollary}
Such an hyperplane is said to be an hyperplane supporting $C$ at $x$
\begin{corollary}
    Let $C$ be a closed convex set in a Euclidean space. 
    Then $C$ is the intersection of all half-spaces containing it.
\end{corollary}
\begin{theorem}
    Let $A, C$ be closed convex subsets of $\mathbb{R}^l$ such that $\text{int} A$ is nonempty and $\text{int}A \cap C = \varnothing$. 
    Then there are $0 \neq x^\ast$ and $b \in \mathbb{R}$ such that, $\forall a \in A$, $\forall c \in C$:
    \[(x^\ast, a) \geq b \geq (x^\ast, c)\]
\end{theorem}
This means a criterion to determine whether a point is in $A$ or in $C$.
\begin{proof}
    Since $\bar{x} = 0 \in (\text{int}A - C)^c$, from the previous separation theorem with $\bar{x} = 0$ there is $x^\ast \neq 0$ such that
    \[(x^\ast, x) \geq 0 \quad \forall x \in \text{int}A - C\]
    Thus, for $x = a - c$ by linearity we obtain
    \[(x^\ast, a) \geq (x^\ast, c) \quad \forall a \in \text{int}A , \forall c \in C\]
    By extension this implies
    \[(x^\ast, a) \geq (x^\ast, c) \quad \forall a \in \text{cl } \text{int}A = A , \forall c \in C\]
\end{proof}
$H = \{x : (x^\ast, x) = b\}$ is called the separating hyperplane: $A$ and $C$ are contained in the two different half-spaces generated by $H$.

\section{Game Theory Assumptions}

Game theory operates under the following key assumptions about the players involved:
\begin{enumerate} 
    \item \textit{Self-interested}. 
    \item \textit{Rational}. 
\end{enumerate}

\subsection{Self-interest} 
Players are assumed to focus solely on their own preferences concerning the outcomes of the game. 
This is a mathematical assumption, not an ethical judgment. 
In fact, it is essential for defining what constitutes a rational choice within the framework of game theory.
  
\subsection{Rationality}
\begin{definition}[\textit{Preference relation}]
    Let $X$ be a set. 
    A preference relation on $X$ is a binary relation $\preceq$ that satisfies the following properties for all $x,y,z\in X$: 
    \begin{itemize}
        \item \textit{Reflexive}: $x\preceq x$ (every element is at least as preferred as itself).
        \item \textit{Complete}: $x \preceq y$ or $y \preceq x$  (any two elements can be compared).
        \item \textit{Transitive}: if $x \preceq y$ and $y \preceq z$, then $x \preceq z$ (preferences are consistent across comparisons).
    \end{itemize}
\end{definition}
The transitive property ensures that preferences can be consistently ranked.
\begin{definition}[\textit{Utility function}]
    Given a preference relation $\preceq$ over a set $X$, a utility function representing $\preceq$ is a function $u:X\rightarrow\mathbb{R}$ such that: 
    \[u(x)\geq u(y)\Leftrightarrow x \preceq y\]
\end{definition}
While a utility function may not always exist in specific cases, it does exist in general settings, particularly when $X$ is finite.
If a utility function does exist, there are infinitely many such functions, differing by any strictly increasing transformation of the original function.

Each player $i$ is assigned a set $X_i$, representing all the choices available to them.
Therefore, the set $X=xX_i$ over which the utility function $u$ is defined represents the combined choices of all players.

\paragraph*{Rationality assumptions}
The following assumptions define the rational behavior of players:
\begin{enumerate}
    \item \textit{Consistent preferences}: players can establish a preference relation over the game's outcomes, and this ordering is consistent.
    \item \textit{Utility representation}: players can define a utility function that represents their preference relations when needed.
    \item \textit{Consistent use of probability}: players apply the laws of probability consistently, including computing expected utilities and updating probabilities according to Bayes' rule.
    \item \textit{Understanding consequences}: players comprehend the outcomes of their actions, the impact on other players, and the resulting chain of consequences.
    \item \textit{Application of decision theory}: players use decision theory to maximize their utility.
        Given a set of alternatives $X$ and a utility function $u$, each player seeks $\bar{x}\in X$ such that: 
        \[u(\bar{x}) \geq u(x)\qquad\forall x \in X\]
\end{enumerate}
One significant consequence of these axioms is the principle of eliminating strictly dominated strategies: a player will not choose an action $a$ if there exists another action $b$ that yields a strictly better outcome, regardless of the actions of other players.
\begin{example}
    Consider the following games: 
    \begin{table}[H]
        \centering
        \begin{tabular}{|c|c|}
        \hline
        \textbf{Gain} & \textbf{Probability} \\ \hline
        2500          & 33\%                  \\ \hline
        2400          & 66\%                  \\ \hline
        0             & 1\%                   \\ \hline
        \end{tabular}
        \caption{Game A}
    \end{table}
    \begin{table}[H]
        \centering
        \begin{tabular}{|c|c|}
        \hline
        \textbf{Gain} & \textbf{Probability} \\ \hline
        2500          & 0\%                  \\ \hline
        2400          & 100\%                  \\ \hline
        0             & 0\%                   \\ \hline
        \end{tabular}
        \caption{Game B}
    \end{table}
    In a sample of 72 participants, 82\% chose to play Game B, indicating a preference for certainty—characteristic of risk-averse individuals. 
    According to expected utility theory, this decision is rational if:
    \[u(2400)>\frac{33}{100}u(2500)+\frac{66}{100}u(2400)\] 
    This simplifies to:
    \[\frac{34}{100}u(2400)>\frac{33}{100}u(2500)\]

    Now consider the following alternatives: 
    \begin{table}[H]
        \centering
        \begin{tabular}{|c|c|}
        \hline
        \textbf{Gain} & \textbf{Probability} \\ \hline
        2500          & 33\%                  \\ \hline
        0             & 67\%                   \\ \hline
        \end{tabular}
        \caption{Game C}
    \end{table}
    \begin{table}[H]
        \centering
        \begin{tabular}{|c|c|}
        \hline
        \textbf{Gain} & \textbf{Probability} \\ \hline
        2400          & 34\%                  \\ \hline
        0             & 66\%                   \\ \hline
        \end{tabular}
        \caption{Game D}
    \end{table}
    In this new setup, 83\% of participants preferred Game C, reflecting a preference for a larger gain even with a lower probability of success.
    Rationality in this scenario requires:
    \[\frac{34}{100}u(2400)<\frac{33}{100}u(2500)\]
    This contradicts the earlier experiment, where the opposite preference was observed.
    Such behavior violates the independence axiom in expected utility theory, which states that consistent preferences should hold under similar probabilistic transformations.
    
    This contradiction is known as the Allais Paradox, demonstrating that individuals do not always act as fully rational decision-makers.
\end{example}
\begin{example}
    A group of players is asked to choose an integer between 1 and 100.
    The mean of all chosen numbers, $M$ is then calculated. 
    The objective of the game is to select the number closest to $qM$, where $0 < q < 1$.

    A purely rational player would conclude that the optimal number to choose is 1, regardless of the value of $q$. 
    However, this player is likely to lose.

    For example, let $q = \frac{1}{2}$. 
    Since $M \leq 100$, in the first step, it seems irrational to choose a number greater than $\frac{1}{2} \cdot 100$, as this is the initial target value based on the game's rules.

    However, in the second step, assuming all players are rational and recognize that others are also rational, each player would realize that others will also choose numbers below 50. 
    Therefore, the new logical step would be to pick a number less than $\left( \frac{1}{2} \right)^2 \cdot 100$.


    This reasoning continues iteratively: at step $n$, it becomes irrational to choose a number greater than $\left( \frac{1}{2} \right)^n \cdot 100$.
    Ultimately, after enough steps, the only rational choice would appear to be selecting the smallest possible number (1).
    
    Despite this reasoning, experiments show that the actual winning number is far higher than 1. 
    In fact, the winning number tends to increase as the value of $q$ increases, revealing that real-life behavior often deviates from purely rational game theory predictions.
\end{example}
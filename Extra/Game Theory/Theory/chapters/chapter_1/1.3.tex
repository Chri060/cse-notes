\section{Bimatrices}

Conventionally, Player 1 chooses a row, while Player 2 chooses a column.
This results in a pair of numbers, corresponding to the utilities of Player 1 and 2, respectively. The options can be summarized in a bimatrix.
\begin{example}
   Consider the following bimatrix:
    \[\begin{pmatrix}
        \begin{pmatrix} 8 & 8 \end{pmatrix} & \begin{pmatrix} 2 & 7 \end{pmatrix} \\
        \begin{pmatrix} 7 & 2 \end{pmatrix} & \begin{pmatrix} 0 & 0 \end{pmatrix}
    \end{pmatrix}\]
    In this example the utilities of player 1 are: 
    \[\begin{pmatrix}
        8 & 2 \\
        7 & 0
    \end{pmatrix}\]
    Since the second row is strictly dominated by the first, Player 1 selects the latter. Likewise, Player 2 selects the first column, which strictly dominates the second one
\end{example}
Even if the Principle of elimination of strictly dominated actions may not be very informative, it has rather surprising consequences.
\begin{example}
    Consider the following two games: 
    \[\begin{pmatrix}
        \begin{pmatrix} 10 & 10 \end{pmatrix} & \begin{pmatrix} 3 & 15 \end{pmatrix} \\
        \begin{pmatrix} 15 & 3 \end{pmatrix} & \begin{pmatrix} 5 & 5 \end{pmatrix}
    \end{pmatrix}\]

    \[\begin{pmatrix}
        \begin{pmatrix} 8 & 8 \end{pmatrix} & \begin{pmatrix} 2 & 7 \end{pmatrix} \\
        \begin{pmatrix} 7 & 2 \end{pmatrix} & \begin{pmatrix} 0 & 0 \end{pmatrix}
    \end{pmatrix}\]

    Observe: relative to any single outcome (i.e they always have greater utilities) in
    Nevertheless, according to the principle,
    interactive situation to play the second game rather than the first! (for, the outcome pair (8, 8) is greater than (5, 5) )

    The first game:
    \[\begin{pmatrix}
        \begin{pmatrix} 8 & 8 \end{pmatrix} & \begin{pmatrix} 2 & 7 \end{pmatrix} \\
        \begin{pmatrix} 7 & 2 \end{pmatrix} & \begin{pmatrix} 0 & 0 \end{pmatrix}
    \end{pmatrix}\]
    The second game contains all outcomes of the first, plus some further outcomes:
    The first game:
    \[\begin{pmatrix}
        \begin{pmatrix} 1 & 1 \end{pmatrix} & \begin{pmatrix} 11 & 0 \end{pmatrix} & \begin{pmatrix} 4 & 0 \end{pmatrix} \\
        \begin{pmatrix} 0 & 11 \end{pmatrix} & \begin{pmatrix} 8 & 8 \end{pmatrix} & \begin{pmatrix} 2 & 7 \end{pmatrix} \\
        \begin{pmatrix} 0 & 4 \end{pmatrix} & \begin{pmatrix} 7 & 2 \end{pmatrix} & \begin{pmatrix} 0 & 0 \end{pmatrix}
    \end{pmatrix}\]

    The second game contains all outcomes of the first, plus some further outcomes:

    Here, the rationality axioms imply that in the first game the players should choose the outcome pair (10, 10), whereas in the second game they should choose the outcome pair (1, 1) in the first row and first column.
    Therefore, just having less available actions can make the players better off!


\end{example}


\begin{example}
    What are the rational outcomes of the following game?
    \[\begin{pmatrix}
        \begin{pmatrix} 0 & 0 \end{pmatrix} & \begin{pmatrix} 1 & 1 \end{pmatrix} \\
        \begin{pmatrix} 1 & 1 \end{pmatrix} & \begin{pmatrix} 0 & 0 \end{pmatrix}
    \end{pmatrix}\]

    We formally do not know but it is obvious that the rational outcomes will be (1, 1)
    However, the actions prescribed by (first row, second column) and (second row, first column) yield the same preferred outcomes but they cannot be distinguished, thereby creating a coordination problem between the players!



\end{example}
\begin{example}
    Consider a voting game with three players with the following preferences: 
    \begin{enumerate}
        \item $A \precneqq B \precneqq C$
        \item $B \precneqq C \precneqq A$
        \item $C \precneqq A \precneqq B$
    \end{enumerate}
    The symbol $A \precneqq B$ means that $A \preceq B$ and not $B \preceq A$.
    The rule is that the alternative that receives most votes will win.
    Yet, in case of three different votes, the alternative selected by Player 1 will win.
    What can we expect to be the rational outcome of the game?
    Let us try with elimination of dominated actions:
    \begin{itemize}
        \item Alternative A is a weakly dominant strategy for Player 1. 
        \item Players 2 and 3 have their worst choice as weakly dominated strategy.
    \end{itemize}
    In order to avoid their worst outcome, Pl2 keeps B and C (ordered in rows) and Pl3 keeps C and A (ordered in columns), while Pl1 always plays A. Thus, the game reduces to a $2 \times 2$ table with the following outcomes:
    \begin{table}[H]
        \centering
        \begin{tabular}{l|ll|}
        \cline{2-3}
        \textbf{}                        & \textbf{C} & \textbf{A} \\ \hline
        \multicolumn{1}{|l|}{\textbf{B}} & A          & A          \\
        \multicolumn{1}{|l|}{\textbf{C}} & C          & A          \\ \hline
        \end{tabular}
    \end{table}
    Since $C \precneqq A$ for both Pl2 and Pl3, they will choose the outcome in the second row and first column. Therefore the final result is C, which is the worst one for Pl1!
\end{example}